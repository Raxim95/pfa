\documentclass{article}
\usepackage[utf8]{inputenc}
\usepackage{array}
\usepackage[a4paper,
  left=15mm,
  top=15mm,]{geometry}
\usepackage{setspace}

\renewcommand{\baselinestretch}{1.1} 


  
\begin{document}

\large
\pagenumbering{gobble}


\begin{tabular}{m{17cm}}
\textbf{1-variant}
\newline

T1. Ellips va uning kanonik tenglamasi (ta'rifi, fokuslari, ellipsning kanonik tenglamasi, ekstsentrisiteti, direktrisalari).\\

T2. ITECH-ning invariantlari (ITECH-ning umumiy tenglamasi, almashtirish, ITECH invariantlari).\\

A1. Aylananing $C$ markazi va $R$ radiusini toping: $x^2+y^2-2x+4y-14=0$.\\

A2. Fokuslari abssissa o'qida va koordinata boshiga nisbatan simmetrik joylashgan ellipsning tenglamasini tuzing: katta o'qi $8$, direktrisalar orasidagi masofa $16$.\\

A3. Qutb tenglamasi bilan berilgan egri chiziqning tipini aniqlang: $\rho=\frac{1}{3-3\cos\theta}$.\\

B1. $\frac{x^{2}}{4} - \frac{y^{2}}{5} = 1$ giperbolasiga $3x + 2y = 0$ to'g'ri chizig'iga perpendikulyar bo'lgan urinma to'g'ri chiziqning tenglamasini tuzing.\\

B2. Koordinata o'qlarini almashtirmasdan ITECH tenglamasini soddalashtiring, yarim o'qlarnin toping: $4x^{2} - 4xy + 7y^{2} - 26x - 18y + 3 = 0$.\\

B3. $x^{2} - 4y^{2} = 16$ giperbola berilgan. Uning ekssentrisitetini, fokuslarining koordinatalarini toping va asimptotalarining tenglamalarini tuzing.\\

C1. Fokusi $F( - 1; - 4)$ nuqtasida joylashgan, mos direktrisasi $x - 2 = 0$ tenglamasi bilan berilgan, $A( - 3; - 5)$ nuqtadan o'tuvchi ellipsning tenglamasini tuzing.  \\

C2. $4x^{2} - 4xy + y^{2} - 2x - 14y + 7 = 0$ ITECH tenglamasini kanonik shaklga olib keling, tipini aniqlang, qanday geometrik obraz ekanligini ko'rsating, chizmasini eski va yangi koordinatalar sistemasiga nisbatan chizing.  \\

C3. $A(\frac{10}{3};\frac{5}{3})$ nuqtasidan $\frac{x^{2}}{20} + \frac{y^{2}}{5} = 1$ ellipsiga yurgizilgan urinmalarning tenglamasini tuzing.  \\

\end{tabular}
\vspace{1cm}


\begin{tabular}{m{17cm}}
\textbf{2-variant}
\newline

T1. Bir pallali giperboloid. Kanonik tenglamasi (giperbolani simmetriya o'qi atrofida aylantirishdan olingan sirt).\\

T2. Giperbola. Kanonik tenglamasi (fokuslar, o'qlar, direktrisalar, giperbola, ekstsentrisitet, kanonik tenglamasi).\\

A1. Tipini aniqlang: $2x^{2}+3y^{2}+8x-6y+11=0$.\\

A2. Aylana tenglamasini tuzing: aylana $A(2;6)$ nuqtadan o'tadi va markazi $C(-1;2)$ nuqtada joylashgan.\\

A3. Fokuslari abssissa o'qida va koordinata boshiga nisbatan simmetrik joylashgan ellipsning tenglamasini tuzing: kichik o'qi $24$, fokuslari orasidagi masofa $2c=10$.\\

B1. $3x + 10y - 25 = 0$ to'g'ri bilan $\frac{x^{2}}{25} + \frac{y^{2}}{4} = 1$ ellipsning kesishish nuqtalarini toping.  \\

B2. $\rho = \frac{10}{2 - cos\theta}$ polyar tenglamasi bilan qanday chiziq berilganini aniqlang.  \\

B3. $2x + 2y - 3 = 0$ to'g'ri chizig'iga perpendikulyar bo'lib $x^{2} = 16y$ parabolasiga urinib o'tuvchi to'g'ri chiziqning tenglamasini tuzing.  \\

C1. $y^{2} = 20x$ parabolasining $M$ nuqtasini toping, agar uning abssissasi 7 ga teng bo'lsa, fokal radiusini va fokal radiusi joylashgan to'g'rini aniqlang.\\

C2. Uchi (-4;0) nuqtasinda, direktrisasi $y - 2 = 0$ to'g'ri chiziq bo'lgan parabolaning tenglamasini tuzing.\\

C3. $32x^{2} + 52xy - 7y^{2} + 180 = 0$ ITECH tenglamasini kanonik shaklga olib keling, tipini aniqlang, qanday geometrik obraz ekanligini ko'rsating, chizmasini eski va yangi koordinatalar sistemasiga nisbatan chizing.  \\

\end{tabular}
\vspace{1cm}


\begin{tabular}{m{17cm}}
\textbf{3-variant}
\newline

T1. ITECH-ning markazini aniqlash formulasi (ITECH-ning umumiy tenglamasi, markazini aniqlash formulasi).\\

T2. Giperbolik paraboloydning to'g'ri chiziq yasovchilari (Giperbolik paraboloydni yasovchi to'g'ri chiziqlar dastasi).\\

A1. Giperbola tenglamasi berilgan: $\frac{x^{2}}{16}-\frac{y^{2}}{9}=1$. Uning qutb tenglamasini tuzing.\\

A2. Tipini aniqlang: $9x^{2}+4y^{2}+18x-8y+49=0$.\\

A3. Aylana tenglamasini tuzing: markazi koordinata boshida joylashgan va radiusi $R=3$ ga teng.\\

B1. Koordinata o'qlarini almashtirmasdan ITECH umumiy tenglamasini soddalashtiring, yarim o'qlarini toping: $13x^{2} + 18xy + 37y^{2} - 26x - 18y + 3 = 0$.  \\

B2. Ellips $3x^{2} + 4y^{2} - 12 = 0$ tenglamasi bilan berilgan. Uning o'qlarining uzunliklarini, fokuslarining koordinatalarini va ekssentrisitetini toping.  \\

B3. $y^{2} = 3x$ parabolasi bilan $\frac{x^{2}}{100} + \frac{y^{2}}{225} = 1$ ellipsining kesishish nuqtalarini toping.  \\

C1. $\frac{x^{2}}{3} - \frac{y^{2}}{5} = 1$, giperbolasiga $P(4;2)$ nuqtadan yurgizilgan urinmalarning tenglamasini tuzing.  \\

C2. $M(2; - \frac{5}{3})$ nuqta $\frac{x^{2}}{9} + \frac{y^{2}}{5} = 1$ ellipsda joylashgan. $M$ nuqtaning fokal radiuslarida yotuvchi to'g'ri chiziq tenglamalarini tuzing.  \\

C3. Fokusi $F(2; - 1)$ nuqtasida joylashgan, mos direktrisasi $x - y - 1 = 0$ tenglamasi bilan berilgan parabolaning tenglamasini tuzing.  \\

\end{tabular}
\vspace{1cm}


\begin{tabular}{m{17cm}}
\textbf{4-variant}
\newline

T1. Giperbolaning urinmasining tenglamasi (giperbolaga berilgan nuqtada yurgizilgan urinma tenglamasi).\\

T2. Koordinata sistemasini almashtirish (birlik vektorlar, o'qlar, parallel ko'chirish, koordinata o'qlarinii burish).\\

A1. Uchi koordinata boshida joylashgan va $Ox$ o'qiga nisbatan o'ng tarafafgi yarim tekislikda joylashgan parabolaning tenglamasini tuzing: parametri $p=3$.\\

A2. Ellips tenglamasi berilgan: $\frac{x^2}{25}+\frac{y^2}{16}=1$. Uning qutb tenglamasini tuzing.\\

A3. Tipini aniqlang: $25x^{2}-20xy+4y^{2}-12x+20y-17=0$.\\

B1. $\rho = \frac{5}{3 - 4cos\theta}$ tenglamasi bilan qanday chiziq berilganini va yarim o'qlarini toping.  \\

B2. $\frac{x^{2}}{4} - \frac{y^{2}}{5} = 1$, giperbolaning $3x - 2y = 0$ to'g'ri chizig'iga parallel bo'lgan urinmasining tenglamasini tuzing.  \\

B3. Koordinata o'qlarini almashtirmasdan ITECH tenglamasini soddalashtiring, qanday geometrik obraz ekanligini ko'rsating: $4x^{2} - 4xy + y^{2} + 4x - 2y + 1 = 0$.  \\

C1. $4x^{2} - 4xy + y^{2} - 6x + 8y + 13 = 0$ ITECH markazga egami? Markazga ega bo'lsa markazini aniqlang?  \\

C2. $\frac{x^{2}}{3} - \frac{y^{2}}{5} = 1$ giperbolasiga $P(1; - 5)$ nuqtasida yurgizilgan urinmalarning tenglamasini tuzing.\\

C3. $y^{2} = 20x$ parabolasining abssissasi 7 ga teng bo'lgan $M$ nuqtasining fokal radiusini toping va fokal radiusi yotgan to'g'ri chiziqning tenglamasini tuzing.  \\

\end{tabular}
\vspace{1cm}


\begin{tabular}{m{17cm}}
\textbf{5-variant}
\newline

T1. Elliptik paraboloid (parabola, o'q, elliptik paraboloid).\\

T2. Parabola va uning kanonik tenglamasi ( ta'rifi, fokusi, direktrisasi, kanonik tenglamasi).\\

A1. Aylana tenglamasini tuzing: $A(3;1)$ va $B(-1;3)$ nuqtalardan o'tadi, markazi $3x-y-2=0$ togri chiziqda joylashgan.\\

A2. Fokuslari abssissa o'qida va koordinata boshiga nisbatan simmetrik joylashgan giperbolaning tenglamasini tuzing: asimptotalar tenglamalari $y=\pm \frac{4}{3}x$ va fokuslari orasidagi masofa $2c=20$.\\

A3. Qutb tenglamasi bilan berilgan egri chiziqning tipini aniqlang: $\rho=\frac{10}{1-\frac{3}{2}\cos\theta}$.\\

B1. $3x + 4y - 12 = 0$ to'g'ri chizig'i va $y^{2} = - 9x$ parabolasining kesishish nuqtalarini toping.\\

B2. $\rho = \frac{6}{1 - cos\theta}$ polyar tenglamasi bilan qanday chiziq berilganini aniqlang.  \\

B3. $2x + 2y - 3 = 0$ to'g'ri chizig'iga parallel bo'lib $\frac{x^{2}}{16} + \frac{y^{2}}{64} = 1$ giperbolasiga urinib o'tuvchi to'g'ri chiziqning tenglamasini tuzing.  \\

C1. Fokusi $F( - 1; - 4)$ nuqtasida bo'lgan, mos direktrisasi $x - 2 = 0$ tenglamasi bilan berilgan, $A( - 3; - 5)$ nuqtadan o'tuvchi ellipsning tenglamasini tuzing.  \\

C2. $16x^{2} - 9y^{2} - 64x - 54y - 161 = 0$ tenglamasi giperbolaning tenglamasi ekanligini ko'rsating va uning markazi $C$ ni, yarim o'qlarini, ekssentrisitetini toping, asimptotalarining tenglamalarini tuzing.  \\

C3. $\frac{x^{2}}{25} + \frac{y^{2}}{16} = 1$, ellipsiga $C(10; - 8)$ nuqtadan yurgizilgan urinmalarining tenglamasini tuzing.  \\

\end{tabular}
\vspace{1cm}


\begin{tabular}{m{17cm}}
\textbf{6-variant}
\newline

T1. ITECH-ning umumiy tenglamasini soddalashtirish (ITECH-ning umumiy tenglamasi, koordinata sistemasin almashtirish ITECH umumiy tenglamasini soddalashtirish).\\

T2. Ellipsoida. Kanonik tenglamasi (ellipsni simmetriya o'qi atrofida aylantirishdan olingan sirt, kanonik tenglamasi).\\

A1. Berilgan chiziqlarning markaziy ekanligini ko'rsating va markazinin toping: $9x^{2}-4xy-7y^{2}-12=0$.\\

A2. Aylananing $C$ markazi va $R$ radiusini toping: $x^2+y^2+4x-2y+5=0$.\\

A3. Fokuslari abssissa o'qida va koordinata boshiga nisbatan simmetrik joylashgan giperbolaning tenglamasini tuzing: direktrisalar orasidagi masofa $32/5$ va o'qi $2b=6$.\\

B1. $41x^{2} + 24xy + 9y^{2} + 24x + 18y - 36 = 0$ ITECH tipini aniqlang va markazlarini toping koordinata o'qlarini almashtirmasdan qanday chiziq ekanligini ko'rsating, yarim o'qlarini toping.  \\

B2. $3x + 4y - 12 = 0$ to'g'ri chizig'i bilan $y^{2} = - 9x$ parabolasining kesishish nuqtalarini toping.  \\

B3. $\rho = \frac{144}{13 - 5cos\theta}$ ellips ekanligini ko'rsating va uning yarim o'qlarini aniqlang.\\

C1. $\frac{x^{2}}{100} + \frac{y^{2}}{36} = 1$ ellipsining o'ng tarafdagi fokusidan 14 ga teng masofada bo'lgan nuqtasini toping.  \\

C2. Agar vaqtning xohlagan momentida $M(x;y)$ nuqta $5x - 16 = 0$ to'g'ri chiziqqa qaraganda $A(5;0)$ nuqtasidan 1,25 marta uzoqroq masofada joylashgan. Shu $M(x;y)$ nuqtaning harakatining tenglamasini tuzing.  \\

C3. $14x^{2} + 24xy + 21y^{2} - 4x + 18y - 139 = 0$ egri chizig'ining tipini aniqlang, agar markazga ega egri chiziq bo'lsa, markazining koordinatalarini toping.  \\

\end{tabular}
\vspace{1cm}


\begin{tabular}{m{17cm}}
\textbf{7-variant}
\newline

T1. Parabolaning urinmasining tenglamasi (parabola, to'g'ri chiziq urinish nuqtasi, urinma tenglamasi).\\

T2. ITECH-ning umumiy tenglamasini koordinata boshin parallel ko'chirish bilan soddalastiring (ITECH-ning umumiy tenglamasini parallel ko'chirish formulasi).\\

A1. Qutb tenglamasi bilan berilgan egri chiziqning tipini aniqlang: $\rho=\frac{12}{2-\cos\theta}$.\\

A2. Berilgan chiziqlarning markaziy ekanligini ko'rsating va markazinin toping: $2x^{2}-6xy+5y^{2}+22x-36y+11=0$.\\

A3. Aylana tenglamasini tuzing: markazi $C(6;-8)$ nuqtada joylashgan va koordinata boshidan o'tadi.\\

B1. $x^{2} + 4y^{2} = 25$ ellipsi bilan $4x - 2y + 23 = 0$ to'g'ri chizig'iga parallel bo'lgan urinma to'g'ri chiziqning tenglamasini tuzing.  \\

B2. ITECH ning umumiy tenglamasini koordinata sistemasini almashtirmasdan soddalashtiring, tipini aniqlang, obrazi qanday chiziq ekanligini ko'rsating: $7x^{2} - 8xy + y^{2} - 16x - 2y - 51 = 0$\\

B3. $y^{2} = 12x$ paraborolasiga $3x - 2y + 30 = 0$ to'g'ri chizig'iga parallel bo'lgan urinmasining tenglamasini tuzing.  \\

C1. Fokuslari $F(3;4)$, $F(-3;-4)$ nuqtalarida joylashgan direktrisalari orasidagi masofa 3,6 ga teng bo'lgan giperbolaning tenglamasini tuzing.  \\

C2. $4x^{2} + 24xy + 11y^{2} + 64x + 42y + 51 = 0$ egri chizig'ining tipini aniqlang, agar markazga ega bo'lsa, uning markazining koordinatalarini toping va koordinata boshini markazga parallel ko'chirish amalini bajaring.\\

C3. Katta o'qi 26 ga, fokuslari $F( - 10;0), F(14;0)$ nuqtalarida joylashgan ellipsning tenglamasini tuzing.  \\

\end{tabular}
\vspace{1cm}


\begin{tabular}{m{17cm}}
\textbf{8-variant}
\newline

T1. Ikkinshi tartibli aylanma sirtlar (koordinata sistemasi, tekislik, vektor egri chiziq, aylanma sirt).\\

T2. Parabolaning polyar koordinatalardagi tenglamasi (polyar koordinata sistemasida parabolaning tenglamasi).\\

A1. Fokuslari abssissa o'qida va koordinata boshiga nisbatan simmetrik joylashgan ellipsning tenglamasini tuzing: yarim o'qlari 5 va 2.\\

A2. Parabola tenglamasi berilgan: $y^2=6x$. Uning qutb tenglamasini tuzing.\\

A3. Tipini aniqlang: $4x^2+9y^2-40x+36y+100=0$.\\

B1. $\frac{x^{2}}{16} - \frac{y^{2}}{64} = 1$ giperbolasiga berilgan $10x - 3y + 9 = 0$ to'g'ri chizig'iga parallel bo'lgan urinmasining tenglamasini tuzing.  \\

B2. $x^{2} - y^{2} = 27$ giperbolasiga $4x + 2y - 7 = 0$ to'g'ri chizigiga parallel bo'lgan urinmasining tenglamasini toping.  \\

B3. Koordinata o'qlarini almashtirmasdan ITECH tenglamasini soddalashtiring, yarim o'qlarnin toping: $4x^{2} - 4xy + 7y^{2} - 26x - 18y + 3 = 0$.\\

C1. $2x^{2} + 3y^{2} + 8x - 6y + 11 = 0$ tenglamasi bilan qanday tipdagi chiziq berilganini aniqlang va uning tenglamasini soddalashtiring va grafigini chizing.  \\

C2. $y^{2} = 20x$ parabolasining $M$ nuqtasini toping, agar uning abssissasi 7 ga teng bo'lsa, fokal radiusini va fokal radiusi joylashgan to'g'rini aniqlang.\\

C3. Fokusi $F(7;2)$ nuqtasida joylashgan, mos direktrisasi $x - 5 = 0$ tenglamasi bilan berilgan parabolaning tenglamasini tuzing.  \\

\end{tabular}
\vspace{1cm}


\begin{tabular}{m{17cm}}
\textbf{9-variant}
\newline

T1. Ikkinchi tartibli sirtning umumiy tenglamasi. Markazin aniqlash formulasi.\\

T2. Sirtning kanonik tenglamalari. Sirt haqqida tushuncha. (Sirtning ta'rifi, formulalari, o'q, yo'naltiruvchi to'g'ri chiziqlar).\\

A1. Aylananing $C$ markazi va $R$ radiusini toping: $x^2+y^2-2x+4y-20=0$.\\

A2. Fokuslari abssissa o'qida va koordinata boshiga nisbatan simmetrik joylashgan giperbolaning tenglamasini tuzing: fokuslari orasidagi masofasi $2c=10$ va o'qi $2b=8$.\\

A3. Qutb tenglamasi bilan berilgan egri chiziqning tipini aniqlang: $\rho=\frac{5}{1-\frac{1}{2}\cos\theta}$.\\

B1. $\frac{x^{2}}{20} - \frac{y^{2}}{5} = 1$ giperbolasiga $4x + 3y - 7 = 0$ to'g'ri chizig'iga perpendikulyar bo'lgan urinmasining tenglamasini tuzing.  \\

B2. Koordinata o'qlarini almashtirmasdan ITECH umumiy tenglamasini soddalashtiring, yarim o'qlarini toping: $13x^{2} + 18xy + 37y^{2} - 26x - 18y + 3 = 0$.  \\

B3. $x^{2} - 4y^{2} = 16$ giperbola berilgan. Uning ekssentrisitetini, fokuslarining koordinatalarini toping va asimptotalarining tenglamalarini tuzing.\\

C1. $4x^{2} - 4xy + y^{2} - 2x - 14y + 7 = 0$ ITECH tenglamasini kanonik shaklga olib keling, tipini aniqlang, qanday geometrik obraz ekanligini ko'rsating, chizmasini eski va yangi koordinatalar sistemasiga nisbatan chizing.  \\

C2. $A(\frac{10}{3};\frac{5}{3})$ nuqtasidan $\frac{x^{2}}{20} + \frac{y^{2}}{5} = 1$ ellipsiga yurgizilgan urinmalarning tenglamasini tuzing.  \\

C3. $M(2; - \frac{5}{3})$ nuqta $\frac{x^{2}}{9} + \frac{y^{2}}{5} = 1$ ellipsda joylashgan. $M$ nuqtaning fokal radiuslarida yotuvchi to'g'ri chiziq tenglamalarini tuzing.  \\

\end{tabular}
\vspace{1cm}


\begin{tabular}{m{17cm}}
\textbf{10-variant}
\newline

T1. Giperbolaning polyar koordinatadagi tenglamasi (Polyar burchagi, polyar radiusi giperbolaning polyar tenglamasi)\\

T2. ITECH-ning umumiy tenglamasini koordinata o'qlarini burish bilan soddalashtirish (ITECH-ning umumiy tenglamalari, koordinata o'qin burish formulasi, tenglamani kanonik turga olib kelish).\\

A1. Tipini aniqlang: $2x^{2}+10xy+12y^{2}-7x+18y-15=0$.\\

A2. Aylana tenglamasini tuzing: markazi koordinata boshida joylashgan va $3x-4y+20=0$ to'g'ri chiziqga urinadi.\\

A3. Fokuslari abssissa o'qida va koordinata boshiga nisbatan simmetrik joylashgan giperbolaning tenglamasini tuzing: fokuslari orasidagi masofa $2c=6$ va ekssentrisitet $\varepsilon=3/2$.\\

B1. $3x + 10y - 25 = 0$ to'g'ri bilan $\frac{x^{2}}{25} + \frac{y^{2}}{4} = 1$ ellipsning kesishish nuqtalarini toping.  \\

B2. $\rho = \frac{10}{2 - cos\theta}$ polyar tenglamasi bilan qanday chiziq berilganini aniqlang.  \\

B3. $\frac{x^{2}}{4} - \frac{y^{2}}{5} = 1$ giperbolasiga $3x + 2y = 0$ to'g'ri chizig'iga perpendikulyar bo'lgan urinma to'g'ri chiziqning tenglamasini tuzing.\\

C1. Agar xohlagan vaqt momentida $M(x;y)$ nuqta $A(8;4)$ nuqtasidan va ordinata o'qidan birxil masofada joylashsa, $M(x;y)$ nuqtaning harakat troektoriyasining tenglamasini tuzing.  \\

C2. $32x^{2} + 52xy - 7y^{2} + 180 = 0$ ITECH tenglamasini kanonik shaklga olib keling, tipini aniqlang, qanday geometrik obraz ekanligini ko'rsating, chizmasini eski va yangi koordinatalar sistemasiga nisbatan chizing.  \\

C3. $\frac{x^{2}}{3} - \frac{y^{2}}{5} = 1$, giperbolasiga $P(4;2)$ nuqtadan yurgizilgan urinmalarning tenglamasini tuzing.  \\

\end{tabular}
\vspace{1cm}


\begin{tabular}{m{17cm}}
\textbf{11-variant}
\newline

T1. Ikki pallali giperboloid Kanonik tenglamasi (giperbolani simmetriya o'qi atrofida aylantirishdan olingan sirt).\\

T2. Ellipsning urinmasining tenglamasi (ellips, to'g'ri chiziq urinish nuqtasi, urinma tenglamasi).\\

A1. Qutb tenglamasi bilan berilgan egri chiziqning tipini aniqlang: $\rho=\frac{5}{3-4\cos\theta}$.\\

A2. Tipini aniqlang: $3x^{2}-8xy+7y^{2}+8x-15y+20=0$.\\

A3. Aylana tenglamasini tuzing: $M_1(-1;5)$, $M_2(-2;-2)$ va $M_3(5;5)$ nuqtalardan o'tadi.\\

B1. Koordinata o'qlarini almashtirmasdan ITECH tenglamasini soddalashtiring, qanday geometrik obraz ekanligini ko'rsating: $4x^{2} - 4xy + y^{2} + 4x - 2y + 1 = 0$.  \\

B2. Ellips $3x^{2} + 4y^{2} - 12 = 0$ tenglamasi bilan berilgan. Uning o'qlarining uzunliklarini, fokuslarining koordinatalarini va ekssentrisitetini toping.  \\

B3. $y^{2} = 3x$ parabolasi bilan $\frac{x^{2}}{100} + \frac{y^{2}}{225} = 1$ ellipsining kesishish nuqtalarini toping.  \\

C1. $y^{2} = 20x$ parabolasining abssissasi 7 ga teng bo'lgan $M$ nuqtasining fokal radiusini toping va fokal radiusi yotgan to'g'ri chiziqning tenglamasini tuzing.  \\

C2. Giperbolaning ekssentrisiteti $\varepsilon = \frac{13}{12}$, fokusi $F(0;13)$ nuqtasida va mos direktrisasi $13y - 144 = 0$ tenglamasi bilan berilgan bo'lsa, giperbolaning tenglamasini tuzing.  \\

C3. $4x^{2} - 4xy + y^{2} - 6x + 8y + 13 = 0$ ITECH markazga egami? Markazga ega bo'lsa markazini aniqlang?  \\

\end{tabular}
\vspace{1cm}


\begin{tabular}{m{17cm}}
\textbf{12-variant}
\newline

T1. ITECH-ning umumiy tenglamasini klassifikatsiyalash (ITECH-ning umumiy tenglamasi, ITECH-ning umumiy tenglamasini soddalashtirish, klassifikatsiyalash).\\

T2. Silindrlik sirtlar (yasovchi to'g'ri chiziq, yo'naltiruvchi egri chiziq, silindrlik sirt).\\

A1. Fokuslari abssissa o'qida va koordinata boshiga nisbatan simmetrik joylashgan ellipsning tenglamasini tuzing: katta o'qi $20$, ekssentrisitet $\varepsilon=3/5$.\\

A2. Giperbola tenglamasi berilgan: $\frac{x^{2}}{25}-\frac{y^{2}}{144}=1$. Uning qutb tenglamasini tuzing.\\

A3. Berilgan chiziqlarning markaziy ekanligini ko'rsating va markazinin toping: $5x^{2}+4xy+2y^{2}+20x+20y-18=0$.\\

B1. $\rho = \frac{5}{3 - 4cos\theta}$ tenglamasi bilan qanday chiziq berilganini va yarim o'qlarini toping.  \\

B2. $2x + 2y - 3 = 0$ to'g'ri chizig'iga perpendikulyar bo'lib $x^{2} = 16y$ parabolasiga urinib o'tuvchi to'g'ri chiziqning tenglamasini tuzing.  \\

B3. $41x^{2} + 24xy + 9y^{2} + 24x + 18y - 36 = 0$ ITECH tipini aniqlang va markazlarini toping koordinata o'qlarini almashtirmasdan qanday chiziq ekanligini ko'rsating, yarim o'qlarini toping.  \\

C1. $\frac{x^{2}}{3} - \frac{y^{2}}{5} = 1$ giperbolasiga $P(1; - 5)$ nuqtasida yurgizilgan urinmalarning tenglamasini tuzing.\\

C2. $\frac{x^{2}}{100} + \frac{y^{2}}{36} = 1$ ellipsining o'ng tarafdagi fokusidan 14 ga teng masofada bo'lgan nuqtasini toping.  \\

C3. Fokusi $F( - 1; - 4)$ nuqtasida joylashgan, mos direktrisasi $x - 2 = 0$ tenglamasi bilan berilgan, $A( - 3; - 5)$ nuqtadan o'tuvchi ellipsning tenglamasini tuzing.  \\

\end{tabular}
\vspace{1cm}


\begin{tabular}{m{17cm}}
\textbf{13-variant}
\newline

T1. ITECH-ning invariantlari (ITECH-ning umumiy tenglamasi, almashtirish, ITECH invariantlari).\\

T2. Ellipsning polyar koordinatalardagi tenglamasi (polyar koordinatalar sistemasida ellipsning tenglamasi).\\

A1. Aylana tenglamasini tuzing: markazi $C(1;-1)$ nuqtada joylashgan va $5x-12y+9-0$ to'g'ri chiziqga urinadi.\\

A2. Uchi koordinata boshida joylashgan va $Ox$ o'qiga nisbatan chap tarafafgi yarim tekislikda joylashgan parabolaning tenglamasini tuzing: parametri $p=0,5$.\\

A3. Qutb tenglamasi bilan berilgan egri chiziqning tipini aniqlang: $\rho=\frac{6}{1-\cos 0}$.\\

B1. $3x + 4y - 12 = 0$ to'g'ri chizig'i va $y^{2} = - 9x$ parabolasining kesishish nuqtalarini toping.\\

B2. $\rho = \frac{6}{1 - cos\theta}$ polyar tenglamasi bilan qanday chiziq berilganini aniqlang.  \\

B3. $\frac{x^{2}}{4} - \frac{y^{2}}{5} = 1$, giperbolaning $3x - 2y = 0$ to'g'ri chizig'iga parallel bo'lgan urinmasining tenglamasini tuzing.  \\

C1. $16x^{2} - 9y^{2} - 64x - 54y - 161 = 0$ tenglamasi giperbolaning tenglamasi ekanligini ko'rsating va uning markazi $C$ ni, yarim o'qlarini, ekssentrisitetini toping, asimptotalarining tenglamalarini tuzing.  \\

C2. $\frac{x^{2}}{25} + \frac{y^{2}}{16} = 1$, ellipsiga $C(10; - 8)$ nuqtadan yurgizilgan urinmalarining tenglamasini tuzing.  \\

C3. $y^{2} = 20x$ parabolasining $M$ nuqtasini toping, agar uning abssissasi 7 ga teng bo'lsa, fokal radiusini va fokal radiusi joylashgan to'g'rini aniqlang.\\

\end{tabular}
\vspace{1cm}


\begin{tabular}{m{17cm}}
\textbf{14-variant}
\newline

T1. ITECH-ning markazini aniqlash formulasi (ITECH-ning umumiy tenglamasi, markazini aniqlash formulasi).\\

T2. Bir pallali giperboloid. Kanonik tenglamasi (giperbolani simmetriya o'qi atrofida aylantirishdan olingan sirt).\\

A1. Tipini aniqlang: $x^{2}-4xy+4y^{2}+7x-12=0$.\\

A2. Aylana tenglamasini tuzing: $A(1;1)$, $B(1;-1)$ va $C(2;0)$ nuqtalardan o'tadi.\\

A3. Fokuslari abssissa o'qida va koordinata boshiga nisbatan simmetrik joylashgan ellipsning tenglamasini tuzing: katta o'qi $10$, fokuslari orasidagi masofa $2c=8$.\\

B1. ITECH ning umumiy tenglamasini koordinata sistemasini almashtirmasdan soddalashtiring, tipini aniqlang, obrazi qanday chiziq ekanligini ko'rsating: $7x^{2} - 8xy + y^{2} - 16x - 2y - 51 = 0$\\

B2. $3x + 4y - 12 = 0$ to'g'ri chizig'i bilan $y^{2} = - 9x$ parabolasining kesishish nuqtalarini toping.  \\

B3. $\rho = \frac{144}{13 - 5cos\theta}$ ellips ekanligini ko'rsating va uning yarim o'qlarini aniqlang.\\

C1. Uchi (-4;0) nuqtasinda, direktrisasi $y - 2 = 0$ to'g'ri chiziq bo'lgan parabolaning tenglamasini tuzing.\\

C2. $14x^{2} + 24xy + 21y^{2} - 4x + 18y - 139 = 0$ egri chizig'ining tipini aniqlang, agar markazga ega egri chiziq bo'lsa, markazining koordinatalarini toping.  \\

C3. Fokusi $F(2; - 1)$ nuqtasida joylashgan, mos direktrisasi $x - y - 1 = 0$ tenglamasi bilan berilgan parabolaning tenglamasini tuzing.  \\

\end{tabular}
\vspace{1cm}


\begin{tabular}{m{17cm}}
\textbf{15-variant}
\newline

T1. Ellips va uning kanonik tenglamasi (ta'rifi, fokuslari, ellipsning kanonik tenglamasi, ekstsentrisiteti, direktrisalari).\\

T2. Koordinata sistemasini almashtirish (birlik vektorlar, o'qlar, parallel ko'chirish, koordinata o'qlarinii burish).\\

A1. Tipini aniqlang: $9x^{2}-16y^{2}-54x-64y-127=0$.\\

A2. Aylananing $C$ markazi va $R$ radiusini toping: $x^2+y^2+6x-4y+14=0$.\\

A3. Fokuslari abssissa o'qida va koordinata boshiga nisbatan simmetrik joylashgan giperbolaning tenglamasini tuzing: direktrisalar orasidagi masofa $8/3$ va ekssentrisitet $\varepsilon=3/2$.\\

B1. $2x + 2y - 3 = 0$ to'g'ri chizig'iga parallel bo'lib $\frac{x^{2}}{16} + \frac{y^{2}}{64} = 1$ giperbolasiga urinib o'tuvchi to'g'ri chiziqning tenglamasini tuzing.  \\

B2. Koordinata o'qlarini almashtirmasdan ITECH tenglamasini soddalashtiring, yarim o'qlarnin toping: $4x^{2} - 4xy + 7y^{2} - 26x - 18y + 3 = 0$.\\

B3. $x^{2} + 4y^{2} = 25$ ellipsi bilan $4x - 2y + 23 = 0$ to'g'ri chizig'iga parallel bo'lgan urinma to'g'ri chiziqning tenglamasini tuzing.  \\

C1. $4x^{2} + 24xy + 11y^{2} + 64x + 42y + 51 = 0$ egri chizig'ining tipini aniqlang, agar markazga ega bo'lsa, uning markazining koordinatalarini toping va koordinata boshini markazga parallel ko'chirish amalini bajaring.\\

C2. Fokusi $F( - 1; - 4)$ nuqtasida bo'lgan, mos direktrisasi $x - 2 = 0$ tenglamasi bilan berilgan, $A( - 3; - 5)$ nuqtadan o'tuvchi ellipsning tenglamasini tuzing.  \\

C3. $2x^{2} + 3y^{2} + 8x - 6y + 11 = 0$ tenglamasi bilan qanday tipdagi chiziq berilganini aniqlang va uning tenglamasini soddalashtiring va grafigini chizing.  \\

\end{tabular}
\vspace{1cm}


\begin{tabular}{m{17cm}}
\textbf{16-variant}
\newline

T1. Giperbolik paraboloydning to'g'ri chiziq yasovchilari (Giperbolik paraboloydni yasovchi to'g'ri chiziqlar dastasi).\\

T2. Giperbola. Kanonik tenglamasi (fokuslar, o'qlar, direktrisalar, giperbola, ekstsentrisitet, kanonik tenglamasi).\\

A1. Tipini aniqlang: $5x^{2}+14xy+11y^{2}+12x-7y+19=0$.\\

A2. Aylana tenglamasini tuzing: aylana diametrining uchlari $A(3;2)$ va $B(-1;6)$ nuqtalarda joylashgan.\\

A3. Fokuslari abssissa o'qida va koordinata boshiga nisbatan simmetrik joylashgan giperbolaning tenglamasini tuzing: o'qlari $2a=10$ va $2b=8$.\\

B1. $y^{2} = 12x$ paraborolasiga $3x - 2y + 30 = 0$ to'g'ri chizig'iga parallel bo'lgan urinmasining tenglamasini tuzing.  \\

B2. $\frac{x^{2}}{16} - \frac{y^{2}}{64} = 1$ giperbolasiga berilgan $10x - 3y + 9 = 0$ to'g'ri chizig'iga parallel bo'lgan urinmasining tenglamasini tuzing.  \\

B3. Koordinata o'qlarini almashtirmasdan ITECH umumiy tenglamasini soddalashtiring, yarim o'qlarini toping: $13x^{2} + 18xy + 37y^{2} - 26x - 18y + 3 = 0$.  \\

C1. $M(2; - \frac{5}{3})$ nuqta $\frac{x^{2}}{9} + \frac{y^{2}}{5} = 1$ ellipsda joylashgan. $M$ nuqtaning fokal radiuslarida yotuvchi to'g'ri chiziq tenglamalarini tuzing.  \\

C2. Agar vaqtning xohlagan momentida $M(x;y)$ nuqta $5x - 16 = 0$ to'g'ri chiziqqa qaraganda $A(5;0)$ nuqtasidan 1,25 marta uzoqroq masofada joylashgan. Shu $M(x;y)$ nuqtaning harakatining tenglamasini tuzing.  \\

C3. $4x^{2} - 4xy + y^{2} - 2x - 14y + 7 = 0$ ITECH tenglamasini kanonik shaklga olib keling, tipini aniqlang, qanday geometrik obraz ekanligini ko'rsating, chizmasini eski va yangi koordinatalar sistemasiga nisbatan chizing.  \\

\end{tabular}
\vspace{1cm}


\begin{tabular}{m{17cm}}
\textbf{17-variant}
\newline

T1. ITECH-ning umumiy tenglamasini soddalashtirish (ITECH-ning umumiy tenglamasi, koordinata sistemasin almashtirish ITECH umumiy tenglamasini soddalashtirish).\\

T2. Elliptik paraboloid (parabola, o'q, elliptik paraboloid).\\

A1. Tipini aniqlang: $4x^{2}-y^{2}+8x-2y+3=0$.\\

A2. Aylana tenglamasini tuzing: markazi $C(2;-3)$ nuqtada joylashgan va radiusi $R=7$ ga teng.\\

A3. Fokuslari abssissa o'qida va koordinata boshiga nisbatan simmetrik joylashgan giperbolaning tenglamasini tuzing: direktrisalar orasidagi masofa $228/13$ va fokuslari orasidagi masofa $2c=26$.\\

B1. $x^{2} - y^{2} = 27$ giperbolasiga $4x + 2y - 7 = 0$ to'g'ri chizigiga parallel bo'lgan urinmasining tenglamasini toping.  \\

B2. Koordinata o'qlarini almashtirmasdan ITECH tenglamasini soddalashtiring, qanday geometrik obraz ekanligini ko'rsating: $4x^{2} - 4xy + y^{2} + 4x - 2y + 1 = 0$.  \\

B3. $x^{2} - 4y^{2} = 16$ giperbola berilgan. Uning ekssentrisitetini, fokuslarining koordinatalarini toping va asimptotalarining tenglamalarini tuzing.\\

C1. $A(\frac{10}{3};\frac{5}{3})$ nuqtasidan $\frac{x^{2}}{20} + \frac{y^{2}}{5} = 1$ ellipsiga yurgizilgan urinmalarning tenglamasini tuzing.  \\

C2. $y^{2} = 20x$ parabolasining abssissasi 7 ga teng bo'lgan $M$ nuqtasining fokal radiusini toping va fokal radiusi yotgan to'g'ri chiziqning tenglamasini tuzing.  \\

C3. Fokuslari $F(3;4)$, $F(-3;-4)$ nuqtalarida joylashgan direktrisalari orasidagi masofa 3,6 ga teng bo'lgan giperbolaning tenglamasini tuzing.  \\

\end{tabular}
\vspace{1cm}


\begin{tabular}{m{17cm}}
\textbf{18-variant}
\newline

T1. Giperbolaning urinmasining tenglamasi (giperbolaga berilgan nuqtada yurgizilgan urinma tenglamasi).\\

T2. ITECH-ning umumiy tenglamasini koordinata boshin parallel ko'chirish bilan soddalastiring (ITECH-ning umumiy tenglamasini parallel ko'chirish formulasi).\\

A1. Tipini aniqlang: $3x^{2}-2xy-3y^{2}+12y-15=0$.\\

A2. Fokuslari abssissa o'qida va koordinata boshiga nisbatan simmetrik joylashgan ellipsning tenglamasini tuzing: kichik o'qi $10$, ekssentrisitet $\varepsilon=12/13$.\\

A3. Berilgan chiziqlarning markaziy ekanligini ko'rsating va markazinin toping: $3x^{2}+5xy+y^{2}-8x-11y-7=0$.\\

B1. $3x + 10y - 25 = 0$ to'g'ri bilan $\frac{x^{2}}{25} + \frac{y^{2}}{4} = 1$ ellipsning kesishish nuqtalarini toping.  \\

B2. $\rho = \frac{10}{2 - cos\theta}$ polyar tenglamasi bilan qanday chiziq berilganini aniqlang.  \\

B3. $\frac{x^{2}}{20} - \frac{y^{2}}{5} = 1$ giperbolasiga $4x + 3y - 7 = 0$ to'g'ri chizig'iga perpendikulyar bo'lgan urinmasining tenglamasini tuzing.  \\

C1. $32x^{2} + 52xy - 7y^{2} + 180 = 0$ ITECH tenglamasini kanonik shaklga olib keling, tipini aniqlang, qanday geometrik obraz ekanligini ko'rsating, chizmasini eski va yangi koordinatalar sistemasiga nisbatan chizing.  \\

C2. $\frac{x^{2}}{3} - \frac{y^{2}}{5} = 1$, giperbolasiga $P(4;2)$ nuqtadan yurgizilgan urinmalarning tenglamasini tuzing.  \\

C3. $\frac{x^{2}}{100} + \frac{y^{2}}{36} = 1$ ellipsining o'ng tarafdagi fokusidan 14 ga teng masofada bo'lgan nuqtasini toping.  \\

\end{tabular}
\vspace{1cm}


\begin{tabular}{m{17cm}}
\textbf{19-variant}
\newline

T1. Ellipsoida. Kanonik tenglamasi (ellipsni simmetriya o'qi atrofida aylantirishdan olingan sirt, kanonik tenglamasi).\\

T2. Parabola va uning kanonik tenglamasi ( ta'rifi, fokusi, direktrisasi, kanonik tenglamasi).\\

A1. Uchi koordinata boshida joylashgan va $Oy$ o'qiga nisbatan quyi tarafafgi yarim tekislikda joylashgan parabolaning tenglamasini tuzing: parametri $p=3$.\\

A2. Uchi koordinata boshida joylashgan va $Oy$ o'qiga nisbatan yuqori yarim tekislikda joylashgan parabolaning tenglamasini tuzing: parametri $p=1/4$.\\

A3. Fokuslari abssissa o'qida va koordinata boshiga nisbatan simmetrik joylashgan ellipsning tenglamasini tuzing: direktrisalar orasidagi masofa $5$ va fokuslari orasidagi masofa $2c=4$.\\

B1. $41x^{2} + 24xy + 9y^{2} + 24x + 18y - 36 = 0$ ITECH tipini aniqlang va markazlarini toping koordinata o'qlarini almashtirmasdan qanday chiziq ekanligini ko'rsating, yarim o'qlarini toping.  \\

B2. Ellips $3x^{2} + 4y^{2} - 12 = 0$ tenglamasi bilan berilgan. Uning o'qlarining uzunliklarini, fokuslarining koordinatalarini va ekssentrisitetini toping.  \\

B3. $y^{2} = 3x$ parabolasi bilan $\frac{x^{2}}{100} + \frac{y^{2}}{225} = 1$ ellipsining kesishish nuqtalarini toping.  \\

C1. Katta o'qi 26 ga, fokuslari $F( - 10;0), F(14;0)$ nuqtalarida joylashgan ellipsning tenglamasini tuzing.  \\

C2. $4x^{2} - 4xy + y^{2} - 6x + 8y + 13 = 0$ ITECH markazga egami? Markazga ega bo'lsa markazini aniqlang?  \\

C3. $\frac{x^{2}}{3} - \frac{y^{2}}{5} = 1$ giperbolasiga $P(1; - 5)$ nuqtasida yurgizilgan urinmalarning tenglamasini tuzing.\\

\end{tabular}
\vspace{1cm}


\begin{tabular}{m{17cm}}
\textbf{20-variant}
\newline

T1. Ikkinchi tartibli sirtning umumiy tenglamasi. Markazin aniqlash formulasi.\\

T2. Ikkinshi tartibli aylanma sirtlar (koordinata sistemasi, tekislik, vektor egri chiziq, aylanma sirt).\\

A1. Fokuslari abssissa o'qida va koordinata boshiga nisbatan simmetrik joylashgan giperbolaning tenglamasini tuzing: katta o'qi $2a=16$ va ekssentrisitet $\varepsilon=5/4$.\\

A2. Fokuslari abssissa o'qida va koordinata boshiga nisbatan simmetrik joylashgan ellipsning tenglamasini tuzing: kichik o'qi $6$, direktrisalar orasidagi masofa $13$.\\

A3. Fokuslari abssissa o'qida va koordinata boshiga nisbatan simmetrik joylashgan ellipsning tenglamasini tuzing: fokuslari orasidagi masofa $2c=6$ va ekssentrisitet $\varepsilon=3/5$.\\

B1. $\rho = \frac{5}{3 - 4cos\theta}$ tenglamasi bilan qanday chiziq berilganini va yarim o'qlarini toping.  \\

B2. $\frac{x^{2}}{4} - \frac{y^{2}}{5} = 1$ giperbolasiga $3x + 2y = 0$ to'g'ri chizig'iga perpendikulyar bo'lgan urinma to'g'ri chiziqning tenglamasini tuzing.\\

B3. ITECH ning umumiy tenglamasini koordinata sistemasini almashtirmasdan soddalashtiring, tipini aniqlang, obrazi qanday chiziq ekanligini ko'rsating: $7x^{2} - 8xy + y^{2} - 16x - 2y - 51 = 0$\\

C1. $y^{2} = 20x$ parabolasining $M$ nuqtasini toping, agar uning abssissasi 7 ga teng bo'lsa, fokal radiusini va fokal radiusi joylashgan to'g'rini aniqlang.\\

C2. Fokusi $F(7;2)$ nuqtasida joylashgan, mos direktrisasi $x - 5 = 0$ tenglamasi bilan berilgan parabolaning tenglamasini tuzing.  \\

C3. $16x^{2} - 9y^{2} - 64x - 54y - 161 = 0$ tenglamasi giperbolaning tenglamasi ekanligini ko'rsating va uning markazi $C$ ni, yarim o'qlarini, ekssentrisitetini toping, asimptotalarining tenglamalarini tuzing.  \\

\end{tabular}
\vspace{1cm}


\begin{tabular}{m{17cm}}
\textbf{21-variant}
\newline

T1. Parabolaning urinmasining tenglamasi (parabola, to'g'ri chiziq urinish nuqtasi, urinma tenglamasi).\\

T2. ITECH-ning umumiy tenglamasini koordinata o'qlarini burish bilan soddalashtirish (ITECH-ning umumiy tenglamalari, koordinata o'qin burish formulasi, tenglamani kanonik turga olib kelish).\\

A1. Fokuslari abssissa o'qida va koordinata boshiga nisbatan simmetrik joylashgan ellipsning tenglamasini tuzing: direktrisalar orasidagi masofa $32$ va $\varepsilon=1/9$.\\

A2. Fokuslari abssissa o'qida va koordinata boshiga nisbatan simmetrik joylashgan giperbolaning tenglamasini tuzing: asimptotalar tenglamalari $y=\pm \frac{3}{4}x$ va direktrisalar orasidagi masofa $64/5$.\\

A3. Aylananing $C$ markazi va $R$ radiusini toping: $x^2+y^2-2x+4y-14=0$.\\

B1. $3x + 4y - 12 = 0$ to'g'ri chizig'i va $y^{2} = - 9x$ parabolasining kesishish nuqtalarini toping.\\

B2. $\rho = \frac{6}{1 - cos\theta}$ polyar tenglamasi bilan qanday chiziq berilganini aniqlang.  \\

B3. $2x + 2y - 3 = 0$ to'g'ri chizig'iga perpendikulyar bo'lib $x^{2} = 16y$ parabolasiga urinib o'tuvchi to'g'ri chiziqning tenglamasini tuzing.  \\

C1. $\frac{x^{2}}{25} + \frac{y^{2}}{16} = 1$, ellipsiga $C(10; - 8)$ nuqtadan yurgizilgan urinmalarining tenglamasini tuzing.  \\

C2. $M(2; - \frac{5}{3})$ nuqta $\frac{x^{2}}{9} + \frac{y^{2}}{5} = 1$ ellipsda joylashgan. $M$ nuqtaning fokal radiuslarida yotuvchi to'g'ri chiziq tenglamalarini tuzing.  \\

C3. Agar xohlagan vaqt momentida $M(x;y)$ nuqta $A(8;4)$ nuqtasidan va ordinata o'qidan birxil masofada joylashsa, $M(x;y)$ nuqtaning harakat troektoriyasining tenglamasini tuzing.  \\

\end{tabular}
\vspace{1cm}


\begin{tabular}{m{17cm}}
\textbf{22-variant}
\newline

T1. Sirtning kanonik tenglamalari. Sirt haqqida tushuncha. (Sirtning ta'rifi, formulalari, o'q, yo'naltiruvchi to'g'ri chiziqlar).\\

T2. Parabolaning polyar koordinatalardagi tenglamasi (polyar koordinata sistemasida parabolaning tenglamasi).\\

A1. Fokuslari abssissa o'qida va koordinata boshiga nisbatan simmetrik joylashgan ellipsning tenglamasini tuzing: katta o'qi $8$, direktrisalar orasidagi masofa $16$.\\

A2. Qutb tenglamasi bilan berilgan egri chiziqning tipini aniqlang: $\rho=\frac{1}{3-3\cos\theta}$.\\

A3. Tipini aniqlang: $2x^{2}+3y^{2}+8x-6y+11=0$.\\

B1. Koordinata o'qlarini almashtirmasdan ITECH tenglamasini soddalashtiring, yarim o'qlarnin toping: $4x^{2} - 4xy + 7y^{2} - 26x - 18y + 3 = 0$.\\

B2. $3x + 4y - 12 = 0$ to'g'ri chizig'i bilan $y^{2} = - 9x$ parabolasining kesishish nuqtalarini toping.  \\

B3. $\rho = \frac{144}{13 - 5cos\theta}$ ellips ekanligini ko'rsating va uning yarim o'qlarini aniqlang.\\

C1. $14x^{2} + 24xy + 21y^{2} - 4x + 18y - 139 = 0$ egri chizig'ining tipini aniqlang, agar markazga ega egri chiziq bo'lsa, markazining koordinatalarini toping.  \\

C2. Giperbolaning ekssentrisiteti $\varepsilon = \frac{13}{12}$, fokusi $F(0;13)$ nuqtasida va mos direktrisasi $13y - 144 = 0$ tenglamasi bilan berilgan bo'lsa, giperbolaning tenglamasini tuzing.  \\

C3. $4x^{2} + 24xy + 11y^{2} + 64x + 42y + 51 = 0$ egri chizig'ining tipini aniqlang, agar markazga ega bo'lsa, uning markazining koordinatalarini toping va koordinata boshini markazga parallel ko'chirish amalini bajaring.\\

\end{tabular}
\vspace{1cm}


\begin{tabular}{m{17cm}}
\textbf{23-variant}
\newline

T1. ITECH-ning umumiy tenglamasini klassifikatsiyalash (ITECH-ning umumiy tenglamasi, ITECH-ning umumiy tenglamasini soddalashtirish, klassifikatsiyalash).\\

T2. Ikki pallali giperboloid Kanonik tenglamasi (giperbolani simmetriya o'qi atrofida aylantirishdan olingan sirt).\\

A1. Aylana tenglamasini tuzing: aylana $A(2;6)$ nuqtadan o'tadi va markazi $C(-1;2)$ nuqtada joylashgan.\\

A2. Fokuslari abssissa o'qida va koordinata boshiga nisbatan simmetrik joylashgan ellipsning tenglamasini tuzing: kichik o'qi $24$, fokuslari orasidagi masofa $2c=10$.\\

A3. Giperbola tenglamasi berilgan: $\frac{x^{2}}{16}-\frac{y^{2}}{9}=1$. Uning qutb tenglamasini tuzing.\\

B1. $\frac{x^{2}}{4} - \frac{y^{2}}{5} = 1$, giperbolaning $3x - 2y = 0$ to'g'ri chizig'iga parallel bo'lgan urinmasining tenglamasini tuzing.  \\

B2. Koordinata o'qlarini almashtirmasdan ITECH umumiy tenglamasini soddalashtiring, yarim o'qlarini toping: $13x^{2} + 18xy + 37y^{2} - 26x - 18y + 3 = 0$.  \\

B3. $2x + 2y - 3 = 0$ to'g'ri chizig'iga parallel bo'lib $\frac{x^{2}}{16} + \frac{y^{2}}{64} = 1$ giperbolasiga urinib o'tuvchi to'g'ri chiziqning tenglamasini tuzing.  \\

C1. Fokusi $F( - 1; - 4)$ nuqtasida joylashgan, mos direktrisasi $x - 2 = 0$ tenglamasi bilan berilgan, $A( - 3; - 5)$ nuqtadan o'tuvchi ellipsning tenglamasini tuzing.  \\

C2. $2x^{2} + 3y^{2} + 8x - 6y + 11 = 0$ tenglamasi bilan qanday tipdagi chiziq berilganini aniqlang va uning tenglamasini soddalashtiring va grafigini chizing.  \\

C3. $y^{2} = 20x$ parabolasining abssissasi 7 ga teng bo'lgan $M$ nuqtasining fokal radiusini toping va fokal radiusi yotgan to'g'ri chiziqning tenglamasini tuzing.  \\

\end{tabular}
\vspace{1cm}


\begin{tabular}{m{17cm}}
\textbf{24-variant}
\newline

T1. Giperbolaning polyar koordinatadagi tenglamasi (Polyar burchagi, polyar radiusi giperbolaning polyar tenglamasi)\\

T2. ITECH-ning invariantlari (ITECH-ning umumiy tenglamasi, almashtirish, ITECH invariantlari).\\

A1. Tipini aniqlang: $9x^{2}+4y^{2}+18x-8y+49=0$.\\

A2. Aylana tenglamasini tuzing: markazi koordinata boshida joylashgan va radiusi $R=3$ ga teng.\\

A3. Uchi koordinata boshida joylashgan va $Ox$ o'qiga nisbatan o'ng tarafafgi yarim tekislikda joylashgan parabolaning tenglamasini tuzing: parametri $p=3$.\\

B1. $x^{2} + 4y^{2} = 25$ ellipsi bilan $4x - 2y + 23 = 0$ to'g'ri chizig'iga parallel bo'lgan urinma to'g'ri chiziqning tenglamasini tuzing.  \\

B2. $y^{2} = 12x$ paraborolasiga $3x - 2y + 30 = 0$ to'g'ri chizig'iga parallel bo'lgan urinmasining tenglamasini tuzing.  \\

B3. Koordinata o'qlarini almashtirmasdan ITECH tenglamasini soddalashtiring, qanday geometrik obraz ekanligini ko'rsating: $4x^{2} - 4xy + y^{2} + 4x - 2y + 1 = 0$.  \\

C1. Uchi (-4;0) nuqtasinda, direktrisasi $y - 2 = 0$ to'g'ri chiziq bo'lgan parabolaning tenglamasini tuzing.\\

C2. $4x^{2} - 4xy + y^{2} - 2x - 14y + 7 = 0$ ITECH tenglamasini kanonik shaklga olib keling, tipini aniqlang, qanday geometrik obraz ekanligini ko'rsating, chizmasini eski va yangi koordinatalar sistemasiga nisbatan chizing.  \\

C3. $A(\frac{10}{3};\frac{5}{3})$ nuqtasidan $\frac{x^{2}}{20} + \frac{y^{2}}{5} = 1$ ellipsiga yurgizilgan urinmalarning tenglamasini tuzing.  \\

\end{tabular}
\vspace{1cm}


\begin{tabular}{m{17cm}}
\textbf{25-variant}
\newline

T1. Silindrlik sirtlar (yasovchi to'g'ri chiziq, yo'naltiruvchi egri chiziq, silindrlik sirt).\\

T2. ITECH-ning markazini aniqlash formulasi (ITECH-ning umumiy tenglamasi, markazini aniqlash formulasi).\\

A1. Ellips tenglamasi berilgan: $\frac{x^2}{25}+\frac{y^2}{16}=1$. Uning qutb tenglamasini tuzing.\\

A2. Tipini aniqlang: $25x^{2}-20xy+4y^{2}-12x+20y-17=0$.\\

A3. Aylana tenglamasini tuzing: $A(3;1)$ va $B(-1;3)$ nuqtalardan o'tadi, markazi $3x-y-2=0$ togri chiziqda joylashgan.\\

B1. $\frac{x^{2}}{16} - \frac{y^{2}}{64} = 1$ giperbolasiga berilgan $10x - 3y + 9 = 0$ to'g'ri chizig'iga parallel bo'lgan urinmasining tenglamasini tuzing.  \\

B2. $41x^{2} + 24xy + 9y^{2} + 24x + 18y - 36 = 0$ ITECH tipini aniqlang va markazlarini toping koordinata o'qlarini almashtirmasdan qanday chiziq ekanligini ko'rsating, yarim o'qlarini toping.  \\

B3. $x^{2} - 4y^{2} = 16$ giperbola berilgan. Uning ekssentrisitetini, fokuslarining koordinatalarini toping va asimptotalarining tenglamalarini tuzing.\\

C1. $\frac{x^{2}}{100} + \frac{y^{2}}{36} = 1$ ellipsining o'ng tarafdagi fokusidan 14 ga teng masofada bo'lgan nuqtasini toping.  \\

C2. Fokusi $F(2; - 1)$ nuqtasida joylashgan, mos direktrisasi $x - y - 1 = 0$ tenglamasi bilan berilgan parabolaning tenglamasini tuzing.  \\

C3. $32x^{2} + 52xy - 7y^{2} + 180 = 0$ ITECH tenglamasini kanonik shaklga olib keling, tipini aniqlang, qanday geometrik obraz ekanligini ko'rsating, chizmasini eski va yangi koordinatalar sistemasiga nisbatan chizing.  \\

\end{tabular}
\vspace{1cm}


\begin{tabular}{m{17cm}}
\textbf{26-variant}
\newline

T1. Ellipsning urinmasining tenglamasi (ellips, to'g'ri chiziq urinish nuqtasi, urinma tenglamasi).\\

T2. Koordinata sistemasini almashtirish (birlik vektorlar, o'qlar, parallel ko'chirish, koordinata o'qlarinii burish).\\

A1. Fokuslari abssissa o'qida va koordinata boshiga nisbatan simmetrik joylashgan giperbolaning tenglamasini tuzing: asimptotalar tenglamalari $y=\pm \frac{4}{3}x$ va fokuslari orasidagi masofa $2c=20$.\\

A2. Qutb tenglamasi bilan berilgan egri chiziqning tipini aniqlang: $\rho=\frac{10}{1-\frac{3}{2}\cos\theta}$.\\

A3. Berilgan chiziqlarning markaziy ekanligini ko'rsating va markazinin toping: $9x^{2}-4xy-7y^{2}-12=0$.\\

B1. $3x + 10y - 25 = 0$ to'g'ri bilan $\frac{x^{2}}{25} + \frac{y^{2}}{4} = 1$ ellipsning kesishish nuqtalarini toping.  \\

B2. $\rho = \frac{10}{2 - cos\theta}$ polyar tenglamasi bilan qanday chiziq berilganini aniqlang.  \\

B3. $x^{2} - y^{2} = 27$ giperbolasiga $4x + 2y - 7 = 0$ to'g'ri chizigiga parallel bo'lgan urinmasining tenglamasini toping.  \\

C1. $\frac{x^{2}}{3} - \frac{y^{2}}{5} = 1$, giperbolasiga $P(4;2)$ nuqtadan yurgizilgan urinmalarning tenglamasini tuzing.  \\

C2. $y^{2} = 20x$ parabolasining $M$ nuqtasini toping, agar uning abssissasi 7 ga teng bo'lsa, fokal radiusini va fokal radiusi joylashgan to'g'rini aniqlang.\\

C3. Fokusi $F( - 1; - 4)$ nuqtasida bo'lgan, mos direktrisasi $x - 2 = 0$ tenglamasi bilan berilgan, $A( - 3; - 5)$ nuqtadan o'tuvchi ellipsning tenglamasini tuzing.  \\

\end{tabular}
\vspace{1cm}


\begin{tabular}{m{17cm}}
\textbf{27-variant}
\newline

T1. Bir pallali giperboloid. Kanonik tenglamasi (giperbolani simmetriya o'qi atrofida aylantirishdan olingan sirt).\\

T2. Ellipsning polyar koordinatalardagi tenglamasi (polyar koordinatalar sistemasida ellipsning tenglamasi).\\

A1. Aylananing $C$ markazi va $R$ radiusini toping: $x^2+y^2+4x-2y+5=0$.\\

A2. Fokuslari abssissa o'qida va koordinata boshiga nisbatan simmetrik joylashgan giperbolaning tenglamasini tuzing: direktrisalar orasidagi masofa $32/5$ va o'qi $2b=6$.\\

A3. Qutb tenglamasi bilan berilgan egri chiziqning tipini aniqlang: $\rho=\frac{12}{2-\cos\theta}$.\\

B1. ITECH ning umumiy tenglamasini koordinata sistemasini almashtirmasdan soddalashtiring, tipini aniqlang, obrazi qanday chiziq ekanligini ko'rsating: $7x^{2} - 8xy + y^{2} - 16x - 2y - 51 = 0$\\

B2. Ellips $3x^{2} + 4y^{2} - 12 = 0$ tenglamasi bilan berilgan. Uning o'qlarining uzunliklarini, fokuslarining koordinatalarini va ekssentrisitetini toping.  \\

B3. $y^{2} = 3x$ parabolasi bilan $\frac{x^{2}}{100} + \frac{y^{2}}{225} = 1$ ellipsining kesishish nuqtalarini toping.  \\

C1. $4x^{2} - 4xy + y^{2} - 6x + 8y + 13 = 0$ ITECH markazga egami? Markazga ega bo'lsa markazini aniqlang?  \\

C2. $\frac{x^{2}}{3} - \frac{y^{2}}{5} = 1$ giperbolasiga $P(1; - 5)$ nuqtasida yurgizilgan urinmalarning tenglamasini tuzing.\\

C3. $M(2; - \frac{5}{3})$ nuqta $\frac{x^{2}}{9} + \frac{y^{2}}{5} = 1$ ellipsda joylashgan. $M$ nuqtaning fokal radiuslarida yotuvchi to'g'ri chiziq tenglamalarini tuzing.  \\

\end{tabular}
\vspace{1cm}


\begin{tabular}{m{17cm}}
\textbf{28-variant}
\newline

T1. ITECH-ning umumiy tenglamasini soddalashtirish (ITECH-ning umumiy tenglamasi, koordinata sistemasin almashtirish ITECH umumiy tenglamasini soddalashtirish).\\

T2. Giperbolik paraboloydning to'g'ri chiziq yasovchilari (Giperbolik paraboloydni yasovchi to'g'ri chiziqlar dastasi).\\

A1. Berilgan chiziqlarning markaziy ekanligini ko'rsating va markazinin toping: $2x^{2}-6xy+5y^{2}+22x-36y+11=0$.\\

A2. Aylana tenglamasini tuzing: markazi $C(6;-8)$ nuqtada joylashgan va koordinata boshidan o'tadi.\\

A3. Fokuslari abssissa o'qida va koordinata boshiga nisbatan simmetrik joylashgan ellipsning tenglamasini tuzing: yarim o'qlari 5 va 2.\\

B1. $\rho = \frac{5}{3 - 4cos\theta}$ tenglamasi bilan qanday chiziq berilganini va yarim o'qlarini toping.  \\

B2. $\frac{x^{2}}{20} - \frac{y^{2}}{5} = 1$ giperbolasiga $4x + 3y - 7 = 0$ to'g'ri chizig'iga perpendikulyar bo'lgan urinmasining tenglamasini tuzing.  \\

B3. Koordinata o'qlarini almashtirmasdan ITECH tenglamasini soddalashtiring, yarim o'qlarnin toping: $4x^{2} - 4xy + 7y^{2} - 26x - 18y + 3 = 0$.\\

C1. Agar vaqtning xohlagan momentida $M(x;y)$ nuqta $5x - 16 = 0$ to'g'ri chiziqqa qaraganda $A(5;0)$ nuqtasidan 1,25 marta uzoqroq masofada joylashgan. Shu $M(x;y)$ nuqtaning harakatining tenglamasini tuzing.  \\

C2. $16x^{2} - 9y^{2} - 64x - 54y - 161 = 0$ tenglamasi giperbolaning tenglamasi ekanligini ko'rsating va uning markazi $C$ ni, yarim o'qlarini, ekssentrisitetini toping, asimptotalarining tenglamalarini tuzing.  \\

C3. $\frac{x^{2}}{25} + \frac{y^{2}}{16} = 1$, ellipsiga $C(10; - 8)$ nuqtadan yurgizilgan urinmalarining tenglamasini tuzing.  \\

\end{tabular}
\vspace{1cm}


\begin{tabular}{m{17cm}}
\textbf{29-variant}
\newline

T1. Ellips va uning kanonik tenglamasi (ta'rifi, fokuslari, ellipsning kanonik tenglamasi, ekstsentrisiteti, direktrisalari).\\

T2. ITECH-ning umumiy tenglamasini koordinata boshin parallel ko'chirish bilan soddalastiring (ITECH-ning umumiy tenglamasini parallel ko'chirish formulasi).\\

A1. Parabola tenglamasi berilgan: $y^2=6x$. Uning qutb tenglamasini tuzing.\\

A2. Tipini aniqlang: $4x^2+9y^2-40x+36y+100=0$.\\

A3. Aylananing $C$ markazi va $R$ radiusini toping: $x^2+y^2-2x+4y-20=0$.\\

B1. $3x + 4y - 12 = 0$ to'g'ri chizig'i va $y^{2} = - 9x$ parabolasining kesishish nuqtalarini toping.\\

B2. $\rho = \frac{6}{1 - cos\theta}$ polyar tenglamasi bilan qanday chiziq berilganini aniqlang.  \\

B3. $\frac{x^{2}}{4} - \frac{y^{2}}{5} = 1$ giperbolasiga $3x + 2y = 0$ to'g'ri chizig'iga perpendikulyar bo'lgan urinma to'g'ri chiziqning tenglamasini tuzing.\\

C1. $y^{2} = 20x$ parabolasining abssissasi 7 ga teng bo'lgan $M$ nuqtasining fokal radiusini toping va fokal radiusi yotgan to'g'ri chiziqning tenglamasini tuzing.  \\

C2. Fokuslari $F(3;4)$, $F(-3;-4)$ nuqtalarida joylashgan direktrisalari orasidagi masofa 3,6 ga teng bo'lgan giperbolaning tenglamasini tuzing.  \\

C3. $14x^{2} + 24xy + 21y^{2} - 4x + 18y - 139 = 0$ egri chizig'ining tipini aniqlang, agar markazga ega egri chiziq bo'lsa, markazining koordinatalarini toping.  \\

\end{tabular}
\vspace{1cm}


\begin{tabular}{m{17cm}}
\textbf{30-variant}
\newline

T1. Elliptik paraboloid (parabola, o'q, elliptik paraboloid).\\

T2. Giperbola. Kanonik tenglamasi (fokuslar, o'qlar, direktrisalar, giperbola, ekstsentrisitet, kanonik tenglamasi).\\

A1. Fokuslari abssissa o'qida va koordinata boshiga nisbatan simmetrik joylashgan giperbolaning tenglamasini tuzing: fokuslari orasidagi masofasi $2c=10$ va o'qi $2b=8$.\\

A2. Qutb tenglamasi bilan berilgan egri chiziqning tipini aniqlang: $\rho=\frac{5}{1-\frac{1}{2}\cos\theta}$.\\

A3. Tipini aniqlang: $2x^{2}+10xy+12y^{2}-7x+18y-15=0$.\\

B1. Koordinata o'qlarini almashtirmasdan ITECH umumiy tenglamasini soddalashtiring, yarim o'qlarini toping: $13x^{2} + 18xy + 37y^{2} - 26x - 18y + 3 = 0$.  \\

B2. $3x + 4y - 12 = 0$ to'g'ri chizig'i bilan $y^{2} = - 9x$ parabolasining kesishish nuqtalarini toping.  \\

B3. $\rho = \frac{144}{13 - 5cos\theta}$ ellips ekanligini ko'rsating va uning yarim o'qlarini aniqlang.\\

C1. Katta o'qi 26 ga, fokuslari $F( - 10;0), F(14;0)$ nuqtalarida joylashgan ellipsning tenglamasini tuzing.  \\

C2. $4x^{2} + 24xy + 11y^{2} + 64x + 42y + 51 = 0$ egri chizig'ining tipini aniqlang, agar markazga ega bo'lsa, uning markazining koordinatalarini toping va koordinata boshini markazga parallel ko'chirish amalini bajaring.\\

C3. Fokusi $F(7;2)$ nuqtasida joylashgan, mos direktrisasi $x - 5 = 0$ tenglamasi bilan berilgan parabolaning tenglamasini tuzing.  \\

\end{tabular}
\vspace{1cm}


\begin{tabular}{m{17cm}}
\textbf{31-variant}
\newline

T1. Ikkinchi tartibli sirtning umumiy tenglamasi. Markazin aniqlash formulasi.\\

T2. Ellipsoida. Kanonik tenglamasi (ellipsni simmetriya o'qi atrofida aylantirishdan olingan sirt, kanonik tenglamasi).\\

A1. Aylana tenglamasini tuzing: markazi koordinata boshida joylashgan va $3x-4y+20=0$ to'g'ri chiziqga urinadi.\\

A2. Fokuslari abssissa o'qida va koordinata boshiga nisbatan simmetrik joylashgan giperbolaning tenglamasini tuzing: fokuslari orasidagi masofa $2c=6$ va ekssentrisitet $\varepsilon=3/2$.\\

A3. Qutb tenglamasi bilan berilgan egri chiziqning tipini aniqlang: $\rho=\frac{5}{3-4\cos\theta}$.\\

B1. $2x + 2y - 3 = 0$ to'g'ri chizig'iga perpendikulyar bo'lib $x^{2} = 16y$ parabolasiga urinib o'tuvchi to'g'ri chiziqning tenglamasini tuzing.  \\

B2. Koordinata o'qlarini almashtirmasdan ITECH tenglamasini soddalashtiring, qanday geometrik obraz ekanligini ko'rsating: $4x^{2} - 4xy + y^{2} + 4x - 2y + 1 = 0$.  \\

B3. $\frac{x^{2}}{4} - \frac{y^{2}}{5} = 1$, giperbolaning $3x - 2y = 0$ to'g'ri chizig'iga parallel bo'lgan urinmasining tenglamasini tuzing.  \\

C1. $2x^{2} + 3y^{2} + 8x - 6y + 11 = 0$ tenglamasi bilan qanday tipdagi chiziq berilganini aniqlang va uning tenglamasini soddalashtiring va grafigini chizing.  \\

C2. $\frac{x^{2}}{100} + \frac{y^{2}}{36} = 1$ ellipsining o'ng tarafdagi fokusidan 14 ga teng masofada bo'lgan nuqtasini toping.  \\

C3. Agar xohlagan vaqt momentida $M(x;y)$ nuqta $A(8;4)$ nuqtasidan va ordinata o'qidan birxil masofada joylashsa, $M(x;y)$ nuqtaning harakat troektoriyasining tenglamasini tuzing.  \\

\end{tabular}
\vspace{1cm}


\begin{tabular}{m{17cm}}
\textbf{32-variant}
\newline

T1. Giperbolaning urinmasining tenglamasi (giperbolaga berilgan nuqtada yurgizilgan urinma tenglamasi).\\

T2. ITECH-ning umumiy tenglamasini koordinata o'qlarini burish bilan soddalashtirish (ITECH-ning umumiy tenglamalari, koordinata o'qin burish formulasi, tenglamani kanonik turga olib kelish).\\

A1. Tipini aniqlang: $3x^{2}-8xy+7y^{2}+8x-15y+20=0$.\\

A2. Aylana tenglamasini tuzing: $M_1(-1;5)$, $M_2(-2;-2)$ va $M_3(5;5)$ nuqtalardan o'tadi.\\

A3. Fokuslari abssissa o'qida va koordinata boshiga nisbatan simmetrik joylashgan ellipsning tenglamasini tuzing: katta o'qi $20$, ekssentrisitet $\varepsilon=3/5$.\\

B1. $2x + 2y - 3 = 0$ to'g'ri chizig'iga parallel bo'lib $\frac{x^{2}}{16} + \frac{y^{2}}{64} = 1$ giperbolasiga urinib o'tuvchi to'g'ri chiziqning tenglamasini tuzing.  \\

B2. $x^{2} + 4y^{2} = 25$ ellipsi bilan $4x - 2y + 23 = 0$ to'g'ri chizig'iga parallel bo'lgan urinma to'g'ri chiziqning tenglamasini tuzing.  \\

B3. $41x^{2} + 24xy + 9y^{2} + 24x + 18y - 36 = 0$ ITECH tipini aniqlang va markazlarini toping koordinata o'qlarini almashtirmasdan qanday chiziq ekanligini ko'rsating, yarim o'qlarini toping.  \\

C1. $4x^{2} - 4xy + y^{2} - 2x - 14y + 7 = 0$ ITECH tenglamasini kanonik shaklga olib keling, tipini aniqlang, qanday geometrik obraz ekanligini ko'rsating, chizmasini eski va yangi koordinatalar sistemasiga nisbatan chizing.  \\

C2. $A(\frac{10}{3};\frac{5}{3})$ nuqtasidan $\frac{x^{2}}{20} + \frac{y^{2}}{5} = 1$ ellipsiga yurgizilgan urinmalarning tenglamasini tuzing.  \\

C3. $y^{2} = 20x$ parabolasining $M$ nuqtasini toping, agar uning abssissasi 7 ga teng bo'lsa, fokal radiusini va fokal radiusi joylashgan to'g'rini aniqlang.\\

\end{tabular}
\vspace{1cm}


\begin{tabular}{m{17cm}}
\textbf{33-variant}
\newline

T1. Ikkinshi tartibli aylanma sirtlar (koordinata sistemasi, tekislik, vektor egri chiziq, aylanma sirt).\\

T2. Parabola va uning kanonik tenglamasi ( ta'rifi, fokusi, direktrisasi, kanonik tenglamasi).\\

A1. Giperbola tenglamasi berilgan: $\frac{x^{2}}{25}-\frac{y^{2}}{144}=1$. Uning qutb tenglamasini tuzing.\\

A2. Berilgan chiziqlarning markaziy ekanligini ko'rsating va markazinin toping: $5x^{2}+4xy+2y^{2}+20x+20y-18=0$.\\

A3. Aylana tenglamasini tuzing: markazi $C(1;-1)$ nuqtada joylashgan va $5x-12y+9-0$ to'g'ri chiziqga urinadi.\\

B1. $y^{2} = 12x$ paraborolasiga $3x - 2y + 30 = 0$ to'g'ri chizig'iga parallel bo'lgan urinmasining tenglamasini tuzing.  \\

B2. ITECH ning umumiy tenglamasini koordinata sistemasini almashtirmasdan soddalashtiring, tipini aniqlang, obrazi qanday chiziq ekanligini ko'rsating: $7x^{2} - 8xy + y^{2} - 16x - 2y - 51 = 0$\\

B3. $x^{2} - 4y^{2} = 16$ giperbola berilgan. Uning ekssentrisitetini, fokuslarining koordinatalarini toping va asimptotalarining tenglamalarini tuzing.\\

C1. Giperbolaning ekssentrisiteti $\varepsilon = \frac{13}{12}$, fokusi $F(0;13)$ nuqtasida va mos direktrisasi $13y - 144 = 0$ tenglamasi bilan berilgan bo'lsa, giperbolaning tenglamasini tuzing.  \\

C2. $32x^{2} + 52xy - 7y^{2} + 180 = 0$ ITECH tenglamasini kanonik shaklga olib keling, tipini aniqlang, qanday geometrik obraz ekanligini ko'rsating, chizmasini eski va yangi koordinatalar sistemasiga nisbatan chizing.  \\

C3. $\frac{x^{2}}{3} - \frac{y^{2}}{5} = 1$, giperbolasiga $P(4;2)$ nuqtadan yurgizilgan urinmalarning tenglamasini tuzing.  \\

\end{tabular}
\vspace{1cm}


\begin{tabular}{m{17cm}}
\textbf{34-variant}
\newline

T1. ITECH-ning umumiy tenglamasini klassifikatsiyalash (ITECH-ning umumiy tenglamasi, ITECH-ning umumiy tenglamasini soddalashtirish, klassifikatsiyalash).\\

T2. Sirtning kanonik tenglamalari. Sirt haqqida tushuncha. (Sirtning ta'rifi, formulalari, o'q, yo'naltiruvchi to'g'ri chiziqlar).\\

A1. Uchi koordinata boshida joylashgan va $Ox$ o'qiga nisbatan chap tarafafgi yarim tekislikda joylashgan parabolaning tenglamasini tuzing: parametri $p=0,5$.\\

A2. Qutb tenglamasi bilan berilgan egri chiziqning tipini aniqlang: $\rho=\frac{6}{1-\cos 0}$.\\

A3. Tipini aniqlang: $x^{2}-4xy+4y^{2}+7x-12=0$.\\

B1. $3x + 10y - 25 = 0$ to'g'ri bilan $\frac{x^{2}}{25} + \frac{y^{2}}{4} = 1$ ellipsning kesishish nuqtalarini toping.  \\

B2. $\rho = \frac{10}{2 - cos\theta}$ polyar tenglamasi bilan qanday chiziq berilganini aniqlang.  \\

B3. $\frac{x^{2}}{16} - \frac{y^{2}}{64} = 1$ giperbolasiga berilgan $10x - 3y + 9 = 0$ to'g'ri chizig'iga parallel bo'lgan urinmasining tenglamasini tuzing.  \\

C1. $M(2; - \frac{5}{3})$ nuqta $\frac{x^{2}}{9} + \frac{y^{2}}{5} = 1$ ellipsda joylashgan. $M$ nuqtaning fokal radiuslarida yotuvchi to'g'ri chiziq tenglamalarini tuzing.  \\

C2. Fokusi $F( - 1; - 4)$ nuqtasida joylashgan, mos direktrisasi $x - 2 = 0$ tenglamasi bilan berilgan, $A( - 3; - 5)$ nuqtadan o'tuvchi ellipsning tenglamasini tuzing.  \\

C3. $4x^{2} - 4xy + y^{2} - 6x + 8y + 13 = 0$ ITECH markazga egami? Markazga ega bo'lsa markazini aniqlang?  \\

\end{tabular}
\vspace{1cm}


\begin{tabular}{m{17cm}}
\textbf{35-variant}
\newline

T1. Parabolaning urinmasining tenglamasi (parabola, to'g'ri chiziq urinish nuqtasi, urinma tenglamasi).\\

T2. ITECH-ning invariantlari (ITECH-ning umumiy tenglamasi, almashtirish, ITECH invariantlari).\\

A1. Aylana tenglamasini tuzing: $A(1;1)$, $B(1;-1)$ va $C(2;0)$ nuqtalardan o'tadi.\\

A2. Fokuslari abssissa o'qida va koordinata boshiga nisbatan simmetrik joylashgan ellipsning tenglamasini tuzing: katta o'qi $10$, fokuslari orasidagi masofa $2c=8$.\\

A3. Tipini aniqlang: $9x^{2}-16y^{2}-54x-64y-127=0$.\\

B1. Koordinata o'qlarini almashtirmasdan ITECH tenglamasini soddalashtiring, yarim o'qlarnin toping: $4x^{2} - 4xy + 7y^{2} - 26x - 18y + 3 = 0$.\\

B2. Ellips $3x^{2} + 4y^{2} - 12 = 0$ tenglamasi bilan berilgan. Uning o'qlarining uzunliklarini, fokuslarining koordinatalarini va ekssentrisitetini toping.  \\

B3. $y^{2} = 3x$ parabolasi bilan $\frac{x^{2}}{100} + \frac{y^{2}}{225} = 1$ ellipsining kesishish nuqtalarini toping.  \\

C1. $\frac{x^{2}}{3} - \frac{y^{2}}{5} = 1$ giperbolasiga $P(1; - 5)$ nuqtasida yurgizilgan urinmalarning tenglamasini tuzing.\\

C2. $y^{2} = 20x$ parabolasining abssissasi 7 ga teng bo'lgan $M$ nuqtasining fokal radiusini toping va fokal radiusi yotgan to'g'ri chiziqning tenglamasini tuzing.  \\

C3. Uchi (-4;0) nuqtasinda, direktrisasi $y - 2 = 0$ to'g'ri chiziq bo'lgan parabolaning tenglamasini tuzing.\\

\end{tabular}
\vspace{1cm}


\begin{tabular}{m{17cm}}
\textbf{36-variant}
\newline

T1. Ikki pallali giperboloid Kanonik tenglamasi (giperbolani simmetriya o'qi atrofida aylantirishdan olingan sirt).\\

T2. Parabolaning polyar koordinatalardagi tenglamasi (polyar koordinata sistemasida parabolaning tenglamasi).\\

A1. Aylananing $C$ markazi va $R$ radiusini toping: $x^2+y^2+6x-4y+14=0$.\\

A2. Fokuslari abssissa o'qida va koordinata boshiga nisbatan simmetrik joylashgan giperbolaning tenglamasini tuzing: direktrisalar orasidagi masofa $8/3$ va ekssentrisitet $\varepsilon=3/2$.\\

A3. Tipini aniqlang: $5x^{2}+14xy+11y^{2}+12x-7y+19=0$.\\

B1. $\rho = \frac{5}{3 - 4cos\theta}$ tenglamasi bilan qanday chiziq berilganini va yarim o'qlarini toping.  \\

B2. $x^{2} - y^{2} = 27$ giperbolasiga $4x + 2y - 7 = 0$ to'g'ri chizigiga parallel bo'lgan urinmasining tenglamasini toping.  \\

B3. Koordinata o'qlarini almashtirmasdan ITECH umumiy tenglamasini soddalashtiring, yarim o'qlarini toping: $13x^{2} + 18xy + 37y^{2} - 26x - 18y + 3 = 0$.  \\

C1. $16x^{2} - 9y^{2} - 64x - 54y - 161 = 0$ tenglamasi giperbolaning tenglamasi ekanligini ko'rsating va uning markazi $C$ ni, yarim o'qlarini, ekssentrisitetini toping, asimptotalarining tenglamalarini tuzing.  \\

C2. $\frac{x^{2}}{25} + \frac{y^{2}}{16} = 1$, ellipsiga $C(10; - 8)$ nuqtadan yurgizilgan urinmalarining tenglamasini tuzing.  \\

C3. $\frac{x^{2}}{100} + \frac{y^{2}}{36} = 1$ ellipsining o'ng tarafdagi fokusidan 14 ga teng masofada bo'lgan nuqtasini toping.  \\

\end{tabular}
\vspace{1cm}


\begin{tabular}{m{17cm}}
\textbf{37-variant}
\newline

T1. ITECH-ning markazini aniqlash formulasi (ITECH-ning umumiy tenglamasi, markazini aniqlash formulasi).\\

T2. Silindrlik sirtlar (yasovchi to'g'ri chiziq, yo'naltiruvchi egri chiziq, silindrlik sirt).\\

A1. Aylana tenglamasini tuzing: aylana diametrining uchlari $A(3;2)$ va $B(-1;6)$ nuqtalarda joylashgan.\\

A2. Fokuslari abssissa o'qida va koordinata boshiga nisbatan simmetrik joylashgan giperbolaning tenglamasini tuzing: o'qlari $2a=10$ va $2b=8$.\\

A3. Tipini aniqlang: $4x^{2}-y^{2}+8x-2y+3=0$.\\

B1. $3x + 4y - 12 = 0$ to'g'ri chizig'i va $y^{2} = - 9x$ parabolasining kesishish nuqtalarini toping.\\

B2. $\rho = \frac{6}{1 - cos\theta}$ polyar tenglamasi bilan qanday chiziq berilganini aniqlang.  \\

B3. $\frac{x^{2}}{20} - \frac{y^{2}}{5} = 1$ giperbolasiga $4x + 3y - 7 = 0$ to'g'ri chizig'iga perpendikulyar bo'lgan urinmasining tenglamasini tuzing.  \\

C1. Fokusi $F(2; - 1)$ nuqtasida joylashgan, mos direktrisasi $x - y - 1 = 0$ tenglamasi bilan berilgan parabolaning tenglamasini tuzing.  \\

C2. $14x^{2} + 24xy + 21y^{2} - 4x + 18y - 139 = 0$ egri chizig'ining tipini aniqlang, agar markazga ega egri chiziq bo'lsa, markazining koordinatalarini toping.  \\

C3. Fokusi $F( - 1; - 4)$ nuqtasida bo'lgan, mos direktrisasi $x - 2 = 0$ tenglamasi bilan berilgan, $A( - 3; - 5)$ nuqtadan o'tuvchi ellipsning tenglamasini tuzing.  \\

\end{tabular}
\vspace{1cm}


\begin{tabular}{m{17cm}}
\textbf{38-variant}
\newline

T1. Koordinata sistemasini almashtirish (birlik vektorlar, o'qlar, parallel ko'chirish, koordinata o'qlarinii burish).\\

T2. Giperbolaning polyar koordinatadagi tenglamasi (Polyar burchagi, polyar radiusi giperbolaning polyar tenglamasi)\\

A1. Aylana tenglamasini tuzing: markazi $C(2;-3)$ nuqtada joylashgan va radiusi $R=7$ ga teng.\\

A2. Fokuslari abssissa o'qida va koordinata boshiga nisbatan simmetrik joylashgan giperbolaning tenglamasini tuzing: direktrisalar orasidagi masofa $228/13$ va fokuslari orasidagi masofa $2c=26$.\\

A3. Tipini aniqlang: $3x^{2}-2xy-3y^{2}+12y-15=0$.\\

B1. Koordinata o'qlarini almashtirmasdan ITECH tenglamasini soddalashtiring, qanday geometrik obraz ekanligini ko'rsating: $4x^{2} - 4xy + y^{2} + 4x - 2y + 1 = 0$.  \\

B2. $3x + 4y - 12 = 0$ to'g'ri chizig'i bilan $y^{2} = - 9x$ parabolasining kesishish nuqtalarini toping.  \\

B3. $\rho = \frac{144}{13 - 5cos\theta}$ ellips ekanligini ko'rsating va uning yarim o'qlarini aniqlang.\\

C1. $4x^{2} + 24xy + 11y^{2} + 64x + 42y + 51 = 0$ egri chizig'ining tipini aniqlang, agar markazga ega bo'lsa, uning markazining koordinatalarini toping va koordinata boshini markazga parallel ko'chirish amalini bajaring.\\

C2. Agar vaqtning xohlagan momentida $M(x;y)$ nuqta $5x - 16 = 0$ to'g'ri chiziqqa qaraganda $A(5;0)$ nuqtasidan 1,25 marta uzoqroq masofada joylashgan. Shu $M(x;y)$ nuqtaning harakatining tenglamasini tuzing.  \\

C3. $2x^{2} + 3y^{2} + 8x - 6y + 11 = 0$ tenglamasi bilan qanday tipdagi chiziq berilganini aniqlang va uning tenglamasini soddalashtiring va grafigini chizing.  \\

\end{tabular}
\vspace{1cm}


\begin{tabular}{m{17cm}}
\textbf{39-variant}
\newline

T1. ITECH-ning umumiy tenglamasini soddalashtirish (ITECH-ning umumiy tenglamasi, koordinata sistemasin almashtirish ITECH umumiy tenglamasini soddalashtirish).\\

T2. Bir pallali giperboloid. Kanonik tenglamasi (giperbolani simmetriya o'qi atrofida aylantirishdan olingan sirt).\\

A1. Fokuslari abssissa o'qida va koordinata boshiga nisbatan simmetrik joylashgan ellipsning tenglamasini tuzing: kichik o'qi $10$, ekssentrisitet $\varepsilon=12/13$.\\

A2. Berilgan chiziqlarning markaziy ekanligini ko'rsating va markazinin toping: $3x^{2}+5xy+y^{2}-8x-11y-7=0$.\\

A3. Uchi koordinata boshida joylashgan va $Oy$ o'qiga nisbatan quyi tarafafgi yarim tekislikda joylashgan parabolaning tenglamasini tuzing: parametri $p=3$.\\

B1. $\frac{x^{2}}{4} - \frac{y^{2}}{5} = 1$ giperbolasiga $3x + 2y = 0$ to'g'ri chizig'iga perpendikulyar bo'lgan urinma to'g'ri chiziqning tenglamasini tuzing.\\

B2. $41x^{2} + 24xy + 9y^{2} + 24x + 18y - 36 = 0$ ITECH tipini aniqlang va markazlarini toping koordinata o'qlarini almashtirmasdan qanday chiziq ekanligini ko'rsating, yarim o'qlarini toping.  \\

B3. $2x + 2y - 3 = 0$ to'g'ri chizig'iga perpendikulyar bo'lib $x^{2} = 16y$ parabolasiga urinib o'tuvchi to'g'ri chiziqning tenglamasini tuzing.  \\

C1. $y^{2} = 20x$ parabolasining $M$ nuqtasini toping, agar uning abssissasi 7 ga teng bo'lsa, fokal radiusini va fokal radiusi joylashgan to'g'rini aniqlang.\\

C2. Fokuslari $F(3;4)$, $F(-3;-4)$ nuqtalarida joylashgan direktrisalari orasidagi masofa 3,6 ga teng bo'lgan giperbolaning tenglamasini tuzing.  \\

C3. $4x^{2} - 4xy + y^{2} - 2x - 14y + 7 = 0$ ITECH tenglamasini kanonik shaklga olib keling, tipini aniqlang, qanday geometrik obraz ekanligini ko'rsating, chizmasini eski va yangi koordinatalar sistemasiga nisbatan chizing.  \\

\end{tabular}
\vspace{1cm}


\begin{tabular}{m{17cm}}
\textbf{40-variant}
\newline

T1. Ellipsning urinmasining tenglamasi (ellips, to'g'ri chiziq urinish nuqtasi, urinma tenglamasi).\\

T2. ITECH-ning umumiy tenglamasini koordinata boshin parallel ko'chirish bilan soddalastiring (ITECH-ning umumiy tenglamasini parallel ko'chirish formulasi).\\

A1. Uchi koordinata boshida joylashgan va $Oy$ o'qiga nisbatan yuqori yarim tekislikda joylashgan parabolaning tenglamasini tuzing: parametri $p=1/4$.\\

A2. Fokuslari abssissa o'qida va koordinata boshiga nisbatan simmetrik joylashgan ellipsning tenglamasini tuzing: direktrisalar orasidagi masofa $5$ va fokuslari orasidagi masofa $2c=4$.\\

A3. Fokuslari abssissa o'qida va koordinata boshiga nisbatan simmetrik joylashgan giperbolaning tenglamasini tuzing: katta o'qi $2a=16$ va ekssentrisitet $\varepsilon=5/4$.\\

B1. $\frac{x^{2}}{4} - \frac{y^{2}}{5} = 1$, giperbolaning $3x - 2y = 0$ to'g'ri chizig'iga parallel bo'lgan urinmasining tenglamasini tuzing.  \\

B2. $2x + 2y - 3 = 0$ to'g'ri chizig'iga parallel bo'lib $\frac{x^{2}}{16} + \frac{y^{2}}{64} = 1$ giperbolasiga urinib o'tuvchi to'g'ri chiziqning tenglamasini tuzing.  \\

B3. ITECH ning umumiy tenglamasini koordinata sistemasini almashtirmasdan soddalashtiring, tipini aniqlang, obrazi qanday chiziq ekanligini ko'rsating: $7x^{2} - 8xy + y^{2} - 16x - 2y - 51 = 0$\\

C1. $A(\frac{10}{3};\frac{5}{3})$ nuqtasidan $\frac{x^{2}}{20} + \frac{y^{2}}{5} = 1$ ellipsiga yurgizilgan urinmalarning tenglamasini tuzing.  \\

C2. $M(2; - \frac{5}{3})$ nuqta $\frac{x^{2}}{9} + \frac{y^{2}}{5} = 1$ ellipsda joylashgan. $M$ nuqtaning fokal radiuslarida yotuvchi to'g'ri chiziq tenglamalarini tuzing.  \\

C3. Katta o'qi 26 ga, fokuslari $F( - 10;0), F(14;0)$ nuqtalarida joylashgan ellipsning tenglamasini tuzing.  \\

\end{tabular}
\vspace{1cm}


\begin{tabular}{m{17cm}}
\textbf{41-variant}
\newline

T1. Giperbolik paraboloydning to'g'ri chiziq yasovchilari (Giperbolik paraboloydni yasovchi to'g'ri chiziqlar dastasi).\\

T2. Ellipsning polyar koordinatalardagi tenglamasi (polyar koordinatalar sistemasida ellipsning tenglamasi).\\

A1. Fokuslari abssissa o'qida va koordinata boshiga nisbatan simmetrik joylashgan ellipsning tenglamasini tuzing: kichik o'qi $6$, direktrisalar orasidagi masofa $13$.\\

A2. Fokuslari abssissa o'qida va koordinata boshiga nisbatan simmetrik joylashgan ellipsning tenglamasini tuzing: fokuslari orasidagi masofa $2c=6$ va ekssentrisitet $\varepsilon=3/5$.\\

A3. Fokuslari abssissa o'qida va koordinata boshiga nisbatan simmetrik joylashgan ellipsning tenglamasini tuzing: direktrisalar orasidagi masofa $32$ va $\varepsilon=1/9$.\\

B1. $x^{2} + 4y^{2} = 25$ ellipsi bilan $4x - 2y + 23 = 0$ to'g'ri chizig'iga parallel bo'lgan urinma to'g'ri chiziqning tenglamasini tuzing.  \\

B2. Koordinata o'qlarini almashtirmasdan ITECH tenglamasini soddalashtiring, yarim o'qlarnin toping: $4x^{2} - 4xy + 7y^{2} - 26x - 18y + 3 = 0$.\\

B3. $x^{2} - 4y^{2} = 16$ giperbola berilgan. Uning ekssentrisitetini, fokuslarining koordinatalarini toping va asimptotalarining tenglamalarini tuzing.\\

C1. $32x^{2} + 52xy - 7y^{2} + 180 = 0$ ITECH tenglamasini kanonik shaklga olib keling, tipini aniqlang, qanday geometrik obraz ekanligini ko'rsating, chizmasini eski va yangi koordinatalar sistemasiga nisbatan chizing.  \\

C2. $\frac{x^{2}}{3} - \frac{y^{2}}{5} = 1$, giperbolasiga $P(4;2)$ nuqtadan yurgizilgan urinmalarning tenglamasini tuzing.  \\

C3. $y^{2} = 20x$ parabolasining abssissasi 7 ga teng bo'lgan $M$ nuqtasining fokal radiusini toping va fokal radiusi yotgan to'g'ri chiziqning tenglamasini tuzing.  \\

\end{tabular}
\vspace{1cm}


\begin{tabular}{m{17cm}}
\textbf{42-variant}
\newline

T1. Ikkinchi tartibli sirtning umumiy tenglamasi. Markazin aniqlash formulasi.\\

T2. Elliptik paraboloid (parabola, o'q, elliptik paraboloid).\\

A1. Fokuslari abssissa o'qida va koordinata boshiga nisbatan simmetrik joylashgan giperbolaning tenglamasini tuzing: asimptotalar tenglamalari $y=\pm \frac{3}{4}x$ va direktrisalar orasidagi masofa $64/5$.\\

A2. Aylananing $C$ markazi va $R$ radiusini toping: $x^2+y^2-2x+4y-14=0$.\\

A3. Fokuslari abssissa o'qida va koordinata boshiga nisbatan simmetrik joylashgan ellipsning tenglamasini tuzing: katta o'qi $8$, direktrisalar orasidagi masofa $16$.\\

B1. $3x + 10y - 25 = 0$ to'g'ri bilan $\frac{x^{2}}{25} + \frac{y^{2}}{4} = 1$ ellipsning kesishish nuqtalarini toping.  \\

B2. $\rho = \frac{10}{2 - cos\theta}$ polyar tenglamasi bilan qanday chiziq berilganini aniqlang.  \\

B3. $y^{2} = 12x$ paraborolasiga $3x - 2y + 30 = 0$ to'g'ri chizig'iga parallel bo'lgan urinmasining tenglamasini tuzing.  \\

C1. Fokusi $F(7;2)$ nuqtasida joylashgan, mos direktrisasi $x - 5 = 0$ tenglamasi bilan berilgan parabolaning tenglamasini tuzing.  \\

C2. $4x^{2} - 4xy + y^{2} - 6x + 8y + 13 = 0$ ITECH markazga egami? Markazga ega bo'lsa markazini aniqlang?  \\

C3. $\frac{x^{2}}{3} - \frac{y^{2}}{5} = 1$ giperbolasiga $P(1; - 5)$ nuqtasida yurgizilgan urinmalarning tenglamasini tuzing.\\

\end{tabular}
\vspace{1cm}


\begin{tabular}{m{17cm}}
\textbf{43-variant}
\newline

T1. Ellips va uning kanonik tenglamasi (ta'rifi, fokuslari, ellipsning kanonik tenglamasi, ekstsentrisiteti, direktrisalari).\\

T2. ITECH-ning umumiy tenglamasini koordinata o'qlarini burish bilan soddalashtirish (ITECH-ning umumiy tenglamalari, koordinata o'qin burish formulasi, tenglamani kanonik turga olib kelish).\\

A1. Qutb tenglamasi bilan berilgan egri chiziqning tipini aniqlang: $\rho=\frac{1}{3-3\cos\theta}$.\\

A2. Tipini aniqlang: $2x^{2}+3y^{2}+8x-6y+11=0$.\\

A3. Aylana tenglamasini tuzing: aylana $A(2;6)$ nuqtadan o'tadi va markazi $C(-1;2)$ nuqtada joylashgan.\\

B1. Koordinata o'qlarini almashtirmasdan ITECH umumiy tenglamasini soddalashtiring, yarim o'qlarini toping: $13x^{2} + 18xy + 37y^{2} - 26x - 18y + 3 = 0$.  \\

B2. Ellips $3x^{2} + 4y^{2} - 12 = 0$ tenglamasi bilan berilgan. Uning o'qlarining uzunliklarini, fokuslarining koordinatalarini va ekssentrisitetini toping.  \\

B3. $y^{2} = 3x$ parabolasi bilan $\frac{x^{2}}{100} + \frac{y^{2}}{225} = 1$ ellipsining kesishish nuqtalarini toping.  \\

C1. $\frac{x^{2}}{100} + \frac{y^{2}}{36} = 1$ ellipsining o'ng tarafdagi fokusidan 14 ga teng masofada bo'lgan nuqtasini toping.  \\

C2. Agar xohlagan vaqt momentida $M(x;y)$ nuqta $A(8;4)$ nuqtasidan va ordinata o'qidan birxil masofada joylashsa, $M(x;y)$ nuqtaning harakat troektoriyasining tenglamasini tuzing.  \\

C3. $16x^{2} - 9y^{2} - 64x - 54y - 161 = 0$ tenglamasi giperbolaning tenglamasi ekanligini ko'rsating va uning markazi $C$ ni, yarim o'qlarini, ekssentrisitetini toping, asimptotalarining tenglamalarini tuzing.  \\

\end{tabular}
\vspace{1cm}


\begin{tabular}{m{17cm}}
\textbf{44-variant}
\newline

T1. Ellipsoida. Kanonik tenglamasi (ellipsni simmetriya o'qi atrofida aylantirishdan olingan sirt, kanonik tenglamasi).\\

T2. Giperbola. Kanonik tenglamasi (fokuslar, o'qlar, direktrisalar, giperbola, ekstsentrisitet, kanonik tenglamasi).\\

A1. Fokuslari abssissa o'qida va koordinata boshiga nisbatan simmetrik joylashgan ellipsning tenglamasini tuzing: kichik o'qi $24$, fokuslari orasidagi masofa $2c=10$.\\

A2. Giperbola tenglamasi berilgan: $\frac{x^{2}}{16}-\frac{y^{2}}{9}=1$. Uning qutb tenglamasini tuzing.\\

A3. Tipini aniqlang: $9x^{2}+4y^{2}+18x-8y+49=0$.\\

B1. $\rho = \frac{5}{3 - 4cos\theta}$ tenglamasi bilan qanday chiziq berilganini va yarim o'qlarini toping.  \\

B2. $\frac{x^{2}}{16} - \frac{y^{2}}{64} = 1$ giperbolasiga berilgan $10x - 3y + 9 = 0$ to'g'ri chizig'iga parallel bo'lgan urinmasining tenglamasini tuzing.  \\

B3. Koordinata o'qlarini almashtirmasdan ITECH tenglamasini soddalashtiring, qanday geometrik obraz ekanligini ko'rsating: $4x^{2} - 4xy + y^{2} + 4x - 2y + 1 = 0$.  \\

C1. $\frac{x^{2}}{25} + \frac{y^{2}}{16} = 1$, ellipsiga $C(10; - 8)$ nuqtadan yurgizilgan urinmalarining tenglamasini tuzing.  \\

C2. $y^{2} = 20x$ parabolasining $M$ nuqtasini toping, agar uning abssissasi 7 ga teng bo'lsa, fokal radiusini va fokal radiusi joylashgan to'g'rini aniqlang.\\

C3. Giperbolaning ekssentrisiteti $\varepsilon = \frac{13}{12}$, fokusi $F(0;13)$ nuqtasida va mos direktrisasi $13y - 144 = 0$ tenglamasi bilan berilgan bo'lsa, giperbolaning tenglamasini tuzing.  \\

\end{tabular}
\vspace{1cm}


\begin{tabular}{m{17cm}}
\textbf{45-variant}
\newline

T1. ITECH-ning umumiy tenglamasini klassifikatsiyalash (ITECH-ning umumiy tenglamasi, ITECH-ning umumiy tenglamasini soddalashtirish, klassifikatsiyalash).\\

T2. Ikkinshi tartibli aylanma sirtlar (koordinata sistemasi, tekislik, vektor egri chiziq, aylanma sirt).\\

A1. Aylana tenglamasini tuzing: markazi koordinata boshida joylashgan va radiusi $R=3$ ga teng.\\

A2. Uchi koordinata boshida joylashgan va $Ox$ o'qiga nisbatan o'ng tarafafgi yarim tekislikda joylashgan parabolaning tenglamasini tuzing: parametri $p=3$.\\

A3. Ellips tenglamasi berilgan: $\frac{x^2}{25}+\frac{y^2}{16}=1$. Uning qutb tenglamasini tuzing.\\

B1. $3x + 4y - 12 = 0$ to'g'ri chizig'i va $y^{2} = - 9x$ parabolasining kesishish nuqtalarini toping.\\

B2. $\rho = \frac{6}{1 - cos\theta}$ polyar tenglamasi bilan qanday chiziq berilganini aniqlang.  \\

B3. $x^{2} - y^{2} = 27$ giperbolasiga $4x + 2y - 7 = 0$ to'g'ri chizigiga parallel bo'lgan urinmasining tenglamasini toping.  \\

C1. $14x^{2} + 24xy + 21y^{2} - 4x + 18y - 139 = 0$ egri chizig'ining tipini aniqlang, agar markazga ega egri chiziq bo'lsa, markazining koordinatalarini toping.  \\

C2. Fokusi $F( - 1; - 4)$ nuqtasida joylashgan, mos direktrisasi $x - 2 = 0$ tenglamasi bilan berilgan, $A( - 3; - 5)$ nuqtadan o'tuvchi ellipsning tenglamasini tuzing.  \\

C3. $4x^{2} + 24xy + 11y^{2} + 64x + 42y + 51 = 0$ egri chizig'ining tipini aniqlang, agar markazga ega bo'lsa, uning markazining koordinatalarini toping va koordinata boshini markazga parallel ko'chirish amalini bajaring.\\

\end{tabular}
\vspace{1cm}


\begin{tabular}{m{17cm}}
\textbf{46-variant}
\newline

T1. Giperbolaning urinmasining tenglamasi (giperbolaga berilgan nuqtada yurgizilgan urinma tenglamasi).\\

T2. ITECH-ning invariantlari (ITECH-ning umumiy tenglamasi, almashtirish, ITECH invariantlari).\\

A1. Tipini aniqlang: $25x^{2}-20xy+4y^{2}-12x+20y-17=0$.\\

A2. Aylana tenglamasini tuzing: $A(3;1)$ va $B(-1;3)$ nuqtalardan o'tadi, markazi $3x-y-2=0$ togri chiziqda joylashgan.\\

A3. Fokuslari abssissa o'qida va koordinata boshiga nisbatan simmetrik joylashgan giperbolaning tenglamasini tuzing: asimptotalar tenglamalari $y=\pm \frac{4}{3}x$ va fokuslari orasidagi masofa $2c=20$.\\

B1. $41x^{2} + 24xy + 9y^{2} + 24x + 18y - 36 = 0$ ITECH tipini aniqlang va markazlarini toping koordinata o'qlarini almashtirmasdan qanday chiziq ekanligini ko'rsating, yarim o'qlarini toping.  \\

B2. $3x + 4y - 12 = 0$ to'g'ri chizig'i bilan $y^{2} = - 9x$ parabolasining kesishish nuqtalarini toping.  \\

B3. $\rho = \frac{144}{13 - 5cos\theta}$ ellips ekanligini ko'rsating va uning yarim o'qlarini aniqlang.\\

C1. Uchi (-4;0) nuqtasinda, direktrisasi $y - 2 = 0$ to'g'ri chiziq bo'lgan parabolaning tenglamasini tuzing.\\

C2. $2x^{2} + 3y^{2} + 8x - 6y + 11 = 0$ tenglamasi bilan qanday tipdagi chiziq berilganini aniqlang va uning tenglamasini soddalashtiring va grafigini chizing.  \\

C3. $M(2; - \frac{5}{3})$ nuqta $\frac{x^{2}}{9} + \frac{y^{2}}{5} = 1$ ellipsda joylashgan. $M$ nuqtaning fokal radiuslarida yotuvchi to'g'ri chiziq tenglamalarini tuzing.  \\

\end{tabular}
\vspace{1cm}


\begin{tabular}{m{17cm}}
\textbf{47-variant}
\newline

T1. Sirtning kanonik tenglamalari. Sirt haqqida tushuncha. (Sirtning ta'rifi, formulalari, o'q, yo'naltiruvchi to'g'ri chiziqlar).\\

T2. Parabola va uning kanonik tenglamasi ( ta'rifi, fokusi, direktrisasi, kanonik tenglamasi).\\

A1. Qutb tenglamasi bilan berilgan egri chiziqning tipini aniqlang: $\rho=\frac{10}{1-\frac{3}{2}\cos\theta}$.\\

A2. Berilgan chiziqlarning markaziy ekanligini ko'rsating va markazinin toping: $9x^{2}-4xy-7y^{2}-12=0$.\\

A3. Aylananing $C$ markazi va $R$ radiusini toping: $x^2+y^2+4x-2y+5=0$.\\

B1. $\frac{x^{2}}{20} - \frac{y^{2}}{5} = 1$ giperbolasiga $4x + 3y - 7 = 0$ to'g'ri chizig'iga perpendikulyar bo'lgan urinmasining tenglamasini tuzing.  \\

B2. ITECH ning umumiy tenglamasini koordinata sistemasini almashtirmasdan soddalashtiring, tipini aniqlang, obrazi qanday chiziq ekanligini ko'rsating: $7x^{2} - 8xy + y^{2} - 16x - 2y - 51 = 0$\\

B3. $\frac{x^{2}}{4} - \frac{y^{2}}{5} = 1$ giperbolasiga $3x + 2y = 0$ to'g'ri chizig'iga perpendikulyar bo'lgan urinma to'g'ri chiziqning tenglamasini tuzing.\\

C1. Fokusi $F(2; - 1)$ nuqtasida joylashgan, mos direktrisasi $x - y - 1 = 0$ tenglamasi bilan berilgan parabolaning tenglamasini tuzing.  \\

C2. $4x^{2} - 4xy + y^{2} - 2x - 14y + 7 = 0$ ITECH tenglamasini kanonik shaklga olib keling, tipini aniqlang, qanday geometrik obraz ekanligini ko'rsating, chizmasini eski va yangi koordinatalar sistemasiga nisbatan chizing.  \\

C3. $A(\frac{10}{3};\frac{5}{3})$ nuqtasidan $\frac{x^{2}}{20} + \frac{y^{2}}{5} = 1$ ellipsiga yurgizilgan urinmalarning tenglamasini tuzing.  \\

\end{tabular}
\vspace{1cm}


\begin{tabular}{m{17cm}}
\textbf{48-variant}
\newline

T1. ITECH-ning markazini aniqlash formulasi (ITECH-ning umumiy tenglamasi, markazini aniqlash formulasi).\\

T2. Ikki pallali giperboloid Kanonik tenglamasi (giperbolani simmetriya o'qi atrofida aylantirishdan olingan sirt).\\

A1. Fokuslari abssissa o'qida va koordinata boshiga nisbatan simmetrik joylashgan giperbolaning tenglamasini tuzing: direktrisalar orasidagi masofa $32/5$ va o'qi $2b=6$.\\

A2. Qutb tenglamasi bilan berilgan egri chiziqning tipini aniqlang: $\rho=\frac{12}{2-\cos\theta}$.\\

A3. Berilgan chiziqlarning markaziy ekanligini ko'rsating va markazinin toping: $2x^{2}-6xy+5y^{2}+22x-36y+11=0$.\\

B1. $2x + 2y - 3 = 0$ to'g'ri chizig'iga perpendikulyar bo'lib $x^{2} = 16y$ parabolasiga urinib o'tuvchi to'g'ri chiziqning tenglamasini tuzing.  \\

B2. $\frac{x^{2}}{4} - \frac{y^{2}}{5} = 1$, giperbolaning $3x - 2y = 0$ to'g'ri chizig'iga parallel bo'lgan urinmasining tenglamasini tuzing.  \\

B3. Koordinata o'qlarini almashtirmasdan ITECH tenglamasini soddalashtiring, yarim o'qlarnin toping: $4x^{2} - 4xy + 7y^{2} - 26x - 18y + 3 = 0$.\\

C1. $y^{2} = 20x$ parabolasining abssissasi 7 ga teng bo'lgan $M$ nuqtasining fokal radiusini toping va fokal radiusi yotgan to'g'ri chiziqning tenglamasini tuzing.  \\

C2. Fokusi $F( - 1; - 4)$ nuqtasida bo'lgan, mos direktrisasi $x - 2 = 0$ tenglamasi bilan berilgan, $A( - 3; - 5)$ nuqtadan o'tuvchi ellipsning tenglamasini tuzing.  \\

C3. $32x^{2} + 52xy - 7y^{2} + 180 = 0$ ITECH tenglamasini kanonik shaklga olib keling, tipini aniqlang, qanday geometrik obraz ekanligini ko'rsating, chizmasini eski va yangi koordinatalar sistemasiga nisbatan chizing.  \\

\end{tabular}
\vspace{1cm}


\begin{tabular}{m{17cm}}
\textbf{49-variant}
\newline

T1. Parabolaning urinmasining tenglamasi (parabola, to'g'ri chiziq urinish nuqtasi, urinma tenglamasi).\\

T2. Koordinata sistemasini almashtirish (birlik vektorlar, o'qlar, parallel ko'chirish, koordinata o'qlarinii burish).\\

A1. Aylana tenglamasini tuzing: markazi $C(6;-8)$ nuqtada joylashgan va koordinata boshidan o'tadi.\\

A2. Fokuslari abssissa o'qida va koordinata boshiga nisbatan simmetrik joylashgan ellipsning tenglamasini tuzing: yarim o'qlari 5 va 2.\\

A3. Parabola tenglamasi berilgan: $y^2=6x$. Uning qutb tenglamasini tuzing.\\

B1. $2x + 2y - 3 = 0$ to'g'ri chizig'iga parallel bo'lib $\frac{x^{2}}{16} + \frac{y^{2}}{64} = 1$ giperbolasiga urinib o'tuvchi to'g'ri chiziqning tenglamasini tuzing.  \\

B2. Koordinata o'qlarini almashtirmasdan ITECH umumiy tenglamasini soddalashtiring, yarim o'qlarini toping: $13x^{2} + 18xy + 37y^{2} - 26x - 18y + 3 = 0$.  \\

B3. $x^{2} - 4y^{2} = 16$ giperbola berilgan. Uning ekssentrisitetini, fokuslarining koordinatalarini toping va asimptotalarining tenglamalarini tuzing.\\

C1. $\frac{x^{2}}{3} - \frac{y^{2}}{5} = 1$, giperbolasiga $P(4;2)$ nuqtadan yurgizilgan urinmalarning tenglamasini tuzing.  \\

C2. $\frac{x^{2}}{100} + \frac{y^{2}}{36} = 1$ ellipsining o'ng tarafdagi fokusidan 14 ga teng masofada bo'lgan nuqtasini toping.  \\

C3. Agar vaqtning xohlagan momentida $M(x;y)$ nuqta $5x - 16 = 0$ to'g'ri chiziqqa qaraganda $A(5;0)$ nuqtasidan 1,25 marta uzoqroq masofada joylashgan. Shu $M(x;y)$ nuqtaning harakatining tenglamasini tuzing.  \\

\end{tabular}
\vspace{1cm}


\begin{tabular}{m{17cm}}
\textbf{50-variant}
\newline

T1. Silindrlik sirtlar (yasovchi to'g'ri chiziq, yo'naltiruvchi egri chiziq, silindrlik sirt).\\

T2. ITECH-ning umumiy tenglamasini soddalashtirish (ITECH-ning umumiy tenglamasi, koordinata sistemasin almashtirish ITECH umumiy tenglamasini soddalashtirish).\\

A1. Tipini aniqlang: $4x^2+9y^2-40x+36y+100=0$.\\

A2. Aylananing $C$ markazi va $R$ radiusini toping: $x^2+y^2-2x+4y-20=0$.\\

A3. Fokuslari abssissa o'qida va koordinata boshiga nisbatan simmetrik joylashgan giperbolaning tenglamasini tuzing: fokuslari orasidagi masofasi $2c=10$ va o'qi $2b=8$.\\

B1. $3x + 10y - 25 = 0$ to'g'ri bilan $\frac{x^{2}}{25} + \frac{y^{2}}{4} = 1$ ellipsning kesishish nuqtalarini toping.  \\

B2. $\rho = \frac{10}{2 - cos\theta}$ polyar tenglamasi bilan qanday chiziq berilganini aniqlang.  \\

B3. $x^{2} + 4y^{2} = 25$ ellipsi bilan $4x - 2y + 23 = 0$ to'g'ri chizig'iga parallel bo'lgan urinma to'g'ri chiziqning tenglamasini tuzing.  \\

C1. $4x^{2} - 4xy + y^{2} - 6x + 8y + 13 = 0$ ITECH markazga egami? Markazga ega bo'lsa markazini aniqlang?  \\

C2. $\frac{x^{2}}{3} - \frac{y^{2}}{5} = 1$ giperbolasiga $P(1; - 5)$ nuqtasida yurgizilgan urinmalarning tenglamasini tuzing.\\

C3. $y^{2} = 20x$ parabolasining $M$ nuqtasini toping, agar uning abssissasi 7 ga teng bo'lsa, fokal radiusini va fokal radiusi joylashgan to'g'rini aniqlang.\\

\end{tabular}
\vspace{1cm}


\begin{tabular}{m{17cm}}
\textbf{51-variant}
\newline

T1. Parabolaning polyar koordinatalardagi tenglamasi (polyar koordinata sistemasida parabolaning tenglamasi).\\

T2. ITECH-ning umumiy tenglamasini koordinata boshin parallel ko'chirish bilan soddalastiring (ITECH-ning umumiy tenglamasini parallel ko'chirish formulasi).\\

A1. Qutb tenglamasi bilan berilgan egri chiziqning tipini aniqlang: $\rho=\frac{5}{1-\frac{1}{2}\cos\theta}$.\\

A2. Tipini aniqlang: $2x^{2}+10xy+12y^{2}-7x+18y-15=0$.\\

A3. Aylana tenglamasini tuzing: markazi koordinata boshida joylashgan va $3x-4y+20=0$ to'g'ri chiziqga urinadi.\\

B1. Koordinata o'qlarini almashtirmasdan ITECH tenglamasini soddalashtiring, qanday geometrik obraz ekanligini ko'rsating: $4x^{2} - 4xy + y^{2} + 4x - 2y + 1 = 0$.  \\

B2. Ellips $3x^{2} + 4y^{2} - 12 = 0$ tenglamasi bilan berilgan. Uning o'qlarining uzunliklarini, fokuslarining koordinatalarini va ekssentrisitetini toping.  \\

B3. $y^{2} = 3x$ parabolasi bilan $\frac{x^{2}}{100} + \frac{y^{2}}{225} = 1$ ellipsining kesishish nuqtalarini toping.  \\

C1. Fokuslari $F(3;4)$, $F(-3;-4)$ nuqtalarida joylashgan direktrisalari orasidagi masofa 3,6 ga teng bo'lgan giperbolaning tenglamasini tuzing.  \\

C2. $16x^{2} - 9y^{2} - 64x - 54y - 161 = 0$ tenglamasi giperbolaning tenglamasi ekanligini ko'rsating va uning markazi $C$ ni, yarim o'qlarini, ekssentrisitetini toping, asimptotalarining tenglamalarini tuzing.  \\

C3. $\frac{x^{2}}{25} + \frac{y^{2}}{16} = 1$, ellipsiga $C(10; - 8)$ nuqtadan yurgizilgan urinmalarining tenglamasini tuzing.  \\

\end{tabular}
\vspace{1cm}


\begin{tabular}{m{17cm}}
\textbf{52-variant}
\newline

T1. Bir pallali giperboloid. Kanonik tenglamasi (giperbolani simmetriya o'qi atrofida aylantirishdan olingan sirt).\\

T2. Giperbolaning polyar koordinatadagi tenglamasi (Polyar burchagi, polyar radiusi giperbolaning polyar tenglamasi)\\

A1. Fokuslari abssissa o'qida va koordinata boshiga nisbatan simmetrik joylashgan giperbolaning tenglamasini tuzing: fokuslari orasidagi masofa $2c=6$ va ekssentrisitet $\varepsilon=3/2$.\\

A2. Qutb tenglamasi bilan berilgan egri chiziqning tipini aniqlang: $\rho=\frac{5}{3-4\cos\theta}$.\\

A3. Tipini aniqlang: $3x^{2}-8xy+7y^{2}+8x-15y+20=0$.\\

B1. $\rho = \frac{5}{3 - 4cos\theta}$ tenglamasi bilan qanday chiziq berilganini va yarim o'qlarini toping.  \\

B2. $y^{2} = 12x$ paraborolasiga $3x - 2y + 30 = 0$ to'g'ri chizig'iga parallel bo'lgan urinmasining tenglamasini tuzing.  \\

B3. $41x^{2} + 24xy + 9y^{2} + 24x + 18y - 36 = 0$ ITECH tipini aniqlang va markazlarini toping koordinata o'qlarini almashtirmasdan qanday chiziq ekanligini ko'rsating, yarim o'qlarini toping.  \\

C1. $M(2; - \frac{5}{3})$ nuqta $\frac{x^{2}}{9} + \frac{y^{2}}{5} = 1$ ellipsda joylashgan. $M$ nuqtaning fokal radiuslarida yotuvchi to'g'ri chiziq tenglamalarini tuzing.  \\

C2. Katta o'qi 26 ga, fokuslari $F( - 10;0), F(14;0)$ nuqtalarida joylashgan ellipsning tenglamasini tuzing.  \\

C3. $14x^{2} + 24xy + 21y^{2} - 4x + 18y - 139 = 0$ egri chizig'ining tipini aniqlang, agar markazga ega egri chiziq bo'lsa, markazining koordinatalarini toping.  \\

\end{tabular}
\vspace{1cm}


\begin{tabular}{m{17cm}}
\textbf{53-variant}
\newline

T1. Ikkinchi tartibli sirtning umumiy tenglamasi. Markazin aniqlash formulasi.\\

T2. Giperbolik paraboloydning to'g'ri chiziq yasovchilari (Giperbolik paraboloydni yasovchi to'g'ri chiziqlar dastasi).\\

A1. Aylana tenglamasini tuzing: $M_1(-1;5)$, $M_2(-2;-2)$ va $M_3(5;5)$ nuqtalardan o'tadi.\\

A2. Fokuslari abssissa o'qida va koordinata boshiga nisbatan simmetrik joylashgan ellipsning tenglamasini tuzing: katta o'qi $20$, ekssentrisitet $\varepsilon=3/5$.\\

A3. Giperbola tenglamasi berilgan: $\frac{x^{2}}{25}-\frac{y^{2}}{144}=1$. Uning qutb tenglamasini tuzing.\\

B1. $3x + 4y - 12 = 0$ to'g'ri chizig'i va $y^{2} = - 9x$ parabolasining kesishish nuqtalarini toping.\\

B2. $\rho = \frac{6}{1 - cos\theta}$ polyar tenglamasi bilan qanday chiziq berilganini aniqlang.  \\

B3. $\frac{x^{2}}{16} - \frac{y^{2}}{64} = 1$ giperbolasiga berilgan $10x - 3y + 9 = 0$ to'g'ri chizig'iga parallel bo'lgan urinmasining tenglamasini tuzing.  \\

C1. Fokusi $F(7;2)$ nuqtasida joylashgan, mos direktrisasi $x - 5 = 0$ tenglamasi bilan berilgan parabolaning tenglamasini tuzing.  \\

C2. $4x^{2} + 24xy + 11y^{2} + 64x + 42y + 51 = 0$ egri chizig'ining tipini aniqlang, agar markazga ega bo'lsa, uning markazining koordinatalarini toping va koordinata boshini markazga parallel ko'chirish amalini bajaring.\\

C3. Agar xohlagan vaqt momentida $M(x;y)$ nuqta $A(8;4)$ nuqtasidan va ordinata o'qidan birxil masofada joylashsa, $M(x;y)$ nuqtaning harakat troektoriyasining tenglamasini tuzing.  \\

\end{tabular}
\vspace{1cm}


\begin{tabular}{m{17cm}}
\textbf{54-variant}
\newline

T1. Ellipsning urinmasining tenglamasi (ellips, to'g'ri chiziq urinish nuqtasi, urinma tenglamasi).\\

T2. ITECH-ning umumiy tenglamasini koordinata o'qlarini burish bilan soddalashtirish (ITECH-ning umumiy tenglamalari, koordinata o'qin burish formulasi, tenglamani kanonik turga olib kelish).\\

A1. Berilgan chiziqlarning markaziy ekanligini ko'rsating va markazinin toping: $5x^{2}+4xy+2y^{2}+20x+20y-18=0$.\\

A2. Aylana tenglamasini tuzing: markazi $C(1;-1)$ nuqtada joylashgan va $5x-12y+9-0$ to'g'ri chiziqga urinadi.\\

A3. Uchi koordinata boshida joylashgan va $Ox$ o'qiga nisbatan chap tarafafgi yarim tekislikda joylashgan parabolaning tenglamasini tuzing: parametri $p=0,5$.\\

B1. ITECH ning umumiy tenglamasini koordinata sistemasini almashtirmasdan soddalashtiring, tipini aniqlang, obrazi qanday chiziq ekanligini ko'rsating: $7x^{2} - 8xy + y^{2} - 16x - 2y - 51 = 0$\\

B2. $3x + 4y - 12 = 0$ to'g'ri chizig'i bilan $y^{2} = - 9x$ parabolasining kesishish nuqtalarini toping.  \\

B3. $\rho = \frac{144}{13 - 5cos\theta}$ ellips ekanligini ko'rsating va uning yarim o'qlarini aniqlang.\\

C1. $2x^{2} + 3y^{2} + 8x - 6y + 11 = 0$ tenglamasi bilan qanday tipdagi chiziq berilganini aniqlang va uning tenglamasini soddalashtiring va grafigini chizing.  \\

C2. $y^{2} = 20x$ parabolasining abssissasi 7 ga teng bo'lgan $M$ nuqtasining fokal radiusini toping va fokal radiusi yotgan to'g'ri chiziqning tenglamasini tuzing.  \\

C3. Giperbolaning ekssentrisiteti $\varepsilon = \frac{13}{12}$, fokusi $F(0;13)$ nuqtasida va mos direktrisasi $13y - 144 = 0$ tenglamasi bilan berilgan bo'lsa, giperbolaning tenglamasini tuzing.  \\

\end{tabular}
\vspace{1cm}


\begin{tabular}{m{17cm}}
\textbf{55-variant}
\newline

T1. Elliptik paraboloid (parabola, o'q, elliptik paraboloid).\\

T2. Ellipsning polyar koordinatalardagi tenglamasi (polyar koordinatalar sistemasida ellipsning tenglamasi).\\

A1. Qutb tenglamasi bilan berilgan egri chiziqning tipini aniqlang: $\rho=\frac{6}{1-\cos 0}$.\\

A2. Tipini aniqlang: $x^{2}-4xy+4y^{2}+7x-12=0$.\\

A3. Aylana tenglamasini tuzing: $A(1;1)$, $B(1;-1)$ va $C(2;0)$ nuqtalardan o'tadi.\\

B1. $x^{2} - y^{2} = 27$ giperbolasiga $4x + 2y - 7 = 0$ to'g'ri chizigiga parallel bo'lgan urinmasining tenglamasini toping.  \\

B2. Koordinata o'qlarini almashtirmasdan ITECH tenglamasini soddalashtiring, yarim o'qlarnin toping: $4x^{2} - 4xy + 7y^{2} - 26x - 18y + 3 = 0$.\\

B3. $\frac{x^{2}}{20} - \frac{y^{2}}{5} = 1$ giperbolasiga $4x + 3y - 7 = 0$ to'g'ri chizig'iga perpendikulyar bo'lgan urinmasining tenglamasini tuzing.  \\

C1. $4x^{2} - 4xy + y^{2} - 2x - 14y + 7 = 0$ ITECH tenglamasini kanonik shaklga olib keling, tipini aniqlang, qanday geometrik obraz ekanligini ko'rsating, chizmasini eski va yangi koordinatalar sistemasiga nisbatan chizing.  \\

C2. $A(\frac{10}{3};\frac{5}{3})$ nuqtasidan $\frac{x^{2}}{20} + \frac{y^{2}}{5} = 1$ ellipsiga yurgizilgan urinmalarning tenglamasini tuzing.  \\

C3. $\frac{x^{2}}{100} + \frac{y^{2}}{36} = 1$ ellipsining o'ng tarafdagi fokusidan 14 ga teng masofada bo'lgan nuqtasini toping.  \\

\end{tabular}
\vspace{1cm}


\begin{tabular}{m{17cm}}
\textbf{56-variant}
\newline

T1. ITECH-ning umumiy tenglamasini klassifikatsiyalash (ITECH-ning umumiy tenglamasi, ITECH-ning umumiy tenglamasini soddalashtirish, klassifikatsiyalash).\\

T2. Ellipsoida. Kanonik tenglamasi (ellipsni simmetriya o'qi atrofida aylantirishdan olingan sirt, kanonik tenglamasi).\\

A1. Fokuslari abssissa o'qida va koordinata boshiga nisbatan simmetrik joylashgan ellipsning tenglamasini tuzing: katta o'qi $10$, fokuslari orasidagi masofa $2c=8$.\\

A2. Tipini aniqlang: $9x^{2}-16y^{2}-54x-64y-127=0$.\\

A3. Aylananing $C$ markazi va $R$ radiusini toping: $x^2+y^2+6x-4y+14=0$.\\

B1. $\frac{x^{2}}{4} - \frac{y^{2}}{5} = 1$ giperbolasiga $3x + 2y = 0$ to'g'ri chizig'iga perpendikulyar bo'lgan urinma to'g'ri chiziqning tenglamasini tuzing.\\

B2. $2x + 2y - 3 = 0$ to'g'ri chizig'iga perpendikulyar bo'lib $x^{2} = 16y$ parabolasiga urinib o'tuvchi to'g'ri chiziqning tenglamasini tuzing.  \\

B3. Koordinata o'qlarini almashtirmasdan ITECH umumiy tenglamasini soddalashtiring, yarim o'qlarini toping: $13x^{2} + 18xy + 37y^{2} - 26x - 18y + 3 = 0$.  \\

C1. Fokusi $F( - 1; - 4)$ nuqtasida joylashgan, mos direktrisasi $x - 2 = 0$ tenglamasi bilan berilgan, $A( - 3; - 5)$ nuqtadan o'tuvchi ellipsning tenglamasini tuzing.  \\

C2. $32x^{2} + 52xy - 7y^{2} + 180 = 0$ ITECH tenglamasini kanonik shaklga olib keling, tipini aniqlang, qanday geometrik obraz ekanligini ko'rsating, chizmasini eski va yangi koordinatalar sistemasiga nisbatan chizing.  \\

C3. $\frac{x^{2}}{3} - \frac{y^{2}}{5} = 1$, giperbolasiga $P(4;2)$ nuqtadan yurgizilgan urinmalarning tenglamasini tuzing.  \\

\end{tabular}
\vspace{1cm}


\begin{tabular}{m{17cm}}
\textbf{57-variant}
\newline

T1. Ellips va uning kanonik tenglamasi (ta'rifi, fokuslari, ellipsning kanonik tenglamasi, ekstsentrisiteti, direktrisalari).\\

T2. ITECH-ning invariantlari (ITECH-ning umumiy tenglamasi, almashtirish, ITECH invariantlari).\\

A1. Fokuslari abssissa o'qida va koordinata boshiga nisbatan simmetrik joylashgan giperbolaning tenglamasini tuzing: direktrisalar orasidagi masofa $8/3$ va ekssentrisitet $\varepsilon=3/2$.\\

A2. Tipini aniqlang: $5x^{2}+14xy+11y^{2}+12x-7y+19=0$.\\

A3. Aylana tenglamasini tuzing: aylana diametrining uchlari $A(3;2)$ va $B(-1;6)$ nuqtalarda joylashgan.\\

B1. $\frac{x^{2}}{4} - \frac{y^{2}}{5} = 1$, giperbolaning $3x - 2y = 0$ to'g'ri chizig'iga parallel bo'lgan urinmasining tenglamasini tuzing.  \\

B2. Koordinata o'qlarini almashtirmasdan ITECH tenglamasini soddalashtiring, qanday geometrik obraz ekanligini ko'rsating: $4x^{2} - 4xy + y^{2} + 4x - 2y + 1 = 0$.  \\

B3. $x^{2} - 4y^{2} = 16$ giperbola berilgan. Uning ekssentrisitetini, fokuslarining koordinatalarini toping va asimptotalarining tenglamalarini tuzing.\\

C1. $y^{2} = 20x$ parabolasining $M$ nuqtasini toping, agar uning abssissasi 7 ga teng bo'lsa, fokal radiusini va fokal radiusi joylashgan to'g'rini aniqlang.\\

C2. Uchi (-4;0) nuqtasinda, direktrisasi $y - 2 = 0$ to'g'ri chiziq bo'lgan parabolaning tenglamasini tuzing.\\

C3. $4x^{2} - 4xy + y^{2} - 6x + 8y + 13 = 0$ ITECH markazga egami? Markazga ega bo'lsa markazini aniqlang?  \\

\end{tabular}
\vspace{1cm}


\begin{tabular}{m{17cm}}
\textbf{58-variant}
\newline

T1. Ikkinshi tartibli aylanma sirtlar (koordinata sistemasi, tekislik, vektor egri chiziq, aylanma sirt).\\

T2. Giperbola. Kanonik tenglamasi (fokuslar, o'qlar, direktrisalar, giperbola, ekstsentrisitet, kanonik tenglamasi).\\

A1. Fokuslari abssissa o'qida va koordinata boshiga nisbatan simmetrik joylashgan giperbolaning tenglamasini tuzing: o'qlari $2a=10$ va $2b=8$.\\

A2. Tipini aniqlang: $4x^{2}-y^{2}+8x-2y+3=0$.\\

A3. Aylana tenglamasini tuzing: markazi $C(2;-3)$ nuqtada joylashgan va radiusi $R=7$ ga teng.\\

B1. $3x + 10y - 25 = 0$ to'g'ri bilan $\frac{x^{2}}{25} + \frac{y^{2}}{4} = 1$ ellipsning kesishish nuqtalarini toping.  \\

B2. $\rho = \frac{10}{2 - cos\theta}$ polyar tenglamasi bilan qanday chiziq berilganini aniqlang.  \\

B3. $2x + 2y - 3 = 0$ to'g'ri chizig'iga parallel bo'lib $\frac{x^{2}}{16} + \frac{y^{2}}{64} = 1$ giperbolasiga urinib o'tuvchi to'g'ri chiziqning tenglamasini tuzing.  \\

C1. $\frac{x^{2}}{3} - \frac{y^{2}}{5} = 1$ giperbolasiga $P(1; - 5)$ nuqtasida yurgizilgan urinmalarning tenglamasini tuzing.\\

C2. $M(2; - \frac{5}{3})$ nuqta $\frac{x^{2}}{9} + \frac{y^{2}}{5} = 1$ ellipsda joylashgan. $M$ nuqtaning fokal radiuslarida yotuvchi to'g'ri chiziq tenglamalarini tuzing.  \\

C3. Fokusi $F(2; - 1)$ nuqtasida joylashgan, mos direktrisasi $x - y - 1 = 0$ tenglamasi bilan berilgan parabolaning tenglamasini tuzing.  \\

\end{tabular}
\vspace{1cm}


\begin{tabular}{m{17cm}}
\textbf{59-variant}
\newline

T1. ITECH-ning markazini aniqlash formulasi (ITECH-ning umumiy tenglamasi, markazini aniqlash formulasi).\\

T2. Sirtning kanonik tenglamalari. Sirt haqqida tushuncha. (Sirtning ta'rifi, formulalari, o'q, yo'naltiruvchi to'g'ri chiziqlar).\\

A1. Fokuslari abssissa o'qida va koordinata boshiga nisbatan simmetrik joylashgan giperbolaning tenglamasini tuzing: direktrisalar orasidagi masofa $228/13$ va fokuslari orasidagi masofa $2c=26$.\\

A2. Tipini aniqlang: $3x^{2}-2xy-3y^{2}+12y-15=0$.\\

A3. Fokuslari abssissa o'qida va koordinata boshiga nisbatan simmetrik joylashgan ellipsning tenglamasini tuzing: kichik o'qi $10$, ekssentrisitet $\varepsilon=12/13$.\\

B1. $41x^{2} + 24xy + 9y^{2} + 24x + 18y - 36 = 0$ ITECH tipini aniqlang va markazlarini toping koordinata o'qlarini almashtirmasdan qanday chiziq ekanligini ko'rsating, yarim o'qlarini toping.  \\

B2. Ellips $3x^{2} + 4y^{2} - 12 = 0$ tenglamasi bilan berilgan. Uning o'qlarining uzunliklarini, fokuslarining koordinatalarini va ekssentrisitetini toping.  \\

B3. $y^{2} = 3x$ parabolasi bilan $\frac{x^{2}}{100} + \frac{y^{2}}{225} = 1$ ellipsining kesishish nuqtalarini toping.  \\

C1. $16x^{2} - 9y^{2} - 64x - 54y - 161 = 0$ tenglamasi giperbolaning tenglamasi ekanligini ko'rsating va uning markazi $C$ ni, yarim o'qlarini, ekssentrisitetini toping, asimptotalarining tenglamalarini tuzing.  \\

C2. $\frac{x^{2}}{25} + \frac{y^{2}}{16} = 1$, ellipsiga $C(10; - 8)$ nuqtadan yurgizilgan urinmalarining tenglamasini tuzing.  \\

C3. $y^{2} = 20x$ parabolasining abssissasi 7 ga teng bo'lgan $M$ nuqtasining fokal radiusini toping va fokal radiusi yotgan to'g'ri chiziqning tenglamasini tuzing.  \\

\end{tabular}
\vspace{1cm}


\begin{tabular}{m{17cm}}
\textbf{60-variant}
\newline

T1. Giperbolaning urinmasining tenglamasi (giperbolaga berilgan nuqtada yurgizilgan urinma tenglamasi).\\

T2. Koordinata sistemasini almashtirish (birlik vektorlar, o'qlar, parallel ko'chirish, koordinata o'qlarinii burish).\\

A1. Berilgan chiziqlarning markaziy ekanligini ko'rsating va markazinin toping: $3x^{2}+5xy+y^{2}-8x-11y-7=0$.\\

A2. Uchi koordinata boshida joylashgan va $Oy$ o'qiga nisbatan quyi tarafafgi yarim tekislikda joylashgan parabolaning tenglamasini tuzing: parametri $p=3$.\\

A3. Uchi koordinata boshida joylashgan va $Oy$ o'qiga nisbatan yuqori yarim tekislikda joylashgan parabolaning tenglamasini tuzing: parametri $p=1/4$.\\

B1. $\rho = \frac{5}{3 - 4cos\theta}$ tenglamasi bilan qanday chiziq berilganini va yarim o'qlarini toping.  \\

B2. $x^{2} + 4y^{2} = 25$ ellipsi bilan $4x - 2y + 23 = 0$ to'g'ri chizig'iga parallel bo'lgan urinma to'g'ri chiziqning tenglamasini tuzing.  \\

B3. ITECH ning umumiy tenglamasini koordinata sistemasini almashtirmasdan soddalashtiring, tipini aniqlang, obrazi qanday chiziq ekanligini ko'rsating: $7x^{2} - 8xy + y^{2} - 16x - 2y - 51 = 0$\\

C1. Fokusi $F( - 1; - 4)$ nuqtasida bo'lgan, mos direktrisasi $x - 2 = 0$ tenglamasi bilan berilgan, $A( - 3; - 5)$ nuqtadan o'tuvchi ellipsning tenglamasini tuzing.  \\

C2. $14x^{2} + 24xy + 21y^{2} - 4x + 18y - 139 = 0$ egri chizig'ining tipini aniqlang, agar markazga ega egri chiziq bo'lsa, markazining koordinatalarini toping.  \\

C3. Agar vaqtning xohlagan momentida $M(x;y)$ nuqta $5x - 16 = 0$ to'g'ri chiziqqa qaraganda $A(5;0)$ nuqtasidan 1,25 marta uzoqroq masofada joylashgan. Shu $M(x;y)$ nuqtaning harakatining tenglamasini tuzing.  \\

\end{tabular}
\vspace{1cm}


\begin{tabular}{m{17cm}}
\textbf{61-variant}
\newline

T1. Ikki pallali giperboloid Kanonik tenglamasi (giperbolani simmetriya o'qi atrofida aylantirishdan olingan sirt).\\

T2. Parabola va uning kanonik tenglamasi ( ta'rifi, fokusi, direktrisasi, kanonik tenglamasi).\\

A1. Fokuslari abssissa o'qida va koordinata boshiga nisbatan simmetrik joylashgan ellipsning tenglamasini tuzing: direktrisalar orasidagi masofa $5$ va fokuslari orasidagi masofa $2c=4$.\\

A2. Fokuslari abssissa o'qida va koordinata boshiga nisbatan simmetrik joylashgan giperbolaning tenglamasini tuzing: katta o'qi $2a=16$ va ekssentrisitet $\varepsilon=5/4$.\\

A3. Fokuslari abssissa o'qida va koordinata boshiga nisbatan simmetrik joylashgan ellipsning tenglamasini tuzing: kichik o'qi $6$, direktrisalar orasidagi masofa $13$.\\

B1. $3x + 4y - 12 = 0$ to'g'ri chizig'i va $y^{2} = - 9x$ parabolasining kesishish nuqtalarini toping.\\

B2. $\rho = \frac{6}{1 - cos\theta}$ polyar tenglamasi bilan qanday chiziq berilganini aniqlang.  \\

B3. $y^{2} = 12x$ paraborolasiga $3x - 2y + 30 = 0$ to'g'ri chizig'iga parallel bo'lgan urinmasining tenglamasini tuzing.  \\

C1. $4x^{2} + 24xy + 11y^{2} + 64x + 42y + 51 = 0$ egri chizig'ining tipini aniqlang, agar markazga ega bo'lsa, uning markazining koordinatalarini toping va koordinata boshini markazga parallel ko'chirish amalini bajaring.\\

C2. Fokuslari $F(3;4)$, $F(-3;-4)$ nuqtalarida joylashgan direktrisalari orasidagi masofa 3,6 ga teng bo'lgan giperbolaning tenglamasini tuzing.  \\

C3. $2x^{2} + 3y^{2} + 8x - 6y + 11 = 0$ tenglamasi bilan qanday tipdagi chiziq berilganini aniqlang va uning tenglamasini soddalashtiring va grafigini chizing.  \\

\end{tabular}
\vspace{1cm}


\begin{tabular}{m{17cm}}
\textbf{62-variant}
\newline

T1. ITECH-ning umumiy tenglamasini soddalashtirish (ITECH-ning umumiy tenglamasi, koordinata sistemasin almashtirish ITECH umumiy tenglamasini soddalashtirish).\\

T2. Silindrlik sirtlar (yasovchi to'g'ri chiziq, yo'naltiruvchi egri chiziq, silindrlik sirt).\\

A1. Fokuslari abssissa o'qida va koordinata boshiga nisbatan simmetrik joylashgan ellipsning tenglamasini tuzing: fokuslari orasidagi masofa $2c=6$ va ekssentrisitet $\varepsilon=3/5$.\\

A2. Fokuslari abssissa o'qida va koordinata boshiga nisbatan simmetrik joylashgan ellipsning tenglamasini tuzing: direktrisalar orasidagi masofa $32$ va $\varepsilon=1/9$.\\

A3. Fokuslari abssissa o'qida va koordinata boshiga nisbatan simmetrik joylashgan giperbolaning tenglamasini tuzing: asimptotalar tenglamalari $y=\pm \frac{3}{4}x$ va direktrisalar orasidagi masofa $64/5$.\\

B1. Koordinata o'qlarini almashtirmasdan ITECH tenglamasini soddalashtiring, yarim o'qlarnin toping: $4x^{2} - 4xy + 7y^{2} - 26x - 18y + 3 = 0$.\\

B2. $3x + 4y - 12 = 0$ to'g'ri chizig'i bilan $y^{2} = - 9x$ parabolasining kesishish nuqtalarini toping.  \\

B3. $\rho = \frac{144}{13 - 5cos\theta}$ ellips ekanligini ko'rsating va uning yarim o'qlarini aniqlang.\\

C1. $\frac{x^{2}}{100} + \frac{y^{2}}{36} = 1$ ellipsining o'ng tarafdagi fokusidan 14 ga teng masofada bo'lgan nuqtasini toping.  \\

C2. Katta o'qi 26 ga, fokuslari $F( - 10;0), F(14;0)$ nuqtalarida joylashgan ellipsning tenglamasini tuzing.  \\

C3. $4x^{2} - 4xy + y^{2} - 2x - 14y + 7 = 0$ ITECH tenglamasini kanonik shaklga olib keling, tipini aniqlang, qanday geometrik obraz ekanligini ko'rsating, chizmasini eski va yangi koordinatalar sistemasiga nisbatan chizing.  \\

\end{tabular}
\vspace{1cm}


\begin{tabular}{m{17cm}}
\textbf{63-variant}
\newline

T1. ITECH-ning umumiy tenglamasini koordinata boshin parallel ko'chirish bilan soddalastiring (ITECH-ning umumiy tenglamasini parallel ko'chirish formulasi).\\

T2. Parabolaning urinmasining tenglamasi (parabola, to'g'ri chiziq urinish nuqtasi, urinma tenglamasi).\\

A1. Aylananing $C$ markazi va $R$ radiusini toping: $x^2+y^2-2x+4y-14=0$.\\

A2. Fokuslari abssissa o'qida va koordinata boshiga nisbatan simmetrik joylashgan ellipsning tenglamasini tuzing: katta o'qi $8$, direktrisalar orasidagi masofa $16$.\\

A3. Qutb tenglamasi bilan berilgan egri chiziqning tipini aniqlang: $\rho=\frac{1}{3-3\cos\theta}$.\\

B1. $\frac{x^{2}}{16} - \frac{y^{2}}{64} = 1$ giperbolasiga berilgan $10x - 3y + 9 = 0$ to'g'ri chizig'iga parallel bo'lgan urinmasining tenglamasini tuzing.  \\

B2. Koordinata o'qlarini almashtirmasdan ITECH umumiy tenglamasini soddalashtiring, yarim o'qlarini toping: $13x^{2} + 18xy + 37y^{2} - 26x - 18y + 3 = 0$.  \\

B3. $x^{2} - y^{2} = 27$ giperbolasiga $4x + 2y - 7 = 0$ to'g'ri chizigiga parallel bo'lgan urinmasining tenglamasini toping.  \\

C1. $A(\frac{10}{3};\frac{5}{3})$ nuqtasidan $\frac{x^{2}}{20} + \frac{y^{2}}{5} = 1$ ellipsiga yurgizilgan urinmalarning tenglamasini tuzing.  \\

C2. $y^{2} = 20x$ parabolasining $M$ nuqtasini toping, agar uning abssissasi 7 ga teng bo'lsa, fokal radiusini va fokal radiusi joylashgan to'g'rini aniqlang.\\

C3. Fokusi $F(7;2)$ nuqtasida joylashgan, mos direktrisasi $x - 5 = 0$ tenglamasi bilan berilgan parabolaning tenglamasini tuzing.  \\

\end{tabular}
\vspace{1cm}


\begin{tabular}{m{17cm}}
\textbf{64-variant}
\newline

T1. Ikkinchi tartibli sirtning umumiy tenglamasi. Markazin aniqlash formulasi.\\

T2. Bir pallali giperboloid. Kanonik tenglamasi (giperbolani simmetriya o'qi atrofida aylantirishdan olingan sirt).\\

A1. Tipini aniqlang: $2x^{2}+3y^{2}+8x-6y+11=0$.\\

A2. Aylana tenglamasini tuzing: aylana $A(2;6)$ nuqtadan o'tadi va markazi $C(-1;2)$ nuqtada joylashgan.\\

A3. Fokuslari abssissa o'qida va koordinata boshiga nisbatan simmetrik joylashgan ellipsning tenglamasini tuzing: kichik o'qi $24$, fokuslari orasidagi masofa $2c=10$.\\

B1. $\frac{x^{2}}{20} - \frac{y^{2}}{5} = 1$ giperbolasiga $4x + 3y - 7 = 0$ to'g'ri chizig'iga perpendikulyar bo'lgan urinmasining tenglamasini tuzing.  \\

B2. $\frac{x^{2}}{4} - \frac{y^{2}}{5} = 1$ giperbolasiga $3x + 2y = 0$ to'g'ri chizig'iga perpendikulyar bo'lgan urinma to'g'ri chiziqning tenglamasini tuzing.\\

B3. Koordinata o'qlarini almashtirmasdan ITECH tenglamasini soddalashtiring, qanday geometrik obraz ekanligini ko'rsating: $4x^{2} - 4xy + y^{2} + 4x - 2y + 1 = 0$.  \\

C1. $32x^{2} + 52xy - 7y^{2} + 180 = 0$ ITECH tenglamasini kanonik shaklga olib keling, tipini aniqlang, qanday geometrik obraz ekanligini ko'rsating, chizmasini eski va yangi koordinatalar sistemasiga nisbatan chizing.  \\

C2. $\frac{x^{2}}{3} - \frac{y^{2}}{5} = 1$, giperbolasiga $P(4;2)$ nuqtadan yurgizilgan urinmalarning tenglamasini tuzing.  \\

C3. $M(2; - \frac{5}{3})$ nuqta $\frac{x^{2}}{9} + \frac{y^{2}}{5} = 1$ ellipsda joylashgan. $M$ nuqtaning fokal radiuslarida yotuvchi to'g'ri chiziq tenglamalarini tuzing.  \\

\end{tabular}
\vspace{1cm}


\begin{tabular}{m{17cm}}
\textbf{65-variant}
\newline

T1. Parabolaning polyar koordinatalardagi tenglamasi (polyar koordinata sistemasida parabolaning tenglamasi).\\

T2. ITECH-ning umumiy tenglamasini koordinata o'qlarini burish bilan soddalashtirish (ITECH-ning umumiy tenglamalari, koordinata o'qin burish formulasi, tenglamani kanonik turga olib kelish).\\

A1. Giperbola tenglamasi berilgan: $\frac{x^{2}}{16}-\frac{y^{2}}{9}=1$. Uning qutb tenglamasini tuzing.\\

A2. Tipini aniqlang: $9x^{2}+4y^{2}+18x-8y+49=0$.\\

A3. Aylana tenglamasini tuzing: markazi koordinata boshida joylashgan va radiusi $R=3$ ga teng.\\

B1. $2x + 2y - 3 = 0$ to'g'ri chizig'iga perpendikulyar bo'lib $x^{2} = 16y$ parabolasiga urinib o'tuvchi to'g'ri chiziqning tenglamasini tuzing.  \\

B2. $41x^{2} + 24xy + 9y^{2} + 24x + 18y - 36 = 0$ ITECH tipini aniqlang va markazlarini toping koordinata o'qlarini almashtirmasdan qanday chiziq ekanligini ko'rsating, yarim o'qlarini toping.  \\

B3. $x^{2} - 4y^{2} = 16$ giperbola berilgan. Uning ekssentrisitetini, fokuslarining koordinatalarini toping va asimptotalarining tenglamalarini tuzing.\\

C1. Agar xohlagan vaqt momentida $M(x;y)$ nuqta $A(8;4)$ nuqtasidan va ordinata o'qidan birxil masofada joylashsa, $M(x;y)$ nuqtaning harakat troektoriyasining tenglamasini tuzing.  \\

C2. $4x^{2} - 4xy + y^{2} - 6x + 8y + 13 = 0$ ITECH markazga egami? Markazga ega bo'lsa markazini aniqlang?  \\

C3. $\frac{x^{2}}{3} - \frac{y^{2}}{5} = 1$ giperbolasiga $P(1; - 5)$ nuqtasida yurgizilgan urinmalarning tenglamasini tuzing.\\

\end{tabular}
\vspace{1cm}


\begin{tabular}{m{17cm}}
\textbf{66-variant}
\newline

T1. Giperbolik paraboloydning to'g'ri chiziq yasovchilari (Giperbolik paraboloydni yasovchi to'g'ri chiziqlar dastasi).\\

T2. Giperbolaning polyar koordinatadagi tenglamasi (Polyar burchagi, polyar radiusi giperbolaning polyar tenglamasi)\\

A1. Uchi koordinata boshida joylashgan va $Ox$ o'qiga nisbatan o'ng tarafafgi yarim tekislikda joylashgan parabolaning tenglamasini tuzing: parametri $p=3$.\\

A2. Ellips tenglamasi berilgan: $\frac{x^2}{25}+\frac{y^2}{16}=1$. Uning qutb tenglamasini tuzing.\\

A3. Tipini aniqlang: $25x^{2}-20xy+4y^{2}-12x+20y-17=0$.\\

B1. $3x + 10y - 25 = 0$ to'g'ri bilan $\frac{x^{2}}{25} + \frac{y^{2}}{4} = 1$ ellipsning kesishish nuqtalarini toping.  \\

B2. $\rho = \frac{10}{2 - cos\theta}$ polyar tenglamasi bilan qanday chiziq berilganini aniqlang.  \\

B3. $\frac{x^{2}}{4} - \frac{y^{2}}{5} = 1$, giperbolaning $3x - 2y = 0$ to'g'ri chizig'iga parallel bo'lgan urinmasining tenglamasini tuzing.  \\

C1. $y^{2} = 20x$ parabolasining abssissasi 7 ga teng bo'lgan $M$ nuqtasining fokal radiusini toping va fokal radiusi yotgan to'g'ri chiziqning tenglamasini tuzing.  \\

C2. Giperbolaning ekssentrisiteti $\varepsilon = \frac{13}{12}$, fokusi $F(0;13)$ nuqtasida va mos direktrisasi $13y - 144 = 0$ tenglamasi bilan berilgan bo'lsa, giperbolaning tenglamasini tuzing.  \\

C3. $16x^{2} - 9y^{2} - 64x - 54y - 161 = 0$ tenglamasi giperbolaning tenglamasi ekanligini ko'rsating va uning markazi $C$ ni, yarim o'qlarini, ekssentrisitetini toping, asimptotalarining tenglamalarini tuzing.  \\

\end{tabular}
\vspace{1cm}


\begin{tabular}{m{17cm}}
\textbf{67-variant}
\newline

T1. ITECH-ning umumiy tenglamasini klassifikatsiyalash (ITECH-ning umumiy tenglamasi, ITECH-ning umumiy tenglamasini soddalashtirish, klassifikatsiyalash).\\

T2. Elliptik paraboloid (parabola, o'q, elliptik paraboloid).\\

A1. Aylana tenglamasini tuzing: $A(3;1)$ va $B(-1;3)$ nuqtalardan o'tadi, markazi $3x-y-2=0$ togri chiziqda joylashgan.\\

A2. Fokuslari abssissa o'qida va koordinata boshiga nisbatan simmetrik joylashgan giperbolaning tenglamasini tuzing: asimptotalar tenglamalari $y=\pm \frac{4}{3}x$ va fokuslari orasidagi masofa $2c=20$.\\

A3. Qutb tenglamasi bilan berilgan egri chiziqning tipini aniqlang: $\rho=\frac{10}{1-\frac{3}{2}\cos\theta}$.\\

B1. ITECH ning umumiy tenglamasini koordinata sistemasini almashtirmasdan soddalashtiring, tipini aniqlang, obrazi qanday chiziq ekanligini ko'rsating: $7x^{2} - 8xy + y^{2} - 16x - 2y - 51 = 0$\\

B2. Ellips $3x^{2} + 4y^{2} - 12 = 0$ tenglamasi bilan berilgan. Uning o'qlarining uzunliklarini, fokuslarining koordinatalarini va ekssentrisitetini toping.  \\

B3. $y^{2} = 3x$ parabolasi bilan $\frac{x^{2}}{100} + \frac{y^{2}}{225} = 1$ ellipsining kesishish nuqtalarini toping.  \\

C1. $\frac{x^{2}}{25} + \frac{y^{2}}{16} = 1$, ellipsiga $C(10; - 8)$ nuqtadan yurgizilgan urinmalarining tenglamasini tuzing.  \\

C2. $\frac{x^{2}}{100} + \frac{y^{2}}{36} = 1$ ellipsining o'ng tarafdagi fokusidan 14 ga teng masofada bo'lgan nuqtasini toping.  \\

C3. Fokusi $F( - 1; - 4)$ nuqtasida joylashgan, mos direktrisasi $x - 2 = 0$ tenglamasi bilan berilgan, $A( - 3; - 5)$ nuqtadan o'tuvchi ellipsning tenglamasini tuzing.  \\

\end{tabular}
\vspace{1cm}


\begin{tabular}{m{17cm}}
\textbf{68-variant}
\newline

T1. Ellipsning urinmasining tenglamasi (ellips, to'g'ri chiziq urinish nuqtasi, urinma tenglamasi).\\

T2. ITECH-ning invariantlari (ITECH-ning umumiy tenglamasi, almashtirish, ITECH invariantlari).\\

A1. Berilgan chiziqlarning markaziy ekanligini ko'rsating va markazinin toping: $9x^{2}-4xy-7y^{2}-12=0$.\\

A2. Aylananing $C$ markazi va $R$ radiusini toping: $x^2+y^2+4x-2y+5=0$.\\

A3. Fokuslari abssissa o'qida va koordinata boshiga nisbatan simmetrik joylashgan giperbolaning tenglamasini tuzing: direktrisalar orasidagi masofa $32/5$ va o'qi $2b=6$.\\

B1. $\rho = \frac{5}{3 - 4cos\theta}$ tenglamasi bilan qanday chiziq berilganini va yarim o'qlarini toping.  \\

B2. $2x + 2y - 3 = 0$ to'g'ri chizig'iga parallel bo'lib $\frac{x^{2}}{16} + \frac{y^{2}}{64} = 1$ giperbolasiga urinib o'tuvchi to'g'ri chiziqning tenglamasini tuzing.  \\

B3. Koordinata o'qlarini almashtirmasdan ITECH tenglamasini soddalashtiring, yarim o'qlarnin toping: $4x^{2} - 4xy + 7y^{2} - 26x - 18y + 3 = 0$.\\

C1. $14x^{2} + 24xy + 21y^{2} - 4x + 18y - 139 = 0$ egri chizig'ining tipini aniqlang, agar markazga ega egri chiziq bo'lsa, markazining koordinatalarini toping.  \\

C2. Uchi (-4;0) nuqtasinda, direktrisasi $y - 2 = 0$ to'g'ri chiziq bo'lgan parabolaning tenglamasini tuzing.\\

C3. $4x^{2} + 24xy + 11y^{2} + 64x + 42y + 51 = 0$ egri chizig'ining tipini aniqlang, agar markazga ega bo'lsa, uning markazining koordinatalarini toping va koordinata boshini markazga parallel ko'chirish amalini bajaring.\\

\end{tabular}
\vspace{1cm}


\begin{tabular}{m{17cm}}
\textbf{69-variant}
\newline

T1. Ellipsoida. Kanonik tenglamasi (ellipsni simmetriya o'qi atrofida aylantirishdan olingan sirt, kanonik tenglamasi).\\

T2. Ellipsning polyar koordinatalardagi tenglamasi (polyar koordinatalar sistemasida ellipsning tenglamasi).\\

A1. Qutb tenglamasi bilan berilgan egri chiziqning tipini aniqlang: $\rho=\frac{12}{2-\cos\theta}$.\\

A2. Berilgan chiziqlarning markaziy ekanligini ko'rsating va markazinin toping: $2x^{2}-6xy+5y^{2}+22x-36y+11=0$.\\

A3. Aylana tenglamasini tuzing: markazi $C(6;-8)$ nuqtada joylashgan va koordinata boshidan o'tadi.\\

B1. $3x + 4y - 12 = 0$ to'g'ri chizig'i va $y^{2} = - 9x$ parabolasining kesishish nuqtalarini toping.\\

B2. $\rho = \frac{6}{1 - cos\theta}$ polyar tenglamasi bilan qanday chiziq berilganini aniqlang.  \\

B3. $x^{2} + 4y^{2} = 25$ ellipsi bilan $4x - 2y + 23 = 0$ to'g'ri chizig'iga parallel bo'lgan urinma to'g'ri chiziqning tenglamasini tuzing.  \\

C1. Fokusi $F(2; - 1)$ nuqtasida joylashgan, mos direktrisasi $x - y - 1 = 0$ tenglamasi bilan berilgan parabolaning tenglamasini tuzing.  \\

C2. $2x^{2} + 3y^{2} + 8x - 6y + 11 = 0$ tenglamasi bilan qanday tipdagi chiziq berilganini aniqlang va uning tenglamasini soddalashtiring va grafigini chizing.  \\

C3. $y^{2} = 20x$ parabolasining $M$ nuqtasini toping, agar uning abssissasi 7 ga teng bo'lsa, fokal radiusini va fokal radiusi joylashgan to'g'rini aniqlang.\\

\end{tabular}
\vspace{1cm}


\begin{tabular}{m{17cm}}
\textbf{70-variant}
\newline

T1. ITECH-ning markazini aniqlash formulasi (ITECH-ning umumiy tenglamasi, markazini aniqlash formulasi).\\

T2. Ikkinshi tartibli aylanma sirtlar (koordinata sistemasi, tekislik, vektor egri chiziq, aylanma sirt).\\

A1. Fokuslari abssissa o'qida va koordinata boshiga nisbatan simmetrik joylashgan ellipsning tenglamasini tuzing: yarim o'qlari 5 va 2.\\

A2. Parabola tenglamasi berilgan: $y^2=6x$. Uning qutb tenglamasini tuzing.\\

A3. Tipini aniqlang: $4x^2+9y^2-40x+36y+100=0$.\\

B1. Koordinata o'qlarini almashtirmasdan ITECH umumiy tenglamasini soddalashtiring, yarim o'qlarini toping: $13x^{2} + 18xy + 37y^{2} - 26x - 18y + 3 = 0$.  \\

B2. $3x + 4y - 12 = 0$ to'g'ri chizig'i bilan $y^{2} = - 9x$ parabolasining kesishish nuqtalarini toping.  \\

B3. $\rho = \frac{144}{13 - 5cos\theta}$ ellips ekanligini ko'rsating va uning yarim o'qlarini aniqlang.\\

C1. Fokusi $F( - 1; - 4)$ nuqtasida bo'lgan, mos direktrisasi $x - 2 = 0$ tenglamasi bilan berilgan, $A( - 3; - 5)$ nuqtadan o'tuvchi ellipsning tenglamasini tuzing.  \\

C2. $4x^{2} - 4xy + y^{2} - 2x - 14y + 7 = 0$ ITECH tenglamasini kanonik shaklga olib keling, tipini aniqlang, qanday geometrik obraz ekanligini ko'rsating, chizmasini eski va yangi koordinatalar sistemasiga nisbatan chizing.  \\

C3. $A(\frac{10}{3};\frac{5}{3})$ nuqtasidan $\frac{x^{2}}{20} + \frac{y^{2}}{5} = 1$ ellipsiga yurgizilgan urinmalarning tenglamasini tuzing.  \\

\end{tabular}
\vspace{1cm}


\begin{tabular}{m{17cm}}
\textbf{71-variant}
\newline

T1. Ellips va uning kanonik tenglamasi (ta'rifi, fokuslari, ellipsning kanonik tenglamasi, ekstsentrisiteti, direktrisalari).\\

T2. Koordinata sistemasini almashtirish (birlik vektorlar, o'qlar, parallel ko'chirish, koordinata o'qlarinii burish).\\

A1. Aylananing $C$ markazi va $R$ radiusini toping: $x^2+y^2-2x+4y-20=0$.\\

A2. Fokuslari abssissa o'qida va koordinata boshiga nisbatan simmetrik joylashgan giperbolaning tenglamasini tuzing: fokuslari orasidagi masofasi $2c=10$ va o'qi $2b=8$.\\

A3. Qutb tenglamasi bilan berilgan egri chiziqning tipini aniqlang: $\rho=\frac{5}{1-\frac{1}{2}\cos\theta}$.\\

B1. $y^{2} = 12x$ paraborolasiga $3x - 2y + 30 = 0$ to'g'ri chizig'iga parallel bo'lgan urinmasining tenglamasini tuzing.  \\

B2. Koordinata o'qlarini almashtirmasdan ITECH tenglamasini soddalashtiring, qanday geometrik obraz ekanligini ko'rsating: $4x^{2} - 4xy + y^{2} + 4x - 2y + 1 = 0$.  \\

B3. $\frac{x^{2}}{16} - \frac{y^{2}}{64} = 1$ giperbolasiga berilgan $10x - 3y + 9 = 0$ to'g'ri chizig'iga parallel bo'lgan urinmasining tenglamasini tuzing.  \\

C1. $M(2; - \frac{5}{3})$ nuqta $\frac{x^{2}}{9} + \frac{y^{2}}{5} = 1$ ellipsda joylashgan. $M$ nuqtaning fokal radiuslarida yotuvchi to'g'ri chiziq tenglamalarini tuzing.  \\

C2. Agar vaqtning xohlagan momentida $M(x;y)$ nuqta $5x - 16 = 0$ to'g'ri chiziqqa qaraganda $A(5;0)$ nuqtasidan 1,25 marta uzoqroq masofada joylashgan. Shu $M(x;y)$ nuqtaning harakatining tenglamasini tuzing.  \\

C3. $32x^{2} + 52xy - 7y^{2} + 180 = 0$ ITECH tenglamasini kanonik shaklga olib keling, tipini aniqlang, qanday geometrik obraz ekanligini ko'rsating, chizmasini eski va yangi koordinatalar sistemasiga nisbatan chizing.  \\

\end{tabular}
\vspace{1cm}


\begin{tabular}{m{17cm}}
\textbf{72-variant}
\newline

T1. Sirtning kanonik tenglamalari. Sirt haqqida tushuncha. (Sirtning ta'rifi, formulalari, o'q, yo'naltiruvchi to'g'ri chiziqlar).\\

T2. Giperbola. Kanonik tenglamasi (fokuslar, o'qlar, direktrisalar, giperbola, ekstsentrisitet, kanonik tenglamasi).\\

A1. Tipini aniqlang: $2x^{2}+10xy+12y^{2}-7x+18y-15=0$.\\

A2. Aylana tenglamasini tuzing: markazi koordinata boshida joylashgan va $3x-4y+20=0$ to'g'ri chiziqga urinadi.\\

A3. Fokuslari abssissa o'qida va koordinata boshiga nisbatan simmetrik joylashgan giperbolaning tenglamasini tuzing: fokuslari orasidagi masofa $2c=6$ va ekssentrisitet $\varepsilon=3/2$.\\

B1. $x^{2} - y^{2} = 27$ giperbolasiga $4x + 2y - 7 = 0$ to'g'ri chizigiga parallel bo'lgan urinmasining tenglamasini toping.  \\

B2. $\frac{x^{2}}{20} - \frac{y^{2}}{5} = 1$ giperbolasiga $4x + 3y - 7 = 0$ to'g'ri chizig'iga perpendikulyar bo'lgan urinmasining tenglamasini tuzing.  \\

B3. $41x^{2} + 24xy + 9y^{2} + 24x + 18y - 36 = 0$ ITECH tipini aniqlang va markazlarini toping koordinata o'qlarini almashtirmasdan qanday chiziq ekanligini ko'rsating, yarim o'qlarini toping.  \\

C1. $\frac{x^{2}}{3} - \frac{y^{2}}{5} = 1$, giperbolasiga $P(4;2)$ nuqtadan yurgizilgan urinmalarning tenglamasini tuzing.  \\

C2. $y^{2} = 20x$ parabolasining abssissasi 7 ga teng bo'lgan $M$ nuqtasining fokal radiusini toping va fokal radiusi yotgan to'g'ri chiziqning tenglamasini tuzing.  \\

C3. Fokuslari $F(3;4)$, $F(-3;-4)$ nuqtalarida joylashgan direktrisalari orasidagi masofa 3,6 ga teng bo'lgan giperbolaning tenglamasini tuzing.  \\

\end{tabular}
\vspace{1cm}


\begin{tabular}{m{17cm}}
\textbf{73-variant}
\newline

T1. ITECH-ning umumiy tenglamasini soddalashtirish (ITECH-ning umumiy tenglamasi, koordinata sistemasin almashtirish ITECH umumiy tenglamasini soddalashtirish).\\

T2. Ikki pallali giperboloid Kanonik tenglamasi (giperbolani simmetriya o'qi atrofida aylantirishdan olingan sirt).\\

A1. Qutb tenglamasi bilan berilgan egri chiziqning tipini aniqlang: $\rho=\frac{5}{3-4\cos\theta}$.\\

A2. Tipini aniqlang: $3x^{2}-8xy+7y^{2}+8x-15y+20=0$.\\

A3. Aylana tenglamasini tuzing: $M_1(-1;5)$, $M_2(-2;-2)$ va $M_3(5;5)$ nuqtalardan o'tadi.\\

B1. $\frac{x^{2}}{4} - \frac{y^{2}}{5} = 1$ giperbolasiga $3x + 2y = 0$ to'g'ri chizig'iga perpendikulyar bo'lgan urinma to'g'ri chiziqning tenglamasini tuzing.\\

B2. ITECH ning umumiy tenglamasini koordinata sistemasini almashtirmasdan soddalashtiring, tipini aniqlang, obrazi qanday chiziq ekanligini ko'rsating: $7x^{2} - 8xy + y^{2} - 16x - 2y - 51 = 0$\\

B3. $x^{2} - 4y^{2} = 16$ giperbola berilgan. Uning ekssentrisitetini, fokuslarining koordinatalarini toping va asimptotalarining tenglamalarini tuzing.\\

C1. $4x^{2} - 4xy + y^{2} - 6x + 8y + 13 = 0$ ITECH markazga egami? Markazga ega bo'lsa markazini aniqlang?  \\

C2. $\frac{x^{2}}{3} - \frac{y^{2}}{5} = 1$ giperbolasiga $P(1; - 5)$ nuqtasida yurgizilgan urinmalarning tenglamasini tuzing.\\

C3. $\frac{x^{2}}{100} + \frac{y^{2}}{36} = 1$ ellipsining o'ng tarafdagi fokusidan 14 ga teng masofada bo'lgan nuqtasini toping.  \\

\end{tabular}
\vspace{1cm}


\begin{tabular}{m{17cm}}
\textbf{74-variant}
\newline

T1. Giperbolaning urinmasining tenglamasi (giperbolaga berilgan nuqtada yurgizilgan urinma tenglamasi).\\

T2. ITECH-ning umumiy tenglamasini koordinata boshin parallel ko'chirish bilan soddalastiring (ITECH-ning umumiy tenglamasini parallel ko'chirish formulasi).\\

A1. Fokuslari abssissa o'qida va koordinata boshiga nisbatan simmetrik joylashgan ellipsning tenglamasini tuzing: katta o'qi $20$, ekssentrisitet $\varepsilon=3/5$.\\

A2. Giperbola tenglamasi berilgan: $\frac{x^{2}}{25}-\frac{y^{2}}{144}=1$. Uning qutb tenglamasini tuzing.\\

A3. Berilgan chiziqlarning markaziy ekanligini ko'rsating va markazinin toping: $5x^{2}+4xy+2y^{2}+20x+20y-18=0$.\\

B1. $3x + 10y - 25 = 0$ to'g'ri bilan $\frac{x^{2}}{25} + \frac{y^{2}}{4} = 1$ ellipsning kesishish nuqtalarini toping.  \\

B2. $\rho = \frac{10}{2 - cos\theta}$ polyar tenglamasi bilan qanday chiziq berilganini aniqlang.  \\

B3. $2x + 2y - 3 = 0$ to'g'ri chizig'iga perpendikulyar bo'lib $x^{2} = 16y$ parabolasiga urinib o'tuvchi to'g'ri chiziqning tenglamasini tuzing.  \\

C1. Katta o'qi 26 ga, fokuslari $F( - 10;0), F(14;0)$ nuqtalarida joylashgan ellipsning tenglamasini tuzing.  \\

C2. $16x^{2} - 9y^{2} - 64x - 54y - 161 = 0$ tenglamasi giperbolaning tenglamasi ekanligini ko'rsating va uning markazi $C$ ni, yarim o'qlarini, ekssentrisitetini toping, asimptotalarining tenglamalarini tuzing.  \\

C3. $\frac{x^{2}}{25} + \frac{y^{2}}{16} = 1$, ellipsiga $C(10; - 8)$ nuqtadan yurgizilgan urinmalarining tenglamasini tuzing.  \\

\end{tabular}
\vspace{1cm}


\begin{tabular}{m{17cm}}
\textbf{75-variant}
\newline

T1. Silindrlik sirtlar (yasovchi to'g'ri chiziq, yo'naltiruvchi egri chiziq, silindrlik sirt).\\

T2. Ikkinchi tartibli sirtning umumiy tenglamasi. Markazin aniqlash formulasi.\\

A1. Aylana tenglamasini tuzing: markazi $C(1;-1)$ nuqtada joylashgan va $5x-12y+9-0$ to'g'ri chiziqga urinadi.\\

A2. Uchi koordinata boshida joylashgan va $Ox$ o'qiga nisbatan chap tarafafgi yarim tekislikda joylashgan parabolaning tenglamasini tuzing: parametri $p=0,5$.\\

A3. Qutb tenglamasi bilan berilgan egri chiziqning tipini aniqlang: $\rho=\frac{6}{1-\cos 0}$.\\

B1. Koordinata o'qlarini almashtirmasdan ITECH tenglamasini soddalashtiring, yarim o'qlarnin toping: $4x^{2} - 4xy + 7y^{2} - 26x - 18y + 3 = 0$.\\

B2. Ellips $3x^{2} + 4y^{2} - 12 = 0$ tenglamasi bilan berilgan. Uning o'qlarining uzunliklarini, fokuslarining koordinatalarini va ekssentrisitetini toping.  \\

B3. $y^{2} = 3x$ parabolasi bilan $\frac{x^{2}}{100} + \frac{y^{2}}{225} = 1$ ellipsining kesishish nuqtalarini toping.  \\

C1. $y^{2} = 20x$ parabolasining $M$ nuqtasini toping, agar uning abssissasi 7 ga teng bo'lsa, fokal radiusini va fokal radiusi joylashgan to'g'rini aniqlang.\\

C2. Fokusi $F(7;2)$ nuqtasida joylashgan, mos direktrisasi $x - 5 = 0$ tenglamasi bilan berilgan parabolaning tenglamasini tuzing.  \\

C3. $14x^{2} + 24xy + 21y^{2} - 4x + 18y - 139 = 0$ egri chizig'ining tipini aniqlang, agar markazga ega egri chiziq bo'lsa, markazining koordinatalarini toping.  \\

\end{tabular}
\vspace{1cm}


\begin{tabular}{m{17cm}}
\textbf{76-variant}
\newline

T1. Parabola va uning kanonik tenglamasi ( ta'rifi, fokusi, direktrisasi, kanonik tenglamasi).\\

T2. ITECH-ning umumiy tenglamasini koordinata o'qlarini burish bilan soddalashtirish (ITECH-ning umumiy tenglamalari, koordinata o'qin burish formulasi, tenglamani kanonik turga olib kelish).\\

A1. Tipini aniqlang: $x^{2}-4xy+4y^{2}+7x-12=0$.\\

A2. Aylana tenglamasini tuzing: $A(1;1)$, $B(1;-1)$ va $C(2;0)$ nuqtalardan o'tadi.\\

A3. Fokuslari abssissa o'qida va koordinata boshiga nisbatan simmetrik joylashgan ellipsning tenglamasini tuzing: katta o'qi $10$, fokuslari orasidagi masofa $2c=8$.\\

B1. $\rho = \frac{5}{3 - 4cos\theta}$ tenglamasi bilan qanday chiziq berilganini va yarim o'qlarini toping.  \\

B2. $\frac{x^{2}}{4} - \frac{y^{2}}{5} = 1$, giperbolaning $3x - 2y = 0$ to'g'ri chizig'iga parallel bo'lgan urinmasining tenglamasini tuzing.  \\

B3. Koordinata o'qlarini almashtirmasdan ITECH umumiy tenglamasini soddalashtiring, yarim o'qlarini toping: $13x^{2} + 18xy + 37y^{2} - 26x - 18y + 3 = 0$.  \\

C1. Agar xohlagan vaqt momentida $M(x;y)$ nuqta $A(8;4)$ nuqtasidan va ordinata o'qidan birxil masofada joylashsa, $M(x;y)$ nuqtaning harakat troektoriyasining tenglamasini tuzing.  \\

C2. $4x^{2} + 24xy + 11y^{2} + 64x + 42y + 51 = 0$ egri chizig'ining tipini aniqlang, agar markazga ega bo'lsa, uning markazining koordinatalarini toping va koordinata boshini markazga parallel ko'chirish amalini bajaring.\\

C3. Giperbolaning ekssentrisiteti $\varepsilon = \frac{13}{12}$, fokusi $F(0;13)$ nuqtasida va mos direktrisasi $13y - 144 = 0$ tenglamasi bilan berilgan bo'lsa, giperbolaning tenglamasini tuzing.  \\

\end{tabular}
\vspace{1cm}


\begin{tabular}{m{17cm}}
\textbf{77-variant}
\newline

T1. Bir pallali giperboloid. Kanonik tenglamasi (giperbolani simmetriya o'qi atrofida aylantirishdan olingan sirt).\\

T2. Parabolaning urinmasining tenglamasi (parabola, to'g'ri chiziq urinish nuqtasi, urinma tenglamasi).\\

A1. Tipini aniqlang: $9x^{2}-16y^{2}-54x-64y-127=0$.\\

A2. Aylananing $C$ markazi va $R$ radiusini toping: $x^2+y^2+6x-4y+14=0$.\\

A3. Fokuslari abssissa o'qida va koordinata boshiga nisbatan simmetrik joylashgan giperbolaning tenglamasini tuzing: direktrisalar orasidagi masofa $8/3$ va ekssentrisitet $\varepsilon=3/2$.\\

B1. $3x + 4y - 12 = 0$ to'g'ri chizig'i va $y^{2} = - 9x$ parabolasining kesishish nuqtalarini toping.\\

B2. $\rho = \frac{6}{1 - cos\theta}$ polyar tenglamasi bilan qanday chiziq berilganini aniqlang.  \\

B3. $2x + 2y - 3 = 0$ to'g'ri chizig'iga parallel bo'lib $\frac{x^{2}}{16} + \frac{y^{2}}{64} = 1$ giperbolasiga urinib o'tuvchi to'g'ri chiziqning tenglamasini tuzing.  \\

C1. $2x^{2} + 3y^{2} + 8x - 6y + 11 = 0$ tenglamasi bilan qanday tipdagi chiziq berilganini aniqlang va uning tenglamasini soddalashtiring va grafigini chizing.  \\

C2. $M(2; - \frac{5}{3})$ nuqta $\frac{x^{2}}{9} + \frac{y^{2}}{5} = 1$ ellipsda joylashgan. $M$ nuqtaning fokal radiuslarida yotuvchi to'g'ri chiziq tenglamalarini tuzing.  \\

C3. Fokusi $F( - 1; - 4)$ nuqtasida joylashgan, mos direktrisasi $x - 2 = 0$ tenglamasi bilan berilgan, $A( - 3; - 5)$ nuqtadan o'tuvchi ellipsning tenglamasini tuzing.  \\

\end{tabular}
\vspace{1cm}


\begin{tabular}{m{17cm}}
\textbf{78-variant}
\newline

T1. ITECH-ning umumiy tenglamasini klassifikatsiyalash (ITECH-ning umumiy tenglamasi, ITECH-ning umumiy tenglamasini soddalashtirish, klassifikatsiyalash).\\

T2. Giperbolik paraboloydning to'g'ri chiziq yasovchilari (Giperbolik paraboloydni yasovchi to'g'ri chiziqlar dastasi).\\

A1. Tipini aniqlang: $5x^{2}+14xy+11y^{2}+12x-7y+19=0$.\\

A2. Aylana tenglamasini tuzing: aylana diametrining uchlari $A(3;2)$ va $B(-1;6)$ nuqtalarda joylashgan.\\

A3. Fokuslari abssissa o'qida va koordinata boshiga nisbatan simmetrik joylashgan giperbolaning tenglamasini tuzing: o'qlari $2a=10$ va $2b=8$.\\

B1. Koordinata o'qlarini almashtirmasdan ITECH tenglamasini soddalashtiring, qanday geometrik obraz ekanligini ko'rsating: $4x^{2} - 4xy + y^{2} + 4x - 2y + 1 = 0$.  \\

B2. $3x + 4y - 12 = 0$ to'g'ri chizig'i bilan $y^{2} = - 9x$ parabolasining kesishish nuqtalarini toping.  \\

B3. $\rho = \frac{144}{13 - 5cos\theta}$ ellips ekanligini ko'rsating va uning yarim o'qlarini aniqlang.\\

C1. $4x^{2} - 4xy + y^{2} - 2x - 14y + 7 = 0$ ITECH tenglamasini kanonik shaklga olib keling, tipini aniqlang, qanday geometrik obraz ekanligini ko'rsating, chizmasini eski va yangi koordinatalar sistemasiga nisbatan chizing.  \\

C2. $A(\frac{10}{3};\frac{5}{3})$ nuqtasidan $\frac{x^{2}}{20} + \frac{y^{2}}{5} = 1$ ellipsiga yurgizilgan urinmalarning tenglamasini tuzing.  \\

C3. $y^{2} = 20x$ parabolasining abssissasi 7 ga teng bo'lgan $M$ nuqtasining fokal radiusini toping va fokal radiusi yotgan to'g'ri chiziqning tenglamasini tuzing.  \\

\end{tabular}
\vspace{1cm}


\begin{tabular}{m{17cm}}
\textbf{79-variant}
\newline

T1. Parabolaning polyar koordinatalardagi tenglamasi (polyar koordinata sistemasida parabolaning tenglamasi).\\

T2. ITECH-ning invariantlari (ITECH-ning umumiy tenglamasi, almashtirish, ITECH invariantlari).\\

A1. Tipini aniqlang: $4x^{2}-y^{2}+8x-2y+3=0$.\\

A2. Aylana tenglamasini tuzing: markazi $C(2;-3)$ nuqtada joylashgan va radiusi $R=7$ ga teng.\\

A3. Fokuslari abssissa o'qida va koordinata boshiga nisbatan simmetrik joylashgan giperbolaning tenglamasini tuzing: direktrisalar orasidagi masofa $228/13$ va fokuslari orasidagi masofa $2c=26$.\\

B1. $x^{2} + 4y^{2} = 25$ ellipsi bilan $4x - 2y + 23 = 0$ to'g'ri chizig'iga parallel bo'lgan urinma to'g'ri chiziqning tenglamasini tuzing.  \\

B2. $41x^{2} + 24xy + 9y^{2} + 24x + 18y - 36 = 0$ ITECH tipini aniqlang va markazlarini toping koordinata o'qlarini almashtirmasdan qanday chiziq ekanligini ko'rsating, yarim o'qlarini toping.  \\

B3. $y^{2} = 12x$ paraborolasiga $3x - 2y + 30 = 0$ to'g'ri chizig'iga parallel bo'lgan urinmasining tenglamasini tuzing.  \\

C1. Uchi (-4;0) nuqtasinda, direktrisasi $y - 2 = 0$ to'g'ri chiziq bo'lgan parabolaning tenglamasini tuzing.\\

C2. $32x^{2} + 52xy - 7y^{2} + 180 = 0$ ITECH tenglamasini kanonik shaklga olib keling, tipini aniqlang, qanday geometrik obraz ekanligini ko'rsating, chizmasini eski va yangi koordinatalar sistemasiga nisbatan chizing.  \\

C3. $\frac{x^{2}}{3} - \frac{y^{2}}{5} = 1$, giperbolasiga $P(4;2)$ nuqtadan yurgizilgan urinmalarning tenglamasini tuzing.  \\

\end{tabular}
\vspace{1cm}


\begin{tabular}{m{17cm}}
\textbf{80-variant}
\newline

T1. Elliptik paraboloid (parabola, o'q, elliptik paraboloid).\\

T2. Giperbolaning polyar koordinatadagi tenglamasi (Polyar burchagi, polyar radiusi giperbolaning polyar tenglamasi)\\

A1. Tipini aniqlang: $3x^{2}-2xy-3y^{2}+12y-15=0$.\\

A2. Fokuslari abssissa o'qida va koordinata boshiga nisbatan simmetrik joylashgan ellipsning tenglamasini tuzing: kichik o'qi $10$, ekssentrisitet $\varepsilon=12/13$.\\

A3. Berilgan chiziqlarning markaziy ekanligini ko'rsating va markazinin toping: $3x^{2}+5xy+y^{2}-8x-11y-7=0$.\\

B1. $\frac{x^{2}}{16} - \frac{y^{2}}{64} = 1$ giperbolasiga berilgan $10x - 3y + 9 = 0$ to'g'ri chizig'iga parallel bo'lgan urinmasining tenglamasini tuzing.  \\

B2. $x^{2} - y^{2} = 27$ giperbolasiga $4x + 2y - 7 = 0$ to'g'ri chizigiga parallel bo'lgan urinmasining tenglamasini toping.  \\

B3. ITECH ning umumiy tenglamasini koordinata sistemasini almashtirmasdan soddalashtiring, tipini aniqlang, obrazi qanday chiziq ekanligini ko'rsating: $7x^{2} - 8xy + y^{2} - 16x - 2y - 51 = 0$\\

C1. $\frac{x^{2}}{100} + \frac{y^{2}}{36} = 1$ ellipsining o'ng tarafdagi fokusidan 14 ga teng masofada bo'lgan nuqtasini toping.  \\

C2. Fokusi $F(2; - 1)$ nuqtasida joylashgan, mos direktrisasi $x - y - 1 = 0$ tenglamasi bilan berilgan parabolaning tenglamasini tuzing.  \\

C3. $4x^{2} - 4xy + y^{2} - 6x + 8y + 13 = 0$ ITECH markazga egami? Markazga ega bo'lsa markazini aniqlang?  \\

\end{tabular}
\vspace{1cm}


\begin{tabular}{m{17cm}}
\textbf{81-variant}
\newline

T1. ITECH-ning markazini aniqlash formulasi (ITECH-ning umumiy tenglamasi, markazini aniqlash formulasi).\\

T2. Ellipsoida. Kanonik tenglamasi (ellipsni simmetriya o'qi atrofida aylantirishdan olingan sirt, kanonik tenglamasi).\\

A1. Uchi koordinata boshida joylashgan va $Oy$ o'qiga nisbatan quyi tarafafgi yarim tekislikda joylashgan parabolaning tenglamasini tuzing: parametri $p=3$.\\

A2. Uchi koordinata boshida joylashgan va $Oy$ o'qiga nisbatan yuqori yarim tekislikda joylashgan parabolaning tenglamasini tuzing: parametri $p=1/4$.\\

A3. Fokuslari abssissa o'qida va koordinata boshiga nisbatan simmetrik joylashgan ellipsning tenglamasini tuzing: direktrisalar orasidagi masofa $5$ va fokuslari orasidagi masofa $2c=4$.\\

B1. $\frac{x^{2}}{20} - \frac{y^{2}}{5} = 1$ giperbolasiga $4x + 3y - 7 = 0$ to'g'ri chizig'iga perpendikulyar bo'lgan urinmasining tenglamasini tuzing.  \\

B2. Koordinata o'qlarini almashtirmasdan ITECH tenglamasini soddalashtiring, yarim o'qlarnin toping: $4x^{2} - 4xy + 7y^{2} - 26x - 18y + 3 = 0$.\\

B3. $x^{2} - 4y^{2} = 16$ giperbola berilgan. Uning ekssentrisitetini, fokuslarining koordinatalarini toping va asimptotalarining tenglamalarini tuzing.\\

C1. $\frac{x^{2}}{3} - \frac{y^{2}}{5} = 1$ giperbolasiga $P(1; - 5)$ nuqtasida yurgizilgan urinmalarning tenglamasini tuzing.\\

C2. $y^{2} = 20x$ parabolasining $M$ nuqtasini toping, agar uning abssissasi 7 ga teng bo'lsa, fokal radiusini va fokal radiusi joylashgan to'g'rini aniqlang.\\

C3. Fokusi $F( - 1; - 4)$ nuqtasida bo'lgan, mos direktrisasi $x - 2 = 0$ tenglamasi bilan berilgan, $A( - 3; - 5)$ nuqtadan o'tuvchi ellipsning tenglamasini tuzing.  \\

\end{tabular}
\vspace{1cm}


\begin{tabular}{m{17cm}}
\textbf{82-variant}
\newline

T1. Ellipsning urinmasining tenglamasi (ellips, to'g'ri chiziq urinish nuqtasi, urinma tenglamasi).\\

T2. Koordinata sistemasini almashtirish (birlik vektorlar, o'qlar, parallel ko'chirish, koordinata o'qlarinii burish).\\

A1. Fokuslari abssissa o'qida va koordinata boshiga nisbatan simmetrik joylashgan giperbolaning tenglamasini tuzing: katta o'qi $2a=16$ va ekssentrisitet $\varepsilon=5/4$.\\

A2. Fokuslari abssissa o'qida va koordinata boshiga nisbatan simmetrik joylashgan ellipsning tenglamasini tuzing: kichik o'qi $6$, direktrisalar orasidagi masofa $13$.\\

A3. Fokuslari abssissa o'qida va koordinata boshiga nisbatan simmetrik joylashgan ellipsning tenglamasini tuzing: fokuslari orasidagi masofa $2c=6$ va ekssentrisitet $\varepsilon=3/5$.\\

B1. $3x + 10y - 25 = 0$ to'g'ri bilan $\frac{x^{2}}{25} + \frac{y^{2}}{4} = 1$ ellipsning kesishish nuqtalarini toping.  \\

B2. $\rho = \frac{10}{2 - cos\theta}$ polyar tenglamasi bilan qanday chiziq berilganini aniqlang.  \\

B3. $\frac{x^{2}}{4} - \frac{y^{2}}{5} = 1$ giperbolasiga $3x + 2y = 0$ to'g'ri chizig'iga perpendikulyar bo'lgan urinma to'g'ri chiziqning tenglamasini tuzing.\\

C1. $16x^{2} - 9y^{2} - 64x - 54y - 161 = 0$ tenglamasi giperbolaning tenglamasi ekanligini ko'rsating va uning markazi $C$ ni, yarim o'qlarini, ekssentrisitetini toping, asimptotalarining tenglamalarini tuzing.  \\

C2. $\frac{x^{2}}{25} + \frac{y^{2}}{16} = 1$, ellipsiga $C(10; - 8)$ nuqtadan yurgizilgan urinmalarining tenglamasini tuzing.  \\

C3. $M(2; - \frac{5}{3})$ nuqta $\frac{x^{2}}{9} + \frac{y^{2}}{5} = 1$ ellipsda joylashgan. $M$ nuqtaning fokal radiuslarida yotuvchi to'g'ri chiziq tenglamalarini tuzing.  \\

\end{tabular}
\vspace{1cm}


\begin{tabular}{m{17cm}}
\textbf{83-variant}
\newline

T1. Ikkinshi tartibli aylanma sirtlar (koordinata sistemasi, tekislik, vektor egri chiziq, aylanma sirt).\\

T2. Ellipsning polyar koordinatalardagi tenglamasi (polyar koordinatalar sistemasida ellipsning tenglamasi).\\

A1. Fokuslari abssissa o'qida va koordinata boshiga nisbatan simmetrik joylashgan ellipsning tenglamasini tuzing: direktrisalar orasidagi masofa $32$ va $\varepsilon=1/9$.\\

A2. Fokuslari abssissa o'qida va koordinata boshiga nisbatan simmetrik joylashgan giperbolaning tenglamasini tuzing: asimptotalar tenglamalari $y=\pm \frac{3}{4}x$ va direktrisalar orasidagi masofa $64/5$.\\

A3. Aylananing $C$ markazi va $R$ radiusini toping: $x^2+y^2-2x+4y-14=0$.\\

B1. Koordinata o'qlarini almashtirmasdan ITECH umumiy tenglamasini soddalashtiring, yarim o'qlarini toping: $13x^{2} + 18xy + 37y^{2} - 26x - 18y + 3 = 0$.  \\

B2. Ellips $3x^{2} + 4y^{2} - 12 = 0$ tenglamasi bilan berilgan. Uning o'qlarining uzunliklarini, fokuslarining koordinatalarini va ekssentrisitetini toping.  \\

B3. $y^{2} = 3x$ parabolasi bilan $\frac{x^{2}}{100} + \frac{y^{2}}{225} = 1$ ellipsining kesishish nuqtalarini toping.  \\

C1. Agar vaqtning xohlagan momentida $M(x;y)$ nuqta $5x - 16 = 0$ to'g'ri chiziqqa qaraganda $A(5;0)$ nuqtasidan 1,25 marta uzoqroq masofada joylashgan. Shu $M(x;y)$ nuqtaning harakatining tenglamasini tuzing.  \\

C2. $14x^{2} + 24xy + 21y^{2} - 4x + 18y - 139 = 0$ egri chizig'ining tipini aniqlang, agar markazga ega egri chiziq bo'lsa, markazining koordinatalarini toping.  \\

C3. Fokuslari $F(3;4)$, $F(-3;-4)$ nuqtalarida joylashgan direktrisalari orasidagi masofa 3,6 ga teng bo'lgan giperbolaning tenglamasini tuzing.  \\

\end{tabular}
\vspace{1cm}


\begin{tabular}{m{17cm}}
\textbf{84-variant}
\newline

T1. ITECH-ning umumiy tenglamasini soddalashtirish (ITECH-ning umumiy tenglamasi, koordinata sistemasin almashtirish ITECH umumiy tenglamasini soddalashtirish).\\

T2. Sirtning kanonik tenglamalari. Sirt haqqida tushuncha. (Sirtning ta'rifi, formulalari, o'q, yo'naltiruvchi to'g'ri chiziqlar).\\

A1. Fokuslari abssissa o'qida va koordinata boshiga nisbatan simmetrik joylashgan ellipsning tenglamasini tuzing: katta o'qi $8$, direktrisalar orasidagi masofa $16$.\\

A2. Qutb tenglamasi bilan berilgan egri chiziqning tipini aniqlang: $\rho=\frac{1}{3-3\cos\theta}$.\\

A3. Tipini aniqlang: $2x^{2}+3y^{2}+8x-6y+11=0$.\\

B1. $\rho = \frac{5}{3 - 4cos\theta}$ tenglamasi bilan qanday chiziq berilganini va yarim o'qlarini toping.  \\

B2. $2x + 2y - 3 = 0$ to'g'ri chizig'iga perpendikulyar bo'lib $x^{2} = 16y$ parabolasiga urinib o'tuvchi to'g'ri chiziqning tenglamasini tuzing.  \\

B3. Koordinata o'qlarini almashtirmasdan ITECH tenglamasini soddalashtiring, qanday geometrik obraz ekanligini ko'rsating: $4x^{2} - 4xy + y^{2} + 4x - 2y + 1 = 0$.  \\

C1. $4x^{2} + 24xy + 11y^{2} + 64x + 42y + 51 = 0$ egri chizig'ining tipini aniqlang, agar markazga ega bo'lsa, uning markazining koordinatalarini toping va koordinata boshini markazga parallel ko'chirish amalini bajaring.\\

C2. Katta o'qi 26 ga, fokuslari $F( - 10;0), F(14;0)$ nuqtalarida joylashgan ellipsning tenglamasini tuzing.  \\

C3. $2x^{2} + 3y^{2} + 8x - 6y + 11 = 0$ tenglamasi bilan qanday tipdagi chiziq berilganini aniqlang va uning tenglamasini soddalashtiring va grafigini chizing.  \\

\end{tabular}
\vspace{1cm}


\begin{tabular}{m{17cm}}
\textbf{85-variant}
\newline

T1. Ellips va uning kanonik tenglamasi (ta'rifi, fokuslari, ellipsning kanonik tenglamasi, ekstsentrisiteti, direktrisalari).\\

T2. ITECH-ning umumiy tenglamasini koordinata boshin parallel ko'chirish bilan soddalastiring (ITECH-ning umumiy tenglamasini parallel ko'chirish formulasi).\\

A1. Aylana tenglamasini tuzing: aylana $A(2;6)$ nuqtadan o'tadi va markazi $C(-1;2)$ nuqtada joylashgan.\\

A2. Fokuslari abssissa o'qida va koordinata boshiga nisbatan simmetrik joylashgan ellipsning tenglamasini tuzing: kichik o'qi $24$, fokuslari orasidagi masofa $2c=10$.\\

A3. Giperbola tenglamasi berilgan: $\frac{x^{2}}{16}-\frac{y^{2}}{9}=1$. Uning qutb tenglamasini tuzing.\\

B1. $3x + 4y - 12 = 0$ to'g'ri chizig'i va $y^{2} = - 9x$ parabolasining kesishish nuqtalarini toping.\\

B2. $\rho = \frac{6}{1 - cos\theta}$ polyar tenglamasi bilan qanday chiziq berilganini aniqlang.  \\

B3. $\frac{x^{2}}{4} - \frac{y^{2}}{5} = 1$, giperbolaning $3x - 2y = 0$ to'g'ri chizig'iga parallel bo'lgan urinmasining tenglamasini tuzing.  \\

C1. $y^{2} = 20x$ parabolasining abssissasi 7 ga teng bo'lgan $M$ nuqtasining fokal radiusini toping va fokal radiusi yotgan to'g'ri chiziqning tenglamasini tuzing.  \\

C2. Fokusi $F(7;2)$ nuqtasida joylashgan, mos direktrisasi $x - 5 = 0$ tenglamasi bilan berilgan parabolaning tenglamasini tuzing.  \\

C3. $4x^{2} - 4xy + y^{2} - 2x - 14y + 7 = 0$ ITECH tenglamasini kanonik shaklga olib keling, tipini aniqlang, qanday geometrik obraz ekanligini ko'rsating, chizmasini eski va yangi koordinatalar sistemasiga nisbatan chizing.  \\

\end{tabular}
\vspace{1cm}


\begin{tabular}{m{17cm}}
\textbf{86-variant}
\newline

T1. Ikki pallali giperboloid Kanonik tenglamasi (giperbolani simmetriya o'qi atrofida aylantirishdan olingan sirt).\\

T2. Giperbola. Kanonik tenglamasi (fokuslar, o'qlar, direktrisalar, giperbola, ekstsentrisitet, kanonik tenglamasi).\\

A1. Tipini aniqlang: $9x^{2}+4y^{2}+18x-8y+49=0$.\\

A2. Aylana tenglamasini tuzing: markazi koordinata boshida joylashgan va radiusi $R=3$ ga teng.\\

A3. Uchi koordinata boshida joylashgan va $Ox$ o'qiga nisbatan o'ng tarafafgi yarim tekislikda joylashgan parabolaning tenglamasini tuzing: parametri $p=3$.\\

B1. $41x^{2} + 24xy + 9y^{2} + 24x + 18y - 36 = 0$ ITECH tipini aniqlang va markazlarini toping koordinata o'qlarini almashtirmasdan qanday chiziq ekanligini ko'rsating, yarim o'qlarini toping.  \\

B2. $3x + 4y - 12 = 0$ to'g'ri chizig'i bilan $y^{2} = - 9x$ parabolasining kesishish nuqtalarini toping.  \\

B3. $\rho = \frac{144}{13 - 5cos\theta}$ ellips ekanligini ko'rsating va uning yarim o'qlarini aniqlang.\\

C1. $A(\frac{10}{3};\frac{5}{3})$ nuqtasidan $\frac{x^{2}}{20} + \frac{y^{2}}{5} = 1$ ellipsiga yurgizilgan urinmalarning tenglamasini tuzing.  \\

C2. $\frac{x^{2}}{100} + \frac{y^{2}}{36} = 1$ ellipsining o'ng tarafdagi fokusidan 14 ga teng masofada bo'lgan nuqtasini toping.  \\

C3. Agar xohlagan vaqt momentida $M(x;y)$ nuqta $A(8;4)$ nuqtasidan va ordinata o'qidan birxil masofada joylashsa, $M(x;y)$ nuqtaning harakat troektoriyasining tenglamasini tuzing.  \\

\end{tabular}
\vspace{1cm}


\begin{tabular}{m{17cm}}
\textbf{87-variant}
\newline

T1. Ikkinchi tartibli sirtning umumiy tenglamasi. Markazin aniqlash formulasi.\\

T2. Silindrlik sirtlar (yasovchi to'g'ri chiziq, yo'naltiruvchi egri chiziq, silindrlik sirt).\\

A1. Ellips tenglamasi berilgan: $\frac{x^2}{25}+\frac{y^2}{16}=1$. Uning qutb tenglamasini tuzing.\\

A2. Tipini aniqlang: $25x^{2}-20xy+4y^{2}-12x+20y-17=0$.\\

A3. Aylana tenglamasini tuzing: $A(3;1)$ va $B(-1;3)$ nuqtalardan o'tadi, markazi $3x-y-2=0$ togri chiziqda joylashgan.\\

B1. $2x + 2y - 3 = 0$ to'g'ri chizig'iga parallel bo'lib $\frac{x^{2}}{16} + \frac{y^{2}}{64} = 1$ giperbolasiga urinib o'tuvchi to'g'ri chiziqning tenglamasini tuzing.  \\

B2. ITECH ning umumiy tenglamasini koordinata sistemasini almashtirmasdan soddalashtiring, tipini aniqlang, obrazi qanday chiziq ekanligini ko'rsating: $7x^{2} - 8xy + y^{2} - 16x - 2y - 51 = 0$\\

B3. $x^{2} + 4y^{2} = 25$ ellipsi bilan $4x - 2y + 23 = 0$ to'g'ri chizig'iga parallel bo'lgan urinma to'g'ri chiziqning tenglamasini tuzing.  \\

C1. $32x^{2} + 52xy - 7y^{2} + 180 = 0$ ITECH tenglamasini kanonik shaklga olib keling, tipini aniqlang, qanday geometrik obraz ekanligini ko'rsating, chizmasini eski va yangi koordinatalar sistemasiga nisbatan chizing.  \\

C2. $\frac{x^{2}}{3} - \frac{y^{2}}{5} = 1$, giperbolasiga $P(4;2)$ nuqtadan yurgizilgan urinmalarning tenglamasini tuzing.  \\

C3. $y^{2} = 20x$ parabolasining $M$ nuqtasini toping, agar uning abssissasi 7 ga teng bo'lsa, fokal radiusini va fokal radiusi joylashgan to'g'rini aniqlang.\\

\end{tabular}
\vspace{1cm}


\begin{tabular}{m{17cm}}
\textbf{88-variant}
\newline

T1. ITECH-ning umumiy tenglamasini koordinata o'qlarini burish bilan soddalashtirish (ITECH-ning umumiy tenglamalari, koordinata o'qin burish formulasi, tenglamani kanonik turga olib kelish).\\

T2. Giperbolaning urinmasining tenglamasi (giperbolaga berilgan nuqtada yurgizilgan urinma tenglamasi).\\

A1. Fokuslari abssissa o'qida va koordinata boshiga nisbatan simmetrik joylashgan giperbolaning tenglamasini tuzing: asimptotalar tenglamalari $y=\pm \frac{4}{3}x$ va fokuslari orasidagi masofa $2c=20$.\\

A2. Qutb tenglamasi bilan berilgan egri chiziqning tipini aniqlang: $\rho=\frac{10}{1-\frac{3}{2}\cos\theta}$.\\

A3. Berilgan chiziqlarning markaziy ekanligini ko'rsating va markazinin toping: $9x^{2}-4xy-7y^{2}-12=0$.\\

B1. $y^{2} = 12x$ paraborolasiga $3x - 2y + 30 = 0$ to'g'ri chizig'iga parallel bo'lgan urinmasining tenglamasini tuzing.  \\

B2. $\frac{x^{2}}{16} - \frac{y^{2}}{64} = 1$ giperbolasiga berilgan $10x - 3y + 9 = 0$ to'g'ri chizig'iga parallel bo'lgan urinmasining tenglamasini tuzing.  \\

B3. Koordinata o'qlarini almashtirmasdan ITECH tenglamasini soddalashtiring, yarim o'qlarnin toping: $4x^{2} - 4xy + 7y^{2} - 26x - 18y + 3 = 0$.\\

C1. Giperbolaning ekssentrisiteti $\varepsilon = \frac{13}{12}$, fokusi $F(0;13)$ nuqtasida va mos direktrisasi $13y - 144 = 0$ tenglamasi bilan berilgan bo'lsa, giperbolaning tenglamasini tuzing.  \\

C2. $4x^{2} - 4xy + y^{2} - 6x + 8y + 13 = 0$ ITECH markazga egami? Markazga ega bo'lsa markazini aniqlang?  \\

C3. $\frac{x^{2}}{3} - \frac{y^{2}}{5} = 1$ giperbolasiga $P(1; - 5)$ nuqtasida yurgizilgan urinmalarning tenglamasini tuzing.\\

\end{tabular}
\vspace{1cm}


\begin{tabular}{m{17cm}}
\textbf{89-variant}
\newline

T1. ITECH-ning umumiy tenglamasini klassifikatsiyalash (ITECH-ning umumiy tenglamasi, ITECH-ning umumiy tenglamasini soddalashtirish, klassifikatsiyalash).\\

T2. Bir pallali giperboloid. Kanonik tenglamasi (giperbolani simmetriya o'qi atrofida aylantirishdan olingan sirt).\\

A1. Aylananing $C$ markazi va $R$ radiusini toping: $x^2+y^2+4x-2y+5=0$.\\

A2. Fokuslari abssissa o'qida va koordinata boshiga nisbatan simmetrik joylashgan giperbolaning tenglamasini tuzing: direktrisalar orasidagi masofa $32/5$ va o'qi $2b=6$.\\

A3. Qutb tenglamasi bilan berilgan egri chiziqning tipini aniqlang: $\rho=\frac{12}{2-\cos\theta}$.\\

B1. $x^{2} - y^{2} = 27$ giperbolasiga $4x + 2y - 7 = 0$ to'g'ri chizigiga parallel bo'lgan urinmasining tenglamasini toping.  \\

B2. Koordinata o'qlarini almashtirmasdan ITECH umumiy tenglamasini soddalashtiring, yarim o'qlarini toping: $13x^{2} + 18xy + 37y^{2} - 26x - 18y + 3 = 0$.  \\

B3. $x^{2} - 4y^{2} = 16$ giperbola berilgan. Uning ekssentrisitetini, fokuslarining koordinatalarini toping va asimptotalarining tenglamalarini tuzing.\\

C1. $M(2; - \frac{5}{3})$ nuqta $\frac{x^{2}}{9} + \frac{y^{2}}{5} = 1$ ellipsda joylashgan. $M$ nuqtaning fokal radiuslarida yotuvchi to'g'ri chiziq tenglamalarini tuzing.  \\

C2. Fokusi $F( - 1; - 4)$ nuqtasida joylashgan, mos direktrisasi $x - 2 = 0$ tenglamasi bilan berilgan, $A( - 3; - 5)$ nuqtadan o'tuvchi ellipsning tenglamasini tuzing.  \\

C3. $16x^{2} - 9y^{2} - 64x - 54y - 161 = 0$ tenglamasi giperbolaning tenglamasi ekanligini ko'rsating va uning markazi $C$ ni, yarim o'qlarini, ekssentrisitetini toping, asimptotalarining tenglamalarini tuzing.  \\

\end{tabular}
\vspace{1cm}


\begin{tabular}{m{17cm}}
\textbf{90-variant}
\newline

T1. Parabola va uning kanonik tenglamasi ( ta'rifi, fokusi, direktrisasi, kanonik tenglamasi).\\

T2. ITECH-ning invariantlari (ITECH-ning umumiy tenglamasi, almashtirish, ITECH invariantlari).\\

A1. Berilgan chiziqlarning markaziy ekanligini ko'rsating va markazinin toping: $2x^{2}-6xy+5y^{2}+22x-36y+11=0$.\\

A2. Aylana tenglamasini tuzing: markazi $C(6;-8)$ nuqtada joylashgan va koordinata boshidan o'tadi.\\

A3. Fokuslari abssissa o'qida va koordinata boshiga nisbatan simmetrik joylashgan ellipsning tenglamasini tuzing: yarim o'qlari 5 va 2.\\

B1. $3x + 10y - 25 = 0$ to'g'ri bilan $\frac{x^{2}}{25} + \frac{y^{2}}{4} = 1$ ellipsning kesishish nuqtalarini toping.  \\

B2. $\rho = \frac{10}{2 - cos\theta}$ polyar tenglamasi bilan qanday chiziq berilganini aniqlang.  \\

B3. $\frac{x^{2}}{20} - \frac{y^{2}}{5} = 1$ giperbolasiga $4x + 3y - 7 = 0$ to'g'ri chizig'iga perpendikulyar bo'lgan urinmasining tenglamasini tuzing.  \\

C1. $\frac{x^{2}}{25} + \frac{y^{2}}{16} = 1$, ellipsiga $C(10; - 8)$ nuqtadan yurgizilgan urinmalarining tenglamasini tuzing.  \\

C2. $y^{2} = 20x$ parabolasining abssissasi 7 ga teng bo'lgan $M$ nuqtasining fokal radiusini toping va fokal radiusi yotgan to'g'ri chiziqning tenglamasini tuzing.  \\

C3. Uchi (-4;0) nuqtasinda, direktrisasi $y - 2 = 0$ to'g'ri chiziq bo'lgan parabolaning tenglamasini tuzing.\\

\end{tabular}
\vspace{1cm}


\begin{tabular}{m{17cm}}
\textbf{91-variant}
\newline

T1. Giperbolik paraboloydning to'g'ri chiziq yasovchilari (Giperbolik paraboloydni yasovchi to'g'ri chiziqlar dastasi).\\

T2. Parabolaning urinmasining tenglamasi (parabola, to'g'ri chiziq urinish nuqtasi, urinma tenglamasi).\\

A1. Parabola tenglamasi berilgan: $y^2=6x$. Uning qutb tenglamasini tuzing.\\

A2. Tipini aniqlang: $4x^2+9y^2-40x+36y+100=0$.\\

A3. Aylananing $C$ markazi va $R$ radiusini toping: $x^2+y^2-2x+4y-20=0$.\\

B1. Koordinata o'qlarini almashtirmasdan ITECH tenglamasini soddalashtiring, qanday geometrik obraz ekanligini ko'rsating: $4x^{2} - 4xy + y^{2} + 4x - 2y + 1 = 0$.  \\

B2. Ellips $3x^{2} + 4y^{2} - 12 = 0$ tenglamasi bilan berilgan. Uning o'qlarining uzunliklarini, fokuslarining koordinatalarini va ekssentrisitetini toping.  \\

B3. $y^{2} = 3x$ parabolasi bilan $\frac{x^{2}}{100} + \frac{y^{2}}{225} = 1$ ellipsining kesishish nuqtalarini toping.  \\

C1. $14x^{2} + 24xy + 21y^{2} - 4x + 18y - 139 = 0$ egri chizig'ining tipini aniqlang, agar markazga ega egri chiziq bo'lsa, markazining koordinatalarini toping.  \\

C2. Fokusi $F(2; - 1)$ nuqtasida joylashgan, mos direktrisasi $x - y - 1 = 0$ tenglamasi bilan berilgan parabolaning tenglamasini tuzing.  \\

C3. $4x^{2} + 24xy + 11y^{2} + 64x + 42y + 51 = 0$ egri chizig'ining tipini aniqlang, agar markazga ega bo'lsa, uning markazining koordinatalarini toping va koordinata boshini markazga parallel ko'chirish amalini bajaring.\\

\end{tabular}
\vspace{1cm}


\begin{tabular}{m{17cm}}
\textbf{92-variant}
\newline

T1. ITECH-ning markazini aniqlash formulasi (ITECH-ning umumiy tenglamasi, markazini aniqlash formulasi).\\

T2. Elliptik paraboloid (parabola, o'q, elliptik paraboloid).\\

A1. Fokuslari abssissa o'qida va koordinata boshiga nisbatan simmetrik joylashgan giperbolaning tenglamasini tuzing: fokuslari orasidagi masofasi $2c=10$ va o'qi $2b=8$.\\

A2. Qutb tenglamasi bilan berilgan egri chiziqning tipini aniqlang: $\rho=\frac{5}{1-\frac{1}{2}\cos\theta}$.\\

A3. Tipini aniqlang: $2x^{2}+10xy+12y^{2}-7x+18y-15=0$.\\

B1. $\rho = \frac{5}{3 - 4cos\theta}$ tenglamasi bilan qanday chiziq berilganini va yarim o'qlarini toping.  \\

B2. $\frac{x^{2}}{4} - \frac{y^{2}}{5} = 1$ giperbolasiga $3x + 2y = 0$ to'g'ri chizig'iga perpendikulyar bo'lgan urinma to'g'ri chiziqning tenglamasini tuzing.\\

B3. $41x^{2} + 24xy + 9y^{2} + 24x + 18y - 36 = 0$ ITECH tipini aniqlang va markazlarini toping koordinata o'qlarini almashtirmasdan qanday chiziq ekanligini ko'rsating, yarim o'qlarini toping.  \\

C1. Fokusi $F( - 1; - 4)$ nuqtasida bo'lgan, mos direktrisasi $x - 2 = 0$ tenglamasi bilan berilgan, $A( - 3; - 5)$ nuqtadan o'tuvchi ellipsning tenglamasini tuzing.  \\

C2. $2x^{2} + 3y^{2} + 8x - 6y + 11 = 0$ tenglamasi bilan qanday tipdagi chiziq berilganini aniqlang va uning tenglamasini soddalashtiring va grafigini chizing.  \\

C3. $\frac{x^{2}}{100} + \frac{y^{2}}{36} = 1$ ellipsining o'ng tarafdagi fokusidan 14 ga teng masofada bo'lgan nuqtasini toping.  \\

\end{tabular}
\vspace{1cm}


\begin{tabular}{m{17cm}}
\textbf{93-variant}
\newline

T1. Parabolaning polyar koordinatalardagi tenglamasi (polyar koordinata sistemasida parabolaning tenglamasi).\\

T2. Koordinata sistemasini almashtirish (birlik vektorlar, o'qlar, parallel ko'chirish, koordinata o'qlarinii burish).\\

A1. Aylana tenglamasini tuzing: markazi koordinata boshida joylashgan va $3x-4y+20=0$ to'g'ri chiziqga urinadi.\\

A2. Fokuslari abssissa o'qida va koordinata boshiga nisbatan simmetrik joylashgan giperbolaning tenglamasini tuzing: fokuslari orasidagi masofa $2c=6$ va ekssentrisitet $\varepsilon=3/2$.\\

A3. Qutb tenglamasi bilan berilgan egri chiziqning tipini aniqlang: $\rho=\frac{5}{3-4\cos\theta}$.\\

B1. $3x + 4y - 12 = 0$ to'g'ri chizig'i va $y^{2} = - 9x$ parabolasining kesishish nuqtalarini toping.\\

B2. $\rho = \frac{6}{1 - cos\theta}$ polyar tenglamasi bilan qanday chiziq berilganini aniqlang.  \\

B3. $2x + 2y - 3 = 0$ to'g'ri chizig'iga perpendikulyar bo'lib $x^{2} = 16y$ parabolasiga urinib o'tuvchi to'g'ri chiziqning tenglamasini tuzing.  \\

C1. Agar vaqtning xohlagan momentida $M(x;y)$ nuqta $5x - 16 = 0$ to'g'ri chiziqqa qaraganda $A(5;0)$ nuqtasidan 1,25 marta uzoqroq masofada joylashgan. Shu $M(x;y)$ nuqtaning harakatining tenglamasini tuzing.  \\

C2. $4x^{2} - 4xy + y^{2} - 2x - 14y + 7 = 0$ ITECH tenglamasini kanonik shaklga olib keling, tipini aniqlang, qanday geometrik obraz ekanligini ko'rsating, chizmasini eski va yangi koordinatalar sistemasiga nisbatan chizing.  \\

C3. $A(\frac{10}{3};\frac{5}{3})$ nuqtasidan $\frac{x^{2}}{20} + \frac{y^{2}}{5} = 1$ ellipsiga yurgizilgan urinmalarning tenglamasini tuzing.  \\

\end{tabular}
\vspace{1cm}


\begin{tabular}{m{17cm}}
\textbf{94-variant}
\newline

T1. Ellipsoida. Kanonik tenglamasi (ellipsni simmetriya o'qi atrofida aylantirishdan olingan sirt, kanonik tenglamasi).\\

T2. Giperbolaning polyar koordinatadagi tenglamasi (Polyar burchagi, polyar radiusi giperbolaning polyar tenglamasi)\\

A1. Tipini aniqlang: $3x^{2}-8xy+7y^{2}+8x-15y+20=0$.\\

A2. Aylana tenglamasini tuzing: $M_1(-1;5)$, $M_2(-2;-2)$ va $M_3(5;5)$ nuqtalardan o'tadi.\\

A3. Fokuslari abssissa o'qida va koordinata boshiga nisbatan simmetrik joylashgan ellipsning tenglamasini tuzing: katta o'qi $20$, ekssentrisitet $\varepsilon=3/5$.\\

B1. ITECH ning umumiy tenglamasini koordinata sistemasini almashtirmasdan soddalashtiring, tipini aniqlang, obrazi qanday chiziq ekanligini ko'rsating: $7x^{2} - 8xy + y^{2} - 16x - 2y - 51 = 0$\\

B2. $3x + 4y - 12 = 0$ to'g'ri chizig'i bilan $y^{2} = - 9x$ parabolasining kesishish nuqtalarini toping.  \\

B3. $\rho = \frac{144}{13 - 5cos\theta}$ ellips ekanligini ko'rsating va uning yarim o'qlarini aniqlang.\\

C1. $y^{2} = 20x$ parabolasining $M$ nuqtasini toping, agar uning abssissasi 7 ga teng bo'lsa, fokal radiusini va fokal radiusi joylashgan to'g'rini aniqlang.\\

C2. Fokuslari $F(3;4)$, $F(-3;-4)$ nuqtalarida joylashgan direktrisalari orasidagi masofa 3,6 ga teng bo'lgan giperbolaning tenglamasini tuzing.  \\

C3. $32x^{2} + 52xy - 7y^{2} + 180 = 0$ ITECH tenglamasini kanonik shaklga olib keling, tipini aniqlang, qanday geometrik obraz ekanligini ko'rsating, chizmasini eski va yangi koordinatalar sistemasiga nisbatan chizing.  \\

\end{tabular}
\vspace{1cm}


\begin{tabular}{m{17cm}}
\textbf{95-variant}
\newline

T1. ITECH-ning umumiy tenglamasini soddalashtirish (ITECH-ning umumiy tenglamasi, koordinata sistemasin almashtirish ITECH umumiy tenglamasini soddalashtirish).\\

T2. Ikkinshi tartibli aylanma sirtlar (koordinata sistemasi, tekislik, vektor egri chiziq, aylanma sirt).\\

A1. Giperbola tenglamasi berilgan: $\frac{x^{2}}{25}-\frac{y^{2}}{144}=1$. Uning qutb tenglamasini tuzing.\\

A2. Berilgan chiziqlarning markaziy ekanligini ko'rsating va markazinin toping: $5x^{2}+4xy+2y^{2}+20x+20y-18=0$.\\

A3. Aylana tenglamasini tuzing: markazi $C(1;-1)$ nuqtada joylashgan va $5x-12y+9-0$ to'g'ri chiziqga urinadi.\\

B1. $\frac{x^{2}}{4} - \frac{y^{2}}{5} = 1$, giperbolaning $3x - 2y = 0$ to'g'ri chizig'iga parallel bo'lgan urinmasining tenglamasini tuzing.  \\

B2. Koordinata o'qlarini almashtirmasdan ITECH tenglamasini soddalashtiring, yarim o'qlarnin toping: $4x^{2} - 4xy + 7y^{2} - 26x - 18y + 3 = 0$.\\

B3. $2x + 2y - 3 = 0$ to'g'ri chizig'iga parallel bo'lib $\frac{x^{2}}{16} + \frac{y^{2}}{64} = 1$ giperbolasiga urinib o'tuvchi to'g'ri chiziqning tenglamasini tuzing.  \\

C1. $\frac{x^{2}}{3} - \frac{y^{2}}{5} = 1$, giperbolasiga $P(4;2)$ nuqtadan yurgizilgan urinmalarning tenglamasini tuzing.  \\

C2. $M(2; - \frac{5}{3})$ nuqta $\frac{x^{2}}{9} + \frac{y^{2}}{5} = 1$ ellipsda joylashgan. $M$ nuqtaning fokal radiuslarida yotuvchi to'g'ri chiziq tenglamalarini tuzing.  \\

C3. Katta o'qi 26 ga, fokuslari $F( - 10;0), F(14;0)$ nuqtalarida joylashgan ellipsning tenglamasini tuzing.  \\

\end{tabular}
\vspace{1cm}


\begin{tabular}{m{17cm}}
\textbf{96-variant}
\newline

T1. Ellipsning urinmasining tenglamasi (ellips, to'g'ri chiziq urinish nuqtasi, urinma tenglamasi).\\

T2. ITECH-ning umumiy tenglamasini koordinata boshin parallel ko'chirish bilan soddalastiring (ITECH-ning umumiy tenglamasini parallel ko'chirish formulasi).\\

A1. Uchi koordinata boshida joylashgan va $Ox$ o'qiga nisbatan chap tarafafgi yarim tekislikda joylashgan parabolaning tenglamasini tuzing: parametri $p=0,5$.\\

A2. Qutb tenglamasi bilan berilgan egri chiziqning tipini aniqlang: $\rho=\frac{6}{1-\cos 0}$.\\

A3. Tipini aniqlang: $x^{2}-4xy+4y^{2}+7x-12=0$.\\

B1. $x^{2} + 4y^{2} = 25$ ellipsi bilan $4x - 2y + 23 = 0$ to'g'ri chizig'iga parallel bo'lgan urinma to'g'ri chiziqning tenglamasini tuzing.  \\

B2. $y^{2} = 12x$ paraborolasiga $3x - 2y + 30 = 0$ to'g'ri chizig'iga parallel bo'lgan urinmasining tenglamasini tuzing.  \\

B3. Koordinata o'qlarini almashtirmasdan ITECH umumiy tenglamasini soddalashtiring, yarim o'qlarini toping: $13x^{2} + 18xy + 37y^{2} - 26x - 18y + 3 = 0$.  \\

C1. $4x^{2} - 4xy + y^{2} - 6x + 8y + 13 = 0$ ITECH markazga egami? Markazga ega bo'lsa markazini aniqlang?  \\

C2. $\frac{x^{2}}{3} - \frac{y^{2}}{5} = 1$ giperbolasiga $P(1; - 5)$ nuqtasida yurgizilgan urinmalarning tenglamasini tuzing.\\

C3. $y^{2} = 20x$ parabolasining abssissasi 7 ga teng bo'lgan $M$ nuqtasining fokal radiusini toping va fokal radiusi yotgan to'g'ri chiziqning tenglamasini tuzing.  \\

\end{tabular}
\vspace{1cm}


\begin{tabular}{m{17cm}}
\textbf{97-variant}
\newline

T1. Sirtning kanonik tenglamalari. Sirt haqqida tushuncha. (Sirtning ta'rifi, formulalari, o'q, yo'naltiruvchi to'g'ri chiziqlar).\\

T2. Ellipsning polyar koordinatalardagi tenglamasi (polyar koordinatalar sistemasida ellipsning tenglamasi).\\

A1. Aylana tenglamasini tuzing: $A(1;1)$, $B(1;-1)$ va $C(2;0)$ nuqtalardan o'tadi.\\

A2. Fokuslari abssissa o'qida va koordinata boshiga nisbatan simmetrik joylashgan ellipsning tenglamasini tuzing: katta o'qi $10$, fokuslari orasidagi masofa $2c=8$.\\

A3. Tipini aniqlang: $9x^{2}-16y^{2}-54x-64y-127=0$.\\

B1. $\frac{x^{2}}{16} - \frac{y^{2}}{64} = 1$ giperbolasiga berilgan $10x - 3y + 9 = 0$ to'g'ri chizig'iga parallel bo'lgan urinmasining tenglamasini tuzing.  \\

B2. Koordinata o'qlarini almashtirmasdan ITECH tenglamasini soddalashtiring, qanday geometrik obraz ekanligini ko'rsating: $4x^{2} - 4xy + y^{2} + 4x - 2y + 1 = 0$.  \\

B3. $x^{2} - 4y^{2} = 16$ giperbola berilgan. Uning ekssentrisitetini, fokuslarining koordinatalarini toping va asimptotalarining tenglamalarini tuzing.\\

C1. Fokusi $F(7;2)$ nuqtasida joylashgan, mos direktrisasi $x - 5 = 0$ tenglamasi bilan berilgan parabolaning tenglamasini tuzing.  \\

C2. $16x^{2} - 9y^{2} - 64x - 54y - 161 = 0$ tenglamasi giperbolaning tenglamasi ekanligini ko'rsating va uning markazi $C$ ni, yarim o'qlarini, ekssentrisitetini toping, asimptotalarining tenglamalarini tuzing.  \\

C3. $\frac{x^{2}}{25} + \frac{y^{2}}{16} = 1$, ellipsiga $C(10; - 8)$ nuqtadan yurgizilgan urinmalarining tenglamasini tuzing.  \\

\end{tabular}
\vspace{1cm}


\begin{tabular}{m{17cm}}
\textbf{98-variant}
\newline

T1. Ikkinchi tartibli sirtning umumiy tenglamasi. Markazin aniqlash formulasi.\\

T2. Ikki pallali giperboloid Kanonik tenglamasi (giperbolani simmetriya o'qi atrofida aylantirishdan olingan sirt).\\

A1. Aylananing $C$ markazi va $R$ radiusini toping: $x^2+y^2+6x-4y+14=0$.\\

A2. Fokuslari abssissa o'qida va koordinata boshiga nisbatan simmetrik joylashgan giperbolaning tenglamasini tuzing: direktrisalar orasidagi masofa $8/3$ va ekssentrisitet $\varepsilon=3/2$.\\

A3. Tipini aniqlang: $5x^{2}+14xy+11y^{2}+12x-7y+19=0$.\\

B1. $3x + 10y - 25 = 0$ to'g'ri bilan $\frac{x^{2}}{25} + \frac{y^{2}}{4} = 1$ ellipsning kesishish nuqtalarini toping.  \\

B2. $\rho = \frac{10}{2 - cos\theta}$ polyar tenglamasi bilan qanday chiziq berilganini aniqlang.  \\

B3. $x^{2} - y^{2} = 27$ giperbolasiga $4x + 2y - 7 = 0$ to'g'ri chizigiga parallel bo'lgan urinmasining tenglamasini toping.  \\

C1. $\frac{x^{2}}{100} + \frac{y^{2}}{36} = 1$ ellipsining o'ng tarafdagi fokusidan 14 ga teng masofada bo'lgan nuqtasini toping.  \\

C2. Agar xohlagan vaqt momentida $M(x;y)$ nuqta $A(8;4)$ nuqtasidan va ordinata o'qidan birxil masofada joylashsa, $M(x;y)$ nuqtaning harakat troektoriyasining tenglamasini tuzing.  \\

C3. $14x^{2} + 24xy + 21y^{2} - 4x + 18y - 139 = 0$ egri chizig'ining tipini aniqlang, agar markazga ega egri chiziq bo'lsa, markazining koordinatalarini toping.  \\

\end{tabular}
\vspace{1cm}


\begin{tabular}{m{17cm}}
\textbf{99-variant}
\newline

T1. Ellips va uning kanonik tenglamasi (ta'rifi, fokuslari, ellipsning kanonik tenglamasi, ekstsentrisiteti, direktrisalari).\\

T2. ITECH-ning umumiy tenglamasini koordinata o'qlarini burish bilan soddalashtirish (ITECH-ning umumiy tenglamalari, koordinata o'qin burish formulasi, tenglamani kanonik turga olib kelish).\\

A1. Aylana tenglamasini tuzing: aylana diametrining uchlari $A(3;2)$ va $B(-1;6)$ nuqtalarda joylashgan.\\

A2. Fokuslari abssissa o'qida va koordinata boshiga nisbatan simmetrik joylashgan giperbolaning tenglamasini tuzing: o'qlari $2a=10$ va $2b=8$.\\

A3. Tipini aniqlang: $4x^{2}-y^{2}+8x-2y+3=0$.\\

B1. $41x^{2} + 24xy + 9y^{2} + 24x + 18y - 36 = 0$ ITECH tipini aniqlang va markazlarini toping koordinata o'qlarini almashtirmasdan qanday chiziq ekanligini ko'rsating, yarim o'qlarini toping.  \\

B2. Ellips $3x^{2} + 4y^{2} - 12 = 0$ tenglamasi bilan berilgan. Uning o'qlarining uzunliklarini, fokuslarining koordinatalarini va ekssentrisitetini toping.  \\

B3. $y^{2} = 3x$ parabolasi bilan $\frac{x^{2}}{100} + \frac{y^{2}}{225} = 1$ ellipsining kesishish nuqtalarini toping.  \\

C1. Giperbolaning ekssentrisiteti $\varepsilon = \frac{13}{12}$, fokusi $F(0;13)$ nuqtasida va mos direktrisasi $13y - 144 = 0$ tenglamasi bilan berilgan bo'lsa, giperbolaning tenglamasini tuzing.  \\

C2. $4x^{2} + 24xy + 11y^{2} + 64x + 42y + 51 = 0$ egri chizig'ining tipini aniqlang, agar markazga ega bo'lsa, uning markazining koordinatalarini toping va koordinata boshini markazga parallel ko'chirish amalini bajaring.\\

C3. Fokusi $F( - 1; - 4)$ nuqtasida joylashgan, mos direktrisasi $x - 2 = 0$ tenglamasi bilan berilgan, $A( - 3; - 5)$ nuqtadan o'tuvchi ellipsning tenglamasini tuzing.  \\

\end{tabular}
\vspace{1cm}


\begin{tabular}{m{17cm}}
\textbf{100-variant}
\newline

T1. Silindrlik sirtlar (yasovchi to'g'ri chiziq, yo'naltiruvchi egri chiziq, silindrlik sirt).\\

T2. ITECH-ning umumiy tenglamasini klassifikatsiyalash (ITECH-ning umumiy tenglamasi, ITECH-ning umumiy tenglamasini soddalashtirish, klassifikatsiyalash).\\

A1. Aylana tenglamasini tuzing: markazi $C(2;-3)$ nuqtada joylashgan va radiusi $R=7$ ga teng.\\

A2. Fokuslari abssissa o'qida va koordinata boshiga nisbatan simmetrik joylashgan giperbolaning tenglamasini tuzing: direktrisalar orasidagi masofa $228/13$ va fokuslari orasidagi masofa $2c=26$.\\

A3. Tipini aniqlang: $3x^{2}-2xy-3y^{2}+12y-15=0$.\\

B1. $\rho = \frac{5}{3 - 4cos\theta}$ tenglamasi bilan qanday chiziq berilganini va yarim o'qlarini toping.  \\

B2. $\frac{x^{2}}{20} - \frac{y^{2}}{5} = 1$ giperbolasiga $4x + 3y - 7 = 0$ to'g'ri chizig'iga perpendikulyar bo'lgan urinmasining tenglamasini tuzing.  \\

B3. ITECH ning umumiy tenglamasini koordinata sistemasini almashtirmasdan soddalashtiring, tipini aniqlang, obrazi qanday chiziq ekanligini ko'rsating: $7x^{2} - 8xy + y^{2} - 16x - 2y - 51 = 0$\\

C1. $2x^{2} + 3y^{2} + 8x - 6y + 11 = 0$ tenglamasi bilan qanday tipdagi chiziq berilganini aniqlang va uning tenglamasini soddalashtiring va grafigini chizing.  \\

C2. $y^{2} = 20x$ parabolasining $M$ nuqtasini toping, agar uning abssissasi 7 ga teng bo'lsa, fokal radiusini va fokal radiusi joylashgan to'g'rini aniqlang.\\

C3. Uchi (-4;0) nuqtasinda, direktrisasi $y - 2 = 0$ to'g'ri chiziq bo'lgan parabolaning tenglamasini tuzing.\\

\end{tabular}
\vspace{1cm}

\end{document}