\documentclass{article}
  \usepackage[utf8]{inputenc}
  \usepackage{array}
  \usepackage[a4paper,
  left=15mm,
  top=15mm,]{geometry}
  \usepackage{setspace}
  
  \renewcommand{\baselinestretch}{1.1} 
  
\begin{document}

\large
\pagenumbering{gobble}


\begin{tabular}{m{17cm}}
\textbf{1-variant}
\newline

T1. Parabolanıń polyar koordinatalardaǵı teńlemesi (polyar koordinata sistemasında parabolanıń teńlemesi).\\

T2. ETIS-tıń ulıwma teńlemesin klassifikatsiyalaw (ETIS-tıń ulıwma teńlemesi, ETIS-tıń ulıwma teńlemesin ápiwaylastırıw, klassifikatsiyalaw).\\

A1. Sheńberdiń $C$ orayı hám $R$ radiusın tabıń: $x^2+y^2+4 x-2 y+5=0$.\\

A2. Fokusları abscissa kósherinde hám koordinata basına qarata simmetriyalıq jaylasqan giperbolanıń teńlemesin dúziń: úlken kósheri $2 a=16$ hám ekscentrisitet $\varepsilon=5/4$.\\

A3. Polyar teńlemesi menen berilgen iymek sızıqtıń tipin anıqlań: $\rho=\frac{6}{1-\cos 0}$.\\

B1. $\frac{x^{2}}{20} - \frac{y^{2}}{5} = 1$ giperbolasına $4x + 3y - 7 = 0$ tuwrısına perpendikulyar bolǵan urınbanıń teńlemesin dúziń.  \\

B2. Koordinata kósherlerin túrlendirmey ETİS teńlemesin ápiwaylastırıń, yarım kósherlerin tabıń $41x^{2} + 2xy + 9y^{2} - 26x - 18y + 3 = 0$.  \\

B3. Ellips $3x^{2} + 4y^{2} - 12 = 0$ teńlemesi menen berilgen. Onıń kósherleriniń uzınlıqların, fokuslarınıń koordinataların hám ekscentrisitetin tabıń.  \\

C1. Úlken kósheri 26 ǵa, fokusları $F( - 10;0)$, $F(14;0)$ noqatlarında jaylasqan ellipstiń teńlemesin dúziń.  \\

C2. $32x^{2} + 52xy - 7y^{2} + 180 = 0$ ETİS teńlemesin ápiwayı túrge alıp keliń, tipin anıqlań, qanday geometriyalıq obrazdı anıqlaytuǵının kórsetiń, sızılmasın góne hám taza koordinatalar sistemasına qarata jasań.  \\

C3. $\frac{x^{2}}{3} - \frac{y^{2}}{5} = 1$ giperbolasına $P(1; - 5)$ noqatında júrgizilgen urınbalardıń teńlemesin dúziń.\\

\end{tabular}
\vspace{1cm}


\begin{tabular}{m{17cm}}
\textbf{2-variant}
\newline

T1. Eki gewekli giperboloid. Kanonikalıq teńlemesi (giperbolanı simmetriya kósheri átirapında aylandırıwdan alınǵan betlik).\\

T2. Ellipstiń urınbasınıń teńlemesi (ellips, tuwrı, urınıw tochka, urınba teńlemesi).\\

A1. Berilgen sızıqlardıń oraylıq ekenligin kórsetiń hám orayın tabıń: $5 x^{2}+4 xy+2 y^{2}+20 x+20 y-18=0$.\\

A2. Sheńber teńlemesin dúziń: $A (1;1) $, $B (1;-1) $ hám $C (2;0) $ noqatlardan ótedi.\\

A3. Uchı koordinata basında jaylasqan hám $Ox$ kósherine qarata joqarı yarım tegislikte jaylasqan parabolanıń teńlemesin dúziń: parametri $p=1/4$.\\

B1. $3x + 10y - 25 = 0$ tuwrı menen $\frac{x^{2}}{25} + \frac{y^{2}}{4} = 1$ ellipstiń kesilisiw noqatların tabıń.\\

B2. $\rho = \frac{6}{1 - cos\theta}$ polyar teńlemesi menen qanday sızıq berilgenin anıqlań.  \\

B3. $x^{2} - y^{2} = 27$ giperbolasına $4x + 2y - 7 = 0$ tuwrısına parallel bolǵan urınbanıń teńlemesin tabıń.  \\

C1. $\frac{x^{2}}{100} + \frac{y^{2}}{36} = 1$ ellipsiniń oń jaqtaǵı fokusınan 14 ge teń aralıqta bolǵan noqattı tabıń.  \\

C2. Fokusı $F(2; - 1)$ noqatında jaylasqan, sáykes direktrisası $x - y - 1 = 0$ teńlemesi menen berilgen parabolanıń teńlemesin dúziń.  \\

C3. $2x^{2} + 10xy + 12y^{2} - 7x + 18y - 15 = 0$ ETİS teńlemesin ápiwayı túrge alıp keliń, tipin anıqlań, qanday geometriyalıq obrazdı anıqlaytuǵının kórsetiń, sızılmasın góne hám taza koordinatalar sistemasına qarata jasań  \\

\end{tabular}
\vspace{1cm}


\begin{tabular}{m{17cm}}
\textbf{3-variant}
\newline

T1. ETIS -tiń ulıwma teńlemesin ápiwaylastırıw (ETIS -tiń ulıwma teńlemesi, koordinata sistemasın túrlendirip ETIS ulıwma teńlemesin ápiwaylastırıw).\\

T2. Ellipslik paraboloid (parabola, kósher, ellipslik paraboloid).\\

A1. Polyar teńlemesi menen berilgen iymek sızıqtıń tipin anıqlań: $\rho=\frac{5}{1-\frac{1}{2}\cos\theta}$.\\

A2. Tipin anıqlań: $9 x^{2}+4 y^{2}+18 x-8 y+49=0$.\\

A3. Sheńberdiń $C$ orayı hám $R$ radiusın tabıń: $x^2+y^2-2 x+4 y-14=0$.\\

B1. Koordinata kósherlerin túrlendirmey ETİS ulıwma teńlemesin ápiwaylastırıń, yarım kósherlerin tabıń: $13x^{2} + 18xy + 37y^{2} - 26x - 18y + 3 = 0$.  \\

B2. $x^{2} - 4y^{2} = 16$ giperbola berilgen. Onıń ekscentrisitetin, fokuslarınıń koordinataların tabıń hám asimptotalarınıń teńlemelerin dúziń.\\

B3. $y^{2} = 3x$ parabolası menen $\frac{x^{2}}{100} + \frac{y^{2}}{225} = 1$ ellipsiniń kesilisiw noqatların tabıń.  \\

C1. $\frac{x^{2}}{2} + \frac{y^{2}}{3} = 1$, ellipsin $x + y - 2 = 0$ noqatınan júrgizilgen urınbalarınıń teńlemesin dúziń.  \\

C2. $y^{2} = 20x$ parabolasınıń $M$ noqatın tabıń, eger onıń abscissası 7 ge teń bolsa, fokal radiusın hám fokal radius jaylasqan tuwrını anıqlań.\\

C3. Fokusı $F(7;2)$ noqatında jaylasqan, sáykes direktrisası $x - 5 = 0$ teńlemesi menen berilgen parabolanıń teńlemesin dúziń.  \\

\end{tabular}
\vspace{1cm}


\begin{tabular}{m{17cm}}
\textbf{4-variant}
\newline

T1. Koordinata sistemasın túrlendiriw (birlik vektorlar, kósherler, parallel kóshiriw, koordinata kósherlerin burıw).\\

T2. ETIS-tıń invariantları (ETIS-tıń ulıwma teńlemesi, túrlendiriw, ETIS invariantları ).\\

A1. Fokusları abscissa kósherinde hám koordinata basına qarata simmetriyalıq jaylasqan ellipstiń teńlemesin dúziń: direktrisaları arasındaǵı aralıq $5$ hám fokusları arasındaǵı aralıq $2 c=4$.\\

A2. Ellips teńlemesi berilgen: $\frac{x^2}{25}+\frac{y^2}{16}=1$. Onıń polyar teńlemesin dúziń.\\

A3. Berilgen sızıqlardıń oraylıq ekenligin kórsetiń hám orayın tabıń: $9 x^{2}-4 xy-7 y^{2}-12=0$.\\

B1. $\rho = \frac{144}{13 - 5cos\theta}$ ellipsti anıqlaytuǵının kórsetiń hám onıń yarım kósherlerin anıqlań.\\

B2. $\frac{x^{2}}{4} - \frac{y^{2}}{5} = 1$ giperbolaǵa $3x - 2y = 0$ tuwrısına parallel bolǵan urınbanıń teńlemesin dúziń.  \\

B3. Koordinata kósherlerin túrlendirmey ETİS teńlemesin ápiwaylastırıń, yarım kósherlerin tabıń $4x^{2} - 4xy + 9y^{2} - 26x - 18y + 3 = 0$.\\

C1. $2x^{2} + 3y^{2} + 8x - 6y + 11 = 0$ teńlemesi menen qanday tiptegi sızıq berilgenin anıqlań hám onıń teńlemesin ápiwaylastırıń hám grafigin jasań.  \\

C2. $\frac{x^{2}}{25} + \frac{y^{2}}{16} = 1$, ellipsine $C(10; - 8)$ noqatınan júrgizilgen urınbalarınıń teńlemesin dúziń.  \\

C3. $y^{2} = 20x$ parabolasınıń abscissası 7 ge teń bolǵan $M$ noqatınıń fokal radiusın tabıń hám fokal radiusı jatqan tuwrınıń teńlemesin dúziń.  \\

\end{tabular}
\vspace{1cm}


\begin{tabular}{m{17cm}}
\textbf{5-variant}
\newline

T1. Betlik haqqında túsinik (tuwrı, iymek sızıq, betliktiń anıqlamaları hám formulaları).\\

T2. Parabola hám onıń kanonikalıq teńlemesi (anıqlaması, fokusı, direktrisası, kanonikalıq teńlemesi).\\

A1. Sheńber teńlemesin dúziń: orayı $C (6 ;-8) $ noqatında jaylasqan hám koordinata basınan ótedi.\\

A2. Fokusları abscissa kósherinde hám koordinata basına qarata simmetriyalıq jaylasqan giperbolanıń teńlemesin dúziń: asimptotalar teńlemeleri $y=\pm \frac{4}{3}x$ hám fokusları arasındaǵı aralıq $2 c=20$.\\

A3. Polyar teńlemesi menen berilgen iymek sızıqtıń tipin anıqlań: $\rho=\frac{12}{2-\cos\theta}$.\\

B1. $3x + 4y - 12 = 0$ tuwrı sızıǵı hám $y^{2} = - 9x$ parabolasınıń kesilisiw noqatların tabıń.  \\

B2. $\rho = \frac{10}{2 - cos\theta}$ polyar teńlemesi menen qanday sızıq berilgenin anıqlań.  \\

B3. $\frac{x^{2}}{16} - \frac{y^{2}}{64} = 1$, giperbolasına berilgen $10x - 3y + 9 = 0$ tuwrı sızıǵına parallel bolǵan urınbanıń teńlemesin dúziń.  \\

C1. Eger waqıttıń qálegen momentinde $M(x;y)$ noqat $5x - 16 = 0$ tuwrı sızıqqa qaraǵanda $A(5;0)$ noqattan 1,25 márte uzaqlıqta jaylasqan. Usı $M(x;y)$ noqattıń háreketiniń teńlemesin dúziń.  \\

C2. $4x^{2} - 4xy + y^{2} - 2x - 14y + 7 = 0$ ETİS teńlemesin ápiwayı túrge alıp keliń, tipin anıqlań, qanday geometriyalıq obrazdı anıqlaytuǵının kórsetiń, sızılmasın góne hám taza koordinatalar sistemasına qarata jasań.  \\

C3. $\frac{x^{2}}{3} - \frac{y^{2}}{5} = 1$, giperbolasına $P(4;2)$ noqatınan júrgizilgen urınbalardıń teńlemesin dúziń.  \\

\end{tabular}
\vspace{1cm}


\begin{tabular}{m{17cm}}
\textbf{6-variant}
\newline

T1. ETIS-tıń orayın anıqlaw forması (ETIS-tıń ulıwma teńlemesi, orayın anıqlaw forması).\\

T2. Ellipsoida. Kanonikalıq teńlemesi (ellipsti simmetriya kósheri dogereginde aylandırıwdan alınǵan betlik, kanonikalıq teńlemesi).\\

A1. Tipin anıqlań: $3 x^{2}-8 xy+7 y^{2}+8 x-15 y+20=0$.\\

A2. Sheńber teńlemesin dúziń: $M_1 (-1;5) $, $M_2 (-2;-2) $ i $M_3 (5;5) $ noqatlardan ótedi.\\

A3. Fokusları abscissa kósherinde hám koordinata basına qarata simmetriyalıq jaylasqan ellipstiń teńlemesin dúziń: kishi kósheri $6$, direktrisaları arasındaǵı aralıq $13$.\\

B1. ETİS-tıń ulıwma teńlemesin koordinata sistemasın túrlendirmey ápiwaylastırıń, tipin anıqlań, obrazı qanday sızıqtı anıqlaytuǵının kórsetiń. $7x^{2} - 8xy + y^{2} - 16x - 2y - 51 = 0$  \\

B2. $\rho = \frac{5}{3 - 4cos\theta}$ teńlemesi menen qanday sızıq berilgenin hám yarım kósherlerin tabıń.  \\

B3. $2x + 2y - 3 = 0$ tuwrısına parallel bolıp $\frac{x^{2}}{16} + \frac{y^{2}}{64} = 1$ giperbolasına urınıwshı tuwrınıń teńlemesin dúziń.  \\

C1. $M(2; - \frac{5}{3})$ noqatı $\frac{x^{2}}{9} + \frac{y^{2}}{5} = 1$ ellipsinde jaylasqan. $M$ noqatınıń fokal radiusları jatıwshı tuwrı sızıq teńlemelerin dúziń.  \\

C2. Fokusı $F( - 1; - 4)$ noqatında jaylasqan, sáykes direktrisası $x - 2 = 0$ teńlemesi menen berilgen, $A( - 3; - 5)$ noqatınan ótiwshi ellipstiń teńlemesin dúziń.  \\

C3. $32x^{2} + 52xy - 9y^{2} + 180 = 0$ ETİS teńlemesin ápiwaylastırıń, tipin anıqlań, qanday geometriyalıq obrazdı anıqlaytuǵının kórsetiń, sızılmasın sızıń.  \\

\end{tabular}
\vspace{1cm}


\begin{tabular}{m{17cm}}
\textbf{7-variant}
\newline

T1. Giperbola. Kanonikalıq teńlemesi (fokuslar, kósherler, direktrisalar, giperbola, ekscentrisitet, kanonikalıq teńlemesi).\\

T2. ETIS-tıń ulıwma teńlemesin koordinata basın parallel kóshiriw arqalı ápiwayılastırıń (ETIS- tıń ulıwma teńlemesin parallel kóshiriw formulası).\\

A1. Giperbola teńlemesi berilgen: $\frac{x^{2}}{25}-\frac{y^{2}}{144}=1$. Onıń polyar teńlemesin dúziń.\\

A2. Tipin anıqlań: $x^{2}-4 xy+4 y^{2}+7 x-12=0$.\\

A3. Sheńberdiń $C$ orayı hám $R$ radiusın tabıń: $x^2+y^2+6 x-4 y+14=0$.\\

B1. $41x^{2} + 24xy + 9y^{2} + 24x + 18y - 36 = 0$ ETİS tipin anıqlań hám orayların tabıń koordinata kósherlerin túrlendirmey qanday sızıqtı anıqlaytuǵının kórsetiń yarım kósherlerin tabıń.  \\

B2. $2x + 2y - 3 = 0$ tuwrısına perpendikulyar bolıp $x^{2} = 16y$ parabolasına urınıwshı tuwrınıń teńlemesin dúziń.  \\

B3. Koordinata kósherlerin túrlendirmey ETİS teńlemesin ápiwaylastırıń, qanday geometriyalıq obrazdı anıqlaytuǵının kórsetiń $4x^{2} - 4xy + y^{2} + 4x - 2y + 1 = 0$.  \\

C1. $A(\frac{10}{3};\frac{5}{3})$ noqattan $\frac{x^{2}}{20} + \frac{y^{2}}{5} = 1$ ellipsine júrgizilgen urınbalardıń teńlemesin dúziń.  \\

C2. Fokusı $F( - 1; - 4)$noqatında bolǵan, sáykes direktrissası $x - 2 = 0$ teńlemesi menen berilgen $A( - 3; - 5)$ noqatınan ótiwshi ellipstiń teńlemesin dúziń.  \\

C3. $2x^{2} + 3y^{2} + 8x - 6y + 11 = 0$ teńlemesin ápiwaylastırıń qanday geometriyalıq obrazdı anıqlaytuǵının tabıń hám grafigin jasań.  \\

\end{tabular}
\vspace{1cm}


\begin{tabular}{m{17cm}}
\textbf{8-variant}
\newline

T1. Giperbolalıq paraboloydtıń tuwrı sızıqlı jasawshıları (Giperbolalıq paraboloydtı jasawshı tuwrı sızıqlar dástesi).\\

T2. Ellipstiń polyar koordinatalardaǵı teńlemesi (polyar koordinatalar sistemasında ellipstiń teńlemesi).\\

A1. Uchı koordinata basında jaylasqan hám $Oy$ kósherine qarata shep táreptegi yarım tegislikte jaylasqan parabolanıń teńlemesin dúziń: parametri $p=0,5$.\\

A2. Parabola teńlemesi berilgen: $y^2=6 x$. Onıń polyar teńlemesin dúziń.\\

A3. Berilgen sızıqlardıń oraylıq ekenligin kórsetiń hám orayın tabıń: $2 x^{2}-6 xy+5 y^{2}+22 x-36 y+11=0$.\\

B1. $y^{2} = 12x$ paraborolasına $3x - 2y + 30 = 0$ tuwrı sızıǵına parallel bolǵan urınbanıń teńlemesin dúziń.  \\

B2. $\frac{x^{2}}{4} - \frac{y^{2}}{5} = 1$, giperbolanıń $3x - 2y = 0$ tuwrı sızıǵına parallel bolǵan urınbasınıń teńlemesin dúziń.  \\

B3. $\frac{x^{2}}{4} - \frac{y^{2}}{5} = 1$ giperbolasına $3x + 2y = 0$ tuwrı sızıǵına perpendikulyar bolǵan urınba tuwrınıń teńlemesin dúziń.\\

C1. Eger qálegen waqıt momentinde $M(x;y)$ noqat $A(8;4)$ noqattan hám ordinata kósherinen birdey aralıqta jaylassa, $M(x;y)$ noqatınıń háreket etiw troektoriyasınıń teńlemesin dúziń.  \\

C2. $2x^{2} + 3y^{2} + 8x - 6y + 11 = 0$ teńlemesin ápiwaylastırıń qanday geometriyalıq obrazdı anıqlaytuǵının tabıń hám grafigin jasań.\\

C3. Giperbolanıń ekscentrisiteti $\varepsilon = \frac{13}{12}$, fokusı $F(0;13)$ noqatında hám sáykes direktrisası $13y - 144 = 0$ teńlemesi menen berilgen bolsa, giperbolanıń teńlemesin dúziń.  \\

\end{tabular}
\vspace{1cm}


\begin{tabular}{m{17cm}}
\textbf{9-variant}
\newline

T1. ETIS-tıń ulıwma teńlemesin koordinata kósherlerin burıw arqalı ápiwaylastırıń (ETIS-tıń ulıwma teńlemeleri, koordinata kósherin burıw formulası, teńlemeni kanonik túrge alıp keliw).\\

T2. Cilindrlik betlikler (jasawshı tuwrı sızıq, baǵıtlawshı iymek sızıq, cilindrlik betlik).\\

A1. Sheńber teńlemesin dúziń: orayı $C (2;-3) $ noqatında jaylasqan hám radiusı $R=7$ ge teń.\\

A2. Fokusları abscissa kósherinde hám koordinata basına qarata simmetriyalıq jaylasqan ellipstiń teńlemesin dúziń: kishi kósheri $10$, ekscentrisitet $\varepsilon=12/13$.\\

A3. Giperbola teńlemesi berilgen: $\frac{x^{2}}{16}-\frac{y^{2}}{9}=1$. Onıń polyar teńlemesin dúziń.\\

B1. Ellips $3x^{2} + 4y^{2} - 12 = 0$ teńlemesi menen berilgen. Onıń kósherleriniń uzınlıqların, fokuslarınıń koordinataların hám ekscentrisitetin tabıń.  \\

B2. $x^{2} + 4y^{2} = 25$ ellipsi menen $4x - 2y + 23 = 0$ tuwrı sızıǵına parallel bolǵan urınba tuwrı sızıqtıń teńlemesin dúziń.  \\

B3. Koordinata kósherlerin túrlendirmey ETİS teńlemesin ápiwaylastırıń, yarım kósherlerin tabıń $41x^{2} + 2xy + 9y^{2} - 26x - 18y + 3 = 0$.  \\

C1. $4x^{2} - 4xy + y^{2} - 6x + 8y + 13 = 0$ ETİS-ǵı orayǵa iyeme? Orayǵa iye bolsa orayın anıqlań: jalǵız orayǵa iyeme-?, sheksiz orayǵa iyeme-?  \\

C2. Tóbesi $A(-4;0)$ noqatında, al, direktrisası $y - 2 = 0$ tuwrı sızıq bolǵan parabolanıń teńlemesin dúziń.\\

C3. $16x^{2} - 9y^{2} - 64x - 54y - 161 = 0$ teńlemesi giperbolanıń teńlemesi ekenin anıqlań hám onıń orayı $C$, yarım kósherleri, ekscentrisitetin, asimptotalarınıń teńlemelerin dúziń.  \\

\end{tabular}
\vspace{1cm}


\begin{tabular}{m{17cm}}
\textbf{10-variant}
\newline

T1. Parabolanıń urınbasınıń teńlemesi (parabola, tuwrı, urınıw noqatı, urınba teńlemesi).\\

T2. Bir gewekli giperboloid. Kanonikalıq teńlemesi (giperbolanı simmetriya kósheri átirapında aylandırıwdan alınǵan betlik).\\

A1. Tipin anıqlań: $25 x^{2}-20 xy+4 y^{2}-12 x+20 y-17=0$.\\

A2. Sheńber teńlemesin dúziń: sheńber $A (2;6 ) $ noqatınan ótedi hám orayı $C (-1;2) $ noqatında jaylasqan .\\

A3. Fokusları abscissa kósherinde hám koordinata basına qarata simmetriyalıq jaylasqan giperbolanıń teńlemesin dúziń: oqları $2 a=10$ hám $2 b=8$.\\

B1. $x^{2} - 4y^{2} = 16$ giperbola berilgen. Onıń ekscentrisitetin, fokuslarınıń koordinataların tabıń hám asimptotalarınıń teńlemelerin dúziń.\\

B2. $3x + 10y - 25 = 0$ tuwrı menen $\frac{x^{2}}{25} + \frac{y^{2}}{4} = 1$ ellipstiń kesilisiw noqatların tabıń.\\

B3. $\rho = \frac{6}{1 - cos\theta}$ polyar teńlemesi menen qanday sızıq berilgenin anıqlań.  \\

C1. Fokuslari $F(3;4), F(-3;-4)$ noqatlarında jaylasqan direktrisaları orasıdaǵı aralıq 3,6 ǵa teń bolǵan giperbolanıń teńlemesin dúziń.  \\

C2. $14x^{2} + 24xy + 21y^{2} - 4x + 18y - 139 = 0$ iymek sızıǵınıń tipin anıqlań, eger oraylı iymek sızıq bolsa orayınıń koordinataların tabıń.  \\

C3. $4x^{2} + 24xy + 11y^{2} + 64x + 42y + 51 = 0$ iymek sızıǵınıń tipin anıqlań eger orayı bar bolsa, onıń orayınıń koordinataların tabıń hám koordinata basın orayǵa parallel kóshiriw ámelin orınlań.  \\

\end{tabular}
\vspace{1cm}


\begin{tabular}{m{17cm}}
\textbf{11-variant}
\newline

T1. Giperbolanıń polyar koordinatadaǵı teńlemesi (Polyar múyeshi, polyar radiusi giperbolanıń polyar teńlemesi).\\

T2. Betliktiń kanonikalıq teńlemeleri. Betlik haqqında túsinik. (Betliktiń anıqlaması, formulaları, kósher, baǵıtlawshı tuwrılar).\\

A1. Polyar teńlemesi menen berilgen iymek sızıqtıń tipin anıqlań: $\rho=\frac{5}{3-4\cos\theta}$.\\

A2. Tipin anıqlań: $3 x^{2}-2 xy-3 y^{2}+12 y-15=0$.\\

A3. Sheńberdiń $C$ orayı hám $R$ radiusın tabıń: $x^2+y^2-2 x+4 y-20=0$.\\

B1. $\frac{x^{2}}{20} - \frac{y^{2}}{5} = 1$ giperbolasına $4x + 3y - 7 = 0$ tuwrısına perpendikulyar bolǵan urınbanıń teńlemesin dúziń.  \\

B2. Koordinata kósherlerin túrlendirmey ETİS ulıwma teńlemesin ápiwaylastırıń, yarım kósherlerin tabıń: $13x^{2} + 18xy + 37y^{2} - 26x - 18y + 3 = 0$.  \\

B3. Ellips $3x^{2} + 4y^{2} - 12 = 0$ teńlemesi menen berilgen. Onıń kósherleriniń uzınlıqların, fokuslarınıń koordinataların hám ekscentrisitetin tabıń.  \\

C1. Úlken kósheri 26 ǵa, fokusları $F( - 10;0)$, $F(14;0)$ noqatlarında jaylasqan ellipstiń teńlemesin dúziń.  \\

C2. $32x^{2} + 52xy - 7y^{2} + 180 = 0$ ETİS teńlemesin ápiwayı túrge alıp keliń, tipin anıqlań, qanday geometriyalıq obrazdı anıqlaytuǵının kórsetiń, sızılmasın góne hám taza koordinatalar sistemasına qarata jasań.  \\

C3. $\frac{x^{2}}{3} - \frac{y^{2}}{5} = 1$ giperbolasına $P(1; - 5)$ noqatında júrgizilgen urınbalardıń teńlemesin dúziń.\\

\end{tabular}
\vspace{1cm}


\begin{tabular}{m{17cm}}
\textbf{12-variant}
\newline

T1. Giperbolanıń urınbasınıń teńlemesi (giperbolaǵa berilgen noqatta júrgizilgen urınba teńlemesi).\\

T2. Ekinshi tártipli betliktiń ulıwma teńlemesi. Orayın anıqlaw formulası.\\

A1. Fokusları abscissa kósherinde hám koordinata basına qarata simmetriyalıq jaylasqan ellipstiń teńlemesin dúziń: yarım oqları 5 hám 2.\\

A2. Polyar teńlemesi menen berilgen iymek sızıqtıń tipin anıqlań: $\rho=\frac{1}{3-3\cos\theta}$.\\

A3. Tipin anıqlań: $2 x^{2}+3 y^{2}+8 x-6 y+11=0$.\\

B1. $y^{2} = 3x$ parabolası menen $\frac{x^{2}}{100} + \frac{y^{2}}{225} = 1$ ellipsiniń kesilisiw noqatların tabıń.  \\

B2. $\rho = \frac{144}{13 - 5cos\theta}$ ellipsti anıqlaytuǵının kórsetiń hám onıń yarım kósherlerin anıqlań.\\

B3. $x^{2} - y^{2} = 27$ giperbolasına $4x + 2y - 7 = 0$ tuwrısına parallel bolǵan urınbanıń teńlemesin tabıń.  \\

C1. $\frac{x^{2}}{100} + \frac{y^{2}}{36} = 1$ ellipsiniń oń jaqtaǵı fokusınan 14 ge teń aralıqta bolǵan noqattı tabıń.  \\

C2. Fokusı $F(2; - 1)$ noqatında jaylasqan, sáykes direktrisası $x - y - 1 = 0$ teńlemesi menen berilgen parabolanıń teńlemesin dúziń.  \\

C3. $2x^{2} + 10xy + 12y^{2} - 7x + 18y - 15 = 0$ ETİS teńlemesin ápiwayı túrge alıp keliń, tipin anıqlań, qanday geometriyalıq obrazdı anıqlaytuǵının kórsetiń, sızılmasın góne hám taza koordinatalar sistemasına qarata jasań  \\

\end{tabular}
\vspace{1cm}


\begin{tabular}{m{17cm}}
\textbf{13-variant}
\newline

T1. Ellips hám onıń kanonikalıq teńlemesi (anıqlaması, fokuslar, ellipstiń kanonikalıq teńlemesi, ekscentrisiteti, direktrisaları).\\

T2. Ekinshi tártipli aylanba betlikler (koordinata sisteması, tegislik, vektor iymek sızıq, aylanba betlik).\\

A1. Sheńber teńlemesin dúziń: orayı koordinata basında jaylasqan hám radiusı $R=3$ ge teń.\\

A2. Uchı koordinata basında jaylasqan hám $Oy$ kósherine qarata oń táreptegi yarım tegislikte jaylasqan parabolanıń teńlemesin dúziń: parametri $p=3$.\\

A3. Polyar teńlemesi menen berilgen iymek sızıqtıń tipin anıqlań: $\rho=\frac{10}{1-\frac{3}{2}\cos\theta}$.\\

B1. Koordinata kósherlerin túrlendirmey ETİS teńlemesin ápiwaylastırıń, yarım kósherlerin tabıń $4x^{2} - 4xy + 9y^{2} - 26x - 18y + 3 = 0$.\\

B2. $3x + 4y - 12 = 0$ tuwrı sızıǵı hám $y^{2} = - 9x$ parabolasınıń kesilisiw noqatların tabıń.  \\

B3. $\rho = \frac{10}{2 - cos\theta}$ polyar teńlemesi menen qanday sızıq berilgenin anıqlań.  \\

C1. $\frac{x^{2}}{2} + \frac{y^{2}}{3} = 1$, ellipsin $x + y - 2 = 0$ noqatınan júrgizilgen urınbalarınıń teńlemesin dúziń.  \\

C2. $y^{2} = 20x$ parabolasınıń $M$ noqatın tabıń, eger onıń abscissası 7 ge teń bolsa, fokal radiusın hám fokal radius jaylasqan tuwrını anıqlań.\\

C3. Fokusı $F(7;2)$ noqatında jaylasqan, sáykes direktrisası $x - 5 = 0$ teńlemesi menen berilgen parabolanıń teńlemesin dúziń.  \\

\end{tabular}
\vspace{1cm}


\begin{tabular}{m{17cm}}
\textbf{14-variant}
\newline

T1. Parabolanıń polyar koordinatalardaǵı teńlemesi (polyar koordinata sistemasında parabolanıń teńlemesi).\\

T2. ETIS-tıń ulıwma teńlemesin klassifikatsiyalaw (ETIS-tıń ulıwma teńlemesi, ETIS-tıń ulıwma teńlemesin ápiwaylastırıw, klassifikatsiyalaw).\\

A1. Berilgen sızıqlardıń oraylıq ekenligin kórsetiń hám orayın tabıń: $3 x^{2}+5 xy+y^{2}-8 x-11 y-7=0$.\\

A2. Sheńber teńlemesin dúziń: orayı $C (1;-1) $ noqatında jaylasqan hám $5 x-12 y+9 -0$ tuwrı sızıǵına urınadı .\\

A3. Fokusları abscissa kósherinde hám koordinata basına qarata simmetriyalıq jaylasqan ellipstiń teńlemesin dúziń: úlken kósheri $8$, direktrisaları arasındaǵı aralıq $16$.\\

B1. $\frac{x^{2}}{4} - \frac{y^{2}}{5} = 1$ giperbolaǵa $3x - 2y = 0$ tuwrısına parallel bolǵan urınbanıń teńlemesin dúziń.  \\

B2. ETİS-tıń ulıwma teńlemesin koordinata sistemasın túrlendirmey ápiwaylastırıń, tipin anıqlań, obrazı qanday sızıqtı anıqlaytuǵının kórsetiń. $7x^{2} - 8xy + y^{2} - 16x - 2y - 51 = 0$  \\

B3. $\rho = \frac{5}{3 - 4cos\theta}$ teńlemesi menen qanday sızıq berilgenin hám yarım kósherlerin tabıń.  \\

C1. $2x^{2} + 3y^{2} + 8x - 6y + 11 = 0$ teńlemesi menen qanday tiptegi sızıq berilgenin anıqlań hám onıń teńlemesin ápiwaylastırıń hám grafigin jasań.  \\

C2. $\frac{x^{2}}{25} + \frac{y^{2}}{16} = 1$, ellipsine $C(10; - 8)$ noqatınan júrgizilgen urınbalarınıń teńlemesin dúziń.  \\

C3. $y^{2} = 20x$ parabolasınıń abscissası 7 ge teń bolǵan $M$ noqatınıń fokal radiusın tabıń hám fokal radiusı jatqan tuwrınıń teńlemesin dúziń.  \\

\end{tabular}
\vspace{1cm}


\begin{tabular}{m{17cm}}
\textbf{15-variant}
\newline

T1. Eki gewekli giperboloid. Kanonikalıq teńlemesi (giperbolanı simmetriya kósheri átirapında aylandırıwdan alınǵan betlik).\\

T2. Ellipstiń urınbasınıń teńlemesi (ellips, tuwrı, urınıw tochka, urınba teńlemesi).\\

A1. Tipin anıqlań: $2 x^{2}+10 xy+12 y^{2}-7 x+18 y-15=0$.\\

A2. Sheńber teńlemesin dúziń: orayı koordinata basında jaylasqan hám $3 x-4 y+20=0$ tuwrı sızıǵına urınadı.\\

A3. Fokusları abscissa kósherinde hám koordinata basına qarata simmetriyalıq jaylasqan ellipstiń teńlemesin dúziń: úlken kósheri $20$, ekscentrisitet $\varepsilon=3/5$.\\

B1. $\frac{x^{2}}{16} - \frac{y^{2}}{64} = 1$, giperbolasına berilgen $10x - 3y + 9 = 0$ tuwrı sızıǵına parallel bolǵan urınbanıń teńlemesin dúziń.  \\

B2. $41x^{2} + 24xy + 9y^{2} + 24x + 18y - 36 = 0$ ETİS tipin anıqlań hám orayların tabıń koordinata kósherlerin túrlendirmey qanday sızıqtı anıqlaytuǵının kórsetiń yarım kósherlerin tabıń.  \\

B3. $2x + 2y - 3 = 0$ tuwrısına parallel bolıp $\frac{x^{2}}{16} + \frac{y^{2}}{64} = 1$ giperbolasına urınıwshı tuwrınıń teńlemesin dúziń.  \\

C1. Eger waqıttıń qálegen momentinde $M(x;y)$ noqat $5x - 16 = 0$ tuwrı sızıqqa qaraǵanda $A(5;0)$ noqattan 1,25 márte uzaqlıqta jaylasqan. Usı $M(x;y)$ noqattıń háreketiniń teńlemesin dúziń.  \\

C2. $4x^{2} - 4xy + y^{2} - 2x - 14y + 7 = 0$ ETİS teńlemesin ápiwayı túrge alıp keliń, tipin anıqlań, qanday geometriyalıq obrazdı anıqlaytuǵının kórsetiń, sızılmasın góne hám taza koordinatalar sistemasına qarata jasań.  \\

C3. $\frac{x^{2}}{3} - \frac{y^{2}}{5} = 1$, giperbolasına $P(4;2)$ noqatınan júrgizilgen urınbalardıń teńlemesin dúziń.  \\

\end{tabular}
\vspace{1cm}


\begin{tabular}{m{17cm}}
\textbf{16-variant}
\newline

T1. ETIS -tiń ulıwma teńlemesin ápiwaylastırıw (ETIS -tiń ulıwma teńlemesi, koordinata sistemasın túrlendirip ETIS ulıwma teńlemesin ápiwaylastırıw).\\

T2. Ellipslik paraboloid (parabola, kósher, ellipslik paraboloid).\\

A1. Tipin anıqlań: $4 x^{2}-y^{2}+8 x-2 y+3=0$.\\

A2. Sheńber teńlemesin dúziń: sheńber diametriniń ushları $A (3;2) $ hám $B (-1;6 ) $ noqatlarında jaylasqan.\\

A3. Fokusları abscissa kósherinde hám koordinata basına qarata simmetriyalıq jaylasqan giperbolanıń teńlemesin dúziń: direktrisaları arasındaǵı aralıq $228/13$ hám fokusları arasındaǵı aralıq $2 c=26$.\\

B1. Koordinata kósherlerin túrlendirmey ETİS teńlemesin ápiwaylastırıń, qanday geometriyalıq obrazdı anıqlaytuǵının kórsetiń $4x^{2} - 4xy + y^{2} + 4x - 2y + 1 = 0$.  \\

B2. $2x + 2y - 3 = 0$ tuwrısına perpendikulyar bolıp $x^{2} = 16y$ parabolasına urınıwshı tuwrınıń teńlemesin dúziń.  \\

B3. $y^{2} = 12x$ paraborolasına $3x - 2y + 30 = 0$ tuwrı sızıǵına parallel bolǵan urınbanıń teńlemesin dúziń.  \\

C1. $M(2; - \frac{5}{3})$ noqatı $\frac{x^{2}}{9} + \frac{y^{2}}{5} = 1$ ellipsinde jaylasqan. $M$ noqatınıń fokal radiusları jatıwshı tuwrı sızıq teńlemelerin dúziń.  \\

C2. Fokusı $F( - 1; - 4)$ noqatında jaylasqan, sáykes direktrisası $x - 2 = 0$ teńlemesi menen berilgen, $A( - 3; - 5)$ noqatınan ótiwshi ellipstiń teńlemesin dúziń.  \\

C3. $32x^{2} + 52xy - 9y^{2} + 180 = 0$ ETİS teńlemesin ápiwaylastırıń, tipin anıqlań, qanday geometriyalıq obrazdı anıqlaytuǵının kórsetiń, sızılmasın sızıń.  \\

\end{tabular}
\vspace{1cm}


\begin{tabular}{m{17cm}}
\textbf{17-variant}
\newline

T1. Koordinata sistemasın túrlendiriw (birlik vektorlar, kósherler, parallel kóshiriw, koordinata kósherlerin burıw).\\

T2. ETIS-tıń invariantları (ETIS-tıń ulıwma teńlemesi, túrlendiriw, ETIS invariantları ).\\

A1. Tipin anıqlań: $9 x^{2}-16 y^{2}-54 x-64 y-127=0$.\\

A2. Sheńber teńlemesin dúziń: $A (3;1) $ hám $B (-1;3) $ noqatlardan ótedi, orayı $3 x-y-2=0$ tuwrı sızıǵında jaylasqan .\\

A3. Uchı koordinata basında jaylasqan hám $Ox$ kósherine qarata tómengi yarım tegislikte jaylasqan parabolanıń teńlemesin dúziń: parametri $p=3$.\\

B1. $\frac{x^{2}}{4} - \frac{y^{2}}{5} = 1$, giperbolanıń $3x - 2y = 0$ tuwrı sızıǵına parallel bolǵan urınbasınıń teńlemesin dúziń.  \\

B2. $x^{2} - 4y^{2} = 16$ giperbola berilgen. Onıń ekscentrisitetin, fokuslarınıń koordinataların tabıń hám asimptotalarınıń teńlemelerin dúziń.\\

B3. $\frac{x^{2}}{4} - \frac{y^{2}}{5} = 1$ giperbolasına $3x + 2y = 0$ tuwrı sızıǵına perpendikulyar bolǵan urınba tuwrınıń teńlemesin dúziń.\\

C1. $A(\frac{10}{3};\frac{5}{3})$ noqattan $\frac{x^{2}}{20} + \frac{y^{2}}{5} = 1$ ellipsine júrgizilgen urınbalardıń teńlemesin dúziń.  \\

C2. Fokusı $F( - 1; - 4)$noqatında bolǵan, sáykes direktrissası $x - 2 = 0$ teńlemesi menen berilgen $A( - 3; - 5)$ noqatınan ótiwshi ellipstiń teńlemesin dúziń.  \\

C3. $2x^{2} + 3y^{2} + 8x - 6y + 11 = 0$ teńlemesin ápiwaylastırıń qanday geometriyalıq obrazdı anıqlaytuǵının tabıń hám grafigin jasań.  \\

\end{tabular}
\vspace{1cm}


\begin{tabular}{m{17cm}}
\textbf{18-variant}
\newline

T1. Betlik haqqında túsinik (tuwrı, iymek sızıq, betliktiń anıqlamaları hám formulaları).\\

T2. Parabola hám onıń kanonikalıq teńlemesi (anıqlaması, fokusı, direktrisası, kanonikalıq teńlemesi).\\

A1. Tipin anıqlań: $5 x^{2}+14 xy+11 y^{2}+12 x-7 y+19=0$.\\

A2. Fokusları abscissa kósherinde hám koordinata basına qarata simmetriyalıq jaylasqan giperbolanıń teńlemesin dúziń: direktrisaları arasındaǵı aralıq $32/5$ hám kósheri $2 b=6$.\\

A3. Tipin anıqlań: $4 x^2+9 y^2-40 x+36 y+100=0$.\\

B1. Koordinata kósherlerin túrlendirmey ETİS teńlemesin ápiwaylastırıń, yarım kósherlerin tabıń $41x^{2} + 2xy + 9y^{2} - 26x - 18y + 3 = 0$.  \\

B2. Ellips $3x^{2} + 4y^{2} - 12 = 0$ teńlemesi menen berilgen. Onıń kósherleriniń uzınlıqların, fokuslarınıń koordinataların hám ekscentrisitetin tabıń.  \\

B3. $3x + 10y - 25 = 0$ tuwrı menen $\frac{x^{2}}{25} + \frac{y^{2}}{4} = 1$ ellipstiń kesilisiw noqatların tabıń.\\

C1. Eger qálegen waqıt momentinde $M(x;y)$ noqat $A(8;4)$ noqattan hám ordinata kósherinen birdey aralıqta jaylassa, $M(x;y)$ noqatınıń háreket etiw troektoriyasınıń teńlemesin dúziń.  \\

C2. $2x^{2} + 3y^{2} + 8x - 6y + 11 = 0$ teńlemesin ápiwaylastırıń qanday geometriyalıq obrazdı anıqlaytuǵının tabıń hám grafigin jasań.\\

C3. Giperbolanıń ekscentrisiteti $\varepsilon = \frac{13}{12}$, fokusı $F(0;13)$ noqatında hám sáykes direktrisası $13y - 144 = 0$ teńlemesi menen berilgen bolsa, giperbolanıń teńlemesin dúziń.  \\

\end{tabular}
\vspace{1cm}


\begin{tabular}{m{17cm}}
\textbf{19-variant}
\newline

T1. ETIS-tıń orayın anıqlaw forması (ETIS-tıń ulıwma teńlemesi, orayın anıqlaw forması).\\

T2. Ellipsoida. Kanonikalıq teńlemesi (ellipsti simmetriya kósheri dogereginde aylandırıwdan alınǵan betlik, kanonikalıq teńlemesi).\\

A1. Fokusları abscissa kósherinde hám koordinata basına qarata simmetriyalıq jaylasqan ellipstiń teńlemesin dúziń: fokusları arasındaǵı aralıq $2 c=6$ hám ekscentrisitet $\varepsilon=3/5$.\\

A2. Fokusları abscissa kósherinde hám koordinata basına qarata simmetriyalıq jaylasqan giperbolanıń teńlemesin dúziń: fokusları arasındaǵı aralıǵı $2 c=10$ hám kósheri $2 b=8$.\\

A3. Fokusları abscissa kósherinde hám koordinata basına qarata simmetriyalıq jaylasqan ellipstiń teńlemesin dúziń: úlken kósheri $10$, fokusları arasındaǵı aralıq $2 c=8$.\\

B1. $\rho = \frac{6}{1 - cos\theta}$ polyar teńlemesi menen qanday sızıq berilgenin anıqlań.  \\

B2. $x^{2} + 4y^{2} = 25$ ellipsi menen $4x - 2y + 23 = 0$ tuwrı sızıǵına parallel bolǵan urınba tuwrı sızıqtıń teńlemesin dúziń.  \\

B3. Koordinata kósherlerin túrlendirmey ETİS ulıwma teńlemesin ápiwaylastırıń, yarım kósherlerin tabıń: $13x^{2} + 18xy + 37y^{2} - 26x - 18y + 3 = 0$.  \\

C1. $4x^{2} - 4xy + y^{2} - 6x + 8y + 13 = 0$ ETİS-ǵı orayǵa iyeme? Orayǵa iye bolsa orayın anıqlań: jalǵız orayǵa iyeme-?, sheksiz orayǵa iyeme-?  \\

C2. Tóbesi $A(-4;0)$ noqatında, al, direktrisası $y - 2 = 0$ tuwrı sızıq bolǵan parabolanıń teńlemesin dúziń.\\

C3. $16x^{2} - 9y^{2} - 64x - 54y - 161 = 0$ teńlemesi giperbolanıń teńlemesi ekenin anıqlań hám onıń orayı $C$, yarım kósherleri, ekscentrisitetin, asimptotalarınıń teńlemelerin dúziń.  \\

\end{tabular}
\vspace{1cm}


\begin{tabular}{m{17cm}}
\textbf{20-variant}
\newline

T1. Giperbola. Kanonikalıq teńlemesi (fokuslar, kósherler, direktrisalar, giperbola, ekscentrisitet, kanonikalıq teńlemesi).\\

T2. ETIS-tıń ulıwma teńlemesin koordinata basın parallel kóshiriw arqalı ápiwayılastırıń (ETIS- tıń ulıwma teńlemesin parallel kóshiriw formulası).\\

A1. Fokusları abscissa kósherinde hám koordinata basına qarata simmetriyalıq jaylasqan ellipstiń teńlemesin dúziń: kishi kósheri $24$, fokusları arasındaǵı aralıq $2 c=10$.\\

A2. Fokusları abscissa kósherinde hám koordinata basına qarata simmetriyalıq jaylasqan giperbolanıń teńlemesin dúziń: fokusları arasındaǵı aralıq $2 c=6$ hám ekscentrisitet $\varepsilon=3/2$.\\

A3. Fokusları abscissa kósherinde hám koordinata basına qarata simmetriyalıq jaylasqan giperbolanıń teńlemesin dúziń: direktrisaları arasındaǵı aralıq $8/3$ hám ekscentrisitet $\varepsilon=3/2$.\\

B1. $x^{2} - 4y^{2} = 16$ giperbola berilgen. Onıń ekscentrisitetin, fokuslarınıń koordinataların tabıń hám asimptotalarınıń teńlemelerin dúziń.\\

B2. $y^{2} = 3x$ parabolası menen $\frac{x^{2}}{100} + \frac{y^{2}}{225} = 1$ ellipsiniń kesilisiw noqatların tabıń.  \\

B3. $\rho = \frac{144}{13 - 5cos\theta}$ ellipsti anıqlaytuǵının kórsetiń hám onıń yarım kósherlerin anıqlań.\\

C1. Fokuslari $F(3;4), F(-3;-4)$ noqatlarında jaylasqan direktrisaları orasıdaǵı aralıq 3,6 ǵa teń bolǵan giperbolanıń teńlemesin dúziń.  \\

C2. $14x^{2} + 24xy + 21y^{2} - 4x + 18y - 139 = 0$ iymek sızıǵınıń tipin anıqlań, eger oraylı iymek sızıq bolsa orayınıń koordinataların tabıń.  \\

C3. $4x^{2} + 24xy + 11y^{2} + 64x + 42y + 51 = 0$ iymek sızıǵınıń tipin anıqlań eger orayı bar bolsa, onıń orayınıń koordinataların tabıń hám koordinata basın orayǵa parallel kóshiriw ámelin orınlań.  \\

\end{tabular}
\vspace{1cm}


\begin{tabular}{m{17cm}}
\textbf{21-variant}
\newline

T1. Giperbolalıq paraboloydtıń tuwrı sızıqlı jasawshıları (Giperbolalıq paraboloydtı jasawshı tuwrı sızıqlar dástesi).\\

T2. Ellipstiń polyar koordinatalardaǵı teńlemesi (polyar koordinatalar sistemasında ellipstiń teńlemesi).\\

A1. Sheńberdiń $C$ orayı hám $R$ radiusın tabıń: $x^2+y^2+4 x-2 y+5=0$.\\

A2. Fokusları abscissa kósherinde hám koordinata basına qarata simmetriyalıq jaylasqan giperbolanıń teńlemesin dúziń: úlken kósheri $2 a=16$ hám ekscentrisitet $\varepsilon=5/4$.\\

A3. Polyar teńlemesi menen berilgen iymek sızıqtıń tipin anıqlań: $\rho=\frac{6}{1-\cos 0}$.\\

B1. $\frac{x^{2}}{20} - \frac{y^{2}}{5} = 1$ giperbolasına $4x + 3y - 7 = 0$ tuwrısına perpendikulyar bolǵan urınbanıń teńlemesin dúziń.  \\

B2. Koordinata kósherlerin túrlendirmey ETİS teńlemesin ápiwaylastırıń, yarım kósherlerin tabıń $4x^{2} - 4xy + 9y^{2} - 26x - 18y + 3 = 0$.\\

B3. $3x + 4y - 12 = 0$ tuwrı sızıǵı hám $y^{2} = - 9x$ parabolasınıń kesilisiw noqatların tabıń.  \\

C1. Úlken kósheri 26 ǵa, fokusları $F( - 10;0)$, $F(14;0)$ noqatlarında jaylasqan ellipstiń teńlemesin dúziń.  \\

C2. $32x^{2} + 52xy - 7y^{2} + 180 = 0$ ETİS teńlemesin ápiwayı túrge alıp keliń, tipin anıqlań, qanday geometriyalıq obrazdı anıqlaytuǵının kórsetiń, sızılmasın góne hám taza koordinatalar sistemasına qarata jasań.  \\

C3. $\frac{x^{2}}{3} - \frac{y^{2}}{5} = 1$ giperbolasına $P(1; - 5)$ noqatında júrgizilgen urınbalardıń teńlemesin dúziń.\\

\end{tabular}
\vspace{1cm}


\begin{tabular}{m{17cm}}
\textbf{22-variant}
\newline

T1. ETIS-tıń ulıwma teńlemesin koordinata kósherlerin burıw arqalı ápiwaylastırıń (ETIS-tıń ulıwma teńlemeleri, koordinata kósherin burıw formulası, teńlemeni kanonik túrge alıp keliw).\\

T2. Cilindrlik betlikler (jasawshı tuwrı sızıq, baǵıtlawshı iymek sızıq, cilindrlik betlik).\\

A1. Berilgen sızıqlardıń oraylıq ekenligin kórsetiń hám orayın tabıń: $5 x^{2}+4 xy+2 y^{2}+20 x+20 y-18=0$.\\

A2. Sheńber teńlemesin dúziń: $A (1;1) $, $B (1;-1) $ hám $C (2;0) $ noqatlardan ótedi.\\

A3. Uchı koordinata basında jaylasqan hám $Ox$ kósherine qarata joqarı yarım tegislikte jaylasqan parabolanıń teńlemesin dúziń: parametri $p=1/4$.\\

B1. $\rho = \frac{10}{2 - cos\theta}$ polyar teńlemesi menen qanday sızıq berilgenin anıqlań.  \\

B2. $x^{2} - y^{2} = 27$ giperbolasına $4x + 2y - 7 = 0$ tuwrısına parallel bolǵan urınbanıń teńlemesin tabıń.  \\

B3. ETİS-tıń ulıwma teńlemesin koordinata sistemasın túrlendirmey ápiwaylastırıń, tipin anıqlań, obrazı qanday sızıqtı anıqlaytuǵının kórsetiń. $7x^{2} - 8xy + y^{2} - 16x - 2y - 51 = 0$  \\

C1. $\frac{x^{2}}{100} + \frac{y^{2}}{36} = 1$ ellipsiniń oń jaqtaǵı fokusınan 14 ge teń aralıqta bolǵan noqattı tabıń.  \\

C2. Fokusı $F(2; - 1)$ noqatında jaylasqan, sáykes direktrisası $x - y - 1 = 0$ teńlemesi menen berilgen parabolanıń teńlemesin dúziń.  \\

C3. $2x^{2} + 10xy + 12y^{2} - 7x + 18y - 15 = 0$ ETİS teńlemesin ápiwayı túrge alıp keliń, tipin anıqlań, qanday geometriyalıq obrazdı anıqlaytuǵının kórsetiń, sızılmasın góne hám taza koordinatalar sistemasına qarata jasań  \\

\end{tabular}
\vspace{1cm}


\begin{tabular}{m{17cm}}
\textbf{23-variant}
\newline

T1. Parabolanıń urınbasınıń teńlemesi (parabola, tuwrı, urınıw noqatı, urınba teńlemesi).\\

T2. Bir gewekli giperboloid. Kanonikalıq teńlemesi (giperbolanı simmetriya kósheri átirapında aylandırıwdan alınǵan betlik).\\

A1. Polyar teńlemesi menen berilgen iymek sızıqtıń tipin anıqlań: $\rho=\frac{5}{1-\frac{1}{2}\cos\theta}$.\\

A2. Tipin anıqlań: $9 x^{2}+4 y^{2}+18 x-8 y+49=0$.\\

A3. Sheńberdiń $C$ orayı hám $R$ radiusın tabıń: $x^2+y^2-2 x+4 y-14=0$.\\

B1. $\rho = \frac{5}{3 - 4cos\theta}$ teńlemesi menen qanday sızıq berilgenin hám yarım kósherlerin tabıń.  \\

B2. $\frac{x^{2}}{4} - \frac{y^{2}}{5} = 1$ giperbolaǵa $3x - 2y = 0$ tuwrısına parallel bolǵan urınbanıń teńlemesin dúziń.  \\

B3. $41x^{2} + 24xy + 9y^{2} + 24x + 18y - 36 = 0$ ETİS tipin anıqlań hám orayların tabıń koordinata kósherlerin túrlendirmey qanday sızıqtı anıqlaytuǵının kórsetiń yarım kósherlerin tabıń.  \\

C1. $\frac{x^{2}}{2} + \frac{y^{2}}{3} = 1$, ellipsin $x + y - 2 = 0$ noqatınan júrgizilgen urınbalarınıń teńlemesin dúziń.  \\

C2. $y^{2} = 20x$ parabolasınıń $M$ noqatın tabıń, eger onıń abscissası 7 ge teń bolsa, fokal radiusın hám fokal radius jaylasqan tuwrını anıqlań.\\

C3. Fokusı $F(7;2)$ noqatında jaylasqan, sáykes direktrisası $x - 5 = 0$ teńlemesi menen berilgen parabolanıń teńlemesin dúziń.  \\

\end{tabular}
\vspace{1cm}


\begin{tabular}{m{17cm}}
\textbf{24-variant}
\newline

T1. Giperbolanıń polyar koordinatadaǵı teńlemesi (Polyar múyeshi, polyar radiusi giperbolanıń polyar teńlemesi).\\

T2. Betliktiń kanonikalıq teńlemeleri. Betlik haqqında túsinik. (Betliktiń anıqlaması, formulaları, kósher, baǵıtlawshı tuwrılar).\\

A1. Fokusları abscissa kósherinde hám koordinata basına qarata simmetriyalıq jaylasqan ellipstiń teńlemesin dúziń: direktrisaları arasındaǵı aralıq $5$ hám fokusları arasındaǵı aralıq $2 c=4$.\\

A2. Ellips teńlemesi berilgen: $\frac{x^2}{25}+\frac{y^2}{16}=1$. Onıń polyar teńlemesin dúziń.\\

A3. Berilgen sızıqlardıń oraylıq ekenligin kórsetiń hám orayın tabıń: $9 x^{2}-4 xy-7 y^{2}-12=0$.\\

B1. $\frac{x^{2}}{16} - \frac{y^{2}}{64} = 1$, giperbolasına berilgen $10x - 3y + 9 = 0$ tuwrı sızıǵına parallel bolǵan urınbanıń teńlemesin dúziń.  \\

B2. Koordinata kósherlerin túrlendirmey ETİS teńlemesin ápiwaylastırıń, qanday geometriyalıq obrazdı anıqlaytuǵının kórsetiń $4x^{2} - 4xy + y^{2} + 4x - 2y + 1 = 0$.  \\

B3. $2x + 2y - 3 = 0$ tuwrısına parallel bolıp $\frac{x^{2}}{16} + \frac{y^{2}}{64} = 1$ giperbolasına urınıwshı tuwrınıń teńlemesin dúziń.  \\

C1. $2x^{2} + 3y^{2} + 8x - 6y + 11 = 0$ teńlemesi menen qanday tiptegi sızıq berilgenin anıqlań hám onıń teńlemesin ápiwaylastırıń hám grafigin jasań.  \\

C2. $\frac{x^{2}}{25} + \frac{y^{2}}{16} = 1$, ellipsine $C(10; - 8)$ noqatınan júrgizilgen urınbalarınıń teńlemesin dúziń.  \\

C3. $y^{2} = 20x$ parabolasınıń abscissası 7 ge teń bolǵan $M$ noqatınıń fokal radiusın tabıń hám fokal radiusı jatqan tuwrınıń teńlemesin dúziń.  \\

\end{tabular}
\vspace{1cm}


\begin{tabular}{m{17cm}}
\textbf{25-variant}
\newline

T1. Giperbolanıń urınbasınıń teńlemesi (giperbolaǵa berilgen noqatta júrgizilgen urınba teńlemesi).\\

T2. Ekinshi tártipli betliktiń ulıwma teńlemesi. Orayın anıqlaw formulası.\\

A1. Sheńber teńlemesin dúziń: orayı $C (6 ;-8) $ noqatında jaylasqan hám koordinata basınan ótedi.\\

A2. Fokusları abscissa kósherinde hám koordinata basına qarata simmetriyalıq jaylasqan giperbolanıń teńlemesin dúziń: asimptotalar teńlemeleri $y=\pm \frac{4}{3}x$ hám fokusları arasındaǵı aralıq $2 c=20$.\\

A3. Polyar teńlemesi menen berilgen iymek sızıqtıń tipin anıqlań: $\rho=\frac{12}{2-\cos\theta}$.\\

B1. $2x + 2y - 3 = 0$ tuwrısına perpendikulyar bolıp $x^{2} = 16y$ parabolasına urınıwshı tuwrınıń teńlemesin dúziń.  \\

B2. $y^{2} = 12x$ paraborolasına $3x - 2y + 30 = 0$ tuwrı sızıǵına parallel bolǵan urınbanıń teńlemesin dúziń.  \\

B3. Ellips $3x^{2} + 4y^{2} - 12 = 0$ teńlemesi menen berilgen. Onıń kósherleriniń uzınlıqların, fokuslarınıń koordinataların hám ekscentrisitetin tabıń.  \\

C1. Eger waqıttıń qálegen momentinde $M(x;y)$ noqat $5x - 16 = 0$ tuwrı sızıqqa qaraǵanda $A(5;0)$ noqattan 1,25 márte uzaqlıqta jaylasqan. Usı $M(x;y)$ noqattıń háreketiniń teńlemesin dúziń.  \\

C2. $4x^{2} - 4xy + y^{2} - 2x - 14y + 7 = 0$ ETİS teńlemesin ápiwayı túrge alıp keliń, tipin anıqlań, qanday geometriyalıq obrazdı anıqlaytuǵının kórsetiń, sızılmasın góne hám taza koordinatalar sistemasına qarata jasań.  \\

C3. $\frac{x^{2}}{3} - \frac{y^{2}}{5} = 1$, giperbolasına $P(4;2)$ noqatınan júrgizilgen urınbalardıń teńlemesin dúziń.  \\

\end{tabular}
\vspace{1cm}


\begin{tabular}{m{17cm}}
\textbf{26-variant}
\newline

T1. Ellips hám onıń kanonikalıq teńlemesi (anıqlaması, fokuslar, ellipstiń kanonikalıq teńlemesi, ekscentrisiteti, direktrisaları).\\

T2. Ekinshi tártipli aylanba betlikler (koordinata sisteması, tegislik, vektor iymek sızıq, aylanba betlik).\\

A1. Tipin anıqlań: $3 x^{2}-8 xy+7 y^{2}+8 x-15 y+20=0$.\\

A2. Sheńber teńlemesin dúziń: $M_1 (-1;5) $, $M_2 (-2;-2) $ i $M_3 (5;5) $ noqatlardan ótedi.\\

A3. Fokusları abscissa kósherinde hám koordinata basına qarata simmetriyalıq jaylasqan ellipstiń teńlemesin dúziń: kishi kósheri $6$, direktrisaları arasındaǵı aralıq $13$.\\

B1. $\frac{x^{2}}{4} - \frac{y^{2}}{5} = 1$, giperbolanıń $3x - 2y = 0$ tuwrı sızıǵına parallel bolǵan urınbasınıń teńlemesin dúziń.  \\

B2. Koordinata kósherlerin túrlendirmey ETİS teńlemesin ápiwaylastırıń, yarım kósherlerin tabıń $41x^{2} + 2xy + 9y^{2} - 26x - 18y + 3 = 0$.  \\

B3. $x^{2} - 4y^{2} = 16$ giperbola berilgen. Onıń ekscentrisitetin, fokuslarınıń koordinataların tabıń hám asimptotalarınıń teńlemelerin dúziń.\\

C1. $M(2; - \frac{5}{3})$ noqatı $\frac{x^{2}}{9} + \frac{y^{2}}{5} = 1$ ellipsinde jaylasqan. $M$ noqatınıń fokal radiusları jatıwshı tuwrı sızıq teńlemelerin dúziń.  \\

C2. Fokusı $F( - 1; - 4)$ noqatında jaylasqan, sáykes direktrisası $x - 2 = 0$ teńlemesi menen berilgen, $A( - 3; - 5)$ noqatınan ótiwshi ellipstiń teńlemesin dúziń.  \\

C3. $32x^{2} + 52xy - 9y^{2} + 180 = 0$ ETİS teńlemesin ápiwaylastırıń, tipin anıqlań, qanday geometriyalıq obrazdı anıqlaytuǵının kórsetiń, sızılmasın sızıń.  \\

\end{tabular}
\vspace{1cm}


\begin{tabular}{m{17cm}}
\textbf{27-variant}
\newline

T1. Parabolanıń polyar koordinatalardaǵı teńlemesi (polyar koordinata sistemasında parabolanıń teńlemesi).\\

T2. ETIS-tıń ulıwma teńlemesin klassifikatsiyalaw (ETIS-tıń ulıwma teńlemesi, ETIS-tıń ulıwma teńlemesin ápiwaylastırıw, klassifikatsiyalaw).\\

A1. Giperbola teńlemesi berilgen: $\frac{x^{2}}{25}-\frac{y^{2}}{144}=1$. Onıń polyar teńlemesin dúziń.\\

A2. Tipin anıqlań: $x^{2}-4 xy+4 y^{2}+7 x-12=0$.\\

A3. Sheńberdiń $C$ orayı hám $R$ radiusın tabıń: $x^2+y^2+6 x-4 y+14=0$.\\

B1. $3x + 10y - 25 = 0$ tuwrı menen $\frac{x^{2}}{25} + \frac{y^{2}}{4} = 1$ ellipstiń kesilisiw noqatların tabıń.\\

B2. $\rho = \frac{6}{1 - cos\theta}$ polyar teńlemesi menen qanday sızıq berilgenin anıqlań.  \\

B3. $\frac{x^{2}}{4} - \frac{y^{2}}{5} = 1$ giperbolasına $3x + 2y = 0$ tuwrı sızıǵına perpendikulyar bolǵan urınba tuwrınıń teńlemesin dúziń.\\

C1. $A(\frac{10}{3};\frac{5}{3})$ noqattan $\frac{x^{2}}{20} + \frac{y^{2}}{5} = 1$ ellipsine júrgizilgen urınbalardıń teńlemesin dúziń.  \\

C2. Fokusı $F( - 1; - 4)$noqatında bolǵan, sáykes direktrissası $x - 2 = 0$ teńlemesi menen berilgen $A( - 3; - 5)$ noqatınan ótiwshi ellipstiń teńlemesin dúziń.  \\

C3. $2x^{2} + 3y^{2} + 8x - 6y + 11 = 0$ teńlemesin ápiwaylastırıń qanday geometriyalıq obrazdı anıqlaytuǵının tabıń hám grafigin jasań.  \\

\end{tabular}
\vspace{1cm}


\begin{tabular}{m{17cm}}
\textbf{28-variant}
\newline

T1. Eki gewekli giperboloid. Kanonikalıq teńlemesi (giperbolanı simmetriya kósheri átirapında aylandırıwdan alınǵan betlik).\\

T2. Ellipstiń urınbasınıń teńlemesi (ellips, tuwrı, urınıw tochka, urınba teńlemesi).\\

A1. Uchı koordinata basında jaylasqan hám $Oy$ kósherine qarata shep táreptegi yarım tegislikte jaylasqan parabolanıń teńlemesin dúziń: parametri $p=0,5$.\\

A2. Parabola teńlemesi berilgen: $y^2=6 x$. Onıń polyar teńlemesin dúziń.\\

A3. Berilgen sızıqlardıń oraylıq ekenligin kórsetiń hám orayın tabıń: $2 x^{2}-6 xy+5 y^{2}+22 x-36 y+11=0$.\\

B1. Koordinata kósherlerin túrlendirmey ETİS ulıwma teńlemesin ápiwaylastırıń, yarım kósherlerin tabıń: $13x^{2} + 18xy + 37y^{2} - 26x - 18y + 3 = 0$.  \\

B2. Ellips $3x^{2} + 4y^{2} - 12 = 0$ teńlemesi menen berilgen. Onıń kósherleriniń uzınlıqların, fokuslarınıń koordinataların hám ekscentrisitetin tabıń.  \\

B3. $y^{2} = 3x$ parabolası menen $\frac{x^{2}}{100} + \frac{y^{2}}{225} = 1$ ellipsiniń kesilisiw noqatların tabıń.  \\

C1. Eger qálegen waqıt momentinde $M(x;y)$ noqat $A(8;4)$ noqattan hám ordinata kósherinen birdey aralıqta jaylassa, $M(x;y)$ noqatınıń háreket etiw troektoriyasınıń teńlemesin dúziń.  \\

C2. $2x^{2} + 3y^{2} + 8x - 6y + 11 = 0$ teńlemesin ápiwaylastırıń qanday geometriyalıq obrazdı anıqlaytuǵının tabıń hám grafigin jasań.\\

C3. Giperbolanıń ekscentrisiteti $\varepsilon = \frac{13}{12}$, fokusı $F(0;13)$ noqatında hám sáykes direktrisası $13y - 144 = 0$ teńlemesi menen berilgen bolsa, giperbolanıń teńlemesin dúziń.  \\

\end{tabular}
\vspace{1cm}


\begin{tabular}{m{17cm}}
\textbf{29-variant}
\newline

T1. ETIS -tiń ulıwma teńlemesin ápiwaylastırıw (ETIS -tiń ulıwma teńlemesi, koordinata sistemasın túrlendirip ETIS ulıwma teńlemesin ápiwaylastırıw).\\

T2. Ellipslik paraboloid (parabola, kósher, ellipslik paraboloid).\\

A1. Sheńber teńlemesin dúziń: orayı $C (2;-3) $ noqatında jaylasqan hám radiusı $R=7$ ge teń.\\

A2. Fokusları abscissa kósherinde hám koordinata basına qarata simmetriyalıq jaylasqan ellipstiń teńlemesin dúziń: kishi kósheri $10$, ekscentrisitet $\varepsilon=12/13$.\\

A3. Giperbola teńlemesi berilgen: $\frac{x^{2}}{16}-\frac{y^{2}}{9}=1$. Onıń polyar teńlemesin dúziń.\\

B1. $\rho = \frac{144}{13 - 5cos\theta}$ ellipsti anıqlaytuǵının kórsetiń hám onıń yarım kósherlerin anıqlań.\\

B2. $x^{2} + 4y^{2} = 25$ ellipsi menen $4x - 2y + 23 = 0$ tuwrı sızıǵına parallel bolǵan urınba tuwrı sızıqtıń teńlemesin dúziń.  \\

B3. Koordinata kósherlerin túrlendirmey ETİS teńlemesin ápiwaylastırıń, yarım kósherlerin tabıń $4x^{2} - 4xy + 9y^{2} - 26x - 18y + 3 = 0$.\\

C1. $4x^{2} - 4xy + y^{2} - 6x + 8y + 13 = 0$ ETİS-ǵı orayǵa iyeme? Orayǵa iye bolsa orayın anıqlań: jalǵız orayǵa iyeme-?, sheksiz orayǵa iyeme-?  \\

C2. Tóbesi $A(-4;0)$ noqatında, al, direktrisası $y - 2 = 0$ tuwrı sızıq bolǵan parabolanıń teńlemesin dúziń.\\

C3. $16x^{2} - 9y^{2} - 64x - 54y - 161 = 0$ teńlemesi giperbolanıń teńlemesi ekenin anıqlań hám onıń orayı $C$, yarım kósherleri, ekscentrisitetin, asimptotalarınıń teńlemelerin dúziń.  \\

\end{tabular}
\vspace{1cm}


\begin{tabular}{m{17cm}}
\textbf{30-variant}
\newline

T1. Koordinata sistemasın túrlendiriw (birlik vektorlar, kósherler, parallel kóshiriw, koordinata kósherlerin burıw).\\

T2. ETIS-tıń invariantları (ETIS-tıń ulıwma teńlemesi, túrlendiriw, ETIS invariantları ).\\

A1. Tipin anıqlań: $25 x^{2}-20 xy+4 y^{2}-12 x+20 y-17=0$.\\

A2. Sheńber teńlemesin dúziń: sheńber $A (2;6 ) $ noqatınan ótedi hám orayı $C (-1;2) $ noqatında jaylasqan .\\

A3. Fokusları abscissa kósherinde hám koordinata basına qarata simmetriyalıq jaylasqan giperbolanıń teńlemesin dúziń: oqları $2 a=10$ hám $2 b=8$.\\

B1. $3x + 4y - 12 = 0$ tuwrı sızıǵı hám $y^{2} = - 9x$ parabolasınıń kesilisiw noqatların tabıń.  \\

B2. $\rho = \frac{10}{2 - cos\theta}$ polyar teńlemesi menen qanday sızıq berilgenin anıqlań.  \\

B3. $\frac{x^{2}}{20} - \frac{y^{2}}{5} = 1$ giperbolasına $4x + 3y - 7 = 0$ tuwrısına perpendikulyar bolǵan urınbanıń teńlemesin dúziń.  \\

C1. Fokuslari $F(3;4), F(-3;-4)$ noqatlarında jaylasqan direktrisaları orasıdaǵı aralıq 3,6 ǵa teń bolǵan giperbolanıń teńlemesin dúziń.  \\

C2. $14x^{2} + 24xy + 21y^{2} - 4x + 18y - 139 = 0$ iymek sızıǵınıń tipin anıqlań, eger oraylı iymek sızıq bolsa orayınıń koordinataların tabıń.  \\

C3. $4x^{2} + 24xy + 11y^{2} + 64x + 42y + 51 = 0$ iymek sızıǵınıń tipin anıqlań eger orayı bar bolsa, onıń orayınıń koordinataların tabıń hám koordinata basın orayǵa parallel kóshiriw ámelin orınlań.  \\

\end{tabular}
\vspace{1cm}


\begin{tabular}{m{17cm}}
\textbf{31-variant}
\newline

T1. Betlik haqqında túsinik (tuwrı, iymek sızıq, betliktiń anıqlamaları hám formulaları).\\

T2. Parabola hám onıń kanonikalıq teńlemesi (anıqlaması, fokusı, direktrisası, kanonikalıq teńlemesi).\\

A1. Polyar teńlemesi menen berilgen iymek sızıqtıń tipin anıqlań: $\rho=\frac{5}{3-4\cos\theta}$.\\

A2. Tipin anıqlań: $3 x^{2}-2 xy-3 y^{2}+12 y-15=0$.\\

A3. Sheńberdiń $C$ orayı hám $R$ radiusın tabıń: $x^2+y^2-2 x+4 y-20=0$.\\

B1. ETİS-tıń ulıwma teńlemesin koordinata sistemasın túrlendirmey ápiwaylastırıń, tipin anıqlań, obrazı qanday sızıqtı anıqlaytuǵının kórsetiń. $7x^{2} - 8xy + y^{2} - 16x - 2y - 51 = 0$  \\

B2. $\rho = \frac{5}{3 - 4cos\theta}$ teńlemesi menen qanday sızıq berilgenin hám yarım kósherlerin tabıń.  \\

B3. $x^{2} - y^{2} = 27$ giperbolasına $4x + 2y - 7 = 0$ tuwrısına parallel bolǵan urınbanıń teńlemesin tabıń.  \\

C1. Úlken kósheri 26 ǵa, fokusları $F( - 10;0)$, $F(14;0)$ noqatlarında jaylasqan ellipstiń teńlemesin dúziń.  \\

C2. $32x^{2} + 52xy - 7y^{2} + 180 = 0$ ETİS teńlemesin ápiwayı túrge alıp keliń, tipin anıqlań, qanday geometriyalıq obrazdı anıqlaytuǵının kórsetiń, sızılmasın góne hám taza koordinatalar sistemasına qarata jasań.  \\

C3. $\frac{x^{2}}{3} - \frac{y^{2}}{5} = 1$ giperbolasına $P(1; - 5)$ noqatında júrgizilgen urınbalardıń teńlemesin dúziń.\\

\end{tabular}
\vspace{1cm}


\begin{tabular}{m{17cm}}
\textbf{32-variant}
\newline

T1. ETIS-tıń orayın anıqlaw forması (ETIS-tıń ulıwma teńlemesi, orayın anıqlaw forması).\\

T2. Ellipsoida. Kanonikalıq teńlemesi (ellipsti simmetriya kósheri dogereginde aylandırıwdan alınǵan betlik, kanonikalıq teńlemesi).\\

A1. Fokusları abscissa kósherinde hám koordinata basına qarata simmetriyalıq jaylasqan ellipstiń teńlemesin dúziń: yarım oqları 5 hám 2.\\

A2. Polyar teńlemesi menen berilgen iymek sızıqtıń tipin anıqlań: $\rho=\frac{1}{3-3\cos\theta}$.\\

A3. Tipin anıqlań: $2 x^{2}+3 y^{2}+8 x-6 y+11=0$.\\

B1. $41x^{2} + 24xy + 9y^{2} + 24x + 18y - 36 = 0$ ETİS tipin anıqlań hám orayların tabıń koordinata kósherlerin túrlendirmey qanday sızıqtı anıqlaytuǵının kórsetiń yarım kósherlerin tabıń.  \\

B2. $\frac{x^{2}}{4} - \frac{y^{2}}{5} = 1$ giperbolaǵa $3x - 2y = 0$ tuwrısına parallel bolǵan urınbanıń teńlemesin dúziń.  \\

B3. Koordinata kósherlerin túrlendirmey ETİS teńlemesin ápiwaylastırıń, qanday geometriyalıq obrazdı anıqlaytuǵının kórsetiń $4x^{2} - 4xy + y^{2} + 4x - 2y + 1 = 0$.  \\

C1. $\frac{x^{2}}{100} + \frac{y^{2}}{36} = 1$ ellipsiniń oń jaqtaǵı fokusınan 14 ge teń aralıqta bolǵan noqattı tabıń.  \\

C2. Fokusı $F(2; - 1)$ noqatında jaylasqan, sáykes direktrisası $x - y - 1 = 0$ teńlemesi menen berilgen parabolanıń teńlemesin dúziń.  \\

C3. $2x^{2} + 10xy + 12y^{2} - 7x + 18y - 15 = 0$ ETİS teńlemesin ápiwayı túrge alıp keliń, tipin anıqlań, qanday geometriyalıq obrazdı anıqlaytuǵının kórsetiń, sızılmasın góne hám taza koordinatalar sistemasına qarata jasań  \\

\end{tabular}
\vspace{1cm}


\begin{tabular}{m{17cm}}
\textbf{33-variant}
\newline

T1. Giperbola. Kanonikalıq teńlemesi (fokuslar, kósherler, direktrisalar, giperbola, ekscentrisitet, kanonikalıq teńlemesi).\\

T2. ETIS-tıń ulıwma teńlemesin koordinata basın parallel kóshiriw arqalı ápiwayılastırıń (ETIS- tıń ulıwma teńlemesin parallel kóshiriw formulası).\\

A1. Sheńber teńlemesin dúziń: orayı koordinata basında jaylasqan hám radiusı $R=3$ ge teń.\\

A2. Uchı koordinata basında jaylasqan hám $Oy$ kósherine qarata oń táreptegi yarım tegislikte jaylasqan parabolanıń teńlemesin dúziń: parametri $p=3$.\\

A3. Polyar teńlemesi menen berilgen iymek sızıqtıń tipin anıqlań: $\rho=\frac{10}{1-\frac{3}{2}\cos\theta}$.\\

B1. $\frac{x^{2}}{16} - \frac{y^{2}}{64} = 1$, giperbolasına berilgen $10x - 3y + 9 = 0$ tuwrı sızıǵına parallel bolǵan urınbanıń teńlemesin dúziń.  \\

B2. $2x + 2y - 3 = 0$ tuwrısına parallel bolıp $\frac{x^{2}}{16} + \frac{y^{2}}{64} = 1$ giperbolasına urınıwshı tuwrınıń teńlemesin dúziń.  \\

B3. $2x + 2y - 3 = 0$ tuwrısına perpendikulyar bolıp $x^{2} = 16y$ parabolasına urınıwshı tuwrınıń teńlemesin dúziń.  \\

C1. $\frac{x^{2}}{2} + \frac{y^{2}}{3} = 1$, ellipsin $x + y - 2 = 0$ noqatınan júrgizilgen urınbalarınıń teńlemesin dúziń.  \\

C2. $y^{2} = 20x$ parabolasınıń $M$ noqatın tabıń, eger onıń abscissası 7 ge teń bolsa, fokal radiusın hám fokal radius jaylasqan tuwrını anıqlań.\\

C3. Fokusı $F(7;2)$ noqatında jaylasqan, sáykes direktrisası $x - 5 = 0$ teńlemesi menen berilgen parabolanıń teńlemesin dúziń.  \\

\end{tabular}
\vspace{1cm}


\begin{tabular}{m{17cm}}
\textbf{34-variant}
\newline

T1. Giperbolalıq paraboloydtıń tuwrı sızıqlı jasawshıları (Giperbolalıq paraboloydtı jasawshı tuwrı sızıqlar dástesi).\\

T2. Ellipstiń polyar koordinatalardaǵı teńlemesi (polyar koordinatalar sistemasında ellipstiń teńlemesi).\\

A1. Berilgen sızıqlardıń oraylıq ekenligin kórsetiń hám orayın tabıń: $3 x^{2}+5 xy+y^{2}-8 x-11 y-7=0$.\\

A2. Sheńber teńlemesin dúziń: orayı $C (1;-1) $ noqatında jaylasqan hám $5 x-12 y+9 -0$ tuwrı sızıǵına urınadı .\\

A3. Fokusları abscissa kósherinde hám koordinata basına qarata simmetriyalıq jaylasqan ellipstiń teńlemesin dúziń: úlken kósheri $8$, direktrisaları arasındaǵı aralıq $16$.\\

B1. $x^{2} - 4y^{2} = 16$ giperbola berilgen. Onıń ekscentrisitetin, fokuslarınıń koordinataların tabıń hám asimptotalarınıń teńlemelerin dúziń.\\

B2. $y^{2} = 12x$ paraborolasına $3x - 2y + 30 = 0$ tuwrı sızıǵına parallel bolǵan urınbanıń teńlemesin dúziń.  \\

B3. Koordinata kósherlerin túrlendirmey ETİS teńlemesin ápiwaylastırıń, yarım kósherlerin tabıń $41x^{2} + 2xy + 9y^{2} - 26x - 18y + 3 = 0$.  \\

C1. $2x^{2} + 3y^{2} + 8x - 6y + 11 = 0$ teńlemesi menen qanday tiptegi sızıq berilgenin anıqlań hám onıń teńlemesin ápiwaylastırıń hám grafigin jasań.  \\

C2. $\frac{x^{2}}{25} + \frac{y^{2}}{16} = 1$, ellipsine $C(10; - 8)$ noqatınan júrgizilgen urınbalarınıń teńlemesin dúziń.  \\

C3. $y^{2} = 20x$ parabolasınıń abscissası 7 ge teń bolǵan $M$ noqatınıń fokal radiusın tabıń hám fokal radiusı jatqan tuwrınıń teńlemesin dúziń.  \\

\end{tabular}
\vspace{1cm}


\begin{tabular}{m{17cm}}
\textbf{35-variant}
\newline

T1. ETIS-tıń ulıwma teńlemesin koordinata kósherlerin burıw arqalı ápiwaylastırıń (ETIS-tıń ulıwma teńlemeleri, koordinata kósherin burıw formulası, teńlemeni kanonik túrge alıp keliw).\\

T2. Cilindrlik betlikler (jasawshı tuwrı sızıq, baǵıtlawshı iymek sızıq, cilindrlik betlik).\\

A1. Tipin anıqlań: $2 x^{2}+10 xy+12 y^{2}-7 x+18 y-15=0$.\\

A2. Sheńber teńlemesin dúziń: orayı koordinata basında jaylasqan hám $3 x-4 y+20=0$ tuwrı sızıǵına urınadı.\\

A3. Fokusları abscissa kósherinde hám koordinata basına qarata simmetriyalıq jaylasqan ellipstiń teńlemesin dúziń: úlken kósheri $20$, ekscentrisitet $\varepsilon=3/5$.\\

B1. Ellips $3x^{2} + 4y^{2} - 12 = 0$ teńlemesi menen berilgen. Onıń kósherleriniń uzınlıqların, fokuslarınıń koordinataların hám ekscentrisitetin tabıń.  \\

B2. $3x + 10y - 25 = 0$ tuwrı menen $\frac{x^{2}}{25} + \frac{y^{2}}{4} = 1$ ellipstiń kesilisiw noqatların tabıń.\\

B3. $\rho = \frac{6}{1 - cos\theta}$ polyar teńlemesi menen qanday sızıq berilgenin anıqlań.  \\

C1. Eger waqıttıń qálegen momentinde $M(x;y)$ noqat $5x - 16 = 0$ tuwrı sızıqqa qaraǵanda $A(5;0)$ noqattan 1,25 márte uzaqlıqta jaylasqan. Usı $M(x;y)$ noqattıń háreketiniń teńlemesin dúziń.  \\

C2. $4x^{2} - 4xy + y^{2} - 2x - 14y + 7 = 0$ ETİS teńlemesin ápiwayı túrge alıp keliń, tipin anıqlań, qanday geometriyalıq obrazdı anıqlaytuǵının kórsetiń, sızılmasın góne hám taza koordinatalar sistemasına qarata jasań.  \\

C3. $\frac{x^{2}}{3} - \frac{y^{2}}{5} = 1$, giperbolasına $P(4;2)$ noqatınan júrgizilgen urınbalardıń teńlemesin dúziń.  \\

\end{tabular}
\vspace{1cm}


\begin{tabular}{m{17cm}}
\textbf{36-variant}
\newline

T1. Parabolanıń urınbasınıń teńlemesi (parabola, tuwrı, urınıw noqatı, urınba teńlemesi).\\

T2. Bir gewekli giperboloid. Kanonikalıq teńlemesi (giperbolanı simmetriya kósheri átirapında aylandırıwdan alınǵan betlik).\\

A1. Tipin anıqlań: $4 x^{2}-y^{2}+8 x-2 y+3=0$.\\

A2. Sheńber teńlemesin dúziń: sheńber diametriniń ushları $A (3;2) $ hám $B (-1;6 ) $ noqatlarında jaylasqan.\\

A3. Fokusları abscissa kósherinde hám koordinata basına qarata simmetriyalıq jaylasqan giperbolanıń teńlemesin dúziń: direktrisaları arasındaǵı aralıq $228/13$ hám fokusları arasındaǵı aralıq $2 c=26$.\\

B1. $\frac{x^{2}}{4} - \frac{y^{2}}{5} = 1$, giperbolanıń $3x - 2y = 0$ tuwrı sızıǵına parallel bolǵan urınbasınıń teńlemesin dúziń.  \\

B2. Koordinata kósherlerin túrlendirmey ETİS ulıwma teńlemesin ápiwaylastırıń, yarım kósherlerin tabıń: $13x^{2} + 18xy + 37y^{2} - 26x - 18y + 3 = 0$.  \\

B3. $x^{2} - 4y^{2} = 16$ giperbola berilgen. Onıń ekscentrisitetin, fokuslarınıń koordinataların tabıń hám asimptotalarınıń teńlemelerin dúziń.\\

C1. $M(2; - \frac{5}{3})$ noqatı $\frac{x^{2}}{9} + \frac{y^{2}}{5} = 1$ ellipsinde jaylasqan. $M$ noqatınıń fokal radiusları jatıwshı tuwrı sızıq teńlemelerin dúziń.  \\

C2. Fokusı $F( - 1; - 4)$ noqatında jaylasqan, sáykes direktrisası $x - 2 = 0$ teńlemesi menen berilgen, $A( - 3; - 5)$ noqatınan ótiwshi ellipstiń teńlemesin dúziń.  \\

C3. $32x^{2} + 52xy - 9y^{2} + 180 = 0$ ETİS teńlemesin ápiwaylastırıń, tipin anıqlań, qanday geometriyalıq obrazdı anıqlaytuǵının kórsetiń, sızılmasın sızıń.  \\

\end{tabular}
\vspace{1cm}


\begin{tabular}{m{17cm}}
\textbf{37-variant}
\newline

T1. Giperbolanıń polyar koordinatadaǵı teńlemesi (Polyar múyeshi, polyar radiusi giperbolanıń polyar teńlemesi).\\

T2. Betliktiń kanonikalıq teńlemeleri. Betlik haqqında túsinik. (Betliktiń anıqlaması, formulaları, kósher, baǵıtlawshı tuwrılar).\\

A1. Tipin anıqlań: $9 x^{2}-16 y^{2}-54 x-64 y-127=0$.\\

A2. Sheńber teńlemesin dúziń: $A (3;1) $ hám $B (-1;3) $ noqatlardan ótedi, orayı $3 x-y-2=0$ tuwrı sızıǵında jaylasqan .\\

A3. Uchı koordinata basında jaylasqan hám $Ox$ kósherine qarata tómengi yarım tegislikte jaylasqan parabolanıń teńlemesin dúziń: parametri $p=3$.\\

B1. $y^{2} = 3x$ parabolası menen $\frac{x^{2}}{100} + \frac{y^{2}}{225} = 1$ ellipsiniń kesilisiw noqatların tabıń.  \\

B2. $\rho = \frac{144}{13 - 5cos\theta}$ ellipsti anıqlaytuǵının kórsetiń hám onıń yarım kósherlerin anıqlań.\\

B3. $\frac{x^{2}}{4} - \frac{y^{2}}{5} = 1$ giperbolasına $3x + 2y = 0$ tuwrı sızıǵına perpendikulyar bolǵan urınba tuwrınıń teńlemesin dúziń.\\

C1. $A(\frac{10}{3};\frac{5}{3})$ noqattan $\frac{x^{2}}{20} + \frac{y^{2}}{5} = 1$ ellipsine júrgizilgen urınbalardıń teńlemesin dúziń.  \\

C2. Fokusı $F( - 1; - 4)$noqatında bolǵan, sáykes direktrissası $x - 2 = 0$ teńlemesi menen berilgen $A( - 3; - 5)$ noqatınan ótiwshi ellipstiń teńlemesin dúziń.  \\

C3. $2x^{2} + 3y^{2} + 8x - 6y + 11 = 0$ teńlemesin ápiwaylastırıń qanday geometriyalıq obrazdı anıqlaytuǵının tabıń hám grafigin jasań.  \\

\end{tabular}
\vspace{1cm}


\begin{tabular}{m{17cm}}
\textbf{38-variant}
\newline

T1. Giperbolanıń urınbasınıń teńlemesi (giperbolaǵa berilgen noqatta júrgizilgen urınba teńlemesi).\\

T2. Ekinshi tártipli betliktiń ulıwma teńlemesi. Orayın anıqlaw formulası.\\

A1. Tipin anıqlań: $5 x^{2}+14 xy+11 y^{2}+12 x-7 y+19=0$.\\

A2. Fokusları abscissa kósherinde hám koordinata basına qarata simmetriyalıq jaylasqan giperbolanıń teńlemesin dúziń: direktrisaları arasındaǵı aralıq $32/5$ hám kósheri $2 b=6$.\\

A3. Tipin anıqlań: $4 x^2+9 y^2-40 x+36 y+100=0$.\\

B1. Koordinata kósherlerin túrlendirmey ETİS teńlemesin ápiwaylastırıń, yarım kósherlerin tabıń $4x^{2} - 4xy + 9y^{2} - 26x - 18y + 3 = 0$.\\

B2. $3x + 4y - 12 = 0$ tuwrı sızıǵı hám $y^{2} = - 9x$ parabolasınıń kesilisiw noqatların tabıń.  \\

B3. $\rho = \frac{10}{2 - cos\theta}$ polyar teńlemesi menen qanday sızıq berilgenin anıqlań.  \\

C1. Eger qálegen waqıt momentinde $M(x;y)$ noqat $A(8;4)$ noqattan hám ordinata kósherinen birdey aralıqta jaylassa, $M(x;y)$ noqatınıń háreket etiw troektoriyasınıń teńlemesin dúziń.  \\

C2. $2x^{2} + 3y^{2} + 8x - 6y + 11 = 0$ teńlemesin ápiwaylastırıń qanday geometriyalıq obrazdı anıqlaytuǵının tabıń hám grafigin jasań.\\

C3. Giperbolanıń ekscentrisiteti $\varepsilon = \frac{13}{12}$, fokusı $F(0;13)$ noqatında hám sáykes direktrisası $13y - 144 = 0$ teńlemesi menen berilgen bolsa, giperbolanıń teńlemesin dúziń.  \\

\end{tabular}
\vspace{1cm}


\begin{tabular}{m{17cm}}
\textbf{39-variant}
\newline

T1. Ellips hám onıń kanonikalıq teńlemesi (anıqlaması, fokuslar, ellipstiń kanonikalıq teńlemesi, ekscentrisiteti, direktrisaları).\\

T2. Ekinshi tártipli aylanba betlikler (koordinata sisteması, tegislik, vektor iymek sızıq, aylanba betlik).\\

A1. Fokusları abscissa kósherinde hám koordinata basına qarata simmetriyalıq jaylasqan ellipstiń teńlemesin dúziń: fokusları arasındaǵı aralıq $2 c=6$ hám ekscentrisitet $\varepsilon=3/5$.\\

A2. Fokusları abscissa kósherinde hám koordinata basına qarata simmetriyalıq jaylasqan giperbolanıń teńlemesin dúziń: fokusları arasındaǵı aralıǵı $2 c=10$ hám kósheri $2 b=8$.\\

A3. Fokusları abscissa kósherinde hám koordinata basına qarata simmetriyalıq jaylasqan ellipstiń teńlemesin dúziń: úlken kósheri $10$, fokusları arasındaǵı aralıq $2 c=8$.\\

B1. $x^{2} + 4y^{2} = 25$ ellipsi menen $4x - 2y + 23 = 0$ tuwrı sızıǵına parallel bolǵan urınba tuwrı sızıqtıń teńlemesin dúziń.  \\

B2. ETİS-tıń ulıwma teńlemesin koordinata sistemasın túrlendirmey ápiwaylastırıń, tipin anıqlań, obrazı qanday sızıqtı anıqlaytuǵının kórsetiń. $7x^{2} - 8xy + y^{2} - 16x - 2y - 51 = 0$  \\

B3. $\rho = \frac{5}{3 - 4cos\theta}$ teńlemesi menen qanday sızıq berilgenin hám yarım kósherlerin tabıń.  \\

C1. $4x^{2} - 4xy + y^{2} - 6x + 8y + 13 = 0$ ETİS-ǵı orayǵa iyeme? Orayǵa iye bolsa orayın anıqlań: jalǵız orayǵa iyeme-?, sheksiz orayǵa iyeme-?  \\

C2. Tóbesi $A(-4;0)$ noqatında, al, direktrisası $y - 2 = 0$ tuwrı sızıq bolǵan parabolanıń teńlemesin dúziń.\\

C3. $16x^{2} - 9y^{2} - 64x - 54y - 161 = 0$ teńlemesi giperbolanıń teńlemesi ekenin anıqlań hám onıń orayı $C$, yarım kósherleri, ekscentrisitetin, asimptotalarınıń teńlemelerin dúziń.  \\

\end{tabular}
\vspace{1cm}


\begin{tabular}{m{17cm}}
\textbf{40-variant}
\newline

T1. Parabolanıń polyar koordinatalardaǵı teńlemesi (polyar koordinata sistemasında parabolanıń teńlemesi).\\

T2. ETIS-tıń ulıwma teńlemesin klassifikatsiyalaw (ETIS-tıń ulıwma teńlemesi, ETIS-tıń ulıwma teńlemesin ápiwaylastırıw, klassifikatsiyalaw).\\

A1. Fokusları abscissa kósherinde hám koordinata basına qarata simmetriyalıq jaylasqan ellipstiń teńlemesin dúziń: kishi kósheri $24$, fokusları arasındaǵı aralıq $2 c=10$.\\

A2. Fokusları abscissa kósherinde hám koordinata basına qarata simmetriyalıq jaylasqan giperbolanıń teńlemesin dúziń: fokusları arasındaǵı aralıq $2 c=6$ hám ekscentrisitet $\varepsilon=3/2$.\\

A3. Fokusları abscissa kósherinde hám koordinata basına qarata simmetriyalıq jaylasqan giperbolanıń teńlemesin dúziń: direktrisaları arasındaǵı aralıq $8/3$ hám ekscentrisitet $\varepsilon=3/2$.\\

B1. $\frac{x^{2}}{20} - \frac{y^{2}}{5} = 1$ giperbolasına $4x + 3y - 7 = 0$ tuwrısına perpendikulyar bolǵan urınbanıń teńlemesin dúziń.  \\

B2. $41x^{2} + 24xy + 9y^{2} + 24x + 18y - 36 = 0$ ETİS tipin anıqlań hám orayların tabıń koordinata kósherlerin túrlendirmey qanday sızıqtı anıqlaytuǵının kórsetiń yarım kósherlerin tabıń.  \\

B3. $x^{2} - y^{2} = 27$ giperbolasına $4x + 2y - 7 = 0$ tuwrısına parallel bolǵan urınbanıń teńlemesin tabıń.  \\

C1. Fokuslari $F(3;4), F(-3;-4)$ noqatlarında jaylasqan direktrisaları orasıdaǵı aralıq 3,6 ǵa teń bolǵan giperbolanıń teńlemesin dúziń.  \\

C2. $14x^{2} + 24xy + 21y^{2} - 4x + 18y - 139 = 0$ iymek sızıǵınıń tipin anıqlań, eger oraylı iymek sızıq bolsa orayınıń koordinataların tabıń.  \\

C3. $4x^{2} + 24xy + 11y^{2} + 64x + 42y + 51 = 0$ iymek sızıǵınıń tipin anıqlań eger orayı bar bolsa, onıń orayınıń koordinataların tabıń hám koordinata basın orayǵa parallel kóshiriw ámelin orınlań.  \\

\end{tabular}
\vspace{1cm}


\begin{tabular}{m{17cm}}
\textbf{41-variant}
\newline

T1. Eki gewekli giperboloid. Kanonikalıq teńlemesi (giperbolanı simmetriya kósheri átirapında aylandırıwdan alınǵan betlik).\\

T2. Ellipstiń urınbasınıń teńlemesi (ellips, tuwrı, urınıw tochka, urınba teńlemesi).\\

A1. Sheńberdiń $C$ orayı hám $R$ radiusın tabıń: $x^2+y^2+4 x-2 y+5=0$.\\

A2. Fokusları abscissa kósherinde hám koordinata basına qarata simmetriyalıq jaylasqan giperbolanıń teńlemesin dúziń: úlken kósheri $2 a=16$ hám ekscentrisitet $\varepsilon=5/4$.\\

A3. Polyar teńlemesi menen berilgen iymek sızıqtıń tipin anıqlań: $\rho=\frac{6}{1-\cos 0}$.\\

B1. Koordinata kósherlerin túrlendirmey ETİS teńlemesin ápiwaylastırıń, qanday geometriyalıq obrazdı anıqlaytuǵının kórsetiń $4x^{2} - 4xy + y^{2} + 4x - 2y + 1 = 0$.  \\

B2. $\frac{x^{2}}{4} - \frac{y^{2}}{5} = 1$ giperbolaǵa $3x - 2y = 0$ tuwrısına parallel bolǵan urınbanıń teńlemesin dúziń.  \\

B3. $\frac{x^{2}}{16} - \frac{y^{2}}{64} = 1$, giperbolasına berilgen $10x - 3y + 9 = 0$ tuwrı sızıǵına parallel bolǵan urınbanıń teńlemesin dúziń.  \\

C1. Úlken kósheri 26 ǵa, fokusları $F( - 10;0)$, $F(14;0)$ noqatlarında jaylasqan ellipstiń teńlemesin dúziń.  \\

C2. $32x^{2} + 52xy - 7y^{2} + 180 = 0$ ETİS teńlemesin ápiwayı túrge alıp keliń, tipin anıqlań, qanday geometriyalıq obrazdı anıqlaytuǵının kórsetiń, sızılmasın góne hám taza koordinatalar sistemasına qarata jasań.  \\

C3. $\frac{x^{2}}{3} - \frac{y^{2}}{5} = 1$ giperbolasına $P(1; - 5)$ noqatında júrgizilgen urınbalardıń teńlemesin dúziń.\\

\end{tabular}
\vspace{1cm}


\begin{tabular}{m{17cm}}
\textbf{42-variant}
\newline

T1. ETIS -tiń ulıwma teńlemesin ápiwaylastırıw (ETIS -tiń ulıwma teńlemesi, koordinata sistemasın túrlendirip ETIS ulıwma teńlemesin ápiwaylastırıw).\\

T2. Ellipslik paraboloid (parabola, kósher, ellipslik paraboloid).\\

A1. Berilgen sızıqlardıń oraylıq ekenligin kórsetiń hám orayın tabıń: $5 x^{2}+4 xy+2 y^{2}+20 x+20 y-18=0$.\\

A2. Sheńber teńlemesin dúziń: $A (1;1) $, $B (1;-1) $ hám $C (2;0) $ noqatlardan ótedi.\\

A3. Uchı koordinata basında jaylasqan hám $Ox$ kósherine qarata joqarı yarım tegislikte jaylasqan parabolanıń teńlemesin dúziń: parametri $p=1/4$.\\

B1. $2x + 2y - 3 = 0$ tuwrısına parallel bolıp $\frac{x^{2}}{16} + \frac{y^{2}}{64} = 1$ giperbolasına urınıwshı tuwrınıń teńlemesin dúziń.  \\

B2. Ellips $3x^{2} + 4y^{2} - 12 = 0$ teńlemesi menen berilgen. Onıń kósherleriniń uzınlıqların, fokuslarınıń koordinataların hám ekscentrisitetin tabıń.  \\

B3. $2x + 2y - 3 = 0$ tuwrısına perpendikulyar bolıp $x^{2} = 16y$ parabolasına urınıwshı tuwrınıń teńlemesin dúziń.  \\

C1. $\frac{x^{2}}{100} + \frac{y^{2}}{36} = 1$ ellipsiniń oń jaqtaǵı fokusınan 14 ge teń aralıqta bolǵan noqattı tabıń.  \\

C2. Fokusı $F(2; - 1)$ noqatında jaylasqan, sáykes direktrisası $x - y - 1 = 0$ teńlemesi menen berilgen parabolanıń teńlemesin dúziń.  \\

C3. $2x^{2} + 10xy + 12y^{2} - 7x + 18y - 15 = 0$ ETİS teńlemesin ápiwayı túrge alıp keliń, tipin anıqlań, qanday geometriyalıq obrazdı anıqlaytuǵının kórsetiń, sızılmasın góne hám taza koordinatalar sistemasına qarata jasań  \\

\end{tabular}
\vspace{1cm}


\begin{tabular}{m{17cm}}
\textbf{43-variant}
\newline

T1. Koordinata sistemasın túrlendiriw (birlik vektorlar, kósherler, parallel kóshiriw, koordinata kósherlerin burıw).\\

T2. ETIS-tıń invariantları (ETIS-tıń ulıwma teńlemesi, túrlendiriw, ETIS invariantları ).\\

A1. Polyar teńlemesi menen berilgen iymek sızıqtıń tipin anıqlań: $\rho=\frac{5}{1-\frac{1}{2}\cos\theta}$.\\

A2. Tipin anıqlań: $9 x^{2}+4 y^{2}+18 x-8 y+49=0$.\\

A3. Sheńberdiń $C$ orayı hám $R$ radiusın tabıń: $x^2+y^2-2 x+4 y-14=0$.\\

B1. Koordinata kósherlerin túrlendirmey ETİS teńlemesin ápiwaylastırıń, yarım kósherlerin tabıń $41x^{2} + 2xy + 9y^{2} - 26x - 18y + 3 = 0$.  \\

B2. $x^{2} - 4y^{2} = 16$ giperbola berilgen. Onıń ekscentrisitetin, fokuslarınıń koordinataların tabıń hám asimptotalarınıń teńlemelerin dúziń.\\

B3. $3x + 10y - 25 = 0$ tuwrı menen $\frac{x^{2}}{25} + \frac{y^{2}}{4} = 1$ ellipstiń kesilisiw noqatların tabıń.\\

C1. $\frac{x^{2}}{2} + \frac{y^{2}}{3} = 1$, ellipsin $x + y - 2 = 0$ noqatınan júrgizilgen urınbalarınıń teńlemesin dúziń.  \\

C2. $y^{2} = 20x$ parabolasınıń $M$ noqatın tabıń, eger onıń abscissası 7 ge teń bolsa, fokal radiusın hám fokal radius jaylasqan tuwrını anıqlań.\\

C3. Fokusı $F(7;2)$ noqatında jaylasqan, sáykes direktrisası $x - 5 = 0$ teńlemesi menen berilgen parabolanıń teńlemesin dúziń.  \\

\end{tabular}
\vspace{1cm}


\begin{tabular}{m{17cm}}
\textbf{44-variant}
\newline

T1. Betlik haqqında túsinik (tuwrı, iymek sızıq, betliktiń anıqlamaları hám formulaları).\\

T2. Parabola hám onıń kanonikalıq teńlemesi (anıqlaması, fokusı, direktrisası, kanonikalıq teńlemesi).\\

A1. Fokusları abscissa kósherinde hám koordinata basına qarata simmetriyalıq jaylasqan ellipstiń teńlemesin dúziń: direktrisaları arasındaǵı aralıq $5$ hám fokusları arasındaǵı aralıq $2 c=4$.\\

A2. Ellips teńlemesi berilgen: $\frac{x^2}{25}+\frac{y^2}{16}=1$. Onıń polyar teńlemesin dúziń.\\

A3. Berilgen sızıqlardıń oraylıq ekenligin kórsetiń hám orayın tabıń: $9 x^{2}-4 xy-7 y^{2}-12=0$.\\

B1. $\rho = \frac{6}{1 - cos\theta}$ polyar teńlemesi menen qanday sızıq berilgenin anıqlań.  \\

B2. $y^{2} = 12x$ paraborolasına $3x - 2y + 30 = 0$ tuwrı sızıǵına parallel bolǵan urınbanıń teńlemesin dúziń.  \\

B3. Koordinata kósherlerin túrlendirmey ETİS ulıwma teńlemesin ápiwaylastırıń, yarım kósherlerin tabıń: $13x^{2} + 18xy + 37y^{2} - 26x - 18y + 3 = 0$.  \\

C1. $2x^{2} + 3y^{2} + 8x - 6y + 11 = 0$ teńlemesi menen qanday tiptegi sızıq berilgenin anıqlań hám onıń teńlemesin ápiwaylastırıń hám grafigin jasań.  \\

C2. $\frac{x^{2}}{25} + \frac{y^{2}}{16} = 1$, ellipsine $C(10; - 8)$ noqatınan júrgizilgen urınbalarınıń teńlemesin dúziń.  \\

C3. $y^{2} = 20x$ parabolasınıń abscissası 7 ge teń bolǵan $M$ noqatınıń fokal radiusın tabıń hám fokal radiusı jatqan tuwrınıń teńlemesin dúziń.  \\

\end{tabular}
\vspace{1cm}


\begin{tabular}{m{17cm}}
\textbf{45-variant}
\newline

T1. ETIS-tıń orayın anıqlaw forması (ETIS-tıń ulıwma teńlemesi, orayın anıqlaw forması).\\

T2. Ellipsoida. Kanonikalıq teńlemesi (ellipsti simmetriya kósheri dogereginde aylandırıwdan alınǵan betlik, kanonikalıq teńlemesi).\\

A1. Sheńber teńlemesin dúziń: orayı $C (6 ;-8) $ noqatında jaylasqan hám koordinata basınan ótedi.\\

A2. Fokusları abscissa kósherinde hám koordinata basına qarata simmetriyalıq jaylasqan giperbolanıń teńlemesin dúziń: asimptotalar teńlemeleri $y=\pm \frac{4}{3}x$ hám fokusları arasındaǵı aralıq $2 c=20$.\\

A3. Polyar teńlemesi menen berilgen iymek sızıqtıń tipin anıqlań: $\rho=\frac{12}{2-\cos\theta}$.\\

B1. Ellips $3x^{2} + 4y^{2} - 12 = 0$ teńlemesi menen berilgen. Onıń kósherleriniń uzınlıqların, fokuslarınıń koordinataların hám ekscentrisitetin tabıń.  \\

B2. $y^{2} = 3x$ parabolası menen $\frac{x^{2}}{100} + \frac{y^{2}}{225} = 1$ ellipsiniń kesilisiw noqatların tabıń.  \\

B3. $\rho = \frac{144}{13 - 5cos\theta}$ ellipsti anıqlaytuǵının kórsetiń hám onıń yarım kósherlerin anıqlań.\\

C1. Eger waqıttıń qálegen momentinde $M(x;y)$ noqat $5x - 16 = 0$ tuwrı sızıqqa qaraǵanda $A(5;0)$ noqattan 1,25 márte uzaqlıqta jaylasqan. Usı $M(x;y)$ noqattıń háreketiniń teńlemesin dúziń.  \\

C2. $4x^{2} - 4xy + y^{2} - 2x - 14y + 7 = 0$ ETİS teńlemesin ápiwayı túrge alıp keliń, tipin anıqlań, qanday geometriyalıq obrazdı anıqlaytuǵının kórsetiń, sızılmasın góne hám taza koordinatalar sistemasına qarata jasań.  \\

C3. $\frac{x^{2}}{3} - \frac{y^{2}}{5} = 1$, giperbolasına $P(4;2)$ noqatınan júrgizilgen urınbalardıń teńlemesin dúziń.  \\

\end{tabular}
\vspace{1cm}


\begin{tabular}{m{17cm}}
\textbf{46-variant}
\newline

T1. Giperbola. Kanonikalıq teńlemesi (fokuslar, kósherler, direktrisalar, giperbola, ekscentrisitet, kanonikalıq teńlemesi).\\

T2. ETIS-tıń ulıwma teńlemesin koordinata basın parallel kóshiriw arqalı ápiwayılastırıń (ETIS- tıń ulıwma teńlemesin parallel kóshiriw formulası).\\

A1. Tipin anıqlań: $3 x^{2}-8 xy+7 y^{2}+8 x-15 y+20=0$.\\

A2. Sheńber teńlemesin dúziń: $M_1 (-1;5) $, $M_2 (-2;-2) $ i $M_3 (5;5) $ noqatlardan ótedi.\\

A3. Fokusları abscissa kósherinde hám koordinata basına qarata simmetriyalıq jaylasqan ellipstiń teńlemesin dúziń: kishi kósheri $6$, direktrisaları arasındaǵı aralıq $13$.\\

B1. $\frac{x^{2}}{4} - \frac{y^{2}}{5} = 1$, giperbolanıń $3x - 2y = 0$ tuwrı sızıǵına parallel bolǵan urınbasınıń teńlemesin dúziń.  \\

B2. Koordinata kósherlerin túrlendirmey ETİS teńlemesin ápiwaylastırıń, yarım kósherlerin tabıń $4x^{2} - 4xy + 9y^{2} - 26x - 18y + 3 = 0$.\\

B3. $3x + 4y - 12 = 0$ tuwrı sızıǵı hám $y^{2} = - 9x$ parabolasınıń kesilisiw noqatların tabıń.  \\

C1. $M(2; - \frac{5}{3})$ noqatı $\frac{x^{2}}{9} + \frac{y^{2}}{5} = 1$ ellipsinde jaylasqan. $M$ noqatınıń fokal radiusları jatıwshı tuwrı sızıq teńlemelerin dúziń.  \\

C2. Fokusı $F( - 1; - 4)$ noqatında jaylasqan, sáykes direktrisası $x - 2 = 0$ teńlemesi menen berilgen, $A( - 3; - 5)$ noqatınan ótiwshi ellipstiń teńlemesin dúziń.  \\

C3. $32x^{2} + 52xy - 9y^{2} + 180 = 0$ ETİS teńlemesin ápiwaylastırıń, tipin anıqlań, qanday geometriyalıq obrazdı anıqlaytuǵının kórsetiń, sızılmasın sızıń.  \\

\end{tabular}
\vspace{1cm}


\begin{tabular}{m{17cm}}
\textbf{47-variant}
\newline

T1. Giperbolalıq paraboloydtıń tuwrı sızıqlı jasawshıları (Giperbolalıq paraboloydtı jasawshı tuwrı sızıqlar dástesi).\\

T2. Ellipstiń polyar koordinatalardaǵı teńlemesi (polyar koordinatalar sistemasında ellipstiń teńlemesi).\\

A1. Giperbola teńlemesi berilgen: $\frac{x^{2}}{25}-\frac{y^{2}}{144}=1$. Onıń polyar teńlemesin dúziń.\\

A2. Tipin anıqlań: $x^{2}-4 xy+4 y^{2}+7 x-12=0$.\\

A3. Sheńberdiń $C$ orayı hám $R$ radiusın tabıń: $x^2+y^2+6 x-4 y+14=0$.\\

B1. $\rho = \frac{10}{2 - cos\theta}$ polyar teńlemesi menen qanday sızıq berilgenin anıqlań.  \\

B2. $\frac{x^{2}}{4} - \frac{y^{2}}{5} = 1$ giperbolasına $3x + 2y = 0$ tuwrı sızıǵına perpendikulyar bolǵan urınba tuwrınıń teńlemesin dúziń.\\

B3. ETİS-tıń ulıwma teńlemesin koordinata sistemasın túrlendirmey ápiwaylastırıń, tipin anıqlań, obrazı qanday sızıqtı anıqlaytuǵının kórsetiń. $7x^{2} - 8xy + y^{2} - 16x - 2y - 51 = 0$  \\

C1. $A(\frac{10}{3};\frac{5}{3})$ noqattan $\frac{x^{2}}{20} + \frac{y^{2}}{5} = 1$ ellipsine júrgizilgen urınbalardıń teńlemesin dúziń.  \\

C2. Fokusı $F( - 1; - 4)$noqatında bolǵan, sáykes direktrissası $x - 2 = 0$ teńlemesi menen berilgen $A( - 3; - 5)$ noqatınan ótiwshi ellipstiń teńlemesin dúziń.  \\

C3. $2x^{2} + 3y^{2} + 8x - 6y + 11 = 0$ teńlemesin ápiwaylastırıń qanday geometriyalıq obrazdı anıqlaytuǵının tabıń hám grafigin jasań.  \\

\end{tabular}
\vspace{1cm}


\begin{tabular}{m{17cm}}
\textbf{48-variant}
\newline

T1. ETIS-tıń ulıwma teńlemesin koordinata kósherlerin burıw arqalı ápiwaylastırıń (ETIS-tıń ulıwma teńlemeleri, koordinata kósherin burıw formulası, teńlemeni kanonik túrge alıp keliw).\\

T2. Cilindrlik betlikler (jasawshı tuwrı sızıq, baǵıtlawshı iymek sızıq, cilindrlik betlik).\\

A1. Uchı koordinata basında jaylasqan hám $Oy$ kósherine qarata shep táreptegi yarım tegislikte jaylasqan parabolanıń teńlemesin dúziń: parametri $p=0,5$.\\

A2. Parabola teńlemesi berilgen: $y^2=6 x$. Onıń polyar teńlemesin dúziń.\\

A3. Berilgen sızıqlardıń oraylıq ekenligin kórsetiń hám orayın tabıń: $2 x^{2}-6 xy+5 y^{2}+22 x-36 y+11=0$.\\

B1. $\rho = \frac{5}{3 - 4cos\theta}$ teńlemesi menen qanday sızıq berilgenin hám yarım kósherlerin tabıń.  \\

B2. $x^{2} + 4y^{2} = 25$ ellipsi menen $4x - 2y + 23 = 0$ tuwrı sızıǵına parallel bolǵan urınba tuwrı sızıqtıń teńlemesin dúziń.  \\

B3. $41x^{2} + 24xy + 9y^{2} + 24x + 18y - 36 = 0$ ETİS tipin anıqlań hám orayların tabıń koordinata kósherlerin túrlendirmey qanday sızıqtı anıqlaytuǵının kórsetiń yarım kósherlerin tabıń.  \\

C1. Eger qálegen waqıt momentinde $M(x;y)$ noqat $A(8;4)$ noqattan hám ordinata kósherinen birdey aralıqta jaylassa, $M(x;y)$ noqatınıń háreket etiw troektoriyasınıń teńlemesin dúziń.  \\

C2. $2x^{2} + 3y^{2} + 8x - 6y + 11 = 0$ teńlemesin ápiwaylastırıń qanday geometriyalıq obrazdı anıqlaytuǵının tabıń hám grafigin jasań.\\

C3. Giperbolanıń ekscentrisiteti $\varepsilon = \frac{13}{12}$, fokusı $F(0;13)$ noqatında hám sáykes direktrisası $13y - 144 = 0$ teńlemesi menen berilgen bolsa, giperbolanıń teńlemesin dúziń.  \\

\end{tabular}
\vspace{1cm}


\begin{tabular}{m{17cm}}
\textbf{49-variant}
\newline

T1. Parabolanıń urınbasınıń teńlemesi (parabola, tuwrı, urınıw noqatı, urınba teńlemesi).\\

T2. Bir gewekli giperboloid. Kanonikalıq teńlemesi (giperbolanı simmetriya kósheri átirapında aylandırıwdan alınǵan betlik).\\

A1. Sheńber teńlemesin dúziń: orayı $C (2;-3) $ noqatında jaylasqan hám radiusı $R=7$ ge teń.\\

A2. Fokusları abscissa kósherinde hám koordinata basına qarata simmetriyalıq jaylasqan ellipstiń teńlemesin dúziń: kishi kósheri $10$, ekscentrisitet $\varepsilon=12/13$.\\

A3. Giperbola teńlemesi berilgen: $\frac{x^{2}}{16}-\frac{y^{2}}{9}=1$. Onıń polyar teńlemesin dúziń.\\

B1. $\frac{x^{2}}{20} - \frac{y^{2}}{5} = 1$ giperbolasına $4x + 3y - 7 = 0$ tuwrısına perpendikulyar bolǵan urınbanıń teńlemesin dúziń.  \\

B2. Koordinata kósherlerin túrlendirmey ETİS teńlemesin ápiwaylastırıń, qanday geometriyalıq obrazdı anıqlaytuǵının kórsetiń $4x^{2} - 4xy + y^{2} + 4x - 2y + 1 = 0$.  \\

B3. $x^{2} - y^{2} = 27$ giperbolasına $4x + 2y - 7 = 0$ tuwrısına parallel bolǵan urınbanıń teńlemesin tabıń.  \\

C1. $4x^{2} - 4xy + y^{2} - 6x + 8y + 13 = 0$ ETİS-ǵı orayǵa iyeme? Orayǵa iye bolsa orayın anıqlań: jalǵız orayǵa iyeme-?, sheksiz orayǵa iyeme-?  \\

C2. Tóbesi $A(-4;0)$ noqatında, al, direktrisası $y - 2 = 0$ tuwrı sızıq bolǵan parabolanıń teńlemesin dúziń.\\

C3. $16x^{2} - 9y^{2} - 64x - 54y - 161 = 0$ teńlemesi giperbolanıń teńlemesi ekenin anıqlań hám onıń orayı $C$, yarım kósherleri, ekscentrisitetin, asimptotalarınıń teńlemelerin dúziń.  \\

\end{tabular}
\vspace{1cm}


\begin{tabular}{m{17cm}}
\textbf{50-variant}
\newline

T1. Giperbolanıń polyar koordinatadaǵı teńlemesi (Polyar múyeshi, polyar radiusi giperbolanıń polyar teńlemesi).\\

T2. Betliktiń kanonikalıq teńlemeleri. Betlik haqqında túsinik. (Betliktiń anıqlaması, formulaları, kósher, baǵıtlawshı tuwrılar).\\

A1. Tipin anıqlań: $25 x^{2}-20 xy+4 y^{2}-12 x+20 y-17=0$.\\

A2. Sheńber teńlemesin dúziń: sheńber $A (2;6 ) $ noqatınan ótedi hám orayı $C (-1;2) $ noqatında jaylasqan .\\

A3. Fokusları abscissa kósherinde hám koordinata basına qarata simmetriyalıq jaylasqan giperbolanıń teńlemesin dúziń: oqları $2 a=10$ hám $2 b=8$.\\

B1. $\frac{x^{2}}{4} - \frac{y^{2}}{5} = 1$ giperbolaǵa $3x - 2y = 0$ tuwrısına parallel bolǵan urınbanıń teńlemesin dúziń.  \\

B2. $\frac{x^{2}}{16} - \frac{y^{2}}{64} = 1$, giperbolasına berilgen $10x - 3y + 9 = 0$ tuwrı sızıǵına parallel bolǵan urınbanıń teńlemesin dúziń.  \\

B3. $x^{2} - 4y^{2} = 16$ giperbola berilgen. Onıń ekscentrisitetin, fokuslarınıń koordinataların tabıń hám asimptotalarınıń teńlemelerin dúziń.\\

C1. Fokuslari $F(3;4), F(-3;-4)$ noqatlarında jaylasqan direktrisaları orasıdaǵı aralıq 3,6 ǵa teń bolǵan giperbolanıń teńlemesin dúziń.  \\

C2. $14x^{2} + 24xy + 21y^{2} - 4x + 18y - 139 = 0$ iymek sızıǵınıń tipin anıqlań, eger oraylı iymek sızıq bolsa orayınıń koordinataların tabıń.  \\

C3. $4x^{2} + 24xy + 11y^{2} + 64x + 42y + 51 = 0$ iymek sızıǵınıń tipin anıqlań eger orayı bar bolsa, onıń orayınıń koordinataların tabıń hám koordinata basın orayǵa parallel kóshiriw ámelin orınlań.  \\

\end{tabular}
\vspace{1cm}


\begin{tabular}{m{17cm}}
\textbf{51-variant}
\newline

T1. Giperbolanıń urınbasınıń teńlemesi (giperbolaǵa berilgen noqatta júrgizilgen urınba teńlemesi).\\

T2. Ekinshi tártipli betliktiń ulıwma teńlemesi. Orayın anıqlaw formulası.\\

A1. Polyar teńlemesi menen berilgen iymek sızıqtıń tipin anıqlań: $\rho=\frac{5}{3-4\cos\theta}$.\\

A2. Tipin anıqlań: $3 x^{2}-2 xy-3 y^{2}+12 y-15=0$.\\

A3. Sheńberdiń $C$ orayı hám $R$ radiusın tabıń: $x^2+y^2-2 x+4 y-20=0$.\\

B1. $2x + 2y - 3 = 0$ tuwrısına parallel bolıp $\frac{x^{2}}{16} + \frac{y^{2}}{64} = 1$ giperbolasına urınıwshı tuwrınıń teńlemesin dúziń.  \\

B2. Koordinata kósherlerin túrlendirmey ETİS teńlemesin ápiwaylastırıń, yarım kósherlerin tabıń $41x^{2} + 2xy + 9y^{2} - 26x - 18y + 3 = 0$.  \\

B3. Ellips $3x^{2} + 4y^{2} - 12 = 0$ teńlemesi menen berilgen. Onıń kósherleriniń uzınlıqların, fokuslarınıń koordinataların hám ekscentrisitetin tabıń.  \\

C1. Úlken kósheri 26 ǵa, fokusları $F( - 10;0)$, $F(14;0)$ noqatlarında jaylasqan ellipstiń teńlemesin dúziń.  \\

C2. $32x^{2} + 52xy - 7y^{2} + 180 = 0$ ETİS teńlemesin ápiwayı túrge alıp keliń, tipin anıqlań, qanday geometriyalıq obrazdı anıqlaytuǵının kórsetiń, sızılmasın góne hám taza koordinatalar sistemasına qarata jasań.  \\

C3. $\frac{x^{2}}{3} - \frac{y^{2}}{5} = 1$ giperbolasına $P(1; - 5)$ noqatında júrgizilgen urınbalardıń teńlemesin dúziń.\\

\end{tabular}
\vspace{1cm}


\begin{tabular}{m{17cm}}
\textbf{52-variant}
\newline

T1. Ellips hám onıń kanonikalıq teńlemesi (anıqlaması, fokuslar, ellipstiń kanonikalıq teńlemesi, ekscentrisiteti, direktrisaları).\\

T2. Ekinshi tártipli aylanba betlikler (koordinata sisteması, tegislik, vektor iymek sızıq, aylanba betlik).\\

A1. Fokusları abscissa kósherinde hám koordinata basına qarata simmetriyalıq jaylasqan ellipstiń teńlemesin dúziń: yarım oqları 5 hám 2.\\

A2. Polyar teńlemesi menen berilgen iymek sızıqtıń tipin anıqlań: $\rho=\frac{1}{3-3\cos\theta}$.\\

A3. Tipin anıqlań: $2 x^{2}+3 y^{2}+8 x-6 y+11=0$.\\

B1. $3x + 10y - 25 = 0$ tuwrı menen $\frac{x^{2}}{25} + \frac{y^{2}}{4} = 1$ ellipstiń kesilisiw noqatların tabıń.\\

B2. $\rho = \frac{6}{1 - cos\theta}$ polyar teńlemesi menen qanday sızıq berilgenin anıqlań.  \\

B3. $2x + 2y - 3 = 0$ tuwrısına perpendikulyar bolıp $x^{2} = 16y$ parabolasına urınıwshı tuwrınıń teńlemesin dúziń.  \\

C1. $\frac{x^{2}}{100} + \frac{y^{2}}{36} = 1$ ellipsiniń oń jaqtaǵı fokusınan 14 ge teń aralıqta bolǵan noqattı tabıń.  \\

C2. Fokusı $F(2; - 1)$ noqatında jaylasqan, sáykes direktrisası $x - y - 1 = 0$ teńlemesi menen berilgen parabolanıń teńlemesin dúziń.  \\

C3. $2x^{2} + 10xy + 12y^{2} - 7x + 18y - 15 = 0$ ETİS teńlemesin ápiwayı túrge alıp keliń, tipin anıqlań, qanday geometriyalıq obrazdı anıqlaytuǵının kórsetiń, sızılmasın góne hám taza koordinatalar sistemasına qarata jasań  \\

\end{tabular}
\vspace{1cm}


\begin{tabular}{m{17cm}}
\textbf{53-variant}
\newline

T1. Parabolanıń polyar koordinatalardaǵı teńlemesi (polyar koordinata sistemasında parabolanıń teńlemesi).\\

T2. ETIS-tıń ulıwma teńlemesin klassifikatsiyalaw (ETIS-tıń ulıwma teńlemesi, ETIS-tıń ulıwma teńlemesin ápiwaylastırıw, klassifikatsiyalaw).\\

A1. Sheńber teńlemesin dúziń: orayı koordinata basında jaylasqan hám radiusı $R=3$ ge teń.\\

A2. Uchı koordinata basında jaylasqan hám $Oy$ kósherine qarata oń táreptegi yarım tegislikte jaylasqan parabolanıń teńlemesin dúziń: parametri $p=3$.\\

A3. Polyar teńlemesi menen berilgen iymek sızıqtıń tipin anıqlań: $\rho=\frac{10}{1-\frac{3}{2}\cos\theta}$.\\

B1. Koordinata kósherlerin túrlendirmey ETİS ulıwma teńlemesin ápiwaylastırıń, yarım kósherlerin tabıń: $13x^{2} + 18xy + 37y^{2} - 26x - 18y + 3 = 0$.  \\

B2. $x^{2} - 4y^{2} = 16$ giperbola berilgen. Onıń ekscentrisitetin, fokuslarınıń koordinataların tabıń hám asimptotalarınıń teńlemelerin dúziń.\\

B3. $y^{2} = 3x$ parabolası menen $\frac{x^{2}}{100} + \frac{y^{2}}{225} = 1$ ellipsiniń kesilisiw noqatların tabıń.  \\

C1. $\frac{x^{2}}{2} + \frac{y^{2}}{3} = 1$, ellipsin $x + y - 2 = 0$ noqatınan júrgizilgen urınbalarınıń teńlemesin dúziń.  \\

C2. $y^{2} = 20x$ parabolasınıń $M$ noqatın tabıń, eger onıń abscissası 7 ge teń bolsa, fokal radiusın hám fokal radius jaylasqan tuwrını anıqlań.\\

C3. Fokusı $F(7;2)$ noqatında jaylasqan, sáykes direktrisası $x - 5 = 0$ teńlemesi menen berilgen parabolanıń teńlemesin dúziń.  \\

\end{tabular}
\vspace{1cm}


\begin{tabular}{m{17cm}}
\textbf{54-variant}
\newline

T1. Eki gewekli giperboloid. Kanonikalıq teńlemesi (giperbolanı simmetriya kósheri átirapında aylandırıwdan alınǵan betlik).\\

T2. Ellipstiń urınbasınıń teńlemesi (ellips, tuwrı, urınıw tochka, urınba teńlemesi).\\

A1. Berilgen sızıqlardıń oraylıq ekenligin kórsetiń hám orayın tabıń: $3 x^{2}+5 xy+y^{2}-8 x-11 y-7=0$.\\

A2. Sheńber teńlemesin dúziń: orayı $C (1;-1) $ noqatında jaylasqan hám $5 x-12 y+9 -0$ tuwrı sızıǵına urınadı .\\

A3. Fokusları abscissa kósherinde hám koordinata basına qarata simmetriyalıq jaylasqan ellipstiń teńlemesin dúziń: úlken kósheri $8$, direktrisaları arasındaǵı aralıq $16$.\\

B1. $\rho = \frac{144}{13 - 5cos\theta}$ ellipsti anıqlaytuǵının kórsetiń hám onıń yarım kósherlerin anıqlań.\\

B2. $y^{2} = 12x$ paraborolasına $3x - 2y + 30 = 0$ tuwrı sızıǵına parallel bolǵan urınbanıń teńlemesin dúziń.  \\

B3. Koordinata kósherlerin túrlendirmey ETİS teńlemesin ápiwaylastırıń, yarım kósherlerin tabıń $4x^{2} - 4xy + 9y^{2} - 26x - 18y + 3 = 0$.\\

C1. $2x^{2} + 3y^{2} + 8x - 6y + 11 = 0$ teńlemesi menen qanday tiptegi sızıq berilgenin anıqlań hám onıń teńlemesin ápiwaylastırıń hám grafigin jasań.  \\

C2. $\frac{x^{2}}{25} + \frac{y^{2}}{16} = 1$, ellipsine $C(10; - 8)$ noqatınan júrgizilgen urınbalarınıń teńlemesin dúziń.  \\

C3. $y^{2} = 20x$ parabolasınıń abscissası 7 ge teń bolǵan $M$ noqatınıń fokal radiusın tabıń hám fokal radiusı jatqan tuwrınıń teńlemesin dúziń.  \\

\end{tabular}
\vspace{1cm}


\begin{tabular}{m{17cm}}
\textbf{55-variant}
\newline

T1. ETIS -tiń ulıwma teńlemesin ápiwaylastırıw (ETIS -tiń ulıwma teńlemesi, koordinata sistemasın túrlendirip ETIS ulıwma teńlemesin ápiwaylastırıw).\\

T2. Ellipslik paraboloid (parabola, kósher, ellipslik paraboloid).\\

A1. Tipin anıqlań: $2 x^{2}+10 xy+12 y^{2}-7 x+18 y-15=0$.\\

A2. Sheńber teńlemesin dúziń: orayı koordinata basında jaylasqan hám $3 x-4 y+20=0$ tuwrı sızıǵına urınadı.\\

A3. Fokusları abscissa kósherinde hám koordinata basına qarata simmetriyalıq jaylasqan ellipstiń teńlemesin dúziń: úlken kósheri $20$, ekscentrisitet $\varepsilon=3/5$.\\

B1. $3x + 4y - 12 = 0$ tuwrı sızıǵı hám $y^{2} = - 9x$ parabolasınıń kesilisiw noqatların tabıń.  \\

B2. $\rho = \frac{10}{2 - cos\theta}$ polyar teńlemesi menen qanday sızıq berilgenin anıqlań.  \\

B3. $\frac{x^{2}}{4} - \frac{y^{2}}{5} = 1$, giperbolanıń $3x - 2y = 0$ tuwrı sızıǵına parallel bolǵan urınbasınıń teńlemesin dúziń.  \\

C1. Eger waqıttıń qálegen momentinde $M(x;y)$ noqat $5x - 16 = 0$ tuwrı sızıqqa qaraǵanda $A(5;0)$ noqattan 1,25 márte uzaqlıqta jaylasqan. Usı $M(x;y)$ noqattıń háreketiniń teńlemesin dúziń.  \\

C2. $4x^{2} - 4xy + y^{2} - 2x - 14y + 7 = 0$ ETİS teńlemesin ápiwayı túrge alıp keliń, tipin anıqlań, qanday geometriyalıq obrazdı anıqlaytuǵının kórsetiń, sızılmasın góne hám taza koordinatalar sistemasına qarata jasań.  \\

C3. $\frac{x^{2}}{3} - \frac{y^{2}}{5} = 1$, giperbolasına $P(4;2)$ noqatınan júrgizilgen urınbalardıń teńlemesin dúziń.  \\

\end{tabular}
\vspace{1cm}


\begin{tabular}{m{17cm}}
\textbf{56-variant}
\newline

T1. Koordinata sistemasın túrlendiriw (birlik vektorlar, kósherler, parallel kóshiriw, koordinata kósherlerin burıw).\\

T2. ETIS-tıń invariantları (ETIS-tıń ulıwma teńlemesi, túrlendiriw, ETIS invariantları ).\\

A1. Tipin anıqlań: $4 x^{2}-y^{2}+8 x-2 y+3=0$.\\

A2. Sheńber teńlemesin dúziń: sheńber diametriniń ushları $A (3;2) $ hám $B (-1;6 ) $ noqatlarında jaylasqan.\\

A3. Fokusları abscissa kósherinde hám koordinata basına qarata simmetriyalıq jaylasqan giperbolanıń teńlemesin dúziń: direktrisaları arasındaǵı aralıq $228/13$ hám fokusları arasındaǵı aralıq $2 c=26$.\\

B1. ETİS-tıń ulıwma teńlemesin koordinata sistemasın túrlendirmey ápiwaylastırıń, tipin anıqlań, obrazı qanday sızıqtı anıqlaytuǵının kórsetiń. $7x^{2} - 8xy + y^{2} - 16x - 2y - 51 = 0$  \\

B2. $\rho = \frac{5}{3 - 4cos\theta}$ teńlemesi menen qanday sızıq berilgenin hám yarım kósherlerin tabıń.  \\

B3. $\frac{x^{2}}{4} - \frac{y^{2}}{5} = 1$ giperbolasına $3x + 2y = 0$ tuwrı sızıǵına perpendikulyar bolǵan urınba tuwrınıń teńlemesin dúziń.\\

C1. $M(2; - \frac{5}{3})$ noqatı $\frac{x^{2}}{9} + \frac{y^{2}}{5} = 1$ ellipsinde jaylasqan. $M$ noqatınıń fokal radiusları jatıwshı tuwrı sızıq teńlemelerin dúziń.  \\

C2. Fokusı $F( - 1; - 4)$ noqatında jaylasqan, sáykes direktrisası $x - 2 = 0$ teńlemesi menen berilgen, $A( - 3; - 5)$ noqatınan ótiwshi ellipstiń teńlemesin dúziń.  \\

C3. $32x^{2} + 52xy - 9y^{2} + 180 = 0$ ETİS teńlemesin ápiwaylastırıń, tipin anıqlań, qanday geometriyalıq obrazdı anıqlaytuǵının kórsetiń, sızılmasın sızıń.  \\

\end{tabular}
\vspace{1cm}


\begin{tabular}{m{17cm}}
\textbf{57-variant}
\newline

T1. Betlik haqqında túsinik (tuwrı, iymek sızıq, betliktiń anıqlamaları hám formulaları).\\

T2. Parabola hám onıń kanonikalıq teńlemesi (anıqlaması, fokusı, direktrisası, kanonikalıq teńlemesi).\\

A1. Tipin anıqlań: $9 x^{2}-16 y^{2}-54 x-64 y-127=0$.\\

A2. Sheńber teńlemesin dúziń: $A (3;1) $ hám $B (-1;3) $ noqatlardan ótedi, orayı $3 x-y-2=0$ tuwrı sızıǵında jaylasqan .\\

A3. Uchı koordinata basında jaylasqan hám $Ox$ kósherine qarata tómengi yarım tegislikte jaylasqan parabolanıń teńlemesin dúziń: parametri $p=3$.\\

B1. $41x^{2} + 24xy + 9y^{2} + 24x + 18y - 36 = 0$ ETİS tipin anıqlań hám orayların tabıń koordinata kósherlerin túrlendirmey qanday sızıqtı anıqlaytuǵının kórsetiń yarım kósherlerin tabıń.  \\

B2. $x^{2} + 4y^{2} = 25$ ellipsi menen $4x - 2y + 23 = 0$ tuwrı sızıǵına parallel bolǵan urınba tuwrı sızıqtıń teńlemesin dúziń.  \\

B3. Koordinata kósherlerin túrlendirmey ETİS teńlemesin ápiwaylastırıń, qanday geometriyalıq obrazdı anıqlaytuǵının kórsetiń $4x^{2} - 4xy + y^{2} + 4x - 2y + 1 = 0$.  \\

C1. $A(\frac{10}{3};\frac{5}{3})$ noqattan $\frac{x^{2}}{20} + \frac{y^{2}}{5} = 1$ ellipsine júrgizilgen urınbalardıń teńlemesin dúziń.  \\

C2. Fokusı $F( - 1; - 4)$noqatında bolǵan, sáykes direktrissası $x - 2 = 0$ teńlemesi menen berilgen $A( - 3; - 5)$ noqatınan ótiwshi ellipstiń teńlemesin dúziń.  \\

C3. $2x^{2} + 3y^{2} + 8x - 6y + 11 = 0$ teńlemesin ápiwaylastırıń qanday geometriyalıq obrazdı anıqlaytuǵının tabıń hám grafigin jasań.  \\

\end{tabular}
\vspace{1cm}


\begin{tabular}{m{17cm}}
\textbf{58-variant}
\newline

T1. ETIS-tıń orayın anıqlaw forması (ETIS-tıń ulıwma teńlemesi, orayın anıqlaw forması).\\

T2. Ellipsoida. Kanonikalıq teńlemesi (ellipsti simmetriya kósheri dogereginde aylandırıwdan alınǵan betlik, kanonikalıq teńlemesi).\\

A1. Tipin anıqlań: $5 x^{2}+14 xy+11 y^{2}+12 x-7 y+19=0$.\\

A2. Fokusları abscissa kósherinde hám koordinata basına qarata simmetriyalıq jaylasqan giperbolanıń teńlemesin dúziń: direktrisaları arasındaǵı aralıq $32/5$ hám kósheri $2 b=6$.\\

A3. Tipin anıqlań: $4 x^2+9 y^2-40 x+36 y+100=0$.\\

B1. $\frac{x^{2}}{20} - \frac{y^{2}}{5} = 1$ giperbolasına $4x + 3y - 7 = 0$ tuwrısına perpendikulyar bolǵan urınbanıń teńlemesin dúziń.  \\

B2. $x^{2} - y^{2} = 27$ giperbolasına $4x + 2y - 7 = 0$ tuwrısına parallel bolǵan urınbanıń teńlemesin tabıń.  \\

B3. $\frac{x^{2}}{4} - \frac{y^{2}}{5} = 1$ giperbolaǵa $3x - 2y = 0$ tuwrısına parallel bolǵan urınbanıń teńlemesin dúziń.  \\

C1. Eger qálegen waqıt momentinde $M(x;y)$ noqat $A(8;4)$ noqattan hám ordinata kósherinen birdey aralıqta jaylassa, $M(x;y)$ noqatınıń háreket etiw troektoriyasınıń teńlemesin dúziń.  \\

C2. $2x^{2} + 3y^{2} + 8x - 6y + 11 = 0$ teńlemesin ápiwaylastırıń qanday geometriyalıq obrazdı anıqlaytuǵının tabıń hám grafigin jasań.\\

C3. Giperbolanıń ekscentrisiteti $\varepsilon = \frac{13}{12}$, fokusı $F(0;13)$ noqatında hám sáykes direktrisası $13y - 144 = 0$ teńlemesi menen berilgen bolsa, giperbolanıń teńlemesin dúziń.  \\

\end{tabular}
\vspace{1cm}


\begin{tabular}{m{17cm}}
\textbf{59-variant}
\newline

T1. Giperbola. Kanonikalıq teńlemesi (fokuslar, kósherler, direktrisalar, giperbola, ekscentrisitet, kanonikalıq teńlemesi).\\

T2. ETIS-tıń ulıwma teńlemesin koordinata basın parallel kóshiriw arqalı ápiwayılastırıń (ETIS- tıń ulıwma teńlemesin parallel kóshiriw formulası).\\

A1. Fokusları abscissa kósherinde hám koordinata basına qarata simmetriyalıq jaylasqan ellipstiń teńlemesin dúziń: fokusları arasındaǵı aralıq $2 c=6$ hám ekscentrisitet $\varepsilon=3/5$.\\

A2. Fokusları abscissa kósherinde hám koordinata basına qarata simmetriyalıq jaylasqan giperbolanıń teńlemesin dúziń: fokusları arasındaǵı aralıǵı $2 c=10$ hám kósheri $2 b=8$.\\

A3. Fokusları abscissa kósherinde hám koordinata basına qarata simmetriyalıq jaylasqan ellipstiń teńlemesin dúziń: úlken kósheri $10$, fokusları arasındaǵı aralıq $2 c=8$.\\

B1. Ellips $3x^{2} + 4y^{2} - 12 = 0$ teńlemesi menen berilgen. Onıń kósherleriniń uzınlıqların, fokuslarınıń koordinataların hám ekscentrisitetin tabıń.  \\

B2. $\frac{x^{2}}{16} - \frac{y^{2}}{64} = 1$, giperbolasına berilgen $10x - 3y + 9 = 0$ tuwrı sızıǵına parallel bolǵan urınbanıń teńlemesin dúziń.  \\

B3. Koordinata kósherlerin túrlendirmey ETİS teńlemesin ápiwaylastırıń, yarım kósherlerin tabıń $41x^{2} + 2xy + 9y^{2} - 26x - 18y + 3 = 0$.  \\

C1. $4x^{2} - 4xy + y^{2} - 6x + 8y + 13 = 0$ ETİS-ǵı orayǵa iyeme? Orayǵa iye bolsa orayın anıqlań: jalǵız orayǵa iyeme-?, sheksiz orayǵa iyeme-?  \\

C2. Tóbesi $A(-4;0)$ noqatında, al, direktrisası $y - 2 = 0$ tuwrı sızıq bolǵan parabolanıń teńlemesin dúziń.\\

C3. $16x^{2} - 9y^{2} - 64x - 54y - 161 = 0$ teńlemesi giperbolanıń teńlemesi ekenin anıqlań hám onıń orayı $C$, yarım kósherleri, ekscentrisitetin, asimptotalarınıń teńlemelerin dúziń.  \\

\end{tabular}
\vspace{1cm}


\begin{tabular}{m{17cm}}
\textbf{60-variant}
\newline

T1. Giperbolalıq paraboloydtıń tuwrı sızıqlı jasawshıları (Giperbolalıq paraboloydtı jasawshı tuwrı sızıqlar dástesi).\\

T2. Ellipstiń polyar koordinatalardaǵı teńlemesi (polyar koordinatalar sistemasında ellipstiń teńlemesi).\\

A1. Fokusları abscissa kósherinde hám koordinata basına qarata simmetriyalıq jaylasqan ellipstiń teńlemesin dúziń: kishi kósheri $24$, fokusları arasındaǵı aralıq $2 c=10$.\\

A2. Fokusları abscissa kósherinde hám koordinata basına qarata simmetriyalıq jaylasqan giperbolanıń teńlemesin dúziń: fokusları arasındaǵı aralıq $2 c=6$ hám ekscentrisitet $\varepsilon=3/2$.\\

A3. Fokusları abscissa kósherinde hám koordinata basına qarata simmetriyalıq jaylasqan giperbolanıń teńlemesin dúziń: direktrisaları arasındaǵı aralıq $8/3$ hám ekscentrisitet $\varepsilon=3/2$.\\

B1. $x^{2} - 4y^{2} = 16$ giperbola berilgen. Onıń ekscentrisitetin, fokuslarınıń koordinataların tabıń hám asimptotalarınıń teńlemelerin dúziń.\\

B2. $3x + 10y - 25 = 0$ tuwrı menen $\frac{x^{2}}{25} + \frac{y^{2}}{4} = 1$ ellipstiń kesilisiw noqatların tabıń.\\

B3. $\rho = \frac{6}{1 - cos\theta}$ polyar teńlemesi menen qanday sızıq berilgenin anıqlań.  \\

C1. Fokuslari $F(3;4), F(-3;-4)$ noqatlarında jaylasqan direktrisaları orasıdaǵı aralıq 3,6 ǵa teń bolǵan giperbolanıń teńlemesin dúziń.  \\

C2. $14x^{2} + 24xy + 21y^{2} - 4x + 18y - 139 = 0$ iymek sızıǵınıń tipin anıqlań, eger oraylı iymek sızıq bolsa orayınıń koordinataların tabıń.  \\

C3. $4x^{2} + 24xy + 11y^{2} + 64x + 42y + 51 = 0$ iymek sızıǵınıń tipin anıqlań eger orayı bar bolsa, onıń orayınıń koordinataların tabıń hám koordinata basın orayǵa parallel kóshiriw ámelin orınlań.  \\

\end{tabular}
\vspace{1cm}


\begin{tabular}{m{17cm}}
\textbf{61-variant}
\newline

T1. ETIS-tıń ulıwma teńlemesin koordinata kósherlerin burıw arqalı ápiwaylastırıń (ETIS-tıń ulıwma teńlemeleri, koordinata kósherin burıw formulası, teńlemeni kanonik túrge alıp keliw).\\

T2. Cilindrlik betlikler (jasawshı tuwrı sızıq, baǵıtlawshı iymek sızıq, cilindrlik betlik).\\

A1. Sheńberdiń $C$ orayı hám $R$ radiusın tabıń: $x^2+y^2+4 x-2 y+5=0$.\\

A2. Fokusları abscissa kósherinde hám koordinata basına qarata simmetriyalıq jaylasqan giperbolanıń teńlemesin dúziń: úlken kósheri $2 a=16$ hám ekscentrisitet $\varepsilon=5/4$.\\

A3. Polyar teńlemesi menen berilgen iymek sızıqtıń tipin anıqlań: $\rho=\frac{6}{1-\cos 0}$.\\

B1. $2x + 2y - 3 = 0$ tuwrısına parallel bolıp $\frac{x^{2}}{16} + \frac{y^{2}}{64} = 1$ giperbolasına urınıwshı tuwrınıń teńlemesin dúziń.  \\

B2. Koordinata kósherlerin túrlendirmey ETİS ulıwma teńlemesin ápiwaylastırıń, yarım kósherlerin tabıń: $13x^{2} + 18xy + 37y^{2} - 26x - 18y + 3 = 0$.  \\

B3. Ellips $3x^{2} + 4y^{2} - 12 = 0$ teńlemesi menen berilgen. Onıń kósherleriniń uzınlıqların, fokuslarınıń koordinataların hám ekscentrisitetin tabıń.  \\

C1. Úlken kósheri 26 ǵa, fokusları $F( - 10;0)$, $F(14;0)$ noqatlarında jaylasqan ellipstiń teńlemesin dúziń.  \\

C2. $32x^{2} + 52xy - 7y^{2} + 180 = 0$ ETİS teńlemesin ápiwayı túrge alıp keliń, tipin anıqlań, qanday geometriyalıq obrazdı anıqlaytuǵının kórsetiń, sızılmasın góne hám taza koordinatalar sistemasına qarata jasań.  \\

C3. $\frac{x^{2}}{3} - \frac{y^{2}}{5} = 1$ giperbolasına $P(1; - 5)$ noqatında júrgizilgen urınbalardıń teńlemesin dúziń.\\

\end{tabular}
\vspace{1cm}


\begin{tabular}{m{17cm}}
\textbf{62-variant}
\newline

T1. Parabolanıń urınbasınıń teńlemesi (parabola, tuwrı, urınıw noqatı, urınba teńlemesi).\\

T2. Bir gewekli giperboloid. Kanonikalıq teńlemesi (giperbolanı simmetriya kósheri átirapında aylandırıwdan alınǵan betlik).\\

A1. Berilgen sızıqlardıń oraylıq ekenligin kórsetiń hám orayın tabıń: $5 x^{2}+4 xy+2 y^{2}+20 x+20 y-18=0$.\\

A2. Sheńber teńlemesin dúziń: $A (1;1) $, $B (1;-1) $ hám $C (2;0) $ noqatlardan ótedi.\\

A3. Uchı koordinata basında jaylasqan hám $Ox$ kósherine qarata joqarı yarım tegislikte jaylasqan parabolanıń teńlemesin dúziń: parametri $p=1/4$.\\

B1. $y^{2} = 3x$ parabolası menen $\frac{x^{2}}{100} + \frac{y^{2}}{225} = 1$ ellipsiniń kesilisiw noqatların tabıń.  \\

B2. $\rho = \frac{144}{13 - 5cos\theta}$ ellipsti anıqlaytuǵının kórsetiń hám onıń yarım kósherlerin anıqlań.\\

B3. $2x + 2y - 3 = 0$ tuwrısına perpendikulyar bolıp $x^{2} = 16y$ parabolasına urınıwshı tuwrınıń teńlemesin dúziń.  \\

C1. $\frac{x^{2}}{100} + \frac{y^{2}}{36} = 1$ ellipsiniń oń jaqtaǵı fokusınan 14 ge teń aralıqta bolǵan noqattı tabıń.  \\

C2. Fokusı $F(2; - 1)$ noqatında jaylasqan, sáykes direktrisası $x - y - 1 = 0$ teńlemesi menen berilgen parabolanıń teńlemesin dúziń.  \\

C3. $2x^{2} + 10xy + 12y^{2} - 7x + 18y - 15 = 0$ ETİS teńlemesin ápiwayı túrge alıp keliń, tipin anıqlań, qanday geometriyalıq obrazdı anıqlaytuǵının kórsetiń, sızılmasın góne hám taza koordinatalar sistemasına qarata jasań  \\

\end{tabular}
\vspace{1cm}


\begin{tabular}{m{17cm}}
\textbf{63-variant}
\newline

T1. Giperbolanıń polyar koordinatadaǵı teńlemesi (Polyar múyeshi, polyar radiusi giperbolanıń polyar teńlemesi).\\

T2. Betliktiń kanonikalıq teńlemeleri. Betlik haqqında túsinik. (Betliktiń anıqlaması, formulaları, kósher, baǵıtlawshı tuwrılar).\\

A1. Polyar teńlemesi menen berilgen iymek sızıqtıń tipin anıqlań: $\rho=\frac{5}{1-\frac{1}{2}\cos\theta}$.\\

A2. Tipin anıqlań: $9 x^{2}+4 y^{2}+18 x-8 y+49=0$.\\

A3. Sheńberdiń $C$ orayı hám $R$ radiusın tabıń: $x^2+y^2-2 x+4 y-14=0$.\\

B1. Koordinata kósherlerin túrlendirmey ETİS teńlemesin ápiwaylastırıń, yarım kósherlerin tabıń $4x^{2} - 4xy + 9y^{2} - 26x - 18y + 3 = 0$.\\

B2. $3x + 4y - 12 = 0$ tuwrı sızıǵı hám $y^{2} = - 9x$ parabolasınıń kesilisiw noqatların tabıń.  \\

B3. $\rho = \frac{10}{2 - cos\theta}$ polyar teńlemesi menen qanday sızıq berilgenin anıqlań.  \\

C1. $\frac{x^{2}}{2} + \frac{y^{2}}{3} = 1$, ellipsin $x + y - 2 = 0$ noqatınan júrgizilgen urınbalarınıń teńlemesin dúziń.  \\

C2. $y^{2} = 20x$ parabolasınıń $M$ noqatın tabıń, eger onıń abscissası 7 ge teń bolsa, fokal radiusın hám fokal radius jaylasqan tuwrını anıqlań.\\

C3. Fokusı $F(7;2)$ noqatında jaylasqan, sáykes direktrisası $x - 5 = 0$ teńlemesi menen berilgen parabolanıń teńlemesin dúziń.  \\

\end{tabular}
\vspace{1cm}


\begin{tabular}{m{17cm}}
\textbf{64-variant}
\newline

T1. Giperbolanıń urınbasınıń teńlemesi (giperbolaǵa berilgen noqatta júrgizilgen urınba teńlemesi).\\

T2. Ekinshi tártipli betliktiń ulıwma teńlemesi. Orayın anıqlaw formulası.\\

A1. Fokusları abscissa kósherinde hám koordinata basına qarata simmetriyalıq jaylasqan ellipstiń teńlemesin dúziń: direktrisaları arasındaǵı aralıq $5$ hám fokusları arasındaǵı aralıq $2 c=4$.\\

A2. Ellips teńlemesi berilgen: $\frac{x^2}{25}+\frac{y^2}{16}=1$. Onıń polyar teńlemesin dúziń.\\

A3. Berilgen sızıqlardıń oraylıq ekenligin kórsetiń hám orayın tabıń: $9 x^{2}-4 xy-7 y^{2}-12=0$.\\

B1. $y^{2} = 12x$ paraborolasına $3x - 2y + 30 = 0$ tuwrı sızıǵına parallel bolǵan urınbanıń teńlemesin dúziń.  \\

B2. ETİS-tıń ulıwma teńlemesin koordinata sistemasın túrlendirmey ápiwaylastırıń, tipin anıqlań, obrazı qanday sızıqtı anıqlaytuǵının kórsetiń. $7x^{2} - 8xy + y^{2} - 16x - 2y - 51 = 0$  \\

B3. $\rho = \frac{5}{3 - 4cos\theta}$ teńlemesi menen qanday sızıq berilgenin hám yarım kósherlerin tabıń.  \\

C1. $2x^{2} + 3y^{2} + 8x - 6y + 11 = 0$ teńlemesi menen qanday tiptegi sızıq berilgenin anıqlań hám onıń teńlemesin ápiwaylastırıń hám grafigin jasań.  \\

C2. $\frac{x^{2}}{25} + \frac{y^{2}}{16} = 1$, ellipsine $C(10; - 8)$ noqatınan júrgizilgen urınbalarınıń teńlemesin dúziń.  \\

C3. $y^{2} = 20x$ parabolasınıń abscissası 7 ge teń bolǵan $M$ noqatınıń fokal radiusın tabıń hám fokal radiusı jatqan tuwrınıń teńlemesin dúziń.  \\

\end{tabular}
\vspace{1cm}


\begin{tabular}{m{17cm}}
\textbf{65-variant}
\newline

T1. Ellips hám onıń kanonikalıq teńlemesi (anıqlaması, fokuslar, ellipstiń kanonikalıq teńlemesi, ekscentrisiteti, direktrisaları).\\

T2. Ekinshi tártipli aylanba betlikler (koordinata sisteması, tegislik, vektor iymek sızıq, aylanba betlik).\\

A1. Sheńber teńlemesin dúziń: orayı $C (6 ;-8) $ noqatında jaylasqan hám koordinata basınan ótedi.\\

A2. Fokusları abscissa kósherinde hám koordinata basına qarata simmetriyalıq jaylasqan giperbolanıń teńlemesin dúziń: asimptotalar teńlemeleri $y=\pm \frac{4}{3}x$ hám fokusları arasındaǵı aralıq $2 c=20$.\\

A3. Polyar teńlemesi menen berilgen iymek sızıqtıń tipin anıqlań: $\rho=\frac{12}{2-\cos\theta}$.\\

B1. $\frac{x^{2}}{4} - \frac{y^{2}}{5} = 1$, giperbolanıń $3x - 2y = 0$ tuwrı sızıǵına parallel bolǵan urınbasınıń teńlemesin dúziń.  \\

B2. $41x^{2} + 24xy + 9y^{2} + 24x + 18y - 36 = 0$ ETİS tipin anıqlań hám orayların tabıń koordinata kósherlerin túrlendirmey qanday sızıqtı anıqlaytuǵının kórsetiń yarım kósherlerin tabıń.  \\

B3. $\frac{x^{2}}{4} - \frac{y^{2}}{5} = 1$ giperbolasına $3x + 2y = 0$ tuwrı sızıǵına perpendikulyar bolǵan urınba tuwrınıń teńlemesin dúziń.\\

C1. Eger waqıttıń qálegen momentinde $M(x;y)$ noqat $5x - 16 = 0$ tuwrı sızıqqa qaraǵanda $A(5;0)$ noqattan 1,25 márte uzaqlıqta jaylasqan. Usı $M(x;y)$ noqattıń háreketiniń teńlemesin dúziń.  \\

C2. $4x^{2} - 4xy + y^{2} - 2x - 14y + 7 = 0$ ETİS teńlemesin ápiwayı túrge alıp keliń, tipin anıqlań, qanday geometriyalıq obrazdı anıqlaytuǵının kórsetiń, sızılmasın góne hám taza koordinatalar sistemasına qarata jasań.  \\

C3. $\frac{x^{2}}{3} - \frac{y^{2}}{5} = 1$, giperbolasına $P(4;2)$ noqatınan júrgizilgen urınbalardıń teńlemesin dúziń.  \\

\end{tabular}
\vspace{1cm}


\begin{tabular}{m{17cm}}
\textbf{66-variant}
\newline

T1. Parabolanıń polyar koordinatalardaǵı teńlemesi (polyar koordinata sistemasında parabolanıń teńlemesi).\\

T2. ETIS-tıń ulıwma teńlemesin klassifikatsiyalaw (ETIS-tıń ulıwma teńlemesi, ETIS-tıń ulıwma teńlemesin ápiwaylastırıw, klassifikatsiyalaw).\\

A1. Tipin anıqlań: $3 x^{2}-8 xy+7 y^{2}+8 x-15 y+20=0$.\\

A2. Sheńber teńlemesin dúziń: $M_1 (-1;5) $, $M_2 (-2;-2) $ i $M_3 (5;5) $ noqatlardan ótedi.\\

A3. Fokusları abscissa kósherinde hám koordinata basına qarata simmetriyalıq jaylasqan ellipstiń teńlemesin dúziń: kishi kósheri $6$, direktrisaları arasındaǵı aralıq $13$.\\

B1. Koordinata kósherlerin túrlendirmey ETİS teńlemesin ápiwaylastırıń, qanday geometriyalıq obrazdı anıqlaytuǵının kórsetiń $4x^{2} - 4xy + y^{2} + 4x - 2y + 1 = 0$.  \\

B2. $x^{2} + 4y^{2} = 25$ ellipsi menen $4x - 2y + 23 = 0$ tuwrı sızıǵına parallel bolǵan urınba tuwrı sızıqtıń teńlemesin dúziń.  \\

B3. $\frac{x^{2}}{20} - \frac{y^{2}}{5} = 1$ giperbolasına $4x + 3y - 7 = 0$ tuwrısına perpendikulyar bolǵan urınbanıń teńlemesin dúziń.  \\

C1. $M(2; - \frac{5}{3})$ noqatı $\frac{x^{2}}{9} + \frac{y^{2}}{5} = 1$ ellipsinde jaylasqan. $M$ noqatınıń fokal radiusları jatıwshı tuwrı sızıq teńlemelerin dúziń.  \\

C2. Fokusı $F( - 1; - 4)$ noqatında jaylasqan, sáykes direktrisası $x - 2 = 0$ teńlemesi menen berilgen, $A( - 3; - 5)$ noqatınan ótiwshi ellipstiń teńlemesin dúziń.  \\

C3. $32x^{2} + 52xy - 9y^{2} + 180 = 0$ ETİS teńlemesin ápiwaylastırıń, tipin anıqlań, qanday geometriyalıq obrazdı anıqlaytuǵının kórsetiń, sızılmasın sızıń.  \\

\end{tabular}
\vspace{1cm}


\begin{tabular}{m{17cm}}
\textbf{67-variant}
\newline

T1. Eki gewekli giperboloid. Kanonikalıq teńlemesi (giperbolanı simmetriya kósheri átirapında aylandırıwdan alınǵan betlik).\\

T2. Ellipstiń urınbasınıń teńlemesi (ellips, tuwrı, urınıw tochka, urınba teńlemesi).\\

A1. Giperbola teńlemesi berilgen: $\frac{x^{2}}{25}-\frac{y^{2}}{144}=1$. Onıń polyar teńlemesin dúziń.\\

A2. Tipin anıqlań: $x^{2}-4 xy+4 y^{2}+7 x-12=0$.\\

A3. Sheńberdiń $C$ orayı hám $R$ radiusın tabıń: $x^2+y^2+6 x-4 y+14=0$.\\

B1. $x^{2} - y^{2} = 27$ giperbolasına $4x + 2y - 7 = 0$ tuwrısına parallel bolǵan urınbanıń teńlemesin tabıń.  \\

B2. $x^{2} - 4y^{2} = 16$ giperbola berilgen. Onıń ekscentrisitetin, fokuslarınıń koordinataların tabıń hám asimptotalarınıń teńlemelerin dúziń.\\

B3. $\frac{x^{2}}{4} - \frac{y^{2}}{5} = 1$ giperbolaǵa $3x - 2y = 0$ tuwrısına parallel bolǵan urınbanıń teńlemesin dúziń.  \\

C1. $A(\frac{10}{3};\frac{5}{3})$ noqattan $\frac{x^{2}}{20} + \frac{y^{2}}{5} = 1$ ellipsine júrgizilgen urınbalardıń teńlemesin dúziń.  \\

C2. Fokusı $F( - 1; - 4)$noqatında bolǵan, sáykes direktrissası $x - 2 = 0$ teńlemesi menen berilgen $A( - 3; - 5)$ noqatınan ótiwshi ellipstiń teńlemesin dúziń.  \\

C3. $2x^{2} + 3y^{2} + 8x - 6y + 11 = 0$ teńlemesin ápiwaylastırıń qanday geometriyalıq obrazdı anıqlaytuǵının tabıń hám grafigin jasań.  \\

\end{tabular}
\vspace{1cm}


\begin{tabular}{m{17cm}}
\textbf{68-variant}
\newline

T1. ETIS -tiń ulıwma teńlemesin ápiwaylastırıw (ETIS -tiń ulıwma teńlemesi, koordinata sistemasın túrlendirip ETIS ulıwma teńlemesin ápiwaylastırıw).\\

T2. Ellipslik paraboloid (parabola, kósher, ellipslik paraboloid).\\

A1. Uchı koordinata basında jaylasqan hám $Oy$ kósherine qarata shep táreptegi yarım tegislikte jaylasqan parabolanıń teńlemesin dúziń: parametri $p=0,5$.\\

A2. Parabola teńlemesi berilgen: $y^2=6 x$. Onıń polyar teńlemesin dúziń.\\

A3. Berilgen sızıqlardıń oraylıq ekenligin kórsetiń hám orayın tabıń: $2 x^{2}-6 xy+5 y^{2}+22 x-36 y+11=0$.\\

B1. Koordinata kósherlerin túrlendirmey ETİS teńlemesin ápiwaylastırıń, yarım kósherlerin tabıń $41x^{2} + 2xy + 9y^{2} - 26x - 18y + 3 = 0$.  \\

B2. Ellips $3x^{2} + 4y^{2} - 12 = 0$ teńlemesi menen berilgen. Onıń kósherleriniń uzınlıqların, fokuslarınıń koordinataların hám ekscentrisitetin tabıń.  \\

B3. $3x + 10y - 25 = 0$ tuwrı menen $\frac{x^{2}}{25} + \frac{y^{2}}{4} = 1$ ellipstiń kesilisiw noqatların tabıń.\\

C1. Eger qálegen waqıt momentinde $M(x;y)$ noqat $A(8;4)$ noqattan hám ordinata kósherinen birdey aralıqta jaylassa, $M(x;y)$ noqatınıń háreket etiw troektoriyasınıń teńlemesin dúziń.  \\

C2. $2x^{2} + 3y^{2} + 8x - 6y + 11 = 0$ teńlemesin ápiwaylastırıń qanday geometriyalıq obrazdı anıqlaytuǵının tabıń hám grafigin jasań.\\

C3. Giperbolanıń ekscentrisiteti $\varepsilon = \frac{13}{12}$, fokusı $F(0;13)$ noqatında hám sáykes direktrisası $13y - 144 = 0$ teńlemesi menen berilgen bolsa, giperbolanıń teńlemesin dúziń.  \\

\end{tabular}
\vspace{1cm}


\begin{tabular}{m{17cm}}
\textbf{69-variant}
\newline

T1. Koordinata sistemasın túrlendiriw (birlik vektorlar, kósherler, parallel kóshiriw, koordinata kósherlerin burıw).\\

T2. ETIS-tıń invariantları (ETIS-tıń ulıwma teńlemesi, túrlendiriw, ETIS invariantları ).\\

A1. Sheńber teńlemesin dúziń: orayı $C (2;-3) $ noqatında jaylasqan hám radiusı $R=7$ ge teń.\\

A2. Fokusları abscissa kósherinde hám koordinata basına qarata simmetriyalıq jaylasqan ellipstiń teńlemesin dúziń: kishi kósheri $10$, ekscentrisitet $\varepsilon=12/13$.\\

A3. Giperbola teńlemesi berilgen: $\frac{x^{2}}{16}-\frac{y^{2}}{9}=1$. Onıń polyar teńlemesin dúziń.\\

B1. $\rho = \frac{6}{1 - cos\theta}$ polyar teńlemesi menen qanday sızıq berilgenin anıqlań.  \\

B2. $\frac{x^{2}}{16} - \frac{y^{2}}{64} = 1$, giperbolasına berilgen $10x - 3y + 9 = 0$ tuwrı sızıǵına parallel bolǵan urınbanıń teńlemesin dúziń.  \\

B3. Koordinata kósherlerin túrlendirmey ETİS ulıwma teńlemesin ápiwaylastırıń, yarım kósherlerin tabıń: $13x^{2} + 18xy + 37y^{2} - 26x - 18y + 3 = 0$.  \\

C1. $4x^{2} - 4xy + y^{2} - 6x + 8y + 13 = 0$ ETİS-ǵı orayǵa iyeme? Orayǵa iye bolsa orayın anıqlań: jalǵız orayǵa iyeme-?, sheksiz orayǵa iyeme-?  \\

C2. Tóbesi $A(-4;0)$ noqatında, al, direktrisası $y - 2 = 0$ tuwrı sızıq bolǵan parabolanıń teńlemesin dúziń.\\

C3. $16x^{2} - 9y^{2} - 64x - 54y - 161 = 0$ teńlemesi giperbolanıń teńlemesi ekenin anıqlań hám onıń orayı $C$, yarım kósherleri, ekscentrisitetin, asimptotalarınıń teńlemelerin dúziń.  \\

\end{tabular}
\vspace{1cm}


\begin{tabular}{m{17cm}}
\textbf{70-variant}
\newline

T1. Betlik haqqında túsinik (tuwrı, iymek sızıq, betliktiń anıqlamaları hám formulaları).\\

T2. Parabola hám onıń kanonikalıq teńlemesi (anıqlaması, fokusı, direktrisası, kanonikalıq teńlemesi).\\

A1. Tipin anıqlań: $25 x^{2}-20 xy+4 y^{2}-12 x+20 y-17=0$.\\

A2. Sheńber teńlemesin dúziń: sheńber $A (2;6 ) $ noqatınan ótedi hám orayı $C (-1;2) $ noqatında jaylasqan .\\

A3. Fokusları abscissa kósherinde hám koordinata basına qarata simmetriyalıq jaylasqan giperbolanıń teńlemesin dúziń: oqları $2 a=10$ hám $2 b=8$.\\

B1. $x^{2} - 4y^{2} = 16$ giperbola berilgen. Onıń ekscentrisitetin, fokuslarınıń koordinataların tabıń hám asimptotalarınıń teńlemelerin dúziń.\\

B2. $y^{2} = 3x$ parabolası menen $\frac{x^{2}}{100} + \frac{y^{2}}{225} = 1$ ellipsiniń kesilisiw noqatların tabıń.  \\

B3. $\rho = \frac{144}{13 - 5cos\theta}$ ellipsti anıqlaytuǵının kórsetiń hám onıń yarım kósherlerin anıqlań.\\

C1. Fokuslari $F(3;4), F(-3;-4)$ noqatlarında jaylasqan direktrisaları orasıdaǵı aralıq 3,6 ǵa teń bolǵan giperbolanıń teńlemesin dúziń.  \\

C2. $14x^{2} + 24xy + 21y^{2} - 4x + 18y - 139 = 0$ iymek sızıǵınıń tipin anıqlań, eger oraylı iymek sızıq bolsa orayınıń koordinataların tabıń.  \\

C3. $4x^{2} + 24xy + 11y^{2} + 64x + 42y + 51 = 0$ iymek sızıǵınıń tipin anıqlań eger orayı bar bolsa, onıń orayınıń koordinataların tabıń hám koordinata basın orayǵa parallel kóshiriw ámelin orınlań.  \\

\end{tabular}
\vspace{1cm}


\begin{tabular}{m{17cm}}
\textbf{71-variant}
\newline

T1. ETIS-tıń orayın anıqlaw forması (ETIS-tıń ulıwma teńlemesi, orayın anıqlaw forması).\\

T2. Ellipsoida. Kanonikalıq teńlemesi (ellipsti simmetriya kósheri dogereginde aylandırıwdan alınǵan betlik, kanonikalıq teńlemesi).\\

A1. Polyar teńlemesi menen berilgen iymek sızıqtıń tipin anıqlań: $\rho=\frac{5}{3-4\cos\theta}$.\\

A2. Tipin anıqlań: $3 x^{2}-2 xy-3 y^{2}+12 y-15=0$.\\

A3. Sheńberdiń $C$ orayı hám $R$ radiusın tabıń: $x^2+y^2-2 x+4 y-20=0$.\\

B1. $2x + 2y - 3 = 0$ tuwrısına parallel bolıp $\frac{x^{2}}{16} + \frac{y^{2}}{64} = 1$ giperbolasına urınıwshı tuwrınıń teńlemesin dúziń.  \\

B2. Koordinata kósherlerin túrlendirmey ETİS teńlemesin ápiwaylastırıń, yarım kósherlerin tabıń $4x^{2} - 4xy + 9y^{2} - 26x - 18y + 3 = 0$.\\

B3. $3x + 4y - 12 = 0$ tuwrı sızıǵı hám $y^{2} = - 9x$ parabolasınıń kesilisiw noqatların tabıń.  \\

C1. Úlken kósheri 26 ǵa, fokusları $F( - 10;0)$, $F(14;0)$ noqatlarında jaylasqan ellipstiń teńlemesin dúziń.  \\

C2. $32x^{2} + 52xy - 7y^{2} + 180 = 0$ ETİS teńlemesin ápiwayı túrge alıp keliń, tipin anıqlań, qanday geometriyalıq obrazdı anıqlaytuǵının kórsetiń, sızılmasın góne hám taza koordinatalar sistemasına qarata jasań.  \\

C3. $\frac{x^{2}}{3} - \frac{y^{2}}{5} = 1$ giperbolasına $P(1; - 5)$ noqatında júrgizilgen urınbalardıń teńlemesin dúziń.\\

\end{tabular}
\vspace{1cm}


\begin{tabular}{m{17cm}}
\textbf{72-variant}
\newline

T1. Giperbola. Kanonikalıq teńlemesi (fokuslar, kósherler, direktrisalar, giperbola, ekscentrisitet, kanonikalıq teńlemesi).\\

T2. ETIS-tıń ulıwma teńlemesin koordinata basın parallel kóshiriw arqalı ápiwayılastırıń (ETIS- tıń ulıwma teńlemesin parallel kóshiriw formulası).\\

A1. Fokusları abscissa kósherinde hám koordinata basına qarata simmetriyalıq jaylasqan ellipstiń teńlemesin dúziń: yarım oqları 5 hám 2.\\

A2. Polyar teńlemesi menen berilgen iymek sızıqtıń tipin anıqlań: $\rho=\frac{1}{3-3\cos\theta}$.\\

A3. Tipin anıqlań: $2 x^{2}+3 y^{2}+8 x-6 y+11=0$.\\

B1. $\rho = \frac{10}{2 - cos\theta}$ polyar teńlemesi menen qanday sızıq berilgenin anıqlań.  \\

B2. $2x + 2y - 3 = 0$ tuwrısına perpendikulyar bolıp $x^{2} = 16y$ parabolasına urınıwshı tuwrınıń teńlemesin dúziń.  \\

B3. ETİS-tıń ulıwma teńlemesin koordinata sistemasın túrlendirmey ápiwaylastırıń, tipin anıqlań, obrazı qanday sızıqtı anıqlaytuǵının kórsetiń. $7x^{2} - 8xy + y^{2} - 16x - 2y - 51 = 0$  \\

C1. $\frac{x^{2}}{100} + \frac{y^{2}}{36} = 1$ ellipsiniń oń jaqtaǵı fokusınan 14 ge teń aralıqta bolǵan noqattı tabıń.  \\

C2. Fokusı $F(2; - 1)$ noqatında jaylasqan, sáykes direktrisası $x - y - 1 = 0$ teńlemesi menen berilgen parabolanıń teńlemesin dúziń.  \\

C3. $2x^{2} + 10xy + 12y^{2} - 7x + 18y - 15 = 0$ ETİS teńlemesin ápiwayı túrge alıp keliń, tipin anıqlań, qanday geometriyalıq obrazdı anıqlaytuǵının kórsetiń, sızılmasın góne hám taza koordinatalar sistemasına qarata jasań  \\

\end{tabular}
\vspace{1cm}


\begin{tabular}{m{17cm}}
\textbf{73-variant}
\newline

T1. Giperbolalıq paraboloydtıń tuwrı sızıqlı jasawshıları (Giperbolalıq paraboloydtı jasawshı tuwrı sızıqlar dástesi).\\

T2. Ellipstiń polyar koordinatalardaǵı teńlemesi (polyar koordinatalar sistemasında ellipstiń teńlemesi).\\

A1. Sheńber teńlemesin dúziń: orayı koordinata basında jaylasqan hám radiusı $R=3$ ge teń.\\

A2. Uchı koordinata basında jaylasqan hám $Oy$ kósherine qarata oń táreptegi yarım tegislikte jaylasqan parabolanıń teńlemesin dúziń: parametri $p=3$.\\

A3. Polyar teńlemesi menen berilgen iymek sızıqtıń tipin anıqlań: $\rho=\frac{10}{1-\frac{3}{2}\cos\theta}$.\\

B1. $\rho = \frac{5}{3 - 4cos\theta}$ teńlemesi menen qanday sızıq berilgenin hám yarım kósherlerin tabıń.  \\

B2. $y^{2} = 12x$ paraborolasına $3x - 2y + 30 = 0$ tuwrı sızıǵına parallel bolǵan urınbanıń teńlemesin dúziń.  \\

B3. $41x^{2} + 24xy + 9y^{2} + 24x + 18y - 36 = 0$ ETİS tipin anıqlań hám orayların tabıń koordinata kósherlerin túrlendirmey qanday sızıqtı anıqlaytuǵının kórsetiń yarım kósherlerin tabıń.  \\

C1. $\frac{x^{2}}{2} + \frac{y^{2}}{3} = 1$, ellipsin $x + y - 2 = 0$ noqatınan júrgizilgen urınbalarınıń teńlemesin dúziń.  \\

C2. $y^{2} = 20x$ parabolasınıń $M$ noqatın tabıń, eger onıń abscissası 7 ge teń bolsa, fokal radiusın hám fokal radius jaylasqan tuwrını anıqlań.\\

C3. Fokusı $F(7;2)$ noqatında jaylasqan, sáykes direktrisası $x - 5 = 0$ teńlemesi menen berilgen parabolanıń teńlemesin dúziń.  \\

\end{tabular}
\vspace{1cm}


\begin{tabular}{m{17cm}}
\textbf{74-variant}
\newline

T1. ETIS-tıń ulıwma teńlemesin koordinata kósherlerin burıw arqalı ápiwaylastırıń (ETIS-tıń ulıwma teńlemeleri, koordinata kósherin burıw formulası, teńlemeni kanonik túrge alıp keliw).\\

T2. Cilindrlik betlikler (jasawshı tuwrı sızıq, baǵıtlawshı iymek sızıq, cilindrlik betlik).\\

A1. Berilgen sızıqlardıń oraylıq ekenligin kórsetiń hám orayın tabıń: $3 x^{2}+5 xy+y^{2}-8 x-11 y-7=0$.\\

A2. Sheńber teńlemesin dúziń: orayı $C (1;-1) $ noqatında jaylasqan hám $5 x-12 y+9 -0$ tuwrı sızıǵına urınadı .\\

A3. Fokusları abscissa kósherinde hám koordinata basına qarata simmetriyalıq jaylasqan ellipstiń teńlemesin dúziń: úlken kósheri $8$, direktrisaları arasındaǵı aralıq $16$.\\

B1. $\frac{x^{2}}{4} - \frac{y^{2}}{5} = 1$, giperbolanıń $3x - 2y = 0$ tuwrı sızıǵına parallel bolǵan urınbasınıń teńlemesin dúziń.  \\

B2. Koordinata kósherlerin túrlendirmey ETİS teńlemesin ápiwaylastırıń, qanday geometriyalıq obrazdı anıqlaytuǵının kórsetiń $4x^{2} - 4xy + y^{2} + 4x - 2y + 1 = 0$.  \\

B3. $\frac{x^{2}}{4} - \frac{y^{2}}{5} = 1$ giperbolasına $3x + 2y = 0$ tuwrı sızıǵına perpendikulyar bolǵan urınba tuwrınıń teńlemesin dúziń.\\

C1. $2x^{2} + 3y^{2} + 8x - 6y + 11 = 0$ teńlemesi menen qanday tiptegi sızıq berilgenin anıqlań hám onıń teńlemesin ápiwaylastırıń hám grafigin jasań.  \\

C2. $\frac{x^{2}}{25} + \frac{y^{2}}{16} = 1$, ellipsine $C(10; - 8)$ noqatınan júrgizilgen urınbalarınıń teńlemesin dúziń.  \\

C3. $y^{2} = 20x$ parabolasınıń abscissası 7 ge teń bolǵan $M$ noqatınıń fokal radiusın tabıń hám fokal radiusı jatqan tuwrınıń teńlemesin dúziń.  \\

\end{tabular}
\vspace{1cm}


\begin{tabular}{m{17cm}}
\textbf{75-variant}
\newline

T1. Parabolanıń urınbasınıń teńlemesi (parabola, tuwrı, urınıw noqatı, urınba teńlemesi).\\

T2. Bir gewekli giperboloid. Kanonikalıq teńlemesi (giperbolanı simmetriya kósheri átirapında aylandırıwdan alınǵan betlik).\\

A1. Tipin anıqlań: $2 x^{2}+10 xy+12 y^{2}-7 x+18 y-15=0$.\\

A2. Sheńber teńlemesin dúziń: orayı koordinata basında jaylasqan hám $3 x-4 y+20=0$ tuwrı sızıǵına urınadı.\\

A3. Fokusları abscissa kósherinde hám koordinata basına qarata simmetriyalıq jaylasqan ellipstiń teńlemesin dúziń: úlken kósheri $20$, ekscentrisitet $\varepsilon=3/5$.\\

B1. $x^{2} + 4y^{2} = 25$ ellipsi menen $4x - 2y + 23 = 0$ tuwrı sızıǵına parallel bolǵan urınba tuwrı sızıqtıń teńlemesin dúziń.  \\

B2. $\frac{x^{2}}{20} - \frac{y^{2}}{5} = 1$ giperbolasına $4x + 3y - 7 = 0$ tuwrısına perpendikulyar bolǵan urınbanıń teńlemesin dúziń.  \\

B3. Ellips $3x^{2} + 4y^{2} - 12 = 0$ teńlemesi menen berilgen. Onıń kósherleriniń uzınlıqların, fokuslarınıń koordinataların hám ekscentrisitetin tabıń.  \\

C1. Eger waqıttıń qálegen momentinde $M(x;y)$ noqat $5x - 16 = 0$ tuwrı sızıqqa qaraǵanda $A(5;0)$ noqattan 1,25 márte uzaqlıqta jaylasqan. Usı $M(x;y)$ noqattıń háreketiniń teńlemesin dúziń.  \\

C2. $4x^{2} - 4xy + y^{2} - 2x - 14y + 7 = 0$ ETİS teńlemesin ápiwayı túrge alıp keliń, tipin anıqlań, qanday geometriyalıq obrazdı anıqlaytuǵının kórsetiń, sızılmasın góne hám taza koordinatalar sistemasına qarata jasań.  \\

C3. $\frac{x^{2}}{3} - \frac{y^{2}}{5} = 1$, giperbolasına $P(4;2)$ noqatınan júrgizilgen urınbalardıń teńlemesin dúziń.  \\

\end{tabular}
\vspace{1cm}


\begin{tabular}{m{17cm}}
\textbf{76-variant}
\newline

T1. Giperbolanıń polyar koordinatadaǵı teńlemesi (Polyar múyeshi, polyar radiusi giperbolanıń polyar teńlemesi).\\

T2. Betliktiń kanonikalıq teńlemeleri. Betlik haqqında túsinik. (Betliktiń anıqlaması, formulaları, kósher, baǵıtlawshı tuwrılar).\\

A1. Tipin anıqlań: $4 x^{2}-y^{2}+8 x-2 y+3=0$.\\

A2. Sheńber teńlemesin dúziń: sheńber diametriniń ushları $A (3;2) $ hám $B (-1;6 ) $ noqatlarında jaylasqan.\\

A3. Fokusları abscissa kósherinde hám koordinata basına qarata simmetriyalıq jaylasqan giperbolanıń teńlemesin dúziń: direktrisaları arasındaǵı aralıq $228/13$ hám fokusları arasındaǵı aralıq $2 c=26$.\\

B1. $x^{2} - y^{2} = 27$ giperbolasına $4x + 2y - 7 = 0$ tuwrısına parallel bolǵan urınbanıń teńlemesin tabıń.  \\

B2. Koordinata kósherlerin túrlendirmey ETİS teńlemesin ápiwaylastırıń, yarım kósherlerin tabıń $41x^{2} + 2xy + 9y^{2} - 26x - 18y + 3 = 0$.  \\

B3. $x^{2} - 4y^{2} = 16$ giperbola berilgen. Onıń ekscentrisitetin, fokuslarınıń koordinataların tabıń hám asimptotalarınıń teńlemelerin dúziń.\\

C1. $M(2; - \frac{5}{3})$ noqatı $\frac{x^{2}}{9} + \frac{y^{2}}{5} = 1$ ellipsinde jaylasqan. $M$ noqatınıń fokal radiusları jatıwshı tuwrı sızıq teńlemelerin dúziń.  \\

C2. Fokusı $F( - 1; - 4)$ noqatında jaylasqan, sáykes direktrisası $x - 2 = 0$ teńlemesi menen berilgen, $A( - 3; - 5)$ noqatınan ótiwshi ellipstiń teńlemesin dúziń.  \\

C3. $32x^{2} + 52xy - 9y^{2} + 180 = 0$ ETİS teńlemesin ápiwaylastırıń, tipin anıqlań, qanday geometriyalıq obrazdı anıqlaytuǵının kórsetiń, sızılmasın sızıń.  \\

\end{tabular}
\vspace{1cm}


\begin{tabular}{m{17cm}}
\textbf{77-variant}
\newline

T1. Giperbolanıń urınbasınıń teńlemesi (giperbolaǵa berilgen noqatta júrgizilgen urınba teńlemesi).\\

T2. Ekinshi tártipli betliktiń ulıwma teńlemesi. Orayın anıqlaw formulası.\\

A1. Tipin anıqlań: $9 x^{2}-16 y^{2}-54 x-64 y-127=0$.\\

A2. Sheńber teńlemesin dúziń: $A (3;1) $ hám $B (-1;3) $ noqatlardan ótedi, orayı $3 x-y-2=0$ tuwrı sızıǵında jaylasqan .\\

A3. Uchı koordinata basında jaylasqan hám $Ox$ kósherine qarata tómengi yarım tegislikte jaylasqan parabolanıń teńlemesin dúziń: parametri $p=3$.\\

B1. $3x + 10y - 25 = 0$ tuwrı menen $\frac{x^{2}}{25} + \frac{y^{2}}{4} = 1$ ellipstiń kesilisiw noqatların tabıń.\\

B2. $\rho = \frac{6}{1 - cos\theta}$ polyar teńlemesi menen qanday sızıq berilgenin anıqlań.  \\

B3. $\frac{x^{2}}{4} - \frac{y^{2}}{5} = 1$ giperbolaǵa $3x - 2y = 0$ tuwrısına parallel bolǵan urınbanıń teńlemesin dúziń.  \\

C1. $A(\frac{10}{3};\frac{5}{3})$ noqattan $\frac{x^{2}}{20} + \frac{y^{2}}{5} = 1$ ellipsine júrgizilgen urınbalardıń teńlemesin dúziń.  \\

C2. Fokusı $F( - 1; - 4)$noqatında bolǵan, sáykes direktrissası $x - 2 = 0$ teńlemesi menen berilgen $A( - 3; - 5)$ noqatınan ótiwshi ellipstiń teńlemesin dúziń.  \\

C3. $2x^{2} + 3y^{2} + 8x - 6y + 11 = 0$ teńlemesin ápiwaylastırıń qanday geometriyalıq obrazdı anıqlaytuǵının tabıń hám grafigin jasań.  \\

\end{tabular}
\vspace{1cm}


\begin{tabular}{m{17cm}}
\textbf{78-variant}
\newline

T1. Ellips hám onıń kanonikalıq teńlemesi (anıqlaması, fokuslar, ellipstiń kanonikalıq teńlemesi, ekscentrisiteti, direktrisaları).\\

T2. Ekinshi tártipli aylanba betlikler (koordinata sisteması, tegislik, vektor iymek sızıq, aylanba betlik).\\

A1. Tipin anıqlań: $5 x^{2}+14 xy+11 y^{2}+12 x-7 y+19=0$.\\

A2. Fokusları abscissa kósherinde hám koordinata basına qarata simmetriyalıq jaylasqan giperbolanıń teńlemesin dúziń: direktrisaları arasındaǵı aralıq $32/5$ hám kósheri $2 b=6$.\\

A3. Tipin anıqlań: $4 x^2+9 y^2-40 x+36 y+100=0$.\\

B1. Koordinata kósherlerin túrlendirmey ETİS ulıwma teńlemesin ápiwaylastırıń, yarım kósherlerin tabıń: $13x^{2} + 18xy + 37y^{2} - 26x - 18y + 3 = 0$.  \\

B2. Ellips $3x^{2} + 4y^{2} - 12 = 0$ teńlemesi menen berilgen. Onıń kósherleriniń uzınlıqların, fokuslarınıń koordinataların hám ekscentrisitetin tabıń.  \\

B3. $y^{2} = 3x$ parabolası menen $\frac{x^{2}}{100} + \frac{y^{2}}{225} = 1$ ellipsiniń kesilisiw noqatların tabıń.  \\

C1. Eger qálegen waqıt momentinde $M(x;y)$ noqat $A(8;4)$ noqattan hám ordinata kósherinen birdey aralıqta jaylassa, $M(x;y)$ noqatınıń háreket etiw troektoriyasınıń teńlemesin dúziń.  \\

C2. $2x^{2} + 3y^{2} + 8x - 6y + 11 = 0$ teńlemesin ápiwaylastırıń qanday geometriyalıq obrazdı anıqlaytuǵının tabıń hám grafigin jasań.\\

C3. Giperbolanıń ekscentrisiteti $\varepsilon = \frac{13}{12}$, fokusı $F(0;13)$ noqatında hám sáykes direktrisası $13y - 144 = 0$ teńlemesi menen berilgen bolsa, giperbolanıń teńlemesin dúziń.  \\

\end{tabular}
\vspace{1cm}


\begin{tabular}{m{17cm}}
\textbf{79-variant}
\newline

T1. Parabolanıń polyar koordinatalardaǵı teńlemesi (polyar koordinata sistemasında parabolanıń teńlemesi).\\

T2. ETIS-tıń ulıwma teńlemesin klassifikatsiyalaw (ETIS-tıń ulıwma teńlemesi, ETIS-tıń ulıwma teńlemesin ápiwaylastırıw, klassifikatsiyalaw).\\

A1. Fokusları abscissa kósherinde hám koordinata basına qarata simmetriyalıq jaylasqan ellipstiń teńlemesin dúziń: fokusları arasındaǵı aralıq $2 c=6$ hám ekscentrisitet $\varepsilon=3/5$.\\

A2. Fokusları abscissa kósherinde hám koordinata basına qarata simmetriyalıq jaylasqan giperbolanıń teńlemesin dúziń: fokusları arasındaǵı aralıǵı $2 c=10$ hám kósheri $2 b=8$.\\

A3. Fokusları abscissa kósherinde hám koordinata basına qarata simmetriyalıq jaylasqan ellipstiń teńlemesin dúziń: úlken kósheri $10$, fokusları arasındaǵı aralıq $2 c=8$.\\

B1. $\rho = \frac{144}{13 - 5cos\theta}$ ellipsti anıqlaytuǵının kórsetiń hám onıń yarım kósherlerin anıqlań.\\

B2. $\frac{x^{2}}{16} - \frac{y^{2}}{64} = 1$, giperbolasına berilgen $10x - 3y + 9 = 0$ tuwrı sızıǵına parallel bolǵan urınbanıń teńlemesin dúziń.  \\

B3. Koordinata kósherlerin túrlendirmey ETİS teńlemesin ápiwaylastırıń, yarım kósherlerin tabıń $4x^{2} - 4xy + 9y^{2} - 26x - 18y + 3 = 0$.\\

C1. $4x^{2} - 4xy + y^{2} - 6x + 8y + 13 = 0$ ETİS-ǵı orayǵa iyeme? Orayǵa iye bolsa orayın anıqlań: jalǵız orayǵa iyeme-?, sheksiz orayǵa iyeme-?  \\

C2. Tóbesi $A(-4;0)$ noqatında, al, direktrisası $y - 2 = 0$ tuwrı sızıq bolǵan parabolanıń teńlemesin dúziń.\\

C3. $16x^{2} - 9y^{2} - 64x - 54y - 161 = 0$ teńlemesi giperbolanıń teńlemesi ekenin anıqlań hám onıń orayı $C$, yarım kósherleri, ekscentrisitetin, asimptotalarınıń teńlemelerin dúziń.  \\

\end{tabular}
\vspace{1cm}


\begin{tabular}{m{17cm}}
\textbf{80-variant}
\newline

T1. Eki gewekli giperboloid. Kanonikalıq teńlemesi (giperbolanı simmetriya kósheri átirapında aylandırıwdan alınǵan betlik).\\

T2. Ellipstiń urınbasınıń teńlemesi (ellips, tuwrı, urınıw tochka, urınba teńlemesi).\\

A1. Fokusları abscissa kósherinde hám koordinata basına qarata simmetriyalıq jaylasqan ellipstiń teńlemesin dúziń: kishi kósheri $24$, fokusları arasındaǵı aralıq $2 c=10$.\\

A2. Fokusları abscissa kósherinde hám koordinata basına qarata simmetriyalıq jaylasqan giperbolanıń teńlemesin dúziń: fokusları arasındaǵı aralıq $2 c=6$ hám ekscentrisitet $\varepsilon=3/2$.\\

A3. Fokusları abscissa kósherinde hám koordinata basına qarata simmetriyalıq jaylasqan giperbolanıń teńlemesin dúziń: direktrisaları arasındaǵı aralıq $8/3$ hám ekscentrisitet $\varepsilon=3/2$.\\

B1. $3x + 4y - 12 = 0$ tuwrı sızıǵı hám $y^{2} = - 9x$ parabolasınıń kesilisiw noqatların tabıń.  \\

B2. $\rho = \frac{10}{2 - cos\theta}$ polyar teńlemesi menen qanday sızıq berilgenin anıqlań.  \\

B3. $2x + 2y - 3 = 0$ tuwrısına parallel bolıp $\frac{x^{2}}{16} + \frac{y^{2}}{64} = 1$ giperbolasına urınıwshı tuwrınıń teńlemesin dúziń.  \\

C1. Fokuslari $F(3;4), F(-3;-4)$ noqatlarında jaylasqan direktrisaları orasıdaǵı aralıq 3,6 ǵa teń bolǵan giperbolanıń teńlemesin dúziń.  \\

C2. $14x^{2} + 24xy + 21y^{2} - 4x + 18y - 139 = 0$ iymek sızıǵınıń tipin anıqlań, eger oraylı iymek sızıq bolsa orayınıń koordinataların tabıń.  \\

C3. $4x^{2} + 24xy + 11y^{2} + 64x + 42y + 51 = 0$ iymek sızıǵınıń tipin anıqlań eger orayı bar bolsa, onıń orayınıń koordinataların tabıń hám koordinata basın orayǵa parallel kóshiriw ámelin orınlań.  \\

\end{tabular}
\vspace{1cm}


\begin{tabular}{m{17cm}}
\textbf{81-variant}
\newline

T1. ETIS -tiń ulıwma teńlemesin ápiwaylastırıw (ETIS -tiń ulıwma teńlemesi, koordinata sistemasın túrlendirip ETIS ulıwma teńlemesin ápiwaylastırıw).\\

T2. Ellipslik paraboloid (parabola, kósher, ellipslik paraboloid).\\

A1. Sheńberdiń $C$ orayı hám $R$ radiusın tabıń: $x^2+y^2+4 x-2 y+5=0$.\\

A2. Fokusları abscissa kósherinde hám koordinata basına qarata simmetriyalıq jaylasqan giperbolanıń teńlemesin dúziń: úlken kósheri $2 a=16$ hám ekscentrisitet $\varepsilon=5/4$.\\

A3. Polyar teńlemesi menen berilgen iymek sızıqtıń tipin anıqlań: $\rho=\frac{6}{1-\cos 0}$.\\

B1. ETİS-tıń ulıwma teńlemesin koordinata sistemasın túrlendirmey ápiwaylastırıń, tipin anıqlań, obrazı qanday sızıqtı anıqlaytuǵının kórsetiń. $7x^{2} - 8xy + y^{2} - 16x - 2y - 51 = 0$  \\

B2. $\rho = \frac{5}{3 - 4cos\theta}$ teńlemesi menen qanday sızıq berilgenin hám yarım kósherlerin tabıń.  \\

B3. $2x + 2y - 3 = 0$ tuwrısına perpendikulyar bolıp $x^{2} = 16y$ parabolasına urınıwshı tuwrınıń teńlemesin dúziń.  \\

C1. Úlken kósheri 26 ǵa, fokusları $F( - 10;0)$, $F(14;0)$ noqatlarında jaylasqan ellipstiń teńlemesin dúziń.  \\

C2. $32x^{2} + 52xy - 7y^{2} + 180 = 0$ ETİS teńlemesin ápiwayı túrge alıp keliń, tipin anıqlań, qanday geometriyalıq obrazdı anıqlaytuǵının kórsetiń, sızılmasın góne hám taza koordinatalar sistemasına qarata jasań.  \\

C3. $\frac{x^{2}}{3} - \frac{y^{2}}{5} = 1$ giperbolasına $P(1; - 5)$ noqatında júrgizilgen urınbalardıń teńlemesin dúziń.\\

\end{tabular}
\vspace{1cm}


\begin{tabular}{m{17cm}}
\textbf{82-variant}
\newline

T1. Koordinata sistemasın túrlendiriw (birlik vektorlar, kósherler, parallel kóshiriw, koordinata kósherlerin burıw).\\

T2. ETIS-tıń invariantları (ETIS-tıń ulıwma teńlemesi, túrlendiriw, ETIS invariantları ).\\

A1. Berilgen sızıqlardıń oraylıq ekenligin kórsetiń hám orayın tabıń: $5 x^{2}+4 xy+2 y^{2}+20 x+20 y-18=0$.\\

A2. Sheńber teńlemesin dúziń: $A (1;1) $, $B (1;-1) $ hám $C (2;0) $ noqatlardan ótedi.\\

A3. Uchı koordinata basında jaylasqan hám $Ox$ kósherine qarata joqarı yarım tegislikte jaylasqan parabolanıń teńlemesin dúziń: parametri $p=1/4$.\\

B1. $41x^{2} + 24xy + 9y^{2} + 24x + 18y - 36 = 0$ ETİS tipin anıqlań hám orayların tabıń koordinata kósherlerin túrlendirmey qanday sızıqtı anıqlaytuǵının kórsetiń yarım kósherlerin tabıń.  \\

B2. $y^{2} = 12x$ paraborolasına $3x - 2y + 30 = 0$ tuwrı sızıǵına parallel bolǵan urınbanıń teńlemesin dúziń.  \\

B3. Koordinata kósherlerin túrlendirmey ETİS teńlemesin ápiwaylastırıń, qanday geometriyalıq obrazdı anıqlaytuǵının kórsetiń $4x^{2} - 4xy + y^{2} + 4x - 2y + 1 = 0$.  \\

C1. $\frac{x^{2}}{100} + \frac{y^{2}}{36} = 1$ ellipsiniń oń jaqtaǵı fokusınan 14 ge teń aralıqta bolǵan noqattı tabıń.  \\

C2. Fokusı $F(2; - 1)$ noqatında jaylasqan, sáykes direktrisası $x - y - 1 = 0$ teńlemesi menen berilgen parabolanıń teńlemesin dúziń.  \\

C3. $2x^{2} + 10xy + 12y^{2} - 7x + 18y - 15 = 0$ ETİS teńlemesin ápiwayı túrge alıp keliń, tipin anıqlań, qanday geometriyalıq obrazdı anıqlaytuǵının kórsetiń, sızılmasın góne hám taza koordinatalar sistemasına qarata jasań  \\

\end{tabular}
\vspace{1cm}


\begin{tabular}{m{17cm}}
\textbf{83-variant}
\newline

T1. Betlik haqqında túsinik (tuwrı, iymek sızıq, betliktiń anıqlamaları hám formulaları).\\

T2. Parabola hám onıń kanonikalıq teńlemesi (anıqlaması, fokusı, direktrisası, kanonikalıq teńlemesi).\\

A1. Polyar teńlemesi menen berilgen iymek sızıqtıń tipin anıqlań: $\rho=\frac{5}{1-\frac{1}{2}\cos\theta}$.\\

A2. Tipin anıqlań: $9 x^{2}+4 y^{2}+18 x-8 y+49=0$.\\

A3. Sheńberdiń $C$ orayı hám $R$ radiusın tabıń: $x^2+y^2-2 x+4 y-14=0$.\\

B1. $\frac{x^{2}}{4} - \frac{y^{2}}{5} = 1$, giperbolanıń $3x - 2y = 0$ tuwrı sızıǵına parallel bolǵan urınbasınıń teńlemesin dúziń.  \\

B2. $\frac{x^{2}}{4} - \frac{y^{2}}{5} = 1$ giperbolasına $3x + 2y = 0$ tuwrı sızıǵına perpendikulyar bolǵan urınba tuwrınıń teńlemesin dúziń.\\

B3. $x^{2} + 4y^{2} = 25$ ellipsi menen $4x - 2y + 23 = 0$ tuwrı sızıǵına parallel bolǵan urınba tuwrı sızıqtıń teńlemesin dúziń.  \\

C1. $\frac{x^{2}}{2} + \frac{y^{2}}{3} = 1$, ellipsin $x + y - 2 = 0$ noqatınan júrgizilgen urınbalarınıń teńlemesin dúziń.  \\

C2. $y^{2} = 20x$ parabolasınıń $M$ noqatın tabıń, eger onıń abscissası 7 ge teń bolsa, fokal radiusın hám fokal radius jaylasqan tuwrını anıqlań.\\

C3. Fokusı $F(7;2)$ noqatında jaylasqan, sáykes direktrisası $x - 5 = 0$ teńlemesi menen berilgen parabolanıń teńlemesin dúziń.  \\

\end{tabular}
\vspace{1cm}


\begin{tabular}{m{17cm}}
\textbf{84-variant}
\newline

T1. ETIS-tıń orayın anıqlaw forması (ETIS-tıń ulıwma teńlemesi, orayın anıqlaw forması).\\

T2. Ellipsoida. Kanonikalıq teńlemesi (ellipsti simmetriya kósheri dogereginde aylandırıwdan alınǵan betlik, kanonikalıq teńlemesi).\\

A1. Fokusları abscissa kósherinde hám koordinata basına qarata simmetriyalıq jaylasqan ellipstiń teńlemesin dúziń: direktrisaları arasındaǵı aralıq $5$ hám fokusları arasındaǵı aralıq $2 c=4$.\\

A2. Ellips teńlemesi berilgen: $\frac{x^2}{25}+\frac{y^2}{16}=1$. Onıń polyar teńlemesin dúziń.\\

A3. Berilgen sızıqlardıń oraylıq ekenligin kórsetiń hám orayın tabıń: $9 x^{2}-4 xy-7 y^{2}-12=0$.\\

B1. $x^{2} - 4y^{2} = 16$ giperbola berilgen. Onıń ekscentrisitetin, fokuslarınıń koordinataların tabıń hám asimptotalarınıń teńlemelerin dúziń.\\

B2. $\frac{x^{2}}{20} - \frac{y^{2}}{5} = 1$ giperbolasına $4x + 3y - 7 = 0$ tuwrısına perpendikulyar bolǵan urınbanıń teńlemesin dúziń.  \\

B3. Koordinata kósherlerin túrlendirmey ETİS teńlemesin ápiwaylastırıń, yarım kósherlerin tabıń $41x^{2} + 2xy + 9y^{2} - 26x - 18y + 3 = 0$.  \\

C1. $2x^{2} + 3y^{2} + 8x - 6y + 11 = 0$ teńlemesi menen qanday tiptegi sızıq berilgenin anıqlań hám onıń teńlemesin ápiwaylastırıń hám grafigin jasań.  \\

C2. $\frac{x^{2}}{25} + \frac{y^{2}}{16} = 1$, ellipsine $C(10; - 8)$ noqatınan júrgizilgen urınbalarınıń teńlemesin dúziń.  \\

C3. $y^{2} = 20x$ parabolasınıń abscissası 7 ge teń bolǵan $M$ noqatınıń fokal radiusın tabıń hám fokal radiusı jatqan tuwrınıń teńlemesin dúziń.  \\

\end{tabular}
\vspace{1cm}


\begin{tabular}{m{17cm}}
\textbf{85-variant}
\newline

T1. Giperbola. Kanonikalıq teńlemesi (fokuslar, kósherler, direktrisalar, giperbola, ekscentrisitet, kanonikalıq teńlemesi).\\

T2. ETIS-tıń ulıwma teńlemesin koordinata basın parallel kóshiriw arqalı ápiwayılastırıń (ETIS- tıń ulıwma teńlemesin parallel kóshiriw formulası).\\

A1. Sheńber teńlemesin dúziń: orayı $C (6 ;-8) $ noqatında jaylasqan hám koordinata basınan ótedi.\\

A2. Fokusları abscissa kósherinde hám koordinata basına qarata simmetriyalıq jaylasqan giperbolanıń teńlemesin dúziń: asimptotalar teńlemeleri $y=\pm \frac{4}{3}x$ hám fokusları arasındaǵı aralıq $2 c=20$.\\

A3. Polyar teńlemesi menen berilgen iymek sızıqtıń tipin anıqlań: $\rho=\frac{12}{2-\cos\theta}$.\\

B1. Ellips $3x^{2} + 4y^{2} - 12 = 0$ teńlemesi menen berilgen. Onıń kósherleriniń uzınlıqların, fokuslarınıń koordinataların hám ekscentrisitetin tabıń.  \\

B2. $3x + 10y - 25 = 0$ tuwrı menen $\frac{x^{2}}{25} + \frac{y^{2}}{4} = 1$ ellipstiń kesilisiw noqatların tabıń.\\

B3. $\rho = \frac{6}{1 - cos\theta}$ polyar teńlemesi menen qanday sızıq berilgenin anıqlań.  \\

C1. Eger waqıttıń qálegen momentinde $M(x;y)$ noqat $5x - 16 = 0$ tuwrı sızıqqa qaraǵanda $A(5;0)$ noqattan 1,25 márte uzaqlıqta jaylasqan. Usı $M(x;y)$ noqattıń háreketiniń teńlemesin dúziń.  \\

C2. $4x^{2} - 4xy + y^{2} - 2x - 14y + 7 = 0$ ETİS teńlemesin ápiwayı túrge alıp keliń, tipin anıqlań, qanday geometriyalıq obrazdı anıqlaytuǵının kórsetiń, sızılmasın góne hám taza koordinatalar sistemasına qarata jasań.  \\

C3. $\frac{x^{2}}{3} - \frac{y^{2}}{5} = 1$, giperbolasına $P(4;2)$ noqatınan júrgizilgen urınbalardıń teńlemesin dúziń.  \\

\end{tabular}
\vspace{1cm}


\begin{tabular}{m{17cm}}
\textbf{86-variant}
\newline

T1. Giperbolalıq paraboloydtıń tuwrı sızıqlı jasawshıları (Giperbolalıq paraboloydtı jasawshı tuwrı sızıqlar dástesi).\\

T2. Ellipstiń polyar koordinatalardaǵı teńlemesi (polyar koordinatalar sistemasında ellipstiń teńlemesi).\\

A1. Tipin anıqlań: $3 x^{2}-8 xy+7 y^{2}+8 x-15 y+20=0$.\\

A2. Sheńber teńlemesin dúziń: $M_1 (-1;5) $, $M_2 (-2;-2) $ i $M_3 (5;5) $ noqatlardan ótedi.\\

A3. Fokusları abscissa kósherinde hám koordinata basına qarata simmetriyalıq jaylasqan ellipstiń teńlemesin dúziń: kishi kósheri $6$, direktrisaları arasındaǵı aralıq $13$.\\

B1. $x^{2} - y^{2} = 27$ giperbolasına $4x + 2y - 7 = 0$ tuwrısına parallel bolǵan urınbanıń teńlemesin tabıń.  \\

B2. Koordinata kósherlerin túrlendirmey ETİS ulıwma teńlemesin ápiwaylastırıń, yarım kósherlerin tabıń: $13x^{2} + 18xy + 37y^{2} - 26x - 18y + 3 = 0$.  \\

B3. $x^{2} - 4y^{2} = 16$ giperbola berilgen. Onıń ekscentrisitetin, fokuslarınıń koordinataların tabıń hám asimptotalarınıń teńlemelerin dúziń.\\

C1. $M(2; - \frac{5}{3})$ noqatı $\frac{x^{2}}{9} + \frac{y^{2}}{5} = 1$ ellipsinde jaylasqan. $M$ noqatınıń fokal radiusları jatıwshı tuwrı sızıq teńlemelerin dúziń.  \\

C2. Fokusı $F( - 1; - 4)$ noqatında jaylasqan, sáykes direktrisası $x - 2 = 0$ teńlemesi menen berilgen, $A( - 3; - 5)$ noqatınan ótiwshi ellipstiń teńlemesin dúziń.  \\

C3. $32x^{2} + 52xy - 9y^{2} + 180 = 0$ ETİS teńlemesin ápiwaylastırıń, tipin anıqlań, qanday geometriyalıq obrazdı anıqlaytuǵının kórsetiń, sızılmasın sızıń.  \\

\end{tabular}
\vspace{1cm}


\begin{tabular}{m{17cm}}
\textbf{87-variant}
\newline

T1. ETIS-tıń ulıwma teńlemesin koordinata kósherlerin burıw arqalı ápiwaylastırıń (ETIS-tıń ulıwma teńlemeleri, koordinata kósherin burıw formulası, teńlemeni kanonik túrge alıp keliw).\\

T2. Cilindrlik betlikler (jasawshı tuwrı sızıq, baǵıtlawshı iymek sızıq, cilindrlik betlik).\\

A1. Giperbola teńlemesi berilgen: $\frac{x^{2}}{25}-\frac{y^{2}}{144}=1$. Onıń polyar teńlemesin dúziń.\\

A2. Tipin anıqlań: $x^{2}-4 xy+4 y^{2}+7 x-12=0$.\\

A3. Sheńberdiń $C$ orayı hám $R$ radiusın tabıń: $x^2+y^2+6 x-4 y+14=0$.\\

B1. $y^{2} = 3x$ parabolası menen $\frac{x^{2}}{100} + \frac{y^{2}}{225} = 1$ ellipsiniń kesilisiw noqatların tabıń.  \\

B2. $\rho = \frac{144}{13 - 5cos\theta}$ ellipsti anıqlaytuǵının kórsetiń hám onıń yarım kósherlerin anıqlań.\\

B3. $\frac{x^{2}}{4} - \frac{y^{2}}{5} = 1$ giperbolaǵa $3x - 2y = 0$ tuwrısına parallel bolǵan urınbanıń teńlemesin dúziń.  \\

C1. $A(\frac{10}{3};\frac{5}{3})$ noqattan $\frac{x^{2}}{20} + \frac{y^{2}}{5} = 1$ ellipsine júrgizilgen urınbalardıń teńlemesin dúziń.  \\

C2. Fokusı $F( - 1; - 4)$noqatında bolǵan, sáykes direktrissası $x - 2 = 0$ teńlemesi menen berilgen $A( - 3; - 5)$ noqatınan ótiwshi ellipstiń teńlemesin dúziń.  \\

C3. $2x^{2} + 3y^{2} + 8x - 6y + 11 = 0$ teńlemesin ápiwaylastırıń qanday geometriyalıq obrazdı anıqlaytuǵının tabıń hám grafigin jasań.  \\

\end{tabular}
\vspace{1cm}


\begin{tabular}{m{17cm}}
\textbf{88-variant}
\newline

T1. Parabolanıń urınbasınıń teńlemesi (parabola, tuwrı, urınıw noqatı, urınba teńlemesi).\\

T2. Bir gewekli giperboloid. Kanonikalıq teńlemesi (giperbolanı simmetriya kósheri átirapında aylandırıwdan alınǵan betlik).\\

A1. Uchı koordinata basında jaylasqan hám $Oy$ kósherine qarata shep táreptegi yarım tegislikte jaylasqan parabolanıń teńlemesin dúziń: parametri $p=0,5$.\\

A2. Parabola teńlemesi berilgen: $y^2=6 x$. Onıń polyar teńlemesin dúziń.\\

A3. Berilgen sızıqlardıń oraylıq ekenligin kórsetiń hám orayın tabıń: $2 x^{2}-6 xy+5 y^{2}+22 x-36 y+11=0$.\\

B1. Koordinata kósherlerin túrlendirmey ETİS teńlemesin ápiwaylastırıń, yarım kósherlerin tabıń $4x^{2} - 4xy + 9y^{2} - 26x - 18y + 3 = 0$.\\

B2. $3x + 4y - 12 = 0$ tuwrı sızıǵı hám $y^{2} = - 9x$ parabolasınıń kesilisiw noqatların tabıń.  \\

B3. $\rho = \frac{10}{2 - cos\theta}$ polyar teńlemesi menen qanday sızıq berilgenin anıqlań.  \\

C1. Eger qálegen waqıt momentinde $M(x;y)$ noqat $A(8;4)$ noqattan hám ordinata kósherinen birdey aralıqta jaylassa, $M(x;y)$ noqatınıń háreket etiw troektoriyasınıń teńlemesin dúziń.  \\

C2. $2x^{2} + 3y^{2} + 8x - 6y + 11 = 0$ teńlemesin ápiwaylastırıń qanday geometriyalıq obrazdı anıqlaytuǵının tabıń hám grafigin jasań.\\

C3. Giperbolanıń ekscentrisiteti $\varepsilon = \frac{13}{12}$, fokusı $F(0;13)$ noqatında hám sáykes direktrisası $13y - 144 = 0$ teńlemesi menen berilgen bolsa, giperbolanıń teńlemesin dúziń.  \\

\end{tabular}
\vspace{1cm}


\begin{tabular}{m{17cm}}
\textbf{89-variant}
\newline

T1. Giperbolanıń polyar koordinatadaǵı teńlemesi (Polyar múyeshi, polyar radiusi giperbolanıń polyar teńlemesi).\\

T2. Betliktiń kanonikalıq teńlemeleri. Betlik haqqında túsinik. (Betliktiń anıqlaması, formulaları, kósher, baǵıtlawshı tuwrılar).\\

A1. Sheńber teńlemesin dúziń: orayı $C (2;-3) $ noqatında jaylasqan hám radiusı $R=7$ ge teń.\\

A2. Fokusları abscissa kósherinde hám koordinata basına qarata simmetriyalıq jaylasqan ellipstiń teńlemesin dúziń: kishi kósheri $10$, ekscentrisitet $\varepsilon=12/13$.\\

A3. Giperbola teńlemesi berilgen: $\frac{x^{2}}{16}-\frac{y^{2}}{9}=1$. Onıń polyar teńlemesin dúziń.\\

B1. $\frac{x^{2}}{16} - \frac{y^{2}}{64} = 1$, giperbolasına berilgen $10x - 3y + 9 = 0$ tuwrı sızıǵına parallel bolǵan urınbanıń teńlemesin dúziń.  \\

B2. ETİS-tıń ulıwma teńlemesin koordinata sistemasın túrlendirmey ápiwaylastırıń, tipin anıqlań, obrazı qanday sızıqtı anıqlaytuǵının kórsetiń. $7x^{2} - 8xy + y^{2} - 16x - 2y - 51 = 0$  \\

B3. $\rho = \frac{5}{3 - 4cos\theta}$ teńlemesi menen qanday sızıq berilgenin hám yarım kósherlerin tabıń.  \\

C1. $4x^{2} - 4xy + y^{2} - 6x + 8y + 13 = 0$ ETİS-ǵı orayǵa iyeme? Orayǵa iye bolsa orayın anıqlań: jalǵız orayǵa iyeme-?, sheksiz orayǵa iyeme-?  \\

C2. Tóbesi $A(-4;0)$ noqatında, al, direktrisası $y - 2 = 0$ tuwrı sızıq bolǵan parabolanıń teńlemesin dúziń.\\

C3. $16x^{2} - 9y^{2} - 64x - 54y - 161 = 0$ teńlemesi giperbolanıń teńlemesi ekenin anıqlań hám onıń orayı $C$, yarım kósherleri, ekscentrisitetin, asimptotalarınıń teńlemelerin dúziń.  \\

\end{tabular}
\vspace{1cm}


\begin{tabular}{m{17cm}}
\textbf{90-variant}
\newline

T1. Giperbolanıń urınbasınıń teńlemesi (giperbolaǵa berilgen noqatta júrgizilgen urınba teńlemesi).\\

T2. Ekinshi tártipli betliktiń ulıwma teńlemesi. Orayın anıqlaw formulası.\\

A1. Tipin anıqlań: $25 x^{2}-20 xy+4 y^{2}-12 x+20 y-17=0$.\\

A2. Sheńber teńlemesin dúziń: sheńber $A (2;6 ) $ noqatınan ótedi hám orayı $C (-1;2) $ noqatında jaylasqan .\\

A3. Fokusları abscissa kósherinde hám koordinata basına qarata simmetriyalıq jaylasqan giperbolanıń teńlemesin dúziń: oqları $2 a=10$ hám $2 b=8$.\\

B1. $2x + 2y - 3 = 0$ tuwrısına parallel bolıp $\frac{x^{2}}{16} + \frac{y^{2}}{64} = 1$ giperbolasına urınıwshı tuwrınıń teńlemesin dúziń.  \\

B2. $41x^{2} + 24xy + 9y^{2} + 24x + 18y - 36 = 0$ ETİS tipin anıqlań hám orayların tabıń koordinata kósherlerin túrlendirmey qanday sızıqtı anıqlaytuǵının kórsetiń yarım kósherlerin tabıń.  \\

B3. $2x + 2y - 3 = 0$ tuwrısına perpendikulyar bolıp $x^{2} = 16y$ parabolasına urınıwshı tuwrınıń teńlemesin dúziń.  \\

C1. Fokuslari $F(3;4), F(-3;-4)$ noqatlarında jaylasqan direktrisaları orasıdaǵı aralıq 3,6 ǵa teń bolǵan giperbolanıń teńlemesin dúziń.  \\

C2. $14x^{2} + 24xy + 21y^{2} - 4x + 18y - 139 = 0$ iymek sızıǵınıń tipin anıqlań, eger oraylı iymek sızıq bolsa orayınıń koordinataların tabıń.  \\

C3. $4x^{2} + 24xy + 11y^{2} + 64x + 42y + 51 = 0$ iymek sızıǵınıń tipin anıqlań eger orayı bar bolsa, onıń orayınıń koordinataların tabıń hám koordinata basın orayǵa parallel kóshiriw ámelin orınlań.  \\

\end{tabular}
\vspace{1cm}


\begin{tabular}{m{17cm}}
\textbf{91-variant}
\newline

T1. Ellips hám onıń kanonikalıq teńlemesi (anıqlaması, fokuslar, ellipstiń kanonikalıq teńlemesi, ekscentrisiteti, direktrisaları).\\

T2. Ekinshi tártipli aylanba betlikler (koordinata sisteması, tegislik, vektor iymek sızıq, aylanba betlik).\\

A1. Polyar teńlemesi menen berilgen iymek sızıqtıń tipin anıqlań: $\rho=\frac{5}{3-4\cos\theta}$.\\

A2. Tipin anıqlań: $3 x^{2}-2 xy-3 y^{2}+12 y-15=0$.\\

A3. Sheńberdiń $C$ orayı hám $R$ radiusın tabıń: $x^2+y^2-2 x+4 y-20=0$.\\

B1. Koordinata kósherlerin túrlendirmey ETİS teńlemesin ápiwaylastırıń, qanday geometriyalıq obrazdı anıqlaytuǵının kórsetiń $4x^{2} - 4xy + y^{2} + 4x - 2y + 1 = 0$.  \\

B2. $y^{2} = 12x$ paraborolasına $3x - 2y + 30 = 0$ tuwrı sızıǵına parallel bolǵan urınbanıń teńlemesin dúziń.  \\

B3. $\frac{x^{2}}{4} - \frac{y^{2}}{5} = 1$, giperbolanıń $3x - 2y = 0$ tuwrı sızıǵına parallel bolǵan urınbasınıń teńlemesin dúziń.  \\

C1. Úlken kósheri 26 ǵa, fokusları $F( - 10;0)$, $F(14;0)$ noqatlarında jaylasqan ellipstiń teńlemesin dúziń.  \\

C2. $32x^{2} + 52xy - 7y^{2} + 180 = 0$ ETİS teńlemesin ápiwayı túrge alıp keliń, tipin anıqlań, qanday geometriyalıq obrazdı anıqlaytuǵının kórsetiń, sızılmasın góne hám taza koordinatalar sistemasına qarata jasań.  \\

C3. $\frac{x^{2}}{3} - \frac{y^{2}}{5} = 1$ giperbolasına $P(1; - 5)$ noqatında júrgizilgen urınbalardıń teńlemesin dúziń.\\

\end{tabular}
\vspace{1cm}


\begin{tabular}{m{17cm}}
\textbf{92-variant}
\newline

T1. Parabolanıń polyar koordinatalardaǵı teńlemesi (polyar koordinata sistemasında parabolanıń teńlemesi).\\

T2. ETIS-tıń ulıwma teńlemesin klassifikatsiyalaw (ETIS-tıń ulıwma teńlemesi, ETIS-tıń ulıwma teńlemesin ápiwaylastırıw, klassifikatsiyalaw).\\

A1. Fokusları abscissa kósherinde hám koordinata basına qarata simmetriyalıq jaylasqan ellipstiń teńlemesin dúziń: yarım oqları 5 hám 2.\\

A2. Polyar teńlemesi menen berilgen iymek sızıqtıń tipin anıqlań: $\rho=\frac{1}{3-3\cos\theta}$.\\

A3. Tipin anıqlań: $2 x^{2}+3 y^{2}+8 x-6 y+11=0$.\\

B1. $\frac{x^{2}}{4} - \frac{y^{2}}{5} = 1$ giperbolasına $3x + 2y = 0$ tuwrı sızıǵına perpendikulyar bolǵan urınba tuwrınıń teńlemesin dúziń.\\

B2. Ellips $3x^{2} + 4y^{2} - 12 = 0$ teńlemesi menen berilgen. Onıń kósherleriniń uzınlıqların, fokuslarınıń koordinataların hám ekscentrisitetin tabıń.  \\

B3. $x^{2} + 4y^{2} = 25$ ellipsi menen $4x - 2y + 23 = 0$ tuwrı sızıǵına parallel bolǵan urınba tuwrı sızıqtıń teńlemesin dúziń.  \\

C1. $\frac{x^{2}}{100} + \frac{y^{2}}{36} = 1$ ellipsiniń oń jaqtaǵı fokusınan 14 ge teń aralıqta bolǵan noqattı tabıń.  \\

C2. Fokusı $F(2; - 1)$ noqatında jaylasqan, sáykes direktrisası $x - y - 1 = 0$ teńlemesi menen berilgen parabolanıń teńlemesin dúziń.  \\

C3. $2x^{2} + 10xy + 12y^{2} - 7x + 18y - 15 = 0$ ETİS teńlemesin ápiwayı túrge alıp keliń, tipin anıqlań, qanday geometriyalıq obrazdı anıqlaytuǵının kórsetiń, sızılmasın góne hám taza koordinatalar sistemasına qarata jasań  \\

\end{tabular}
\vspace{1cm}


\begin{tabular}{m{17cm}}
\textbf{93-variant}
\newline

T1. Eki gewekli giperboloid. Kanonikalıq teńlemesi (giperbolanı simmetriya kósheri átirapında aylandırıwdan alınǵan betlik).\\

T2. Ellipstiń urınbasınıń teńlemesi (ellips, tuwrı, urınıw tochka, urınba teńlemesi).\\

A1. Sheńber teńlemesin dúziń: orayı koordinata basında jaylasqan hám radiusı $R=3$ ge teń.\\

A2. Uchı koordinata basında jaylasqan hám $Oy$ kósherine qarata oń táreptegi yarım tegislikte jaylasqan parabolanıń teńlemesin dúziń: parametri $p=3$.\\

A3. Polyar teńlemesi menen berilgen iymek sızıqtıń tipin anıqlań: $\rho=\frac{10}{1-\frac{3}{2}\cos\theta}$.\\

B1. Koordinata kósherlerin túrlendirmey ETİS teńlemesin ápiwaylastırıń, yarım kósherlerin tabıń $41x^{2} + 2xy + 9y^{2} - 26x - 18y + 3 = 0$.  \\

B2. $x^{2} - 4y^{2} = 16$ giperbola berilgen. Onıń ekscentrisitetin, fokuslarınıń koordinataların tabıń hám asimptotalarınıń teńlemelerin dúziń.\\

B3. $3x + 10y - 25 = 0$ tuwrı menen $\frac{x^{2}}{25} + \frac{y^{2}}{4} = 1$ ellipstiń kesilisiw noqatların tabıń.\\

C1. $\frac{x^{2}}{2} + \frac{y^{2}}{3} = 1$, ellipsin $x + y - 2 = 0$ noqatınan júrgizilgen urınbalarınıń teńlemesin dúziń.  \\

C2. $y^{2} = 20x$ parabolasınıń $M$ noqatın tabıń, eger onıń abscissası 7 ge teń bolsa, fokal radiusın hám fokal radius jaylasqan tuwrını anıqlań.\\

C3. Fokusı $F(7;2)$ noqatında jaylasqan, sáykes direktrisası $x - 5 = 0$ teńlemesi menen berilgen parabolanıń teńlemesin dúziń.  \\

\end{tabular}
\vspace{1cm}


\begin{tabular}{m{17cm}}
\textbf{94-variant}
\newline

T1. ETIS -tiń ulıwma teńlemesin ápiwaylastırıw (ETIS -tiń ulıwma teńlemesi, koordinata sistemasın túrlendirip ETIS ulıwma teńlemesin ápiwaylastırıw).\\

T2. Ellipslik paraboloid (parabola, kósher, ellipslik paraboloid).\\

A1. Berilgen sızıqlardıń oraylıq ekenligin kórsetiń hám orayın tabıń: $3 x^{2}+5 xy+y^{2}-8 x-11 y-7=0$.\\

A2. Sheńber teńlemesin dúziń: orayı $C (1;-1) $ noqatında jaylasqan hám $5 x-12 y+9 -0$ tuwrı sızıǵına urınadı .\\

A3. Fokusları abscissa kósherinde hám koordinata basına qarata simmetriyalıq jaylasqan ellipstiń teńlemesin dúziń: úlken kósheri $8$, direktrisaları arasındaǵı aralıq $16$.\\

B1. $\rho = \frac{6}{1 - cos\theta}$ polyar teńlemesi menen qanday sızıq berilgenin anıqlań.  \\

B2. $\frac{x^{2}}{20} - \frac{y^{2}}{5} = 1$ giperbolasına $4x + 3y - 7 = 0$ tuwrısına perpendikulyar bolǵan urınbanıń teńlemesin dúziń.  \\

B3. Koordinata kósherlerin túrlendirmey ETİS ulıwma teńlemesin ápiwaylastırıń, yarım kósherlerin tabıń: $13x^{2} + 18xy + 37y^{2} - 26x - 18y + 3 = 0$.  \\

C1. $2x^{2} + 3y^{2} + 8x - 6y + 11 = 0$ teńlemesi menen qanday tiptegi sızıq berilgenin anıqlań hám onıń teńlemesin ápiwaylastırıń hám grafigin jasań.  \\

C2. $\frac{x^{2}}{25} + \frac{y^{2}}{16} = 1$, ellipsine $C(10; - 8)$ noqatınan júrgizilgen urınbalarınıń teńlemesin dúziń.  \\

C3. $y^{2} = 20x$ parabolasınıń abscissası 7 ge teń bolǵan $M$ noqatınıń fokal radiusın tabıń hám fokal radiusı jatqan tuwrınıń teńlemesin dúziń.  \\

\end{tabular}
\vspace{1cm}


\begin{tabular}{m{17cm}}
\textbf{95-variant}
\newline

T1. Koordinata sistemasın túrlendiriw (birlik vektorlar, kósherler, parallel kóshiriw, koordinata kósherlerin burıw).\\

T2. ETIS-tıń invariantları (ETIS-tıń ulıwma teńlemesi, túrlendiriw, ETIS invariantları ).\\

A1. Tipin anıqlań: $2 x^{2}+10 xy+12 y^{2}-7 x+18 y-15=0$.\\

A2. Sheńber teńlemesin dúziń: orayı koordinata basında jaylasqan hám $3 x-4 y+20=0$ tuwrı sızıǵına urınadı.\\

A3. Fokusları abscissa kósherinde hám koordinata basına qarata simmetriyalıq jaylasqan ellipstiń teńlemesin dúziń: úlken kósheri $20$, ekscentrisitet $\varepsilon=3/5$.\\

B1. Ellips $3x^{2} + 4y^{2} - 12 = 0$ teńlemesi menen berilgen. Onıń kósherleriniń uzınlıqların, fokuslarınıń koordinataların hám ekscentrisitetin tabıń.  \\

B2. $y^{2} = 3x$ parabolası menen $\frac{x^{2}}{100} + \frac{y^{2}}{225} = 1$ ellipsiniń kesilisiw noqatların tabıń.  \\

B3. $\rho = \frac{144}{13 - 5cos\theta}$ ellipsti anıqlaytuǵının kórsetiń hám onıń yarım kósherlerin anıqlań.\\

C1. Eger waqıttıń qálegen momentinde $M(x;y)$ noqat $5x - 16 = 0$ tuwrı sızıqqa qaraǵanda $A(5;0)$ noqattan 1,25 márte uzaqlıqta jaylasqan. Usı $M(x;y)$ noqattıń háreketiniń teńlemesin dúziń.  \\

C2. $4x^{2} - 4xy + y^{2} - 2x - 14y + 7 = 0$ ETİS teńlemesin ápiwayı túrge alıp keliń, tipin anıqlań, qanday geometriyalıq obrazdı anıqlaytuǵının kórsetiń, sızılmasın góne hám taza koordinatalar sistemasına qarata jasań.  \\

C3. $\frac{x^{2}}{3} - \frac{y^{2}}{5} = 1$, giperbolasına $P(4;2)$ noqatınan júrgizilgen urınbalardıń teńlemesin dúziń.  \\

\end{tabular}
\vspace{1cm}


\begin{tabular}{m{17cm}}
\textbf{96-variant}
\newline

T1. Betlik haqqında túsinik (tuwrı, iymek sızıq, betliktiń anıqlamaları hám formulaları).\\

T2. Parabola hám onıń kanonikalıq teńlemesi (anıqlaması, fokusı, direktrisası, kanonikalıq teńlemesi).\\

A1. Tipin anıqlań: $4 x^{2}-y^{2}+8 x-2 y+3=0$.\\

A2. Sheńber teńlemesin dúziń: sheńber diametriniń ushları $A (3;2) $ hám $B (-1;6 ) $ noqatlarında jaylasqan.\\

A3. Fokusları abscissa kósherinde hám koordinata basına qarata simmetriyalıq jaylasqan giperbolanıń teńlemesin dúziń: direktrisaları arasındaǵı aralıq $228/13$ hám fokusları arasındaǵı aralıq $2 c=26$.\\

B1. $x^{2} - y^{2} = 27$ giperbolasına $4x + 2y - 7 = 0$ tuwrısına parallel bolǵan urınbanıń teńlemesin tabıń.  \\

B2. Koordinata kósherlerin túrlendirmey ETİS teńlemesin ápiwaylastırıń, yarım kósherlerin tabıń $4x^{2} - 4xy + 9y^{2} - 26x - 18y + 3 = 0$.\\

B3. $3x + 4y - 12 = 0$ tuwrı sızıǵı hám $y^{2} = - 9x$ parabolasınıń kesilisiw noqatların tabıń.  \\

C1. $M(2; - \frac{5}{3})$ noqatı $\frac{x^{2}}{9} + \frac{y^{2}}{5} = 1$ ellipsinde jaylasqan. $M$ noqatınıń fokal radiusları jatıwshı tuwrı sızıq teńlemelerin dúziń.  \\

C2. Fokusı $F( - 1; - 4)$ noqatında jaylasqan, sáykes direktrisası $x - 2 = 0$ teńlemesi menen berilgen, $A( - 3; - 5)$ noqatınan ótiwshi ellipstiń teńlemesin dúziń.  \\

C3. $32x^{2} + 52xy - 9y^{2} + 180 = 0$ ETİS teńlemesin ápiwaylastırıń, tipin anıqlań, qanday geometriyalıq obrazdı anıqlaytuǵının kórsetiń, sızılmasın sızıń.  \\

\end{tabular}
\vspace{1cm}


\begin{tabular}{m{17cm}}
\textbf{97-variant}
\newline

T1. ETIS-tıń orayın anıqlaw forması (ETIS-tıń ulıwma teńlemesi, orayın anıqlaw forması).\\

T2. Ellipsoida. Kanonikalıq teńlemesi (ellipsti simmetriya kósheri dogereginde aylandırıwdan alınǵan betlik, kanonikalıq teńlemesi).\\

A1. Tipin anıqlań: $9 x^{2}-16 y^{2}-54 x-64 y-127=0$.\\

A2. Sheńber teńlemesin dúziń: $A (3;1) $ hám $B (-1;3) $ noqatlardan ótedi, orayı $3 x-y-2=0$ tuwrı sızıǵında jaylasqan .\\

A3. Uchı koordinata basında jaylasqan hám $Ox$ kósherine qarata tómengi yarım tegislikte jaylasqan parabolanıń teńlemesin dúziń: parametri $p=3$.\\

B1. $\rho = \frac{10}{2 - cos\theta}$ polyar teńlemesi menen qanday sızıq berilgenin anıqlań.  \\

B2. $\frac{x^{2}}{4} - \frac{y^{2}}{5} = 1$ giperbolaǵa $3x - 2y = 0$ tuwrısına parallel bolǵan urınbanıń teńlemesin dúziń.  \\

B3. ETİS-tıń ulıwma teńlemesin koordinata sistemasın túrlendirmey ápiwaylastırıń, tipin anıqlań, obrazı qanday sızıqtı anıqlaytuǵının kórsetiń. $7x^{2} - 8xy + y^{2} - 16x - 2y - 51 = 0$  \\

C1. $A(\frac{10}{3};\frac{5}{3})$ noqattan $\frac{x^{2}}{20} + \frac{y^{2}}{5} = 1$ ellipsine júrgizilgen urınbalardıń teńlemesin dúziń.  \\

C2. Fokusı $F( - 1; - 4)$noqatında bolǵan, sáykes direktrissası $x - 2 = 0$ teńlemesi menen berilgen $A( - 3; - 5)$ noqatınan ótiwshi ellipstiń teńlemesin dúziń.  \\

C3. $2x^{2} + 3y^{2} + 8x - 6y + 11 = 0$ teńlemesin ápiwaylastırıń qanday geometriyalıq obrazdı anıqlaytuǵının tabıń hám grafigin jasań.  \\

\end{tabular}
\vspace{1cm}


\begin{tabular}{m{17cm}}
\textbf{98-variant}
\newline

T1. Giperbola. Kanonikalıq teńlemesi (fokuslar, kósherler, direktrisalar, giperbola, ekscentrisitet, kanonikalıq teńlemesi).\\

T2. ETIS-tıń ulıwma teńlemesin koordinata basın parallel kóshiriw arqalı ápiwayılastırıń (ETIS- tıń ulıwma teńlemesin parallel kóshiriw formulası).\\

A1. Tipin anıqlań: $5 x^{2}+14 xy+11 y^{2}+12 x-7 y+19=0$.\\

A2. Fokusları abscissa kósherinde hám koordinata basına qarata simmetriyalıq jaylasqan giperbolanıń teńlemesin dúziń: direktrisaları arasındaǵı aralıq $32/5$ hám kósheri $2 b=6$.\\

A3. Tipin anıqlań: $4 x^2+9 y^2-40 x+36 y+100=0$.\\

B1. $\rho = \frac{5}{3 - 4cos\theta}$ teńlemesi menen qanday sızıq berilgenin hám yarım kósherlerin tabıń.  \\

B2. $\frac{x^{2}}{16} - \frac{y^{2}}{64} = 1$, giperbolasına berilgen $10x - 3y + 9 = 0$ tuwrı sızıǵına parallel bolǵan urınbanıń teńlemesin dúziń.  \\

B3. $41x^{2} + 24xy + 9y^{2} + 24x + 18y - 36 = 0$ ETİS tipin anıqlań hám orayların tabıń koordinata kósherlerin túrlendirmey qanday sızıqtı anıqlaytuǵının kórsetiń yarım kósherlerin tabıń.  \\

C1. Eger qálegen waqıt momentinde $M(x;y)$ noqat $A(8;4)$ noqattan hám ordinata kósherinen birdey aralıqta jaylassa, $M(x;y)$ noqatınıń háreket etiw troektoriyasınıń teńlemesin dúziń.  \\

C2. $2x^{2} + 3y^{2} + 8x - 6y + 11 = 0$ teńlemesin ápiwaylastırıń qanday geometriyalıq obrazdı anıqlaytuǵının tabıń hám grafigin jasań.\\

C3. Giperbolanıń ekscentrisiteti $\varepsilon = \frac{13}{12}$, fokusı $F(0;13)$ noqatında hám sáykes direktrisası $13y - 144 = 0$ teńlemesi menen berilgen bolsa, giperbolanıń teńlemesin dúziń.  \\

\end{tabular}
\vspace{1cm}


\begin{tabular}{m{17cm}}
\textbf{99-variant}
\newline

T1. Giperbolalıq paraboloydtıń tuwrı sızıqlı jasawshıları (Giperbolalıq paraboloydtı jasawshı tuwrı sızıqlar dástesi).\\

T2. Ellipstiń polyar koordinatalardaǵı teńlemesi (polyar koordinatalar sistemasında ellipstiń teńlemesi).\\

A1. Fokusları abscissa kósherinde hám koordinata basına qarata simmetriyalıq jaylasqan ellipstiń teńlemesin dúziń: fokusları arasındaǵı aralıq $2 c=6$ hám ekscentrisitet $\varepsilon=3/5$.\\

A2. Fokusları abscissa kósherinde hám koordinata basına qarata simmetriyalıq jaylasqan giperbolanıń teńlemesin dúziń: fokusları arasındaǵı aralıǵı $2 c=10$ hám kósheri $2 b=8$.\\

A3. Fokusları abscissa kósherinde hám koordinata basına qarata simmetriyalıq jaylasqan ellipstiń teńlemesin dúziń: úlken kósheri $10$, fokusları arasındaǵı aralıq $2 c=8$.\\

B1. $2x + 2y - 3 = 0$ tuwrısına parallel bolıp $\frac{x^{2}}{16} + \frac{y^{2}}{64} = 1$ giperbolasına urınıwshı tuwrınıń teńlemesin dúziń.  \\

B2. Koordinata kósherlerin túrlendirmey ETİS teńlemesin ápiwaylastırıń, qanday geometriyalıq obrazdı anıqlaytuǵının kórsetiń $4x^{2} - 4xy + y^{2} + 4x - 2y + 1 = 0$.  \\

B3. $2x + 2y - 3 = 0$ tuwrısına perpendikulyar bolıp $x^{2} = 16y$ parabolasına urınıwshı tuwrınıń teńlemesin dúziń.  \\

C1. $4x^{2} - 4xy + y^{2} - 6x + 8y + 13 = 0$ ETİS-ǵı orayǵa iyeme? Orayǵa iye bolsa orayın anıqlań: jalǵız orayǵa iyeme-?, sheksiz orayǵa iyeme-?  \\

C2. Tóbesi $A(-4;0)$ noqatında, al, direktrisası $y - 2 = 0$ tuwrı sızıq bolǵan parabolanıń teńlemesin dúziń.\\

C3. $16x^{2} - 9y^{2} - 64x - 54y - 161 = 0$ teńlemesi giperbolanıń teńlemesi ekenin anıqlań hám onıń orayı $C$, yarım kósherleri, ekscentrisitetin, asimptotalarınıń teńlemelerin dúziń.  \\

\end{tabular}
\vspace{1cm}


\begin{tabular}{m{17cm}}
\textbf{100-variant}
\newline

T1. ETIS-tıń ulıwma teńlemesin koordinata kósherlerin burıw arqalı ápiwaylastırıń (ETIS-tıń ulıwma teńlemeleri, koordinata kósherin burıw formulası, teńlemeni kanonik túrge alıp keliw).\\

T2. Cilindrlik betlikler (jasawshı tuwrı sızıq, baǵıtlawshı iymek sızıq, cilindrlik betlik).\\

A1. Fokusları abscissa kósherinde hám koordinata basına qarata simmetriyalıq jaylasqan ellipstiń teńlemesin dúziń: kishi kósheri $24$, fokusları arasındaǵı aralıq $2 c=10$.\\

A2. Fokusları abscissa kósherinde hám koordinata basına qarata simmetriyalıq jaylasqan giperbolanıń teńlemesin dúziń: fokusları arasındaǵı aralıq $2 c=6$ hám ekscentrisitet $\varepsilon=3/2$.\\

A3. Fokusları abscissa kósherinde hám koordinata basına qarata simmetriyalıq jaylasqan giperbolanıń teńlemesin dúziń: direktrisaları arasındaǵı aralıq $8/3$ hám ekscentrisitet $\varepsilon=3/2$.\\

B1. $y^{2} = 12x$ paraborolasına $3x - 2y + 30 = 0$ tuwrı sızıǵına parallel bolǵan urınbanıń teńlemesin dúziń.  \\

B2. $\frac{x^{2}}{4} - \frac{y^{2}}{5} = 1$, giperbolanıń $3x - 2y = 0$ tuwrı sızıǵına parallel bolǵan urınbasınıń teńlemesin dúziń.  \\

B3. $x^{2} - 4y^{2} = 16$ giperbola berilgen. Onıń ekscentrisitetin, fokuslarınıń koordinataların tabıń hám asimptotalarınıń teńlemelerin dúziń.\\

C1. Fokuslari $F(3;4), F(-3;-4)$ noqatlarında jaylasqan direktrisaları orasıdaǵı aralıq 3,6 ǵa teń bolǵan giperbolanıń teńlemesin dúziń.  \\

C2. $14x^{2} + 24xy + 21y^{2} - 4x + 18y - 139 = 0$ iymek sızıǵınıń tipin anıqlań, eger oraylı iymek sızıq bolsa orayınıń koordinataların tabıń.  \\

C3. $4x^{2} + 24xy + 11y^{2} + 64x + 42y + 51 = 0$ iymek sızıǵınıń tipin anıqlań eger orayı bar bolsa, onıń orayınıń koordinataların tabıń hám koordinata basın orayǵa parallel kóshiriw ámelin orınlań.  \\

\end{tabular}
\vspace{1cm}



\end{document}
