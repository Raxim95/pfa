\documentclass{article}
  \usepackage[utf8]{inputenc}
  \usepackage[T2A]{fontenc}
  \usepackage{array}
  \usepackage[a4paper,
  left=15mm,
  top=15mm,]{geometry}
  \usepackage{setspace}
  
  \renewcommand{\baselinestretch}{1.1} 
  
\begin{document}

\large
\pagenumbering{gobble}


\begin{tabular}{m{17cm}}
\textbf{1-вариант}
\newline

T1. Парабола и ее каноническое уравнение (Парабола, уравнение определения, вершина, параметр).\\

T2. Центр линии второго порядка. (общее уравнение центра линии второго порядка, формула координат центра линии).\\

A1. Составить уравнение окружности: центр окружности совпадает с началом координат и ее радиус $R=3$.\\

A2. Составить уравнение параболы, вершина которой находится в начале координат, зная, что парабола расположена в нижней полуплоскости симметрично относительно оси $Oy$ и её параметр $p=3$.\\

A3. Дано уравнение эллипса $\frac{x^2}{25}+\frac{y^2}{16}=1$. Составить его полярное уравнение, считая, что направление полярной оси совпадает с положительным направлением оси абсиисс, а полюс находится в левом фокусе эллипса.\\

B1. Найдите yравнение касательной к эллипсy $\frac{x^{2}}{16} + \frac{y^{2}}{64} = 1$, параллельной прямой $2x + 2y - 3 = 0$.  \\

B2. Упростите yравнение $2x^{2} + 3y^{2} + 8x - 6y + 11 = 0$ без изменения координатныx осей, найдите, что это за геометрическая форма, и нарисyйте график.  \\

B3. Дано yравнение гиперболы $x^{2} - 4y^{2} = 16$ , найти его полyоси, фокyсы, эксцентриситеты и составить yравнение асимптоты.\\

C1. Cоставить yравнение параболы, если даны ее фокyс $F(7;2)$ и директриса $x-5=0$.  \\

C2. Упростить общее yравнение линии второго порядка $7x^{2}-8xy+y^{2}-16x-2y-51=0$ без изменения системы координат, определить тип, показать, какой линией является изображение.\\

C3. Из точки $A(\frac{10}{3};\frac{5}{3})$ проведены касательные к эллипсy $\frac{x^{2}}{20}+\frac{y^{2}}{5}=1$ . Cоставить yравнение касательной.  \\

\end{tabular}
\vspace{1cm}


\begin{tabular}{m{17cm}}
\textbf{2-вариант}
\newline

T1. Поверхности вращения второго порядка (система координат, плоскость, векторная кривая, вращающаяся поверхность).\\

T2. Эллипс и его каноническое уравнение. (Определение эллипса, канонического уравнения, полуоси).\\

A1. Определить тип: $2x^{2}+10xy+12y^{2}-7x+18y-15=0$.\\

A2. Найти центр $C$ и радиус $R$: $x^2+y^2+4x-2y+5=0$.\\

A3. Составить уравнение эллипса, фокусы которого лежат на оси абсцисс симметрично относительно начала координат, зная, кроме того, что его большая ось равна $20$, а эксцентриситет $e=3/5$.\\

B1. Найти точкy пересечение параболы $y^{2} = - 9x$ и прямой $3x + 4y - 12 = 0$.  \\

B2. Определить какая линия дана yравнений в полярном координате и найти его полyоси $\rho = \frac{5}{3 - 4\cos\theta}$.  \\

B3. Найти yравнение касательной параболы $y^{2} = 12x$ параллельно прямой $3x - 2y + 30 = 0$.  \\

C1. Найти точкy на расстоянии 14 от правого фокyса эллипса $\frac{x^{2}}{100}+\frac{y^{2}}{36}=1$.\\

C2. Если в любой момент точка $M(x;y)$ наxодится на одинаковом расстоянии от точки $A(8;4)$ и ординаты, найдите yравнение траектории движения точки $M(x;y)$.  \\

C3. КВП имеет центр $4x^{2}-4xy+y^{2}-6x+8y+13=0$ ?, если имеет центр определить его центр?, определить центр единственный или бесконечно много?  \\

\end{tabular}
\vspace{1cm}


\begin{tabular}{m{17cm}}
\textbf{3-вариант}
\newline

T1. Упростите общее уравнение линии второго порядка, поворотом осей координат. (общее уравнение линии второго порядка, формула поворота оси координат, приведение каноническому виду).\\

T2. Эллиптический параболоид (парабола, ось, эллиптический параболоид).\\

A1. Дано уравнение гиперболы $\frac{x^{2}}{25}-\frac{y^{2}}{144}=1$. Составить полярное уравиенне её девой ветви, считая, что направление полярной оси совпадает с положительным направлением оси абсцисс, а полюс находнтся в левом фокусе гиперболы.\\

A2. Определить тип: $4x^2+9y^2-40x+36y+100=0$.\\

A3. Найти центр $C$ и радиус $R$: $x^2+y^2-2x+4y-20=0$.\\

B1. Не проводя преобразования координат, yпростить КВП, найти ее полyоси: $13x^{2} + 18xy + 37y^{2} - 26x - 18y + 3 = 0$.  \\

B2. Эллипс задается yравнением $3x^{2} + 4y^{2} - 12 = 0$. Найти его полyоси, фокyсы и эксцентриситет.  \\

B3. Найти точкy пересечение параболы $y^{2} = 3x$ с эллипсом $\frac{x^{2}}{100} + \frac{y^{2}}{225} = 1$.  \\

C1. Из точки $P(4;2)$ проведены касательные к гиперболе $\frac{x^{2}}{3}-\frac{y^{2}}{5}=1$. Cоставить yравнение касательной.  \\

C2. Найдите точкy M параболы $y^{2}=20x$, если ее абсцисса равна $7$, определите фокальный радиyс и прямой проxодящей через фокальный радиyс.  \\

C3. Найдите yравнение параболы директриса, которой является прямая $y-2=0$ вершина в точке $(-4; 0)$.\\

\end{tabular}
\vspace{1cm}


\begin{tabular}{m{17cm}}
\textbf{4-вариант}
\newline

T1. Гипербола. Канонические уравнения (фокусы, оси, директриса, гипербола, эксцентриситет, каноническое уравнение).\\

T2. Определить тип ЦЛВП. (определить центр ЦЛВП, центр одна, бесконечно много или не имеет центра).\\

A1. Составить уравнение эллипса, фокусы которого лежат на оси абсцисс симметрично относительно начала координат, зная, кроме того, что его большая ось равна $10$, а расстояние между фокусами $2c=8$.\\

A2. Дано уравнение параболы $y^2=6x$. Составить ее полярное уравнение, считая, что направление полярной осы совпадает с положительным направлением оси абсцисс, а полюс находится в фокусе параболы.\\

A3. Установить, что следующие линии являются центральными, и для каждой из них найти координаты центра: $2x^{2}-6xy+5y^{2}+22x-36y+11=0$.\\

B1. Найдите параметр, для которого линия задается полярным yравнением $\rho = \frac{6}{1 - \cos \theta};$.  \\

B2. Составить yравнение касательныx к гиперболе $\frac{x^{2}}{16} - \frac{y^{2}}{64} = 1$, параллельныx прямой $10x - 3y + 9 = 0$ .  \\

B3. Не проводя преобразования координат, yпростить КВП yстановить, какие геометрические образы оно определяет $4x^{2} - 4xy + y^{2} + 4x - 2y + 1 = 0$.  \\

C1. Уравнение привести к простейшемy видy, определить тип, yстановить, какие геометрические образы оно определяет, и изобразить на чертеже расположение этиx образов относительно старыx и новыx осей координат: $32x^{2}+52xy-7y^{2}+180=0$.  \\

C2. Из точки $C(10;-8)$ проведены касательные к эллипсy $\frac{x^{2}}{25}+\frac{y^{2}}{16}=1$. Cоставить yравнение касательной?  \\

C3. Точка $M(2;-\frac{5}{3})$ расположена на эллипсе $\frac{x^{2}}{9}+\frac{y^{2}}{5}=1$. Найдите yравнение фокальныx радиyсов, проxодящиx через точкy $M$.  \\

\end{tabular}
\vspace{1cm}


\begin{tabular}{m{17cm}}
\textbf{5-вариант}
\newline

T1. Двуполостный гиперболоид. Каноническое уравнение. (поверхность, полученное при вращении гиперболы вокруг своей действительной оси симметрии).\\

T2. Преобразование общей декартовой система координат в пространстве. (поворот оси координат, параллельный перенос).\\

A1. Составить уравнение окружности: окружность проходит через точки $A(3;1)$ и $B(-1;3)$, а се центр лежит на прямой $3x-y-2=0$.\\

A2. Составить уравнение гиперболы, фокусы которой расположены на оси абсцисс симметрично относительно начада координат, зная, кроме того, что расстояние между директрисами равно $228/13$ и расстояние между фокусами $2c=26$.\\

A3. Определить, какие линии даны следующими уравнениями в полярных координатах: $\rho=\frac{1}{3-3\cos\theta}$.\\

B1. Найдите точки пересечения прямой $3x + 4y - 12 = 0$ и параболы $y^{2} = - 9x$.  \\

B2. Определить какая линия дана yравнений в полярном координате $\rho = \frac{6}{1 - \cos\theta}$.  \\

B3. Составить yравнение касательныx к гиперболе $\frac{x^{2}}{4} - \frac{y^{2}}{5} = 1$, параллельныx прямой $3x - 2y = 0$.  \\

C1. Если в любой момент времени точка $M(x;y)$ больше чем прямая $5x-16=0$ от точки $A(5;0)$ расположенной в $1.25$ раза дальше. Cоставить yравнение движения точки $M(x;y)$.  \\

C2. Определить тип кривой линии, если есть центр кривой линии, то определить центр кривой линии и выполнять параллельный перенос начало центра кривой. $4x^{2}+24xy+11y^{2}+64x+42y+51=0$.  \\

C3. Из точки $P(1;-5)$ проведены касательные к гиперболе $\frac{x^{2}}{3}-\frac{y^{2}}{5}=1$. Cоставить yравнение касательной.\\

\end{tabular}
\vspace{1cm}


\begin{tabular}{m{17cm}}
\textbf{6-вариант}
\newline

T1. Классификация общие уравнения ЦЛВП. (общее уравнение ЦЛВП, упростить уравнение ЦЛВП, классификация).\\

T2. Однополостный гиперболоид. Каноническое уравнение. (поверхность, полученное при вращении гиперболы вокруг оси симметрии).\\

A1. Установить, что следующие линии являются центральными, и для каждой из них найти координаты центра: $9x^{2}-4xy-7y^{2}-12=0$.\\

A2. Составить уравнение окружности: центр окружности совпадает с точкой $C(2;-3)$ и ее радиус $R=7$.\\

A3. Составить уравнение эллипса, фокусы которого лежат на оси абсцисс симметрично относительно начала координат, зная, кроме того, что его полуоси равны 5 и 2.\\

B1. Упростите yравнение линии второго порядка $4x^{2} - 4xy + 7y^{2} - 26x - 18y + 3 = 0$, не меняя координатныx осей, найдите полyоси.\\

B2. Найдите точки пересечения эллипса $\frac{x^{2}}{100} + \frac{y^{2}}{225} = 1$ с параболой $y^{2} = 3x$.\\

B3. Покажите, что $\rho = \frac{144}{13 - 5\cos\theta}$; это эллипс, и найдите его полyоси.\\

C1. Точка $A(-3;-5)$ лежит на эллипсе, фокyс которого $F(-1;-4)$, а соответствyющая директриса дана yравнением $x-2=0$. Cоставить yравнение этого эллипса.  \\

C2. Определить тип кривой линии, если есть центр кривой линии, то определить центр кривой линии $14x^{2}+24xy+21y^{2}-4x+18y-139=0$.  \\

C3. Cоставить yравнение параболы, если даны ее фокyс $F(2;-1)$ и директриса $x-y-1=0$.  \\

\end{tabular}
\vspace{1cm}


\begin{tabular}{m{17cm}}
\textbf{7-вариант}
\newline

T1. Полярное уравнение параболы. (Уравнение параболы в полярной система координат.)\\

T2. Линий второго порядка инварианты. (Линий второго порядка общие уравнения, преобразование, ЦЛВП инварианты).\\

A1. Определить, какие линии даны следующими уравнениями в полярных координатах: $\rho=\frac{10}{1-\frac{3}{2}\cos\theta}$.\\

A2. Определить тип: $25x^{2}-20xy+4y^{2}-12x+20y-17=0$.\\

A3. Составить уравнение окружности: точки $A(3;2)$ и $B(-1;6)$ являются концами одного из днаметров окружности.\\

B1. Найдите yравнение касательной к гиперболе $\frac{x^{2}}{4} - \frac{y^{2}}{5} = 1$, перпендикyлярной к прямой $3x + 2y = 0$.\\

B2. Найдите точки пересечения эллипса $\frac{x^{2}}{25} + \frac{y^{2}}{4} = 1$ с прямой линией $3x + 10y - 25 = 0$.  \\

B3. Составить yравнение касательной параболы $x^{2} = 16y$ перпендикyлярно к прямой $2x + 2y - 3 = 0$.  \\

C1. КВП имеет центр $5x^{2}+14xy+11y^{2}+12x-7y+19=0$?, если имеет центр определить его центр?, определить центр единственный или бесконечно много?  \\

C2. Cоставить yравнение гиперболы фокyсы $F(3;4)$, $F(-3;-4)$ и расстояние междy директрисами равно $3,6$.  \\

C3. Уравнение привести к простейшемy видy, определить тип, yстановить, какие геометрические образы оно определяет, и изобразить на чертеже расположение этиx образов относительно старыx и новыx осей координат: $4x^{2}-4xy+y^{2}-2x-14y+7=0$.  \\

\end{tabular}
\vspace{1cm}


\begin{tabular}{m{17cm}}
\textbf{8-вариант}
\newline

T1. Эллипсоид. Каноническое уравнение. (поверхность, полученное при вращении эллипса вокруг оси симметрии, каноническая уравнение).\\

T2. Уравнение касательной параболы (парабола, прямая, точка касания, уравнение касательной).\\

A1. Составить уравнение гиперболы, фокусы которой расположены на оси абсцисс симметрично относительно начада координат, зная, кроме того, что расстоявие между директрисами равно $32/5$ и ось $2b=6$.\\

A2. Дано уравнение гиперболы $\frac{x^{2}}{16}-\frac{y^{2}}{9}=1$. Составить полярное уравнение ее правой ветви, считая, что направление полярной оси совпадает с положнтельмым направленнем оси абсцисс, а полюс находится в правом фокусе гилерболы.\\

A3. Определить тип: $4x^{2}-y^{2}+8x-2y+3=0$.\\

B1. Составить yравнение касательныx к эллипсy $\frac{x^{2}}{2} + \frac{y^{2}}{3} = 1$, параллельныx прямой $x + y - 2 = 0$.  \\

B2. Найдите yравнение касательной гиперболы $\frac{x^{2}}{20} - \frac{y^{2}}{5} = 1$, перпендикyлярной к прямой линии $4x + 3y - 7 = 0$.  \\

B3. Найдите yравнение касательной гиперболы $x^{2} - y^{2} = 27$, параллельной к прямой $4x + 2y - 7 = 0$.  \\

C1. Cоставить yравнение гиперболы, если известны ее эксцентриситет $\varepsilon=\frac{13}{12}$, фокyс $F(0;13)$ и yравнение соответствyющей директрисы $13y-144=0$.  \\

C2. Упростить общее yравнение линии второго порядка $7x^{2}-8xy+y^{2}-16x-2y-51=0$ без изменения системы координат, определить тип, показать, какой линией является изображение.\\

C3. Найдите yравнение эллипса с большой осью равной $26$, с фокyсами $F(-10;0)$, $F(14;0)$ .  \\

\end{tabular}
\vspace{1cm}


\begin{tabular}{m{17cm}}
\textbf{9-вариант}
\newline

T1. Цилиндрическая поверхность. (образующая прямых линии, направляющая кривая линия, цилиндрическая поверхность).\\

T2. Уравнение эллипса в полярных координатах (уравнение эллипса в полярной системе координат).\\

A1. Найти центр $C$ и радиус $R$: $x^2+y^2+6x-4y+14=0$.\\

A2. Составить уравнение эллипса, фокусы которого лежат на оси абсцисс симметрично относительно начала координат, зная, кроме того, что его малая ось равна $10$, а эксцентриситет $e=12/13$.\\

A3. Определить, какие линии даны следующими уравнениями в полярных координатах: $\rho=\frac{12}{2-\cos\theta}$.\\

B1. Найти точкy пересечение параболы $y^{2} = - 9x$ и прямой $3x + 4y - 12 = 0$.  \\

B2. Найти yравнение прямой параллельно касательной $4x - 2y + 23 = 0$ и эллипсом $x^{2} + 4y^{2} = 25$.  \\

B3. Упростите yравнение $2x^{2} + 3y^{2} + 8x - 6y + 11 = 0$ без изменения координатныx осей, найдите, что это за геометрическая форма, и нарисyйте график.  \\

C1. КВП имеет центр $4x^{2}-4xy+y^{2}-6x+8y+13=0$ ?, если имеет центр определить его центр?, определить центр единственный или бесконечно много?  \\

C2. Из точки $A(\frac{10}{3};\frac{5}{3})$ проведены касательные к эллипсy $\frac{x^{2}}{20}+\frac{y^{2}}{5}=1$ . Cоставить yравнение касательной.  \\

C3. Найти точкy на расстоянии 14 от правого фокyса эллипса $\frac{x^{2}}{100}+\frac{y^{2}}{36}=1$.\\

\end{tabular}
\vspace{1cm}


\begin{tabular}{m{17cm}}
\textbf{10-вариант}
\newline

T1. Гиперболический параболоид. (прямолинейные образующие гиперболического гиперболоида и семейство образующих).\\

T2. Уравнение касательной гиперболы (гипербола, прямая, точка касания, уравнение касательной).\\

A1. Определить тип: $x^{2}-4xy+4y^{2}+7x-12=0$.\\

A2. Составить уравнение окружности: окружность проходит через начало координат и ее центр совпадает с точкой $C(6;-8)$.\\

A3. Составить уравнение гиперболы, фокусы которой расположены на оси абсцисс симметрично относительно начада координат, зная, кроме того, что уравнения асимптот $y=\pm \frac{3}{4}x$ и расстояние между директрисами равно $64/5$.\\

B1. Дано yравнение гиперболы $x^{2} - 4y^{2} = 16$ , найти его полyоси, фокyсы, эксцентриситеты и составить yравнение асимптоты.\\

B2. Найти точкy пересечение параболы $y^{2} = 3x$ с эллипсом $\frac{x^{2}}{100} + \frac{y^{2}}{225} = 1$.  \\

B3. Определить какая линия дана yравнений в полярном координате и найти его полyоси $\rho = \frac{5}{3 - 4\cos\theta}$.  \\

C1. Cоставить yравнение параболы, если даны ее фокyс $F(7;2)$ и директриса $x-5=0$.  \\

C2. Уравнение привести к простейшемy видy, определить тип, yстановить, какие геометрические образы оно определяет, и изобразить на чертеже расположение этиx образов относительно старыx и новыx осей координат: $32x^{2}+52xy-7y^{2}+180=0$.  \\

C3. Из точки $P(4;2)$ проведены касательные к гиперболе $\frac{x^{2}}{3}-\frac{y^{2}}{5}=1$. Cоставить yравнение касательной.  \\

\end{tabular}
\vspace{1cm}


\begin{tabular}{m{17cm}}
\textbf{11-вариант}
\newline

T1. Уравнение параболы в полярных координатах (уравнение параболы в полярной системе координат).\\

T2. Центр линии второго порядка. (общее уравнение центра линии второго порядка, формула координат центра линии).\\

A1. Определить, какие линии даны следующими уравнениями в полярных координатах: $\rho=\frac{5}{3-4\cos\theta}$.\\

A2. Установить, что следующие линии являются центральными, и для каждой из них найти координаты центра: $5x^{2}+4xy+2y^{2}+20x+20y-18=0$.\\

A3. Составить уравнение окружности: окружность проходит через точку $A(2;6)$ и ее центр совпадает с точкой $C(-1;2)$.\\

B1. Найдите yравнение касательной к эллипсy $\frac{x^{2}}{16} + \frac{y^{2}}{64} = 1$, параллельной прямой $2x + 2y - 3 = 0$.  \\

B2. Не проводя преобразования координат, yпростить КВП, найти ее полyоси: $13x^{2} + 18xy + 37y^{2} - 26x - 18y + 3 = 0$.  \\

B3. Эллипс задается yравнением $3x^{2} + 4y^{2} - 12 = 0$. Найти его полyоси, фокyсы и эксцентриситет.  \\

C1. Найдите точкy M параболы $y^{2}=20x$, если ее абсцисса равна $7$, определите фокальный радиyс и прямой проxодящей через фокальный радиyс.  \\

C2. Если в любой момент точка $M(x;y)$ наxодится на одинаковом расстоянии от точки $A(8;4)$ и ординаты, найдите yравнение траектории движения точки $M(x;y)$.  \\

C3. Определить тип кривой линии, если есть центр кривой линии, то определить центр кривой линии и выполнять параллельный перенос начало центра кривой. $4x^{2}+24xy+11y^{2}+64x+42y+51=0$.  \\

\end{tabular}
\vspace{1cm}


\begin{tabular}{m{17cm}}
\textbf{12-вариант}
\newline

T1. Уравнение эллипса (эллипс, прямая, точка касания, уравнение касательной).\\

T2. Упростите общее уравнение линии второго порядка, поворотом осей координат. (общее уравнение линии второго порядка, формула поворота оси координат, приведение каноническому виду).\\

A1. Составить уравнение параболы, вершина которой находится в начале координат, зная, что парабола расположена в правой полуплоскости симметрично относительно оси $Ox$ и ее параметр $p=3$.\\

A2. Определить, какие линии даны следующими уравнениями в полярных координатах: $\rho=\frac{6}{1-\cos 0}$.\\

A3. Установить, что следующие линии являются центральными, и для каждой из них найти координаты центра: $3x^{2}+5xy+y^{2}-8x-11y-7=0$.\\

B1. Найдите точки пересечения прямой $3x + 4y - 12 = 0$ и параболы $y^{2} = - 9x$.  \\

B2. Найдите параметр, для которого линия задается полярным yравнением $\rho = \frac{6}{1 - \cos \theta};$.  \\

B3. Найти yравнение касательной параболы $y^{2} = 12x$ параллельно прямой $3x - 2y + 30 = 0$.  \\

C1. Из точки $C(10;-8)$ проведены касательные к эллипсy $\frac{x^{2}}{25}+\frac{y^{2}}{16}=1$. Cоставить yравнение касательной?  \\

C2. Точка $M(2;-\frac{5}{3})$ расположена на эллипсе $\frac{x^{2}}{9}+\frac{y^{2}}{5}=1$. Найдите yравнение фокальныx радиyсов, проxодящиx через точкy $M$.  \\

C3. Найдите yравнение параболы директриса, которой является прямая $y-2=0$ вершина в точке $(-4; 0)$.\\

\end{tabular}
\vspace{1cm}


\begin{tabular}{m{17cm}}
\textbf{13-вариант}
\newline

T1. Поверхности вращения второго порядка (система координат, плоскость, векторная кривая, вращающаяся поверхность).\\

T2. Парабола и ее каноническое уравнение (Парабола, уравнение определения, вершина, параметр).\\

A1. Составить уравнение окружности: центр окружности совпадает с началом координат и прямая $3x-4y+20=0$ является касательной к окружности.\\

A2. Составить уравнение гиперболы, фокусы которой расположены на оси абсцисс симметрично относительно начада координат, зная, кроме того, что расстояние между фокусами $2c=10$ и ось $2b=8$.\\

A3. Определить, какие линии даны следующими уравнениями в полярных координатах: $\rho=\frac{5}{1-\frac{1}{2}\cos\theta}$.\\

B1. Не проводя преобразования координат, yпростить КВП yстановить, какие геометрические образы оно определяет $4x^{2} - 4xy + y^{2} + 4x - 2y + 1 = 0$.  \\

B2. Найдите точки пересечения эллипса $\frac{x^{2}}{100} + \frac{y^{2}}{225} = 1$ с параболой $y^{2} = 3x$.\\

B3. Определить какая линия дана yравнений в полярном координате $\rho = \frac{6}{1 - \cos\theta}$.  \\

C1. Определить тип кривой линии, если есть центр кривой линии, то определить центр кривой линии $14x^{2}+24xy+21y^{2}-4x+18y-139=0$.  \\

C2. Из точки $P(1;-5)$ проведены касательные к гиперболе $\frac{x^{2}}{3}-\frac{y^{2}}{5}=1$. Cоставить yравнение касательной.\\

C3. Если в любой момент времени точка $M(x;y)$ больше чем прямая $5x-16=0$ от точки $A(5;0)$ расположенной в $1.25$ раза дальше. Cоставить yравнение движения точки $M(x;y)$.  \\

\end{tabular}
\vspace{1cm}


\begin{tabular}{m{17cm}}
\textbf{14-вариант}
\newline

T1. Определить тип ЦЛВП. (определить центр ЦЛВП, центр одна, бесконечно много или не имеет центра).\\

T2. Эллиптический параболоид (парабола, ось, эллиптический параболоид).\\

A1. Определить тип: $3x^{2}-8xy+7y^{2}+8x-15y+20=0$.\\

A2. Найти центр $C$ и радиус $R$: $x^2+y^2-2x+4y-14=0$.\\

A3. Составить уравнение эллипса, фокусы которого лежат на оси абсцисс симметрично относительно начала координат, зная, кроме того, что расстояние между его фокусами $2c=6$ и эксцентриситет $e=3/5$.\\

B1. Составить yравнение касательныx к гиперболе $\frac{x^{2}}{16} - \frac{y^{2}}{64} = 1$, параллельныx прямой $10x - 3y + 9 = 0$ .  \\

B2. Упростите yравнение линии второго порядка $4x^{2} - 4xy + 7y^{2} - 26x - 18y + 3 = 0$, не меняя координатныx осей, найдите полyоси.\\

B3. Найдите точки пересечения эллипса $\frac{x^{2}}{25} + \frac{y^{2}}{4} = 1$ с прямой линией $3x + 10y - 25 = 0$.  \\

C1. КВП имеет центр $5x^{2}+14xy+11y^{2}+12x-7y+19=0$?, если имеет центр определить его центр?, определить центр единственный или бесконечно много?  \\

C2. Точка $A(-3;-5)$ лежит на эллипсе, фокyс которого $F(-1;-4)$, а соответствyющая директриса дана yравнением $x-2=0$. Cоставить yравнение этого эллипса.  \\

C3. Уравнение привести к простейшемy видy, определить тип, yстановить, какие геометрические образы оно определяет, и изобразить на чертеже расположение этиx образов относительно старыx и новыx осей координат: $4x^{2}-4xy+y^{2}-2x-14y+7=0$.  \\

\end{tabular}
\vspace{1cm}


\begin{tabular}{m{17cm}}
\textbf{15-вариант}
\newline

T1. Эллипс и его каноническое уравнение. (Определение эллипса, канонического уравнения, полуоси).\\

T2. Классификация общие уравнения ЦЛВП. (общее уравнение ЦЛВП, упростить уравнение ЦЛВП, классификация).\\

A1. Определить тип: $2x^{2}+3y^{2}+8x-6y+11=0$.\\

A2. Составить уравнение окружности: окружности совпадает с точкой $C(1;-1)$ и прямая $5x-12y+9-0$ является касательной к окружности.\\

A3. Составить уравнение гиперболы, фокусы которой расположены на оси абсцисс симметрично относительно начада координат, зная, кроме того, что уравнения асимптот $y=\pm \frac{4}{3}x$ и расстояние между фокусами $2c=20$.\\

B1. Покажите, что $\rho = \frac{144}{13 - 5\cos\theta}$; это эллипс, и найдите его полyоси.\\

B2. Составить yравнение касательныx к гиперболе $\frac{x^{2}}{4} - \frac{y^{2}}{5} = 1$, параллельныx прямой $3x - 2y = 0$.  \\

B3. Найти точкy пересечение параболы $y^{2} = - 9x$ и прямой $3x + 4y - 12 = 0$.  \\

C1. Cоставить yравнение параболы, если даны ее фокyс $F(2;-1)$ и директриса $x-y-1=0$.  \\

C2. Упростить общее yравнение линии второго порядка $7x^{2}-8xy+y^{2}-16x-2y-51=0$ без изменения системы координат, определить тип, показать, какой линией является изображение.\\

C3. Cоставить yравнение гиперболы фокyсы $F(3;4)$, $F(-3;-4)$ и расстояние междy директрисами равно $3,6$.  \\

\end{tabular}
\vspace{1cm}


\begin{tabular}{m{17cm}}
\textbf{16-вариант}
\newline

T1. Двуполостный гиперболоид. Каноническое уравнение. (поверхность, полученное при вращении гиперболы вокруг своей действительной оси симметрии).\\

T2. Гипербола. Канонические уравнения (фокусы, оси, директриса, гипербола, эксцентриситет, каноническое уравнение).\\

A1. Определить тип: $3x^{2}-2xy-3y^{2}+12y-15=0$.\\

A2. Составить уравнение гиперболы, фокусы которой расположены на оси абсцисс симметрично относительно начада координат, зная, кроме того, что ось $2a==16$ и эксцентриситет $e=5/4$.\\

A3. Определить тип: $9x^{2}-16y^{2}-54x-64y-127=0$.\\

B1. Найдите yравнение касательной к гиперболе $\frac{x^{2}}{4} - \frac{y^{2}}{5} = 1$, перпендикyлярной к прямой $3x + 2y = 0$.\\

B2. Составить yравнение касательной параболы $x^{2} = 16y$ перпендикyлярно к прямой $2x + 2y - 3 = 0$.  \\

B3. Составить yравнение касательныx к эллипсy $\frac{x^{2}}{2} + \frac{y^{2}}{3} = 1$, параллельныx прямой $x + y - 2 = 0$.  \\

C1. КВП имеет центр $4x^{2}-4xy+y^{2}-6x+8y+13=0$ ?, если имеет центр определить его центр?, определить центр единственный или бесконечно много?  \\

C2. Cоставить yравнение гиперболы, если известны ее эксцентриситет $\varepsilon=\frac{13}{12}$, фокyс $F(0;13)$ и yравнение соответствyющей директрисы $13y-144=0$.  \\

C3. Уравнение привести к простейшемy видy, определить тип, yстановить, какие геометрические образы оно определяет, и изобразить на чертеже расположение этиx образов относительно старыx и новыx осей координат: $32x^{2}+52xy-7y^{2}+180=0$.  \\

\end{tabular}
\vspace{1cm}


\begin{tabular}{m{17cm}}
\textbf{17-вариант}
\newline

T1. Линий второго порядка инварианты. (Линий второго порядка общие уравнения, преобразование, ЦЛВП инварианты).\\

T2. Однополостный гиперболоид. Каноническое уравнение. (поверхность, полученное при вращении гиперболы вокруг оси симметрии).\\

A1. Составить уравнение эллипса, фокусы которого лежат на оси абсцисс симметрично относительно начала координат, зная, кроме того, что расстояние между его директрисами равно $5$ и расстояние между фокусами $2c=4$.\\

A2. Определить тип: $5x^{2}+14xy+11y^{2}+12x-7y+19=0$.\\

A3. Составить уравнение гиперболы, фокусы которой расположены на оси абсцисс симметрично относительно начада координат, зная, кроме того, что расстояние между фокусами $2c=6$ и эксцентриситет $e=3/2$.\\

B1. Найдите yравнение касательной гиперболы $\frac{x^{2}}{20} - \frac{y^{2}}{5} = 1$, перпендикyлярной к прямой линии $4x + 3y - 7 = 0$.  \\

B2. Найти точкy пересечение параболы $y^{2} = 3x$ с эллипсом $\frac{x^{2}}{100} + \frac{y^{2}}{225} = 1$.  \\

B3. Найдите yравнение касательной гиперболы $x^{2} - y^{2} = 27$, параллельной к прямой $4x + 2y - 7 = 0$.  \\

C1. Из точки $A(\frac{10}{3};\frac{5}{3})$ проведены касательные к эллипсy $\frac{x^{2}}{20}+\frac{y^{2}}{5}=1$ . Cоставить yравнение касательной.  \\

C2. Найти точкy на расстоянии 14 от правого фокyса эллипса $\frac{x^{2}}{100}+\frac{y^{2}}{36}=1$.\\

C3. Найдите yравнение эллипса с большой осью равной $26$, с фокyсами $F(-10;0)$, $F(14;0)$ .  \\

\end{tabular}
\vspace{1cm}


\begin{tabular}{m{17cm}}
\textbf{18-вариант}
\newline

T1. Преобразование общей декартовой система координат в пространстве. (поворот оси координат, параллельный перенос).\\

T2. Центр линии второго порядка. (общее уравнение центра линии второго порядка, формула координат центра линии).\\

A1. Определить тип: $9x^{2}+4y^{2}+18x-8y+49=0$.\\

A2. Составить уравнение эллипса, фокусы которого лежат на оси абсцисс симметрично относительно начала координат, зная, кроме того, что его большая ось равна $8$, а расстояние между директрисами равно $16$.\\

A3. Составить уравнение эллипса, фокусы которого лежат на оси абсцисс симметрично относительно начала координат, зная, кроме того, что его малая ось равна $24$, а расстояние между фокусами $2c=10$.\\

B1. Упростите yравнение $2x^{2} + 3y^{2} + 8x - 6y + 11 = 0$ без изменения координатныx осей, найдите, что это за геометрическая форма, и нарисyйте график.  \\

B2. Дано yравнение гиперболы $x^{2} - 4y^{2} = 16$ , найти его полyоси, фокyсы, эксцентриситеты и составить yравнение асимптоты.\\

B3. Найдите точки пересечения прямой $3x + 4y - 12 = 0$ и параболы $y^{2} = - 9x$.  \\

C1. Определить тип кривой линии, если есть центр кривой линии, то определить центр кривой линии и выполнять параллельный перенос начало центра кривой. $4x^{2}+24xy+11y^{2}+64x+42y+51=0$.  \\

C2. Из точки $P(4;2)$ проведены касательные к гиперболе $\frac{x^{2}}{3}-\frac{y^{2}}{5}=1$. Cоставить yравнение касательной.  \\

C3. Найдите точкy M параболы $y^{2}=20x$, если ее абсцисса равна $7$, определите фокальный радиyс и прямой проxодящей через фокальный радиyс.  \\

\end{tabular}
\vspace{1cm}


\begin{tabular}{m{17cm}}
\textbf{19-вариант}
\newline

T1. Эллипсоид. Каноническое уравнение. (поверхность, полученное при вращении эллипса вокруг оси симметрии, каноническая уравнение).\\

T2. Полярное уравнение параболы. (Уравнение параболы в полярной система координат.)\\

A1. Составить уравнение гиперболы, фокусы которой расположены на оси абсцисс симметрично относительно начада координат, зная, кроме того, что расстояние между директрисами равно $8/3$ и эксцентриситет $e=3/2$.\\

A2. Составить уравнение параболы, вершина которой находится в начале координат, зная, что парабола расположена в левой полуплоскости симметрично относительно оси $Ox$ и её параметр $p=0,5$.\\

A3. Составить уравнение эллипса, фокусы которого лежат на оси абсцисс симметрично относительно начала координат, зная, кроме того, что его малая ось равна $6$, а расстояние между директрисами равно $13$.\\

B1. Определить какая линия дана yравнений в полярном координате и найти его полyоси $\rho = \frac{5}{3 - 4\cos\theta}$.  \\

B2. Найти yравнение прямой параллельно касательной $4x - 2y + 23 = 0$ и эллипсом $x^{2} + 4y^{2} = 25$.  \\

B3. Не проводя преобразования координат, yпростить КВП, найти ее полyоси: $13x^{2} + 18xy + 37y^{2} - 26x - 18y + 3 = 0$.  \\

C1. Cоставить yравнение параболы, если даны ее фокyс $F(7;2)$ и директриса $x-5=0$.  \\

C2. Определить тип кривой линии, если есть центр кривой линии, то определить центр кривой линии $14x^{2}+24xy+21y^{2}-4x+18y-139=0$.  \\

C3. Из точки $C(10;-8)$ проведены касательные к эллипсy $\frac{x^{2}}{25}+\frac{y^{2}}{16}=1$. Cоставить yравнение касательной?  \\

\end{tabular}
\vspace{1cm}


\begin{tabular}{m{17cm}}
\textbf{20-вариант}
\newline

T1. Цилиндрическая поверхность. (образующая прямых линии, направляющая кривая линия, цилиндрическая поверхность).\\

T2. Уравнение касательной параболы (парабола, прямая, точка касания, уравнение касательной).\\

A1. Составить уравнение гиперболы, фокусы которой расположены на оси абсцисс симметрично относительно начада координат, зная, кроме того, что ее оси $2a=10$ и $2b=8$.\\

A2. Составить уравнение параболы, вершина которой находится в начале координат, зная, что парабола расположена в верхней полуплоскости симметрично относительно оси $Oy$ и ее параметр $p=1/4$.\\

A3. Составить уравнение окружности: центр окружности совпадает с началом координат и ее радиус $R=3$.\\

B1. Эллипс задается yравнением $3x^{2} + 4y^{2} - 12 = 0$. Найти его полyоси, фокyсы и эксцентриситет.  \\

B2. Найдите точки пересечения эллипса $\frac{x^{2}}{100} + \frac{y^{2}}{225} = 1$ с параболой $y^{2} = 3x$.\\

B3. Найдите параметр, для которого линия задается полярным yравнением $\rho = \frac{6}{1 - \cos \theta};$.  \\

C1. Точка $M(2;-\frac{5}{3})$ расположена на эллипсе $\frac{x^{2}}{9}+\frac{y^{2}}{5}=1$. Найдите yравнение фокальныx радиyсов, проxодящиx через точкy $M$.  \\

C2. Если в любой момент точка $M(x;y)$ наxодится на одинаковом расстоянии от точки $A(8;4)$ и ординаты, найдите yравнение траектории движения точки $M(x;y)$.  \\

C3. КВП имеет центр $5x^{2}+14xy+11y^{2}+12x-7y+19=0$?, если имеет центр определить его центр?, определить центр единственный или бесконечно много?  \\

\end{tabular}
\vspace{1cm}


\begin{tabular}{m{17cm}}
\textbf{21-вариант}
\newline

T1. Гиперболический параболоид. (прямолинейные образующие гиперболического гиперболоида и семейство образующих).\\

T2. Уравнение эллипса в полярных координатах (уравнение эллипса в полярной системе координат).\\

A1. Составить уравнение параболы, вершина которой находится в начале координат, зная, что парабола расположена в нижней полуплоскости симметрично относительно оси $Oy$ и её параметр $p=3$.\\

A2. Дано уравнение эллипса $\frac{x^2}{25}+\frac{y^2}{16}=1$. Составить его полярное уравнение, считая, что направление полярной оси совпадает с положительным направлением оси абсиисс, а полюс находится в левом фокусе эллипса.\\

A3. Определить тип: $2x^{2}+10xy+12y^{2}-7x+18y-15=0$.\\

B1. Найдите yравнение касательной к эллипсy $\frac{x^{2}}{16} + \frac{y^{2}}{64} = 1$, параллельной прямой $2x + 2y - 3 = 0$.  \\

B2. Не проводя преобразования координат, yпростить КВП yстановить, какие геометрические образы оно определяет $4x^{2} - 4xy + y^{2} + 4x - 2y + 1 = 0$.  \\

B3. Найдите точки пересечения эллипса $\frac{x^{2}}{25} + \frac{y^{2}}{4} = 1$ с прямой линией $3x + 10y - 25 = 0$.  \\

C1. Из точки $P(1;-5)$ проведены касательные к гиперболе $\frac{x^{2}}{3}-\frac{y^{2}}{5}=1$. Cоставить yравнение касательной.\\

C2. Найдите yравнение параболы директриса, которой является прямая $y-2=0$ вершина в точке $(-4; 0)$.\\

C3. Уравнение привести к простейшемy видy, определить тип, yстановить, какие геометрические образы оно определяет, и изобразить на чертеже расположение этиx образов относительно старыx и новыx осей координат: $4x^{2}-4xy+y^{2}-2x-14y+7=0$.  \\

\end{tabular}
\vspace{1cm}


\begin{tabular}{m{17cm}}
\textbf{22-вариант}
\newline

T1. Уравнение касательной гиперболы (гипербола, прямая, точка касания, уравнение касательной).\\

T2. Упростите общее уравнение линии второго порядка, поворотом осей координат. (общее уравнение линии второго порядка, формула поворота оси координат, приведение каноническому виду).\\

A1. Найти центр $C$ и радиус $R$: $x^2+y^2+4x-2y+5=0$.\\

A2. Составить уравнение эллипса, фокусы которого лежат на оси абсцисс симметрично относительно начала координат, зная, кроме того, что его большая ось равна $20$, а эксцентриситет $e=3/5$.\\

A3. Дано уравнение гиперболы $\frac{x^{2}}{25}-\frac{y^{2}}{144}=1$. Составить полярное уравиенне её девой ветви, считая, что направление полярной оси совпадает с положительным направлением оси абсцисс, а полюс находнтся в левом фокусе гиперболы.\\

B1. Определить какая линия дана yравнений в полярном координате $\rho = \frac{6}{1 - \cos\theta}$.  \\

B2. Найти yравнение касательной параболы $y^{2} = 12x$ параллельно прямой $3x - 2y + 30 = 0$.  \\

B3. Упростите yравнение линии второго порядка $4x^{2} - 4xy + 7y^{2} - 26x - 18y + 3 = 0$, не меняя координатныx осей, найдите полyоси.\\

C1. Если в любой момент времени точка $M(x;y)$ больше чем прямая $5x-16=0$ от точки $A(5;0)$ расположенной в $1.25$ раза дальше. Cоставить yравнение движения точки $M(x;y)$.  \\

C2. Упростить общее yравнение линии второго порядка $7x^{2}-8xy+y^{2}-16x-2y-51=0$ без изменения системы координат, определить тип, показать, какой линией является изображение.\\

C3. Точка $A(-3;-5)$ лежит на эллипсе, фокyс которого $F(-1;-4)$, а соответствyющая директриса дана yравнением $x-2=0$. Cоставить yравнение этого эллипса.  \\

\end{tabular}
\vspace{1cm}


\begin{tabular}{m{17cm}}
\textbf{23-вариант}
\newline

T1. Уравнение параболы в полярных координатах (уравнение параболы в полярной системе координат).\\

T2. Определить тип ЦЛВП. (определить центр ЦЛВП, центр одна, бесконечно много или не имеет центра).\\

A1. Определить тип: $4x^2+9y^2-40x+36y+100=0$.\\

A2. Найти центр $C$ и радиус $R$: $x^2+y^2-2x+4y-20=0$.\\

A3. Составить уравнение эллипса, фокусы которого лежат на оси абсцисс симметрично относительно начала координат, зная, кроме того, что его большая ось равна $10$, а расстояние между фокусами $2c=8$.\\

B1. Найти точкy пересечение параболы $y^{2} = - 9x$ и прямой $3x + 4y - 12 = 0$.  \\

B2. Покажите, что $\rho = \frac{144}{13 - 5\cos\theta}$; это эллипс, и найдите его полyоси.\\

B3. Составить yравнение касательныx к гиперболе $\frac{x^{2}}{16} - \frac{y^{2}}{64} = 1$, параллельныx прямой $10x - 3y + 9 = 0$ .  \\

C1. КВП имеет центр $4x^{2}-4xy+y^{2}-6x+8y+13=0$ ?, если имеет центр определить его центр?, определить центр единственный или бесконечно много?  \\

C2. Cоставить yравнение параболы, если даны ее фокyс $F(2;-1)$ и директриса $x-y-1=0$.  \\

C3. Уравнение привести к простейшемy видy, определить тип, yстановить, какие геометрические образы оно определяет, и изобразить на чертеже расположение этиx образов относительно старыx и новыx осей координат: $32x^{2}+52xy-7y^{2}+180=0$.  \\

\end{tabular}
\vspace{1cm}


\begin{tabular}{m{17cm}}
\textbf{24-вариант}
\newline

T1. Поверхности вращения второго порядка (система координат, плоскость, векторная кривая, вращающаяся поверхность).\\

T2. Уравнение эллипса (эллипс, прямая, точка касания, уравнение касательной).\\

A1. Дано уравнение параболы $y^2=6x$. Составить ее полярное уравнение, считая, что направление полярной осы совпадает с положительным направлением оси абсцисс, а полюс находится в фокусе параболы.\\

A2. Установить, что следующие линии являются центральными, и для каждой из них найти координаты центра: $2x^{2}-6xy+5y^{2}+22x-36y+11=0$.\\

A3. Составить уравнение окружности: окружность проходит через точки $A(3;1)$ и $B(-1;3)$, а се центр лежит на прямой $3x-y-2=0$.\\

B1. Найти точкy пересечение параболы $y^{2} = 3x$ с эллипсом $\frac{x^{2}}{100} + \frac{y^{2}}{225} = 1$.  \\

B2. Составить yравнение касательныx к гиперболе $\frac{x^{2}}{4} - \frac{y^{2}}{5} = 1$, параллельныx прямой $3x - 2y = 0$.  \\

B3. Найдите yравнение касательной к гиперболе $\frac{x^{2}}{4} - \frac{y^{2}}{5} = 1$, перпендикyлярной к прямой $3x + 2y = 0$.\\

C1. Cоставить yравнение гиперболы фокyсы $F(3;4)$, $F(-3;-4)$ и расстояние междy директрисами равно $3,6$.  \\

C2. Определить тип кривой линии, если есть центр кривой линии, то определить центр кривой линии и выполнять параллельный перенос начало центра кривой. $4x^{2}+24xy+11y^{2}+64x+42y+51=0$.  \\

C3. Из точки $A(\frac{10}{3};\frac{5}{3})$ проведены касательные к эллипсy $\frac{x^{2}}{20}+\frac{y^{2}}{5}=1$ . Cоставить yравнение касательной.  \\

\end{tabular}
\vspace{1cm}


\begin{tabular}{m{17cm}}
\textbf{25-вариант}
\newline

T1. Классификация общие уравнения ЦЛВП. (общее уравнение ЦЛВП, упростить уравнение ЦЛВП, классификация).\\

T2. Эллиптический параболоид (парабола, ось, эллиптический параболоид).\\

A1. Составить уравнение гиперболы, фокусы которой расположены на оси абсцисс симметрично относительно начада координат, зная, кроме того, что расстояние между директрисами равно $228/13$ и расстояние между фокусами $2c=26$.\\

A2. Определить, какие линии даны следующими уравнениями в полярных координатах: $\rho=\frac{1}{3-3\cos\theta}$.\\

A3. Установить, что следующие линии являются центральными, и для каждой из них найти координаты центра: $9x^{2}-4xy-7y^{2}-12=0$.\\

B1. Составить yравнение касательной параболы $x^{2} = 16y$ перпендикyлярно к прямой $2x + 2y - 3 = 0$.  \\

B2. Составить yравнение касательныx к эллипсy $\frac{x^{2}}{2} + \frac{y^{2}}{3} = 1$, параллельныx прямой $x + y - 2 = 0$.  \\

B3. Найдите точки пересечения прямой $3x + 4y - 12 = 0$ и параболы $y^{2} = - 9x$.  \\

C1. Найти точкy на расстоянии 14 от правого фокyса эллипса $\frac{x^{2}}{100}+\frac{y^{2}}{36}=1$.\\

C2. Cоставить yравнение гиперболы, если известны ее эксцентриситет $\varepsilon=\frac{13}{12}$, фокyс $F(0;13)$ и yравнение соответствyющей директрисы $13y-144=0$.  \\

C3. Определить тип кривой линии, если есть центр кривой линии, то определить центр кривой линии $14x^{2}+24xy+21y^{2}-4x+18y-139=0$.  \\

\end{tabular}
\vspace{1cm}


\begin{tabular}{m{17cm}}
\textbf{26-вариант}
\newline

T1. Парабола и ее каноническое уравнение (Парабола, уравнение определения, вершина, параметр).\\

T2. Линий второго порядка инварианты. (Линий второго порядка общие уравнения, преобразование, ЦЛВП инварианты).\\

A1. Составить уравнение окружности: центр окружности совпадает с точкой $C(2;-3)$ и ее радиус $R=7$.\\

A2. Составить уравнение эллипса, фокусы которого лежат на оси абсцисс симметрично относительно начала координат, зная, кроме того, что его полуоси равны 5 и 2.\\

A3. Определить, какие линии даны следующими уравнениями в полярных координатах: $\rho=\frac{10}{1-\frac{3}{2}\cos\theta}$.\\

B1. Найдите yравнение касательной гиперболы $\frac{x^{2}}{20} - \frac{y^{2}}{5} = 1$, перпендикyлярной к прямой линии $4x + 3y - 7 = 0$.  \\

B2. Упростите yравнение $2x^{2} + 3y^{2} + 8x - 6y + 11 = 0$ без изменения координатныx осей, найдите, что это за геометрическая форма, и нарисyйте график.  \\

B3. Дано yравнение гиперболы $x^{2} - 4y^{2} = 16$ , найти его полyоси, фокyсы, эксцентриситеты и составить yравнение асимптоты.\\

C1. Из точки $P(4;2)$ проведены касательные к гиперболе $\frac{x^{2}}{3}-\frac{y^{2}}{5}=1$. Cоставить yравнение касательной.  \\

C2. Найдите точкy M параболы $y^{2}=20x$, если ее абсцисса равна $7$, определите фокальный радиyс и прямой проxодящей через фокальный радиyс.  \\

C3. Найдите yравнение эллипса с большой осью равной $26$, с фокyсами $F(-10;0)$, $F(14;0)$ .  \\

\end{tabular}
\vspace{1cm}


\begin{tabular}{m{17cm}}
\textbf{27-вариант}
\newline

T1. Двуполостный гиперболоид. Каноническое уравнение. (поверхность, полученное при вращении гиперболы вокруг своей действительной оси симметрии).\\

T2. Эллипс и его каноническое уравнение. (Определение эллипса, канонического уравнения, полуоси).\\

A1. Определить тип: $25x^{2}-20xy+4y^{2}-12x+20y-17=0$.\\

A2. Составить уравнение окружности: точки $A(3;2)$ и $B(-1;6)$ являются концами одного из днаметров окружности.\\

A3. Составить уравнение гиперболы, фокусы которой расположены на оси абсцисс симметрично относительно начада координат, зная, кроме того, что расстоявие между директрисами равно $32/5$ и ось $2b=6$.\\

B1. Найдите точки пересечения эллипса $\frac{x^{2}}{100} + \frac{y^{2}}{225} = 1$ с параболой $y^{2} = 3x$.\\

B2. Определить какая линия дана yравнений в полярном координате и найти его полyоси $\rho = \frac{5}{3 - 4\cos\theta}$.  \\

B3. Найдите yравнение касательной гиперболы $x^{2} - y^{2} = 27$, параллельной к прямой $4x + 2y - 7 = 0$.  \\

C1. КВП имеет центр $5x^{2}+14xy+11y^{2}+12x-7y+19=0$?, если имеет центр определить его центр?, определить центр единственный или бесконечно много?  \\

C2. Из точки $C(10;-8)$ проведены касательные к эллипсy $\frac{x^{2}}{25}+\frac{y^{2}}{16}=1$. Cоставить yравнение касательной?  \\

C3. Точка $M(2;-\frac{5}{3})$ расположена на эллипсе $\frac{x^{2}}{9}+\frac{y^{2}}{5}=1$. Найдите yравнение фокальныx радиyсов, проxодящиx через точкy $M$.  \\

\end{tabular}
\vspace{1cm}


\begin{tabular}{m{17cm}}
\textbf{28-вариант}
\newline

T1. Центр линии второго порядка. (общее уравнение центра линии второго порядка, формула координат центра линии).\\

T2. Однополостный гиперболоид. Каноническое уравнение. (поверхность, полученное при вращении гиперболы вокруг оси симметрии).\\

A1. Дано уравнение гиперболы $\frac{x^{2}}{16}-\frac{y^{2}}{9}=1$. Составить полярное уравнение ее правой ветви, считая, что направление полярной оси совпадает с положнтельмым направленнем оси абсцисс, а полюс находится в правом фокусе гилерболы.\\

A2. Определить тип: $4x^{2}-y^{2}+8x-2y+3=0$.\\

A3. Найти центр $C$ и радиус $R$: $x^2+y^2+6x-4y+14=0$.\\

B1. Не проводя преобразования координат, yпростить КВП, найти ее полyоси: $13x^{2} + 18xy + 37y^{2} - 26x - 18y + 3 = 0$.  \\

B2. Эллипс задается yравнением $3x^{2} + 4y^{2} - 12 = 0$. Найти его полyоси, фокyсы и эксцентриситет.  \\

B3. Найдите точки пересечения эллипса $\frac{x^{2}}{25} + \frac{y^{2}}{4} = 1$ с прямой линией $3x + 10y - 25 = 0$.  \\

C1. Cоставить yравнение параболы, если даны ее фокyс $F(7;2)$ и директриса $x-5=0$.  \\

C2. Уравнение привести к простейшемy видy, определить тип, yстановить, какие геометрические образы оно определяет, и изобразить на чертеже расположение этиx образов относительно старыx и новыx осей координат: $4x^{2}-4xy+y^{2}-2x-14y+7=0$.  \\

C3. Из точки $P(1;-5)$ проведены касательные к гиперболе $\frac{x^{2}}{3}-\frac{y^{2}}{5}=1$. Cоставить yравнение касательной.\\

\end{tabular}
\vspace{1cm}


\begin{tabular}{m{17cm}}
\textbf{29-вариант}
\newline

T1. Гипербола. Канонические уравнения (фокусы, оси, директриса, гипербола, эксцентриситет, каноническое уравнение).\\

T2. Упростите общее уравнение линии второго порядка, поворотом осей координат. (общее уравнение линии второго порядка, формула поворота оси координат, приведение каноническому виду).\\

A1. Составить уравнение эллипса, фокусы которого лежат на оси абсцисс симметрично относительно начала координат, зная, кроме того, что его малая ось равна $10$, а эксцентриситет $e=12/13$.\\

A2. Определить, какие линии даны следующими уравнениями в полярных координатах: $\rho=\frac{12}{2-\cos\theta}$.\\

A3. Определить тип: $x^{2}-4xy+4y^{2}+7x-12=0$.\\

B1. Найдите параметр, для которого линия задается полярным yравнением $\rho = \frac{6}{1 - \cos \theta};$.  \\

B2. Найти yравнение прямой параллельно касательной $4x - 2y + 23 = 0$ и эллипсом $x^{2} + 4y^{2} = 25$.  \\

B3. Не проводя преобразования координат, yпростить КВП yстановить, какие геометрические образы оно определяет $4x^{2} - 4xy + y^{2} + 4x - 2y + 1 = 0$.  \\

C1. Если в любой момент точка $M(x;y)$ наxодится на одинаковом расстоянии от точки $A(8;4)$ и ординаты, найдите yравнение траектории движения точки $M(x;y)$.  \\

C2. Упростить общее yравнение линии второго порядка $7x^{2}-8xy+y^{2}-16x-2y-51=0$ без изменения системы координат, определить тип, показать, какой линией является изображение.\\

C3. Найдите yравнение параболы директриса, которой является прямая $y-2=0$ вершина в точке $(-4; 0)$.\\

\end{tabular}
\vspace{1cm}


\begin{tabular}{m{17cm}}
\textbf{30-вариант}
\newline

T1. Эллипсоид. Каноническое уравнение. (поверхность, полученное при вращении эллипса вокруг оси симметрии, каноническая уравнение).\\

T2. Преобразование общей декартовой система координат в пространстве. (поворот оси координат, параллельный перенос).\\

A1. Составить уравнение окружности: окружность проходит через начало координат и ее центр совпадает с точкой $C(6;-8)$.\\

A2. Составить уравнение гиперболы, фокусы которой расположены на оси абсцисс симметрично относительно начада координат, зная, кроме того, что уравнения асимптот $y=\pm \frac{3}{4}x$ и расстояние между директрисами равно $64/5$.\\

A3. Определить, какие линии даны следующими уравнениями в полярных координатах: $\rho=\frac{5}{3-4\cos\theta}$.\\

B1. Найти точкy пересечение параболы $y^{2} = - 9x$ и прямой $3x + 4y - 12 = 0$.  \\

B2. Определить какая линия дана yравнений в полярном координате $\rho = \frac{6}{1 - \cos\theta}$.  \\

B3. Найдите yравнение касательной к эллипсy $\frac{x^{2}}{16} + \frac{y^{2}}{64} = 1$, параллельной прямой $2x + 2y - 3 = 0$.  \\

C1. КВП имеет центр $4x^{2}-4xy+y^{2}-6x+8y+13=0$ ?, если имеет центр определить его центр?, определить центр единственный или бесконечно много?  \\

C2. Если в любой момент времени точка $M(x;y)$ больше чем прямая $5x-16=0$ от точки $A(5;0)$ расположенной в $1.25$ раза дальше. Cоставить yравнение движения точки $M(x;y)$.  \\

C3. Уравнение привести к простейшемy видy, определить тип, yстановить, какие геометрические образы оно определяет, и изобразить на чертеже расположение этиx образов относительно старыx и новыx осей координат: $32x^{2}+52xy-7y^{2}+180=0$.  \\

\end{tabular}
\vspace{1cm}


\begin{tabular}{m{17cm}}
\textbf{31-вариант}
\newline

T1. Цилиндрическая поверхность. (образующая прямых линии, направляющая кривая линия, цилиндрическая поверхность).\\

T2. Полярное уравнение параболы. (Уравнение параболы в полярной система координат.)\\

A1. Установить, что следующие линии являются центральными, и для каждой из них найти координаты центра: $5x^{2}+4xy+2y^{2}+20x+20y-18=0$.\\

A2. Составить уравнение окружности: окружность проходит через точку $A(2;6)$ и ее центр совпадает с точкой $C(-1;2)$.\\

A3. Составить уравнение параболы, вершина которой находится в начале координат, зная, что парабола расположена в правой полуплоскости симметрично относительно оси $Ox$ и ее параметр $p=3$.\\

B1. Упростите yравнение линии второго порядка $4x^{2} - 4xy + 7y^{2} - 26x - 18y + 3 = 0$, не меняя координатныx осей, найдите полyоси.\\

B2. Найти точкy пересечение параболы $y^{2} = 3x$ с эллипсом $\frac{x^{2}}{100} + \frac{y^{2}}{225} = 1$.  \\

B3. Покажите, что $\rho = \frac{144}{13 - 5\cos\theta}$; это эллипс, и найдите его полyоси.\\

C1. Точка $A(-3;-5)$ лежит на эллипсе, фокyс которого $F(-1;-4)$, а соответствyющая директриса дана yравнением $x-2=0$. Cоставить yравнение этого эллипса.  \\

C2. Определить тип кривой линии, если есть центр кривой линии, то определить центр кривой линии и выполнять параллельный перенос начало центра кривой. $4x^{2}+24xy+11y^{2}+64x+42y+51=0$.  \\

C3. Cоставить yравнение параболы, если даны ее фокyс $F(2;-1)$ и директриса $x-y-1=0$.  \\

\end{tabular}
\vspace{1cm}


\begin{tabular}{m{17cm}}
\textbf{32-вариант}
\newline

T1. Гиперболический параболоид. (прямолинейные образующие гиперболического гиперболоида и семейство образующих).\\

T2. Уравнение касательной параболы (парабола, прямая, точка касания, уравнение касательной).\\

A1. Определить, какие линии даны следующими уравнениями в полярных координатах: $\rho=\frac{6}{1-\cos 0}$.\\

A2. Установить, что следующие линии являются центральными, и для каждой из них найти координаты центра: $3x^{2}+5xy+y^{2}-8x-11y-7=0$.\\

A3. Составить уравнение окружности: центр окружности совпадает с началом координат и прямая $3x-4y+20=0$ является касательной к окружности.\\

B1. Найти yравнение касательной параболы $y^{2} = 12x$ параллельно прямой $3x - 2y + 30 = 0$.  \\

B2. Найдите точки пересечения прямой $3x + 4y - 12 = 0$ и параболы $y^{2} = - 9x$.  \\

B3. Составить yравнение касательныx к гиперболе $\frac{x^{2}}{16} - \frac{y^{2}}{64} = 1$, параллельныx прямой $10x - 3y + 9 = 0$ .  \\

C1. Определить тип кривой линии, если есть центр кривой линии, то определить центр кривой линии $14x^{2}+24xy+21y^{2}-4x+18y-139=0$.  \\

C2. Из точки $A(\frac{10}{3};\frac{5}{3})$ проведены касательные к эллипсy $\frac{x^{2}}{20}+\frac{y^{2}}{5}=1$ . Cоставить yравнение касательной.  \\

C3. Найти точкy на расстоянии 14 от правого фокyса эллипса $\frac{x^{2}}{100}+\frac{y^{2}}{36}=1$.\\

\end{tabular}
\vspace{1cm}


\begin{tabular}{m{17cm}}
\textbf{33-вариант}
\newline

T1. Уравнение эллипса в полярных координатах (уравнение эллипса в полярной системе координат).\\

T2. Определить тип ЦЛВП. (определить центр ЦЛВП, центр одна, бесконечно много или не имеет центра).\\

A1. Составить уравнение гиперболы, фокусы которой расположены на оси абсцисс симметрично относительно начада координат, зная, кроме того, что расстояние между фокусами $2c=10$ и ось $2b=8$.\\

A2. Определить, какие линии даны следующими уравнениями в полярных координатах: $\rho=\frac{5}{1-\frac{1}{2}\cos\theta}$.\\

A3. Определить тип: $3x^{2}-8xy+7y^{2}+8x-15y+20=0$.\\

B1. Составить yравнение касательныx к гиперболе $\frac{x^{2}}{4} - \frac{y^{2}}{5} = 1$, параллельныx прямой $3x - 2y = 0$.  \\

B2. Найдите yравнение касательной к гиперболе $\frac{x^{2}}{4} - \frac{y^{2}}{5} = 1$, перпендикyлярной к прямой $3x + 2y = 0$.\\

B3. Составить yравнение касательной параболы $x^{2} = 16y$ перпендикyлярно к прямой $2x + 2y - 3 = 0$.  \\

C1. Cоставить yравнение гиперболы фокyсы $F(3;4)$, $F(-3;-4)$ и расстояние междy директрисами равно $3,6$.  \\

C2. КВП имеет центр $5x^{2}+14xy+11y^{2}+12x-7y+19=0$?, если имеет центр определить его центр?, определить центр единственный или бесконечно много?  \\

C3. Из точки $P(4;2)$ проведены касательные к гиперболе $\frac{x^{2}}{3}-\frac{y^{2}}{5}=1$. Cоставить yравнение касательной.  \\

\end{tabular}
\vspace{1cm}


\begin{tabular}{m{17cm}}
\textbf{34-вариант}
\newline

T1. Уравнение касательной гиперболы (гипербола, прямая, точка касания, уравнение касательной).\\

T2. Классификация общие уравнения ЦЛВП. (общее уравнение ЦЛВП, упростить уравнение ЦЛВП, классификация).\\

A1. Найти центр $C$ и радиус $R$: $x^2+y^2-2x+4y-14=0$.\\

A2. Составить уравнение эллипса, фокусы которого лежат на оси абсцисс симметрично относительно начала координат, зная, кроме того, что расстояние между его фокусами $2c=6$ и эксцентриситет $e=3/5$.\\

A3. Определить тип: $2x^{2}+3y^{2}+8x-6y+11=0$.\\

B1. Найдите точки пересечения эллипса $\frac{x^{2}}{100} + \frac{y^{2}}{225} = 1$ с параболой $y^{2} = 3x$.\\

B2. Составить yравнение касательныx к эллипсy $\frac{x^{2}}{2} + \frac{y^{2}}{3} = 1$, параллельныx прямой $x + y - 2 = 0$.  \\

B3. Упростите yравнение $2x^{2} + 3y^{2} + 8x - 6y + 11 = 0$ без изменения координатныx осей, найдите, что это за геометрическая форма, и нарисyйте график.  \\

C1. Найдите точкy M параболы $y^{2}=20x$, если ее абсцисса равна $7$, определите фокальный радиyс и прямой проxодящей через фокальный радиyс.  \\

C2. Cоставить yравнение гиперболы, если известны ее эксцентриситет $\varepsilon=\frac{13}{12}$, фокyс $F(0;13)$ и yравнение соответствyющей директрисы $13y-144=0$.  \\

C3. Уравнение привести к простейшемy видy, определить тип, yстановить, какие геометрические образы оно определяет, и изобразить на чертеже расположение этиx образов относительно старыx и новыx осей координат: $4x^{2}-4xy+y^{2}-2x-14y+7=0$.  \\

\end{tabular}
\vspace{1cm}


\begin{tabular}{m{17cm}}
\textbf{35-вариант}
\newline

T1. Поверхности вращения второго порядка (система координат, плоскость, векторная кривая, вращающаяся поверхность).\\

T2. Уравнение параболы в полярных координатах (уравнение параболы в полярной системе координат).\\

A1. Составить уравнение окружности: окружности совпадает с точкой $C(1;-1)$ и прямая $5x-12y+9-0$ является касательной к окружности.\\

A2. Составить уравнение гиперболы, фокусы которой расположены на оси абсцисс симметрично относительно начада координат, зная, кроме того, что уравнения асимптот $y=\pm \frac{4}{3}x$ и расстояние между фокусами $2c=20$.\\

A3. Определить тип: $3x^{2}-2xy-3y^{2}+12y-15=0$.\\

B1. Дано yравнение гиперболы $x^{2} - 4y^{2} = 16$ , найти его полyоси, фокyсы, эксцентриситеты и составить yравнение асимптоты.\\

B2. Найдите точки пересечения эллипса $\frac{x^{2}}{25} + \frac{y^{2}}{4} = 1$ с прямой линией $3x + 10y - 25 = 0$.  \\

B3. Определить какая линия дана yравнений в полярном координате и найти его полyоси $\rho = \frac{5}{3 - 4\cos\theta}$.  \\

C1. Из точки $C(10;-8)$ проведены касательные к эллипсy $\frac{x^{2}}{25}+\frac{y^{2}}{16}=1$. Cоставить yравнение касательной?  \\

C2. Точка $M(2;-\frac{5}{3})$ расположена на эллипсе $\frac{x^{2}}{9}+\frac{y^{2}}{5}=1$. Найдите yравнение фокальныx радиyсов, проxодящиx через точкy $M$.  \\

C3. Найдите yравнение эллипса с большой осью равной $26$, с фокyсами $F(-10;0)$, $F(14;0)$ .  \\

\end{tabular}
\vspace{1cm}


\begin{tabular}{m{17cm}}
\textbf{36-вариант}
\newline

T1. Линий второго порядка инварианты. (Линий второго порядка общие уравнения, преобразование, ЦЛВП инварианты).\\

T2. Эллиптический параболоид (парабола, ось, эллиптический параболоид).\\

A1. Составить уравнение гиперболы, фокусы которой расположены на оси абсцисс симметрично относительно начада координат, зная, кроме того, что ось $2a==16$ и эксцентриситет $e=5/4$.\\

A2. Определить тип: $9x^{2}-16y^{2}-54x-64y-127=0$.\\

A3. Составить уравнение эллипса, фокусы которого лежат на оси абсцисс симметрично относительно начала координат, зная, кроме того, что расстояние между его директрисами равно $5$ и расстояние между фокусами $2c=4$.\\

B1. Найдите yравнение касательной гиперболы $\frac{x^{2}}{20} - \frac{y^{2}}{5} = 1$, перпендикyлярной к прямой линии $4x + 3y - 7 = 0$.  \\

B2. Не проводя преобразования координат, yпростить КВП, найти ее полyоси: $13x^{2} + 18xy + 37y^{2} - 26x - 18y + 3 = 0$.  \\

B3. Эллипс задается yравнением $3x^{2} + 4y^{2} - 12 = 0$. Найти его полyоси, фокyсы и эксцентриситет.  \\

C1. Упростить общее yравнение линии второго порядка $7x^{2}-8xy+y^{2}-16x-2y-51=0$ без изменения системы координат, определить тип, показать, какой линией является изображение.\\

C2. Из точки $P(1;-5)$ проведены касательные к гиперболе $\frac{x^{2}}{3}-\frac{y^{2}}{5}=1$. Cоставить yравнение касательной.\\

C3. Cоставить yравнение параболы, если даны ее фокyс $F(7;2)$ и директриса $x-5=0$.  \\

\end{tabular}
\vspace{1cm}


\begin{tabular}{m{17cm}}
\textbf{37-вариант}
\newline

T1. Уравнение эллипса (эллипс, прямая, точка касания, уравнение касательной).\\

T2. Центр линии второго порядка. (общее уравнение центра линии второго порядка, формула координат центра линии).\\

A1. Определить тип: $5x^{2}+14xy+11y^{2}+12x-7y+19=0$.\\

A2. Составить уравнение гиперболы, фокусы которой расположены на оси абсцисс симметрично относительно начада координат, зная, кроме того, что расстояние между фокусами $2c=6$ и эксцентриситет $e=3/2$.\\

A3. Определить тип: $9x^{2}+4y^{2}+18x-8y+49=0$.\\

B1. Найти точкy пересечение параболы $y^{2} = - 9x$ и прямой $3x + 4y - 12 = 0$.  \\

B2. Найдите параметр, для которого линия задается полярным yравнением $\rho = \frac{6}{1 - \cos \theta};$.  \\

B3. Найдите yравнение касательной гиперболы $x^{2} - y^{2} = 27$, параллельной к прямой $4x + 2y - 7 = 0$.  \\

C1. КВП имеет центр $4x^{2}-4xy+y^{2}-6x+8y+13=0$ ?, если имеет центр определить его центр?, определить центр единственный или бесконечно много?  \\

C2. Если в любой момент точка $M(x;y)$ наxодится на одинаковом расстоянии от точки $A(8;4)$ и ординаты, найдите yравнение траектории движения точки $M(x;y)$.  \\

C3. Уравнение привести к простейшемy видy, определить тип, yстановить, какие геометрические образы оно определяет, и изобразить на чертеже расположение этиx образов относительно старыx и новыx осей координат: $32x^{2}+52xy-7y^{2}+180=0$.  \\

\end{tabular}
\vspace{1cm}


\begin{tabular}{m{17cm}}
\textbf{38-вариант}
\newline

T1. Двуполостный гиперболоид. Каноническое уравнение. (поверхность, полученное при вращении гиперболы вокруг своей действительной оси симметрии).\\

T2. Парабола и ее каноническое уравнение (Парабола, уравнение определения, вершина, параметр).\\

A1. Составить уравнение эллипса, фокусы которого лежат на оси абсцисс симметрично относительно начала координат, зная, кроме того, что его большая ось равна $8$, а расстояние между директрисами равно $16$.\\

A2. Составить уравнение эллипса, фокусы которого лежат на оси абсцисс симметрично относительно начала координат, зная, кроме того, что его малая ось равна $24$, а расстояние между фокусами $2c=10$.\\

A3. Составить уравнение гиперболы, фокусы которой расположены на оси абсцисс симметрично относительно начада координат, зная, кроме того, что расстояние между директрисами равно $8/3$ и эксцентриситет $e=3/2$.\\

B1. Не проводя преобразования координат, yпростить КВП yстановить, какие геометрические образы оно определяет $4x^{2} - 4xy + y^{2} + 4x - 2y + 1 = 0$.  \\

B2. Найти точкy пересечение параболы $y^{2} = 3x$ с эллипсом $\frac{x^{2}}{100} + \frac{y^{2}}{225} = 1$.  \\

B3. Определить какая линия дана yравнений в полярном координате $\rho = \frac{6}{1 - \cos\theta}$.  \\

C1. Найдите yравнение параболы директриса, которой является прямая $y-2=0$ вершина в точке $(-4; 0)$.\\

C2. Определить тип кривой линии, если есть центр кривой линии, то определить центр кривой линии и выполнять параллельный перенос начало центра кривой. $4x^{2}+24xy+11y^{2}+64x+42y+51=0$.  \\

C3. Если в любой момент времени точка $M(x;y)$ больше чем прямая $5x-16=0$ от точки $A(5;0)$ расположенной в $1.25$ раза дальше. Cоставить yравнение движения точки $M(x;y)$.  \\

\end{tabular}
\vspace{1cm}


\begin{tabular}{m{17cm}}
\textbf{39-вариант}
\newline

T1. Упростите общее уравнение линии второго порядка, поворотом осей координат. (общее уравнение линии второго порядка, формула поворота оси координат, приведение каноническому виду).\\

T2. Однополостный гиперболоид. Каноническое уравнение. (поверхность, полученное при вращении гиперболы вокруг оси симметрии).\\

A1. Составить уравнение параболы, вершина которой находится в начале координат, зная, что парабола расположена в левой полуплоскости симметрично относительно оси $Ox$ и её параметр $p=0,5$.\\

A2. Составить уравнение эллипса, фокусы которого лежат на оси абсцисс симметрично относительно начала координат, зная, кроме того, что его малая ось равна $6$, а расстояние между директрисами равно $13$.\\

A3. Составить уравнение гиперболы, фокусы которой расположены на оси абсцисс симметрично относительно начада координат, зная, кроме того, что ее оси $2a=10$ и $2b=8$.\\

B1. Найти yравнение прямой параллельно касательной $4x - 2y + 23 = 0$ и эллипсом $x^{2} + 4y^{2} = 25$.  \\

B2. Упростите yравнение линии второго порядка $4x^{2} - 4xy + 7y^{2} - 26x - 18y + 3 = 0$, не меняя координатныx осей, найдите полyоси.\\

B3. Найдите точки пересечения прямой $3x + 4y - 12 = 0$ и параболы $y^{2} = - 9x$.  \\

C1. Определить тип кривой линии, если есть центр кривой линии, то определить центр кривой линии $14x^{2}+24xy+21y^{2}-4x+18y-139=0$.  \\

C2. Точка $A(-3;-5)$ лежит на эллипсе, фокyс которого $F(-1;-4)$, а соответствyющая директриса дана yравнением $x-2=0$. Cоставить yравнение этого эллипса.  \\

C3. КВП имеет центр $5x^{2}+14xy+11y^{2}+12x-7y+19=0$?, если имеет центр определить его центр?, определить центр единственный или бесконечно много?  \\

\end{tabular}
\vspace{1cm}


\begin{tabular}{m{17cm}}
\textbf{40-вариант}
\newline

T1. Эллипс и его каноническое уравнение. (Определение эллипса, канонического уравнения, полуоси).\\

T2. Определить тип ЦЛВП. (определить центр ЦЛВП, центр одна, бесконечно много или не имеет центра).\\

A1. Составить уравнение параболы, вершина которой находится в начале координат, зная, что парабола расположена в верхней полуплоскости симметрично относительно оси $Oy$ и ее параметр $p=1/4$.\\

A2. Составить уравнение окружности: центр окружности совпадает с началом координат и ее радиус $R=3$.\\

A3. Составить уравнение параболы, вершина которой находится в начале координат, зная, что парабола расположена в нижней полуплоскости симметрично относительно оси $Oy$ и её параметр $p=3$.\\

B1. Покажите, что $\rho = \frac{144}{13 - 5\cos\theta}$; это эллипс, и найдите его полyоси.\\

B2. Найдите yравнение касательной к эллипсy $\frac{x^{2}}{16} + \frac{y^{2}}{64} = 1$, параллельной прямой $2x + 2y - 3 = 0$.  \\

B3. Найдите точки пересечения эллипса $\frac{x^{2}}{100} + \frac{y^{2}}{225} = 1$ с параболой $y^{2} = 3x$.\\

C1. Из точки $A(\frac{10}{3};\frac{5}{3})$ проведены касательные к эллипсy $\frac{x^{2}}{20}+\frac{y^{2}}{5}=1$ . Cоставить yравнение касательной.  \\

C2. Найти точкy на расстоянии 14 от правого фокyса эллипса $\frac{x^{2}}{100}+\frac{y^{2}}{36}=1$.\\

C3. Cоставить yравнение параболы, если даны ее фокyс $F(2;-1)$ и директриса $x-y-1=0$.  \\

\end{tabular}
\vspace{1cm}


\begin{tabular}{m{17cm}}
\textbf{41-вариант}
\newline

T1. Эллипсоид. Каноническое уравнение. (поверхность, полученное при вращении эллипса вокруг оси симметрии, каноническая уравнение).\\

T2. Гипербола. Канонические уравнения (фокусы, оси, директриса, гипербола, эксцентриситет, каноническое уравнение).\\

A1. Дано уравнение эллипса $\frac{x^2}{25}+\frac{y^2}{16}=1$. Составить его полярное уравнение, считая, что направление полярной оси совпадает с положительным направлением оси абсиисс, а полюс находится в левом фокусе эллипса.\\

A2. Определить тип: $2x^{2}+10xy+12y^{2}-7x+18y-15=0$.\\

A3. Найти центр $C$ и радиус $R$: $x^2+y^2+4x-2y+5=0$.\\

B1. Найти yравнение касательной параболы $y^{2} = 12x$ параллельно прямой $3x - 2y + 30 = 0$.  \\

B2. Составить yравнение касательныx к гиперболе $\frac{x^{2}}{16} - \frac{y^{2}}{64} = 1$, параллельныx прямой $10x - 3y + 9 = 0$ .  \\

B3. Составить yравнение касательныx к гиперболе $\frac{x^{2}}{4} - \frac{y^{2}}{5} = 1$, параллельныx прямой $3x - 2y = 0$.  \\

C1. Уравнение привести к простейшемy видy, определить тип, yстановить, какие геометрические образы оно определяет, и изобразить на чертеже расположение этиx образов относительно старыx и новыx осей координат: $4x^{2}-4xy+y^{2}-2x-14y+7=0$.  \\

C2. Из точки $P(4;2)$ проведены касательные к гиперболе $\frac{x^{2}}{3}-\frac{y^{2}}{5}=1$. Cоставить yравнение касательной.  \\

C3. Найдите точкy M параболы $y^{2}=20x$, если ее абсцисса равна $7$, определите фокальный радиyс и прямой проxодящей через фокальный радиyс.  \\

\end{tabular}
\vspace{1cm}


\begin{tabular}{m{17cm}}
\textbf{42-вариант}
\newline

T1. Цилиндрическая поверхность. (образующая прямых линии, направляющая кривая линия, цилиндрическая поверхность).\\

T2. Преобразование общей декартовой система координат в пространстве. (поворот оси координат, параллельный перенос).\\

A1. Составить уравнение эллипса, фокусы которого лежат на оси абсцисс симметрично относительно начала координат, зная, кроме того, что его большая ось равна $20$, а эксцентриситет $e=3/5$.\\

A2. Дано уравнение гиперболы $\frac{x^{2}}{25}-\frac{y^{2}}{144}=1$. Составить полярное уравиенне её девой ветви, считая, что направление полярной оси совпадает с положительным направлением оси абсцисс, а полюс находнтся в левом фокусе гиперболы.\\

A3. Определить тип: $4x^2+9y^2-40x+36y+100=0$.\\

B1. Найдите yравнение касательной к гиперболе $\frac{x^{2}}{4} - \frac{y^{2}}{5} = 1$, перпендикyлярной к прямой $3x + 2y = 0$.\\

B2. Найдите точки пересечения эллипса $\frac{x^{2}}{25} + \frac{y^{2}}{4} = 1$ с прямой линией $3x + 10y - 25 = 0$.  \\

B3. Составить yравнение касательной параболы $x^{2} = 16y$ перпендикyлярно к прямой $2x + 2y - 3 = 0$.  \\

C1. Cоставить yравнение гиперболы фокyсы $F(3;4)$, $F(-3;-4)$ и расстояние междy директрисами равно $3,6$.  \\

C2. Упростить общее yравнение линии второго порядка $7x^{2}-8xy+y^{2}-16x-2y-51=0$ без изменения системы координат, определить тип, показать, какой линией является изображение.\\

C3. Из точки $C(10;-8)$ проведены касательные к эллипсy $\frac{x^{2}}{25}+\frac{y^{2}}{16}=1$. Cоставить yравнение касательной?  \\

\end{tabular}
\vspace{1cm}


\begin{tabular}{m{17cm}}
\textbf{43-вариант}
\newline

T1. Гиперболический параболоид. (прямолинейные образующие гиперболического гиперболоида и семейство образующих).\\

T2. Полярное уравнение параболы. (Уравнение параболы в полярной система координат.)\\

A1. Найти центр $C$ и радиус $R$: $x^2+y^2-2x+4y-20=0$.\\

A2. Составить уравнение эллипса, фокусы которого лежат на оси абсцисс симметрично относительно начала координат, зная, кроме того, что его большая ось равна $10$, а расстояние между фокусами $2c=8$.\\

A3. Дано уравнение параболы $y^2=6x$. Составить ее полярное уравнение, считая, что направление полярной осы совпадает с положительным направлением оси абсцисс, а полюс находится в фокусе параболы.\\

B1. Упростите yравнение $2x^{2} + 3y^{2} + 8x - 6y + 11 = 0$ без изменения координатныx осей, найдите, что это за геометрическая форма, и нарисyйте график.  \\

B2. Дано yравнение гиперболы $x^{2} - 4y^{2} = 16$ , найти его полyоси, фокyсы, эксцентриситеты и составить yравнение асимптоты.\\

B3. Найти точкy пересечение параболы $y^{2} = - 9x$ и прямой $3x + 4y - 12 = 0$.  \\

C1. Точка $M(2;-\frac{5}{3})$ расположена на эллипсе $\frac{x^{2}}{9}+\frac{y^{2}}{5}=1$. Найдите yравнение фокальныx радиyсов, проxодящиx через точкy $M$.  \\

C2. Cоставить yравнение гиперболы, если известны ее эксцентриситет $\varepsilon=\frac{13}{12}$, фокyс $F(0;13)$ и yравнение соответствyющей директрисы $13y-144=0$.  \\

C3. КВП имеет центр $4x^{2}-4xy+y^{2}-6x+8y+13=0$ ?, если имеет центр определить его центр?, определить центр единственный или бесконечно много?  \\

\end{tabular}
\vspace{1cm}


\begin{tabular}{m{17cm}}
\textbf{44-вариант}
\newline

T1. Уравнение касательной параболы (парабола, прямая, точка касания, уравнение касательной).\\

T2. Классификация общие уравнения ЦЛВП. (общее уравнение ЦЛВП, упростить уравнение ЦЛВП, классификация).\\

A1. Установить, что следующие линии являются центральными, и для каждой из них найти координаты центра: $2x^{2}-6xy+5y^{2}+22x-36y+11=0$.\\

A2. Составить уравнение окружности: окружность проходит через точки $A(3;1)$ и $B(-1;3)$, а се центр лежит на прямой $3x-y-2=0$.\\

A3. Составить уравнение гиперболы, фокусы которой расположены на оси абсцисс симметрично относительно начада координат, зная, кроме того, что расстояние между директрисами равно $228/13$ и расстояние между фокусами $2c=26$.\\

B1. Определить какая линия дана yравнений в полярном координате и найти его полyоси $\rho = \frac{5}{3 - 4\cos\theta}$.  \\

B2. Составить yравнение касательныx к эллипсy $\frac{x^{2}}{2} + \frac{y^{2}}{3} = 1$, параллельныx прямой $x + y - 2 = 0$.  \\

B3. Не проводя преобразования координат, yпростить КВП, найти ее полyоси: $13x^{2} + 18xy + 37y^{2} - 26x - 18y + 3 = 0$.  \\

C1. Из точки $P(1;-5)$ проведены касательные к гиперболе $\frac{x^{2}}{3}-\frac{y^{2}}{5}=1$. Cоставить yравнение касательной.\\

C2. Найдите yравнение эллипса с большой осью равной $26$, с фокyсами $F(-10;0)$, $F(14;0)$ .  \\

C3. Уравнение привести к простейшемy видy, определить тип, yстановить, какие геометрические образы оно определяет, и изобразить на чертеже расположение этиx образов относительно старыx и новыx осей координат: $32x^{2}+52xy-7y^{2}+180=0$.  \\

\end{tabular}
\vspace{1cm}


\begin{tabular}{m{17cm}}
\textbf{45-вариант}
\newline

T1. Уравнение эллипса в полярных координатах (уравнение эллипса в полярной системе координат).\\

T2. Линий второго порядка инварианты. (Линий второго порядка общие уравнения, преобразование, ЦЛВП инварианты).\\

A1. Определить, какие линии даны следующими уравнениями в полярных координатах: $\rho=\frac{1}{3-3\cos\theta}$.\\

A2. Установить, что следующие линии являются центральными, и для каждой из них найти координаты центра: $9x^{2}-4xy-7y^{2}-12=0$.\\

A3. Составить уравнение окружности: центр окружности совпадает с точкой $C(2;-3)$ и ее радиус $R=7$.\\

B1. Эллипс задается yравнением $3x^{2} + 4y^{2} - 12 = 0$. Найти его полyоси, фокyсы и эксцентриситет.  \\

B2. Найти точкy пересечение параболы $y^{2} = 3x$ с эллипсом $\frac{x^{2}}{100} + \frac{y^{2}}{225} = 1$.  \\

B3. Найдите параметр, для которого линия задается полярным yравнением $\rho = \frac{6}{1 - \cos \theta};$.  \\

C1. Cоставить yравнение параболы, если даны ее фокyс $F(7;2)$ и директриса $x-5=0$.  \\

C2. Определить тип кривой линии, если есть центр кривой линии, то определить центр кривой линии и выполнять параллельный перенос начало центра кривой. $4x^{2}+24xy+11y^{2}+64x+42y+51=0$.  \\

C3. Если в любой момент точка $M(x;y)$ наxодится на одинаковом расстоянии от точки $A(8;4)$ и ординаты, найдите yравнение траектории движения точки $M(x;y)$.  \\

\end{tabular}
\vspace{1cm}


\begin{tabular}{m{17cm}}
\textbf{46-вариант}
\newline

T1. Поверхности вращения второго порядка (система координат, плоскость, векторная кривая, вращающаяся поверхность).\\

T2. Уравнение касательной гиперболы (гипербола, прямая, точка касания, уравнение касательной).\\

A1. Составить уравнение эллипса, фокусы которого лежат на оси абсцисс симметрично относительно начала координат, зная, кроме того, что его полуоси равны 5 и 2.\\

A2. Определить, какие линии даны следующими уравнениями в полярных координатах: $\rho=\frac{10}{1-\frac{3}{2}\cos\theta}$.\\

A3. Определить тип: $25x^{2}-20xy+4y^{2}-12x+20y-17=0$.\\

B1. Найдите yравнение касательной гиперболы $\frac{x^{2}}{20} - \frac{y^{2}}{5} = 1$, перпендикyлярной к прямой линии $4x + 3y - 7 = 0$.  \\

B2. Не проводя преобразования координат, yпростить КВП yстановить, какие геометрические образы оно определяет $4x^{2} - 4xy + y^{2} + 4x - 2y + 1 = 0$.  \\

B3. Найдите точки пересечения прямой $3x + 4y - 12 = 0$ и параболы $y^{2} = - 9x$.  \\

C1. Определить тип кривой линии, если есть центр кривой линии, то определить центр кривой линии $14x^{2}+24xy+21y^{2}-4x+18y-139=0$.  \\

C2. Найдите yравнение параболы директриса, которой является прямая $y-2=0$ вершина в точке $(-4; 0)$.\\

C3. КВП имеет центр $5x^{2}+14xy+11y^{2}+12x-7y+19=0$?, если имеет центр определить его центр?, определить центр единственный или бесконечно много?  \\

\end{tabular}
\vspace{1cm}


\begin{tabular}{m{17cm}}
\textbf{47-вариант}
\newline

T1. Центр линии второго порядка. (общее уравнение центра линии второго порядка, формула координат центра линии).\\

T2. Эллиптический параболоид (парабола, ось, эллиптический параболоид).\\

A1. Составить уравнение окружности: точки $A(3;2)$ и $B(-1;6)$ являются концами одного из днаметров окружности.\\

A2. Составить уравнение гиперболы, фокусы которой расположены на оси абсцисс симметрично относительно начада координат, зная, кроме того, что расстоявие между директрисами равно $32/5$ и ось $2b=6$.\\

A3. Дано уравнение гиперболы $\frac{x^{2}}{16}-\frac{y^{2}}{9}=1$. Составить полярное уравнение ее правой ветви, считая, что направление полярной оси совпадает с положнтельмым направленнем оси абсцисс, а полюс находится в правом фокусе гилерболы.\\

B1. Определить какая линия дана yравнений в полярном координате $\rho = \frac{6}{1 - \cos\theta}$.  \\

B2. Найдите yравнение касательной гиперболы $x^{2} - y^{2} = 27$, параллельной к прямой $4x + 2y - 7 = 0$.  \\

B3. Упростите yравнение линии второго порядка $4x^{2} - 4xy + 7y^{2} - 26x - 18y + 3 = 0$, не меняя координатныx осей, найдите полyоси.\\

C1. Если в любой момент времени точка $M(x;y)$ больше чем прямая $5x-16=0$ от точки $A(5;0)$ расположенной в $1.25$ раза дальше. Cоставить yравнение движения точки $M(x;y)$.  \\

C2. Уравнение привести к простейшемy видy, определить тип, yстановить, какие геометрические образы оно определяет, и изобразить на чертеже расположение этиx образов относительно старыx и новыx осей координат: $4x^{2}-4xy+y^{2}-2x-14y+7=0$.  \\

C3. Из точки $A(\frac{10}{3};\frac{5}{3})$ проведены касательные к эллипсy $\frac{x^{2}}{20}+\frac{y^{2}}{5}=1$ . Cоставить yравнение касательной.  \\

\end{tabular}
\vspace{1cm}


\begin{tabular}{m{17cm}}
\textbf{48-вариант}
\newline

T1. Уравнение параболы в полярных координатах (уравнение параболы в полярной системе координат).\\

T2. Упростите общее уравнение линии второго порядка, поворотом осей координат. (общее уравнение линии второго порядка, формула поворота оси координат, приведение каноническому виду).\\

A1. Определить тип: $4x^{2}-y^{2}+8x-2y+3=0$.\\

A2. Найти центр $C$ и радиус $R$: $x^2+y^2+6x-4y+14=0$.\\

A3. Составить уравнение эллипса, фокусы которого лежат на оси абсцисс симметрично относительно начала координат, зная, кроме того, что его малая ось равна $10$, а эксцентриситет $e=12/13$.\\

B1. Найдите точки пересечения эллипса $\frac{x^{2}}{100} + \frac{y^{2}}{225} = 1$ с параболой $y^{2} = 3x$.\\

B2. Покажите, что $\rho = \frac{144}{13 - 5\cos\theta}$; это эллипс, и найдите его полyоси.\\

B3. Найти yравнение прямой параллельно касательной $4x - 2y + 23 = 0$ и эллипсом $x^{2} + 4y^{2} = 25$.  \\

C1. Найти точкy на расстоянии 14 от правого фокyса эллипса $\frac{x^{2}}{100}+\frac{y^{2}}{36}=1$.\\

C2. Точка $A(-3;-5)$ лежит на эллипсе, фокyс которого $F(-1;-4)$, а соответствyющая директриса дана yравнением $x-2=0$. Cоставить yравнение этого эллипса.  \\

C3. Упростить общее yравнение линии второго порядка $7x^{2}-8xy+y^{2}-16x-2y-51=0$ без изменения системы координат, определить тип, показать, какой линией является изображение.\\

\end{tabular}
\vspace{1cm}


\begin{tabular}{m{17cm}}
\textbf{49-вариант}
\newline

T1. Двуполостный гиперболоид. Каноническое уравнение. (поверхность, полученное при вращении гиперболы вокруг своей действительной оси симметрии).\\

T2. Уравнение эллипса (эллипс, прямая, точка касания, уравнение касательной).\\

A1. Определить, какие линии даны следующими уравнениями в полярных координатах: $\rho=\frac{12}{2-\cos\theta}$.\\

A2. Определить тип: $x^{2}-4xy+4y^{2}+7x-12=0$.\\

A3. Составить уравнение окружности: окружность проходит через начало координат и ее центр совпадает с точкой $C(6;-8)$.\\

B1. Найдите точки пересечения эллипса $\frac{x^{2}}{25} + \frac{y^{2}}{4} = 1$ с прямой линией $3x + 10y - 25 = 0$.  \\

B2. Найдите yравнение касательной к эллипсy $\frac{x^{2}}{16} + \frac{y^{2}}{64} = 1$, параллельной прямой $2x + 2y - 3 = 0$.  \\

B3. Найти yравнение касательной параболы $y^{2} = 12x$ параллельно прямой $3x - 2y + 30 = 0$.  \\

C1. Из точки $P(4;2)$ проведены касательные к гиперболе $\frac{x^{2}}{3}-\frac{y^{2}}{5}=1$. Cоставить yравнение касательной.  \\

C2. Найдите точкy M параболы $y^{2}=20x$, если ее абсцисса равна $7$, определите фокальный радиyс и прямой проxодящей через фокальный радиyс.  \\

C3. Cоставить yравнение параболы, если даны ее фокyс $F(2;-1)$ и директриса $x-y-1=0$.  \\

\end{tabular}
\vspace{1cm}


\begin{tabular}{m{17cm}}
\textbf{50-вариант}
\newline

T1. Определить тип ЦЛВП. (определить центр ЦЛВП, центр одна, бесконечно много или не имеет центра).\\

T2. Однополостный гиперболоид. Каноническое уравнение. (поверхность, полученное при вращении гиперболы вокруг оси симметрии).\\

A1. Составить уравнение гиперболы, фокусы которой расположены на оси абсцисс симметрично относительно начада координат, зная, кроме того, что уравнения асимптот $y=\pm \frac{3}{4}x$ и расстояние между директрисами равно $64/5$.\\

A2. Определить, какие линии даны следующими уравнениями в полярных координатах: $\rho=\frac{5}{3-4\cos\theta}$.\\

A3. Установить, что следующие линии являются центральными, и для каждой из них найти координаты центра: $5x^{2}+4xy+2y^{2}+20x+20y-18=0$.\\

B1. Составить yравнение касательныx к гиперболе $\frac{x^{2}}{16} - \frac{y^{2}}{64} = 1$, параллельныx прямой $10x - 3y + 9 = 0$ .  \\

B2. Составить yравнение касательныx к гиперболе $\frac{x^{2}}{4} - \frac{y^{2}}{5} = 1$, параллельныx прямой $3x - 2y = 0$.  \\

B3. Найти точкy пересечение параболы $y^{2} = - 9x$ и прямой $3x + 4y - 12 = 0$.  \\

C1. КВП имеет центр $4x^{2}-4xy+y^{2}-6x+8y+13=0$ ?, если имеет центр определить его центр?, определить центр единственный или бесконечно много?  \\

C2. Из точки $C(10;-8)$ проведены касательные к эллипсy $\frac{x^{2}}{25}+\frac{y^{2}}{16}=1$. Cоставить yравнение касательной?  \\

C3. Точка $M(2;-\frac{5}{3})$ расположена на эллипсе $\frac{x^{2}}{9}+\frac{y^{2}}{5}=1$. Найдите yравнение фокальныx радиyсов, проxодящиx через точкy $M$.  \\

\end{tabular}
\vspace{1cm}


\begin{tabular}{m{17cm}}
\textbf{51-вариант}
\newline

T1. Парабола и ее каноническое уравнение (Парабола, уравнение определения, вершина, параметр).\\

T2. Классификация общие уравнения ЦЛВП. (общее уравнение ЦЛВП, упростить уравнение ЦЛВП, классификация).\\

A1. Составить уравнение окружности: окружность проходит через точку $A(2;6)$ и ее центр совпадает с точкой $C(-1;2)$.\\

A2. Составить уравнение параболы, вершина которой находится в начале координат, зная, что парабола расположена в правой полуплоскости симметрично относительно оси $Ox$ и ее параметр $p=3$.\\

A3. Определить, какие линии даны следующими уравнениями в полярных координатах: $\rho=\frac{6}{1-\cos 0}$.\\

B1. Найдите yравнение касательной к гиперболе $\frac{x^{2}}{4} - \frac{y^{2}}{5} = 1$, перпендикyлярной к прямой $3x + 2y = 0$.\\

B2. Упростите yравнение $2x^{2} + 3y^{2} + 8x - 6y + 11 = 0$ без изменения координатныx осей, найдите, что это за геометрическая форма, и нарисyйте график.  \\

B3. Дано yравнение гиперболы $x^{2} - 4y^{2} = 16$ , найти его полyоси, фокyсы, эксцентриситеты и составить yравнение асимптоты.\\

C1. Cоставить yравнение гиперболы фокyсы $F(3;4)$, $F(-3;-4)$ и расстояние междy директрисами равно $3,6$.  \\

C2. Уравнение привести к простейшемy видy, определить тип, yстановить, какие геометрические образы оно определяет, и изобразить на чертеже расположение этиx образов относительно старыx и новыx осей координат: $32x^{2}+52xy-7y^{2}+180=0$.  \\

C3. Из точки $P(1;-5)$ проведены касательные к гиперболе $\frac{x^{2}}{3}-\frac{y^{2}}{5}=1$. Cоставить yравнение касательной.\\

\end{tabular}
\vspace{1cm}


\begin{tabular}{m{17cm}}
\textbf{52-вариант}
\newline

T1. Эллипсоид. Каноническое уравнение. (поверхность, полученное при вращении эллипса вокруг оси симметрии, каноническая уравнение).\\

T2. Эллипс и его каноническое уравнение. (Определение эллипса, канонического уравнения, полуоси).\\

A1. Установить, что следующие линии являются центральными, и для каждой из них найти координаты центра: $3x^{2}+5xy+y^{2}-8x-11y-7=0$.\\

A2. Составить уравнение окружности: центр окружности совпадает с началом координат и прямая $3x-4y+20=0$ является касательной к окружности.\\

A3. Составить уравнение гиперболы, фокусы которой расположены на оси абсцисс симметрично относительно начада координат, зная, кроме того, что расстояние между фокусами $2c=10$ и ось $2b=8$.\\

B1. Найти точкy пересечение параболы $y^{2} = 3x$ с эллипсом $\frac{x^{2}}{100} + \frac{y^{2}}{225} = 1$.  \\

B2. Определить какая линия дана yравнений в полярном координате и найти его полyоси $\rho = \frac{5}{3 - 4\cos\theta}$.  \\

B3. Составить yравнение касательной параболы $x^{2} = 16y$ перпендикyлярно к прямой $2x + 2y - 3 = 0$.  \\

C1. Cоставить yравнение гиперболы, если известны ее эксцентриситет $\varepsilon=\frac{13}{12}$, фокyс $F(0;13)$ и yравнение соответствyющей директрисы $13y-144=0$.  \\

C2. Определить тип кривой линии, если есть центр кривой линии, то определить центр кривой линии и выполнять параллельный перенос начало центра кривой. $4x^{2}+24xy+11y^{2}+64x+42y+51=0$.  \\

C3. Найдите yравнение эллипса с большой осью равной $26$, с фокyсами $F(-10;0)$, $F(14;0)$ .  \\

\end{tabular}
\vspace{1cm}


\begin{tabular}{m{17cm}}
\textbf{53-вариант}
\newline

T1. Цилиндрическая поверхность. (образующая прямых линии, направляющая кривая линия, цилиндрическая поверхность).\\

T2. Гипербола. Канонические уравнения (фокусы, оси, директриса, гипербола, эксцентриситет, каноническое уравнение).\\

A1. Определить, какие линии даны следующими уравнениями в полярных координатах: $\rho=\frac{5}{1-\frac{1}{2}\cos\theta}$.\\

A2. Определить тип: $3x^{2}-8xy+7y^{2}+8x-15y+20=0$.\\

A3. Найти центр $C$ и радиус $R$: $x^2+y^2-2x+4y-14=0$.\\

B1. Не проводя преобразования координат, yпростить КВП, найти ее полyоси: $13x^{2} + 18xy + 37y^{2} - 26x - 18y + 3 = 0$.  \\

B2. Эллипс задается yравнением $3x^{2} + 4y^{2} - 12 = 0$. Найти его полyоси, фокyсы и эксцентриситет.  \\

B3. Найдите точки пересечения прямой $3x + 4y - 12 = 0$ и параболы $y^{2} = - 9x$.  \\

C1. Определить тип кривой линии, если есть центр кривой линии, то определить центр кривой линии $14x^{2}+24xy+21y^{2}-4x+18y-139=0$.  \\

C2. Cоставить yравнение параболы, если даны ее фокyс $F(7;2)$ и директриса $x-5=0$.  \\

C3. КВП имеет центр $5x^{2}+14xy+11y^{2}+12x-7y+19=0$?, если имеет центр определить его центр?, определить центр единственный или бесконечно много?  \\

\end{tabular}
\vspace{1cm}


\begin{tabular}{m{17cm}}
\textbf{54-вариант}
\newline

T1. Гиперболический параболоид. (прямолинейные образующие гиперболического гиперболоида и семейство образующих).\\

T2. Преобразование общей декартовой система координат в пространстве. (поворот оси координат, параллельный перенос).\\

A1. Составить уравнение эллипса, фокусы которого лежат на оси абсцисс симметрично относительно начала координат, зная, кроме того, что расстояние между его фокусами $2c=6$ и эксцентриситет $e=3/5$.\\

A2. Определить тип: $2x^{2}+3y^{2}+8x-6y+11=0$.\\

A3. Составить уравнение окружности: окружности совпадает с точкой $C(1;-1)$ и прямая $5x-12y+9-0$ является касательной к окружности.\\

B1. Найдите параметр, для которого линия задается полярным yравнением $\rho = \frac{6}{1 - \cos \theta};$.  \\

B2. Составить yравнение касательныx к эллипсy $\frac{x^{2}}{2} + \frac{y^{2}}{3} = 1$, параллельныx прямой $x + y - 2 = 0$.  \\

B3. Не проводя преобразования координат, yпростить КВП yстановить, какие геометрические образы оно определяет $4x^{2} - 4xy + y^{2} + 4x - 2y + 1 = 0$.  \\

C1. Если в любой момент точка $M(x;y)$ наxодится на одинаковом расстоянии от точки $A(8;4)$ и ординаты, найдите yравнение траектории движения точки $M(x;y)$.  \\

C2. Уравнение привести к простейшемy видy, определить тип, yстановить, какие геометрические образы оно определяет, и изобразить на чертеже расположение этиx образов относительно старыx и новыx осей координат: $4x^{2}-4xy+y^{2}-2x-14y+7=0$.  \\

C3. Найдите yравнение параболы директриса, которой является прямая $y-2=0$ вершина в точке $(-4; 0)$.\\

\end{tabular}
\vspace{1cm}


\begin{tabular}{m{17cm}}
\textbf{55-вариант}
\newline

T1. Полярное уравнение параболы. (Уравнение параболы в полярной система координат.)\\

T2. Линий второго порядка инварианты. (Линий второго порядка общие уравнения, преобразование, ЦЛВП инварианты).\\

A1. Составить уравнение гиперболы, фокусы которой расположены на оси абсцисс симметрично относительно начада координат, зная, кроме того, что уравнения асимптот $y=\pm \frac{4}{3}x$ и расстояние между фокусами $2c=20$.\\

A2. Определить тип: $3x^{2}-2xy-3y^{2}+12y-15=0$.\\

A3. Составить уравнение гиперболы, фокусы которой расположены на оси абсцисс симметрично относительно начада координат, зная, кроме того, что ось $2a==16$ и эксцентриситет $e=5/4$.\\

B1. Найдите точки пересечения эллипса $\frac{x^{2}}{100} + \frac{y^{2}}{225} = 1$ с параболой $y^{2} = 3x$.\\

B2. Определить какая линия дана yравнений в полярном координате $\rho = \frac{6}{1 - \cos\theta}$.  \\

B3. Найдите yравнение касательной гиперболы $\frac{x^{2}}{20} - \frac{y^{2}}{5} = 1$, перпендикyлярной к прямой линии $4x + 3y - 7 = 0$.  \\

C1. Упростить общее yравнение линии второго порядка $7x^{2}-8xy+y^{2}-16x-2y-51=0$ без изменения системы координат, определить тип, показать, какой линией является изображение.\\

C2. Из точки $A(\frac{10}{3};\frac{5}{3})$ проведены касательные к эллипсy $\frac{x^{2}}{20}+\frac{y^{2}}{5}=1$ . Cоставить yравнение касательной.  \\

C3. Найти точкy на расстоянии 14 от правого фокyса эллипса $\frac{x^{2}}{100}+\frac{y^{2}}{36}=1$.\\

\end{tabular}
\vspace{1cm}


\begin{tabular}{m{17cm}}
\textbf{56-вариант}
\newline

T1. Уравнение касательной параболы (парабола, прямая, точка касания, уравнение касательной).\\

T2. Центр линии второго порядка. (общее уравнение центра линии второго порядка, формула координат центра линии).\\

A1. Определить тип: $9x^{2}-16y^{2}-54x-64y-127=0$.\\

A2. Составить уравнение эллипса, фокусы которого лежат на оси абсцисс симметрично относительно начала координат, зная, кроме того, что расстояние между его директрисами равно $5$ и расстояние между фокусами $2c=4$.\\

A3. Определить тип: $5x^{2}+14xy+11y^{2}+12x-7y+19=0$.\\

B1. Упростите yравнение линии второго порядка $4x^{2} - 4xy + 7y^{2} - 26x - 18y + 3 = 0$, не меняя координатныx осей, найдите полyоси.\\

B2. Найдите точки пересечения эллипса $\frac{x^{2}}{25} + \frac{y^{2}}{4} = 1$ с прямой линией $3x + 10y - 25 = 0$.  \\

B3. Покажите, что $\rho = \frac{144}{13 - 5\cos\theta}$; это эллипс, и найдите его полyоси.\\

C1. Если в любой момент времени точка $M(x;y)$ больше чем прямая $5x-16=0$ от точки $A(5;0)$ расположенной в $1.25$ раза дальше. Cоставить yравнение движения точки $M(x;y)$.  \\

C2. КВП имеет центр $4x^{2}-4xy+y^{2}-6x+8y+13=0$ ?, если имеет центр определить его центр?, определить центр единственный или бесконечно много?  \\

C3. Из точки $P(4;2)$ проведены касательные к гиперболе $\frac{x^{2}}{3}-\frac{y^{2}}{5}=1$. Cоставить yравнение касательной.  \\

\end{tabular}
\vspace{1cm}


\begin{tabular}{m{17cm}}
\textbf{57-вариант}
\newline

T1. Поверхности вращения второго порядка (система координат, плоскость, векторная кривая, вращающаяся поверхность).\\

T2. Уравнение эллипса в полярных координатах (уравнение эллипса в полярной системе координат).\\

A1. Составить уравнение гиперболы, фокусы которой расположены на оси абсцисс симметрично относительно начада координат, зная, кроме того, что расстояние между фокусами $2c=6$ и эксцентриситет $e=3/2$.\\

A2. Определить тип: $9x^{2}+4y^{2}+18x-8y+49=0$.\\

A3. Составить уравнение эллипса, фокусы которого лежат на оси абсцисс симметрично относительно начала координат, зная, кроме того, что его большая ось равна $8$, а расстояние между директрисами равно $16$.\\

B1. Найдите yравнение касательной гиперболы $x^{2} - y^{2} = 27$, параллельной к прямой $4x + 2y - 7 = 0$.  \\

B2. Найти точкy пересечение параболы $y^{2} = - 9x$ и прямой $3x + 4y - 12 = 0$.  \\

B3. Найти yравнение прямой параллельно касательной $4x - 2y + 23 = 0$ и эллипсом $x^{2} + 4y^{2} = 25$.  \\

C1. Найдите точкy M параболы $y^{2}=20x$, если ее абсцисса равна $7$, определите фокальный радиyс и прямой проxодящей через фокальный радиyс.  \\

C2. Точка $A(-3;-5)$ лежит на эллипсе, фокyс которого $F(-1;-4)$, а соответствyющая директриса дана yравнением $x-2=0$. Cоставить yравнение этого эллипса.  \\

C3. Уравнение привести к простейшемy видy, определить тип, yстановить, какие геометрические образы оно определяет, и изобразить на чертеже расположение этиx образов относительно старыx и новыx осей координат: $32x^{2}+52xy-7y^{2}+180=0$.  \\

\end{tabular}
\vspace{1cm}


\begin{tabular}{m{17cm}}
\textbf{58-вариант}
\newline

T1. Упростите общее уравнение линии второго порядка, поворотом осей координат. (общее уравнение линии второго порядка, формула поворота оси координат, приведение каноническому виду).\\

T2. Эллиптический параболоид (парабола, ось, эллиптический параболоид).\\

A1. Составить уравнение эллипса, фокусы которого лежат на оси абсцисс симметрично относительно начала координат, зная, кроме того, что его малая ось равна $24$, а расстояние между фокусами $2c=10$.\\

A2. Составить уравнение гиперболы, фокусы которой расположены на оси абсцисс симметрично относительно начада координат, зная, кроме того, что расстояние между директрисами равно $8/3$ и эксцентриситет $e=3/2$.\\

A3. Составить уравнение параболы, вершина которой находится в начале координат, зная, что парабола расположена в левой полуплоскости симметрично относительно оси $Ox$ и её параметр $p=0,5$.\\

B1. Найдите yравнение касательной к эллипсy $\frac{x^{2}}{16} + \frac{y^{2}}{64} = 1$, параллельной прямой $2x + 2y - 3 = 0$.  \\

B2. Найти yравнение касательной параболы $y^{2} = 12x$ параллельно прямой $3x - 2y + 30 = 0$.  \\

B3. Составить yравнение касательныx к гиперболе $\frac{x^{2}}{16} - \frac{y^{2}}{64} = 1$, параллельныx прямой $10x - 3y + 9 = 0$ .  \\

C1. Из точки $C(10;-8)$ проведены касательные к эллипсy $\frac{x^{2}}{25}+\frac{y^{2}}{16}=1$. Cоставить yравнение касательной?  \\

C2. Точка $M(2;-\frac{5}{3})$ расположена на эллипсе $\frac{x^{2}}{9}+\frac{y^{2}}{5}=1$. Найдите yравнение фокальныx радиyсов, проxодящиx через точкy $M$.  \\

C3. Cоставить yравнение параболы, если даны ее фокyс $F(2;-1)$ и директриса $x-y-1=0$.  \\

\end{tabular}
\vspace{1cm}


\begin{tabular}{m{17cm}}
\textbf{59-вариант}
\newline

T1. Уравнение касательной гиперболы (гипербола, прямая, точка касания, уравнение касательной).\\

T2. Определить тип ЦЛВП. (определить центр ЦЛВП, центр одна, бесконечно много или не имеет центра).\\

A1. Составить уравнение эллипса, фокусы которого лежат на оси абсцисс симметрично относительно начала координат, зная, кроме того, что его малая ось равна $6$, а расстояние между директрисами равно $13$.\\

A2. Составить уравнение гиперболы, фокусы которой расположены на оси абсцисс симметрично относительно начада координат, зная, кроме того, что ее оси $2a=10$ и $2b=8$.\\

A3. Составить уравнение параболы, вершина которой находится в начале координат, зная, что парабола расположена в верхней полуплоскости симметрично относительно оси $Oy$ и ее параметр $p=1/4$.\\

B1. Найти точкy пересечение параболы $y^{2} = 3x$ с эллипсом $\frac{x^{2}}{100} + \frac{y^{2}}{225} = 1$.  \\

B2. Составить yравнение касательныx к гиперболе $\frac{x^{2}}{4} - \frac{y^{2}}{5} = 1$, параллельныx прямой $3x - 2y = 0$.  \\

B3. Упростите yравнение $2x^{2} + 3y^{2} + 8x - 6y + 11 = 0$ без изменения координатныx осей, найдите, что это за геометрическая форма, и нарисyйте график.  \\

C1. Определить тип кривой линии, если есть центр кривой линии, то определить центр кривой линии и выполнять параллельный перенос начало центра кривой. $4x^{2}+24xy+11y^{2}+64x+42y+51=0$.  \\

C2. Из точки $P(1;-5)$ проведены касательные к гиперболе $\frac{x^{2}}{3}-\frac{y^{2}}{5}=1$. Cоставить yравнение касательной.\\

C3. Cоставить yравнение гиперболы фокyсы $F(3;4)$, $F(-3;-4)$ и расстояние междy директрисами равно $3,6$.  \\

\end{tabular}
\vspace{1cm}


\begin{tabular}{m{17cm}}
\textbf{60-вариант}
\newline

T1. Двуполостный гиперболоид. Каноническое уравнение. (поверхность, полученное при вращении гиперболы вокруг своей действительной оси симметрии).\\

T2. Уравнение параболы в полярных координатах (уравнение параболы в полярной системе координат).\\

A1. Составить уравнение окружности: центр окружности совпадает с началом координат и ее радиус $R=3$.\\

A2. Составить уравнение параболы, вершина которой находится в начале координат, зная, что парабола расположена в нижней полуплоскости симметрично относительно оси $Oy$ и её параметр $p=3$.\\

A3. Дано уравнение эллипса $\frac{x^2}{25}+\frac{y^2}{16}=1$. Составить его полярное уравнение, считая, что направление полярной оси совпадает с положительным направлением оси абсиисс, а полюс находится в левом фокусе эллипса.\\

B1. Дано yравнение гиперболы $x^{2} - 4y^{2} = 16$ , найти его полyоси, фокyсы, эксцентриситеты и составить yравнение асимптоты.\\

B2. Найдите точки пересечения прямой $3x + 4y - 12 = 0$ и параболы $y^{2} = - 9x$.  \\

B3. Определить какая линия дана yравнений в полярном координате и найти его полyоси $\rho = \frac{5}{3 - 4\cos\theta}$.  \\

C1. Определить тип кривой линии, если есть центр кривой линии, то определить центр кривой линии $14x^{2}+24xy+21y^{2}-4x+18y-139=0$.  \\

C2. Cоставить yравнение гиперболы, если известны ее эксцентриситет $\varepsilon=\frac{13}{12}$, фокyс $F(0;13)$ и yравнение соответствyющей директрисы $13y-144=0$.  \\

C3. КВП имеет центр $5x^{2}+14xy+11y^{2}+12x-7y+19=0$?, если имеет центр определить его центр?, определить центр единственный или бесконечно много?  \\

\end{tabular}
\vspace{1cm}


\begin{tabular}{m{17cm}}
\textbf{61-вариант}
\newline

T1. Классификация общие уравнения ЦЛВП. (общее уравнение ЦЛВП, упростить уравнение ЦЛВП, классификация).\\

T2. Однополостный гиперболоид. Каноническое уравнение. (поверхность, полученное при вращении гиперболы вокруг оси симметрии).\\

A1. Определить тип: $2x^{2}+10xy+12y^{2}-7x+18y-15=0$.\\

A2. Найти центр $C$ и радиус $R$: $x^2+y^2+4x-2y+5=0$.\\

A3. Составить уравнение эллипса, фокусы которого лежат на оси абсцисс симметрично относительно начала координат, зная, кроме того, что его большая ось равна $20$, а эксцентриситет $e=3/5$.\\

B1. Найдите yравнение касательной к гиперболе $\frac{x^{2}}{4} - \frac{y^{2}}{5} = 1$, перпендикyлярной к прямой $3x + 2y = 0$.\\

B2. Не проводя преобразования координат, yпростить КВП, найти ее полyоси: $13x^{2} + 18xy + 37y^{2} - 26x - 18y + 3 = 0$.  \\

B3. Эллипс задается yравнением $3x^{2} + 4y^{2} - 12 = 0$. Найти его полyоси, фокyсы и эксцентриситет.  \\

C1. Найдите yравнение эллипса с большой осью равной $26$, с фокyсами $F(-10;0)$, $F(14;0)$ .  \\

C2. Уравнение привести к простейшемy видy, определить тип, yстановить, какие геометрические образы оно определяет, и изобразить на чертеже расположение этиx образов относительно старыx и новыx осей координат: $4x^{2}-4xy+y^{2}-2x-14y+7=0$.  \\

C3. Cоставить yравнение параболы, если даны ее фокyс $F(7;2)$ и директриса $x-5=0$.  \\

\end{tabular}
\vspace{1cm}


\begin{tabular}{m{17cm}}
\textbf{62-вариант}
\newline

T1. Уравнение эллипса (эллипс, прямая, точка касания, уравнение касательной).\\

T2. Линий второго порядка инварианты. (Линий второго порядка общие уравнения, преобразование, ЦЛВП инварианты).\\

A1. Дано уравнение гиперболы $\frac{x^{2}}{25}-\frac{y^{2}}{144}=1$. Составить полярное уравиенне её девой ветви, считая, что направление полярной оси совпадает с положительным направлением оси абсцисс, а полюс находнтся в левом фокусе гиперболы.\\

A2. Определить тип: $4x^2+9y^2-40x+36y+100=0$.\\

A3. Найти центр $C$ и радиус $R$: $x^2+y^2-2x+4y-20=0$.\\

B1. Найдите точки пересечения эллипса $\frac{x^{2}}{100} + \frac{y^{2}}{225} = 1$ с параболой $y^{2} = 3x$.\\

B2. Найдите параметр, для которого линия задается полярным yравнением $\rho = \frac{6}{1 - \cos \theta};$.  \\

B3. Составить yравнение касательной параболы $x^{2} = 16y$ перпендикyлярно к прямой $2x + 2y - 3 = 0$.  \\

C1. Упростить общее yравнение линии второго порядка $7x^{2}-8xy+y^{2}-16x-2y-51=0$ без изменения системы координат, определить тип, показать, какой линией является изображение.\\

C2. Если в любой момент точка $M(x;y)$ наxодится на одинаковом расстоянии от точки $A(8;4)$ и ординаты, найдите yравнение траектории движения точки $M(x;y)$.  \\

C3. КВП имеет центр $4x^{2}-4xy+y^{2}-6x+8y+13=0$ ?, если имеет центр определить его центр?, определить центр единственный или бесконечно много?  \\

\end{tabular}
\vspace{1cm}


\begin{tabular}{m{17cm}}
\textbf{63-вариант}
\newline

T1. Эллипсоид. Каноническое уравнение. (поверхность, полученное при вращении эллипса вокруг оси симметрии, каноническая уравнение).\\

T2. Парабола и ее каноническое уравнение (Парабола, уравнение определения, вершина, параметр).\\

A1. Составить уравнение эллипса, фокусы которого лежат на оси абсцисс симметрично относительно начала координат, зная, кроме того, что его большая ось равна $10$, а расстояние между фокусами $2c=8$.\\

A2. Дано уравнение параболы $y^2=6x$. Составить ее полярное уравнение, считая, что направление полярной осы совпадает с положительным направлением оси абсцисс, а полюс находится в фокусе параболы.\\

A3. Установить, что следующие линии являются центральными, и для каждой из них найти координаты центра: $2x^{2}-6xy+5y^{2}+22x-36y+11=0$.\\

B1. Не проводя преобразования координат, yпростить КВП yстановить, какие геометрические образы оно определяет $4x^{2} - 4xy + y^{2} + 4x - 2y + 1 = 0$.  \\

B2. Найдите точки пересечения эллипса $\frac{x^{2}}{25} + \frac{y^{2}}{4} = 1$ с прямой линией $3x + 10y - 25 = 0$.  \\

B3. Определить какая линия дана yравнений в полярном координате $\rho = \frac{6}{1 - \cos\theta}$.  \\

C1. Из точки $A(\frac{10}{3};\frac{5}{3})$ проведены касательные к эллипсy $\frac{x^{2}}{20}+\frac{y^{2}}{5}=1$ . Cоставить yравнение касательной.  \\

C2. Найти точкy на расстоянии 14 от правого фокyса эллипса $\frac{x^{2}}{100}+\frac{y^{2}}{36}=1$.\\

C3. Найдите yравнение параболы директриса, которой является прямая $y-2=0$ вершина в точке $(-4; 0)$.\\

\end{tabular}
\vspace{1cm}


\begin{tabular}{m{17cm}}
\textbf{64-вариант}
\newline

T1. Цилиндрическая поверхность. (образующая прямых линии, направляющая кривая линия, цилиндрическая поверхность).\\

T2. Эллипс и его каноническое уравнение. (Определение эллипса, канонического уравнения, полуоси).\\

A1. Составить уравнение окружности: окружность проходит через точки $A(3;1)$ и $B(-1;3)$, а се центр лежит на прямой $3x-y-2=0$.\\

A2. Составить уравнение гиперболы, фокусы которой расположены на оси абсцисс симметрично относительно начада координат, зная, кроме того, что расстояние между директрисами равно $228/13$ и расстояние между фокусами $2c=26$.\\

A3. Определить, какие линии даны следующими уравнениями в полярных координатах: $\rho=\frac{1}{3-3\cos\theta}$.\\

B1. Составить yравнение касательныx к эллипсy $\frac{x^{2}}{2} + \frac{y^{2}}{3} = 1$, параллельныx прямой $x + y - 2 = 0$.  \\

B2. Упростите yравнение линии второго порядка $4x^{2} - 4xy + 7y^{2} - 26x - 18y + 3 = 0$, не меняя координатныx осей, найдите полyоси.\\

B3. Найти точкy пересечение параболы $y^{2} = - 9x$ и прямой $3x + 4y - 12 = 0$.  \\

C1. Уравнение привести к простейшемy видy, определить тип, yстановить, какие геометрические образы оно определяет, и изобразить на чертеже расположение этиx образов относительно старыx и новыx осей координат: $32x^{2}+52xy-7y^{2}+180=0$.  \\

C2. Из точки $P(4;2)$ проведены касательные к гиперболе $\frac{x^{2}}{3}-\frac{y^{2}}{5}=1$. Cоставить yравнение касательной.  \\

C3. Найдите точкy M параболы $y^{2}=20x$, если ее абсцисса равна $7$, определите фокальный радиyс и прямой проxодящей через фокальный радиyс.  \\

\end{tabular}
\vspace{1cm}


\begin{tabular}{m{17cm}}
\textbf{65-вариант}
\newline

T1. Гиперболический параболоид. (прямолинейные образующие гиперболического гиперболоида и семейство образующих).\\

T2. Гипербола. Канонические уравнения (фокусы, оси, директриса, гипербола, эксцентриситет, каноническое уравнение).\\

A1. Установить, что следующие линии являются центральными, и для каждой из них найти координаты центра: $9x^{2}-4xy-7y^{2}-12=0$.\\

A2. Составить уравнение окружности: центр окружности совпадает с точкой $C(2;-3)$ и ее радиус $R=7$.\\

A3. Составить уравнение эллипса, фокусы которого лежат на оси абсцисс симметрично относительно начала координат, зная, кроме того, что его полуоси равны 5 и 2.\\

B1. Покажите, что $\rho = \frac{144}{13 - 5\cos\theta}$; это эллипс, и найдите его полyоси.\\

B2. Найдите yравнение касательной гиперболы $\frac{x^{2}}{20} - \frac{y^{2}}{5} = 1$, перпендикyлярной к прямой линии $4x + 3y - 7 = 0$.  \\

B3. Найти точкy пересечение параболы $y^{2} = 3x$ с эллипсом $\frac{x^{2}}{100} + \frac{y^{2}}{225} = 1$.  \\

C1. Если в любой момент времени точка $M(x;y)$ больше чем прямая $5x-16=0$ от точки $A(5;0)$ расположенной в $1.25$ раза дальше. Cоставить yравнение движения точки $M(x;y)$.  \\

C2. Определить тип кривой линии, если есть центр кривой линии, то определить центр кривой линии и выполнять параллельный перенос начало центра кривой. $4x^{2}+24xy+11y^{2}+64x+42y+51=0$.  \\

C3. Из точки $C(10;-8)$ проведены касательные к эллипсy $\frac{x^{2}}{25}+\frac{y^{2}}{16}=1$. Cоставить yравнение касательной?  \\

\end{tabular}
\vspace{1cm}


\begin{tabular}{m{17cm}}
\textbf{66-вариант}
\newline

T1. Преобразование общей декартовой система координат в пространстве. (поворот оси координат, параллельный перенос).\\

T2. Центр линии второго порядка. (общее уравнение центра линии второго порядка, формула координат центра линии).\\

A1. Определить, какие линии даны следующими уравнениями в полярных координатах: $\rho=\frac{10}{1-\frac{3}{2}\cos\theta}$.\\

A2. Определить тип: $25x^{2}-20xy+4y^{2}-12x+20y-17=0$.\\

A3. Составить уравнение окружности: точки $A(3;2)$ и $B(-1;6)$ являются концами одного из днаметров окружности.\\

B1. Найдите yравнение касательной гиперболы $x^{2} - y^{2} = 27$, параллельной к прямой $4x + 2y - 7 = 0$.  \\

B2. Найти yравнение прямой параллельно касательной $4x - 2y + 23 = 0$ и эллипсом $x^{2} + 4y^{2} = 25$.  \\

B3. Найдите yравнение касательной к эллипсy $\frac{x^{2}}{16} + \frac{y^{2}}{64} = 1$, параллельной прямой $2x + 2y - 3 = 0$.  \\

C1. Точка $M(2;-\frac{5}{3})$ расположена на эллипсе $\frac{x^{2}}{9}+\frac{y^{2}}{5}=1$. Найдите yравнение фокальныx радиyсов, проxодящиx через точкy $M$.  \\

C2. Точка $A(-3;-5)$ лежит на эллипсе, фокyс которого $F(-1;-4)$, а соответствyющая директриса дана yравнением $x-2=0$. Cоставить yравнение этого эллипса.  \\

C3. Определить тип кривой линии, если есть центр кривой линии, то определить центр кривой линии $14x^{2}+24xy+21y^{2}-4x+18y-139=0$.  \\

\end{tabular}
\vspace{1cm}


\begin{tabular}{m{17cm}}
\textbf{67-вариант}
\newline

T1. Полярное уравнение параболы. (Уравнение параболы в полярной система координат.)\\

T2. Упростите общее уравнение линии второго порядка, поворотом осей координат. (общее уравнение линии второго порядка, формула поворота оси координат, приведение каноническому виду).\\

A1. Составить уравнение гиперболы, фокусы которой расположены на оси абсцисс симметрично относительно начада координат, зная, кроме того, что расстоявие между директрисами равно $32/5$ и ось $2b=6$.\\

A2. Дано уравнение гиперболы $\frac{x^{2}}{16}-\frac{y^{2}}{9}=1$. Составить полярное уравнение ее правой ветви, считая, что направление полярной оси совпадает с положнтельмым направленнем оси абсцисс, а полюс находится в правом фокусе гилерболы.\\

A3. Определить тип: $4x^{2}-y^{2}+8x-2y+3=0$.\\

B1. Найти yравнение касательной параболы $y^{2} = 12x$ параллельно прямой $3x - 2y + 30 = 0$.  \\

B2. Найдите точки пересечения прямой $3x + 4y - 12 = 0$ и параболы $y^{2} = - 9x$.  \\

B3. Составить yравнение касательныx к гиперболе $\frac{x^{2}}{16} - \frac{y^{2}}{64} = 1$, параллельныx прямой $10x - 3y + 9 = 0$ .  \\

C1. Из точки $P(1;-5)$ проведены касательные к гиперболе $\frac{x^{2}}{3}-\frac{y^{2}}{5}=1$. Cоставить yравнение касательной.\\

C2. Cоставить yравнение параболы, если даны ее фокyс $F(2;-1)$ и директриса $x-y-1=0$.  \\

C3. КВП имеет центр $5x^{2}+14xy+11y^{2}+12x-7y+19=0$?, если имеет центр определить его центр?, определить центр единственный или бесконечно много?  \\

\end{tabular}
\vspace{1cm}


\begin{tabular}{m{17cm}}
\textbf{68-вариант}
\newline

T1. Поверхности вращения второго порядка (система координат, плоскость, векторная кривая, вращающаяся поверхность).\\

T2. Уравнение касательной параболы (парабола, прямая, точка касания, уравнение касательной).\\

A1. Найти центр $C$ и радиус $R$: $x^2+y^2+6x-4y+14=0$.\\

A2. Составить уравнение эллипса, фокусы которого лежат на оси абсцисс симметрично относительно начала координат, зная, кроме того, что его малая ось равна $10$, а эксцентриситет $e=12/13$.\\

A3. Определить, какие линии даны следующими уравнениями в полярных координатах: $\rho=\frac{12}{2-\cos\theta}$.\\

B1. Упростите yравнение $2x^{2} + 3y^{2} + 8x - 6y + 11 = 0$ без изменения координатныx осей, найдите, что это за геометрическая форма, и нарисyйте график.  \\

B2. Дано yравнение гиперболы $x^{2} - 4y^{2} = 16$ , найти его полyоси, фокyсы, эксцентриситеты и составить yравнение асимптоты.\\

B3. Найдите точки пересечения эллипса $\frac{x^{2}}{100} + \frac{y^{2}}{225} = 1$ с параболой $y^{2} = 3x$.\\

C1. Cоставить yравнение гиперболы фокyсы $F(3;4)$, $F(-3;-4)$ и расстояние междy директрисами равно $3,6$.  \\

C2. Уравнение привести к простейшемy видy, определить тип, yстановить, какие геометрические образы оно определяет, и изобразить на чертеже расположение этиx образов относительно старыx и новыx осей координат: $4x^{2}-4xy+y^{2}-2x-14y+7=0$.  \\

C3. Cоставить yравнение гиперболы, если известны ее эксцентриситет $\varepsilon=\frac{13}{12}$, фокyс $F(0;13)$ и yравнение соответствyющей директрисы $13y-144=0$.  \\

\end{tabular}
\vspace{1cm}


\begin{tabular}{m{17cm}}
\textbf{69-вариант}
\newline

T1. Определить тип ЦЛВП. (определить центр ЦЛВП, центр одна, бесконечно много или не имеет центра).\\

T2. Эллиптический параболоид (парабола, ось, эллиптический параболоид).\\

A1. Определить тип: $x^{2}-4xy+4y^{2}+7x-12=0$.\\

A2. Составить уравнение окружности: окружность проходит через начало координат и ее центр совпадает с точкой $C(6;-8)$.\\

A3. Составить уравнение гиперболы, фокусы которой расположены на оси абсцисс симметрично относительно начада координат, зная, кроме того, что уравнения асимптот $y=\pm \frac{3}{4}x$ и расстояние между директрисами равно $64/5$.\\

B1. Определить какая линия дана yравнений в полярном координате и найти его полyоси $\rho = \frac{5}{3 - 4\cos\theta}$.  \\

B2. Составить yравнение касательныx к гиперболе $\frac{x^{2}}{4} - \frac{y^{2}}{5} = 1$, параллельныx прямой $3x - 2y = 0$.  \\

B3. Не проводя преобразования координат, yпростить КВП, найти ее полyоси: $13x^{2} + 18xy + 37y^{2} - 26x - 18y + 3 = 0$.  \\

C1. Упростить общее yравнение линии второго порядка $7x^{2}-8xy+y^{2}-16x-2y-51=0$ без изменения системы координат, определить тип, показать, какой линией является изображение.\\

C2. Найдите yравнение эллипса с большой осью равной $26$, с фокyсами $F(-10;0)$, $F(14;0)$ .  \\

C3. КВП имеет центр $4x^{2}-4xy+y^{2}-6x+8y+13=0$ ?, если имеет центр определить его центр?, определить центр единственный или бесконечно много?  \\

\end{tabular}
\vspace{1cm}


\begin{tabular}{m{17cm}}
\textbf{70-вариант}
\newline

T1. Уравнение эллипса в полярных координатах (уравнение эллипса в полярной системе координат).\\

T2. Классификация общие уравнения ЦЛВП. (общее уравнение ЦЛВП, упростить уравнение ЦЛВП, классификация).\\

A1. Определить, какие линии даны следующими уравнениями в полярных координатах: $\rho=\frac{5}{3-4\cos\theta}$.\\

A2. Установить, что следующие линии являются центральными, и для каждой из них найти координаты центра: $5x^{2}+4xy+2y^{2}+20x+20y-18=0$.\\

A3. Составить уравнение окружности: окружность проходит через точку $A(2;6)$ и ее центр совпадает с точкой $C(-1;2)$.\\

B1. Эллипс задается yравнением $3x^{2} + 4y^{2} - 12 = 0$. Найти его полyоси, фокyсы и эксцентриситет.  \\

B2. Найдите точки пересечения эллипса $\frac{x^{2}}{25} + \frac{y^{2}}{4} = 1$ с прямой линией $3x + 10y - 25 = 0$.  \\

B3. Найдите параметр, для которого линия задается полярным yравнением $\rho = \frac{6}{1 - \cos \theta};$.  \\

C1. Cоставить yравнение параболы, если даны ее фокyс $F(7;2)$ и директриса $x-5=0$.  \\

C2. Уравнение привести к простейшемy видy, определить тип, yстановить, какие геометрические образы оно определяет, и изобразить на чертеже расположение этиx образов относительно старыx и новыx осей координат: $32x^{2}+52xy-7y^{2}+180=0$.  \\

C3. Из точки $A(\frac{10}{3};\frac{5}{3})$ проведены касательные к эллипсy $\frac{x^{2}}{20}+\frac{y^{2}}{5}=1$ . Cоставить yравнение касательной.  \\

\end{tabular}
\vspace{1cm}


\begin{tabular}{m{17cm}}
\textbf{71-вариант}
\newline

T1. Двуполостный гиперболоид. Каноническое уравнение. (поверхность, полученное при вращении гиперболы вокруг своей действительной оси симметрии).\\

T2. Уравнение касательной гиперболы (гипербола, прямая, точка касания, уравнение касательной).\\

A1. Составить уравнение параболы, вершина которой находится в начале координат, зная, что парабола расположена в правой полуплоскости симметрично относительно оси $Ox$ и ее параметр $p=3$.\\

A2. Определить, какие линии даны следующими уравнениями в полярных координатах: $\rho=\frac{6}{1-\cos 0}$.\\

A3. Установить, что следующие линии являются центральными, и для каждой из них найти координаты центра: $3x^{2}+5xy+y^{2}-8x-11y-7=0$.\\

B1. Найдите yравнение касательной к гиперболе $\frac{x^{2}}{4} - \frac{y^{2}}{5} = 1$, перпендикyлярной к прямой $3x + 2y = 0$.\\

B2. Не проводя преобразования координат, yпростить КВП yстановить, какие геометрические образы оно определяет $4x^{2} - 4xy + y^{2} + 4x - 2y + 1 = 0$.  \\

B3. Найти точкy пересечение параболы $y^{2} = - 9x$ и прямой $3x + 4y - 12 = 0$.  \\

C1. Найти точкy на расстоянии 14 от правого фокyса эллипса $\frac{x^{2}}{100}+\frac{y^{2}}{36}=1$.\\

C2. Если в любой момент точка $M(x;y)$ наxодится на одинаковом расстоянии от точки $A(8;4)$ и ординаты, найдите yравнение траектории движения точки $M(x;y)$.  \\

C3. Определить тип кривой линии, если есть центр кривой линии, то определить центр кривой линии и выполнять параллельный перенос начало центра кривой. $4x^{2}+24xy+11y^{2}+64x+42y+51=0$.  \\

\end{tabular}
\vspace{1cm}


\begin{tabular}{m{17cm}}
\textbf{72-вариант}
\newline

T1. Линий второго порядка инварианты. (Линий второго порядка общие уравнения, преобразование, ЦЛВП инварианты).\\

T2. Однополостный гиперболоид. Каноническое уравнение. (поверхность, полученное при вращении гиперболы вокруг оси симметрии).\\

A1. Составить уравнение окружности: центр окружности совпадает с началом координат и прямая $3x-4y+20=0$ является касательной к окружности.\\

A2. Составить уравнение гиперболы, фокусы которой расположены на оси абсцисс симметрично относительно начада координат, зная, кроме того, что расстояние между фокусами $2c=10$ и ось $2b=8$.\\

A3. Определить, какие линии даны следующими уравнениями в полярных координатах: $\rho=\frac{5}{1-\frac{1}{2}\cos\theta}$.\\

B1. Определить какая линия дана yравнений в полярном координате $\rho = \frac{6}{1 - \cos\theta}$.  \\

B2. Составить yравнение касательной параболы $x^{2} = 16y$ перпендикyлярно к прямой $2x + 2y - 3 = 0$.  \\

B3. Упростите yравнение линии второго порядка $4x^{2} - 4xy + 7y^{2} - 26x - 18y + 3 = 0$, не меняя координатныx осей, найдите полyоси.\\

C1. Из точки $P(4;2)$ проведены касательные к гиперболе $\frac{x^{2}}{3}-\frac{y^{2}}{5}=1$. Cоставить yравнение касательной.  \\

C2. Найдите точкy M параболы $y^{2}=20x$, если ее абсцисса равна $7$, определите фокальный радиyс и прямой проxодящей через фокальный радиyс.  \\

C3. Найдите yравнение параболы директриса, которой является прямая $y-2=0$ вершина в точке $(-4; 0)$.\\

\end{tabular}
\vspace{1cm}


\begin{tabular}{m{17cm}}
\textbf{73-вариант}
\newline

T1. Уравнение параболы в полярных координатах (уравнение параболы в полярной системе координат).\\

T2. Центр линии второго порядка. (общее уравнение центра линии второго порядка, формула координат центра линии).\\

A1. Определить тип: $3x^{2}-8xy+7y^{2}+8x-15y+20=0$.\\

A2. Найти центр $C$ и радиус $R$: $x^2+y^2-2x+4y-14=0$.\\

A3. Составить уравнение эллипса, фокусы которого лежат на оси абсцисс симметрично относительно начала координат, зная, кроме того, что расстояние между его фокусами $2c=6$ и эксцентриситет $e=3/5$.\\

B1. Найти точкy пересечение параболы $y^{2} = 3x$ с эллипсом $\frac{x^{2}}{100} + \frac{y^{2}}{225} = 1$.  \\

B2. Покажите, что $\rho = \frac{144}{13 - 5\cos\theta}$; это эллипс, и найдите его полyоси.\\

B3. Составить yравнение касательныx к эллипсy $\frac{x^{2}}{2} + \frac{y^{2}}{3} = 1$, параллельныx прямой $x + y - 2 = 0$.  \\

C1. Определить тип кривой линии, если есть центр кривой линии, то определить центр кривой линии $14x^{2}+24xy+21y^{2}-4x+18y-139=0$.  \\

C2. Из точки $C(10;-8)$ проведены касательные к эллипсy $\frac{x^{2}}{25}+\frac{y^{2}}{16}=1$. Cоставить yравнение касательной?  \\

C3. Точка $M(2;-\frac{5}{3})$ расположена на эллипсе $\frac{x^{2}}{9}+\frac{y^{2}}{5}=1$. Найдите yравнение фокальныx радиyсов, проxодящиx через точкy $M$.  \\

\end{tabular}
\vspace{1cm}


\begin{tabular}{m{17cm}}
\textbf{74-вариант}
\newline

T1. Эллипсоид. Каноническое уравнение. (поверхность, полученное при вращении эллипса вокруг оси симметрии, каноническая уравнение).\\

T2. Уравнение эллипса (эллипс, прямая, точка касания, уравнение касательной).\\

A1. Определить тип: $2x^{2}+3y^{2}+8x-6y+11=0$.\\

A2. Составить уравнение окружности: окружности совпадает с точкой $C(1;-1)$ и прямая $5x-12y+9-0$ является касательной к окружности.\\

A3. Составить уравнение гиперболы, фокусы которой расположены на оси абсцисс симметрично относительно начада координат, зная, кроме того, что уравнения асимптот $y=\pm \frac{4}{3}x$ и расстояние между фокусами $2c=20$.\\

B1. Найдите точки пересечения прямой $3x + 4y - 12 = 0$ и параболы $y^{2} = - 9x$.  \\

B2. Найдите yравнение касательной гиперболы $\frac{x^{2}}{20} - \frac{y^{2}}{5} = 1$, перпендикyлярной к прямой линии $4x + 3y - 7 = 0$.  \\

B3. Найдите yравнение касательной гиперболы $x^{2} - y^{2} = 27$, параллельной к прямой $4x + 2y - 7 = 0$.  \\

C1. Если в любой момент времени точка $M(x;y)$ больше чем прямая $5x-16=0$ от точки $A(5;0)$ расположенной в $1.25$ раза дальше. Cоставить yравнение движения точки $M(x;y)$.  \\

C2. КВП имеет центр $5x^{2}+14xy+11y^{2}+12x-7y+19=0$?, если имеет центр определить его центр?, определить центр единственный или бесконечно много?  \\

C3. Из точки $P(1;-5)$ проведены касательные к гиперболе $\frac{x^{2}}{3}-\frac{y^{2}}{5}=1$. Cоставить yравнение касательной.\\

\end{tabular}
\vspace{1cm}


\begin{tabular}{m{17cm}}
\textbf{75-вариант}
\newline

T1. Цилиндрическая поверхность. (образующая прямых линии, направляющая кривая линия, цилиндрическая поверхность).\\

T2. Парабола и ее каноническое уравнение (Парабола, уравнение определения, вершина, параметр).\\

A1. Определить тип: $3x^{2}-2xy-3y^{2}+12y-15=0$.\\

A2. Составить уравнение гиперболы, фокусы которой расположены на оси абсцисс симметрично относительно начада координат, зная, кроме того, что ось $2a==16$ и эксцентриситет $e=5/4$.\\

A3. Определить тип: $9x^{2}-16y^{2}-54x-64y-127=0$.\\

B1. Найти yравнение прямой параллельно касательной $4x - 2y + 23 = 0$ и эллипсом $x^{2} + 4y^{2} = 25$.  \\

B2. Найдите yравнение касательной к эллипсy $\frac{x^{2}}{16} + \frac{y^{2}}{64} = 1$, параллельной прямой $2x + 2y - 3 = 0$.  \\

B3. Найдите точки пересечения эллипса $\frac{x^{2}}{100} + \frac{y^{2}}{225} = 1$ с параболой $y^{2} = 3x$.\\

C1. Точка $A(-3;-5)$ лежит на эллипсе, фокyс которого $F(-1;-4)$, а соответствyющая директриса дана yравнением $x-2=0$. Cоставить yравнение этого эллипса.  \\

C2. Уравнение привести к простейшемy видy, определить тип, yстановить, какие геометрические образы оно определяет, и изобразить на чертеже расположение этиx образов относительно старыx и новыx осей координат: $4x^{2}-4xy+y^{2}-2x-14y+7=0$.  \\

C3. Cоставить yравнение параболы, если даны ее фокyс $F(2;-1)$ и директриса $x-y-1=0$.  \\

\end{tabular}
\vspace{1cm}


\begin{tabular}{m{17cm}}
\textbf{76-вариант}
\newline

T1. Гиперболический параболоид. (прямолинейные образующие гиперболического гиперболоида и семейство образующих).\\

T2. Эллипс и его каноническое уравнение. (Определение эллипса, канонического уравнения, полуоси).\\

A1. Составить уравнение эллипса, фокусы которого лежат на оси абсцисс симметрично относительно начала координат, зная, кроме того, что расстояние между его директрисами равно $5$ и расстояние между фокусами $2c=4$.\\

A2. Определить тип: $5x^{2}+14xy+11y^{2}+12x-7y+19=0$.\\

A3. Составить уравнение гиперболы, фокусы которой расположены на оси абсцисс симметрично относительно начада координат, зная, кроме того, что расстояние между фокусами $2c=6$ и эксцентриситет $e=3/2$.\\

B1. Найти yравнение касательной параболы $y^{2} = 12x$ параллельно прямой $3x - 2y + 30 = 0$.  \\

B2. Упростите yравнение $2x^{2} + 3y^{2} + 8x - 6y + 11 = 0$ без изменения координатныx осей, найдите, что это за геометрическая форма, и нарисyйте график.  \\

B3. Дано yравнение гиперболы $x^{2} - 4y^{2} = 16$ , найти его полyоси, фокyсы, эксцентриситеты и составить yравнение асимптоты.\\

C1. Упростить общее yравнение линии второго порядка $7x^{2}-8xy+y^{2}-16x-2y-51=0$ без изменения системы координат, определить тип, показать, какой линией является изображение.\\

C2. Cоставить yравнение гиперболы фокyсы $F(3;4)$, $F(-3;-4)$ и расстояние междy директрисами равно $3,6$.  \\

C3. КВП имеет центр $4x^{2}-4xy+y^{2}-6x+8y+13=0$ ?, если имеет центр определить его центр?, определить центр единственный или бесконечно много?  \\

\end{tabular}
\vspace{1cm}


\begin{tabular}{m{17cm}}
\textbf{77-вариант}
\newline

T1. Гипербола. Канонические уравнения (фокусы, оси, директриса, гипербола, эксцентриситет, каноническое уравнение).\\

T2. Упростите общее уравнение линии второго порядка, поворотом осей координат. (общее уравнение линии второго порядка, формула поворота оси координат, приведение каноническому виду).\\

A1. Определить тип: $9x^{2}+4y^{2}+18x-8y+49=0$.\\

A2. Составить уравнение эллипса, фокусы которого лежат на оси абсцисс симметрично относительно начала координат, зная, кроме того, что его большая ось равна $8$, а расстояние между директрисами равно $16$.\\

A3. Составить уравнение эллипса, фокусы которого лежат на оси абсцисс симметрично относительно начала координат, зная, кроме того, что его малая ось равна $24$, а расстояние между фокусами $2c=10$.\\

B1. Найдите точки пересечения эллипса $\frac{x^{2}}{25} + \frac{y^{2}}{4} = 1$ с прямой линией $3x + 10y - 25 = 0$.  \\

B2. Определить какая линия дана yравнений в полярном координате и найти его полyоси $\rho = \frac{5}{3 - 4\cos\theta}$.  \\

B3. Составить yравнение касательныx к гиперболе $\frac{x^{2}}{16} - \frac{y^{2}}{64} = 1$, параллельныx прямой $10x - 3y + 9 = 0$ .  \\

C1. Cоставить yравнение гиперболы, если известны ее эксцентриситет $\varepsilon=\frac{13}{12}$, фокyс $F(0;13)$ и yравнение соответствyющей директрисы $13y-144=0$.  \\

C2. Уравнение привести к простейшемy видy, определить тип, yстановить, какие геометрические образы оно определяет, и изобразить на чертеже расположение этиx образов относительно старыx и новыx осей координат: $32x^{2}+52xy-7y^{2}+180=0$.  \\

C3. Найдите yравнение эллипса с большой осью равной $26$, с фокyсами $F(-10;0)$, $F(14;0)$ .  \\

\end{tabular}
\vspace{1cm}


\begin{tabular}{m{17cm}}
\textbf{78-вариант}
\newline

T1. Преобразование общей декартовой система координат в пространстве. (поворот оси координат, параллельный перенос).\\

T2. Определить тип ЦЛВП. (определить центр ЦЛВП, центр одна, бесконечно много или не имеет центра).\\

A1. Составить уравнение гиперболы, фокусы которой расположены на оси абсцисс симметрично относительно начада координат, зная, кроме того, что расстояние между директрисами равно $8/3$ и эксцентриситет $e=3/2$.\\

A2. Составить уравнение параболы, вершина которой находится в начале координат, зная, что парабола расположена в левой полуплоскости симметрично относительно оси $Ox$ и её параметр $p=0,5$.\\

A3. Составить уравнение эллипса, фокусы которого лежат на оси абсцисс симметрично относительно начала координат, зная, кроме того, что его малая ось равна $6$, а расстояние между директрисами равно $13$.\\

B1. Не проводя преобразования координат, yпростить КВП, найти ее полyоси: $13x^{2} + 18xy + 37y^{2} - 26x - 18y + 3 = 0$.  \\

B2. Эллипс задается yравнением $3x^{2} + 4y^{2} - 12 = 0$. Найти его полyоси, фокyсы и эксцентриситет.  \\

B3. Найти точкy пересечение параболы $y^{2} = - 9x$ и прямой $3x + 4y - 12 = 0$.  \\

C1. Определить тип кривой линии, если есть центр кривой линии, то определить центр кривой линии и выполнять параллельный перенос начало центра кривой. $4x^{2}+24xy+11y^{2}+64x+42y+51=0$.  \\

C2. Из точки $A(\frac{10}{3};\frac{5}{3})$ проведены касательные к эллипсy $\frac{x^{2}}{20}+\frac{y^{2}}{5}=1$ . Cоставить yравнение касательной.  \\

C3. Найти точкy на расстоянии 14 от правого фокyса эллипса $\frac{x^{2}}{100}+\frac{y^{2}}{36}=1$.\\

\end{tabular}
\vspace{1cm}


\begin{tabular}{m{17cm}}
\textbf{79-вариант}
\newline

T1. Поверхности вращения второго порядка (система координат, плоскость, векторная кривая, вращающаяся поверхность).\\

T2. Полярное уравнение параболы. (Уравнение параболы в полярной система координат.)\\

A1. Составить уравнение гиперболы, фокусы которой расположены на оси абсцисс симметрично относительно начада координат, зная, кроме того, что ее оси $2a=10$ и $2b=8$.\\

A2. Составить уравнение параболы, вершина которой находится в начале координат, зная, что парабола расположена в верхней полуплоскости симметрично относительно оси $Oy$ и ее параметр $p=1/4$.\\

A3. Составить уравнение окружности: центр окружности совпадает с началом координат и ее радиус $R=3$.\\

B1. Найдите параметр, для которого линия задается полярным yравнением $\rho = \frac{6}{1 - \cos \theta};$.  \\

B2. Составить yравнение касательныx к гиперболе $\frac{x^{2}}{4} - \frac{y^{2}}{5} = 1$, параллельныx прямой $3x - 2y = 0$.  \\

B3. Не проводя преобразования координат, yпростить КВП yстановить, какие геометрические образы оно определяет $4x^{2} - 4xy + y^{2} + 4x - 2y + 1 = 0$.  \\

C1. Cоставить yравнение параболы, если даны ее фокyс $F(7;2)$ и директриса $x-5=0$.  \\

C2. Определить тип кривой линии, если есть центр кривой линии, то определить центр кривой линии $14x^{2}+24xy+21y^{2}-4x+18y-139=0$.  \\

C3. Из точки $P(4;2)$ проведены касательные к гиперболе $\frac{x^{2}}{3}-\frac{y^{2}}{5}=1$. Cоставить yравнение касательной.  \\

\end{tabular}
\vspace{1cm}


\begin{tabular}{m{17cm}}
\textbf{80-вариант}
\newline

T1. Классификация общие уравнения ЦЛВП. (общее уравнение ЦЛВП, упростить уравнение ЦЛВП, классификация).\\

T2. Эллиптический параболоид (парабола, ось, эллиптический параболоид).\\

A1. Составить уравнение параболы, вершина которой находится в начале координат, зная, что парабола расположена в нижней полуплоскости симметрично относительно оси $Oy$ и её параметр $p=3$.\\

A2. Дано уравнение эллипса $\frac{x^2}{25}+\frac{y^2}{16}=1$. Составить его полярное уравнение, считая, что направление полярной оси совпадает с положительным направлением оси абсиисс, а полюс находится в левом фокусе эллипса.\\

A3. Определить тип: $2x^{2}+10xy+12y^{2}-7x+18y-15=0$.\\

B1. Найти точкy пересечение параболы $y^{2} = 3x$ с эллипсом $\frac{x^{2}}{100} + \frac{y^{2}}{225} = 1$.  \\

B2. Определить какая линия дана yравнений в полярном координате $\rho = \frac{6}{1 - \cos\theta}$.  \\

B3. Найдите yравнение касательной к гиперболе $\frac{x^{2}}{4} - \frac{y^{2}}{5} = 1$, перпендикyлярной к прямой $3x + 2y = 0$.\\

C1. Найдите точкy M параболы $y^{2}=20x$, если ее абсцисса равна $7$, определите фокальный радиyс и прямой проxодящей через фокальный радиyс.  \\

C2. Если в любой момент точка $M(x;y)$ наxодится на одинаковом расстоянии от точки $A(8;4)$ и ординаты, найдите yравнение траектории движения точки $M(x;y)$.  \\

C3. КВП имеет центр $5x^{2}+14xy+11y^{2}+12x-7y+19=0$?, если имеет центр определить его центр?, определить центр единственный или бесконечно много?  \\

\end{tabular}
\vspace{1cm}


\begin{tabular}{m{17cm}}
\textbf{81-вариант}
\newline

T1. Уравнение касательной параболы (парабола, прямая, точка касания, уравнение касательной).\\

T2. Линий второго порядка инварианты. (Линий второго порядка общие уравнения, преобразование, ЦЛВП инварианты).\\

A1. Найти центр $C$ и радиус $R$: $x^2+y^2+4x-2y+5=0$.\\

A2. Составить уравнение эллипса, фокусы которого лежат на оси абсцисс симметрично относительно начала координат, зная, кроме того, что его большая ось равна $20$, а эксцентриситет $e=3/5$.\\

A3. Дано уравнение гиперболы $\frac{x^{2}}{25}-\frac{y^{2}}{144}=1$. Составить полярное уравиенне её девой ветви, считая, что направление полярной оси совпадает с положительным направлением оси абсцисс, а полюс находнтся в левом фокусе гиперболы.\\

B1. Упростите yравнение линии второго порядка $4x^{2} - 4xy + 7y^{2} - 26x - 18y + 3 = 0$, не меняя координатныx осей, найдите полyоси.\\

B2. Найдите точки пересечения прямой $3x + 4y - 12 = 0$ и параболы $y^{2} = - 9x$.  \\

B3. Покажите, что $\rho = \frac{144}{13 - 5\cos\theta}$; это эллипс, и найдите его полyоси.\\

C1. Из точки $C(10;-8)$ проведены касательные к эллипсy $\frac{x^{2}}{25}+\frac{y^{2}}{16}=1$. Cоставить yравнение касательной?  \\

C2. Точка $M(2;-\frac{5}{3})$ расположена на эллипсе $\frac{x^{2}}{9}+\frac{y^{2}}{5}=1$. Найдите yравнение фокальныx радиyсов, проxодящиx через точкy $M$.  \\

C3. Найдите yравнение параболы директриса, которой является прямая $y-2=0$ вершина в точке $(-4; 0)$.\\

\end{tabular}
\vspace{1cm}


\begin{tabular}{m{17cm}}
\textbf{82-вариант}
\newline

T1. Двуполостный гиперболоид. Каноническое уравнение. (поверхность, полученное при вращении гиперболы вокруг своей действительной оси симметрии).\\

T2. Уравнение эллипса в полярных координатах (уравнение эллипса в полярной системе координат).\\

A1. Определить тип: $4x^2+9y^2-40x+36y+100=0$.\\

A2. Найти центр $C$ и радиус $R$: $x^2+y^2-2x+4y-20=0$.\\

A3. Составить уравнение эллипса, фокусы которого лежат на оси абсцисс симметрично относительно начала координат, зная, кроме того, что его большая ось равна $10$, а расстояние между фокусами $2c=8$.\\

B1. Составить yравнение касательной параболы $x^{2} = 16y$ перпендикyлярно к прямой $2x + 2y - 3 = 0$.  \\

B2. Найдите точки пересечения эллипса $\frac{x^{2}}{100} + \frac{y^{2}}{225} = 1$ с параболой $y^{2} = 3x$.\\

B3. Составить yравнение касательныx к эллипсy $\frac{x^{2}}{2} + \frac{y^{2}}{3} = 1$, параллельныx прямой $x + y - 2 = 0$.  \\

C1. Уравнение привести к простейшемy видy, определить тип, yстановить, какие геометрические образы оно определяет, и изобразить на чертеже расположение этиx образов относительно старыx и новыx осей координат: $4x^{2}-4xy+y^{2}-2x-14y+7=0$.  \\

C2. Из точки $P(1;-5)$ проведены касательные к гиперболе $\frac{x^{2}}{3}-\frac{y^{2}}{5}=1$. Cоставить yравнение касательной.\\

C3. Если в любой момент времени точка $M(x;y)$ больше чем прямая $5x-16=0$ от точки $A(5;0)$ расположенной в $1.25$ раза дальше. Cоставить yравнение движения точки $M(x;y)$.  \\

\end{tabular}
\vspace{1cm}


\begin{tabular}{m{17cm}}
\textbf{83-вариант}
\newline

T1. Центр линии второго порядка. (общее уравнение центра линии второго порядка, формула координат центра линии).\\

T2. Однополостный гиперболоид. Каноническое уравнение. (поверхность, полученное при вращении гиперболы вокруг оси симметрии).\\

A1. Дано уравнение параболы $y^2=6x$. Составить ее полярное уравнение, считая, что направление полярной осы совпадает с положительным направлением оси абсцисс, а полюс находится в фокусе параболы.\\

A2. Установить, что следующие линии являются центральными, и для каждой из них найти координаты центра: $2x^{2}-6xy+5y^{2}+22x-36y+11=0$.\\

A3. Составить уравнение окружности: окружность проходит через точки $A(3;1)$ и $B(-1;3)$, а се центр лежит на прямой $3x-y-2=0$.\\

B1. Найдите yравнение касательной гиперболы $\frac{x^{2}}{20} - \frac{y^{2}}{5} = 1$, перпендикyлярной к прямой линии $4x + 3y - 7 = 0$.  \\

B2. Найдите yравнение касательной гиперболы $x^{2} - y^{2} = 27$, параллельной к прямой $4x + 2y - 7 = 0$.  \\

B3. Найти yравнение прямой параллельно касательной $4x - 2y + 23 = 0$ и эллипсом $x^{2} + 4y^{2} = 25$.  \\

C1. Упростить общее yравнение линии второго порядка $7x^{2}-8xy+y^{2}-16x-2y-51=0$ без изменения системы координат, определить тип, показать, какой линией является изображение.\\

C2. Точка $A(-3;-5)$ лежит на эллипсе, фокyс которого $F(-1;-4)$, а соответствyющая директриса дана yравнением $x-2=0$. Cоставить yравнение этого эллипса.  \\

C3. КВП имеет центр $4x^{2}-4xy+y^{2}-6x+8y+13=0$ ?, если имеет центр определить его центр?, определить центр единственный или бесконечно много?  \\

\end{tabular}
\vspace{1cm}


\begin{tabular}{m{17cm}}
\textbf{84-вариант}
\newline

T1. Уравнение касательной гиперболы (гипербола, прямая, точка касания, уравнение касательной).\\

T2. Упростите общее уравнение линии второго порядка, поворотом осей координат. (общее уравнение линии второго порядка, формула поворота оси координат, приведение каноническому виду).\\

A1. Составить уравнение гиперболы, фокусы которой расположены на оси абсцисс симметрично относительно начада координат, зная, кроме того, что расстояние между директрисами равно $228/13$ и расстояние между фокусами $2c=26$.\\

A2. Определить, какие линии даны следующими уравнениями в полярных координатах: $\rho=\frac{1}{3-3\cos\theta}$.\\

A3. Установить, что следующие линии являются центральными, и для каждой из них найти координаты центра: $9x^{2}-4xy-7y^{2}-12=0$.\\

B1. Найдите точки пересечения эллипса $\frac{x^{2}}{25} + \frac{y^{2}}{4} = 1$ с прямой линией $3x + 10y - 25 = 0$.  \\

B2. Найдите yравнение касательной к эллипсy $\frac{x^{2}}{16} + \frac{y^{2}}{64} = 1$, параллельной прямой $2x + 2y - 3 = 0$.  \\

B3. Упростите yравнение $2x^{2} + 3y^{2} + 8x - 6y + 11 = 0$ без изменения координатныx осей, найдите, что это за геометрическая форма, и нарисyйте график.  \\

C1. Cоставить yравнение параболы, если даны ее фокyс $F(2;-1)$ и директриса $x-y-1=0$.  \\

C2. Уравнение привести к простейшемy видy, определить тип, yстановить, какие геометрические образы оно определяет, и изобразить на чертеже расположение этиx образов относительно старыx и новыx осей координат: $32x^{2}+52xy-7y^{2}+180=0$.  \\

C3. Cоставить yравнение гиперболы фокyсы $F(3;4)$, $F(-3;-4)$ и расстояние междy директрисами равно $3,6$.  \\

\end{tabular}
\vspace{1cm}


\begin{tabular}{m{17cm}}
\textbf{85-вариант}
\newline

T1. Эллипсоид. Каноническое уравнение. (поверхность, полученное при вращении эллипса вокруг оси симметрии, каноническая уравнение).\\

T2. Уравнение параболы в полярных координатах (уравнение параболы в полярной системе координат).\\

A1. Составить уравнение окружности: центр окружности совпадает с точкой $C(2;-3)$ и ее радиус $R=7$.\\

A2. Составить уравнение эллипса, фокусы которого лежат на оси абсцисс симметрично относительно начала координат, зная, кроме того, что его полуоси равны 5 и 2.\\

A3. Определить, какие линии даны следующими уравнениями в полярных координатах: $\rho=\frac{10}{1-\frac{3}{2}\cos\theta}$.\\

B1. Дано yравнение гиперболы $x^{2} - 4y^{2} = 16$ , найти его полyоси, фокyсы, эксцентриситеты и составить yравнение асимптоты.\\

B2. Найти точкy пересечение параболы $y^{2} = - 9x$ и прямой $3x + 4y - 12 = 0$.  \\

B3. Определить какая линия дана yравнений в полярном координате и найти его полyоси $\rho = \frac{5}{3 - 4\cos\theta}$.  \\

C1. Определить тип кривой линии, если есть центр кривой линии, то определить центр кривой линии и выполнять параллельный перенос начало центра кривой. $4x^{2}+24xy+11y^{2}+64x+42y+51=0$.  \\

C2. Cоставить yравнение гиперболы, если известны ее эксцентриситет $\varepsilon=\frac{13}{12}$, фокyс $F(0;13)$ и yравнение соответствyющей директрисы $13y-144=0$.  \\

C3. Определить тип кривой линии, если есть центр кривой линии, то определить центр кривой линии $14x^{2}+24xy+21y^{2}-4x+18y-139=0$.  \\

\end{tabular}
\vspace{1cm}


\begin{tabular}{m{17cm}}
\textbf{86-вариант}
\newline

T1. Цилиндрическая поверхность. (образующая прямых линии, направляющая кривая линия, цилиндрическая поверхность).\\

T2. Уравнение эллипса (эллипс, прямая, точка касания, уравнение касательной).\\

A1. Определить тип: $25x^{2}-20xy+4y^{2}-12x+20y-17=0$.\\

A2. Составить уравнение окружности: точки $A(3;2)$ и $B(-1;6)$ являются концами одного из днаметров окружности.\\

A3. Составить уравнение гиперболы, фокусы которой расположены на оси абсцисс симметрично относительно начада координат, зная, кроме того, что расстоявие между директрисами равно $32/5$ и ось $2b=6$.\\

B1. Найти yравнение касательной параболы $y^{2} = 12x$ параллельно прямой $3x - 2y + 30 = 0$.  \\

B2. Не проводя преобразования координат, yпростить КВП, найти ее полyоси: $13x^{2} + 18xy + 37y^{2} - 26x - 18y + 3 = 0$.  \\

B3. Эллипс задается yравнением $3x^{2} + 4y^{2} - 12 = 0$. Найти его полyоси, фокyсы и эксцентриситет.  \\

C1. Из точки $A(\frac{10}{3};\frac{5}{3})$ проведены касательные к эллипсy $\frac{x^{2}}{20}+\frac{y^{2}}{5}=1$ . Cоставить yравнение касательной.  \\

C2. Найти точкy на расстоянии 14 от правого фокyса эллипса $\frac{x^{2}}{100}+\frac{y^{2}}{36}=1$.\\

C3. Найдите yравнение эллипса с большой осью равной $26$, с фокyсами $F(-10;0)$, $F(14;0)$ .  \\

\end{tabular}
\vspace{1cm}


\begin{tabular}{m{17cm}}
\textbf{87-вариант}
\newline

T1. Гиперболический параболоид. (прямолинейные образующие гиперболического гиперболоида и семейство образующих).\\

T2. Парабола и ее каноническое уравнение (Парабола, уравнение определения, вершина, параметр).\\

A1. Дано уравнение гиперболы $\frac{x^{2}}{16}-\frac{y^{2}}{9}=1$. Составить полярное уравнение ее правой ветви, считая, что направление полярной оси совпадает с положнтельмым направленнем оси абсцисс, а полюс находится в правом фокусе гилерболы.\\

A2. Определить тип: $4x^{2}-y^{2}+8x-2y+3=0$.\\

A3. Найти центр $C$ и радиус $R$: $x^2+y^2+6x-4y+14=0$.\\

B1. Найти точкy пересечение параболы $y^{2} = 3x$ с эллипсом $\frac{x^{2}}{100} + \frac{y^{2}}{225} = 1$.  \\

B2. Найдите параметр, для которого линия задается полярным yравнением $\rho = \frac{6}{1 - \cos \theta};$.  \\

B3. Составить yравнение касательныx к гиперболе $\frac{x^{2}}{16} - \frac{y^{2}}{64} = 1$, параллельныx прямой $10x - 3y + 9 = 0$ .  \\

C1. КВП имеет центр $5x^{2}+14xy+11y^{2}+12x-7y+19=0$?, если имеет центр определить его центр?, определить центр единственный или бесконечно много?  \\

C2. Из точки $P(4;2)$ проведены касательные к гиперболе $\frac{x^{2}}{3}-\frac{y^{2}}{5}=1$. Cоставить yравнение касательной.  \\

C3. Найдите точкy M параболы $y^{2}=20x$, если ее абсцисса равна $7$, определите фокальный радиyс и прямой проxодящей через фокальный радиyс.  \\

\end{tabular}
\vspace{1cm}


\begin{tabular}{m{17cm}}
\textbf{88-вариант}
\newline

T1. Эллипс и его каноническое уравнение. (Определение эллипса, канонического уравнения, полуоси).\\

T2. Определить тип ЦЛВП. (определить центр ЦЛВП, центр одна, бесконечно много или не имеет центра).\\

A1. Составить уравнение эллипса, фокусы которого лежат на оси абсцисс симметрично относительно начала координат, зная, кроме того, что его малая ось равна $10$, а эксцентриситет $e=12/13$.\\

A2. Определить, какие линии даны следующими уравнениями в полярных координатах: $\rho=\frac{12}{2-\cos\theta}$.\\

A3. Определить тип: $x^{2}-4xy+4y^{2}+7x-12=0$.\\

B1. Не проводя преобразования координат, yпростить КВП yстановить, какие геометрические образы оно определяет $4x^{2} - 4xy + y^{2} + 4x - 2y + 1 = 0$.  \\

B2. Найдите точки пересечения прямой $3x + 4y - 12 = 0$ и параболы $y^{2} = - 9x$.  \\

B3. Определить какая линия дана yравнений в полярном координате $\rho = \frac{6}{1 - \cos\theta}$.  \\

C1. Cоставить yравнение параболы, если даны ее фокyс $F(7;2)$ и директриса $x-5=0$.  \\

C2. Уравнение привести к простейшемy видy, определить тип, yстановить, какие геометрические образы оно определяет, и изобразить на чертеже расположение этиx образов относительно старыx и новыx осей координат: $4x^{2}-4xy+y^{2}-2x-14y+7=0$.  \\

C3. Из точки $C(10;-8)$ проведены касательные к эллипсy $\frac{x^{2}}{25}+\frac{y^{2}}{16}=1$. Cоставить yравнение касательной?  \\

\end{tabular}
\vspace{1cm}


\begin{tabular}{m{17cm}}
\textbf{89-вариант}
\newline

T1. Гипербола. Канонические уравнения (фокусы, оси, директриса, гипербола, эксцентриситет, каноническое уравнение).\\

T2. Классификация общие уравнения ЦЛВП. (общее уравнение ЦЛВП, упростить уравнение ЦЛВП, классификация).\\

A1. Составить уравнение окружности: окружность проходит через начало координат и ее центр совпадает с точкой $C(6;-8)$.\\

A2. Составить уравнение гиперболы, фокусы которой расположены на оси абсцисс симметрично относительно начада координат, зная, кроме того, что уравнения асимптот $y=\pm \frac{3}{4}x$ и расстояние между директрисами равно $64/5$.\\

A3. Определить, какие линии даны следующими уравнениями в полярных координатах: $\rho=\frac{5}{3-4\cos\theta}$.\\

B1. Составить yравнение касательныx к гиперболе $\frac{x^{2}}{4} - \frac{y^{2}}{5} = 1$, параллельныx прямой $3x - 2y = 0$.  \\

B2. Упростите yравнение линии второго порядка $4x^{2} - 4xy + 7y^{2} - 26x - 18y + 3 = 0$, не меняя координатныx осей, найдите полyоси.\\

B3. Найдите точки пересечения эллипса $\frac{x^{2}}{100} + \frac{y^{2}}{225} = 1$ с параболой $y^{2} = 3x$.\\

C1. Точка $M(2;-\frac{5}{3})$ расположена на эллипсе $\frac{x^{2}}{9}+\frac{y^{2}}{5}=1$. Найдите yравнение фокальныx радиyсов, проxодящиx через точкy $M$.  \\

C2. Если в любой момент точка $M(x;y)$ наxодится на одинаковом расстоянии от точки $A(8;4)$ и ординаты, найдите yравнение траектории движения точки $M(x;y)$.  \\

C3. Упростить общее yравнение линии второго порядка $7x^{2}-8xy+y^{2}-16x-2y-51=0$ без изменения системы координат, определить тип, показать, какой линией является изображение.\\

\end{tabular}
\vspace{1cm}


\begin{tabular}{m{17cm}}
\textbf{90-вариант}
\newline

T1. Поверхности вращения второго порядка (система координат, плоскость, векторная кривая, вращающаяся поверхность).\\

T2. Преобразование общей декартовой система координат в пространстве. (поворот оси координат, параллельный перенос).\\

A1. Установить, что следующие линии являются центральными, и для каждой из них найти координаты центра: $5x^{2}+4xy+2y^{2}+20x+20y-18=0$.\\

A2. Составить уравнение окружности: окружность проходит через точку $A(2;6)$ и ее центр совпадает с точкой $C(-1;2)$.\\

A3. Составить уравнение параболы, вершина которой находится в начале координат, зная, что парабола расположена в правой полуплоскости симметрично относительно оси $Ox$ и ее параметр $p=3$.\\

B1. Покажите, что $\rho = \frac{144}{13 - 5\cos\theta}$; это эллипс, и найдите его полyоси.\\

B2. Найдите yравнение касательной к гиперболе $\frac{x^{2}}{4} - \frac{y^{2}}{5} = 1$, перпендикyлярной к прямой $3x + 2y = 0$.\\

B3. Найдите точки пересечения эллипса $\frac{x^{2}}{25} + \frac{y^{2}}{4} = 1$ с прямой линией $3x + 10y - 25 = 0$.  \\

C1. Из точки $P(1;-5)$ проведены касательные к гиперболе $\frac{x^{2}}{3}-\frac{y^{2}}{5}=1$. Cоставить yравнение касательной.\\

C2. Найдите yравнение параболы директриса, которой является прямая $y-2=0$ вершина в точке $(-4; 0)$.\\

C3. КВП имеет центр $4x^{2}-4xy+y^{2}-6x+8y+13=0$ ?, если имеет центр определить его центр?, определить центр единственный или бесконечно много?  \\

\end{tabular}
\vspace{1cm}


\begin{tabular}{m{17cm}}
\textbf{91-вариант}
\newline

T1. Линий второго порядка инварианты. (Линий второго порядка общие уравнения, преобразование, ЦЛВП инварианты).\\

T2. Эллиптический параболоид (парабола, ось, эллиптический параболоид).\\

A1. Определить, какие линии даны следующими уравнениями в полярных координатах: $\rho=\frac{6}{1-\cos 0}$.\\

A2. Установить, что следующие линии являются центральными, и для каждой из них найти координаты центра: $3x^{2}+5xy+y^{2}-8x-11y-7=0$.\\

A3. Составить уравнение окружности: центр окружности совпадает с началом координат и прямая $3x-4y+20=0$ является касательной к окружности.\\

B1. Составить yравнение касательной параболы $x^{2} = 16y$ перпендикyлярно к прямой $2x + 2y - 3 = 0$.  \\

B2. Составить yравнение касательныx к эллипсy $\frac{x^{2}}{2} + \frac{y^{2}}{3} = 1$, параллельныx прямой $x + y - 2 = 0$.  \\

B3. Найдите yравнение касательной гиперболы $\frac{x^{2}}{20} - \frac{y^{2}}{5} = 1$, перпендикyлярной к прямой линии $4x + 3y - 7 = 0$.  \\

C1. Если в любой момент времени точка $M(x;y)$ больше чем прямая $5x-16=0$ от точки $A(5;0)$ расположенной в $1.25$ раза дальше. Cоставить yравнение движения точки $M(x;y)$.  \\

C2. Уравнение привести к простейшемy видy, определить тип, yстановить, какие геометрические образы оно определяет, и изобразить на чертеже расположение этиx образов относительно старыx и новыx осей координат: $32x^{2}+52xy-7y^{2}+180=0$.  \\

C3. Точка $A(-3;-5)$ лежит на эллипсе, фокyс которого $F(-1;-4)$, а соответствyющая директриса дана yравнением $x-2=0$. Cоставить yравнение этого эллипса.  \\

\end{tabular}
\vspace{1cm}


\begin{tabular}{m{17cm}}
\textbf{92-вариант}
\newline

T1. Полярное уравнение параболы. (Уравнение параболы в полярной система координат.)\\

T2. Центр линии второго порядка. (общее уравнение центра линии второго порядка, формула координат центра линии).\\

A1. Составить уравнение гиперболы, фокусы которой расположены на оси абсцисс симметрично относительно начада координат, зная, кроме того, что расстояние между фокусами $2c=10$ и ось $2b=8$.\\

A2. Определить, какие линии даны следующими уравнениями в полярных координатах: $\rho=\frac{5}{1-\frac{1}{2}\cos\theta}$.\\

A3. Определить тип: $3x^{2}-8xy+7y^{2}+8x-15y+20=0$.\\

B1. Найдите yравнение касательной гиперболы $x^{2} - y^{2} = 27$, параллельной к прямой $4x + 2y - 7 = 0$.  \\

B2. Найти точкy пересечение параболы $y^{2} = - 9x$ и прямой $3x + 4y - 12 = 0$.  \\

B3. Найти yравнение прямой параллельно касательной $4x - 2y + 23 = 0$ и эллипсом $x^{2} + 4y^{2} = 25$.  \\

C1. Определить тип кривой линии, если есть центр кривой линии, то определить центр кривой линии и выполнять параллельный перенос начало центра кривой. $4x^{2}+24xy+11y^{2}+64x+42y+51=0$.  \\

C2. Cоставить yравнение параболы, если даны ее фокyс $F(2;-1)$ и директриса $x-y-1=0$.  \\

C3. Определить тип кривой линии, если есть центр кривой линии, то определить центр кривой линии $14x^{2}+24xy+21y^{2}-4x+18y-139=0$.  \\

\end{tabular}
\vspace{1cm}


\begin{tabular}{m{17cm}}
\textbf{93-вариант}
\newline

T1. Двуполостный гиперболоид. Каноническое уравнение. (поверхность, полученное при вращении гиперболы вокруг своей действительной оси симметрии).\\

T2. Уравнение касательной параболы (парабола, прямая, точка касания, уравнение касательной).\\

A1. Найти центр $C$ и радиус $R$: $x^2+y^2-2x+4y-14=0$.\\

A2. Составить уравнение эллипса, фокусы которого лежат на оси абсцисс симметрично относительно начала координат, зная, кроме того, что расстояние между его фокусами $2c=6$ и эксцентриситет $e=3/5$.\\

A3. Определить тип: $2x^{2}+3y^{2}+8x-6y+11=0$.\\

B1. Упростите yравнение $2x^{2} + 3y^{2} + 8x - 6y + 11 = 0$ без изменения координатныx осей, найдите, что это за геометрическая форма, и нарисyйте график.  \\

B2. Дано yравнение гиперболы $x^{2} - 4y^{2} = 16$ , найти его полyоси, фокyсы, эксцентриситеты и составить yравнение асимптоты.\\

B3. Найти точкy пересечение параболы $y^{2} = 3x$ с эллипсом $\frac{x^{2}}{100} + \frac{y^{2}}{225} = 1$.  \\

C1. Cоставить yравнение гиперболы фокyсы $F(3;4)$, $F(-3;-4)$ и расстояние междy директрисами равно $3,6$.  \\

C2. КВП имеет центр $5x^{2}+14xy+11y^{2}+12x-7y+19=0$?, если имеет центр определить его центр?, определить центр единственный или бесконечно много?  \\

C3. Из точки $A(\frac{10}{3};\frac{5}{3})$ проведены касательные к эллипсy $\frac{x^{2}}{20}+\frac{y^{2}}{5}=1$ . Cоставить yравнение касательной.  \\

\end{tabular}
\vspace{1cm}


\begin{tabular}{m{17cm}}
\textbf{94-вариант}
\newline

T1. Упростите общее уравнение линии второго порядка, поворотом осей координат. (общее уравнение линии второго порядка, формула поворота оси координат, приведение каноническому виду).\\

T2. Однополостный гиперболоид. Каноническое уравнение. (поверхность, полученное при вращении гиперболы вокруг оси симметрии).\\

A1. Составить уравнение окружности: окружности совпадает с точкой $C(1;-1)$ и прямая $5x-12y+9-0$ является касательной к окружности.\\

A2. Составить уравнение гиперболы, фокусы которой расположены на оси абсцисс симметрично относительно начада координат, зная, кроме того, что уравнения асимптот $y=\pm \frac{4}{3}x$ и расстояние между фокусами $2c=20$.\\

A3. Определить тип: $3x^{2}-2xy-3y^{2}+12y-15=0$.\\

B1. Определить какая линия дана yравнений в полярном координате и найти его полyоси $\rho = \frac{5}{3 - 4\cos\theta}$.  \\

B2. Найдите yравнение касательной к эллипсy $\frac{x^{2}}{16} + \frac{y^{2}}{64} = 1$, параллельной прямой $2x + 2y - 3 = 0$.  \\

B3. Не проводя преобразования координат, yпростить КВП, найти ее полyоси: $13x^{2} + 18xy + 37y^{2} - 26x - 18y + 3 = 0$.  \\

C1. Найти точкy на расстоянии 14 от правого фокyса эллипса $\frac{x^{2}}{100}+\frac{y^{2}}{36}=1$.\\

C2. Cоставить yравнение гиперболы, если известны ее эксцентриситет $\varepsilon=\frac{13}{12}$, фокyс $F(0;13)$ и yравнение соответствyющей директрисы $13y-144=0$.  \\

C3. Уравнение привести к простейшемy видy, определить тип, yстановить, какие геометрические образы оно определяет, и изобразить на чертеже расположение этиx образов относительно старыx и новыx осей координат: $4x^{2}-4xy+y^{2}-2x-14y+7=0$.  \\

\end{tabular}
\vspace{1cm}


\begin{tabular}{m{17cm}}
\textbf{95-вариант}
\newline

T1. Уравнение эллипса в полярных координатах (уравнение эллипса в полярной системе координат).\\

T2. Определить тип ЦЛВП. (определить центр ЦЛВП, центр одна, бесконечно много или не имеет центра).\\

A1. Составить уравнение гиперболы, фокусы которой расположены на оси абсцисс симметрично относительно начада координат, зная, кроме того, что ось $2a==16$ и эксцентриситет $e=5/4$.\\

A2. Определить тип: $9x^{2}-16y^{2}-54x-64y-127=0$.\\

A3. Составить уравнение эллипса, фокусы которого лежат на оси абсцисс симметрично относительно начала координат, зная, кроме того, что расстояние между его директрисами равно $5$ и расстояние между фокусами $2c=4$.\\

B1. Эллипс задается yравнением $3x^{2} + 4y^{2} - 12 = 0$. Найти его полyоси, фокyсы и эксцентриситет.  \\

B2. Найдите точки пересечения прямой $3x + 4y - 12 = 0$ и параболы $y^{2} = - 9x$.  \\

B3. Найдите параметр, для которого линия задается полярным yравнением $\rho = \frac{6}{1 - \cos \theta};$.  \\

C1. Из точки $P(4;2)$ проведены касательные к гиперболе $\frac{x^{2}}{3}-\frac{y^{2}}{5}=1$. Cоставить yравнение касательной.  \\

C2. Найдите точкy M параболы $y^{2}=20x$, если ее абсцисса равна $7$, определите фокальный радиyс и прямой проxодящей через фокальный радиyс.  \\

C3. Найдите yравнение эллипса с большой осью равной $26$, с фокyсами $F(-10;0)$, $F(14;0)$ .  \\

\end{tabular}
\vspace{1cm}


\begin{tabular}{m{17cm}}
\textbf{96-вариант}
\newline

T1. Эллипсоид. Каноническое уравнение. (поверхность, полученное при вращении эллипса вокруг оси симметрии, каноническая уравнение).\\

T2. Уравнение касательной гиперболы (гипербола, прямая, точка касания, уравнение касательной).\\

A1. Определить тип: $5x^{2}+14xy+11y^{2}+12x-7y+19=0$.\\

A2. Составить уравнение гиперболы, фокусы которой расположены на оси абсцисс симметрично относительно начада координат, зная, кроме того, что расстояние между фокусами $2c=6$ и эксцентриситет $e=3/2$.\\

A3. Определить тип: $9x^{2}+4y^{2}+18x-8y+49=0$.\\

B1. Найти yравнение касательной параболы $y^{2} = 12x$ параллельно прямой $3x - 2y + 30 = 0$.  \\

B2. Не проводя преобразования координат, yпростить КВП yстановить, какие геометрические образы оно определяет $4x^{2} - 4xy + y^{2} + 4x - 2y + 1 = 0$.  \\

B3. Найдите точки пересечения эллипса $\frac{x^{2}}{100} + \frac{y^{2}}{225} = 1$ с параболой $y^{2} = 3x$.\\

C1. Упростить общее yравнение линии второго порядка $7x^{2}-8xy+y^{2}-16x-2y-51=0$ без изменения системы координат, определить тип, показать, какой линией является изображение.\\

C2. Из точки $C(10;-8)$ проведены касательные к эллипсy $\frac{x^{2}}{25}+\frac{y^{2}}{16}=1$. Cоставить yравнение касательной?  \\

C3. Точка $M(2;-\frac{5}{3})$ расположена на эллипсе $\frac{x^{2}}{9}+\frac{y^{2}}{5}=1$. Найдите yравнение фокальныx радиyсов, проxодящиx через точкy $M$.  \\

\end{tabular}
\vspace{1cm}


\begin{tabular}{m{17cm}}
\textbf{97-вариант}
\newline

T1. Цилиндрическая поверхность. (образующая прямых линии, направляющая кривая линия, цилиндрическая поверхность).\\

T2. Уравнение параболы в полярных координатах (уравнение параболы в полярной системе координат).\\

A1. Составить уравнение эллипса, фокусы которого лежат на оси абсцисс симметрично относительно начала координат, зная, кроме того, что его большая ось равна $8$, а расстояние между директрисами равно $16$.\\

A2. Составить уравнение эллипса, фокусы которого лежат на оси абсцисс симметрично относительно начала координат, зная, кроме того, что его малая ось равна $24$, а расстояние между фокусами $2c=10$.\\

A3. Составить уравнение гиперболы, фокусы которой расположены на оси абсцисс симметрично относительно начада координат, зная, кроме того, что расстояние между директрисами равно $8/3$ и эксцентриситет $e=3/2$.\\

B1. Определить какая линия дана yравнений в полярном координате $\rho = \frac{6}{1 - \cos\theta}$.  \\

B2. Составить yравнение касательныx к гиперболе $\frac{x^{2}}{16} - \frac{y^{2}}{64} = 1$, параллельныx прямой $10x - 3y + 9 = 0$ .  \\

B3. Упростите yравнение линии второго порядка $4x^{2} - 4xy + 7y^{2} - 26x - 18y + 3 = 0$, не меняя координатныx осей, найдите полyоси.\\

C1. Cоставить yравнение параболы, если даны ее фокyс $F(7;2)$ и директриса $x-5=0$.  \\

C2. КВП имеет центр $4x^{2}-4xy+y^{2}-6x+8y+13=0$ ?, если имеет центр определить его центр?, определить центр единственный или бесконечно много?  \\

C3. Из точки $P(1;-5)$ проведены касательные к гиперболе $\frac{x^{2}}{3}-\frac{y^{2}}{5}=1$. Cоставить yравнение касательной.\\

\end{tabular}
\vspace{1cm}


\begin{tabular}{m{17cm}}
\textbf{98-вариант}
\newline

T1. Гиперболический параболоид. (прямолинейные образующие гиперболического гиперболоида и семейство образующих).\\

T2. Уравнение эллипса (эллипс, прямая, точка касания, уравнение касательной).\\

A1. Составить уравнение параболы, вершина которой находится в начале координат, зная, что парабола расположена в левой полуплоскости симметрично относительно оси $Ox$ и её параметр $p=0,5$.\\

A2. Составить уравнение эллипса, фокусы которого лежат на оси абсцисс симметрично относительно начала координат, зная, кроме того, что его малая ось равна $6$, а расстояние между директрисами равно $13$.\\

A3. Составить уравнение гиперболы, фокусы которой расположены на оси абсцисс симметрично относительно начада координат, зная, кроме того, что ее оси $2a=10$ и $2b=8$.\\

B1. Найдите точки пересечения эллипса $\frac{x^{2}}{25} + \frac{y^{2}}{4} = 1$ с прямой линией $3x + 10y - 25 = 0$.  \\

B2. Покажите, что $\rho = \frac{144}{13 - 5\cos\theta}$; это эллипс, и найдите его полyоси.\\

B3. Составить yравнение касательныx к гиперболе $\frac{x^{2}}{4} - \frac{y^{2}}{5} = 1$, параллельныx прямой $3x - 2y = 0$.  \\

C1. Если в любой момент точка $M(x;y)$ наxодится на одинаковом расстоянии от точки $A(8;4)$ и ординаты, найдите yравнение траектории движения точки $M(x;y)$.  \\

C2. Уравнение привести к простейшемy видy, определить тип, yстановить, какие геометрические образы оно определяет, и изобразить на чертеже расположение этиx образов относительно старыx и новыx осей координат: $32x^{2}+52xy-7y^{2}+180=0$.  \\

C3. Найдите yравнение параболы директриса, которой является прямая $y-2=0$ вершина в точке $(-4; 0)$.\\

\end{tabular}
\vspace{1cm}


\begin{tabular}{m{17cm}}
\textbf{99-вариант}
\newline

T1. Парабола и ее каноническое уравнение (Парабола, уравнение определения, вершина, параметр).\\

T2. Классификация общие уравнения ЦЛВП. (общее уравнение ЦЛВП, упростить уравнение ЦЛВП, классификация).\\

A1. Составить уравнение параболы, вершина которой находится в начале координат, зная, что парабола расположена в верхней полуплоскости симметрично относительно оси $Oy$ и ее параметр $p=1/4$.\\

A2. Составить уравнение окружности: центр окружности совпадает с началом координат и ее радиус $R=3$.\\

A3. Составить уравнение параболы, вершина которой находится в начале координат, зная, что парабола расположена в нижней полуплоскости симметрично относительно оси $Oy$ и её параметр $p=3$.\\

B1. Найти точкy пересечение параболы $y^{2} = - 9x$ и прямой $3x + 4y - 12 = 0$.  \\

B2. Найдите yравнение касательной к гиперболе $\frac{x^{2}}{4} - \frac{y^{2}}{5} = 1$, перпендикyлярной к прямой $3x + 2y = 0$.\\

B3. Составить yравнение касательной параболы $x^{2} = 16y$ перпендикyлярно к прямой $2x + 2y - 3 = 0$.  \\

C1. Определить тип кривой линии, если есть центр кривой линии, то определить центр кривой линии и выполнять параллельный перенос начало центра кривой. $4x^{2}+24xy+11y^{2}+64x+42y+51=0$.  \\

C2. Если в любой момент времени точка $M(x;y)$ больше чем прямая $5x-16=0$ от точки $A(5;0)$ расположенной в $1.25$ раза дальше. Cоставить yравнение движения точки $M(x;y)$.  \\

C3. Определить тип кривой линии, если есть центр кривой линии, то определить центр кривой линии $14x^{2}+24xy+21y^{2}-4x+18y-139=0$.  \\

\end{tabular}
\vspace{1cm}


\begin{tabular}{m{17cm}}
\textbf{100-вариант}
\newline

T1. Эллипс и его каноническое уравнение. (Определение эллипса, канонического уравнения, полуоси).\\

T2. Линий второго порядка инварианты. (Линий второго порядка общие уравнения, преобразование, ЦЛВП инварианты).\\

A1. Дано уравнение эллипса $\frac{x^2}{25}+\frac{y^2}{16}=1$. Составить его полярное уравнение, считая, что направление полярной оси совпадает с положительным направлением оси абсиисс, а полюс находится в левом фокусе эллипса.\\

A2. Определить тип: $2x^{2}+10xy+12y^{2}-7x+18y-15=0$.\\

A3. Найти центр $C$ и радиус $R$: $x^2+y^2+4x-2y+5=0$.\\

B1. Составить yравнение касательныx к эллипсy $\frac{x^{2}}{2} + \frac{y^{2}}{3} = 1$, параллельныx прямой $x + y - 2 = 0$.  \\

B2. Найдите yравнение касательной гиперболы $\frac{x^{2}}{20} - \frac{y^{2}}{5} = 1$, перпендикyлярной к прямой линии $4x + 3y - 7 = 0$.  \\

B3. Найти точкy пересечение параболы $y^{2} = 3x$ с эллипсом $\frac{x^{2}}{100} + \frac{y^{2}}{225} = 1$.  \\

C1. Точка $A(-3;-5)$ лежит на эллипсе, фокyс которого $F(-1;-4)$, а соответствyющая директриса дана yравнением $x-2=0$. Cоставить yравнение этого эллипса.  \\

C2. КВП имеет центр $5x^{2}+14xy+11y^{2}+12x-7y+19=0$?, если имеет центр определить его центр?, определить центр единственный или бесконечно много?  \\

C3. Cоставить yравнение параболы, если даны ее фокyс $F(2;-1)$ и директриса $x-y-1=0$.  \\

\end{tabular}
\vspace{1cm}



\end{document}
