\begin{enumerate}
\item Составить уравнение окружности: центр окружности совпадает с началом координат и ее радиус $R=3$
\item Составить уравнение окружности: центр окружности совпадает с точкой $C(2;-3)$ и ее радиус$R=7$
\item Составить уравнение окружности: окружность проходит через начало координат и ее центр совпадает с точкой $C(6;-8)$
\item Составить уравнение окружности: окружность проходит через точку $A(2;6)$ и ее центр совпадает с точкой $C(-1;2)$
\item Составить уравнение окружности: точки $A(3;2)$ и $B(-1;6)$ являются концами одного из днаметров окружности
\item Составить уравнение окружности: центр окружности совпадает с началом координат и прямая $3x-4y+20=0$ является касательной к окружности
\item Составить уравнение окружности: окружности совпадает с точкой $C(1;-1)$ и прямая $5x-12y+9-0$ является касательной к окружности
\item Составить уравнение окружности: окружность проходит через точки $A(3;1)$ и $B(-1;3)$, а се центр лежит на прямой $3x-y-2=0$
\item Составить уравнение окружности: окружность проходит через три точки $A(1;1)$, $B(1;-1)$ и $C(2;0)$
\item Составить уравнение окружности: окружность проходит через три точки: $M_1(-1;5)$, $M_2(-2;-2)$ и $M_3(5;5)$
\item Найти центр $C$ и радиус $R$: $x^2+y^2-2x+4y-20=0$
\item Найти центр $C$ и радиус $R$: $x^2+y^2-2x+4y-14=0$
\item Найти центр $C$ и радиус $R$: $x^2+y^2+4x-2y+5=0$
\item Найти центр $C$ и радиус $R$: $x^2+y^2+6x-4y+14=0$
\item Составить уравнение эллипса, фокусы которого лежат на оси абсцисс симметрично относительно начала координат, зная, кроме того, что его полуоси равны 5 и 2
\item Составить уравнение эллипса, фокусы которого лежат на оси абсцисс симметрично относительно начала координат, зная, кроме того, что его большая ось равна $10$, а расстояние между фокусами $2c=8$
\item Составить уравнение эллипса, фокусы которого лежат на оси абсцисс симметрично относительно начала координат, зная, кроме того, что его малая ось равна $24$, а расстояние между фокусами $2c=10$
\item Составить уравнение эллипса, фокусы которого лежат на оси абсцисс симметрично относительно начала координат, зная, кроме того, что расстояние между его фокусами $2c=6$ и эксцентриситет $e=3/5$
\item Составить уравнение эллипса, фокусы которого лежат на оси абсцисс симметрично относительно начала координат, зная, кроме того, что его большая ось равна $20$, а эксцентриситет $e=3/5$
\item Составить уравнение эллипса, фокусы которого лежат на оси абсцисс симметрично относительно начала координат, зная, кроме того, что его малая ось равна $10$, а эксцентриситет $e=12/13$
\item Составить уравнение эллипса, фокусы которого лежат на оси абсцисс симметрично относительно начала координат, зная, кроме того, что расстояние между его директрисами равно $5$ и расстояние между фокусами $2c=4$
\item Составить уравнение эллипса, фокусы которого лежат на оси абсцисс симметрично относительно начала координат, зная, кроме того, что его большая ось равна $8$, а расстояние между директрисами равно $16$
\item Составить уравнение эллипса, фокусы которого лежат на оси абсцисс симметрично относительно начала координат, зная, кроме того, что его малая ось равна $6$, а расстояние между директрисами равно $13$
\item Составить уравнение эллипса, фокусы которого лежат на оси абсцисс симметрично относительно начала координат, зная, кроме того, что расстояние между его директрисами равно $32$ и $e=1/9$
\item Составить уравнение гиперболы, фокусы которой расположены на оси абсцисс симметрично относительно начада координат, зная, кроме того, что ее оси $2a=10$ и $2b=8$
\item Составить уравнение гиперболы, фокусы которой расположены на оси абсцисс симметрично относительно начада координат, зная, кроме того, что расстояние между фокусами $2c=10$ и ось $2b=8$
\item Составить уравнение гиперболы, фокусы которой расположены на оси абсцисс симметрично относительно начада координат, зная, кроме того, что расстояние между фокусамк $2c=6$ и эксцентриситет $e=3/2$
\item Составить уравнение гиперболы, фокусы которой расположены на оси абсцисс симметрично относительно начада координат, зная, кроме того, что ось $2a==16$ и эксцентриситет $e=5/4$
\item Составить уравнение гиперболы, фокусы которой расположены на оси абсцисс симметрично относительно начада координат, зная, кроме того, что уравнения асиматот $y=\pm \frac{4}{3}x$ и расстояние между фокусами $2c=20$
\item Составить уравнение гиперболы, фокусы которой расположены на оси абсцисс симметрично относительно начада координат, зная, кроме того, что расстояние между директрисами равно $228/13$ и расстояние между фокусами $2c=26$
\item Составить уравнение гиперболы, фокусы которой расположены на оси абсцисс симметрично относительно начада координат, зная, кроме того, что расстоявие между директрисами равно $32/5$ и ось $2b=6$
\item Составить уравнение гиперболы, фокусы которой расположены на оси абсцисс симметрично относительно начада координат, зная, кроме того, что расстояние между директрисами равно $8/3$ и эксцентриситет $e=3/2$
\item Составить уравнение гиперболы, фокусы которой расположены на оси абсцисс симметрично относительно начада координат, зная, кроме того, что уравнения асимтот $y=\pm \frac{3}{4}x$ и расстояние между директрисами равно $64/5$
\item Составить уравнение параболы, вершина которой находится в начале координат, зная, что парабола расположена в правой полуплоскости симметрично относительно оси $Ox$ и ее параметр $p=3$
\item Составить уравнение параболы, вершина которой находится в начале координат, зная, что парабола расположена в левой полуплоскости симметрично относительно оси $Ox$ и её параметр $p=0,5$
\item Составить уравнение параболы, вершина которой находится в начале координат, зная, что парабола расположена в верхней полуплоскости симметрично относительно оси $Oy$ и ее параметр $p=1/4$
\item Составить уравнение параболы, вершина которой находится в начале координат, зная, что парабола расположена в нижней полуплоскости симметрично относительно оси $Oy$ и её параметр $p=3$
\item Дано уравнение эллипса $\frac{x^2}{25}+\frac{y^2}{16}=1$. Составить его полярное уравнение, считая, что направление полярной оси совпадает с положительным направлением оси абсиисс, а полюс находится в левом фокусе элдипса
\item Дано уравнение гиперболы $\frac{x^{2}}{16}-\frac{y^{2}}{9}=1$. Составить полярное уравнение ее правой ветви, считая, что направление полярной оси совпадает с положнтельмым направленнем оси абсцисс, а полюс находится в правом фокусе гилерболы
\item Дано уравнение гиперболы $\frac{x^{2}}{25}-\frac{y^{2}}{144}=1$. Составить полярное уравиенне её девой ветви, считая, что направление полярной оси совпадает с положительным направлением оси абсцисс, а полюс находнтся в левом фокусе гиперболы
\item Дано уравнение параболы $y^2=6x$. Составить ее полярное уравнение, считая, что направление полярной осы совпадает с положительным направлением оси абсцисс, а полюс находится в фокусе параболы
\item Определить, какие линии даны следующими уравнениями в полярных координатах: $\rho=\frac{5}{1-\frac{1}{2}\cos\theta}$
\item Определить, какие линии даны следующими уравнениями в полярных координатах: $\rho=\frac{6}{1-\cos 0}$
\item Определить, какие линии даны следующими уравнениями в полярных координатах: $\rho=\frac{10}{1-\frac{3}{2}\cos\theta}$
\item Определить, какие линии даны следующими уравнениями в полярных координатах: $\rho=\frac{12}{2-\cos\theta}$
\item Определить, какие линии даны следующими уравнениями в полярных координатах: $\rho=\frac{5}{3-4\cos\theta}$
\item Определить, какие линии даны следующими уравнениями в полярных координатах: $\rho=\frac{1}{3-3\cos\theta}$
\item Установить, что следующие линии являются центральными, и для каждой из них найти координаты центра: $3x^{2}+5xy+y^{2}-8x-11y-7=0$
\item Установить, что следующие линии являются центральными, и для каждой из них найти координаты центра: $5x^{2}+4xy+2y^{2}+20x+20y-18=0$
\item Установить, что следующие линии являются центральными, и для каждой из них найти координаты центра: $9x^{2}-4xy-7y^{2}-12=0$
\item Установить, что следующие линии являются центральными, и для каждой из них найти координаты центра: $2x^{2}-6xy+5y^{2}+22x-36y+11=0$
\item Определить тип: $4x^2+9y^2-40x+36y+100=0$
\item Определить тип: $9x^{2}-16y^{2}-54x-64y-127=0$
\item Определить тип: $9x^{2}+4y^{2}+18x-8y+49=0$
\item Определить тип: $4x^{2}-y^{2}+8x-2y+3=0$
\item Определить тип: $2x^{2}+3y^{2}+8x-6y+11=0$
\item Определить тип: $2x^{2}+10xy+12y^{2}-7x+18y-15=0$
\item Определить тип: $3x^{2}-8xy+7y^{2}+8x-15y+20=0$
\item Определить тип: $25x^{2}-20xy+4y^{2}-12x+20y-17=0$
\item Определить тип: $5x^{2}+14xy+11y^{2}+12x-7y+19=0$
\item Определить тип: $x^{2}-4xy+4y^{2}+7x-12=0$
\item Определить тип: $3x^{2}-2xy-3y^{2}+12y-15=0$
\end{enumerate}