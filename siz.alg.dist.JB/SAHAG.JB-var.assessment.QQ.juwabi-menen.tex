\documentclass{article}
\usepackage[fontsize=12pt]{fontsize}
\usepackage[utf8]{inputenc}
\usepackage[T2A]{fontenc}
\usepackage{array}
\usepackage[a4paper,
left=15mm,
top=15mm,]{geometry}
\usepackage{amsmath}
\usepackage{setspace}


\renewcommand{\baselinestretch}{1} 

\begin{document}

\pagenumbering{gobble}


\textbf{1-variant}\\

\bgroup
\def\arraystretch{1.6} % 1 is the default, change whatever you need

\begin{tabular}{|m{5.7cm}|m{9.5cm}|}
\hline
Familiyası hám atı & \\
\hline
Fakulteti  & \\
\hline
Toparı hám tálim baǵdarı  & \\
\hline
\end{tabular}

\vspace{1cm}

\begin{tabular}{|m{0.7cm}|m{10cm}|m{4cm}|}
\hline
№ & Soraw & Juwap \\
\hline
1. & Eki vektor qashan kollinear dep ataladı? & bir tuwrıda yamasa parallel tuwrıda jaylasqan bolsa \\
\hline
2. & Tuwrı múyeshli koordinatalar sisteması dep nege aytamız? & Masshtab birlikleri berilgen o'zara perpendikulyar $OX$ hám $OY$ kósherleri \\
\hline
3. & $OXY$ tegisliginiń teńlemesi? & $z=0$ \\
\hline
4. & Giperbolanıń kanonikalıq teńlemesi? & $\frac{x^2}{a^2}-\frac{y^2}{b^2}=1$ \\
\hline
5. & $\overline{a}=\{5,-6, 1 \}, \overline{b}=\{-4, 3, 0 \} $, $\overline{c}=\left\{ 5,-8, 10 \right\}$ vektorları berilgen. $2{\overline{a}}^{2}+4{\overline{b}}^{2}-5{\overline{c}}^{2}$ ańlatpasınıń mánisin tabıń. & $-721$ \\
\hline
6. & $(2, 3)$ hám $(4, 3)$ noqatlarınan ótiwshi tuwrı sızıqtıń teńlemesin dúziń. & $ y-3=0$ \\
\hline
7. & $x^{2}+y^{2}-2x+4y=0$ sheńberdiń teńlemesin kanonikalıq túrdegi teńlemege alıp keliń. & $(x-1)^{2}+(y+2)^{2}=5$ \\
\hline
8. & $A(4, 3), B(7, 7)$ noqatları arasındaǵı aralıqtı tabıń. & $d(AB)=5$ \\
\hline
9. & $3x^{2}+10xy+3y^{2}-2x-14y-13=0$ teńlemesiniń tipin anıqlań. & giperbola \\
\hline
10. & $x^{2}-4y^{2}+6x+5=0$ giperbolanıń kanonikalıq teńlemesin dúziń. & $\frac{(x+3)^{2}}{4}-\frac{y^{2}}{1}=1$ \\
\hline
\end{tabular}

\vspace{1cm}

\begin{tabular}{lll}
Tuwrı juwaplar sanı: \underline{\hspace{1.5cm}} & 
Bahası: \underline{\hspace{1.5cm}} & 
Imtixan alıwshınıń qolı: \underline{\hspace{2cm}} \\
\end{tabular}

\egroup

\newpage


\textbf{2-variant}\\

\bgroup
\def\arraystretch{1.6} % 1 is the default, change whatever you need

\begin{tabular}{|m{5.7cm}|m{9.5cm}|}
\hline
Familiyası hám atı & \\
\hline
Fakulteti  & \\
\hline
Toparı hám tálim baǵdarı  & \\
\hline
\end{tabular}

\vspace{1cm}

\begin{tabular}{|m{0.7cm}|m{10cm}|m{4cm}|}
\hline
№ & Soraw & Juwap \\
\hline
1. & Eki vektordıń vektor kóbeymesiniń uzınlıǵın tabıw formulası? & $\left| \lbrack ab\rbrack \right|=|a||b|\sin\varphi$ \\
\hline
2. & Tegislikdegi qálegen noqatınan berilgen eki noqatqa shekemgi bolǵan aralıqlardıń ayırmasınıń modulı ózgermeytuǵın bolǵan noqatlardıń geometriyalıq ornı ne dep ataladı? & giperbola \\
\hline
3. & Eki tuwrı sızıq arasındaǵı múyeshti tabıw formulası? & $\text{tg}\varphi=\frac{k_2-k_1}{1+k_1k_2}$ \\
\hline
4. & $\frac{x^2}{a^2}-\frac{y^2}{b^2}=1$ giperbolanıń $(x_0;y_0)$ noqatındaǵı urınbasınıń teńlemesin kórsetiń. & $\frac{x_0x}{a^2}-\frac{y_0y}{b^2}=1$ \\
\hline
5. & $M_{1}M_{2}$ kesindiniń ortasınıń koordinatalarınıń tabıń, eger $M_{1} (2, 3), M_{2} (4, 7)$ bolsa. & $(3,5)$ \\
\hline
6. & $x+y-3=0$ hám $2x+3y-8=0$ tuwrıları óz-ara qanday jaylasqan? & kesilisedi \\
\hline
7. & $x^{2}+y^{2}-2x+4y-20=0$ sheńberdiń $C$ orayın hám $R$ radiusın tabıń. & $C(1;-2), R=5$ \\
\hline
8. & $(x+1)^{2}+(y-2) ^{2}+(z+2) ^{2}=49$ sferanıń orayınıń koordinataların tabıń. & $(-1,2,-2)$ \\
\hline
9. & Eger $2a=16, e=\frac{5}{4}$ bolsa, fokusı abscissa kósherinde, koordinata basına salıstırǵanda simmetriyalıq jaylasqan giperbolanıń teńlemesin dúziń. & $\frac{x^{2}}{64}-\frac{y^{2}}{36}=1$ \\
\hline
10. & Eger $2b=24, 2 c=10$ bolsa, onda abscissa kósherinde koordinata basına salıstırǵanda simmetriyalıq jaylasqan fokuslarǵa iye, ellipstiń teńlemesin dúziń. & $\frac{x^{2}}{169}+\frac{y^{2}}{144}=1$ \\
\hline
\end{tabular}

\vspace{1cm}

\begin{tabular}{lll}
Tuwrı juwaplar sanı: \underline{\hspace{1.5cm}} & 
Bahası: \underline{\hspace{1.5cm}} & 
Imtixan alıwshınıń qolı: \underline{\hspace{2cm}} \\
\end{tabular}

\egroup

\newpage


\textbf{3-variant}\\

\bgroup
\def\arraystretch{1.6} % 1 is the default, change whatever you need

\begin{tabular}{|m{5.7cm}|m{9.5cm}|}
\hline
Familiyası hám atı & \\
\hline
Fakulteti  & \\
\hline
Toparı hám tálim baǵdarı  & \\
\hline
\end{tabular}

\vspace{1cm}

\begin{tabular}{|m{0.7cm}|m{10cm}|m{4cm}|}
\hline
№ & Soraw & Juwap \\
\hline
1. & Vektorlardıń kósherdegi proekciyasınıń formulası? & $x=|a|\cos\varphi, y=|a|\sin\varphi$ \\
\hline
2. & $Ax+By+D=0$ teńlemesi arqalı ... tegisliktiń teńlemesi berilgen? & $OZ$ kósherine parallel \\
\hline
3. & $\frac{x^2}{a^2}+\frac{y^2}{b^2}=1$ ellipstiń $(x_0;y_0)$ noqatındaǵı urınbasınıń teńlemesin tabıń. & $\frac{x_0x}{a^2}+\frac{y_0y}{b^2}=1$ \\
\hline
4. & Vektorlardı qosıw koordinatalarda qanday formula menen anıqlanadı? & $\overline{a}+\overline{b}=\{x_1+x_2;y_1+y_2\}$ \\
\hline
5. & $M_{1} (12;-1)$ hám $M_{2} (0;4)$ noqatlardıń arasındaǵı aralıqtı tabıń. & $13$ \\
\hline
6. & $x+y=0$ teńlemesi menen berilgen tuwrı sızıqtıń múyeshlik koefficientin anıqlań. & $- 1$ \\
\hline
7. & Orayı $C (-1;2)$ noqatında, $A (-2;6 )$ noqatınan ótetuǵın sheńberdiń teńlemesin dúziń. & $(x+1)^{2}+(y-2)^{2}=17$ \\
\hline
8. & $x+2=0$ keńislik qanday geometriyalıq betlikti anıqlaydı? &  $OYZ$ tegisligine parallel bolǵan tegislikti \\
\hline
9. & $\frac{x^{2}}{225}-\frac{y^{2}}{64}=-1$ giperbola fokusınıń koordinatalarınıń tabıń. & $F_{1}(0;-17), F_{2}(0;17)$ \\
\hline
10. & $9x^{2}+25y^{2}=225$ ellipsi berilgen, ellipstiń fokusların, ekscentrisitetin tabıń. & $F_1\left(-4;0 \right) , F_2\left( 4;0 \right) , e = \frac{4}{5}$ \\
\hline
\end{tabular}

\vspace{1cm}

\begin{tabular}{lll}
Tuwrı juwaplar sanı: \underline{\hspace{1.5cm}} & 
Bahası: \underline{\hspace{1.5cm}} & 
Imtixan alıwshınıń qolı: \underline{\hspace{2cm}} \\
\end{tabular}

\egroup

\newpage


\textbf{4-variant}\\

\bgroup
\def\arraystretch{1.6} % 1 is the default, change whatever you need

\begin{tabular}{|m{5.7cm}|m{9.5cm}|}
\hline
Familiyası hám atı & \\
\hline
Fakulteti  & \\
\hline
Toparı hám tálim baǵdarı  & \\
\hline
\end{tabular}

\vspace{1cm}

\begin{tabular}{|m{0.7cm}|m{10cm}|m{4cm}|}
\hline
№ & Soraw & Juwap \\
\hline
1. & $OY$ kósheriniń teńlemesi? & $x=0$ \\
\hline
2. & Egerde $a=\{ x_1; y_1; z_1\}, b=\{ x_2, y_2; z_2\}$ bolsa, vektor kóbeymeniń koordinatalarda ańlatılıwı qanday boladı? &  $\lbrack ab\rbrack=\{y_1z_2-y_2z_1; z_1x_2-z_2x_1; x_1y_2-x_2y_1\}$ \\
\hline
3. & $A_1x+B_1y+C_1z+D_1=0$ hám $Ax_2+By_2+Cz_2+D_2=0$ tegislikleri perpendikulyar bolıwı shárti & $A_1\cdot A_2+B_1\cdot B_2+C_1\cdot C_2=0$ \\
\hline
4. & Úsh vektordıń aralas kóbeymesi ushın $(abc)=0$ teńligi orınlı bolsa ne dep ataladı? & $\overline{a}$, $\overline{b}$ hám $\overline{c}$ vektorları komplanar \\
\hline
5. & $A (-1;0;1),\ B (1;-1;0)$ noqatları berilgen. $\overline{BA}$ vektorın tabıń. & $\left\{ - 2;1;1 \right\}$ \\
\hline
6. & $2x+3y+4=0$ tuwrısına parallel hám $M_{0} (2;1)$ noqattan ótetuǵın tuwrınıń teńlemesin dúziń. & $2x+3y-7=0$ \\
\hline
7. & $x+y-12=0$ tuwrısı $x^{2}+y^{2}-2y=0$ sheńberge salıstırǵanda qanday jaylasqan? & sırtında jaylasqan \\
\hline
8. & $\left| \overline{a} \right|=8, \left| \overline{b} \right|=5, \alpha=60^{0}$ bolsa, $( \overline{a}\overline{b} )$ ni tabıń. & $20$ \\
\hline
9. & $2x+3y-6=0$ tuwrınıń teńlemesin kesindilerde berilgen teńleme túrinde kórsetiń. & $\frac{x}{3} + \frac{ y }{ 2 } =  1$ \\
\hline
10. & $\overline{a}=\left\{ 4,-2,-4 \right\}$ hám $\overline{b}=\left\{ 6,-3, 2 \right\}$ vektorları berilgen, $(\overline{a}-\overline{b}) ^{2}$-? & $41$ \\
\hline
\end{tabular}

\vspace{1cm}

\begin{tabular}{lll}
Tuwrı juwaplar sanı: \underline{\hspace{1.5cm}} & 
Bahası: \underline{\hspace{1.5cm}} & 
Imtixan alıwshınıń qolı: \underline{\hspace{2cm}} \\
\end{tabular}

\egroup

\newpage


\textbf{5-variant}\\

\bgroup
\def\arraystretch{1.6} % 1 is the default, change whatever you need

\begin{tabular}{|m{5.7cm}|m{9.5cm}|}
\hline
Familiyası hám atı & \\
\hline
Fakulteti  & \\
\hline
Toparı hám tálim baǵdarı  & \\
\hline
\end{tabular}

\vspace{1cm}

\begin{tabular}{|m{0.7cm}|m{10cm}|m{4cm}|}
\hline
№ & Soraw & Juwap \\
\hline
1. & $A_1x+B_1y+C_1z+D_1=0$ hám $Ax_2+By_2+Cz_2+D_2=0$ tegislikleri ústpe-úst túsiwi shárti? & $\frac{A_1}{A_2}=\frac{B_1}{B_2}=\frac{C_1}{C_2}=\frac{D_1}{D_2}$ \\
\hline
2. & Eki vektordıń skalyar kóbeymesiniń formulası? & $(ab)=|a||b|\cos\varphi$ \\
\hline
3. & $A_1x+B_1y+C_1z+D_1=0$ hám $Ax_2+By_2+Cz_2+D_2=0$ tegislikleri parallel bolıwı shárti & $\frac{A_1}{A_2}=\frac{B_1}{B_2}=\frac{C_1}{C_2}$ \\
\hline
4. & $Ax+C=0$ tuwrı sızıqtıń grafigi koordinata kósherlerine salıstırǵanda qanday jaylasqan? & $OY$ kósherine parallel \\
\hline
5. & $5x-y+7=0$ hám $3x+2y=0$ tuwrıları arasındaǵı múyeshni tabıń. & $\varphi=\frac{\pi}{4}$ \\
\hline
6. & $\overline{a}=\left\{ 2, 1, 0 \right\}$ hám $\overline{b}=\left\{ 1, 0,-1 \right\}$ bolsa, $\overline{a}-\overline{b}$ ni tabıń. & $\overline{a} -\overline{b} = \left\{ 1,1,1 \right\}$ \\
\hline
7. & Koordinatalar kósherleri hám $ 3x+4y-12=0 $ tuwrı sızıǵı menen shegaralanǵan úshmúyeshliktiń maydanın tabıń. & $ S=6 $ \\
\hline
8. & $x-2y+1=0$ teńlemesi menen berilgen tuwrınıń normal túrdegi teńlemesin kórsetiń. & $\frac{x}{- \sqrt{5}}+\frac{2y}{\sqrt{5}}-\frac{1}{\sqrt{5}}=0$ \\
\hline
9. & $3x-y+5=0$, $x+3y-4=0$ tuwrı sızıqları arasındaǵı múyeshti tabıń. & $90^{0}$ \\
\hline
10. & $\overline{a}=\{5,-6, 1 \}, \overline{b}=\{-4, 3, 0 \} $, $\overline{c}=\left\{ 5,-8, 10 \right\}$ vektorları berilgen. $2{\overline{a}}^{2}+4{\overline{b}}^{2}-5{\overline{c}}^{2}$ ańlatpasınıń mánisin tabıń. & $-721$ \\
\hline
\end{tabular}

\vspace{1cm}

\begin{tabular}{lll}
Tuwrı juwaplar sanı: \underline{\hspace{1.5cm}} & 
Bahası: \underline{\hspace{1.5cm}} & 
Imtixan alıwshınıń qolı: \underline{\hspace{2cm}} \\
\end{tabular}

\egroup

\newpage


\textbf{6-variant}\\

\bgroup
\def\arraystretch{1.6} % 1 is the default, change whatever you need

\begin{tabular}{|m{5.7cm}|m{9.5cm}|}
\hline
Familiyası hám atı & \\
\hline
Fakulteti  & \\
\hline
Toparı hám tálim baǵdarı  & \\
\hline
\end{tabular}

\vspace{1cm}

\begin{tabular}{|m{0.7cm}|m{10cm}|m{4cm}|}
\hline
№ & Soraw & Juwap \\
\hline
1. & Eki vektor qashan kollinear dep ataladı? & bir tuwrıda yamasa parallel tuwrıda jaylasqan bolsa \\
\hline
2. & Tuwrı múyeshli koordinatalar sisteması dep nege aytamız? & Masshtab birlikleri berilgen o'zara perpendikulyar $OX$ hám $OY$ kósherleri \\
\hline
3. & $OXY$ tegisliginiń teńlemesi? & $z=0$ \\
\hline
4. & Giperbolanıń kanonikalıq teńlemesi? & $\frac{x^2}{a^2}-\frac{y^2}{b^2}=1$ \\
\hline
5. & $(2, 3)$ hám $(4, 3)$ noqatlarınan ótiwshi tuwrı sızıqtıń teńlemesin dúziń. & $ y-3=0$ \\
\hline
6. & $x^{2}+y^{2}-2x+4y=0$ sheńberdiń teńlemesin kanonikalıq túrdegi teńlemege alıp keliń. & $(x-1)^{2}+(y+2)^{2}=5$ \\
\hline
7. & $A(4, 3), B(7, 7)$ noqatları arasındaǵı aralıqtı tabıń. & $d(AB)=5$ \\
\hline
8. & $3x^{2}+10xy+3y^{2}-2x-14y-13=0$ teńlemesiniń tipin anıqlań. & giperbola \\
\hline
9. & $x^{2}-4y^{2}+6x+5=0$ giperbolanıń kanonikalıq teńlemesin dúziń. & $\frac{(x+3)^{2}}{4}-\frac{y^{2}}{1}=1$ \\
\hline
10. & $M_{1}M_{2}$ kesindiniń ortasınıń koordinatalarınıń tabıń, eger $M_{1} (2, 3), M_{2} (4, 7)$ bolsa. & $(3,5)$ \\
\hline
\end{tabular}

\vspace{1cm}

\begin{tabular}{lll}
Tuwrı juwaplar sanı: \underline{\hspace{1.5cm}} & 
Bahası: \underline{\hspace{1.5cm}} & 
Imtixan alıwshınıń qolı: \underline{\hspace{2cm}} \\
\end{tabular}

\egroup

\newpage


\textbf{7-variant}\\

\bgroup
\def\arraystretch{1.6} % 1 is the default, change whatever you need

\begin{tabular}{|m{5.7cm}|m{9.5cm}|}
\hline
Familiyası hám atı & \\
\hline
Fakulteti  & \\
\hline
Toparı hám tálim baǵdarı  & \\
\hline
\end{tabular}

\vspace{1cm}

\begin{tabular}{|m{0.7cm}|m{10cm}|m{4cm}|}
\hline
№ & Soraw & Juwap \\
\hline
1. & Eki vektordıń vektor kóbeymesiniń uzınlıǵın tabıw formulası? & $\left| \lbrack ab\rbrack \right|=|a||b|\sin\varphi$ \\
\hline
2. & Tegislikdegi qálegen noqatınan berilgen eki noqatqa shekemgi bolǵan aralıqlardıń ayırmasınıń modulı ózgermeytuǵın bolǵan noqatlardıń geometriyalıq ornı ne dep ataladı? & giperbola \\
\hline
3. & Eki tuwrı sızıq arasındaǵı múyeshti tabıw formulası? & $\text{tg}\varphi=\frac{k_2-k_1}{1+k_1k_2}$ \\
\hline
4. & $\frac{x^2}{a^2}-\frac{y^2}{b^2}=1$ giperbolanıń $(x_0;y_0)$ noqatındaǵı urınbasınıń teńlemesin kórsetiń. & $\frac{x_0x}{a^2}-\frac{y_0y}{b^2}=1$ \\
\hline
5. & $x+y-3=0$ hám $2x+3y-8=0$ tuwrıları óz-ara qanday jaylasqan? & kesilisedi \\
\hline
6. & $x^{2}+y^{2}-2x+4y-20=0$ sheńberdiń $C$ orayın hám $R$ radiusın tabıń. & $C(1;-2), R=5$ \\
\hline
7. & $(x+1)^{2}+(y-2) ^{2}+(z+2) ^{2}=49$ sferanıń orayınıń koordinataların tabıń. & $(-1,2,-2)$ \\
\hline
8. & Eger $2a=16, e=\frac{5}{4}$ bolsa, fokusı abscissa kósherinde, koordinata basına salıstırǵanda simmetriyalıq jaylasqan giperbolanıń teńlemesin dúziń. & $\frac{x^{2}}{64}-\frac{y^{2}}{36}=1$ \\
\hline
9. & Eger $2b=24, 2 c=10$ bolsa, onda abscissa kósherinde koordinata basına salıstırǵanda simmetriyalıq jaylasqan fokuslarǵa iye, ellipstiń teńlemesin dúziń. & $\frac{x^{2}}{169}+\frac{y^{2}}{144}=1$ \\
\hline
10. & $M_{1} (12;-1)$ hám $M_{2} (0;4)$ noqatlardıń arasındaǵı aralıqtı tabıń. & $13$ \\
\hline
\end{tabular}

\vspace{1cm}

\begin{tabular}{lll}
Tuwrı juwaplar sanı: \underline{\hspace{1.5cm}} & 
Bahası: \underline{\hspace{1.5cm}} & 
Imtixan alıwshınıń qolı: \underline{\hspace{2cm}} \\
\end{tabular}

\egroup

\newpage


\textbf{8-variant}\\

\bgroup
\def\arraystretch{1.6} % 1 is the default, change whatever you need

\begin{tabular}{|m{5.7cm}|m{9.5cm}|}
\hline
Familiyası hám atı & \\
\hline
Fakulteti  & \\
\hline
Toparı hám tálim baǵdarı  & \\
\hline
\end{tabular}

\vspace{1cm}

\begin{tabular}{|m{0.7cm}|m{10cm}|m{4cm}|}
\hline
№ & Soraw & Juwap \\
\hline
1. & Vektorlardıń kósherdegi proekciyasınıń formulası? & $x=|a|\cos\varphi, y=|a|\sin\varphi$ \\
\hline
2. & $Ax+By+D=0$ teńlemesi arqalı ... tegisliktiń teńlemesi berilgen? & $OZ$ kósherine parallel \\
\hline
3. & $\frac{x^2}{a^2}+\frac{y^2}{b^2}=1$ ellipstiń $(x_0;y_0)$ noqatındaǵı urınbasınıń teńlemesin tabıń. & $\frac{x_0x}{a^2}+\frac{y_0y}{b^2}=1$ \\
\hline
4. & Vektorlardı qosıw koordinatalarda qanday formula menen anıqlanadı? & $\overline{a}+\overline{b}=\{x_1+x_2;y_1+y_2\}$ \\
\hline
5. & $x+y=0$ teńlemesi menen berilgen tuwrı sızıqtıń múyeshlik koefficientin anıqlań. & $- 1$ \\
\hline
6. & Orayı $C (-1;2)$ noqatında, $A (-2;6 )$ noqatınan ótetuǵın sheńberdiń teńlemesin dúziń. & $(x+1)^{2}+(y-2)^{2}=17$ \\
\hline
7. & $x+2=0$ keńislik qanday geometriyalıq betlikti anıqlaydı? &  $OYZ$ tegisligine parallel bolǵan tegislikti \\
\hline
8. & $\frac{x^{2}}{225}-\frac{y^{2}}{64}=-1$ giperbola fokusınıń koordinatalarınıń tabıń. & $F_{1}(0;-17), F_{2}(0;17)$ \\
\hline
9. & $9x^{2}+25y^{2}=225$ ellipsi berilgen, ellipstiń fokusların, ekscentrisitetin tabıń. & $F_1\left(-4;0 \right) , F_2\left( 4;0 \right) , e = \frac{4}{5}$ \\
\hline
10. & $A (-1;0;1),\ B (1;-1;0)$ noqatları berilgen. $\overline{BA}$ vektorın tabıń. & $\left\{ - 2;1;1 \right\}$ \\
\hline
\end{tabular}

\vspace{1cm}

\begin{tabular}{lll}
Tuwrı juwaplar sanı: \underline{\hspace{1.5cm}} & 
Bahası: \underline{\hspace{1.5cm}} & 
Imtixan alıwshınıń qolı: \underline{\hspace{2cm}} \\
\end{tabular}

\egroup

\newpage


\textbf{9-variant}\\

\bgroup
\def\arraystretch{1.6} % 1 is the default, change whatever you need

\begin{tabular}{|m{5.7cm}|m{9.5cm}|}
\hline
Familiyası hám atı & \\
\hline
Fakulteti  & \\
\hline
Toparı hám tálim baǵdarı  & \\
\hline
\end{tabular}

\vspace{1cm}

\begin{tabular}{|m{0.7cm}|m{10cm}|m{4cm}|}
\hline
№ & Soraw & Juwap \\
\hline
1. & $OY$ kósheriniń teńlemesi? & $x=0$ \\
\hline
2. & Egerde $a=\{ x_1; y_1; z_1\}, b=\{ x_2, y_2; z_2\}$ bolsa, vektor kóbeymeniń koordinatalarda ańlatılıwı qanday boladı? &  $\lbrack ab\rbrack=\{y_1z_2-y_2z_1; z_1x_2-z_2x_1; x_1y_2-x_2y_1\}$ \\
\hline
3. & $A_1x+B_1y+C_1z+D_1=0$ hám $Ax_2+By_2+Cz_2+D_2=0$ tegislikleri perpendikulyar bolıwı shárti & $A_1\cdot A_2+B_1\cdot B_2+C_1\cdot C_2=0$ \\
\hline
4. & Úsh vektordıń aralas kóbeymesi ushın $(abc)=0$ teńligi orınlı bolsa ne dep ataladı? & $\overline{a}$, $\overline{b}$ hám $\overline{c}$ vektorları komplanar \\
\hline
5. & $2x+3y+4=0$ tuwrısına parallel hám $M_{0} (2;1)$ noqattan ótetuǵın tuwrınıń teńlemesin dúziń. & $2x+3y-7=0$ \\
\hline
6. & $x+y-12=0$ tuwrısı $x^{2}+y^{2}-2y=0$ sheńberge salıstırǵanda qanday jaylasqan? & sırtında jaylasqan \\
\hline
7. & $\left| \overline{a} \right|=8, \left| \overline{b} \right|=5, \alpha=60^{0}$ bolsa, $( \overline{a}\overline{b} )$ ni tabıń. & $20$ \\
\hline
8. & $2x+3y-6=0$ tuwrınıń teńlemesin kesindilerde berilgen teńleme túrinde kórsetiń. & $\frac{x}{3} + \frac{ y }{ 2 } =  1$ \\
\hline
9. & $\overline{a}=\left\{ 4,-2,-4 \right\}$ hám $\overline{b}=\left\{ 6,-3, 2 \right\}$ vektorları berilgen, $(\overline{a}-\overline{b}) ^{2}$-? & $41$ \\
\hline
10. & $5x-y+7=0$ hám $3x+2y=0$ tuwrıları arasındaǵı múyeshni tabıń. & $\varphi=\frac{\pi}{4}$ \\
\hline
\end{tabular}

\vspace{1cm}

\begin{tabular}{lll}
Tuwrı juwaplar sanı: \underline{\hspace{1.5cm}} & 
Bahası: \underline{\hspace{1.5cm}} & 
Imtixan alıwshınıń qolı: \underline{\hspace{2cm}} \\
\end{tabular}

\egroup

\newpage


\textbf{10-variant}\\

\bgroup
\def\arraystretch{1.6} % 1 is the default, change whatever you need

\begin{tabular}{|m{5.7cm}|m{9.5cm}|}
\hline
Familiyası hám atı & \\
\hline
Fakulteti  & \\
\hline
Toparı hám tálim baǵdarı  & \\
\hline
\end{tabular}

\vspace{1cm}

\begin{tabular}{|m{0.7cm}|m{10cm}|m{4cm}|}
\hline
№ & Soraw & Juwap \\
\hline
1. & $A_1x+B_1y+C_1z+D_1=0$ hám $Ax_2+By_2+Cz_2+D_2=0$ tegislikleri ústpe-úst túsiwi shárti? & $\frac{A_1}{A_2}=\frac{B_1}{B_2}=\frac{C_1}{C_2}=\frac{D_1}{D_2}$ \\
\hline
2. & Eki vektordıń skalyar kóbeymesiniń formulası? & $(ab)=|a||b|\cos\varphi$ \\
\hline
3. & $A_1x+B_1y+C_1z+D_1=0$ hám $Ax_2+By_2+Cz_2+D_2=0$ tegislikleri parallel bolıwı shárti & $\frac{A_1}{A_2}=\frac{B_1}{B_2}=\frac{C_1}{C_2}$ \\
\hline
4. & $Ax+C=0$ tuwrı sızıqtıń grafigi koordinata kósherlerine salıstırǵanda qanday jaylasqan? & $OY$ kósherine parallel \\
\hline
5. & $\overline{a}=\left\{ 2, 1, 0 \right\}$ hám $\overline{b}=\left\{ 1, 0,-1 \right\}$ bolsa, $\overline{a}-\overline{b}$ ni tabıń. & $\overline{a} -\overline{b} = \left\{ 1,1,1 \right\}$ \\
\hline
6. & Koordinatalar kósherleri hám $ 3x+4y-12=0 $ tuwrı sızıǵı menen shegaralanǵan úshmúyeshliktiń maydanın tabıń. & $ S=6 $ \\
\hline
7. & $x-2y+1=0$ teńlemesi menen berilgen tuwrınıń normal túrdegi teńlemesin kórsetiń. & $\frac{x}{- \sqrt{5}}+\frac{2y}{\sqrt{5}}-\frac{1}{\sqrt{5}}=0$ \\
\hline
8. & $3x-y+5=0$, $x+3y-4=0$ tuwrı sızıqları arasındaǵı múyeshti tabıń. & $90^{0}$ \\
\hline
9. & $\overline{a}=\{5,-6, 1 \}, \overline{b}=\{-4, 3, 0 \} $, $\overline{c}=\left\{ 5,-8, 10 \right\}$ vektorları berilgen. $2{\overline{a}}^{2}+4{\overline{b}}^{2}-5{\overline{c}}^{2}$ ańlatpasınıń mánisin tabıń. & $-721$ \\
\hline
10. & $(2, 3)$ hám $(4, 3)$ noqatlarınan ótiwshi tuwrı sızıqtıń teńlemesin dúziń. & $ y-3=0$ \\
\hline
\end{tabular}

\vspace{1cm}

\begin{tabular}{lll}
Tuwrı juwaplar sanı: \underline{\hspace{1.5cm}} & 
Bahası: \underline{\hspace{1.5cm}} & 
Imtixan alıwshınıń qolı: \underline{\hspace{2cm}} \\
\end{tabular}

\egroup

\newpage


\textbf{11-variant}\\

\bgroup
\def\arraystretch{1.6} % 1 is the default, change whatever you need

\begin{tabular}{|m{5.7cm}|m{9.5cm}|}
\hline
Familiyası hám atı & \\
\hline
Fakulteti  & \\
\hline
Toparı hám tálim baǵdarı  & \\
\hline
\end{tabular}

\vspace{1cm}

\begin{tabular}{|m{0.7cm}|m{10cm}|m{4cm}|}
\hline
№ & Soraw & Juwap \\
\hline
1. & Eki vektor qashan kollinear dep ataladı? & bir tuwrıda yamasa parallel tuwrıda jaylasqan bolsa \\
\hline
2. & Tuwrı múyeshli koordinatalar sisteması dep nege aytamız? & Masshtab birlikleri berilgen o'zara perpendikulyar $OX$ hám $OY$ kósherleri \\
\hline
3. & $OXY$ tegisliginiń teńlemesi? & $z=0$ \\
\hline
4. & Giperbolanıń kanonikalıq teńlemesi? & $\frac{x^2}{a^2}-\frac{y^2}{b^2}=1$ \\
\hline
5. & $x^{2}+y^{2}-2x+4y=0$ sheńberdiń teńlemesin kanonikalıq túrdegi teńlemege alıp keliń. & $(x-1)^{2}+(y+2)^{2}=5$ \\
\hline
6. & $A(4, 3), B(7, 7)$ noqatları arasındaǵı aralıqtı tabıń. & $d(AB)=5$ \\
\hline
7. & $3x^{2}+10xy+3y^{2}-2x-14y-13=0$ teńlemesiniń tipin anıqlań. & giperbola \\
\hline
8. & $x^{2}-4y^{2}+6x+5=0$ giperbolanıń kanonikalıq teńlemesin dúziń. & $\frac{(x+3)^{2}}{4}-\frac{y^{2}}{1}=1$ \\
\hline
9. & $M_{1}M_{2}$ kesindiniń ortasınıń koordinatalarınıń tabıń, eger $M_{1} (2, 3), M_{2} (4, 7)$ bolsa. & $(3,5)$ \\
\hline
10. & $x+y-3=0$ hám $2x+3y-8=0$ tuwrıları óz-ara qanday jaylasqan? & kesilisedi \\
\hline
\end{tabular}

\vspace{1cm}

\begin{tabular}{lll}
Tuwrı juwaplar sanı: \underline{\hspace{1.5cm}} & 
Bahası: \underline{\hspace{1.5cm}} & 
Imtixan alıwshınıń qolı: \underline{\hspace{2cm}} \\
\end{tabular}

\egroup

\newpage


\textbf{12-variant}\\

\bgroup
\def\arraystretch{1.6} % 1 is the default, change whatever you need

\begin{tabular}{|m{5.7cm}|m{9.5cm}|}
\hline
Familiyası hám atı & \\
\hline
Fakulteti  & \\
\hline
Toparı hám tálim baǵdarı  & \\
\hline
\end{tabular}

\vspace{1cm}

\begin{tabular}{|m{0.7cm}|m{10cm}|m{4cm}|}
\hline
№ & Soraw & Juwap \\
\hline
1. & Eki vektordıń vektor kóbeymesiniń uzınlıǵın tabıw formulası? & $\left| \lbrack ab\rbrack \right|=|a||b|\sin\varphi$ \\
\hline
2. & Tegislikdegi qálegen noqatınan berilgen eki noqatqa shekemgi bolǵan aralıqlardıń ayırmasınıń modulı ózgermeytuǵın bolǵan noqatlardıń geometriyalıq ornı ne dep ataladı? & giperbola \\
\hline
3. & Eki tuwrı sızıq arasındaǵı múyeshti tabıw formulası? & $\text{tg}\varphi=\frac{k_2-k_1}{1+k_1k_2}$ \\
\hline
4. & $\frac{x^2}{a^2}-\frac{y^2}{b^2}=1$ giperbolanıń $(x_0;y_0)$ noqatındaǵı urınbasınıń teńlemesin kórsetiń. & $\frac{x_0x}{a^2}-\frac{y_0y}{b^2}=1$ \\
\hline
5. & $x^{2}+y^{2}-2x+4y-20=0$ sheńberdiń $C$ orayın hám $R$ radiusın tabıń. & $C(1;-2), R=5$ \\
\hline
6. & $(x+1)^{2}+(y-2) ^{2}+(z+2) ^{2}=49$ sferanıń orayınıń koordinataların tabıń. & $(-1,2,-2)$ \\
\hline
7. & Eger $2a=16, e=\frac{5}{4}$ bolsa, fokusı abscissa kósherinde, koordinata basına salıstırǵanda simmetriyalıq jaylasqan giperbolanıń teńlemesin dúziń. & $\frac{x^{2}}{64}-\frac{y^{2}}{36}=1$ \\
\hline
8. & Eger $2b=24, 2 c=10$ bolsa, onda abscissa kósherinde koordinata basına salıstırǵanda simmetriyalıq jaylasqan fokuslarǵa iye, ellipstiń teńlemesin dúziń. & $\frac{x^{2}}{169}+\frac{y^{2}}{144}=1$ \\
\hline
9. & $M_{1} (12;-1)$ hám $M_{2} (0;4)$ noqatlardıń arasındaǵı aralıqtı tabıń. & $13$ \\
\hline
10. & $x+y=0$ teńlemesi menen berilgen tuwrı sızıqtıń múyeshlik koefficientin anıqlań. & $- 1$ \\
\hline
\end{tabular}

\vspace{1cm}

\begin{tabular}{lll}
Tuwrı juwaplar sanı: \underline{\hspace{1.5cm}} & 
Bahası: \underline{\hspace{1.5cm}} & 
Imtixan alıwshınıń qolı: \underline{\hspace{2cm}} \\
\end{tabular}

\egroup

\newpage


\textbf{13-variant}\\

\bgroup
\def\arraystretch{1.6} % 1 is the default, change whatever you need

\begin{tabular}{|m{5.7cm}|m{9.5cm}|}
\hline
Familiyası hám atı & \\
\hline
Fakulteti  & \\
\hline
Toparı hám tálim baǵdarı  & \\
\hline
\end{tabular}

\vspace{1cm}

\begin{tabular}{|m{0.7cm}|m{10cm}|m{4cm}|}
\hline
№ & Soraw & Juwap \\
\hline
1. & Vektorlardıń kósherdegi proekciyasınıń formulası? & $x=|a|\cos\varphi, y=|a|\sin\varphi$ \\
\hline
2. & $Ax+By+D=0$ teńlemesi arqalı ... tegisliktiń teńlemesi berilgen? & $OZ$ kósherine parallel \\
\hline
3. & $\frac{x^2}{a^2}+\frac{y^2}{b^2}=1$ ellipstiń $(x_0;y_0)$ noqatındaǵı urınbasınıń teńlemesin tabıń. & $\frac{x_0x}{a^2}+\frac{y_0y}{b^2}=1$ \\
\hline
4. & Vektorlardı qosıw koordinatalarda qanday formula menen anıqlanadı? & $\overline{a}+\overline{b}=\{x_1+x_2;y_1+y_2\}$ \\
\hline
5. & Orayı $C (-1;2)$ noqatında, $A (-2;6 )$ noqatınan ótetuǵın sheńberdiń teńlemesin dúziń. & $(x+1)^{2}+(y-2)^{2}=17$ \\
\hline
6. & $x+2=0$ keńislik qanday geometriyalıq betlikti anıqlaydı? &  $OYZ$ tegisligine parallel bolǵan tegislikti \\
\hline
7. & $\frac{x^{2}}{225}-\frac{y^{2}}{64}=-1$ giperbola fokusınıń koordinatalarınıń tabıń. & $F_{1}(0;-17), F_{2}(0;17)$ \\
\hline
8. & $9x^{2}+25y^{2}=225$ ellipsi berilgen, ellipstiń fokusların, ekscentrisitetin tabıń. & $F_1\left(-4;0 \right) , F_2\left( 4;0 \right) , e = \frac{4}{5}$ \\
\hline
9. & $A (-1;0;1),\ B (1;-1;0)$ noqatları berilgen. $\overline{BA}$ vektorın tabıń. & $\left\{ - 2;1;1 \right\}$ \\
\hline
10. & $2x+3y+4=0$ tuwrısına parallel hám $M_{0} (2;1)$ noqattan ótetuǵın tuwrınıń teńlemesin dúziń. & $2x+3y-7=0$ \\
\hline
\end{tabular}

\vspace{1cm}

\begin{tabular}{lll}
Tuwrı juwaplar sanı: \underline{\hspace{1.5cm}} & 
Bahası: \underline{\hspace{1.5cm}} & 
Imtixan alıwshınıń qolı: \underline{\hspace{2cm}} \\
\end{tabular}

\egroup

\newpage


\textbf{14-variant}\\

\bgroup
\def\arraystretch{1.6} % 1 is the default, change whatever you need

\begin{tabular}{|m{5.7cm}|m{9.5cm}|}
\hline
Familiyası hám atı & \\
\hline
Fakulteti  & \\
\hline
Toparı hám tálim baǵdarı  & \\
\hline
\end{tabular}

\vspace{1cm}

\begin{tabular}{|m{0.7cm}|m{10cm}|m{4cm}|}
\hline
№ & Soraw & Juwap \\
\hline
1. & $OY$ kósheriniń teńlemesi? & $x=0$ \\
\hline
2. & Egerde $a=\{ x_1; y_1; z_1\}, b=\{ x_2, y_2; z_2\}$ bolsa, vektor kóbeymeniń koordinatalarda ańlatılıwı qanday boladı? &  $\lbrack ab\rbrack=\{y_1z_2-y_2z_1; z_1x_2-z_2x_1; x_1y_2-x_2y_1\}$ \\
\hline
3. & $A_1x+B_1y+C_1z+D_1=0$ hám $Ax_2+By_2+Cz_2+D_2=0$ tegislikleri perpendikulyar bolıwı shárti & $A_1\cdot A_2+B_1\cdot B_2+C_1\cdot C_2=0$ \\
\hline
4. & Úsh vektordıń aralas kóbeymesi ushın $(abc)=0$ teńligi orınlı bolsa ne dep ataladı? & $\overline{a}$, $\overline{b}$ hám $\overline{c}$ vektorları komplanar \\
\hline
5. & $x+y-12=0$ tuwrısı $x^{2}+y^{2}-2y=0$ sheńberge salıstırǵanda qanday jaylasqan? & sırtında jaylasqan \\
\hline
6. & $\left| \overline{a} \right|=8, \left| \overline{b} \right|=5, \alpha=60^{0}$ bolsa, $( \overline{a}\overline{b} )$ ni tabıń. & $20$ \\
\hline
7. & $2x+3y-6=0$ tuwrınıń teńlemesin kesindilerde berilgen teńleme túrinde kórsetiń. & $\frac{x}{3} + \frac{ y }{ 2 } =  1$ \\
\hline
8. & $\overline{a}=\left\{ 4,-2,-4 \right\}$ hám $\overline{b}=\left\{ 6,-3, 2 \right\}$ vektorları berilgen, $(\overline{a}-\overline{b}) ^{2}$-? & $41$ \\
\hline
9. & $5x-y+7=0$ hám $3x+2y=0$ tuwrıları arasındaǵı múyeshni tabıń. & $\varphi=\frac{\pi}{4}$ \\
\hline
10. & $\overline{a}=\left\{ 2, 1, 0 \right\}$ hám $\overline{b}=\left\{ 1, 0,-1 \right\}$ bolsa, $\overline{a}-\overline{b}$ ni tabıń. & $\overline{a} -\overline{b} = \left\{ 1,1,1 \right\}$ \\
\hline
\end{tabular}

\vspace{1cm}

\begin{tabular}{lll}
Tuwrı juwaplar sanı: \underline{\hspace{1.5cm}} & 
Bahası: \underline{\hspace{1.5cm}} & 
Imtixan alıwshınıń qolı: \underline{\hspace{2cm}} \\
\end{tabular}

\egroup

\newpage


\textbf{15-variant}\\

\bgroup
\def\arraystretch{1.6} % 1 is the default, change whatever you need

\begin{tabular}{|m{5.7cm}|m{9.5cm}|}
\hline
Familiyası hám atı & \\
\hline
Fakulteti  & \\
\hline
Toparı hám tálim baǵdarı  & \\
\hline
\end{tabular}

\vspace{1cm}

\begin{tabular}{|m{0.7cm}|m{10cm}|m{4cm}|}
\hline
№ & Soraw & Juwap \\
\hline
1. & $A_1x+B_1y+C_1z+D_1=0$ hám $Ax_2+By_2+Cz_2+D_2=0$ tegislikleri ústpe-úst túsiwi shárti? & $\frac{A_1}{A_2}=\frac{B_1}{B_2}=\frac{C_1}{C_2}=\frac{D_1}{D_2}$ \\
\hline
2. & Eki vektordıń skalyar kóbeymesiniń formulası? & $(ab)=|a||b|\cos\varphi$ \\
\hline
3. & $A_1x+B_1y+C_1z+D_1=0$ hám $Ax_2+By_2+Cz_2+D_2=0$ tegislikleri parallel bolıwı shárti & $\frac{A_1}{A_2}=\frac{B_1}{B_2}=\frac{C_1}{C_2}$ \\
\hline
4. & $Ax+C=0$ tuwrı sızıqtıń grafigi koordinata kósherlerine salıstırǵanda qanday jaylasqan? & $OY$ kósherine parallel \\
\hline
5. & Koordinatalar kósherleri hám $ 3x+4y-12=0 $ tuwrı sızıǵı menen shegaralanǵan úshmúyeshliktiń maydanın tabıń. & $ S=6 $ \\
\hline
6. & $x-2y+1=0$ teńlemesi menen berilgen tuwrınıń normal túrdegi teńlemesin kórsetiń. & $\frac{x}{- \sqrt{5}}+\frac{2y}{\sqrt{5}}-\frac{1}{\sqrt{5}}=0$ \\
\hline
7. & $3x-y+5=0$, $x+3y-4=0$ tuwrı sızıqları arasındaǵı múyeshti tabıń. & $90^{0}$ \\
\hline
8. & $\overline{a}=\{5,-6, 1 \}, \overline{b}=\{-4, 3, 0 \} $, $\overline{c}=\left\{ 5,-8, 10 \right\}$ vektorları berilgen. $2{\overline{a}}^{2}+4{\overline{b}}^{2}-5{\overline{c}}^{2}$ ańlatpasınıń mánisin tabıń. & $-721$ \\
\hline
9. & $(2, 3)$ hám $(4, 3)$ noqatlarınan ótiwshi tuwrı sızıqtıń teńlemesin dúziń. & $ y-3=0$ \\
\hline
10. & $x^{2}+y^{2}-2x+4y=0$ sheńberdiń teńlemesin kanonikalıq túrdegi teńlemege alıp keliń. & $(x-1)^{2}+(y+2)^{2}=5$ \\
\hline
\end{tabular}

\vspace{1cm}

\begin{tabular}{lll}
Tuwrı juwaplar sanı: \underline{\hspace{1.5cm}} & 
Bahası: \underline{\hspace{1.5cm}} & 
Imtixan alıwshınıń qolı: \underline{\hspace{2cm}} \\
\end{tabular}

\egroup

\newpage


\textbf{16-variant}\\

\bgroup
\def\arraystretch{1.6} % 1 is the default, change whatever you need

\begin{tabular}{|m{5.7cm}|m{9.5cm}|}
\hline
Familiyası hám atı & \\
\hline
Fakulteti  & \\
\hline
Toparı hám tálim baǵdarı  & \\
\hline
\end{tabular}

\vspace{1cm}

\begin{tabular}{|m{0.7cm}|m{10cm}|m{4cm}|}
\hline
№ & Soraw & Juwap \\
\hline
1. & Eki vektor qashan kollinear dep ataladı? & bir tuwrıda yamasa parallel tuwrıda jaylasqan bolsa \\
\hline
2. & Tuwrı múyeshli koordinatalar sisteması dep nege aytamız? & Masshtab birlikleri berilgen o'zara perpendikulyar $OX$ hám $OY$ kósherleri \\
\hline
3. & $OXY$ tegisliginiń teńlemesi? & $z=0$ \\
\hline
4. & Giperbolanıń kanonikalıq teńlemesi? & $\frac{x^2}{a^2}-\frac{y^2}{b^2}=1$ \\
\hline
5. & $A(4, 3), B(7, 7)$ noqatları arasındaǵı aralıqtı tabıń. & $d(AB)=5$ \\
\hline
6. & $3x^{2}+10xy+3y^{2}-2x-14y-13=0$ teńlemesiniń tipin anıqlań. & giperbola \\
\hline
7. & $x^{2}-4y^{2}+6x+5=0$ giperbolanıń kanonikalıq teńlemesin dúziń. & $\frac{(x+3)^{2}}{4}-\frac{y^{2}}{1}=1$ \\
\hline
8. & $M_{1}M_{2}$ kesindiniń ortasınıń koordinatalarınıń tabıń, eger $M_{1} (2, 3), M_{2} (4, 7)$ bolsa. & $(3,5)$ \\
\hline
9. & $x+y-3=0$ hám $2x+3y-8=0$ tuwrıları óz-ara qanday jaylasqan? & kesilisedi \\
\hline
10. & $x^{2}+y^{2}-2x+4y-20=0$ sheńberdiń $C$ orayın hám $R$ radiusın tabıń. & $C(1;-2), R=5$ \\
\hline
\end{tabular}

\vspace{1cm}

\begin{tabular}{lll}
Tuwrı juwaplar sanı: \underline{\hspace{1.5cm}} & 
Bahası: \underline{\hspace{1.5cm}} & 
Imtixan alıwshınıń qolı: \underline{\hspace{2cm}} \\
\end{tabular}

\egroup

\newpage


\textbf{17-variant}\\

\bgroup
\def\arraystretch{1.6} % 1 is the default, change whatever you need

\begin{tabular}{|m{5.7cm}|m{9.5cm}|}
\hline
Familiyası hám atı & \\
\hline
Fakulteti  & \\
\hline
Toparı hám tálim baǵdarı  & \\
\hline
\end{tabular}

\vspace{1cm}

\begin{tabular}{|m{0.7cm}|m{10cm}|m{4cm}|}
\hline
№ & Soraw & Juwap \\
\hline
1. & Eki vektordıń vektor kóbeymesiniń uzınlıǵın tabıw formulası? & $\left| \lbrack ab\rbrack \right|=|a||b|\sin\varphi$ \\
\hline
2. & Tegislikdegi qálegen noqatınan berilgen eki noqatqa shekemgi bolǵan aralıqlardıń ayırmasınıń modulı ózgermeytuǵın bolǵan noqatlardıń geometriyalıq ornı ne dep ataladı? & giperbola \\
\hline
3. & Eki tuwrı sızıq arasındaǵı múyeshti tabıw formulası? & $\text{tg}\varphi=\frac{k_2-k_1}{1+k_1k_2}$ \\
\hline
4. & $\frac{x^2}{a^2}-\frac{y^2}{b^2}=1$ giperbolanıń $(x_0;y_0)$ noqatındaǵı urınbasınıń teńlemesin kórsetiń. & $\frac{x_0x}{a^2}-\frac{y_0y}{b^2}=1$ \\
\hline
5. & $(x+1)^{2}+(y-2) ^{2}+(z+2) ^{2}=49$ sferanıń orayınıń koordinataların tabıń. & $(-1,2,-2)$ \\
\hline
6. & Eger $2a=16, e=\frac{5}{4}$ bolsa, fokusı abscissa kósherinde, koordinata basına salıstırǵanda simmetriyalıq jaylasqan giperbolanıń teńlemesin dúziń. & $\frac{x^{2}}{64}-\frac{y^{2}}{36}=1$ \\
\hline
7. & Eger $2b=24, 2 c=10$ bolsa, onda abscissa kósherinde koordinata basına salıstırǵanda simmetriyalıq jaylasqan fokuslarǵa iye, ellipstiń teńlemesin dúziń. & $\frac{x^{2}}{169}+\frac{y^{2}}{144}=1$ \\
\hline
8. & $M_{1} (12;-1)$ hám $M_{2} (0;4)$ noqatlardıń arasındaǵı aralıqtı tabıń. & $13$ \\
\hline
9. & $x+y=0$ teńlemesi menen berilgen tuwrı sızıqtıń múyeshlik koefficientin anıqlań. & $- 1$ \\
\hline
10. & Orayı $C (-1;2)$ noqatında, $A (-2;6 )$ noqatınan ótetuǵın sheńberdiń teńlemesin dúziń. & $(x+1)^{2}+(y-2)^{2}=17$ \\
\hline
\end{tabular}

\vspace{1cm}

\begin{tabular}{lll}
Tuwrı juwaplar sanı: \underline{\hspace{1.5cm}} & 
Bahası: \underline{\hspace{1.5cm}} & 
Imtixan alıwshınıń qolı: \underline{\hspace{2cm}} \\
\end{tabular}

\egroup

\newpage


\textbf{18-variant}\\

\bgroup
\def\arraystretch{1.6} % 1 is the default, change whatever you need

\begin{tabular}{|m{5.7cm}|m{9.5cm}|}
\hline
Familiyası hám atı & \\
\hline
Fakulteti  & \\
\hline
Toparı hám tálim baǵdarı  & \\
\hline
\end{tabular}

\vspace{1cm}

\begin{tabular}{|m{0.7cm}|m{10cm}|m{4cm}|}
\hline
№ & Soraw & Juwap \\
\hline
1. & Vektorlardıń kósherdegi proekciyasınıń formulası? & $x=|a|\cos\varphi, y=|a|\sin\varphi$ \\
\hline
2. & $Ax+By+D=0$ teńlemesi arqalı ... tegisliktiń teńlemesi berilgen? & $OZ$ kósherine parallel \\
\hline
3. & $\frac{x^2}{a^2}+\frac{y^2}{b^2}=1$ ellipstiń $(x_0;y_0)$ noqatındaǵı urınbasınıń teńlemesin tabıń. & $\frac{x_0x}{a^2}+\frac{y_0y}{b^2}=1$ \\
\hline
4. & Vektorlardı qosıw koordinatalarda qanday formula menen anıqlanadı? & $\overline{a}+\overline{b}=\{x_1+x_2;y_1+y_2\}$ \\
\hline
5. & $x+2=0$ keńislik qanday geometriyalıq betlikti anıqlaydı? &  $OYZ$ tegisligine parallel bolǵan tegislikti \\
\hline
6. & $\frac{x^{2}}{225}-\frac{y^{2}}{64}=-1$ giperbola fokusınıń koordinatalarınıń tabıń. & $F_{1}(0;-17), F_{2}(0;17)$ \\
\hline
7. & $9x^{2}+25y^{2}=225$ ellipsi berilgen, ellipstiń fokusların, ekscentrisitetin tabıń. & $F_1\left(-4;0 \right) , F_2\left( 4;0 \right) , e = \frac{4}{5}$ \\
\hline
8. & $A (-1;0;1),\ B (1;-1;0)$ noqatları berilgen. $\overline{BA}$ vektorın tabıń. & $\left\{ - 2;1;1 \right\}$ \\
\hline
9. & $2x+3y+4=0$ tuwrısına parallel hám $M_{0} (2;1)$ noqattan ótetuǵın tuwrınıń teńlemesin dúziń. & $2x+3y-7=0$ \\
\hline
10. & $x+y-12=0$ tuwrısı $x^{2}+y^{2}-2y=0$ sheńberge salıstırǵanda qanday jaylasqan? & sırtında jaylasqan \\
\hline
\end{tabular}

\vspace{1cm}

\begin{tabular}{lll}
Tuwrı juwaplar sanı: \underline{\hspace{1.5cm}} & 
Bahası: \underline{\hspace{1.5cm}} & 
Imtixan alıwshınıń qolı: \underline{\hspace{2cm}} \\
\end{tabular}

\egroup

\newpage


\textbf{19-variant}\\

\bgroup
\def\arraystretch{1.6} % 1 is the default, change whatever you need

\begin{tabular}{|m{5.7cm}|m{9.5cm}|}
\hline
Familiyası hám atı & \\
\hline
Fakulteti  & \\
\hline
Toparı hám tálim baǵdarı  & \\
\hline
\end{tabular}

\vspace{1cm}

\begin{tabular}{|m{0.7cm}|m{10cm}|m{4cm}|}
\hline
№ & Soraw & Juwap \\
\hline
1. & $OY$ kósheriniń teńlemesi? & $x=0$ \\
\hline
2. & Egerde $a=\{ x_1; y_1; z_1\}, b=\{ x_2, y_2; z_2\}$ bolsa, vektor kóbeymeniń koordinatalarda ańlatılıwı qanday boladı? &  $\lbrack ab\rbrack=\{y_1z_2-y_2z_1; z_1x_2-z_2x_1; x_1y_2-x_2y_1\}$ \\
\hline
3. & $A_1x+B_1y+C_1z+D_1=0$ hám $Ax_2+By_2+Cz_2+D_2=0$ tegislikleri perpendikulyar bolıwı shárti & $A_1\cdot A_2+B_1\cdot B_2+C_1\cdot C_2=0$ \\
\hline
4. & Úsh vektordıń aralas kóbeymesi ushın $(abc)=0$ teńligi orınlı bolsa ne dep ataladı? & $\overline{a}$, $\overline{b}$ hám $\overline{c}$ vektorları komplanar \\
\hline
5. & $\left| \overline{a} \right|=8, \left| \overline{b} \right|=5, \alpha=60^{0}$ bolsa, $( \overline{a}\overline{b} )$ ni tabıń. & $20$ \\
\hline
6. & $2x+3y-6=0$ tuwrınıń teńlemesin kesindilerde berilgen teńleme túrinde kórsetiń. & $\frac{x}{3} + \frac{ y }{ 2 } =  1$ \\
\hline
7. & $\overline{a}=\left\{ 4,-2,-4 \right\}$ hám $\overline{b}=\left\{ 6,-3, 2 \right\}$ vektorları berilgen, $(\overline{a}-\overline{b}) ^{2}$-? & $41$ \\
\hline
8. & $5x-y+7=0$ hám $3x+2y=0$ tuwrıları arasındaǵı múyeshni tabıń. & $\varphi=\frac{\pi}{4}$ \\
\hline
9. & $\overline{a}=\left\{ 2, 1, 0 \right\}$ hám $\overline{b}=\left\{ 1, 0,-1 \right\}$ bolsa, $\overline{a}-\overline{b}$ ni tabıń. & $\overline{a} -\overline{b} = \left\{ 1,1,1 \right\}$ \\
\hline
10. & Koordinatalar kósherleri hám $ 3x+4y-12=0 $ tuwrı sızıǵı menen shegaralanǵan úshmúyeshliktiń maydanın tabıń. & $ S=6 $ \\
\hline
\end{tabular}

\vspace{1cm}

\begin{tabular}{lll}
Tuwrı juwaplar sanı: \underline{\hspace{1.5cm}} & 
Bahası: \underline{\hspace{1.5cm}} & 
Imtixan alıwshınıń qolı: \underline{\hspace{2cm}} \\
\end{tabular}

\egroup

\newpage


\textbf{20-variant}\\

\bgroup
\def\arraystretch{1.6} % 1 is the default, change whatever you need

\begin{tabular}{|m{5.7cm}|m{9.5cm}|}
\hline
Familiyası hám atı & \\
\hline
Fakulteti  & \\
\hline
Toparı hám tálim baǵdarı  & \\
\hline
\end{tabular}

\vspace{1cm}

\begin{tabular}{|m{0.7cm}|m{10cm}|m{4cm}|}
\hline
№ & Soraw & Juwap \\
\hline
1. & $A_1x+B_1y+C_1z+D_1=0$ hám $Ax_2+By_2+Cz_2+D_2=0$ tegislikleri ústpe-úst túsiwi shárti? & $\frac{A_1}{A_2}=\frac{B_1}{B_2}=\frac{C_1}{C_2}=\frac{D_1}{D_2}$ \\
\hline
2. & Eki vektordıń skalyar kóbeymesiniń formulası? & $(ab)=|a||b|\cos\varphi$ \\
\hline
3. & $A_1x+B_1y+C_1z+D_1=0$ hám $Ax_2+By_2+Cz_2+D_2=0$ tegislikleri parallel bolıwı shárti & $\frac{A_1}{A_2}=\frac{B_1}{B_2}=\frac{C_1}{C_2}$ \\
\hline
4. & $Ax+C=0$ tuwrı sızıqtıń grafigi koordinata kósherlerine salıstırǵanda qanday jaylasqan? & $OY$ kósherine parallel \\
\hline
5. & $x-2y+1=0$ teńlemesi menen berilgen tuwrınıń normal túrdegi teńlemesin kórsetiń. & $\frac{x}{- \sqrt{5}}+\frac{2y}{\sqrt{5}}-\frac{1}{\sqrt{5}}=0$ \\
\hline
6. & $3x-y+5=0$, $x+3y-4=0$ tuwrı sızıqları arasındaǵı múyeshti tabıń. & $90^{0}$ \\
\hline
7. & $\overline{a}=\{5,-6, 1 \}, \overline{b}=\{-4, 3, 0 \} $, $\overline{c}=\left\{ 5,-8, 10 \right\}$ vektorları berilgen. $2{\overline{a}}^{2}+4{\overline{b}}^{2}-5{\overline{c}}^{2}$ ańlatpasınıń mánisin tabıń. & $-721$ \\
\hline
8. & $(2, 3)$ hám $(4, 3)$ noqatlarınan ótiwshi tuwrı sızıqtıń teńlemesin dúziń. & $ y-3=0$ \\
\hline
9. & $x^{2}+y^{2}-2x+4y=0$ sheńberdiń teńlemesin kanonikalıq túrdegi teńlemege alıp keliń. & $(x-1)^{2}+(y+2)^{2}=5$ \\
\hline
10. & $A(4, 3), B(7, 7)$ noqatları arasındaǵı aralıqtı tabıń. & $d(AB)=5$ \\
\hline
\end{tabular}

\vspace{1cm}

\begin{tabular}{lll}
Tuwrı juwaplar sanı: \underline{\hspace{1.5cm}} & 
Bahası: \underline{\hspace{1.5cm}} & 
Imtixan alıwshınıń qolı: \underline{\hspace{2cm}} \\
\end{tabular}

\egroup

\newpage


\textbf{21-variant}\\

\bgroup
\def\arraystretch{1.6} % 1 is the default, change whatever you need

\begin{tabular}{|m{5.7cm}|m{9.5cm}|}
\hline
Familiyası hám atı & \\
\hline
Fakulteti  & \\
\hline
Toparı hám tálim baǵdarı  & \\
\hline
\end{tabular}

\vspace{1cm}

\begin{tabular}{|m{0.7cm}|m{10cm}|m{4cm}|}
\hline
№ & Soraw & Juwap \\
\hline
1. & Eki vektor qashan kollinear dep ataladı? & bir tuwrıda yamasa parallel tuwrıda jaylasqan bolsa \\
\hline
2. & Tuwrı múyeshli koordinatalar sisteması dep nege aytamız? & Masshtab birlikleri berilgen o'zara perpendikulyar $OX$ hám $OY$ kósherleri \\
\hline
3. & $OXY$ tegisliginiń teńlemesi? & $z=0$ \\
\hline
4. & Giperbolanıń kanonikalıq teńlemesi? & $\frac{x^2}{a^2}-\frac{y^2}{b^2}=1$ \\
\hline
5. & $3x^{2}+10xy+3y^{2}-2x-14y-13=0$ teńlemesiniń tipin anıqlań. & giperbola \\
\hline
6. & $x^{2}-4y^{2}+6x+5=0$ giperbolanıń kanonikalıq teńlemesin dúziń. & $\frac{(x+3)^{2}}{4}-\frac{y^{2}}{1}=1$ \\
\hline
7. & $M_{1}M_{2}$ kesindiniń ortasınıń koordinatalarınıń tabıń, eger $M_{1} (2, 3), M_{2} (4, 7)$ bolsa. & $(3,5)$ \\
\hline
8. & $x+y-3=0$ hám $2x+3y-8=0$ tuwrıları óz-ara qanday jaylasqan? & kesilisedi \\
\hline
9. & $x^{2}+y^{2}-2x+4y-20=0$ sheńberdiń $C$ orayın hám $R$ radiusın tabıń. & $C(1;-2), R=5$ \\
\hline
10. & $(x+1)^{2}+(y-2) ^{2}+(z+2) ^{2}=49$ sferanıń orayınıń koordinataların tabıń. & $(-1,2,-2)$ \\
\hline
\end{tabular}

\vspace{1cm}

\begin{tabular}{lll}
Tuwrı juwaplar sanı: \underline{\hspace{1.5cm}} & 
Bahası: \underline{\hspace{1.5cm}} & 
Imtixan alıwshınıń qolı: \underline{\hspace{2cm}} \\
\end{tabular}

\egroup

\newpage


\textbf{22-variant}\\

\bgroup
\def\arraystretch{1.6} % 1 is the default, change whatever you need

\begin{tabular}{|m{5.7cm}|m{9.5cm}|}
\hline
Familiyası hám atı & \\
\hline
Fakulteti  & \\
\hline
Toparı hám tálim baǵdarı  & \\
\hline
\end{tabular}

\vspace{1cm}

\begin{tabular}{|m{0.7cm}|m{10cm}|m{4cm}|}
\hline
№ & Soraw & Juwap \\
\hline
1. & Eki vektordıń vektor kóbeymesiniń uzınlıǵın tabıw formulası? & $\left| \lbrack ab\rbrack \right|=|a||b|\sin\varphi$ \\
\hline
2. & Tegislikdegi qálegen noqatınan berilgen eki noqatqa shekemgi bolǵan aralıqlardıń ayırmasınıń modulı ózgermeytuǵın bolǵan noqatlardıń geometriyalıq ornı ne dep ataladı? & giperbola \\
\hline
3. & Eki tuwrı sızıq arasındaǵı múyeshti tabıw formulası? & $\text{tg}\varphi=\frac{k_2-k_1}{1+k_1k_2}$ \\
\hline
4. & $\frac{x^2}{a^2}-\frac{y^2}{b^2}=1$ giperbolanıń $(x_0;y_0)$ noqatındaǵı urınbasınıń teńlemesin kórsetiń. & $\frac{x_0x}{a^2}-\frac{y_0y}{b^2}=1$ \\
\hline
5. & Eger $2a=16, e=\frac{5}{4}$ bolsa, fokusı abscissa kósherinde, koordinata basına salıstırǵanda simmetriyalıq jaylasqan giperbolanıń teńlemesin dúziń. & $\frac{x^{2}}{64}-\frac{y^{2}}{36}=1$ \\
\hline
6. & Eger $2b=24, 2 c=10$ bolsa, onda abscissa kósherinde koordinata basına salıstırǵanda simmetriyalıq jaylasqan fokuslarǵa iye, ellipstiń teńlemesin dúziń. & $\frac{x^{2}}{169}+\frac{y^{2}}{144}=1$ \\
\hline
7. & $M_{1} (12;-1)$ hám $M_{2} (0;4)$ noqatlardıń arasındaǵı aralıqtı tabıń. & $13$ \\
\hline
8. & $x+y=0$ teńlemesi menen berilgen tuwrı sızıqtıń múyeshlik koefficientin anıqlań. & $- 1$ \\
\hline
9. & Orayı $C (-1;2)$ noqatında, $A (-2;6 )$ noqatınan ótetuǵın sheńberdiń teńlemesin dúziń. & $(x+1)^{2}+(y-2)^{2}=17$ \\
\hline
10. & $x+2=0$ keńislik qanday geometriyalıq betlikti anıqlaydı? &  $OYZ$ tegisligine parallel bolǵan tegislikti \\
\hline
\end{tabular}

\vspace{1cm}

\begin{tabular}{lll}
Tuwrı juwaplar sanı: \underline{\hspace{1.5cm}} & 
Bahası: \underline{\hspace{1.5cm}} & 
Imtixan alıwshınıń qolı: \underline{\hspace{2cm}} \\
\end{tabular}

\egroup

\newpage


\textbf{23-variant}\\

\bgroup
\def\arraystretch{1.6} % 1 is the default, change whatever you need

\begin{tabular}{|m{5.7cm}|m{9.5cm}|}
\hline
Familiyası hám atı & \\
\hline
Fakulteti  & \\
\hline
Toparı hám tálim baǵdarı  & \\
\hline
\end{tabular}

\vspace{1cm}

\begin{tabular}{|m{0.7cm}|m{10cm}|m{4cm}|}
\hline
№ & Soraw & Juwap \\
\hline
1. & Vektorlardıń kósherdegi proekciyasınıń formulası? & $x=|a|\cos\varphi, y=|a|\sin\varphi$ \\
\hline
2. & $Ax+By+D=0$ teńlemesi arqalı ... tegisliktiń teńlemesi berilgen? & $OZ$ kósherine parallel \\
\hline
3. & $\frac{x^2}{a^2}+\frac{y^2}{b^2}=1$ ellipstiń $(x_0;y_0)$ noqatındaǵı urınbasınıń teńlemesin tabıń. & $\frac{x_0x}{a^2}+\frac{y_0y}{b^2}=1$ \\
\hline
4. & Vektorlardı qosıw koordinatalarda qanday formula menen anıqlanadı? & $\overline{a}+\overline{b}=\{x_1+x_2;y_1+y_2\}$ \\
\hline
5. & $\frac{x^{2}}{225}-\frac{y^{2}}{64}=-1$ giperbola fokusınıń koordinatalarınıń tabıń. & $F_{1}(0;-17), F_{2}(0;17)$ \\
\hline
6. & $9x^{2}+25y^{2}=225$ ellipsi berilgen, ellipstiń fokusların, ekscentrisitetin tabıń. & $F_1\left(-4;0 \right) , F_2\left( 4;0 \right) , e = \frac{4}{5}$ \\
\hline
7. & $A (-1;0;1),\ B (1;-1;0)$ noqatları berilgen. $\overline{BA}$ vektorın tabıń. & $\left\{ - 2;1;1 \right\}$ \\
\hline
8. & $2x+3y+4=0$ tuwrısına parallel hám $M_{0} (2;1)$ noqattan ótetuǵın tuwrınıń teńlemesin dúziń. & $2x+3y-7=0$ \\
\hline
9. & $x+y-12=0$ tuwrısı $x^{2}+y^{2}-2y=0$ sheńberge salıstırǵanda qanday jaylasqan? & sırtında jaylasqan \\
\hline
10. & $\left| \overline{a} \right|=8, \left| \overline{b} \right|=5, \alpha=60^{0}$ bolsa, $( \overline{a}\overline{b} )$ ni tabıń. & $20$ \\
\hline
\end{tabular}

\vspace{1cm}

\begin{tabular}{lll}
Tuwrı juwaplar sanı: \underline{\hspace{1.5cm}} & 
Bahası: \underline{\hspace{1.5cm}} & 
Imtixan alıwshınıń qolı: \underline{\hspace{2cm}} \\
\end{tabular}

\egroup

\newpage


\textbf{24-variant}\\

\bgroup
\def\arraystretch{1.6} % 1 is the default, change whatever you need

\begin{tabular}{|m{5.7cm}|m{9.5cm}|}
\hline
Familiyası hám atı & \\
\hline
Fakulteti  & \\
\hline
Toparı hám tálim baǵdarı  & \\
\hline
\end{tabular}

\vspace{1cm}

\begin{tabular}{|m{0.7cm}|m{10cm}|m{4cm}|}
\hline
№ & Soraw & Juwap \\
\hline
1. & $OY$ kósheriniń teńlemesi? & $x=0$ \\
\hline
2. & Egerde $a=\{ x_1; y_1; z_1\}, b=\{ x_2, y_2; z_2\}$ bolsa, vektor kóbeymeniń koordinatalarda ańlatılıwı qanday boladı? &  $\lbrack ab\rbrack=\{y_1z_2-y_2z_1; z_1x_2-z_2x_1; x_1y_2-x_2y_1\}$ \\
\hline
3. & $A_1x+B_1y+C_1z+D_1=0$ hám $Ax_2+By_2+Cz_2+D_2=0$ tegislikleri perpendikulyar bolıwı shárti & $A_1\cdot A_2+B_1\cdot B_2+C_1\cdot C_2=0$ \\
\hline
4. & Úsh vektordıń aralas kóbeymesi ushın $(abc)=0$ teńligi orınlı bolsa ne dep ataladı? & $\overline{a}$, $\overline{b}$ hám $\overline{c}$ vektorları komplanar \\
\hline
5. & $2x+3y-6=0$ tuwrınıń teńlemesin kesindilerde berilgen teńleme túrinde kórsetiń. & $\frac{x}{3} + \frac{ y }{ 2 } =  1$ \\
\hline
6. & $\overline{a}=\left\{ 4,-2,-4 \right\}$ hám $\overline{b}=\left\{ 6,-3, 2 \right\}$ vektorları berilgen, $(\overline{a}-\overline{b}) ^{2}$-? & $41$ \\
\hline
7. & $5x-y+7=0$ hám $3x+2y=0$ tuwrıları arasındaǵı múyeshni tabıń. & $\varphi=\frac{\pi}{4}$ \\
\hline
8. & $\overline{a}=\left\{ 2, 1, 0 \right\}$ hám $\overline{b}=\left\{ 1, 0,-1 \right\}$ bolsa, $\overline{a}-\overline{b}$ ni tabıń. & $\overline{a} -\overline{b} = \left\{ 1,1,1 \right\}$ \\
\hline
9. & Koordinatalar kósherleri hám $ 3x+4y-12=0 $ tuwrı sızıǵı menen shegaralanǵan úshmúyeshliktiń maydanın tabıń. & $ S=6 $ \\
\hline
10. & $x-2y+1=0$ teńlemesi menen berilgen tuwrınıń normal túrdegi teńlemesin kórsetiń. & $\frac{x}{- \sqrt{5}}+\frac{2y}{\sqrt{5}}-\frac{1}{\sqrt{5}}=0$ \\
\hline
\end{tabular}

\vspace{1cm}

\begin{tabular}{lll}
Tuwrı juwaplar sanı: \underline{\hspace{1.5cm}} & 
Bahası: \underline{\hspace{1.5cm}} & 
Imtixan alıwshınıń qolı: \underline{\hspace{2cm}} \\
\end{tabular}

\egroup

\newpage


\textbf{25-variant}\\

\bgroup
\def\arraystretch{1.6} % 1 is the default, change whatever you need

\begin{tabular}{|m{5.7cm}|m{9.5cm}|}
\hline
Familiyası hám atı & \\
\hline
Fakulteti  & \\
\hline
Toparı hám tálim baǵdarı  & \\
\hline
\end{tabular}

\vspace{1cm}

\begin{tabular}{|m{0.7cm}|m{10cm}|m{4cm}|}
\hline
№ & Soraw & Juwap \\
\hline
1. & $A_1x+B_1y+C_1z+D_1=0$ hám $Ax_2+By_2+Cz_2+D_2=0$ tegislikleri ústpe-úst túsiwi shárti? & $\frac{A_1}{A_2}=\frac{B_1}{B_2}=\frac{C_1}{C_2}=\frac{D_1}{D_2}$ \\
\hline
2. & Eki vektordıń skalyar kóbeymesiniń formulası? & $(ab)=|a||b|\cos\varphi$ \\
\hline
3. & $A_1x+B_1y+C_1z+D_1=0$ hám $Ax_2+By_2+Cz_2+D_2=0$ tegislikleri parallel bolıwı shárti & $\frac{A_1}{A_2}=\frac{B_1}{B_2}=\frac{C_1}{C_2}$ \\
\hline
4. & $Ax+C=0$ tuwrı sızıqtıń grafigi koordinata kósherlerine salıstırǵanda qanday jaylasqan? & $OY$ kósherine parallel \\
\hline
5. & $3x-y+5=0$, $x+3y-4=0$ tuwrı sızıqları arasındaǵı múyeshti tabıń. & $90^{0}$ \\
\hline
6. & $\overline{a}=\{5,-6, 1 \}, \overline{b}=\{-4, 3, 0 \} $, $\overline{c}=\left\{ 5,-8, 10 \right\}$ vektorları berilgen. $2{\overline{a}}^{2}+4{\overline{b}}^{2}-5{\overline{c}}^{2}$ ańlatpasınıń mánisin tabıń. & $-721$ \\
\hline
7. & $(2, 3)$ hám $(4, 3)$ noqatlarınan ótiwshi tuwrı sızıqtıń teńlemesin dúziń. & $ y-3=0$ \\
\hline
8. & $x^{2}+y^{2}-2x+4y=0$ sheńberdiń teńlemesin kanonikalıq túrdegi teńlemege alıp keliń. & $(x-1)^{2}+(y+2)^{2}=5$ \\
\hline
9. & $A(4, 3), B(7, 7)$ noqatları arasındaǵı aralıqtı tabıń. & $d(AB)=5$ \\
\hline
10. & $3x^{2}+10xy+3y^{2}-2x-14y-13=0$ teńlemesiniń tipin anıqlań. & giperbola \\
\hline
\end{tabular}

\vspace{1cm}

\begin{tabular}{lll}
Tuwrı juwaplar sanı: \underline{\hspace{1.5cm}} & 
Bahası: \underline{\hspace{1.5cm}} & 
Imtixan alıwshınıń qolı: \underline{\hspace{2cm}} \\
\end{tabular}

\egroup

\newpage


\textbf{26-variant}\\

\bgroup
\def\arraystretch{1.6} % 1 is the default, change whatever you need

\begin{tabular}{|m{5.7cm}|m{9.5cm}|}
\hline
Familiyası hám atı & \\
\hline
Fakulteti  & \\
\hline
Toparı hám tálim baǵdarı  & \\
\hline
\end{tabular}

\vspace{1cm}

\begin{tabular}{|m{0.7cm}|m{10cm}|m{4cm}|}
\hline
№ & Soraw & Juwap \\
\hline
1. & Eki vektor qashan kollinear dep ataladı? & bir tuwrıda yamasa parallel tuwrıda jaylasqan bolsa \\
\hline
2. & Tuwrı múyeshli koordinatalar sisteması dep nege aytamız? & Masshtab birlikleri berilgen o'zara perpendikulyar $OX$ hám $OY$ kósherleri \\
\hline
3. & $OXY$ tegisliginiń teńlemesi? & $z=0$ \\
\hline
4. & Giperbolanıń kanonikalıq teńlemesi? & $\frac{x^2}{a^2}-\frac{y^2}{b^2}=1$ \\
\hline
5. & $x^{2}-4y^{2}+6x+5=0$ giperbolanıń kanonikalıq teńlemesin dúziń. & $\frac{(x+3)^{2}}{4}-\frac{y^{2}}{1}=1$ \\
\hline
6. & $M_{1}M_{2}$ kesindiniń ortasınıń koordinatalarınıń tabıń, eger $M_{1} (2, 3), M_{2} (4, 7)$ bolsa. & $(3,5)$ \\
\hline
7. & $x+y-3=0$ hám $2x+3y-8=0$ tuwrıları óz-ara qanday jaylasqan? & kesilisedi \\
\hline
8. & $x^{2}+y^{2}-2x+4y-20=0$ sheńberdiń $C$ orayın hám $R$ radiusın tabıń. & $C(1;-2), R=5$ \\
\hline
9. & $(x+1)^{2}+(y-2) ^{2}+(z+2) ^{2}=49$ sferanıń orayınıń koordinataların tabıń. & $(-1,2,-2)$ \\
\hline
10. & Eger $2a=16, e=\frac{5}{4}$ bolsa, fokusı abscissa kósherinde, koordinata basına salıstırǵanda simmetriyalıq jaylasqan giperbolanıń teńlemesin dúziń. & $\frac{x^{2}}{64}-\frac{y^{2}}{36}=1$ \\
\hline
\end{tabular}

\vspace{1cm}

\begin{tabular}{lll}
Tuwrı juwaplar sanı: \underline{\hspace{1.5cm}} & 
Bahası: \underline{\hspace{1.5cm}} & 
Imtixan alıwshınıń qolı: \underline{\hspace{2cm}} \\
\end{tabular}

\egroup

\newpage


\textbf{27-variant}\\

\bgroup
\def\arraystretch{1.6} % 1 is the default, change whatever you need

\begin{tabular}{|m{5.7cm}|m{9.5cm}|}
\hline
Familiyası hám atı & \\
\hline
Fakulteti  & \\
\hline
Toparı hám tálim baǵdarı  & \\
\hline
\end{tabular}

\vspace{1cm}

\begin{tabular}{|m{0.7cm}|m{10cm}|m{4cm}|}
\hline
№ & Soraw & Juwap \\
\hline
1. & Eki vektordıń vektor kóbeymesiniń uzınlıǵın tabıw formulası? & $\left| \lbrack ab\rbrack \right|=|a||b|\sin\varphi$ \\
\hline
2. & Tegislikdegi qálegen noqatınan berilgen eki noqatqa shekemgi bolǵan aralıqlardıń ayırmasınıń modulı ózgermeytuǵın bolǵan noqatlardıń geometriyalıq ornı ne dep ataladı? & giperbola \\
\hline
3. & Eki tuwrı sızıq arasındaǵı múyeshti tabıw formulası? & $\text{tg}\varphi=\frac{k_2-k_1}{1+k_1k_2}$ \\
\hline
4. & $\frac{x^2}{a^2}-\frac{y^2}{b^2}=1$ giperbolanıń $(x_0;y_0)$ noqatındaǵı urınbasınıń teńlemesin kórsetiń. & $\frac{x_0x}{a^2}-\frac{y_0y}{b^2}=1$ \\
\hline
5. & Eger $2b=24, 2 c=10$ bolsa, onda abscissa kósherinde koordinata basına salıstırǵanda simmetriyalıq jaylasqan fokuslarǵa iye, ellipstiń teńlemesin dúziń. & $\frac{x^{2}}{169}+\frac{y^{2}}{144}=1$ \\
\hline
6. & $M_{1} (12;-1)$ hám $M_{2} (0;4)$ noqatlardıń arasındaǵı aralıqtı tabıń. & $13$ \\
\hline
7. & $x+y=0$ teńlemesi menen berilgen tuwrı sızıqtıń múyeshlik koefficientin anıqlań. & $- 1$ \\
\hline
8. & Orayı $C (-1;2)$ noqatında, $A (-2;6 )$ noqatınan ótetuǵın sheńberdiń teńlemesin dúziń. & $(x+1)^{2}+(y-2)^{2}=17$ \\
\hline
9. & $x+2=0$ keńislik qanday geometriyalıq betlikti anıqlaydı? &  $OYZ$ tegisligine parallel bolǵan tegislikti \\
\hline
10. & $\frac{x^{2}}{225}-\frac{y^{2}}{64}=-1$ giperbola fokusınıń koordinatalarınıń tabıń. & $F_{1}(0;-17), F_{2}(0;17)$ \\
\hline
\end{tabular}

\vspace{1cm}

\begin{tabular}{lll}
Tuwrı juwaplar sanı: \underline{\hspace{1.5cm}} & 
Bahası: \underline{\hspace{1.5cm}} & 
Imtixan alıwshınıń qolı: \underline{\hspace{2cm}} \\
\end{tabular}

\egroup

\newpage


\textbf{28-variant}\\

\bgroup
\def\arraystretch{1.6} % 1 is the default, change whatever you need

\begin{tabular}{|m{5.7cm}|m{9.5cm}|}
\hline
Familiyası hám atı & \\
\hline
Fakulteti  & \\
\hline
Toparı hám tálim baǵdarı  & \\
\hline
\end{tabular}

\vspace{1cm}

\begin{tabular}{|m{0.7cm}|m{10cm}|m{4cm}|}
\hline
№ & Soraw & Juwap \\
\hline
1. & Vektorlardıń kósherdegi proekciyasınıń formulası? & $x=|a|\cos\varphi, y=|a|\sin\varphi$ \\
\hline
2. & $Ax+By+D=0$ teńlemesi arqalı ... tegisliktiń teńlemesi berilgen? & $OZ$ kósherine parallel \\
\hline
3. & $\frac{x^2}{a^2}+\frac{y^2}{b^2}=1$ ellipstiń $(x_0;y_0)$ noqatındaǵı urınbasınıń teńlemesin tabıń. & $\frac{x_0x}{a^2}+\frac{y_0y}{b^2}=1$ \\
\hline
4. & Vektorlardı qosıw koordinatalarda qanday formula menen anıqlanadı? & $\overline{a}+\overline{b}=\{x_1+x_2;y_1+y_2\}$ \\
\hline
5. & $9x^{2}+25y^{2}=225$ ellipsi berilgen, ellipstiń fokusların, ekscentrisitetin tabıń. & $F_1\left(-4;0 \right) , F_2\left( 4;0 \right) , e = \frac{4}{5}$ \\
\hline
6. & $A (-1;0;1),\ B (1;-1;0)$ noqatları berilgen. $\overline{BA}$ vektorın tabıń. & $\left\{ - 2;1;1 \right\}$ \\
\hline
7. & $2x+3y+4=0$ tuwrısına parallel hám $M_{0} (2;1)$ noqattan ótetuǵın tuwrınıń teńlemesin dúziń. & $2x+3y-7=0$ \\
\hline
8. & $x+y-12=0$ tuwrısı $x^{2}+y^{2}-2y=0$ sheńberge salıstırǵanda qanday jaylasqan? & sırtında jaylasqan \\
\hline
9. & $\left| \overline{a} \right|=8, \left| \overline{b} \right|=5, \alpha=60^{0}$ bolsa, $( \overline{a}\overline{b} )$ ni tabıń. & $20$ \\
\hline
10. & $2x+3y-6=0$ tuwrınıń teńlemesin kesindilerde berilgen teńleme túrinde kórsetiń. & $\frac{x}{3} + \frac{ y }{ 2 } =  1$ \\
\hline
\end{tabular}

\vspace{1cm}

\begin{tabular}{lll}
Tuwrı juwaplar sanı: \underline{\hspace{1.5cm}} & 
Bahası: \underline{\hspace{1.5cm}} & 
Imtixan alıwshınıń qolı: \underline{\hspace{2cm}} \\
\end{tabular}

\egroup

\newpage


\textbf{29-variant}\\

\bgroup
\def\arraystretch{1.6} % 1 is the default, change whatever you need

\begin{tabular}{|m{5.7cm}|m{9.5cm}|}
\hline
Familiyası hám atı & \\
\hline
Fakulteti  & \\
\hline
Toparı hám tálim baǵdarı  & \\
\hline
\end{tabular}

\vspace{1cm}

\begin{tabular}{|m{0.7cm}|m{10cm}|m{4cm}|}
\hline
№ & Soraw & Juwap \\
\hline
1. & $OY$ kósheriniń teńlemesi? & $x=0$ \\
\hline
2. & Egerde $a=\{ x_1; y_1; z_1\}, b=\{ x_2, y_2; z_2\}$ bolsa, vektor kóbeymeniń koordinatalarda ańlatılıwı qanday boladı? &  $\lbrack ab\rbrack=\{y_1z_2-y_2z_1; z_1x_2-z_2x_1; x_1y_2-x_2y_1\}$ \\
\hline
3. & $A_1x+B_1y+C_1z+D_1=0$ hám $Ax_2+By_2+Cz_2+D_2=0$ tegislikleri perpendikulyar bolıwı shárti & $A_1\cdot A_2+B_1\cdot B_2+C_1\cdot C_2=0$ \\
\hline
4. & Úsh vektordıń aralas kóbeymesi ushın $(abc)=0$ teńligi orınlı bolsa ne dep ataladı? & $\overline{a}$, $\overline{b}$ hám $\overline{c}$ vektorları komplanar \\
\hline
5. & $\overline{a}=\left\{ 4,-2,-4 \right\}$ hám $\overline{b}=\left\{ 6,-3, 2 \right\}$ vektorları berilgen, $(\overline{a}-\overline{b}) ^{2}$-? & $41$ \\
\hline
6. & $5x-y+7=0$ hám $3x+2y=0$ tuwrıları arasındaǵı múyeshni tabıń. & $\varphi=\frac{\pi}{4}$ \\
\hline
7. & $\overline{a}=\left\{ 2, 1, 0 \right\}$ hám $\overline{b}=\left\{ 1, 0,-1 \right\}$ bolsa, $\overline{a}-\overline{b}$ ni tabıń. & $\overline{a} -\overline{b} = \left\{ 1,1,1 \right\}$ \\
\hline
8. & Koordinatalar kósherleri hám $ 3x+4y-12=0 $ tuwrı sızıǵı menen shegaralanǵan úshmúyeshliktiń maydanın tabıń. & $ S=6 $ \\
\hline
9. & $x-2y+1=0$ teńlemesi menen berilgen tuwrınıń normal túrdegi teńlemesin kórsetiń. & $\frac{x}{- \sqrt{5}}+\frac{2y}{\sqrt{5}}-\frac{1}{\sqrt{5}}=0$ \\
\hline
10. & $3x-y+5=0$, $x+3y-4=0$ tuwrı sızıqları arasındaǵı múyeshti tabıń. & $90^{0}$ \\
\hline
\end{tabular}

\vspace{1cm}

\begin{tabular}{lll}
Tuwrı juwaplar sanı: \underline{\hspace{1.5cm}} & 
Bahası: \underline{\hspace{1.5cm}} & 
Imtixan alıwshınıń qolı: \underline{\hspace{2cm}} \\
\end{tabular}

\egroup

\newpage


\textbf{30-variant}\\

\bgroup
\def\arraystretch{1.6} % 1 is the default, change whatever you need

\begin{tabular}{|m{5.7cm}|m{9.5cm}|}
\hline
Familiyası hám atı & \\
\hline
Fakulteti  & \\
\hline
Toparı hám tálim baǵdarı  & \\
\hline
\end{tabular}

\vspace{1cm}

\begin{tabular}{|m{0.7cm}|m{10cm}|m{4cm}|}
\hline
№ & Soraw & Juwap \\
\hline
1. & $A_1x+B_1y+C_1z+D_1=0$ hám $Ax_2+By_2+Cz_2+D_2=0$ tegislikleri ústpe-úst túsiwi shárti? & $\frac{A_1}{A_2}=\frac{B_1}{B_2}=\frac{C_1}{C_2}=\frac{D_1}{D_2}$ \\
\hline
2. & Eki vektordıń skalyar kóbeymesiniń formulası? & $(ab)=|a||b|\cos\varphi$ \\
\hline
3. & $A_1x+B_1y+C_1z+D_1=0$ hám $Ax_2+By_2+Cz_2+D_2=0$ tegislikleri parallel bolıwı shárti & $\frac{A_1}{A_2}=\frac{B_1}{B_2}=\frac{C_1}{C_2}$ \\
\hline
4. & $Ax+C=0$ tuwrı sızıqtıń grafigi koordinata kósherlerine salıstırǵanda qanday jaylasqan? & $OY$ kósherine parallel \\
\hline
5. & $\overline{a}=\{5,-6, 1 \}, \overline{b}=\{-4, 3, 0 \} $, $\overline{c}=\left\{ 5,-8, 10 \right\}$ vektorları berilgen. $2{\overline{a}}^{2}+4{\overline{b}}^{2}-5{\overline{c}}^{2}$ ańlatpasınıń mánisin tabıń. & $-721$ \\
\hline
6. & $(2, 3)$ hám $(4, 3)$ noqatlarınan ótiwshi tuwrı sızıqtıń teńlemesin dúziń. & $ y-3=0$ \\
\hline
7. & $x^{2}+y^{2}-2x+4y=0$ sheńberdiń teńlemesin kanonikalıq túrdegi teńlemege alıp keliń. & $(x-1)^{2}+(y+2)^{2}=5$ \\
\hline
8. & $A(4, 3), B(7, 7)$ noqatları arasındaǵı aralıqtı tabıń. & $d(AB)=5$ \\
\hline
9. & $3x^{2}+10xy+3y^{2}-2x-14y-13=0$ teńlemesiniń tipin anıqlań. & giperbola \\
\hline
10. & $x^{2}-4y^{2}+6x+5=0$ giperbolanıń kanonikalıq teńlemesin dúziń. & $\frac{(x+3)^{2}}{4}-\frac{y^{2}}{1}=1$ \\
\hline
\end{tabular}

\vspace{1cm}

\begin{tabular}{lll}
Tuwrı juwaplar sanı: \underline{\hspace{1.5cm}} & 
Bahası: \underline{\hspace{1.5cm}} & 
Imtixan alıwshınıń qolı: \underline{\hspace{2cm}} \\
\end{tabular}

\egroup

\newpage


\textbf{31-variant}\\

\bgroup
\def\arraystretch{1.6} % 1 is the default, change whatever you need

\begin{tabular}{|m{5.7cm}|m{9.5cm}|}
\hline
Familiyası hám atı & \\
\hline
Fakulteti  & \\
\hline
Toparı hám tálim baǵdarı  & \\
\hline
\end{tabular}

\vspace{1cm}

\begin{tabular}{|m{0.7cm}|m{10cm}|m{4cm}|}
\hline
№ & Soraw & Juwap \\
\hline
1. & Eki vektor qashan kollinear dep ataladı? & bir tuwrıda yamasa parallel tuwrıda jaylasqan bolsa \\
\hline
2. & Tuwrı múyeshli koordinatalar sisteması dep nege aytamız? & Masshtab birlikleri berilgen o'zara perpendikulyar $OX$ hám $OY$ kósherleri \\
\hline
3. & $OXY$ tegisliginiń teńlemesi? & $z=0$ \\
\hline
4. & Giperbolanıń kanonikalıq teńlemesi? & $\frac{x^2}{a^2}-\frac{y^2}{b^2}=1$ \\
\hline
5. & $M_{1}M_{2}$ kesindiniń ortasınıń koordinatalarınıń tabıń, eger $M_{1} (2, 3), M_{2} (4, 7)$ bolsa. & $(3,5)$ \\
\hline
6. & $x+y-3=0$ hám $2x+3y-8=0$ tuwrıları óz-ara qanday jaylasqan? & kesilisedi \\
\hline
7. & $x^{2}+y^{2}-2x+4y-20=0$ sheńberdiń $C$ orayın hám $R$ radiusın tabıń. & $C(1;-2), R=5$ \\
\hline
8. & $(x+1)^{2}+(y-2) ^{2}+(z+2) ^{2}=49$ sferanıń orayınıń koordinataların tabıń. & $(-1,2,-2)$ \\
\hline
9. & Eger $2a=16, e=\frac{5}{4}$ bolsa, fokusı abscissa kósherinde, koordinata basına salıstırǵanda simmetriyalıq jaylasqan giperbolanıń teńlemesin dúziń. & $\frac{x^{2}}{64}-\frac{y^{2}}{36}=1$ \\
\hline
10. & Eger $2b=24, 2 c=10$ bolsa, onda abscissa kósherinde koordinata basına salıstırǵanda simmetriyalıq jaylasqan fokuslarǵa iye, ellipstiń teńlemesin dúziń. & $\frac{x^{2}}{169}+\frac{y^{2}}{144}=1$ \\
\hline
\end{tabular}

\vspace{1cm}

\begin{tabular}{lll}
Tuwrı juwaplar sanı: \underline{\hspace{1.5cm}} & 
Bahası: \underline{\hspace{1.5cm}} & 
Imtixan alıwshınıń qolı: \underline{\hspace{2cm}} \\
\end{tabular}

\egroup

\newpage


\textbf{32-variant}\\

\bgroup
\def\arraystretch{1.6} % 1 is the default, change whatever you need

\begin{tabular}{|m{5.7cm}|m{9.5cm}|}
\hline
Familiyası hám atı & \\
\hline
Fakulteti  & \\
\hline
Toparı hám tálim baǵdarı  & \\
\hline
\end{tabular}

\vspace{1cm}

\begin{tabular}{|m{0.7cm}|m{10cm}|m{4cm}|}
\hline
№ & Soraw & Juwap \\
\hline
1. & Eki vektordıń vektor kóbeymesiniń uzınlıǵın tabıw formulası? & $\left| \lbrack ab\rbrack \right|=|a||b|\sin\varphi$ \\
\hline
2. & Tegislikdegi qálegen noqatınan berilgen eki noqatqa shekemgi bolǵan aralıqlardıń ayırmasınıń modulı ózgermeytuǵın bolǵan noqatlardıń geometriyalıq ornı ne dep ataladı? & giperbola \\
\hline
3. & Eki tuwrı sızıq arasındaǵı múyeshti tabıw formulası? & $\text{tg}\varphi=\frac{k_2-k_1}{1+k_1k_2}$ \\
\hline
4. & $\frac{x^2}{a^2}-\frac{y^2}{b^2}=1$ giperbolanıń $(x_0;y_0)$ noqatındaǵı urınbasınıń teńlemesin kórsetiń. & $\frac{x_0x}{a^2}-\frac{y_0y}{b^2}=1$ \\
\hline
5. & $M_{1} (12;-1)$ hám $M_{2} (0;4)$ noqatlardıń arasındaǵı aralıqtı tabıń. & $13$ \\
\hline
6. & $x+y=0$ teńlemesi menen berilgen tuwrı sızıqtıń múyeshlik koefficientin anıqlań. & $- 1$ \\
\hline
7. & Orayı $C (-1;2)$ noqatında, $A (-2;6 )$ noqatınan ótetuǵın sheńberdiń teńlemesin dúziń. & $(x+1)^{2}+(y-2)^{2}=17$ \\
\hline
8. & $x+2=0$ keńislik qanday geometriyalıq betlikti anıqlaydı? &  $OYZ$ tegisligine parallel bolǵan tegislikti \\
\hline
9. & $\frac{x^{2}}{225}-\frac{y^{2}}{64}=-1$ giperbola fokusınıń koordinatalarınıń tabıń. & $F_{1}(0;-17), F_{2}(0;17)$ \\
\hline
10. & $9x^{2}+25y^{2}=225$ ellipsi berilgen, ellipstiń fokusların, ekscentrisitetin tabıń. & $F_1\left(-4;0 \right) , F_2\left( 4;0 \right) , e = \frac{4}{5}$ \\
\hline
\end{tabular}

\vspace{1cm}

\begin{tabular}{lll}
Tuwrı juwaplar sanı: \underline{\hspace{1.5cm}} & 
Bahası: \underline{\hspace{1.5cm}} & 
Imtixan alıwshınıń qolı: \underline{\hspace{2cm}} \\
\end{tabular}

\egroup

\newpage


\textbf{33-variant}\\

\bgroup
\def\arraystretch{1.6} % 1 is the default, change whatever you need

\begin{tabular}{|m{5.7cm}|m{9.5cm}|}
\hline
Familiyası hám atı & \\
\hline
Fakulteti  & \\
\hline
Toparı hám tálim baǵdarı  & \\
\hline
\end{tabular}

\vspace{1cm}

\begin{tabular}{|m{0.7cm}|m{10cm}|m{4cm}|}
\hline
№ & Soraw & Juwap \\
\hline
1. & Vektorlardıń kósherdegi proekciyasınıń formulası? & $x=|a|\cos\varphi, y=|a|\sin\varphi$ \\
\hline
2. & $Ax+By+D=0$ teńlemesi arqalı ... tegisliktiń teńlemesi berilgen? & $OZ$ kósherine parallel \\
\hline
3. & $\frac{x^2}{a^2}+\frac{y^2}{b^2}=1$ ellipstiń $(x_0;y_0)$ noqatındaǵı urınbasınıń teńlemesin tabıń. & $\frac{x_0x}{a^2}+\frac{y_0y}{b^2}=1$ \\
\hline
4. & Vektorlardı qosıw koordinatalarda qanday formula menen anıqlanadı? & $\overline{a}+\overline{b}=\{x_1+x_2;y_1+y_2\}$ \\
\hline
5. & $A (-1;0;1),\ B (1;-1;0)$ noqatları berilgen. $\overline{BA}$ vektorın tabıń. & $\left\{ - 2;1;1 \right\}$ \\
\hline
6. & $2x+3y+4=0$ tuwrısına parallel hám $M_{0} (2;1)$ noqattan ótetuǵın tuwrınıń teńlemesin dúziń. & $2x+3y-7=0$ \\
\hline
7. & $x+y-12=0$ tuwrısı $x^{2}+y^{2}-2y=0$ sheńberge salıstırǵanda qanday jaylasqan? & sırtında jaylasqan \\
\hline
8. & $\left| \overline{a} \right|=8, \left| \overline{b} \right|=5, \alpha=60^{0}$ bolsa, $( \overline{a}\overline{b} )$ ni tabıń. & $20$ \\
\hline
9. & $2x+3y-6=0$ tuwrınıń teńlemesin kesindilerde berilgen teńleme túrinde kórsetiń. & $\frac{x}{3} + \frac{ y }{ 2 } =  1$ \\
\hline
10. & $\overline{a}=\left\{ 4,-2,-4 \right\}$ hám $\overline{b}=\left\{ 6,-3, 2 \right\}$ vektorları berilgen, $(\overline{a}-\overline{b}) ^{2}$-? & $41$ \\
\hline
\end{tabular}

\vspace{1cm}

\begin{tabular}{lll}
Tuwrı juwaplar sanı: \underline{\hspace{1.5cm}} & 
Bahası: \underline{\hspace{1.5cm}} & 
Imtixan alıwshınıń qolı: \underline{\hspace{2cm}} \\
\end{tabular}

\egroup

\newpage


\textbf{34-variant}\\

\bgroup
\def\arraystretch{1.6} % 1 is the default, change whatever you need

\begin{tabular}{|m{5.7cm}|m{9.5cm}|}
\hline
Familiyası hám atı & \\
\hline
Fakulteti  & \\
\hline
Toparı hám tálim baǵdarı  & \\
\hline
\end{tabular}

\vspace{1cm}

\begin{tabular}{|m{0.7cm}|m{10cm}|m{4cm}|}
\hline
№ & Soraw & Juwap \\
\hline
1. & $OY$ kósheriniń teńlemesi? & $x=0$ \\
\hline
2. & Egerde $a=\{ x_1; y_1; z_1\}, b=\{ x_2, y_2; z_2\}$ bolsa, vektor kóbeymeniń koordinatalarda ańlatılıwı qanday boladı? &  $\lbrack ab\rbrack=\{y_1z_2-y_2z_1; z_1x_2-z_2x_1; x_1y_2-x_2y_1\}$ \\
\hline
3. & $A_1x+B_1y+C_1z+D_1=0$ hám $Ax_2+By_2+Cz_2+D_2=0$ tegislikleri perpendikulyar bolıwı shárti & $A_1\cdot A_2+B_1\cdot B_2+C_1\cdot C_2=0$ \\
\hline
4. & Úsh vektordıń aralas kóbeymesi ushın $(abc)=0$ teńligi orınlı bolsa ne dep ataladı? & $\overline{a}$, $\overline{b}$ hám $\overline{c}$ vektorları komplanar \\
\hline
5. & $5x-y+7=0$ hám $3x+2y=0$ tuwrıları arasındaǵı múyeshni tabıń. & $\varphi=\frac{\pi}{4}$ \\
\hline
6. & $\overline{a}=\left\{ 2, 1, 0 \right\}$ hám $\overline{b}=\left\{ 1, 0,-1 \right\}$ bolsa, $\overline{a}-\overline{b}$ ni tabıń. & $\overline{a} -\overline{b} = \left\{ 1,1,1 \right\}$ \\
\hline
7. & Koordinatalar kósherleri hám $ 3x+4y-12=0 $ tuwrı sızıǵı menen shegaralanǵan úshmúyeshliktiń maydanın tabıń. & $ S=6 $ \\
\hline
8. & $x-2y+1=0$ teńlemesi menen berilgen tuwrınıń normal túrdegi teńlemesin kórsetiń. & $\frac{x}{- \sqrt{5}}+\frac{2y}{\sqrt{5}}-\frac{1}{\sqrt{5}}=0$ \\
\hline
9. & $3x-y+5=0$, $x+3y-4=0$ tuwrı sızıqları arasındaǵı múyeshti tabıń. & $90^{0}$ \\
\hline
10. & $\overline{a}=\{5,-6, 1 \}, \overline{b}=\{-4, 3, 0 \} $, $\overline{c}=\left\{ 5,-8, 10 \right\}$ vektorları berilgen. $2{\overline{a}}^{2}+4{\overline{b}}^{2}-5{\overline{c}}^{2}$ ańlatpasınıń mánisin tabıń. & $-721$ \\
\hline
\end{tabular}

\vspace{1cm}

\begin{tabular}{lll}
Tuwrı juwaplar sanı: \underline{\hspace{1.5cm}} & 
Bahası: \underline{\hspace{1.5cm}} & 
Imtixan alıwshınıń qolı: \underline{\hspace{2cm}} \\
\end{tabular}

\egroup

\newpage


\textbf{35-variant}\\

\bgroup
\def\arraystretch{1.6} % 1 is the default, change whatever you need

\begin{tabular}{|m{5.7cm}|m{9.5cm}|}
\hline
Familiyası hám atı & \\
\hline
Fakulteti  & \\
\hline
Toparı hám tálim baǵdarı  & \\
\hline
\end{tabular}

\vspace{1cm}

\begin{tabular}{|m{0.7cm}|m{10cm}|m{4cm}|}
\hline
№ & Soraw & Juwap \\
\hline
1. & $A_1x+B_1y+C_1z+D_1=0$ hám $Ax_2+By_2+Cz_2+D_2=0$ tegislikleri ústpe-úst túsiwi shárti? & $\frac{A_1}{A_2}=\frac{B_1}{B_2}=\frac{C_1}{C_2}=\frac{D_1}{D_2}$ \\
\hline
2. & Eki vektordıń skalyar kóbeymesiniń formulası? & $(ab)=|a||b|\cos\varphi$ \\
\hline
3. & $A_1x+B_1y+C_1z+D_1=0$ hám $Ax_2+By_2+Cz_2+D_2=0$ tegislikleri parallel bolıwı shárti & $\frac{A_1}{A_2}=\frac{B_1}{B_2}=\frac{C_1}{C_2}$ \\
\hline
4. & $Ax+C=0$ tuwrı sızıqtıń grafigi koordinata kósherlerine salıstırǵanda qanday jaylasqan? & $OY$ kósherine parallel \\
\hline
5. & $(2, 3)$ hám $(4, 3)$ noqatlarınan ótiwshi tuwrı sızıqtıń teńlemesin dúziń. & $ y-3=0$ \\
\hline
6. & $x^{2}+y^{2}-2x+4y=0$ sheńberdiń teńlemesin kanonikalıq túrdegi teńlemege alıp keliń. & $(x-1)^{2}+(y+2)^{2}=5$ \\
\hline
7. & $A(4, 3), B(7, 7)$ noqatları arasındaǵı aralıqtı tabıń. & $d(AB)=5$ \\
\hline
8. & $3x^{2}+10xy+3y^{2}-2x-14y-13=0$ teńlemesiniń tipin anıqlań. & giperbola \\
\hline
9. & $x^{2}-4y^{2}+6x+5=0$ giperbolanıń kanonikalıq teńlemesin dúziń. & $\frac{(x+3)^{2}}{4}-\frac{y^{2}}{1}=1$ \\
\hline
10. & $M_{1}M_{2}$ kesindiniń ortasınıń koordinatalarınıń tabıń, eger $M_{1} (2, 3), M_{2} (4, 7)$ bolsa. & $(3,5)$ \\
\hline
\end{tabular}

\vspace{1cm}

\begin{tabular}{lll}
Tuwrı juwaplar sanı: \underline{\hspace{1.5cm}} & 
Bahası: \underline{\hspace{1.5cm}} & 
Imtixan alıwshınıń qolı: \underline{\hspace{2cm}} \\
\end{tabular}

\egroup

\newpage


\textbf{36-variant}\\

\bgroup
\def\arraystretch{1.6} % 1 is the default, change whatever you need

\begin{tabular}{|m{5.7cm}|m{9.5cm}|}
\hline
Familiyası hám atı & \\
\hline
Fakulteti  & \\
\hline
Toparı hám tálim baǵdarı  & \\
\hline
\end{tabular}

\vspace{1cm}

\begin{tabular}{|m{0.7cm}|m{10cm}|m{4cm}|}
\hline
№ & Soraw & Juwap \\
\hline
1. & Eki vektor qashan kollinear dep ataladı? & bir tuwrıda yamasa parallel tuwrıda jaylasqan bolsa \\
\hline
2. & Tuwrı múyeshli koordinatalar sisteması dep nege aytamız? & Masshtab birlikleri berilgen o'zara perpendikulyar $OX$ hám $OY$ kósherleri \\
\hline
3. & $OXY$ tegisliginiń teńlemesi? & $z=0$ \\
\hline
4. & Giperbolanıń kanonikalıq teńlemesi? & $\frac{x^2}{a^2}-\frac{y^2}{b^2}=1$ \\
\hline
5. & $x+y-3=0$ hám $2x+3y-8=0$ tuwrıları óz-ara qanday jaylasqan? & kesilisedi \\
\hline
6. & $x^{2}+y^{2}-2x+4y-20=0$ sheńberdiń $C$ orayın hám $R$ radiusın tabıń. & $C(1;-2), R=5$ \\
\hline
7. & $(x+1)^{2}+(y-2) ^{2}+(z+2) ^{2}=49$ sferanıń orayınıń koordinataların tabıń. & $(-1,2,-2)$ \\
\hline
8. & Eger $2a=16, e=\frac{5}{4}$ bolsa, fokusı abscissa kósherinde, koordinata basına salıstırǵanda simmetriyalıq jaylasqan giperbolanıń teńlemesin dúziń. & $\frac{x^{2}}{64}-\frac{y^{2}}{36}=1$ \\
\hline
9. & Eger $2b=24, 2 c=10$ bolsa, onda abscissa kósherinde koordinata basına salıstırǵanda simmetriyalıq jaylasqan fokuslarǵa iye, ellipstiń teńlemesin dúziń. & $\frac{x^{2}}{169}+\frac{y^{2}}{144}=1$ \\
\hline
10. & $M_{1} (12;-1)$ hám $M_{2} (0;4)$ noqatlardıń arasındaǵı aralıqtı tabıń. & $13$ \\
\hline
\end{tabular}

\vspace{1cm}

\begin{tabular}{lll}
Tuwrı juwaplar sanı: \underline{\hspace{1.5cm}} & 
Bahası: \underline{\hspace{1.5cm}} & 
Imtixan alıwshınıń qolı: \underline{\hspace{2cm}} \\
\end{tabular}

\egroup

\newpage


\textbf{37-variant}\\

\bgroup
\def\arraystretch{1.6} % 1 is the default, change whatever you need

\begin{tabular}{|m{5.7cm}|m{9.5cm}|}
\hline
Familiyası hám atı & \\
\hline
Fakulteti  & \\
\hline
Toparı hám tálim baǵdarı  & \\
\hline
\end{tabular}

\vspace{1cm}

\begin{tabular}{|m{0.7cm}|m{10cm}|m{4cm}|}
\hline
№ & Soraw & Juwap \\
\hline
1. & Eki vektordıń vektor kóbeymesiniń uzınlıǵın tabıw formulası? & $\left| \lbrack ab\rbrack \right|=|a||b|\sin\varphi$ \\
\hline
2. & Tegislikdegi qálegen noqatınan berilgen eki noqatqa shekemgi bolǵan aralıqlardıń ayırmasınıń modulı ózgermeytuǵın bolǵan noqatlardıń geometriyalıq ornı ne dep ataladı? & giperbola \\
\hline
3. & Eki tuwrı sızıq arasındaǵı múyeshti tabıw formulası? & $\text{tg}\varphi=\frac{k_2-k_1}{1+k_1k_2}$ \\
\hline
4. & $\frac{x^2}{a^2}-\frac{y^2}{b^2}=1$ giperbolanıń $(x_0;y_0)$ noqatındaǵı urınbasınıń teńlemesin kórsetiń. & $\frac{x_0x}{a^2}-\frac{y_0y}{b^2}=1$ \\
\hline
5. & $x+y=0$ teńlemesi menen berilgen tuwrı sızıqtıń múyeshlik koefficientin anıqlań. & $- 1$ \\
\hline
6. & Orayı $C (-1;2)$ noqatında, $A (-2;6 )$ noqatınan ótetuǵın sheńberdiń teńlemesin dúziń. & $(x+1)^{2}+(y-2)^{2}=17$ \\
\hline
7. & $x+2=0$ keńislik qanday geometriyalıq betlikti anıqlaydı? &  $OYZ$ tegisligine parallel bolǵan tegislikti \\
\hline
8. & $\frac{x^{2}}{225}-\frac{y^{2}}{64}=-1$ giperbola fokusınıń koordinatalarınıń tabıń. & $F_{1}(0;-17), F_{2}(0;17)$ \\
\hline
9. & $9x^{2}+25y^{2}=225$ ellipsi berilgen, ellipstiń fokusların, ekscentrisitetin tabıń. & $F_1\left(-4;0 \right) , F_2\left( 4;0 \right) , e = \frac{4}{5}$ \\
\hline
10. & $A (-1;0;1),\ B (1;-1;0)$ noqatları berilgen. $\overline{BA}$ vektorın tabıń. & $\left\{ - 2;1;1 \right\}$ \\
\hline
\end{tabular}

\vspace{1cm}

\begin{tabular}{lll}
Tuwrı juwaplar sanı: \underline{\hspace{1.5cm}} & 
Bahası: \underline{\hspace{1.5cm}} & 
Imtixan alıwshınıń qolı: \underline{\hspace{2cm}} \\
\end{tabular}

\egroup

\newpage


\textbf{38-variant}\\

\bgroup
\def\arraystretch{1.6} % 1 is the default, change whatever you need

\begin{tabular}{|m{5.7cm}|m{9.5cm}|}
\hline
Familiyası hám atı & \\
\hline
Fakulteti  & \\
\hline
Toparı hám tálim baǵdarı  & \\
\hline
\end{tabular}

\vspace{1cm}

\begin{tabular}{|m{0.7cm}|m{10cm}|m{4cm}|}
\hline
№ & Soraw & Juwap \\
\hline
1. & Vektorlardıń kósherdegi proekciyasınıń formulası? & $x=|a|\cos\varphi, y=|a|\sin\varphi$ \\
\hline
2. & $Ax+By+D=0$ teńlemesi arqalı ... tegisliktiń teńlemesi berilgen? & $OZ$ kósherine parallel \\
\hline
3. & $\frac{x^2}{a^2}+\frac{y^2}{b^2}=1$ ellipstiń $(x_0;y_0)$ noqatındaǵı urınbasınıń teńlemesin tabıń. & $\frac{x_0x}{a^2}+\frac{y_0y}{b^2}=1$ \\
\hline
4. & Vektorlardı qosıw koordinatalarda qanday formula menen anıqlanadı? & $\overline{a}+\overline{b}=\{x_1+x_2;y_1+y_2\}$ \\
\hline
5. & $2x+3y+4=0$ tuwrısına parallel hám $M_{0} (2;1)$ noqattan ótetuǵın tuwrınıń teńlemesin dúziń. & $2x+3y-7=0$ \\
\hline
6. & $x+y-12=0$ tuwrısı $x^{2}+y^{2}-2y=0$ sheńberge salıstırǵanda qanday jaylasqan? & sırtında jaylasqan \\
\hline
7. & $\left| \overline{a} \right|=8, \left| \overline{b} \right|=5, \alpha=60^{0}$ bolsa, $( \overline{a}\overline{b} )$ ni tabıń. & $20$ \\
\hline
8. & $2x+3y-6=0$ tuwrınıń teńlemesin kesindilerde berilgen teńleme túrinde kórsetiń. & $\frac{x}{3} + \frac{ y }{ 2 } =  1$ \\
\hline
9. & $\overline{a}=\left\{ 4,-2,-4 \right\}$ hám $\overline{b}=\left\{ 6,-3, 2 \right\}$ vektorları berilgen, $(\overline{a}-\overline{b}) ^{2}$-? & $41$ \\
\hline
10. & $5x-y+7=0$ hám $3x+2y=0$ tuwrıları arasındaǵı múyeshni tabıń. & $\varphi=\frac{\pi}{4}$ \\
\hline
\end{tabular}

\vspace{1cm}

\begin{tabular}{lll}
Tuwrı juwaplar sanı: \underline{\hspace{1.5cm}} & 
Bahası: \underline{\hspace{1.5cm}} & 
Imtixan alıwshınıń qolı: \underline{\hspace{2cm}} \\
\end{tabular}

\egroup

\newpage


\textbf{39-variant}\\

\bgroup
\def\arraystretch{1.6} % 1 is the default, change whatever you need

\begin{tabular}{|m{5.7cm}|m{9.5cm}|}
\hline
Familiyası hám atı & \\
\hline
Fakulteti  & \\
\hline
Toparı hám tálim baǵdarı  & \\
\hline
\end{tabular}

\vspace{1cm}

\begin{tabular}{|m{0.7cm}|m{10cm}|m{4cm}|}
\hline
№ & Soraw & Juwap \\
\hline
1. & $OY$ kósheriniń teńlemesi? & $x=0$ \\
\hline
2. & Egerde $a=\{ x_1; y_1; z_1\}, b=\{ x_2, y_2; z_2\}$ bolsa, vektor kóbeymeniń koordinatalarda ańlatılıwı qanday boladı? &  $\lbrack ab\rbrack=\{y_1z_2-y_2z_1; z_1x_2-z_2x_1; x_1y_2-x_2y_1\}$ \\
\hline
3. & $A_1x+B_1y+C_1z+D_1=0$ hám $Ax_2+By_2+Cz_2+D_2=0$ tegislikleri perpendikulyar bolıwı shárti & $A_1\cdot A_2+B_1\cdot B_2+C_1\cdot C_2=0$ \\
\hline
4. & Úsh vektordıń aralas kóbeymesi ushın $(abc)=0$ teńligi orınlı bolsa ne dep ataladı? & $\overline{a}$, $\overline{b}$ hám $\overline{c}$ vektorları komplanar \\
\hline
5. & $\overline{a}=\left\{ 2, 1, 0 \right\}$ hám $\overline{b}=\left\{ 1, 0,-1 \right\}$ bolsa, $\overline{a}-\overline{b}$ ni tabıń. & $\overline{a} -\overline{b} = \left\{ 1,1,1 \right\}$ \\
\hline
6. & Koordinatalar kósherleri hám $ 3x+4y-12=0 $ tuwrı sızıǵı menen shegaralanǵan úshmúyeshliktiń maydanın tabıń. & $ S=6 $ \\
\hline
7. & $x-2y+1=0$ teńlemesi menen berilgen tuwrınıń normal túrdegi teńlemesin kórsetiń. & $\frac{x}{- \sqrt{5}}+\frac{2y}{\sqrt{5}}-\frac{1}{\sqrt{5}}=0$ \\
\hline
8. & $3x-y+5=0$, $x+3y-4=0$ tuwrı sızıqları arasındaǵı múyeshti tabıń. & $90^{0}$ \\
\hline
9. & $\overline{a}=\{5,-6, 1 \}, \overline{b}=\{-4, 3, 0 \} $, $\overline{c}=\left\{ 5,-8, 10 \right\}$ vektorları berilgen. $2{\overline{a}}^{2}+4{\overline{b}}^{2}-5{\overline{c}}^{2}$ ańlatpasınıń mánisin tabıń. & $-721$ \\
\hline
10. & $(2, 3)$ hám $(4, 3)$ noqatlarınan ótiwshi tuwrı sızıqtıń teńlemesin dúziń. & $ y-3=0$ \\
\hline
\end{tabular}

\vspace{1cm}

\begin{tabular}{lll}
Tuwrı juwaplar sanı: \underline{\hspace{1.5cm}} & 
Bahası: \underline{\hspace{1.5cm}} & 
Imtixan alıwshınıń qolı: \underline{\hspace{2cm}} \\
\end{tabular}

\egroup

\newpage


\textbf{40-variant}\\

\bgroup
\def\arraystretch{1.6} % 1 is the default, change whatever you need

\begin{tabular}{|m{5.7cm}|m{9.5cm}|}
\hline
Familiyası hám atı & \\
\hline
Fakulteti  & \\
\hline
Toparı hám tálim baǵdarı  & \\
\hline
\end{tabular}

\vspace{1cm}

\begin{tabular}{|m{0.7cm}|m{10cm}|m{4cm}|}
\hline
№ & Soraw & Juwap \\
\hline
1. & $A_1x+B_1y+C_1z+D_1=0$ hám $Ax_2+By_2+Cz_2+D_2=0$ tegislikleri ústpe-úst túsiwi shárti? & $\frac{A_1}{A_2}=\frac{B_1}{B_2}=\frac{C_1}{C_2}=\frac{D_1}{D_2}$ \\
\hline
2. & Eki vektordıń skalyar kóbeymesiniń formulası? & $(ab)=|a||b|\cos\varphi$ \\
\hline
3. & $A_1x+B_1y+C_1z+D_1=0$ hám $Ax_2+By_2+Cz_2+D_2=0$ tegislikleri parallel bolıwı shárti & $\frac{A_1}{A_2}=\frac{B_1}{B_2}=\frac{C_1}{C_2}$ \\
\hline
4. & $Ax+C=0$ tuwrı sızıqtıń grafigi koordinata kósherlerine salıstırǵanda qanday jaylasqan? & $OY$ kósherine parallel \\
\hline
5. & $x^{2}+y^{2}-2x+4y=0$ sheńberdiń teńlemesin kanonikalıq túrdegi teńlemege alıp keliń. & $(x-1)^{2}+(y+2)^{2}=5$ \\
\hline
6. & $A(4, 3), B(7, 7)$ noqatları arasındaǵı aralıqtı tabıń. & $d(AB)=5$ \\
\hline
7. & $3x^{2}+10xy+3y^{2}-2x-14y-13=0$ teńlemesiniń tipin anıqlań. & giperbola \\
\hline
8. & $x^{2}-4y^{2}+6x+5=0$ giperbolanıń kanonikalıq teńlemesin dúziń. & $\frac{(x+3)^{2}}{4}-\frac{y^{2}}{1}=1$ \\
\hline
9. & $M_{1}M_{2}$ kesindiniń ortasınıń koordinatalarınıń tabıń, eger $M_{1} (2, 3), M_{2} (4, 7)$ bolsa. & $(3,5)$ \\
\hline
10. & $x+y-3=0$ hám $2x+3y-8=0$ tuwrıları óz-ara qanday jaylasqan? & kesilisedi \\
\hline
\end{tabular}

\vspace{1cm}

\begin{tabular}{lll}
Tuwrı juwaplar sanı: \underline{\hspace{1.5cm}} & 
Bahası: \underline{\hspace{1.5cm}} & 
Imtixan alıwshınıń qolı: \underline{\hspace{2cm}} \\
\end{tabular}

\egroup

\newpage


\textbf{41-variant}\\

\bgroup
\def\arraystretch{1.6} % 1 is the default, change whatever you need

\begin{tabular}{|m{5.7cm}|m{9.5cm}|}
\hline
Familiyası hám atı & \\
\hline
Fakulteti  & \\
\hline
Toparı hám tálim baǵdarı  & \\
\hline
\end{tabular}

\vspace{1cm}

\begin{tabular}{|m{0.7cm}|m{10cm}|m{4cm}|}
\hline
№ & Soraw & Juwap \\
\hline
1. & Eki vektor qashan kollinear dep ataladı? & bir tuwrıda yamasa parallel tuwrıda jaylasqan bolsa \\
\hline
2. & Tuwrı múyeshli koordinatalar sisteması dep nege aytamız? & Masshtab birlikleri berilgen o'zara perpendikulyar $OX$ hám $OY$ kósherleri \\
\hline
3. & $OXY$ tegisliginiń teńlemesi? & $z=0$ \\
\hline
4. & Giperbolanıń kanonikalıq teńlemesi? & $\frac{x^2}{a^2}-\frac{y^2}{b^2}=1$ \\
\hline
5. & $x^{2}+y^{2}-2x+4y-20=0$ sheńberdiń $C$ orayın hám $R$ radiusın tabıń. & $C(1;-2), R=5$ \\
\hline
6. & $(x+1)^{2}+(y-2) ^{2}+(z+2) ^{2}=49$ sferanıń orayınıń koordinataların tabıń. & $(-1,2,-2)$ \\
\hline
7. & Eger $2a=16, e=\frac{5}{4}$ bolsa, fokusı abscissa kósherinde, koordinata basına salıstırǵanda simmetriyalıq jaylasqan giperbolanıń teńlemesin dúziń. & $\frac{x^{2}}{64}-\frac{y^{2}}{36}=1$ \\
\hline
8. & Eger $2b=24, 2 c=10$ bolsa, onda abscissa kósherinde koordinata basına salıstırǵanda simmetriyalıq jaylasqan fokuslarǵa iye, ellipstiń teńlemesin dúziń. & $\frac{x^{2}}{169}+\frac{y^{2}}{144}=1$ \\
\hline
9. & $M_{1} (12;-1)$ hám $M_{2} (0;4)$ noqatlardıń arasındaǵı aralıqtı tabıń. & $13$ \\
\hline
10. & $x+y=0$ teńlemesi menen berilgen tuwrı sızıqtıń múyeshlik koefficientin anıqlań. & $- 1$ \\
\hline
\end{tabular}

\vspace{1cm}

\begin{tabular}{lll}
Tuwrı juwaplar sanı: \underline{\hspace{1.5cm}} & 
Bahası: \underline{\hspace{1.5cm}} & 
Imtixan alıwshınıń qolı: \underline{\hspace{2cm}} \\
\end{tabular}

\egroup

\newpage


\textbf{42-variant}\\

\bgroup
\def\arraystretch{1.6} % 1 is the default, change whatever you need

\begin{tabular}{|m{5.7cm}|m{9.5cm}|}
\hline
Familiyası hám atı & \\
\hline
Fakulteti  & \\
\hline
Toparı hám tálim baǵdarı  & \\
\hline
\end{tabular}

\vspace{1cm}

\begin{tabular}{|m{0.7cm}|m{10cm}|m{4cm}|}
\hline
№ & Soraw & Juwap \\
\hline
1. & Eki vektordıń vektor kóbeymesiniń uzınlıǵın tabıw formulası? & $\left| \lbrack ab\rbrack \right|=|a||b|\sin\varphi$ \\
\hline
2. & Tegislikdegi qálegen noqatınan berilgen eki noqatqa shekemgi bolǵan aralıqlardıń ayırmasınıń modulı ózgermeytuǵın bolǵan noqatlardıń geometriyalıq ornı ne dep ataladı? & giperbola \\
\hline
3. & Eki tuwrı sızıq arasındaǵı múyeshti tabıw formulası? & $\text{tg}\varphi=\frac{k_2-k_1}{1+k_1k_2}$ \\
\hline
4. & $\frac{x^2}{a^2}-\frac{y^2}{b^2}=1$ giperbolanıń $(x_0;y_0)$ noqatındaǵı urınbasınıń teńlemesin kórsetiń. & $\frac{x_0x}{a^2}-\frac{y_0y}{b^2}=1$ \\
\hline
5. & Orayı $C (-1;2)$ noqatında, $A (-2;6 )$ noqatınan ótetuǵın sheńberdiń teńlemesin dúziń. & $(x+1)^{2}+(y-2)^{2}=17$ \\
\hline
6. & $x+2=0$ keńislik qanday geometriyalıq betlikti anıqlaydı? &  $OYZ$ tegisligine parallel bolǵan tegislikti \\
\hline
7. & $\frac{x^{2}}{225}-\frac{y^{2}}{64}=-1$ giperbola fokusınıń koordinatalarınıń tabıń. & $F_{1}(0;-17), F_{2}(0;17)$ \\
\hline
8. & $9x^{2}+25y^{2}=225$ ellipsi berilgen, ellipstiń fokusların, ekscentrisitetin tabıń. & $F_1\left(-4;0 \right) , F_2\left( 4;0 \right) , e = \frac{4}{5}$ \\
\hline
9. & $A (-1;0;1),\ B (1;-1;0)$ noqatları berilgen. $\overline{BA}$ vektorın tabıń. & $\left\{ - 2;1;1 \right\}$ \\
\hline
10. & $2x+3y+4=0$ tuwrısına parallel hám $M_{0} (2;1)$ noqattan ótetuǵın tuwrınıń teńlemesin dúziń. & $2x+3y-7=0$ \\
\hline
\end{tabular}

\vspace{1cm}

\begin{tabular}{lll}
Tuwrı juwaplar sanı: \underline{\hspace{1.5cm}} & 
Bahası: \underline{\hspace{1.5cm}} & 
Imtixan alıwshınıń qolı: \underline{\hspace{2cm}} \\
\end{tabular}

\egroup

\newpage


\textbf{43-variant}\\

\bgroup
\def\arraystretch{1.6} % 1 is the default, change whatever you need

\begin{tabular}{|m{5.7cm}|m{9.5cm}|}
\hline
Familiyası hám atı & \\
\hline
Fakulteti  & \\
\hline
Toparı hám tálim baǵdarı  & \\
\hline
\end{tabular}

\vspace{1cm}

\begin{tabular}{|m{0.7cm}|m{10cm}|m{4cm}|}
\hline
№ & Soraw & Juwap \\
\hline
1. & Vektorlardıń kósherdegi proekciyasınıń formulası? & $x=|a|\cos\varphi, y=|a|\sin\varphi$ \\
\hline
2. & $Ax+By+D=0$ teńlemesi arqalı ... tegisliktiń teńlemesi berilgen? & $OZ$ kósherine parallel \\
\hline
3. & $\frac{x^2}{a^2}+\frac{y^2}{b^2}=1$ ellipstiń $(x_0;y_0)$ noqatındaǵı urınbasınıń teńlemesin tabıń. & $\frac{x_0x}{a^2}+\frac{y_0y}{b^2}=1$ \\
\hline
4. & Vektorlardı qosıw koordinatalarda qanday formula menen anıqlanadı? & $\overline{a}+\overline{b}=\{x_1+x_2;y_1+y_2\}$ \\
\hline
5. & $x+y-12=0$ tuwrısı $x^{2}+y^{2}-2y=0$ sheńberge salıstırǵanda qanday jaylasqan? & sırtında jaylasqan \\
\hline
6. & $\left| \overline{a} \right|=8, \left| \overline{b} \right|=5, \alpha=60^{0}$ bolsa, $( \overline{a}\overline{b} )$ ni tabıń. & $20$ \\
\hline
7. & $2x+3y-6=0$ tuwrınıń teńlemesin kesindilerde berilgen teńleme túrinde kórsetiń. & $\frac{x}{3} + \frac{ y }{ 2 } =  1$ \\
\hline
8. & $\overline{a}=\left\{ 4,-2,-4 \right\}$ hám $\overline{b}=\left\{ 6,-3, 2 \right\}$ vektorları berilgen, $(\overline{a}-\overline{b}) ^{2}$-? & $41$ \\
\hline
9. & $5x-y+7=0$ hám $3x+2y=0$ tuwrıları arasındaǵı múyeshni tabıń. & $\varphi=\frac{\pi}{4}$ \\
\hline
10. & $\overline{a}=\left\{ 2, 1, 0 \right\}$ hám $\overline{b}=\left\{ 1, 0,-1 \right\}$ bolsa, $\overline{a}-\overline{b}$ ni tabıń. & $\overline{a} -\overline{b} = \left\{ 1,1,1 \right\}$ \\
\hline
\end{tabular}

\vspace{1cm}

\begin{tabular}{lll}
Tuwrı juwaplar sanı: \underline{\hspace{1.5cm}} & 
Bahası: \underline{\hspace{1.5cm}} & 
Imtixan alıwshınıń qolı: \underline{\hspace{2cm}} \\
\end{tabular}

\egroup

\newpage


\textbf{44-variant}\\

\bgroup
\def\arraystretch{1.6} % 1 is the default, change whatever you need

\begin{tabular}{|m{5.7cm}|m{9.5cm}|}
\hline
Familiyası hám atı & \\
\hline
Fakulteti  & \\
\hline
Toparı hám tálim baǵdarı  & \\
\hline
\end{tabular}

\vspace{1cm}

\begin{tabular}{|m{0.7cm}|m{10cm}|m{4cm}|}
\hline
№ & Soraw & Juwap \\
\hline
1. & $OY$ kósheriniń teńlemesi? & $x=0$ \\
\hline
2. & Egerde $a=\{ x_1; y_1; z_1\}, b=\{ x_2, y_2; z_2\}$ bolsa, vektor kóbeymeniń koordinatalarda ańlatılıwı qanday boladı? &  $\lbrack ab\rbrack=\{y_1z_2-y_2z_1; z_1x_2-z_2x_1; x_1y_2-x_2y_1\}$ \\
\hline
3. & $A_1x+B_1y+C_1z+D_1=0$ hám $Ax_2+By_2+Cz_2+D_2=0$ tegislikleri perpendikulyar bolıwı shárti & $A_1\cdot A_2+B_1\cdot B_2+C_1\cdot C_2=0$ \\
\hline
4. & Úsh vektordıń aralas kóbeymesi ushın $(abc)=0$ teńligi orınlı bolsa ne dep ataladı? & $\overline{a}$, $\overline{b}$ hám $\overline{c}$ vektorları komplanar \\
\hline
5. & Koordinatalar kósherleri hám $ 3x+4y-12=0 $ tuwrı sızıǵı menen shegaralanǵan úshmúyeshliktiń maydanın tabıń. & $ S=6 $ \\
\hline
6. & $x-2y+1=0$ teńlemesi menen berilgen tuwrınıń normal túrdegi teńlemesin kórsetiń. & $\frac{x}{- \sqrt{5}}+\frac{2y}{\sqrt{5}}-\frac{1}{\sqrt{5}}=0$ \\
\hline
7. & $3x-y+5=0$, $x+3y-4=0$ tuwrı sızıqları arasındaǵı múyeshti tabıń. & $90^{0}$ \\
\hline
8. & $\overline{a}=\{5,-6, 1 \}, \overline{b}=\{-4, 3, 0 \} $, $\overline{c}=\left\{ 5,-8, 10 \right\}$ vektorları berilgen. $2{\overline{a}}^{2}+4{\overline{b}}^{2}-5{\overline{c}}^{2}$ ańlatpasınıń mánisin tabıń. & $-721$ \\
\hline
9. & $(2, 3)$ hám $(4, 3)$ noqatlarınan ótiwshi tuwrı sızıqtıń teńlemesin dúziń. & $ y-3=0$ \\
\hline
10. & $x^{2}+y^{2}-2x+4y=0$ sheńberdiń teńlemesin kanonikalıq túrdegi teńlemege alıp keliń. & $(x-1)^{2}+(y+2)^{2}=5$ \\
\hline
\end{tabular}

\vspace{1cm}

\begin{tabular}{lll}
Tuwrı juwaplar sanı: \underline{\hspace{1.5cm}} & 
Bahası: \underline{\hspace{1.5cm}} & 
Imtixan alıwshınıń qolı: \underline{\hspace{2cm}} \\
\end{tabular}

\egroup

\newpage


\textbf{45-variant}\\

\bgroup
\def\arraystretch{1.6} % 1 is the default, change whatever you need

\begin{tabular}{|m{5.7cm}|m{9.5cm}|}
\hline
Familiyası hám atı & \\
\hline
Fakulteti  & \\
\hline
Toparı hám tálim baǵdarı  & \\
\hline
\end{tabular}

\vspace{1cm}

\begin{tabular}{|m{0.7cm}|m{10cm}|m{4cm}|}
\hline
№ & Soraw & Juwap \\
\hline
1. & $A_1x+B_1y+C_1z+D_1=0$ hám $Ax_2+By_2+Cz_2+D_2=0$ tegislikleri ústpe-úst túsiwi shárti? & $\frac{A_1}{A_2}=\frac{B_1}{B_2}=\frac{C_1}{C_2}=\frac{D_1}{D_2}$ \\
\hline
2. & Eki vektordıń skalyar kóbeymesiniń formulası? & $(ab)=|a||b|\cos\varphi$ \\
\hline
3. & $A_1x+B_1y+C_1z+D_1=0$ hám $Ax_2+By_2+Cz_2+D_2=0$ tegislikleri parallel bolıwı shárti & $\frac{A_1}{A_2}=\frac{B_1}{B_2}=\frac{C_1}{C_2}$ \\
\hline
4. & $Ax+C=0$ tuwrı sızıqtıń grafigi koordinata kósherlerine salıstırǵanda qanday jaylasqan? & $OY$ kósherine parallel \\
\hline
5. & $A(4, 3), B(7, 7)$ noqatları arasındaǵı aralıqtı tabıń. & $d(AB)=5$ \\
\hline
6. & $3x^{2}+10xy+3y^{2}-2x-14y-13=0$ teńlemesiniń tipin anıqlań. & giperbola \\
\hline
7. & $x^{2}-4y^{2}+6x+5=0$ giperbolanıń kanonikalıq teńlemesin dúziń. & $\frac{(x+3)^{2}}{4}-\frac{y^{2}}{1}=1$ \\
\hline
8. & $M_{1}M_{2}$ kesindiniń ortasınıń koordinatalarınıń tabıń, eger $M_{1} (2, 3), M_{2} (4, 7)$ bolsa. & $(3,5)$ \\
\hline
9. & $x+y-3=0$ hám $2x+3y-8=0$ tuwrıları óz-ara qanday jaylasqan? & kesilisedi \\
\hline
10. & $x^{2}+y^{2}-2x+4y-20=0$ sheńberdiń $C$ orayın hám $R$ radiusın tabıń. & $C(1;-2), R=5$ \\
\hline
\end{tabular}

\vspace{1cm}

\begin{tabular}{lll}
Tuwrı juwaplar sanı: \underline{\hspace{1.5cm}} & 
Bahası: \underline{\hspace{1.5cm}} & 
Imtixan alıwshınıń qolı: \underline{\hspace{2cm}} \\
\end{tabular}

\egroup

\newpage


\textbf{46-variant}\\

\bgroup
\def\arraystretch{1.6} % 1 is the default, change whatever you need

\begin{tabular}{|m{5.7cm}|m{9.5cm}|}
\hline
Familiyası hám atı & \\
\hline
Fakulteti  & \\
\hline
Toparı hám tálim baǵdarı  & \\
\hline
\end{tabular}

\vspace{1cm}

\begin{tabular}{|m{0.7cm}|m{10cm}|m{4cm}|}
\hline
№ & Soraw & Juwap \\
\hline
1. & Eki vektor qashan kollinear dep ataladı? & bir tuwrıda yamasa parallel tuwrıda jaylasqan bolsa \\
\hline
2. & Tuwrı múyeshli koordinatalar sisteması dep nege aytamız? & Masshtab birlikleri berilgen o'zara perpendikulyar $OX$ hám $OY$ kósherleri \\
\hline
3. & $OXY$ tegisliginiń teńlemesi? & $z=0$ \\
\hline
4. & Giperbolanıń kanonikalıq teńlemesi? & $\frac{x^2}{a^2}-\frac{y^2}{b^2}=1$ \\
\hline
5. & $(x+1)^{2}+(y-2) ^{2}+(z+2) ^{2}=49$ sferanıń orayınıń koordinataların tabıń. & $(-1,2,-2)$ \\
\hline
6. & Eger $2a=16, e=\frac{5}{4}$ bolsa, fokusı abscissa kósherinde, koordinata basına salıstırǵanda simmetriyalıq jaylasqan giperbolanıń teńlemesin dúziń. & $\frac{x^{2}}{64}-\frac{y^{2}}{36}=1$ \\
\hline
7. & Eger $2b=24, 2 c=10$ bolsa, onda abscissa kósherinde koordinata basına salıstırǵanda simmetriyalıq jaylasqan fokuslarǵa iye, ellipstiń teńlemesin dúziń. & $\frac{x^{2}}{169}+\frac{y^{2}}{144}=1$ \\
\hline
8. & $M_{1} (12;-1)$ hám $M_{2} (0;4)$ noqatlardıń arasındaǵı aralıqtı tabıń. & $13$ \\
\hline
9. & $x+y=0$ teńlemesi menen berilgen tuwrı sızıqtıń múyeshlik koefficientin anıqlań. & $- 1$ \\
\hline
10. & Orayı $C (-1;2)$ noqatında, $A (-2;6 )$ noqatınan ótetuǵın sheńberdiń teńlemesin dúziń. & $(x+1)^{2}+(y-2)^{2}=17$ \\
\hline
\end{tabular}

\vspace{1cm}

\begin{tabular}{lll}
Tuwrı juwaplar sanı: \underline{\hspace{1.5cm}} & 
Bahası: \underline{\hspace{1.5cm}} & 
Imtixan alıwshınıń qolı: \underline{\hspace{2cm}} \\
\end{tabular}

\egroup

\newpage


\textbf{47-variant}\\

\bgroup
\def\arraystretch{1.6} % 1 is the default, change whatever you need

\begin{tabular}{|m{5.7cm}|m{9.5cm}|}
\hline
Familiyası hám atı & \\
\hline
Fakulteti  & \\
\hline
Toparı hám tálim baǵdarı  & \\
\hline
\end{tabular}

\vspace{1cm}

\begin{tabular}{|m{0.7cm}|m{10cm}|m{4cm}|}
\hline
№ & Soraw & Juwap \\
\hline
1. & Eki vektordıń vektor kóbeymesiniń uzınlıǵın tabıw formulası? & $\left| \lbrack ab\rbrack \right|=|a||b|\sin\varphi$ \\
\hline
2. & Tegislikdegi qálegen noqatınan berilgen eki noqatqa shekemgi bolǵan aralıqlardıń ayırmasınıń modulı ózgermeytuǵın bolǵan noqatlardıń geometriyalıq ornı ne dep ataladı? & giperbola \\
\hline
3. & Eki tuwrı sızıq arasındaǵı múyeshti tabıw formulası? & $\text{tg}\varphi=\frac{k_2-k_1}{1+k_1k_2}$ \\
\hline
4. & $\frac{x^2}{a^2}-\frac{y^2}{b^2}=1$ giperbolanıń $(x_0;y_0)$ noqatındaǵı urınbasınıń teńlemesin kórsetiń. & $\frac{x_0x}{a^2}-\frac{y_0y}{b^2}=1$ \\
\hline
5. & $x+2=0$ keńislik qanday geometriyalıq betlikti anıqlaydı? &  $OYZ$ tegisligine parallel bolǵan tegislikti \\
\hline
6. & $\frac{x^{2}}{225}-\frac{y^{2}}{64}=-1$ giperbola fokusınıń koordinatalarınıń tabıń. & $F_{1}(0;-17), F_{2}(0;17)$ \\
\hline
7. & $9x^{2}+25y^{2}=225$ ellipsi berilgen, ellipstiń fokusların, ekscentrisitetin tabıń. & $F_1\left(-4;0 \right) , F_2\left( 4;0 \right) , e = \frac{4}{5}$ \\
\hline
8. & $A (-1;0;1),\ B (1;-1;0)$ noqatları berilgen. $\overline{BA}$ vektorın tabıń. & $\left\{ - 2;1;1 \right\}$ \\
\hline
9. & $2x+3y+4=0$ tuwrısına parallel hám $M_{0} (2;1)$ noqattan ótetuǵın tuwrınıń teńlemesin dúziń. & $2x+3y-7=0$ \\
\hline
10. & $x+y-12=0$ tuwrısı $x^{2}+y^{2}-2y=0$ sheńberge salıstırǵanda qanday jaylasqan? & sırtında jaylasqan \\
\hline
\end{tabular}

\vspace{1cm}

\begin{tabular}{lll}
Tuwrı juwaplar sanı: \underline{\hspace{1.5cm}} & 
Bahası: \underline{\hspace{1.5cm}} & 
Imtixan alıwshınıń qolı: \underline{\hspace{2cm}} \\
\end{tabular}

\egroup

\newpage


\textbf{48-variant}\\

\bgroup
\def\arraystretch{1.6} % 1 is the default, change whatever you need

\begin{tabular}{|m{5.7cm}|m{9.5cm}|}
\hline
Familiyası hám atı & \\
\hline
Fakulteti  & \\
\hline
Toparı hám tálim baǵdarı  & \\
\hline
\end{tabular}

\vspace{1cm}

\begin{tabular}{|m{0.7cm}|m{10cm}|m{4cm}|}
\hline
№ & Soraw & Juwap \\
\hline
1. & Vektorlardıń kósherdegi proekciyasınıń formulası? & $x=|a|\cos\varphi, y=|a|\sin\varphi$ \\
\hline
2. & $Ax+By+D=0$ teńlemesi arqalı ... tegisliktiń teńlemesi berilgen? & $OZ$ kósherine parallel \\
\hline
3. & $\frac{x^2}{a^2}+\frac{y^2}{b^2}=1$ ellipstiń $(x_0;y_0)$ noqatındaǵı urınbasınıń teńlemesin tabıń. & $\frac{x_0x}{a^2}+\frac{y_0y}{b^2}=1$ \\
\hline
4. & Vektorlardı qosıw koordinatalarda qanday formula menen anıqlanadı? & $\overline{a}+\overline{b}=\{x_1+x_2;y_1+y_2\}$ \\
\hline
5. & $\left| \overline{a} \right|=8, \left| \overline{b} \right|=5, \alpha=60^{0}$ bolsa, $( \overline{a}\overline{b} )$ ni tabıń. & $20$ \\
\hline
6. & $2x+3y-6=0$ tuwrınıń teńlemesin kesindilerde berilgen teńleme túrinde kórsetiń. & $\frac{x}{3} + \frac{ y }{ 2 } =  1$ \\
\hline
7. & $\overline{a}=\left\{ 4,-2,-4 \right\}$ hám $\overline{b}=\left\{ 6,-3, 2 \right\}$ vektorları berilgen, $(\overline{a}-\overline{b}) ^{2}$-? & $41$ \\
\hline
8. & $5x-y+7=0$ hám $3x+2y=0$ tuwrıları arasındaǵı múyeshni tabıń. & $\varphi=\frac{\pi}{4}$ \\
\hline
9. & $\overline{a}=\left\{ 2, 1, 0 \right\}$ hám $\overline{b}=\left\{ 1, 0,-1 \right\}$ bolsa, $\overline{a}-\overline{b}$ ni tabıń. & $\overline{a} -\overline{b} = \left\{ 1,1,1 \right\}$ \\
\hline
10. & Koordinatalar kósherleri hám $ 3x+4y-12=0 $ tuwrı sızıǵı menen shegaralanǵan úshmúyeshliktiń maydanın tabıń. & $ S=6 $ \\
\hline
\end{tabular}

\vspace{1cm}

\begin{tabular}{lll}
Tuwrı juwaplar sanı: \underline{\hspace{1.5cm}} & 
Bahası: \underline{\hspace{1.5cm}} & 
Imtixan alıwshınıń qolı: \underline{\hspace{2cm}} \\
\end{tabular}

\egroup

\newpage


\textbf{49-variant}\\

\bgroup
\def\arraystretch{1.6} % 1 is the default, change whatever you need

\begin{tabular}{|m{5.7cm}|m{9.5cm}|}
\hline
Familiyası hám atı & \\
\hline
Fakulteti  & \\
\hline
Toparı hám tálim baǵdarı  & \\
\hline
\end{tabular}

\vspace{1cm}

\begin{tabular}{|m{0.7cm}|m{10cm}|m{4cm}|}
\hline
№ & Soraw & Juwap \\
\hline
1. & $OY$ kósheriniń teńlemesi? & $x=0$ \\
\hline
2. & Egerde $a=\{ x_1; y_1; z_1\}, b=\{ x_2, y_2; z_2\}$ bolsa, vektor kóbeymeniń koordinatalarda ańlatılıwı qanday boladı? &  $\lbrack ab\rbrack=\{y_1z_2-y_2z_1; z_1x_2-z_2x_1; x_1y_2-x_2y_1\}$ \\
\hline
3. & $A_1x+B_1y+C_1z+D_1=0$ hám $Ax_2+By_2+Cz_2+D_2=0$ tegislikleri perpendikulyar bolıwı shárti & $A_1\cdot A_2+B_1\cdot B_2+C_1\cdot C_2=0$ \\
\hline
4. & Úsh vektordıń aralas kóbeymesi ushın $(abc)=0$ teńligi orınlı bolsa ne dep ataladı? & $\overline{a}$, $\overline{b}$ hám $\overline{c}$ vektorları komplanar \\
\hline
5. & $x-2y+1=0$ teńlemesi menen berilgen tuwrınıń normal túrdegi teńlemesin kórsetiń. & $\frac{x}{- \sqrt{5}}+\frac{2y}{\sqrt{5}}-\frac{1}{\sqrt{5}}=0$ \\
\hline
6. & $3x-y+5=0$, $x+3y-4=0$ tuwrı sızıqları arasındaǵı múyeshti tabıń. & $90^{0}$ \\
\hline
7. & $\overline{a}=\{5,-6, 1 \}, \overline{b}=\{-4, 3, 0 \} $, $\overline{c}=\left\{ 5,-8, 10 \right\}$ vektorları berilgen. $2{\overline{a}}^{2}+4{\overline{b}}^{2}-5{\overline{c}}^{2}$ ańlatpasınıń mánisin tabıń. & $-721$ \\
\hline
8. & $(2, 3)$ hám $(4, 3)$ noqatlarınan ótiwshi tuwrı sızıqtıń teńlemesin dúziń. & $ y-3=0$ \\
\hline
9. & $x^{2}+y^{2}-2x+4y=0$ sheńberdiń teńlemesin kanonikalıq túrdegi teńlemege alıp keliń. & $(x-1)^{2}+(y+2)^{2}=5$ \\
\hline
10. & $A(4, 3), B(7, 7)$ noqatları arasındaǵı aralıqtı tabıń. & $d(AB)=5$ \\
\hline
\end{tabular}

\vspace{1cm}

\begin{tabular}{lll}
Tuwrı juwaplar sanı: \underline{\hspace{1.5cm}} & 
Bahası: \underline{\hspace{1.5cm}} & 
Imtixan alıwshınıń qolı: \underline{\hspace{2cm}} \\
\end{tabular}

\egroup

\newpage


\textbf{50-variant}\\

\bgroup
\def\arraystretch{1.6} % 1 is the default, change whatever you need

\begin{tabular}{|m{5.7cm}|m{9.5cm}|}
\hline
Familiyası hám atı & \\
\hline
Fakulteti  & \\
\hline
Toparı hám tálim baǵdarı  & \\
\hline
\end{tabular}

\vspace{1cm}

\begin{tabular}{|m{0.7cm}|m{10cm}|m{4cm}|}
\hline
№ & Soraw & Juwap \\
\hline
1. & $A_1x+B_1y+C_1z+D_1=0$ hám $Ax_2+By_2+Cz_2+D_2=0$ tegislikleri ústpe-úst túsiwi shárti? & $\frac{A_1}{A_2}=\frac{B_1}{B_2}=\frac{C_1}{C_2}=\frac{D_1}{D_2}$ \\
\hline
2. & Eki vektordıń skalyar kóbeymesiniń formulası? & $(ab)=|a||b|\cos\varphi$ \\
\hline
3. & $A_1x+B_1y+C_1z+D_1=0$ hám $Ax_2+By_2+Cz_2+D_2=0$ tegislikleri parallel bolıwı shárti & $\frac{A_1}{A_2}=\frac{B_1}{B_2}=\frac{C_1}{C_2}$ \\
\hline
4. & $Ax+C=0$ tuwrı sızıqtıń grafigi koordinata kósherlerine salıstırǵanda qanday jaylasqan? & $OY$ kósherine parallel \\
\hline
5. & $3x^{2}+10xy+3y^{2}-2x-14y-13=0$ teńlemesiniń tipin anıqlań. & giperbola \\
\hline
6. & $x^{2}-4y^{2}+6x+5=0$ giperbolanıń kanonikalıq teńlemesin dúziń. & $\frac{(x+3)^{2}}{4}-\frac{y^{2}}{1}=1$ \\
\hline
7. & $M_{1}M_{2}$ kesindiniń ortasınıń koordinatalarınıń tabıń, eger $M_{1} (2, 3), M_{2} (4, 7)$ bolsa. & $(3,5)$ \\
\hline
8. & $x+y-3=0$ hám $2x+3y-8=0$ tuwrıları óz-ara qanday jaylasqan? & kesilisedi \\
\hline
9. & $x^{2}+y^{2}-2x+4y-20=0$ sheńberdiń $C$ orayın hám $R$ radiusın tabıń. & $C(1;-2), R=5$ \\
\hline
10. & $(x+1)^{2}+(y-2) ^{2}+(z+2) ^{2}=49$ sferanıń orayınıń koordinataların tabıń. & $(-1,2,-2)$ \\
\hline
\end{tabular}

\vspace{1cm}

\begin{tabular}{lll}
Tuwrı juwaplar sanı: \underline{\hspace{1.5cm}} & 
Bahası: \underline{\hspace{1.5cm}} & 
Imtixan alıwshınıń qolı: \underline{\hspace{2cm}} \\
\end{tabular}

\egroup

\newpage


\textbf{51-variant}\\

\bgroup
\def\arraystretch{1.6} % 1 is the default, change whatever you need

\begin{tabular}{|m{5.7cm}|m{9.5cm}|}
\hline
Familiyası hám atı & \\
\hline
Fakulteti  & \\
\hline
Toparı hám tálim baǵdarı  & \\
\hline
\end{tabular}

\vspace{1cm}

\begin{tabular}{|m{0.7cm}|m{10cm}|m{4cm}|}
\hline
№ & Soraw & Juwap \\
\hline
1. & Eki vektor qashan kollinear dep ataladı? & bir tuwrıda yamasa parallel tuwrıda jaylasqan bolsa \\
\hline
2. & Tuwrı múyeshli koordinatalar sisteması dep nege aytamız? & Masshtab birlikleri berilgen o'zara perpendikulyar $OX$ hám $OY$ kósherleri \\
\hline
3. & $OXY$ tegisliginiń teńlemesi? & $z=0$ \\
\hline
4. & Giperbolanıń kanonikalıq teńlemesi? & $\frac{x^2}{a^2}-\frac{y^2}{b^2}=1$ \\
\hline
5. & Eger $2a=16, e=\frac{5}{4}$ bolsa, fokusı abscissa kósherinde, koordinata basına salıstırǵanda simmetriyalıq jaylasqan giperbolanıń teńlemesin dúziń. & $\frac{x^{2}}{64}-\frac{y^{2}}{36}=1$ \\
\hline
6. & Eger $2b=24, 2 c=10$ bolsa, onda abscissa kósherinde koordinata basına salıstırǵanda simmetriyalıq jaylasqan fokuslarǵa iye, ellipstiń teńlemesin dúziń. & $\frac{x^{2}}{169}+\frac{y^{2}}{144}=1$ \\
\hline
7. & $M_{1} (12;-1)$ hám $M_{2} (0;4)$ noqatlardıń arasındaǵı aralıqtı tabıń. & $13$ \\
\hline
8. & $x+y=0$ teńlemesi menen berilgen tuwrı sızıqtıń múyeshlik koefficientin anıqlań. & $- 1$ \\
\hline
9. & Orayı $C (-1;2)$ noqatında, $A (-2;6 )$ noqatınan ótetuǵın sheńberdiń teńlemesin dúziń. & $(x+1)^{2}+(y-2)^{2}=17$ \\
\hline
10. & $x+2=0$ keńislik qanday geometriyalıq betlikti anıqlaydı? &  $OYZ$ tegisligine parallel bolǵan tegislikti \\
\hline
\end{tabular}

\vspace{1cm}

\begin{tabular}{lll}
Tuwrı juwaplar sanı: \underline{\hspace{1.5cm}} & 
Bahası: \underline{\hspace{1.5cm}} & 
Imtixan alıwshınıń qolı: \underline{\hspace{2cm}} \\
\end{tabular}

\egroup

\newpage


\textbf{52-variant}\\

\bgroup
\def\arraystretch{1.6} % 1 is the default, change whatever you need

\begin{tabular}{|m{5.7cm}|m{9.5cm}|}
\hline
Familiyası hám atı & \\
\hline
Fakulteti  & \\
\hline
Toparı hám tálim baǵdarı  & \\
\hline
\end{tabular}

\vspace{1cm}

\begin{tabular}{|m{0.7cm}|m{10cm}|m{4cm}|}
\hline
№ & Soraw & Juwap \\
\hline
1. & Eki vektordıń vektor kóbeymesiniń uzınlıǵın tabıw formulası? & $\left| \lbrack ab\rbrack \right|=|a||b|\sin\varphi$ \\
\hline
2. & Tegislikdegi qálegen noqatınan berilgen eki noqatqa shekemgi bolǵan aralıqlardıń ayırmasınıń modulı ózgermeytuǵın bolǵan noqatlardıń geometriyalıq ornı ne dep ataladı? & giperbola \\
\hline
3. & Eki tuwrı sızıq arasındaǵı múyeshti tabıw formulası? & $\text{tg}\varphi=\frac{k_2-k_1}{1+k_1k_2}$ \\
\hline
4. & $\frac{x^2}{a^2}-\frac{y^2}{b^2}=1$ giperbolanıń $(x_0;y_0)$ noqatındaǵı urınbasınıń teńlemesin kórsetiń. & $\frac{x_0x}{a^2}-\frac{y_0y}{b^2}=1$ \\
\hline
5. & $\frac{x^{2}}{225}-\frac{y^{2}}{64}=-1$ giperbola fokusınıń koordinatalarınıń tabıń. & $F_{1}(0;-17), F_{2}(0;17)$ \\
\hline
6. & $9x^{2}+25y^{2}=225$ ellipsi berilgen, ellipstiń fokusların, ekscentrisitetin tabıń. & $F_1\left(-4;0 \right) , F_2\left( 4;0 \right) , e = \frac{4}{5}$ \\
\hline
7. & $A (-1;0;1),\ B (1;-1;0)$ noqatları berilgen. $\overline{BA}$ vektorın tabıń. & $\left\{ - 2;1;1 \right\}$ \\
\hline
8. & $2x+3y+4=0$ tuwrısına parallel hám $M_{0} (2;1)$ noqattan ótetuǵın tuwrınıń teńlemesin dúziń. & $2x+3y-7=0$ \\
\hline
9. & $x+y-12=0$ tuwrısı $x^{2}+y^{2}-2y=0$ sheńberge salıstırǵanda qanday jaylasqan? & sırtında jaylasqan \\
\hline
10. & $\left| \overline{a} \right|=8, \left| \overline{b} \right|=5, \alpha=60^{0}$ bolsa, $( \overline{a}\overline{b} )$ ni tabıń. & $20$ \\
\hline
\end{tabular}

\vspace{1cm}

\begin{tabular}{lll}
Tuwrı juwaplar sanı: \underline{\hspace{1.5cm}} & 
Bahası: \underline{\hspace{1.5cm}} & 
Imtixan alıwshınıń qolı: \underline{\hspace{2cm}} \\
\end{tabular}

\egroup

\newpage


\textbf{53-variant}\\

\bgroup
\def\arraystretch{1.6} % 1 is the default, change whatever you need

\begin{tabular}{|m{5.7cm}|m{9.5cm}|}
\hline
Familiyası hám atı & \\
\hline
Fakulteti  & \\
\hline
Toparı hám tálim baǵdarı  & \\
\hline
\end{tabular}

\vspace{1cm}

\begin{tabular}{|m{0.7cm}|m{10cm}|m{4cm}|}
\hline
№ & Soraw & Juwap \\
\hline
1. & Vektorlardıń kósherdegi proekciyasınıń formulası? & $x=|a|\cos\varphi, y=|a|\sin\varphi$ \\
\hline
2. & $Ax+By+D=0$ teńlemesi arqalı ... tegisliktiń teńlemesi berilgen? & $OZ$ kósherine parallel \\
\hline
3. & $\frac{x^2}{a^2}+\frac{y^2}{b^2}=1$ ellipstiń $(x_0;y_0)$ noqatındaǵı urınbasınıń teńlemesin tabıń. & $\frac{x_0x}{a^2}+\frac{y_0y}{b^2}=1$ \\
\hline
4. & Vektorlardı qosıw koordinatalarda qanday formula menen anıqlanadı? & $\overline{a}+\overline{b}=\{x_1+x_2;y_1+y_2\}$ \\
\hline
5. & $2x+3y-6=0$ tuwrınıń teńlemesin kesindilerde berilgen teńleme túrinde kórsetiń. & $\frac{x}{3} + \frac{ y }{ 2 } =  1$ \\
\hline
6. & $\overline{a}=\left\{ 4,-2,-4 \right\}$ hám $\overline{b}=\left\{ 6,-3, 2 \right\}$ vektorları berilgen, $(\overline{a}-\overline{b}) ^{2}$-? & $41$ \\
\hline
7. & $5x-y+7=0$ hám $3x+2y=0$ tuwrıları arasındaǵı múyeshni tabıń. & $\varphi=\frac{\pi}{4}$ \\
\hline
8. & $\overline{a}=\left\{ 2, 1, 0 \right\}$ hám $\overline{b}=\left\{ 1, 0,-1 \right\}$ bolsa, $\overline{a}-\overline{b}$ ni tabıń. & $\overline{a} -\overline{b} = \left\{ 1,1,1 \right\}$ \\
\hline
9. & Koordinatalar kósherleri hám $ 3x+4y-12=0 $ tuwrı sızıǵı menen shegaralanǵan úshmúyeshliktiń maydanın tabıń. & $ S=6 $ \\
\hline
10. & $x-2y+1=0$ teńlemesi menen berilgen tuwrınıń normal túrdegi teńlemesin kórsetiń. & $\frac{x}{- \sqrt{5}}+\frac{2y}{\sqrt{5}}-\frac{1}{\sqrt{5}}=0$ \\
\hline
\end{tabular}

\vspace{1cm}

\begin{tabular}{lll}
Tuwrı juwaplar sanı: \underline{\hspace{1.5cm}} & 
Bahası: \underline{\hspace{1.5cm}} & 
Imtixan alıwshınıń qolı: \underline{\hspace{2cm}} \\
\end{tabular}

\egroup

\newpage


\textbf{54-variant}\\

\bgroup
\def\arraystretch{1.6} % 1 is the default, change whatever you need

\begin{tabular}{|m{5.7cm}|m{9.5cm}|}
\hline
Familiyası hám atı & \\
\hline
Fakulteti  & \\
\hline
Toparı hám tálim baǵdarı  & \\
\hline
\end{tabular}

\vspace{1cm}

\begin{tabular}{|m{0.7cm}|m{10cm}|m{4cm}|}
\hline
№ & Soraw & Juwap \\
\hline
1. & $OY$ kósheriniń teńlemesi? & $x=0$ \\
\hline
2. & Egerde $a=\{ x_1; y_1; z_1\}, b=\{ x_2, y_2; z_2\}$ bolsa, vektor kóbeymeniń koordinatalarda ańlatılıwı qanday boladı? &  $\lbrack ab\rbrack=\{y_1z_2-y_2z_1; z_1x_2-z_2x_1; x_1y_2-x_2y_1\}$ \\
\hline
3. & $A_1x+B_1y+C_1z+D_1=0$ hám $Ax_2+By_2+Cz_2+D_2=0$ tegislikleri perpendikulyar bolıwı shárti & $A_1\cdot A_2+B_1\cdot B_2+C_1\cdot C_2=0$ \\
\hline
4. & Úsh vektordıń aralas kóbeymesi ushın $(abc)=0$ teńligi orınlı bolsa ne dep ataladı? & $\overline{a}$, $\overline{b}$ hám $\overline{c}$ vektorları komplanar \\
\hline
5. & $3x-y+5=0$, $x+3y-4=0$ tuwrı sızıqları arasındaǵı múyeshti tabıń. & $90^{0}$ \\
\hline
6. & $\overline{a}=\{5,-6, 1 \}, \overline{b}=\{-4, 3, 0 \} $, $\overline{c}=\left\{ 5,-8, 10 \right\}$ vektorları berilgen. $2{\overline{a}}^{2}+4{\overline{b}}^{2}-5{\overline{c}}^{2}$ ańlatpasınıń mánisin tabıń. & $-721$ \\
\hline
7. & $(2, 3)$ hám $(4, 3)$ noqatlarınan ótiwshi tuwrı sızıqtıń teńlemesin dúziń. & $ y-3=0$ \\
\hline
8. & $x^{2}+y^{2}-2x+4y=0$ sheńberdiń teńlemesin kanonikalıq túrdegi teńlemege alıp keliń. & $(x-1)^{2}+(y+2)^{2}=5$ \\
\hline
9. & $A(4, 3), B(7, 7)$ noqatları arasındaǵı aralıqtı tabıń. & $d(AB)=5$ \\
\hline
10. & $3x^{2}+10xy+3y^{2}-2x-14y-13=0$ teńlemesiniń tipin anıqlań. & giperbola \\
\hline
\end{tabular}

\vspace{1cm}

\begin{tabular}{lll}
Tuwrı juwaplar sanı: \underline{\hspace{1.5cm}} & 
Bahası: \underline{\hspace{1.5cm}} & 
Imtixan alıwshınıń qolı: \underline{\hspace{2cm}} \\
\end{tabular}

\egroup

\newpage


\textbf{55-variant}\\

\bgroup
\def\arraystretch{1.6} % 1 is the default, change whatever you need

\begin{tabular}{|m{5.7cm}|m{9.5cm}|}
\hline
Familiyası hám atı & \\
\hline
Fakulteti  & \\
\hline
Toparı hám tálim baǵdarı  & \\
\hline
\end{tabular}

\vspace{1cm}

\begin{tabular}{|m{0.7cm}|m{10cm}|m{4cm}|}
\hline
№ & Soraw & Juwap \\
\hline
1. & $A_1x+B_1y+C_1z+D_1=0$ hám $Ax_2+By_2+Cz_2+D_2=0$ tegislikleri ústpe-úst túsiwi shárti? & $\frac{A_1}{A_2}=\frac{B_1}{B_2}=\frac{C_1}{C_2}=\frac{D_1}{D_2}$ \\
\hline
2. & Eki vektordıń skalyar kóbeymesiniń formulası? & $(ab)=|a||b|\cos\varphi$ \\
\hline
3. & $A_1x+B_1y+C_1z+D_1=0$ hám $Ax_2+By_2+Cz_2+D_2=0$ tegislikleri parallel bolıwı shárti & $\frac{A_1}{A_2}=\frac{B_1}{B_2}=\frac{C_1}{C_2}$ \\
\hline
4. & $Ax+C=0$ tuwrı sızıqtıń grafigi koordinata kósherlerine salıstırǵanda qanday jaylasqan? & $OY$ kósherine parallel \\
\hline
5. & $x^{2}-4y^{2}+6x+5=0$ giperbolanıń kanonikalıq teńlemesin dúziń. & $\frac{(x+3)^{2}}{4}-\frac{y^{2}}{1}=1$ \\
\hline
6. & $M_{1}M_{2}$ kesindiniń ortasınıń koordinatalarınıń tabıń, eger $M_{1} (2, 3), M_{2} (4, 7)$ bolsa. & $(3,5)$ \\
\hline
7. & $x+y-3=0$ hám $2x+3y-8=0$ tuwrıları óz-ara qanday jaylasqan? & kesilisedi \\
\hline
8. & $x^{2}+y^{2}-2x+4y-20=0$ sheńberdiń $C$ orayın hám $R$ radiusın tabıń. & $C(1;-2), R=5$ \\
\hline
9. & $(x+1)^{2}+(y-2) ^{2}+(z+2) ^{2}=49$ sferanıń orayınıń koordinataların tabıń. & $(-1,2,-2)$ \\
\hline
10. & Eger $2a=16, e=\frac{5}{4}$ bolsa, fokusı abscissa kósherinde, koordinata basına salıstırǵanda simmetriyalıq jaylasqan giperbolanıń teńlemesin dúziń. & $\frac{x^{2}}{64}-\frac{y^{2}}{36}=1$ \\
\hline
\end{tabular}

\vspace{1cm}

\begin{tabular}{lll}
Tuwrı juwaplar sanı: \underline{\hspace{1.5cm}} & 
Bahası: \underline{\hspace{1.5cm}} & 
Imtixan alıwshınıń qolı: \underline{\hspace{2cm}} \\
\end{tabular}

\egroup

\newpage


\textbf{56-variant}\\

\bgroup
\def\arraystretch{1.6} % 1 is the default, change whatever you need

\begin{tabular}{|m{5.7cm}|m{9.5cm}|}
\hline
Familiyası hám atı & \\
\hline
Fakulteti  & \\
\hline
Toparı hám tálim baǵdarı  & \\
\hline
\end{tabular}

\vspace{1cm}

\begin{tabular}{|m{0.7cm}|m{10cm}|m{4cm}|}
\hline
№ & Soraw & Juwap \\
\hline
1. & Eki vektor qashan kollinear dep ataladı? & bir tuwrıda yamasa parallel tuwrıda jaylasqan bolsa \\
\hline
2. & Tuwrı múyeshli koordinatalar sisteması dep nege aytamız? & Masshtab birlikleri berilgen o'zara perpendikulyar $OX$ hám $OY$ kósherleri \\
\hline
3. & $OXY$ tegisliginiń teńlemesi? & $z=0$ \\
\hline
4. & Giperbolanıń kanonikalıq teńlemesi? & $\frac{x^2}{a^2}-\frac{y^2}{b^2}=1$ \\
\hline
5. & Eger $2b=24, 2 c=10$ bolsa, onda abscissa kósherinde koordinata basına salıstırǵanda simmetriyalıq jaylasqan fokuslarǵa iye, ellipstiń teńlemesin dúziń. & $\frac{x^{2}}{169}+\frac{y^{2}}{144}=1$ \\
\hline
6. & $M_{1} (12;-1)$ hám $M_{2} (0;4)$ noqatlardıń arasındaǵı aralıqtı tabıń. & $13$ \\
\hline
7. & $x+y=0$ teńlemesi menen berilgen tuwrı sızıqtıń múyeshlik koefficientin anıqlań. & $- 1$ \\
\hline
8. & Orayı $C (-1;2)$ noqatında, $A (-2;6 )$ noqatınan ótetuǵın sheńberdiń teńlemesin dúziń. & $(x+1)^{2}+(y-2)^{2}=17$ \\
\hline
9. & $x+2=0$ keńislik qanday geometriyalıq betlikti anıqlaydı? &  $OYZ$ tegisligine parallel bolǵan tegislikti \\
\hline
10. & $\frac{x^{2}}{225}-\frac{y^{2}}{64}=-1$ giperbola fokusınıń koordinatalarınıń tabıń. & $F_{1}(0;-17), F_{2}(0;17)$ \\
\hline
\end{tabular}

\vspace{1cm}

\begin{tabular}{lll}
Tuwrı juwaplar sanı: \underline{\hspace{1.5cm}} & 
Bahası: \underline{\hspace{1.5cm}} & 
Imtixan alıwshınıń qolı: \underline{\hspace{2cm}} \\
\end{tabular}

\egroup

\newpage


\textbf{57-variant}\\

\bgroup
\def\arraystretch{1.6} % 1 is the default, change whatever you need

\begin{tabular}{|m{5.7cm}|m{9.5cm}|}
\hline
Familiyası hám atı & \\
\hline
Fakulteti  & \\
\hline
Toparı hám tálim baǵdarı  & \\
\hline
\end{tabular}

\vspace{1cm}

\begin{tabular}{|m{0.7cm}|m{10cm}|m{4cm}|}
\hline
№ & Soraw & Juwap \\
\hline
1. & Eki vektordıń vektor kóbeymesiniń uzınlıǵın tabıw formulası? & $\left| \lbrack ab\rbrack \right|=|a||b|\sin\varphi$ \\
\hline
2. & Tegislikdegi qálegen noqatınan berilgen eki noqatqa shekemgi bolǵan aralıqlardıń ayırmasınıń modulı ózgermeytuǵın bolǵan noqatlardıń geometriyalıq ornı ne dep ataladı? & giperbola \\
\hline
3. & Eki tuwrı sızıq arasındaǵı múyeshti tabıw formulası? & $\text{tg}\varphi=\frac{k_2-k_1}{1+k_1k_2}$ \\
\hline
4. & $\frac{x^2}{a^2}-\frac{y^2}{b^2}=1$ giperbolanıń $(x_0;y_0)$ noqatındaǵı urınbasınıń teńlemesin kórsetiń. & $\frac{x_0x}{a^2}-\frac{y_0y}{b^2}=1$ \\
\hline
5. & $9x^{2}+25y^{2}=225$ ellipsi berilgen, ellipstiń fokusların, ekscentrisitetin tabıń. & $F_1\left(-4;0 \right) , F_2\left( 4;0 \right) , e = \frac{4}{5}$ \\
\hline
6. & $A (-1;0;1),\ B (1;-1;0)$ noqatları berilgen. $\overline{BA}$ vektorın tabıń. & $\left\{ - 2;1;1 \right\}$ \\
\hline
7. & $2x+3y+4=0$ tuwrısına parallel hám $M_{0} (2;1)$ noqattan ótetuǵın tuwrınıń teńlemesin dúziń. & $2x+3y-7=0$ \\
\hline
8. & $x+y-12=0$ tuwrısı $x^{2}+y^{2}-2y=0$ sheńberge salıstırǵanda qanday jaylasqan? & sırtında jaylasqan \\
\hline
9. & $\left| \overline{a} \right|=8, \left| \overline{b} \right|=5, \alpha=60^{0}$ bolsa, $( \overline{a}\overline{b} )$ ni tabıń. & $20$ \\
\hline
10. & $2x+3y-6=0$ tuwrınıń teńlemesin kesindilerde berilgen teńleme túrinde kórsetiń. & $\frac{x}{3} + \frac{ y }{ 2 } =  1$ \\
\hline
\end{tabular}

\vspace{1cm}

\begin{tabular}{lll}
Tuwrı juwaplar sanı: \underline{\hspace{1.5cm}} & 
Bahası: \underline{\hspace{1.5cm}} & 
Imtixan alıwshınıń qolı: \underline{\hspace{2cm}} \\
\end{tabular}

\egroup

\newpage


\textbf{58-variant}\\

\bgroup
\def\arraystretch{1.6} % 1 is the default, change whatever you need

\begin{tabular}{|m{5.7cm}|m{9.5cm}|}
\hline
Familiyası hám atı & \\
\hline
Fakulteti  & \\
\hline
Toparı hám tálim baǵdarı  & \\
\hline
\end{tabular}

\vspace{1cm}

\begin{tabular}{|m{0.7cm}|m{10cm}|m{4cm}|}
\hline
№ & Soraw & Juwap \\
\hline
1. & Vektorlardıń kósherdegi proekciyasınıń formulası? & $x=|a|\cos\varphi, y=|a|\sin\varphi$ \\
\hline
2. & $Ax+By+D=0$ teńlemesi arqalı ... tegisliktiń teńlemesi berilgen? & $OZ$ kósherine parallel \\
\hline
3. & $\frac{x^2}{a^2}+\frac{y^2}{b^2}=1$ ellipstiń $(x_0;y_0)$ noqatındaǵı urınbasınıń teńlemesin tabıń. & $\frac{x_0x}{a^2}+\frac{y_0y}{b^2}=1$ \\
\hline
4. & Vektorlardı qosıw koordinatalarda qanday formula menen anıqlanadı? & $\overline{a}+\overline{b}=\{x_1+x_2;y_1+y_2\}$ \\
\hline
5. & $\overline{a}=\left\{ 4,-2,-4 \right\}$ hám $\overline{b}=\left\{ 6,-3, 2 \right\}$ vektorları berilgen, $(\overline{a}-\overline{b}) ^{2}$-? & $41$ \\
\hline
6. & $5x-y+7=0$ hám $3x+2y=0$ tuwrıları arasındaǵı múyeshni tabıń. & $\varphi=\frac{\pi}{4}$ \\
\hline
7. & $\overline{a}=\left\{ 2, 1, 0 \right\}$ hám $\overline{b}=\left\{ 1, 0,-1 \right\}$ bolsa, $\overline{a}-\overline{b}$ ni tabıń. & $\overline{a} -\overline{b} = \left\{ 1,1,1 \right\}$ \\
\hline
8. & Koordinatalar kósherleri hám $ 3x+4y-12=0 $ tuwrı sızıǵı menen shegaralanǵan úshmúyeshliktiń maydanın tabıń. & $ S=6 $ \\
\hline
9. & $x-2y+1=0$ teńlemesi menen berilgen tuwrınıń normal túrdegi teńlemesin kórsetiń. & $\frac{x}{- \sqrt{5}}+\frac{2y}{\sqrt{5}}-\frac{1}{\sqrt{5}}=0$ \\
\hline
10. & $3x-y+5=0$, $x+3y-4=0$ tuwrı sızıqları arasındaǵı múyeshti tabıń. & $90^{0}$ \\
\hline
\end{tabular}

\vspace{1cm}

\begin{tabular}{lll}
Tuwrı juwaplar sanı: \underline{\hspace{1.5cm}} & 
Bahası: \underline{\hspace{1.5cm}} & 
Imtixan alıwshınıń qolı: \underline{\hspace{2cm}} \\
\end{tabular}

\egroup

\newpage


\textbf{59-variant}\\

\bgroup
\def\arraystretch{1.6} % 1 is the default, change whatever you need

\begin{tabular}{|m{5.7cm}|m{9.5cm}|}
\hline
Familiyası hám atı & \\
\hline
Fakulteti  & \\
\hline
Toparı hám tálim baǵdarı  & \\
\hline
\end{tabular}

\vspace{1cm}

\begin{tabular}{|m{0.7cm}|m{10cm}|m{4cm}|}
\hline
№ & Soraw & Juwap \\
\hline
1. & $OY$ kósheriniń teńlemesi? & $x=0$ \\
\hline
2. & Egerde $a=\{ x_1; y_1; z_1\}, b=\{ x_2, y_2; z_2\}$ bolsa, vektor kóbeymeniń koordinatalarda ańlatılıwı qanday boladı? &  $\lbrack ab\rbrack=\{y_1z_2-y_2z_1; z_1x_2-z_2x_1; x_1y_2-x_2y_1\}$ \\
\hline
3. & $A_1x+B_1y+C_1z+D_1=0$ hám $Ax_2+By_2+Cz_2+D_2=0$ tegislikleri perpendikulyar bolıwı shárti & $A_1\cdot A_2+B_1\cdot B_2+C_1\cdot C_2=0$ \\
\hline
4. & Úsh vektordıń aralas kóbeymesi ushın $(abc)=0$ teńligi orınlı bolsa ne dep ataladı? & $\overline{a}$, $\overline{b}$ hám $\overline{c}$ vektorları komplanar \\
\hline
5. & $\overline{a}=\{5,-6, 1 \}, \overline{b}=\{-4, 3, 0 \} $, $\overline{c}=\left\{ 5,-8, 10 \right\}$ vektorları berilgen. $2{\overline{a}}^{2}+4{\overline{b}}^{2}-5{\overline{c}}^{2}$ ańlatpasınıń mánisin tabıń. & $-721$ \\
\hline
6. & $(2, 3)$ hám $(4, 3)$ noqatlarınan ótiwshi tuwrı sızıqtıń teńlemesin dúziń. & $ y-3=0$ \\
\hline
7. & $x^{2}+y^{2}-2x+4y=0$ sheńberdiń teńlemesin kanonikalıq túrdegi teńlemege alıp keliń. & $(x-1)^{2}+(y+2)^{2}=5$ \\
\hline
8. & $A(4, 3), B(7, 7)$ noqatları arasındaǵı aralıqtı tabıń. & $d(AB)=5$ \\
\hline
9. & $3x^{2}+10xy+3y^{2}-2x-14y-13=0$ teńlemesiniń tipin anıqlań. & giperbola \\
\hline
10. & $x^{2}-4y^{2}+6x+5=0$ giperbolanıń kanonikalıq teńlemesin dúziń. & $\frac{(x+3)^{2}}{4}-\frac{y^{2}}{1}=1$ \\
\hline
\end{tabular}

\vspace{1cm}

\begin{tabular}{lll}
Tuwrı juwaplar sanı: \underline{\hspace{1.5cm}} & 
Bahası: \underline{\hspace{1.5cm}} & 
Imtixan alıwshınıń qolı: \underline{\hspace{2cm}} \\
\end{tabular}

\egroup

\newpage


\textbf{60-variant}\\

\bgroup
\def\arraystretch{1.6} % 1 is the default, change whatever you need

\begin{tabular}{|m{5.7cm}|m{9.5cm}|}
\hline
Familiyası hám atı & \\
\hline
Fakulteti  & \\
\hline
Toparı hám tálim baǵdarı  & \\
\hline
\end{tabular}

\vspace{1cm}

\begin{tabular}{|m{0.7cm}|m{10cm}|m{4cm}|}
\hline
№ & Soraw & Juwap \\
\hline
1. & $A_1x+B_1y+C_1z+D_1=0$ hám $Ax_2+By_2+Cz_2+D_2=0$ tegislikleri ústpe-úst túsiwi shárti? & $\frac{A_1}{A_2}=\frac{B_1}{B_2}=\frac{C_1}{C_2}=\frac{D_1}{D_2}$ \\
\hline
2. & Eki vektordıń skalyar kóbeymesiniń formulası? & $(ab)=|a||b|\cos\varphi$ \\
\hline
3. & $A_1x+B_1y+C_1z+D_1=0$ hám $Ax_2+By_2+Cz_2+D_2=0$ tegislikleri parallel bolıwı shárti & $\frac{A_1}{A_2}=\frac{B_1}{B_2}=\frac{C_1}{C_2}$ \\
\hline
4. & $Ax+C=0$ tuwrı sızıqtıń grafigi koordinata kósherlerine salıstırǵanda qanday jaylasqan? & $OY$ kósherine parallel \\
\hline
5. & $M_{1}M_{2}$ kesindiniń ortasınıń koordinatalarınıń tabıń, eger $M_{1} (2, 3), M_{2} (4, 7)$ bolsa. & $(3,5)$ \\
\hline
6. & $x+y-3=0$ hám $2x+3y-8=0$ tuwrıları óz-ara qanday jaylasqan? & kesilisedi \\
\hline
7. & $x^{2}+y^{2}-2x+4y-20=0$ sheńberdiń $C$ orayın hám $R$ radiusın tabıń. & $C(1;-2), R=5$ \\
\hline
8. & $(x+1)^{2}+(y-2) ^{2}+(z+2) ^{2}=49$ sferanıń orayınıń koordinataların tabıń. & $(-1,2,-2)$ \\
\hline
9. & Eger $2a=16, e=\frac{5}{4}$ bolsa, fokusı abscissa kósherinde, koordinata basına salıstırǵanda simmetriyalıq jaylasqan giperbolanıń teńlemesin dúziń. & $\frac{x^{2}}{64}-\frac{y^{2}}{36}=1$ \\
\hline
10. & Eger $2b=24, 2 c=10$ bolsa, onda abscissa kósherinde koordinata basına salıstırǵanda simmetriyalıq jaylasqan fokuslarǵa iye, ellipstiń teńlemesin dúziń. & $\frac{x^{2}}{169}+\frac{y^{2}}{144}=1$ \\
\hline
\end{tabular}

\vspace{1cm}

\begin{tabular}{lll}
Tuwrı juwaplar sanı: \underline{\hspace{1.5cm}} & 
Bahası: \underline{\hspace{1.5cm}} & 
Imtixan alıwshınıń qolı: \underline{\hspace{2cm}} \\
\end{tabular}

\egroup

\newpage


\textbf{61-variant}\\

\bgroup
\def\arraystretch{1.6} % 1 is the default, change whatever you need

\begin{tabular}{|m{5.7cm}|m{9.5cm}|}
\hline
Familiyası hám atı & \\
\hline
Fakulteti  & \\
\hline
Toparı hám tálim baǵdarı  & \\
\hline
\end{tabular}

\vspace{1cm}

\begin{tabular}{|m{0.7cm}|m{10cm}|m{4cm}|}
\hline
№ & Soraw & Juwap \\
\hline
1. & Eki vektor qashan kollinear dep ataladı? & bir tuwrıda yamasa parallel tuwrıda jaylasqan bolsa \\
\hline
2. & Tuwrı múyeshli koordinatalar sisteması dep nege aytamız? & Masshtab birlikleri berilgen o'zara perpendikulyar $OX$ hám $OY$ kósherleri \\
\hline
3. & $OXY$ tegisliginiń teńlemesi? & $z=0$ \\
\hline
4. & Giperbolanıń kanonikalıq teńlemesi? & $\frac{x^2}{a^2}-\frac{y^2}{b^2}=1$ \\
\hline
5. & $M_{1} (12;-1)$ hám $M_{2} (0;4)$ noqatlardıń arasındaǵı aralıqtı tabıń. & $13$ \\
\hline
6. & $x+y=0$ teńlemesi menen berilgen tuwrı sızıqtıń múyeshlik koefficientin anıqlań. & $- 1$ \\
\hline
7. & Orayı $C (-1;2)$ noqatında, $A (-2;6 )$ noqatınan ótetuǵın sheńberdiń teńlemesin dúziń. & $(x+1)^{2}+(y-2)^{2}=17$ \\
\hline
8. & $x+2=0$ keńislik qanday geometriyalıq betlikti anıqlaydı? &  $OYZ$ tegisligine parallel bolǵan tegislikti \\
\hline
9. & $\frac{x^{2}}{225}-\frac{y^{2}}{64}=-1$ giperbola fokusınıń koordinatalarınıń tabıń. & $F_{1}(0;-17), F_{2}(0;17)$ \\
\hline
10. & $9x^{2}+25y^{2}=225$ ellipsi berilgen, ellipstiń fokusların, ekscentrisitetin tabıń. & $F_1\left(-4;0 \right) , F_2\left( 4;0 \right) , e = \frac{4}{5}$ \\
\hline
\end{tabular}

\vspace{1cm}

\begin{tabular}{lll}
Tuwrı juwaplar sanı: \underline{\hspace{1.5cm}} & 
Bahası: \underline{\hspace{1.5cm}} & 
Imtixan alıwshınıń qolı: \underline{\hspace{2cm}} \\
\end{tabular}

\egroup

\newpage


\textbf{62-variant}\\

\bgroup
\def\arraystretch{1.6} % 1 is the default, change whatever you need

\begin{tabular}{|m{5.7cm}|m{9.5cm}|}
\hline
Familiyası hám atı & \\
\hline
Fakulteti  & \\
\hline
Toparı hám tálim baǵdarı  & \\
\hline
\end{tabular}

\vspace{1cm}

\begin{tabular}{|m{0.7cm}|m{10cm}|m{4cm}|}
\hline
№ & Soraw & Juwap \\
\hline
1. & Eki vektordıń vektor kóbeymesiniń uzınlıǵın tabıw formulası? & $\left| \lbrack ab\rbrack \right|=|a||b|\sin\varphi$ \\
\hline
2. & Tegislikdegi qálegen noqatınan berilgen eki noqatqa shekemgi bolǵan aralıqlardıń ayırmasınıń modulı ózgermeytuǵın bolǵan noqatlardıń geometriyalıq ornı ne dep ataladı? & giperbola \\
\hline
3. & Eki tuwrı sızıq arasındaǵı múyeshti tabıw formulası? & $\text{tg}\varphi=\frac{k_2-k_1}{1+k_1k_2}$ \\
\hline
4. & $\frac{x^2}{a^2}-\frac{y^2}{b^2}=1$ giperbolanıń $(x_0;y_0)$ noqatındaǵı urınbasınıń teńlemesin kórsetiń. & $\frac{x_0x}{a^2}-\frac{y_0y}{b^2}=1$ \\
\hline
5. & $A (-1;0;1),\ B (1;-1;0)$ noqatları berilgen. $\overline{BA}$ vektorın tabıń. & $\left\{ - 2;1;1 \right\}$ \\
\hline
6. & $2x+3y+4=0$ tuwrısına parallel hám $M_{0} (2;1)$ noqattan ótetuǵın tuwrınıń teńlemesin dúziń. & $2x+3y-7=0$ \\
\hline
7. & $x+y-12=0$ tuwrısı $x^{2}+y^{2}-2y=0$ sheńberge salıstırǵanda qanday jaylasqan? & sırtında jaylasqan \\
\hline
8. & $\left| \overline{a} \right|=8, \left| \overline{b} \right|=5, \alpha=60^{0}$ bolsa, $( \overline{a}\overline{b} )$ ni tabıń. & $20$ \\
\hline
9. & $2x+3y-6=0$ tuwrınıń teńlemesin kesindilerde berilgen teńleme túrinde kórsetiń. & $\frac{x}{3} + \frac{ y }{ 2 } =  1$ \\
\hline
10. & $\overline{a}=\left\{ 4,-2,-4 \right\}$ hám $\overline{b}=\left\{ 6,-3, 2 \right\}$ vektorları berilgen, $(\overline{a}-\overline{b}) ^{2}$-? & $41$ \\
\hline
\end{tabular}

\vspace{1cm}

\begin{tabular}{lll}
Tuwrı juwaplar sanı: \underline{\hspace{1.5cm}} & 
Bahası: \underline{\hspace{1.5cm}} & 
Imtixan alıwshınıń qolı: \underline{\hspace{2cm}} \\
\end{tabular}

\egroup

\newpage


\textbf{63-variant}\\

\bgroup
\def\arraystretch{1.6} % 1 is the default, change whatever you need

\begin{tabular}{|m{5.7cm}|m{9.5cm}|}
\hline
Familiyası hám atı & \\
\hline
Fakulteti  & \\
\hline
Toparı hám tálim baǵdarı  & \\
\hline
\end{tabular}

\vspace{1cm}

\begin{tabular}{|m{0.7cm}|m{10cm}|m{4cm}|}
\hline
№ & Soraw & Juwap \\
\hline
1. & Vektorlardıń kósherdegi proekciyasınıń formulası? & $x=|a|\cos\varphi, y=|a|\sin\varphi$ \\
\hline
2. & $Ax+By+D=0$ teńlemesi arqalı ... tegisliktiń teńlemesi berilgen? & $OZ$ kósherine parallel \\
\hline
3. & $\frac{x^2}{a^2}+\frac{y^2}{b^2}=1$ ellipstiń $(x_0;y_0)$ noqatındaǵı urınbasınıń teńlemesin tabıń. & $\frac{x_0x}{a^2}+\frac{y_0y}{b^2}=1$ \\
\hline
4. & Vektorlardı qosıw koordinatalarda qanday formula menen anıqlanadı? & $\overline{a}+\overline{b}=\{x_1+x_2;y_1+y_2\}$ \\
\hline
5. & $5x-y+7=0$ hám $3x+2y=0$ tuwrıları arasındaǵı múyeshni tabıń. & $\varphi=\frac{\pi}{4}$ \\
\hline
6. & $\overline{a}=\left\{ 2, 1, 0 \right\}$ hám $\overline{b}=\left\{ 1, 0,-1 \right\}$ bolsa, $\overline{a}-\overline{b}$ ni tabıń. & $\overline{a} -\overline{b} = \left\{ 1,1,1 \right\}$ \\
\hline
7. & Koordinatalar kósherleri hám $ 3x+4y-12=0 $ tuwrı sızıǵı menen shegaralanǵan úshmúyeshliktiń maydanın tabıń. & $ S=6 $ \\
\hline
8. & $x-2y+1=0$ teńlemesi menen berilgen tuwrınıń normal túrdegi teńlemesin kórsetiń. & $\frac{x}{- \sqrt{5}}+\frac{2y}{\sqrt{5}}-\frac{1}{\sqrt{5}}=0$ \\
\hline
9. & $3x-y+5=0$, $x+3y-4=0$ tuwrı sızıqları arasındaǵı múyeshti tabıń. & $90^{0}$ \\
\hline
10. & $\overline{a}=\{5,-6, 1 \}, \overline{b}=\{-4, 3, 0 \} $, $\overline{c}=\left\{ 5,-8, 10 \right\}$ vektorları berilgen. $2{\overline{a}}^{2}+4{\overline{b}}^{2}-5{\overline{c}}^{2}$ ańlatpasınıń mánisin tabıń. & $-721$ \\
\hline
\end{tabular}

\vspace{1cm}

\begin{tabular}{lll}
Tuwrı juwaplar sanı: \underline{\hspace{1.5cm}} & 
Bahası: \underline{\hspace{1.5cm}} & 
Imtixan alıwshınıń qolı: \underline{\hspace{2cm}} \\
\end{tabular}

\egroup

\newpage


\textbf{64-variant}\\

\bgroup
\def\arraystretch{1.6} % 1 is the default, change whatever you need

\begin{tabular}{|m{5.7cm}|m{9.5cm}|}
\hline
Familiyası hám atı & \\
\hline
Fakulteti  & \\
\hline
Toparı hám tálim baǵdarı  & \\
\hline
\end{tabular}

\vspace{1cm}

\begin{tabular}{|m{0.7cm}|m{10cm}|m{4cm}|}
\hline
№ & Soraw & Juwap \\
\hline
1. & $OY$ kósheriniń teńlemesi? & $x=0$ \\
\hline
2. & Egerde $a=\{ x_1; y_1; z_1\}, b=\{ x_2, y_2; z_2\}$ bolsa, vektor kóbeymeniń koordinatalarda ańlatılıwı qanday boladı? &  $\lbrack ab\rbrack=\{y_1z_2-y_2z_1; z_1x_2-z_2x_1; x_1y_2-x_2y_1\}$ \\
\hline
3. & $A_1x+B_1y+C_1z+D_1=0$ hám $Ax_2+By_2+Cz_2+D_2=0$ tegislikleri perpendikulyar bolıwı shárti & $A_1\cdot A_2+B_1\cdot B_2+C_1\cdot C_2=0$ \\
\hline
4. & Úsh vektordıń aralas kóbeymesi ushın $(abc)=0$ teńligi orınlı bolsa ne dep ataladı? & $\overline{a}$, $\overline{b}$ hám $\overline{c}$ vektorları komplanar \\
\hline
5. & $(2, 3)$ hám $(4, 3)$ noqatlarınan ótiwshi tuwrı sızıqtıń teńlemesin dúziń. & $ y-3=0$ \\
\hline
6. & $x^{2}+y^{2}-2x+4y=0$ sheńberdiń teńlemesin kanonikalıq túrdegi teńlemege alıp keliń. & $(x-1)^{2}+(y+2)^{2}=5$ \\
\hline
7. & $A(4, 3), B(7, 7)$ noqatları arasındaǵı aralıqtı tabıń. & $d(AB)=5$ \\
\hline
8. & $3x^{2}+10xy+3y^{2}-2x-14y-13=0$ teńlemesiniń tipin anıqlań. & giperbola \\
\hline
9. & $x^{2}-4y^{2}+6x+5=0$ giperbolanıń kanonikalıq teńlemesin dúziń. & $\frac{(x+3)^{2}}{4}-\frac{y^{2}}{1}=1$ \\
\hline
10. & $M_{1}M_{2}$ kesindiniń ortasınıń koordinatalarınıń tabıń, eger $M_{1} (2, 3), M_{2} (4, 7)$ bolsa. & $(3,5)$ \\
\hline
\end{tabular}

\vspace{1cm}

\begin{tabular}{lll}
Tuwrı juwaplar sanı: \underline{\hspace{1.5cm}} & 
Bahası: \underline{\hspace{1.5cm}} & 
Imtixan alıwshınıń qolı: \underline{\hspace{2cm}} \\
\end{tabular}

\egroup

\newpage


\textbf{65-variant}\\

\bgroup
\def\arraystretch{1.6} % 1 is the default, change whatever you need

\begin{tabular}{|m{5.7cm}|m{9.5cm}|}
\hline
Familiyası hám atı & \\
\hline
Fakulteti  & \\
\hline
Toparı hám tálim baǵdarı  & \\
\hline
\end{tabular}

\vspace{1cm}

\begin{tabular}{|m{0.7cm}|m{10cm}|m{4cm}|}
\hline
№ & Soraw & Juwap \\
\hline
1. & $A_1x+B_1y+C_1z+D_1=0$ hám $Ax_2+By_2+Cz_2+D_2=0$ tegislikleri ústpe-úst túsiwi shárti? & $\frac{A_1}{A_2}=\frac{B_1}{B_2}=\frac{C_1}{C_2}=\frac{D_1}{D_2}$ \\
\hline
2. & Eki vektordıń skalyar kóbeymesiniń formulası? & $(ab)=|a||b|\cos\varphi$ \\
\hline
3. & $A_1x+B_1y+C_1z+D_1=0$ hám $Ax_2+By_2+Cz_2+D_2=0$ tegislikleri parallel bolıwı shárti & $\frac{A_1}{A_2}=\frac{B_1}{B_2}=\frac{C_1}{C_2}$ \\
\hline
4. & $Ax+C=0$ tuwrı sızıqtıń grafigi koordinata kósherlerine salıstırǵanda qanday jaylasqan? & $OY$ kósherine parallel \\
\hline
5. & $x+y-3=0$ hám $2x+3y-8=0$ tuwrıları óz-ara qanday jaylasqan? & kesilisedi \\
\hline
6. & $x^{2}+y^{2}-2x+4y-20=0$ sheńberdiń $C$ orayın hám $R$ radiusın tabıń. & $C(1;-2), R=5$ \\
\hline
7. & $(x+1)^{2}+(y-2) ^{2}+(z+2) ^{2}=49$ sferanıń orayınıń koordinataların tabıń. & $(-1,2,-2)$ \\
\hline
8. & Eger $2a=16, e=\frac{5}{4}$ bolsa, fokusı abscissa kósherinde, koordinata basına salıstırǵanda simmetriyalıq jaylasqan giperbolanıń teńlemesin dúziń. & $\frac{x^{2}}{64}-\frac{y^{2}}{36}=1$ \\
\hline
9. & Eger $2b=24, 2 c=10$ bolsa, onda abscissa kósherinde koordinata basına salıstırǵanda simmetriyalıq jaylasqan fokuslarǵa iye, ellipstiń teńlemesin dúziń. & $\frac{x^{2}}{169}+\frac{y^{2}}{144}=1$ \\
\hline
10. & $M_{1} (12;-1)$ hám $M_{2} (0;4)$ noqatlardıń arasındaǵı aralıqtı tabıń. & $13$ \\
\hline
\end{tabular}

\vspace{1cm}

\begin{tabular}{lll}
Tuwrı juwaplar sanı: \underline{\hspace{1.5cm}} & 
Bahası: \underline{\hspace{1.5cm}} & 
Imtixan alıwshınıń qolı: \underline{\hspace{2cm}} \\
\end{tabular}

\egroup

\newpage


\textbf{66-variant}\\

\bgroup
\def\arraystretch{1.6} % 1 is the default, change whatever you need

\begin{tabular}{|m{5.7cm}|m{9.5cm}|}
\hline
Familiyası hám atı & \\
\hline
Fakulteti  & \\
\hline
Toparı hám tálim baǵdarı  & \\
\hline
\end{tabular}

\vspace{1cm}

\begin{tabular}{|m{0.7cm}|m{10cm}|m{4cm}|}
\hline
№ & Soraw & Juwap \\
\hline
1. & Eki vektor qashan kollinear dep ataladı? & bir tuwrıda yamasa parallel tuwrıda jaylasqan bolsa \\
\hline
2. & Tuwrı múyeshli koordinatalar sisteması dep nege aytamız? & Masshtab birlikleri berilgen o'zara perpendikulyar $OX$ hám $OY$ kósherleri \\
\hline
3. & $OXY$ tegisliginiń teńlemesi? & $z=0$ \\
\hline
4. & Giperbolanıń kanonikalıq teńlemesi? & $\frac{x^2}{a^2}-\frac{y^2}{b^2}=1$ \\
\hline
5. & $x+y=0$ teńlemesi menen berilgen tuwrı sızıqtıń múyeshlik koefficientin anıqlań. & $- 1$ \\
\hline
6. & Orayı $C (-1;2)$ noqatında, $A (-2;6 )$ noqatınan ótetuǵın sheńberdiń teńlemesin dúziń. & $(x+1)^{2}+(y-2)^{2}=17$ \\
\hline
7. & $x+2=0$ keńislik qanday geometriyalıq betlikti anıqlaydı? &  $OYZ$ tegisligine parallel bolǵan tegislikti \\
\hline
8. & $\frac{x^{2}}{225}-\frac{y^{2}}{64}=-1$ giperbola fokusınıń koordinatalarınıń tabıń. & $F_{1}(0;-17), F_{2}(0;17)$ \\
\hline
9. & $9x^{2}+25y^{2}=225$ ellipsi berilgen, ellipstiń fokusların, ekscentrisitetin tabıń. & $F_1\left(-4;0 \right) , F_2\left( 4;0 \right) , e = \frac{4}{5}$ \\
\hline
10. & $A (-1;0;1),\ B (1;-1;0)$ noqatları berilgen. $\overline{BA}$ vektorın tabıń. & $\left\{ - 2;1;1 \right\}$ \\
\hline
\end{tabular}

\vspace{1cm}

\begin{tabular}{lll}
Tuwrı juwaplar sanı: \underline{\hspace{1.5cm}} & 
Bahası: \underline{\hspace{1.5cm}} & 
Imtixan alıwshınıń qolı: \underline{\hspace{2cm}} \\
\end{tabular}

\egroup

\newpage


\textbf{67-variant}\\

\bgroup
\def\arraystretch{1.6} % 1 is the default, change whatever you need

\begin{tabular}{|m{5.7cm}|m{9.5cm}|}
\hline
Familiyası hám atı & \\
\hline
Fakulteti  & \\
\hline
Toparı hám tálim baǵdarı  & \\
\hline
\end{tabular}

\vspace{1cm}

\begin{tabular}{|m{0.7cm}|m{10cm}|m{4cm}|}
\hline
№ & Soraw & Juwap \\
\hline
1. & Eki vektordıń vektor kóbeymesiniń uzınlıǵın tabıw formulası? & $\left| \lbrack ab\rbrack \right|=|a||b|\sin\varphi$ \\
\hline
2. & Tegislikdegi qálegen noqatınan berilgen eki noqatqa shekemgi bolǵan aralıqlardıń ayırmasınıń modulı ózgermeytuǵın bolǵan noqatlardıń geometriyalıq ornı ne dep ataladı? & giperbola \\
\hline
3. & Eki tuwrı sızıq arasındaǵı múyeshti tabıw formulası? & $\text{tg}\varphi=\frac{k_2-k_1}{1+k_1k_2}$ \\
\hline
4. & $\frac{x^2}{a^2}-\frac{y^2}{b^2}=1$ giperbolanıń $(x_0;y_0)$ noqatındaǵı urınbasınıń teńlemesin kórsetiń. & $\frac{x_0x}{a^2}-\frac{y_0y}{b^2}=1$ \\
\hline
5. & $2x+3y+4=0$ tuwrısına parallel hám $M_{0} (2;1)$ noqattan ótetuǵın tuwrınıń teńlemesin dúziń. & $2x+3y-7=0$ \\
\hline
6. & $x+y-12=0$ tuwrısı $x^{2}+y^{2}-2y=0$ sheńberge salıstırǵanda qanday jaylasqan? & sırtında jaylasqan \\
\hline
7. & $\left| \overline{a} \right|=8, \left| \overline{b} \right|=5, \alpha=60^{0}$ bolsa, $( \overline{a}\overline{b} )$ ni tabıń. & $20$ \\
\hline
8. & $2x+3y-6=0$ tuwrınıń teńlemesin kesindilerde berilgen teńleme túrinde kórsetiń. & $\frac{x}{3} + \frac{ y }{ 2 } =  1$ \\
\hline
9. & $\overline{a}=\left\{ 4,-2,-4 \right\}$ hám $\overline{b}=\left\{ 6,-3, 2 \right\}$ vektorları berilgen, $(\overline{a}-\overline{b}) ^{2}$-? & $41$ \\
\hline
10. & $5x-y+7=0$ hám $3x+2y=0$ tuwrıları arasındaǵı múyeshni tabıń. & $\varphi=\frac{\pi}{4}$ \\
\hline
\end{tabular}

\vspace{1cm}

\begin{tabular}{lll}
Tuwrı juwaplar sanı: \underline{\hspace{1.5cm}} & 
Bahası: \underline{\hspace{1.5cm}} & 
Imtixan alıwshınıń qolı: \underline{\hspace{2cm}} \\
\end{tabular}

\egroup

\newpage


\textbf{68-variant}\\

\bgroup
\def\arraystretch{1.6} % 1 is the default, change whatever you need

\begin{tabular}{|m{5.7cm}|m{9.5cm}|}
\hline
Familiyası hám atı & \\
\hline
Fakulteti  & \\
\hline
Toparı hám tálim baǵdarı  & \\
\hline
\end{tabular}

\vspace{1cm}

\begin{tabular}{|m{0.7cm}|m{10cm}|m{4cm}|}
\hline
№ & Soraw & Juwap \\
\hline
1. & Vektorlardıń kósherdegi proekciyasınıń formulası? & $x=|a|\cos\varphi, y=|a|\sin\varphi$ \\
\hline
2. & $Ax+By+D=0$ teńlemesi arqalı ... tegisliktiń teńlemesi berilgen? & $OZ$ kósherine parallel \\
\hline
3. & $\frac{x^2}{a^2}+\frac{y^2}{b^2}=1$ ellipstiń $(x_0;y_0)$ noqatındaǵı urınbasınıń teńlemesin tabıń. & $\frac{x_0x}{a^2}+\frac{y_0y}{b^2}=1$ \\
\hline
4. & Vektorlardı qosıw koordinatalarda qanday formula menen anıqlanadı? & $\overline{a}+\overline{b}=\{x_1+x_2;y_1+y_2\}$ \\
\hline
5. & $\overline{a}=\left\{ 2, 1, 0 \right\}$ hám $\overline{b}=\left\{ 1, 0,-1 \right\}$ bolsa, $\overline{a}-\overline{b}$ ni tabıń. & $\overline{a} -\overline{b} = \left\{ 1,1,1 \right\}$ \\
\hline
6. & Koordinatalar kósherleri hám $ 3x+4y-12=0 $ tuwrı sızıǵı menen shegaralanǵan úshmúyeshliktiń maydanın tabıń. & $ S=6 $ \\
\hline
7. & $x-2y+1=0$ teńlemesi menen berilgen tuwrınıń normal túrdegi teńlemesin kórsetiń. & $\frac{x}{- \sqrt{5}}+\frac{2y}{\sqrt{5}}-\frac{1}{\sqrt{5}}=0$ \\
\hline
8. & $3x-y+5=0$, $x+3y-4=0$ tuwrı sızıqları arasındaǵı múyeshti tabıń. & $90^{0}$ \\
\hline
9. & $\overline{a}=\{5,-6, 1 \}, \overline{b}=\{-4, 3, 0 \} $, $\overline{c}=\left\{ 5,-8, 10 \right\}$ vektorları berilgen. $2{\overline{a}}^{2}+4{\overline{b}}^{2}-5{\overline{c}}^{2}$ ańlatpasınıń mánisin tabıń. & $-721$ \\
\hline
10. & $(2, 3)$ hám $(4, 3)$ noqatlarınan ótiwshi tuwrı sızıqtıń teńlemesin dúziń. & $ y-3=0$ \\
\hline
\end{tabular}

\vspace{1cm}

\begin{tabular}{lll}
Tuwrı juwaplar sanı: \underline{\hspace{1.5cm}} & 
Bahası: \underline{\hspace{1.5cm}} & 
Imtixan alıwshınıń qolı: \underline{\hspace{2cm}} \\
\end{tabular}

\egroup

\newpage


\textbf{69-variant}\\

\bgroup
\def\arraystretch{1.6} % 1 is the default, change whatever you need

\begin{tabular}{|m{5.7cm}|m{9.5cm}|}
\hline
Familiyası hám atı & \\
\hline
Fakulteti  & \\
\hline
Toparı hám tálim baǵdarı  & \\
\hline
\end{tabular}

\vspace{1cm}

\begin{tabular}{|m{0.7cm}|m{10cm}|m{4cm}|}
\hline
№ & Soraw & Juwap \\
\hline
1. & $OY$ kósheriniń teńlemesi? & $x=0$ \\
\hline
2. & Egerde $a=\{ x_1; y_1; z_1\}, b=\{ x_2, y_2; z_2\}$ bolsa, vektor kóbeymeniń koordinatalarda ańlatılıwı qanday boladı? &  $\lbrack ab\rbrack=\{y_1z_2-y_2z_1; z_1x_2-z_2x_1; x_1y_2-x_2y_1\}$ \\
\hline
3. & $A_1x+B_1y+C_1z+D_1=0$ hám $Ax_2+By_2+Cz_2+D_2=0$ tegislikleri perpendikulyar bolıwı shárti & $A_1\cdot A_2+B_1\cdot B_2+C_1\cdot C_2=0$ \\
\hline
4. & Úsh vektordıń aralas kóbeymesi ushın $(abc)=0$ teńligi orınlı bolsa ne dep ataladı? & $\overline{a}$, $\overline{b}$ hám $\overline{c}$ vektorları komplanar \\
\hline
5. & $x^{2}+y^{2}-2x+4y=0$ sheńberdiń teńlemesin kanonikalıq túrdegi teńlemege alıp keliń. & $(x-1)^{2}+(y+2)^{2}=5$ \\
\hline
6. & $A(4, 3), B(7, 7)$ noqatları arasındaǵı aralıqtı tabıń. & $d(AB)=5$ \\
\hline
7. & $3x^{2}+10xy+3y^{2}-2x-14y-13=0$ teńlemesiniń tipin anıqlań. & giperbola \\
\hline
8. & $x^{2}-4y^{2}+6x+5=0$ giperbolanıń kanonikalıq teńlemesin dúziń. & $\frac{(x+3)^{2}}{4}-\frac{y^{2}}{1}=1$ \\
\hline
9. & $M_{1}M_{2}$ kesindiniń ortasınıń koordinatalarınıń tabıń, eger $M_{1} (2, 3), M_{2} (4, 7)$ bolsa. & $(3,5)$ \\
\hline
10. & $x+y-3=0$ hám $2x+3y-8=0$ tuwrıları óz-ara qanday jaylasqan? & kesilisedi \\
\hline
\end{tabular}

\vspace{1cm}

\begin{tabular}{lll}
Tuwrı juwaplar sanı: \underline{\hspace{1.5cm}} & 
Bahası: \underline{\hspace{1.5cm}} & 
Imtixan alıwshınıń qolı: \underline{\hspace{2cm}} \\
\end{tabular}

\egroup

\newpage


\textbf{70-variant}\\

\bgroup
\def\arraystretch{1.6} % 1 is the default, change whatever you need

\begin{tabular}{|m{5.7cm}|m{9.5cm}|}
\hline
Familiyası hám atı & \\
\hline
Fakulteti  & \\
\hline
Toparı hám tálim baǵdarı  & \\
\hline
\end{tabular}

\vspace{1cm}

\begin{tabular}{|m{0.7cm}|m{10cm}|m{4cm}|}
\hline
№ & Soraw & Juwap \\
\hline
1. & $A_1x+B_1y+C_1z+D_1=0$ hám $Ax_2+By_2+Cz_2+D_2=0$ tegislikleri ústpe-úst túsiwi shárti? & $\frac{A_1}{A_2}=\frac{B_1}{B_2}=\frac{C_1}{C_2}=\frac{D_1}{D_2}$ \\
\hline
2. & Eki vektordıń skalyar kóbeymesiniń formulası? & $(ab)=|a||b|\cos\varphi$ \\
\hline
3. & $A_1x+B_1y+C_1z+D_1=0$ hám $Ax_2+By_2+Cz_2+D_2=0$ tegislikleri parallel bolıwı shárti & $\frac{A_1}{A_2}=\frac{B_1}{B_2}=\frac{C_1}{C_2}$ \\
\hline
4. & $Ax+C=0$ tuwrı sızıqtıń grafigi koordinata kósherlerine salıstırǵanda qanday jaylasqan? & $OY$ kósherine parallel \\
\hline
5. & $x^{2}+y^{2}-2x+4y-20=0$ sheńberdiń $C$ orayın hám $R$ radiusın tabıń. & $C(1;-2), R=5$ \\
\hline
6. & $(x+1)^{2}+(y-2) ^{2}+(z+2) ^{2}=49$ sferanıń orayınıń koordinataların tabıń. & $(-1,2,-2)$ \\
\hline
7. & Eger $2a=16, e=\frac{5}{4}$ bolsa, fokusı abscissa kósherinde, koordinata basına salıstırǵanda simmetriyalıq jaylasqan giperbolanıń teńlemesin dúziń. & $\frac{x^{2}}{64}-\frac{y^{2}}{36}=1$ \\
\hline
8. & Eger $2b=24, 2 c=10$ bolsa, onda abscissa kósherinde koordinata basına salıstırǵanda simmetriyalıq jaylasqan fokuslarǵa iye, ellipstiń teńlemesin dúziń. & $\frac{x^{2}}{169}+\frac{y^{2}}{144}=1$ \\
\hline
9. & $M_{1} (12;-1)$ hám $M_{2} (0;4)$ noqatlardıń arasındaǵı aralıqtı tabıń. & $13$ \\
\hline
10. & $x+y=0$ teńlemesi menen berilgen tuwrı sızıqtıń múyeshlik koefficientin anıqlań. & $- 1$ \\
\hline
\end{tabular}

\vspace{1cm}

\begin{tabular}{lll}
Tuwrı juwaplar sanı: \underline{\hspace{1.5cm}} & 
Bahası: \underline{\hspace{1.5cm}} & 
Imtixan alıwshınıń qolı: \underline{\hspace{2cm}} \\
\end{tabular}

\egroup

\newpage


\textbf{71-variant}\\

\bgroup
\def\arraystretch{1.6} % 1 is the default, change whatever you need

\begin{tabular}{|m{5.7cm}|m{9.5cm}|}
\hline
Familiyası hám atı & \\
\hline
Fakulteti  & \\
\hline
Toparı hám tálim baǵdarı  & \\
\hline
\end{tabular}

\vspace{1cm}

\begin{tabular}{|m{0.7cm}|m{10cm}|m{4cm}|}
\hline
№ & Soraw & Juwap \\
\hline
1. & Eki vektor qashan kollinear dep ataladı? & bir tuwrıda yamasa parallel tuwrıda jaylasqan bolsa \\
\hline
2. & Tuwrı múyeshli koordinatalar sisteması dep nege aytamız? & Masshtab birlikleri berilgen o'zara perpendikulyar $OX$ hám $OY$ kósherleri \\
\hline
3. & $OXY$ tegisliginiń teńlemesi? & $z=0$ \\
\hline
4. & Giperbolanıń kanonikalıq teńlemesi? & $\frac{x^2}{a^2}-\frac{y^2}{b^2}=1$ \\
\hline
5. & Orayı $C (-1;2)$ noqatında, $A (-2;6 )$ noqatınan ótetuǵın sheńberdiń teńlemesin dúziń. & $(x+1)^{2}+(y-2)^{2}=17$ \\
\hline
6. & $x+2=0$ keńislik qanday geometriyalıq betlikti anıqlaydı? &  $OYZ$ tegisligine parallel bolǵan tegislikti \\
\hline
7. & $\frac{x^{2}}{225}-\frac{y^{2}}{64}=-1$ giperbola fokusınıń koordinatalarınıń tabıń. & $F_{1}(0;-17), F_{2}(0;17)$ \\
\hline
8. & $9x^{2}+25y^{2}=225$ ellipsi berilgen, ellipstiń fokusların, ekscentrisitetin tabıń. & $F_1\left(-4;0 \right) , F_2\left( 4;0 \right) , e = \frac{4}{5}$ \\
\hline
9. & $A (-1;0;1),\ B (1;-1;0)$ noqatları berilgen. $\overline{BA}$ vektorın tabıń. & $\left\{ - 2;1;1 \right\}$ \\
\hline
10. & $2x+3y+4=0$ tuwrısına parallel hám $M_{0} (2;1)$ noqattan ótetuǵın tuwrınıń teńlemesin dúziń. & $2x+3y-7=0$ \\
\hline
\end{tabular}

\vspace{1cm}

\begin{tabular}{lll}
Tuwrı juwaplar sanı: \underline{\hspace{1.5cm}} & 
Bahası: \underline{\hspace{1.5cm}} & 
Imtixan alıwshınıń qolı: \underline{\hspace{2cm}} \\
\end{tabular}

\egroup

\newpage


\textbf{72-variant}\\

\bgroup
\def\arraystretch{1.6} % 1 is the default, change whatever you need

\begin{tabular}{|m{5.7cm}|m{9.5cm}|}
\hline
Familiyası hám atı & \\
\hline
Fakulteti  & \\
\hline
Toparı hám tálim baǵdarı  & \\
\hline
\end{tabular}

\vspace{1cm}

\begin{tabular}{|m{0.7cm}|m{10cm}|m{4cm}|}
\hline
№ & Soraw & Juwap \\
\hline
1. & Eki vektordıń vektor kóbeymesiniń uzınlıǵın tabıw formulası? & $\left| \lbrack ab\rbrack \right|=|a||b|\sin\varphi$ \\
\hline
2. & Tegislikdegi qálegen noqatınan berilgen eki noqatqa shekemgi bolǵan aralıqlardıń ayırmasınıń modulı ózgermeytuǵın bolǵan noqatlardıń geometriyalıq ornı ne dep ataladı? & giperbola \\
\hline
3. & Eki tuwrı sızıq arasındaǵı múyeshti tabıw formulası? & $\text{tg}\varphi=\frac{k_2-k_1}{1+k_1k_2}$ \\
\hline
4. & $\frac{x^2}{a^2}-\frac{y^2}{b^2}=1$ giperbolanıń $(x_0;y_0)$ noqatındaǵı urınbasınıń teńlemesin kórsetiń. & $\frac{x_0x}{a^2}-\frac{y_0y}{b^2}=1$ \\
\hline
5. & $x+y-12=0$ tuwrısı $x^{2}+y^{2}-2y=0$ sheńberge salıstırǵanda qanday jaylasqan? & sırtında jaylasqan \\
\hline
6. & $\left| \overline{a} \right|=8, \left| \overline{b} \right|=5, \alpha=60^{0}$ bolsa, $( \overline{a}\overline{b} )$ ni tabıń. & $20$ \\
\hline
7. & $2x+3y-6=0$ tuwrınıń teńlemesin kesindilerde berilgen teńleme túrinde kórsetiń. & $\frac{x}{3} + \frac{ y }{ 2 } =  1$ \\
\hline
8. & $\overline{a}=\left\{ 4,-2,-4 \right\}$ hám $\overline{b}=\left\{ 6,-3, 2 \right\}$ vektorları berilgen, $(\overline{a}-\overline{b}) ^{2}$-? & $41$ \\
\hline
9. & $5x-y+7=0$ hám $3x+2y=0$ tuwrıları arasındaǵı múyeshni tabıń. & $\varphi=\frac{\pi}{4}$ \\
\hline
10. & $\overline{a}=\left\{ 2, 1, 0 \right\}$ hám $\overline{b}=\left\{ 1, 0,-1 \right\}$ bolsa, $\overline{a}-\overline{b}$ ni tabıń. & $\overline{a} -\overline{b} = \left\{ 1,1,1 \right\}$ \\
\hline
\end{tabular}

\vspace{1cm}

\begin{tabular}{lll}
Tuwrı juwaplar sanı: \underline{\hspace{1.5cm}} & 
Bahası: \underline{\hspace{1.5cm}} & 
Imtixan alıwshınıń qolı: \underline{\hspace{2cm}} \\
\end{tabular}

\egroup

\newpage


\textbf{73-variant}\\

\bgroup
\def\arraystretch{1.6} % 1 is the default, change whatever you need

\begin{tabular}{|m{5.7cm}|m{9.5cm}|}
\hline
Familiyası hám atı & \\
\hline
Fakulteti  & \\
\hline
Toparı hám tálim baǵdarı  & \\
\hline
\end{tabular}

\vspace{1cm}

\begin{tabular}{|m{0.7cm}|m{10cm}|m{4cm}|}
\hline
№ & Soraw & Juwap \\
\hline
1. & Vektorlardıń kósherdegi proekciyasınıń formulası? & $x=|a|\cos\varphi, y=|a|\sin\varphi$ \\
\hline
2. & $Ax+By+D=0$ teńlemesi arqalı ... tegisliktiń teńlemesi berilgen? & $OZ$ kósherine parallel \\
\hline
3. & $\frac{x^2}{a^2}+\frac{y^2}{b^2}=1$ ellipstiń $(x_0;y_0)$ noqatındaǵı urınbasınıń teńlemesin tabıń. & $\frac{x_0x}{a^2}+\frac{y_0y}{b^2}=1$ \\
\hline
4. & Vektorlardı qosıw koordinatalarda qanday formula menen anıqlanadı? & $\overline{a}+\overline{b}=\{x_1+x_2;y_1+y_2\}$ \\
\hline
5. & Koordinatalar kósherleri hám $ 3x+4y-12=0 $ tuwrı sızıǵı menen shegaralanǵan úshmúyeshliktiń maydanın tabıń. & $ S=6 $ \\
\hline
6. & $x-2y+1=0$ teńlemesi menen berilgen tuwrınıń normal túrdegi teńlemesin kórsetiń. & $\frac{x}{- \sqrt{5}}+\frac{2y}{\sqrt{5}}-\frac{1}{\sqrt{5}}=0$ \\
\hline
7. & $3x-y+5=0$, $x+3y-4=0$ tuwrı sızıqları arasındaǵı múyeshti tabıń. & $90^{0}$ \\
\hline
8. & $\overline{a}=\{5,-6, 1 \}, \overline{b}=\{-4, 3, 0 \} $, $\overline{c}=\left\{ 5,-8, 10 \right\}$ vektorları berilgen. $2{\overline{a}}^{2}+4{\overline{b}}^{2}-5{\overline{c}}^{2}$ ańlatpasınıń mánisin tabıń. & $-721$ \\
\hline
9. & $(2, 3)$ hám $(4, 3)$ noqatlarınan ótiwshi tuwrı sızıqtıń teńlemesin dúziń. & $ y-3=0$ \\
\hline
10. & $x^{2}+y^{2}-2x+4y=0$ sheńberdiń teńlemesin kanonikalıq túrdegi teńlemege alıp keliń. & $(x-1)^{2}+(y+2)^{2}=5$ \\
\hline
\end{tabular}

\vspace{1cm}

\begin{tabular}{lll}
Tuwrı juwaplar sanı: \underline{\hspace{1.5cm}} & 
Bahası: \underline{\hspace{1.5cm}} & 
Imtixan alıwshınıń qolı: \underline{\hspace{2cm}} \\
\end{tabular}

\egroup

\newpage


\textbf{74-variant}\\

\bgroup
\def\arraystretch{1.6} % 1 is the default, change whatever you need

\begin{tabular}{|m{5.7cm}|m{9.5cm}|}
\hline
Familiyası hám atı & \\
\hline
Fakulteti  & \\
\hline
Toparı hám tálim baǵdarı  & \\
\hline
\end{tabular}

\vspace{1cm}

\begin{tabular}{|m{0.7cm}|m{10cm}|m{4cm}|}
\hline
№ & Soraw & Juwap \\
\hline
1. & $OY$ kósheriniń teńlemesi? & $x=0$ \\
\hline
2. & Egerde $a=\{ x_1; y_1; z_1\}, b=\{ x_2, y_2; z_2\}$ bolsa, vektor kóbeymeniń koordinatalarda ańlatılıwı qanday boladı? &  $\lbrack ab\rbrack=\{y_1z_2-y_2z_1; z_1x_2-z_2x_1; x_1y_2-x_2y_1\}$ \\
\hline
3. & $A_1x+B_1y+C_1z+D_1=0$ hám $Ax_2+By_2+Cz_2+D_2=0$ tegislikleri perpendikulyar bolıwı shárti & $A_1\cdot A_2+B_1\cdot B_2+C_1\cdot C_2=0$ \\
\hline
4. & Úsh vektordıń aralas kóbeymesi ushın $(abc)=0$ teńligi orınlı bolsa ne dep ataladı? & $\overline{a}$, $\overline{b}$ hám $\overline{c}$ vektorları komplanar \\
\hline
5. & $A(4, 3), B(7, 7)$ noqatları arasındaǵı aralıqtı tabıń. & $d(AB)=5$ \\
\hline
6. & $3x^{2}+10xy+3y^{2}-2x-14y-13=0$ teńlemesiniń tipin anıqlań. & giperbola \\
\hline
7. & $x^{2}-4y^{2}+6x+5=0$ giperbolanıń kanonikalıq teńlemesin dúziń. & $\frac{(x+3)^{2}}{4}-\frac{y^{2}}{1}=1$ \\
\hline
8. & $M_{1}M_{2}$ kesindiniń ortasınıń koordinatalarınıń tabıń, eger $M_{1} (2, 3), M_{2} (4, 7)$ bolsa. & $(3,5)$ \\
\hline
9. & $x+y-3=0$ hám $2x+3y-8=0$ tuwrıları óz-ara qanday jaylasqan? & kesilisedi \\
\hline
10. & $x^{2}+y^{2}-2x+4y-20=0$ sheńberdiń $C$ orayın hám $R$ radiusın tabıń. & $C(1;-2), R=5$ \\
\hline
\end{tabular}

\vspace{1cm}

\begin{tabular}{lll}
Tuwrı juwaplar sanı: \underline{\hspace{1.5cm}} & 
Bahası: \underline{\hspace{1.5cm}} & 
Imtixan alıwshınıń qolı: \underline{\hspace{2cm}} \\
\end{tabular}

\egroup

\newpage


\textbf{75-variant}\\

\bgroup
\def\arraystretch{1.6} % 1 is the default, change whatever you need

\begin{tabular}{|m{5.7cm}|m{9.5cm}|}
\hline
Familiyası hám atı & \\
\hline
Fakulteti  & \\
\hline
Toparı hám tálim baǵdarı  & \\
\hline
\end{tabular}

\vspace{1cm}

\begin{tabular}{|m{0.7cm}|m{10cm}|m{4cm}|}
\hline
№ & Soraw & Juwap \\
\hline
1. & $A_1x+B_1y+C_1z+D_1=0$ hám $Ax_2+By_2+Cz_2+D_2=0$ tegislikleri ústpe-úst túsiwi shárti? & $\frac{A_1}{A_2}=\frac{B_1}{B_2}=\frac{C_1}{C_2}=\frac{D_1}{D_2}$ \\
\hline
2. & Eki vektordıń skalyar kóbeymesiniń formulası? & $(ab)=|a||b|\cos\varphi$ \\
\hline
3. & $A_1x+B_1y+C_1z+D_1=0$ hám $Ax_2+By_2+Cz_2+D_2=0$ tegislikleri parallel bolıwı shárti & $\frac{A_1}{A_2}=\frac{B_1}{B_2}=\frac{C_1}{C_2}$ \\
\hline
4. & $Ax+C=0$ tuwrı sızıqtıń grafigi koordinata kósherlerine salıstırǵanda qanday jaylasqan? & $OY$ kósherine parallel \\
\hline
5. & $(x+1)^{2}+(y-2) ^{2}+(z+2) ^{2}=49$ sferanıń orayınıń koordinataların tabıń. & $(-1,2,-2)$ \\
\hline
6. & Eger $2a=16, e=\frac{5}{4}$ bolsa, fokusı abscissa kósherinde, koordinata basına salıstırǵanda simmetriyalıq jaylasqan giperbolanıń teńlemesin dúziń. & $\frac{x^{2}}{64}-\frac{y^{2}}{36}=1$ \\
\hline
7. & Eger $2b=24, 2 c=10$ bolsa, onda abscissa kósherinde koordinata basına salıstırǵanda simmetriyalıq jaylasqan fokuslarǵa iye, ellipstiń teńlemesin dúziń. & $\frac{x^{2}}{169}+\frac{y^{2}}{144}=1$ \\
\hline
8. & $M_{1} (12;-1)$ hám $M_{2} (0;4)$ noqatlardıń arasındaǵı aralıqtı tabıń. & $13$ \\
\hline
9. & $x+y=0$ teńlemesi menen berilgen tuwrı sızıqtıń múyeshlik koefficientin anıqlań. & $- 1$ \\
\hline
10. & Orayı $C (-1;2)$ noqatında, $A (-2;6 )$ noqatınan ótetuǵın sheńberdiń teńlemesin dúziń. & $(x+1)^{2}+(y-2)^{2}=17$ \\
\hline
\end{tabular}

\vspace{1cm}

\begin{tabular}{lll}
Tuwrı juwaplar sanı: \underline{\hspace{1.5cm}} & 
Bahası: \underline{\hspace{1.5cm}} & 
Imtixan alıwshınıń qolı: \underline{\hspace{2cm}} \\
\end{tabular}

\egroup

\newpage


\textbf{76-variant}\\

\bgroup
\def\arraystretch{1.6} % 1 is the default, change whatever you need

\begin{tabular}{|m{5.7cm}|m{9.5cm}|}
\hline
Familiyası hám atı & \\
\hline
Fakulteti  & \\
\hline
Toparı hám tálim baǵdarı  & \\
\hline
\end{tabular}

\vspace{1cm}

\begin{tabular}{|m{0.7cm}|m{10cm}|m{4cm}|}
\hline
№ & Soraw & Juwap \\
\hline
1. & Eki vektor qashan kollinear dep ataladı? & bir tuwrıda yamasa parallel tuwrıda jaylasqan bolsa \\
\hline
2. & Tuwrı múyeshli koordinatalar sisteması dep nege aytamız? & Masshtab birlikleri berilgen o'zara perpendikulyar $OX$ hám $OY$ kósherleri \\
\hline
3. & $OXY$ tegisliginiń teńlemesi? & $z=0$ \\
\hline
4. & Giperbolanıń kanonikalıq teńlemesi? & $\frac{x^2}{a^2}-\frac{y^2}{b^2}=1$ \\
\hline
5. & $x+2=0$ keńislik qanday geometriyalıq betlikti anıqlaydı? &  $OYZ$ tegisligine parallel bolǵan tegislikti \\
\hline
6. & $\frac{x^{2}}{225}-\frac{y^{2}}{64}=-1$ giperbola fokusınıń koordinatalarınıń tabıń. & $F_{1}(0;-17), F_{2}(0;17)$ \\
\hline
7. & $9x^{2}+25y^{2}=225$ ellipsi berilgen, ellipstiń fokusların, ekscentrisitetin tabıń. & $F_1\left(-4;0 \right) , F_2\left( 4;0 \right) , e = \frac{4}{5}$ \\
\hline
8. & $A (-1;0;1),\ B (1;-1;0)$ noqatları berilgen. $\overline{BA}$ vektorın tabıń. & $\left\{ - 2;1;1 \right\}$ \\
\hline
9. & $2x+3y+4=0$ tuwrısına parallel hám $M_{0} (2;1)$ noqattan ótetuǵın tuwrınıń teńlemesin dúziń. & $2x+3y-7=0$ \\
\hline
10. & $x+y-12=0$ tuwrısı $x^{2}+y^{2}-2y=0$ sheńberge salıstırǵanda qanday jaylasqan? & sırtında jaylasqan \\
\hline
\end{tabular}

\vspace{1cm}

\begin{tabular}{lll}
Tuwrı juwaplar sanı: \underline{\hspace{1.5cm}} & 
Bahası: \underline{\hspace{1.5cm}} & 
Imtixan alıwshınıń qolı: \underline{\hspace{2cm}} \\
\end{tabular}

\egroup

\newpage


\textbf{77-variant}\\

\bgroup
\def\arraystretch{1.6} % 1 is the default, change whatever you need

\begin{tabular}{|m{5.7cm}|m{9.5cm}|}
\hline
Familiyası hám atı & \\
\hline
Fakulteti  & \\
\hline
Toparı hám tálim baǵdarı  & \\
\hline
\end{tabular}

\vspace{1cm}

\begin{tabular}{|m{0.7cm}|m{10cm}|m{4cm}|}
\hline
№ & Soraw & Juwap \\
\hline
1. & Eki vektordıń vektor kóbeymesiniń uzınlıǵın tabıw formulası? & $\left| \lbrack ab\rbrack \right|=|a||b|\sin\varphi$ \\
\hline
2. & Tegislikdegi qálegen noqatınan berilgen eki noqatqa shekemgi bolǵan aralıqlardıń ayırmasınıń modulı ózgermeytuǵın bolǵan noqatlardıń geometriyalıq ornı ne dep ataladı? & giperbola \\
\hline
3. & Eki tuwrı sızıq arasındaǵı múyeshti tabıw formulası? & $\text{tg}\varphi=\frac{k_2-k_1}{1+k_1k_2}$ \\
\hline
4. & $\frac{x^2}{a^2}-\frac{y^2}{b^2}=1$ giperbolanıń $(x_0;y_0)$ noqatındaǵı urınbasınıń teńlemesin kórsetiń. & $\frac{x_0x}{a^2}-\frac{y_0y}{b^2}=1$ \\
\hline
5. & $\left| \overline{a} \right|=8, \left| \overline{b} \right|=5, \alpha=60^{0}$ bolsa, $( \overline{a}\overline{b} )$ ni tabıń. & $20$ \\
\hline
6. & $2x+3y-6=0$ tuwrınıń teńlemesin kesindilerde berilgen teńleme túrinde kórsetiń. & $\frac{x}{3} + \frac{ y }{ 2 } =  1$ \\
\hline
7. & $\overline{a}=\left\{ 4,-2,-4 \right\}$ hám $\overline{b}=\left\{ 6,-3, 2 \right\}$ vektorları berilgen, $(\overline{a}-\overline{b}) ^{2}$-? & $41$ \\
\hline
8. & $5x-y+7=0$ hám $3x+2y=0$ tuwrıları arasındaǵı múyeshni tabıń. & $\varphi=\frac{\pi}{4}$ \\
\hline
9. & $\overline{a}=\left\{ 2, 1, 0 \right\}$ hám $\overline{b}=\left\{ 1, 0,-1 \right\}$ bolsa, $\overline{a}-\overline{b}$ ni tabıń. & $\overline{a} -\overline{b} = \left\{ 1,1,1 \right\}$ \\
\hline
10. & Koordinatalar kósherleri hám $ 3x+4y-12=0 $ tuwrı sızıǵı menen shegaralanǵan úshmúyeshliktiń maydanın tabıń. & $ S=6 $ \\
\hline
\end{tabular}

\vspace{1cm}

\begin{tabular}{lll}
Tuwrı juwaplar sanı: \underline{\hspace{1.5cm}} & 
Bahası: \underline{\hspace{1.5cm}} & 
Imtixan alıwshınıń qolı: \underline{\hspace{2cm}} \\
\end{tabular}

\egroup

\newpage


\textbf{78-variant}\\

\bgroup
\def\arraystretch{1.6} % 1 is the default, change whatever you need

\begin{tabular}{|m{5.7cm}|m{9.5cm}|}
\hline
Familiyası hám atı & \\
\hline
Fakulteti  & \\
\hline
Toparı hám tálim baǵdarı  & \\
\hline
\end{tabular}

\vspace{1cm}

\begin{tabular}{|m{0.7cm}|m{10cm}|m{4cm}|}
\hline
№ & Soraw & Juwap \\
\hline
1. & Vektorlardıń kósherdegi proekciyasınıń formulası? & $x=|a|\cos\varphi, y=|a|\sin\varphi$ \\
\hline
2. & $Ax+By+D=0$ teńlemesi arqalı ... tegisliktiń teńlemesi berilgen? & $OZ$ kósherine parallel \\
\hline
3. & $\frac{x^2}{a^2}+\frac{y^2}{b^2}=1$ ellipstiń $(x_0;y_0)$ noqatındaǵı urınbasınıń teńlemesin tabıń. & $\frac{x_0x}{a^2}+\frac{y_0y}{b^2}=1$ \\
\hline
4. & Vektorlardı qosıw koordinatalarda qanday formula menen anıqlanadı? & $\overline{a}+\overline{b}=\{x_1+x_2;y_1+y_2\}$ \\
\hline
5. & $x-2y+1=0$ teńlemesi menen berilgen tuwrınıń normal túrdegi teńlemesin kórsetiń. & $\frac{x}{- \sqrt{5}}+\frac{2y}{\sqrt{5}}-\frac{1}{\sqrt{5}}=0$ \\
\hline
6. & $3x-y+5=0$, $x+3y-4=0$ tuwrı sızıqları arasındaǵı múyeshti tabıń. & $90^{0}$ \\
\hline
7. & $\overline{a}=\{5,-6, 1 \}, \overline{b}=\{-4, 3, 0 \} $, $\overline{c}=\left\{ 5,-8, 10 \right\}$ vektorları berilgen. $2{\overline{a}}^{2}+4{\overline{b}}^{2}-5{\overline{c}}^{2}$ ańlatpasınıń mánisin tabıń. & $-721$ \\
\hline
8. & $(2, 3)$ hám $(4, 3)$ noqatlarınan ótiwshi tuwrı sızıqtıń teńlemesin dúziń. & $ y-3=0$ \\
\hline
9. & $x^{2}+y^{2}-2x+4y=0$ sheńberdiń teńlemesin kanonikalıq túrdegi teńlemege alıp keliń. & $(x-1)^{2}+(y+2)^{2}=5$ \\
\hline
10. & $A(4, 3), B(7, 7)$ noqatları arasındaǵı aralıqtı tabıń. & $d(AB)=5$ \\
\hline
\end{tabular}

\vspace{1cm}

\begin{tabular}{lll}
Tuwrı juwaplar sanı: \underline{\hspace{1.5cm}} & 
Bahası: \underline{\hspace{1.5cm}} & 
Imtixan alıwshınıń qolı: \underline{\hspace{2cm}} \\
\end{tabular}

\egroup

\newpage


\textbf{79-variant}\\

\bgroup
\def\arraystretch{1.6} % 1 is the default, change whatever you need

\begin{tabular}{|m{5.7cm}|m{9.5cm}|}
\hline
Familiyası hám atı & \\
\hline
Fakulteti  & \\
\hline
Toparı hám tálim baǵdarı  & \\
\hline
\end{tabular}

\vspace{1cm}

\begin{tabular}{|m{0.7cm}|m{10cm}|m{4cm}|}
\hline
№ & Soraw & Juwap \\
\hline
1. & $OY$ kósheriniń teńlemesi? & $x=0$ \\
\hline
2. & Egerde $a=\{ x_1; y_1; z_1\}, b=\{ x_2, y_2; z_2\}$ bolsa, vektor kóbeymeniń koordinatalarda ańlatılıwı qanday boladı? &  $\lbrack ab\rbrack=\{y_1z_2-y_2z_1; z_1x_2-z_2x_1; x_1y_2-x_2y_1\}$ \\
\hline
3. & $A_1x+B_1y+C_1z+D_1=0$ hám $Ax_2+By_2+Cz_2+D_2=0$ tegislikleri perpendikulyar bolıwı shárti & $A_1\cdot A_2+B_1\cdot B_2+C_1\cdot C_2=0$ \\
\hline
4. & Úsh vektordıń aralas kóbeymesi ushın $(abc)=0$ teńligi orınlı bolsa ne dep ataladı? & $\overline{a}$, $\overline{b}$ hám $\overline{c}$ vektorları komplanar \\
\hline
5. & $3x^{2}+10xy+3y^{2}-2x-14y-13=0$ teńlemesiniń tipin anıqlań. & giperbola \\
\hline
6. & $x^{2}-4y^{2}+6x+5=0$ giperbolanıń kanonikalıq teńlemesin dúziń. & $\frac{(x+3)^{2}}{4}-\frac{y^{2}}{1}=1$ \\
\hline
7. & $M_{1}M_{2}$ kesindiniń ortasınıń koordinatalarınıń tabıń, eger $M_{1} (2, 3), M_{2} (4, 7)$ bolsa. & $(3,5)$ \\
\hline
8. & $x+y-3=0$ hám $2x+3y-8=0$ tuwrıları óz-ara qanday jaylasqan? & kesilisedi \\
\hline
9. & $x^{2}+y^{2}-2x+4y-20=0$ sheńberdiń $C$ orayın hám $R$ radiusın tabıń. & $C(1;-2), R=5$ \\
\hline
10. & $(x+1)^{2}+(y-2) ^{2}+(z+2) ^{2}=49$ sferanıń orayınıń koordinataların tabıń. & $(-1,2,-2)$ \\
\hline
\end{tabular}

\vspace{1cm}

\begin{tabular}{lll}
Tuwrı juwaplar sanı: \underline{\hspace{1.5cm}} & 
Bahası: \underline{\hspace{1.5cm}} & 
Imtixan alıwshınıń qolı: \underline{\hspace{2cm}} \\
\end{tabular}

\egroup

\newpage


\textbf{80-variant}\\

\bgroup
\def\arraystretch{1.6} % 1 is the default, change whatever you need

\begin{tabular}{|m{5.7cm}|m{9.5cm}|}
\hline
Familiyası hám atı & \\
\hline
Fakulteti  & \\
\hline
Toparı hám tálim baǵdarı  & \\
\hline
\end{tabular}

\vspace{1cm}

\begin{tabular}{|m{0.7cm}|m{10cm}|m{4cm}|}
\hline
№ & Soraw & Juwap \\
\hline
1. & $A_1x+B_1y+C_1z+D_1=0$ hám $Ax_2+By_2+Cz_2+D_2=0$ tegislikleri ústpe-úst túsiwi shárti? & $\frac{A_1}{A_2}=\frac{B_1}{B_2}=\frac{C_1}{C_2}=\frac{D_1}{D_2}$ \\
\hline
2. & Eki vektordıń skalyar kóbeymesiniń formulası? & $(ab)=|a||b|\cos\varphi$ \\
\hline
3. & $A_1x+B_1y+C_1z+D_1=0$ hám $Ax_2+By_2+Cz_2+D_2=0$ tegislikleri parallel bolıwı shárti & $\frac{A_1}{A_2}=\frac{B_1}{B_2}=\frac{C_1}{C_2}$ \\
\hline
4. & $Ax+C=0$ tuwrı sızıqtıń grafigi koordinata kósherlerine salıstırǵanda qanday jaylasqan? & $OY$ kósherine parallel \\
\hline
5. & Eger $2a=16, e=\frac{5}{4}$ bolsa, fokusı abscissa kósherinde, koordinata basına salıstırǵanda simmetriyalıq jaylasqan giperbolanıń teńlemesin dúziń. & $\frac{x^{2}}{64}-\frac{y^{2}}{36}=1$ \\
\hline
6. & Eger $2b=24, 2 c=10$ bolsa, onda abscissa kósherinde koordinata basına salıstırǵanda simmetriyalıq jaylasqan fokuslarǵa iye, ellipstiń teńlemesin dúziń. & $\frac{x^{2}}{169}+\frac{y^{2}}{144}=1$ \\
\hline
7. & $M_{1} (12;-1)$ hám $M_{2} (0;4)$ noqatlardıń arasındaǵı aralıqtı tabıń. & $13$ \\
\hline
8. & $x+y=0$ teńlemesi menen berilgen tuwrı sızıqtıń múyeshlik koefficientin anıqlań. & $- 1$ \\
\hline
9. & Orayı $C (-1;2)$ noqatında, $A (-2;6 )$ noqatınan ótetuǵın sheńberdiń teńlemesin dúziń. & $(x+1)^{2}+(y-2)^{2}=17$ \\
\hline
10. & $x+2=0$ keńislik qanday geometriyalıq betlikti anıqlaydı? &  $OYZ$ tegisligine parallel bolǵan tegislikti \\
\hline
\end{tabular}

\vspace{1cm}

\begin{tabular}{lll}
Tuwrı juwaplar sanı: \underline{\hspace{1.5cm}} & 
Bahası: \underline{\hspace{1.5cm}} & 
Imtixan alıwshınıń qolı: \underline{\hspace{2cm}} \\
\end{tabular}

\egroup

\newpage


\textbf{81-variant}\\

\bgroup
\def\arraystretch{1.6} % 1 is the default, change whatever you need

\begin{tabular}{|m{5.7cm}|m{9.5cm}|}
\hline
Familiyası hám atı & \\
\hline
Fakulteti  & \\
\hline
Toparı hám tálim baǵdarı  & \\
\hline
\end{tabular}

\vspace{1cm}

\begin{tabular}{|m{0.7cm}|m{10cm}|m{4cm}|}
\hline
№ & Soraw & Juwap \\
\hline
1. & Eki vektor qashan kollinear dep ataladı? & bir tuwrıda yamasa parallel tuwrıda jaylasqan bolsa \\
\hline
2. & Tuwrı múyeshli koordinatalar sisteması dep nege aytamız? & Masshtab birlikleri berilgen o'zara perpendikulyar $OX$ hám $OY$ kósherleri \\
\hline
3. & $OXY$ tegisliginiń teńlemesi? & $z=0$ \\
\hline
4. & Giperbolanıń kanonikalıq teńlemesi? & $\frac{x^2}{a^2}-\frac{y^2}{b^2}=1$ \\
\hline
5. & $\frac{x^{2}}{225}-\frac{y^{2}}{64}=-1$ giperbola fokusınıń koordinatalarınıń tabıń. & $F_{1}(0;-17), F_{2}(0;17)$ \\
\hline
6. & $9x^{2}+25y^{2}=225$ ellipsi berilgen, ellipstiń fokusların, ekscentrisitetin tabıń. & $F_1\left(-4;0 \right) , F_2\left( 4;0 \right) , e = \frac{4}{5}$ \\
\hline
7. & $A (-1;0;1),\ B (1;-1;0)$ noqatları berilgen. $\overline{BA}$ vektorın tabıń. & $\left\{ - 2;1;1 \right\}$ \\
\hline
8. & $2x+3y+4=0$ tuwrısına parallel hám $M_{0} (2;1)$ noqattan ótetuǵın tuwrınıń teńlemesin dúziń. & $2x+3y-7=0$ \\
\hline
9. & $x+y-12=0$ tuwrısı $x^{2}+y^{2}-2y=0$ sheńberge salıstırǵanda qanday jaylasqan? & sırtında jaylasqan \\
\hline
10. & $\left| \overline{a} \right|=8, \left| \overline{b} \right|=5, \alpha=60^{0}$ bolsa, $( \overline{a}\overline{b} )$ ni tabıń. & $20$ \\
\hline
\end{tabular}

\vspace{1cm}

\begin{tabular}{lll}
Tuwrı juwaplar sanı: \underline{\hspace{1.5cm}} & 
Bahası: \underline{\hspace{1.5cm}} & 
Imtixan alıwshınıń qolı: \underline{\hspace{2cm}} \\
\end{tabular}

\egroup

\newpage


\textbf{82-variant}\\

\bgroup
\def\arraystretch{1.6} % 1 is the default, change whatever you need

\begin{tabular}{|m{5.7cm}|m{9.5cm}|}
\hline
Familiyası hám atı & \\
\hline
Fakulteti  & \\
\hline
Toparı hám tálim baǵdarı  & \\
\hline
\end{tabular}

\vspace{1cm}

\begin{tabular}{|m{0.7cm}|m{10cm}|m{4cm}|}
\hline
№ & Soraw & Juwap \\
\hline
1. & Eki vektordıń vektor kóbeymesiniń uzınlıǵın tabıw formulası? & $\left| \lbrack ab\rbrack \right|=|a||b|\sin\varphi$ \\
\hline
2. & Tegislikdegi qálegen noqatınan berilgen eki noqatqa shekemgi bolǵan aralıqlardıń ayırmasınıń modulı ózgermeytuǵın bolǵan noqatlardıń geometriyalıq ornı ne dep ataladı? & giperbola \\
\hline
3. & Eki tuwrı sızıq arasındaǵı múyeshti tabıw formulası? & $\text{tg}\varphi=\frac{k_2-k_1}{1+k_1k_2}$ \\
\hline
4. & $\frac{x^2}{a^2}-\frac{y^2}{b^2}=1$ giperbolanıń $(x_0;y_0)$ noqatındaǵı urınbasınıń teńlemesin kórsetiń. & $\frac{x_0x}{a^2}-\frac{y_0y}{b^2}=1$ \\
\hline
5. & $2x+3y-6=0$ tuwrınıń teńlemesin kesindilerde berilgen teńleme túrinde kórsetiń. & $\frac{x}{3} + \frac{ y }{ 2 } =  1$ \\
\hline
6. & $\overline{a}=\left\{ 4,-2,-4 \right\}$ hám $\overline{b}=\left\{ 6,-3, 2 \right\}$ vektorları berilgen, $(\overline{a}-\overline{b}) ^{2}$-? & $41$ \\
\hline
7. & $5x-y+7=0$ hám $3x+2y=0$ tuwrıları arasındaǵı múyeshni tabıń. & $\varphi=\frac{\pi}{4}$ \\
\hline
8. & $\overline{a}=\left\{ 2, 1, 0 \right\}$ hám $\overline{b}=\left\{ 1, 0,-1 \right\}$ bolsa, $\overline{a}-\overline{b}$ ni tabıń. & $\overline{a} -\overline{b} = \left\{ 1,1,1 \right\}$ \\
\hline
9. & Koordinatalar kósherleri hám $ 3x+4y-12=0 $ tuwrı sızıǵı menen shegaralanǵan úshmúyeshliktiń maydanın tabıń. & $ S=6 $ \\
\hline
10. & $x-2y+1=0$ teńlemesi menen berilgen tuwrınıń normal túrdegi teńlemesin kórsetiń. & $\frac{x}{- \sqrt{5}}+\frac{2y}{\sqrt{5}}-\frac{1}{\sqrt{5}}=0$ \\
\hline
\end{tabular}

\vspace{1cm}

\begin{tabular}{lll}
Tuwrı juwaplar sanı: \underline{\hspace{1.5cm}} & 
Bahası: \underline{\hspace{1.5cm}} & 
Imtixan alıwshınıń qolı: \underline{\hspace{2cm}} \\
\end{tabular}

\egroup

\newpage


\textbf{83-variant}\\

\bgroup
\def\arraystretch{1.6} % 1 is the default, change whatever you need

\begin{tabular}{|m{5.7cm}|m{9.5cm}|}
\hline
Familiyası hám atı & \\
\hline
Fakulteti  & \\
\hline
Toparı hám tálim baǵdarı  & \\
\hline
\end{tabular}

\vspace{1cm}

\begin{tabular}{|m{0.7cm}|m{10cm}|m{4cm}|}
\hline
№ & Soraw & Juwap \\
\hline
1. & Vektorlardıń kósherdegi proekciyasınıń formulası? & $x=|a|\cos\varphi, y=|a|\sin\varphi$ \\
\hline
2. & $Ax+By+D=0$ teńlemesi arqalı ... tegisliktiń teńlemesi berilgen? & $OZ$ kósherine parallel \\
\hline
3. & $\frac{x^2}{a^2}+\frac{y^2}{b^2}=1$ ellipstiń $(x_0;y_0)$ noqatındaǵı urınbasınıń teńlemesin tabıń. & $\frac{x_0x}{a^2}+\frac{y_0y}{b^2}=1$ \\
\hline
4. & Vektorlardı qosıw koordinatalarda qanday formula menen anıqlanadı? & $\overline{a}+\overline{b}=\{x_1+x_2;y_1+y_2\}$ \\
\hline
5. & $3x-y+5=0$, $x+3y-4=0$ tuwrı sızıqları arasındaǵı múyeshti tabıń. & $90^{0}$ \\
\hline
6. & $\overline{a}=\{5,-6, 1 \}, \overline{b}=\{-4, 3, 0 \} $, $\overline{c}=\left\{ 5,-8, 10 \right\}$ vektorları berilgen. $2{\overline{a}}^{2}+4{\overline{b}}^{2}-5{\overline{c}}^{2}$ ańlatpasınıń mánisin tabıń. & $-721$ \\
\hline
7. & $(2, 3)$ hám $(4, 3)$ noqatlarınan ótiwshi tuwrı sızıqtıń teńlemesin dúziń. & $ y-3=0$ \\
\hline
8. & $x^{2}+y^{2}-2x+4y=0$ sheńberdiń teńlemesin kanonikalıq túrdegi teńlemege alıp keliń. & $(x-1)^{2}+(y+2)^{2}=5$ \\
\hline
9. & $A(4, 3), B(7, 7)$ noqatları arasındaǵı aralıqtı tabıń. & $d(AB)=5$ \\
\hline
10. & $3x^{2}+10xy+3y^{2}-2x-14y-13=0$ teńlemesiniń tipin anıqlań. & giperbola \\
\hline
\end{tabular}

\vspace{1cm}

\begin{tabular}{lll}
Tuwrı juwaplar sanı: \underline{\hspace{1.5cm}} & 
Bahası: \underline{\hspace{1.5cm}} & 
Imtixan alıwshınıń qolı: \underline{\hspace{2cm}} \\
\end{tabular}

\egroup

\newpage


\textbf{84-variant}\\

\bgroup
\def\arraystretch{1.6} % 1 is the default, change whatever you need

\begin{tabular}{|m{5.7cm}|m{9.5cm}|}
\hline
Familiyası hám atı & \\
\hline
Fakulteti  & \\
\hline
Toparı hám tálim baǵdarı  & \\
\hline
\end{tabular}

\vspace{1cm}

\begin{tabular}{|m{0.7cm}|m{10cm}|m{4cm}|}
\hline
№ & Soraw & Juwap \\
\hline
1. & $OY$ kósheriniń teńlemesi? & $x=0$ \\
\hline
2. & Egerde $a=\{ x_1; y_1; z_1\}, b=\{ x_2, y_2; z_2\}$ bolsa, vektor kóbeymeniń koordinatalarda ańlatılıwı qanday boladı? &  $\lbrack ab\rbrack=\{y_1z_2-y_2z_1; z_1x_2-z_2x_1; x_1y_2-x_2y_1\}$ \\
\hline
3. & $A_1x+B_1y+C_1z+D_1=0$ hám $Ax_2+By_2+Cz_2+D_2=0$ tegislikleri perpendikulyar bolıwı shárti & $A_1\cdot A_2+B_1\cdot B_2+C_1\cdot C_2=0$ \\
\hline
4. & Úsh vektordıń aralas kóbeymesi ushın $(abc)=0$ teńligi orınlı bolsa ne dep ataladı? & $\overline{a}$, $\overline{b}$ hám $\overline{c}$ vektorları komplanar \\
\hline
5. & $x^{2}-4y^{2}+6x+5=0$ giperbolanıń kanonikalıq teńlemesin dúziń. & $\frac{(x+3)^{2}}{4}-\frac{y^{2}}{1}=1$ \\
\hline
6. & $M_{1}M_{2}$ kesindiniń ortasınıń koordinatalarınıń tabıń, eger $M_{1} (2, 3), M_{2} (4, 7)$ bolsa. & $(3,5)$ \\
\hline
7. & $x+y-3=0$ hám $2x+3y-8=0$ tuwrıları óz-ara qanday jaylasqan? & kesilisedi \\
\hline
8. & $x^{2}+y^{2}-2x+4y-20=0$ sheńberdiń $C$ orayın hám $R$ radiusın tabıń. & $C(1;-2), R=5$ \\
\hline
9. & $(x+1)^{2}+(y-2) ^{2}+(z+2) ^{2}=49$ sferanıń orayınıń koordinataların tabıń. & $(-1,2,-2)$ \\
\hline
10. & Eger $2a=16, e=\frac{5}{4}$ bolsa, fokusı abscissa kósherinde, koordinata basına salıstırǵanda simmetriyalıq jaylasqan giperbolanıń teńlemesin dúziń. & $\frac{x^{2}}{64}-\frac{y^{2}}{36}=1$ \\
\hline
\end{tabular}

\vspace{1cm}

\begin{tabular}{lll}
Tuwrı juwaplar sanı: \underline{\hspace{1.5cm}} & 
Bahası: \underline{\hspace{1.5cm}} & 
Imtixan alıwshınıń qolı: \underline{\hspace{2cm}} \\
\end{tabular}

\egroup

\newpage


\textbf{85-variant}\\

\bgroup
\def\arraystretch{1.6} % 1 is the default, change whatever you need

\begin{tabular}{|m{5.7cm}|m{9.5cm}|}
\hline
Familiyası hám atı & \\
\hline
Fakulteti  & \\
\hline
Toparı hám tálim baǵdarı  & \\
\hline
\end{tabular}

\vspace{1cm}

\begin{tabular}{|m{0.7cm}|m{10cm}|m{4cm}|}
\hline
№ & Soraw & Juwap \\
\hline
1. & $A_1x+B_1y+C_1z+D_1=0$ hám $Ax_2+By_2+Cz_2+D_2=0$ tegislikleri ústpe-úst túsiwi shárti? & $\frac{A_1}{A_2}=\frac{B_1}{B_2}=\frac{C_1}{C_2}=\frac{D_1}{D_2}$ \\
\hline
2. & Eki vektordıń skalyar kóbeymesiniń formulası? & $(ab)=|a||b|\cos\varphi$ \\
\hline
3. & $A_1x+B_1y+C_1z+D_1=0$ hám $Ax_2+By_2+Cz_2+D_2=0$ tegislikleri parallel bolıwı shárti & $\frac{A_1}{A_2}=\frac{B_1}{B_2}=\frac{C_1}{C_2}$ \\
\hline
4. & $Ax+C=0$ tuwrı sızıqtıń grafigi koordinata kósherlerine salıstırǵanda qanday jaylasqan? & $OY$ kósherine parallel \\
\hline
5. & Eger $2b=24, 2 c=10$ bolsa, onda abscissa kósherinde koordinata basına salıstırǵanda simmetriyalıq jaylasqan fokuslarǵa iye, ellipstiń teńlemesin dúziń. & $\frac{x^{2}}{169}+\frac{y^{2}}{144}=1$ \\
\hline
6. & $M_{1} (12;-1)$ hám $M_{2} (0;4)$ noqatlardıń arasındaǵı aralıqtı tabıń. & $13$ \\
\hline
7. & $x+y=0$ teńlemesi menen berilgen tuwrı sızıqtıń múyeshlik koefficientin anıqlań. & $- 1$ \\
\hline
8. & Orayı $C (-1;2)$ noqatında, $A (-2;6 )$ noqatınan ótetuǵın sheńberdiń teńlemesin dúziń. & $(x+1)^{2}+(y-2)^{2}=17$ \\
\hline
9. & $x+2=0$ keńislik qanday geometriyalıq betlikti anıqlaydı? &  $OYZ$ tegisligine parallel bolǵan tegislikti \\
\hline
10. & $\frac{x^{2}}{225}-\frac{y^{2}}{64}=-1$ giperbola fokusınıń koordinatalarınıń tabıń. & $F_{1}(0;-17), F_{2}(0;17)$ \\
\hline
\end{tabular}

\vspace{1cm}

\begin{tabular}{lll}
Tuwrı juwaplar sanı: \underline{\hspace{1.5cm}} & 
Bahası: \underline{\hspace{1.5cm}} & 
Imtixan alıwshınıń qolı: \underline{\hspace{2cm}} \\
\end{tabular}

\egroup

\newpage


\textbf{86-variant}\\

\bgroup
\def\arraystretch{1.6} % 1 is the default, change whatever you need

\begin{tabular}{|m{5.7cm}|m{9.5cm}|}
\hline
Familiyası hám atı & \\
\hline
Fakulteti  & \\
\hline
Toparı hám tálim baǵdarı  & \\
\hline
\end{tabular}

\vspace{1cm}

\begin{tabular}{|m{0.7cm}|m{10cm}|m{4cm}|}
\hline
№ & Soraw & Juwap \\
\hline
1. & Eki vektor qashan kollinear dep ataladı? & bir tuwrıda yamasa parallel tuwrıda jaylasqan bolsa \\
\hline
2. & Tuwrı múyeshli koordinatalar sisteması dep nege aytamız? & Masshtab birlikleri berilgen o'zara perpendikulyar $OX$ hám $OY$ kósherleri \\
\hline
3. & $OXY$ tegisliginiń teńlemesi? & $z=0$ \\
\hline
4. & Giperbolanıń kanonikalıq teńlemesi? & $\frac{x^2}{a^2}-\frac{y^2}{b^2}=1$ \\
\hline
5. & $9x^{2}+25y^{2}=225$ ellipsi berilgen, ellipstiń fokusların, ekscentrisitetin tabıń. & $F_1\left(-4;0 \right) , F_2\left( 4;0 \right) , e = \frac{4}{5}$ \\
\hline
6. & $A (-1;0;1),\ B (1;-1;0)$ noqatları berilgen. $\overline{BA}$ vektorın tabıń. & $\left\{ - 2;1;1 \right\}$ \\
\hline
7. & $2x+3y+4=0$ tuwrısına parallel hám $M_{0} (2;1)$ noqattan ótetuǵın tuwrınıń teńlemesin dúziń. & $2x+3y-7=0$ \\
\hline
8. & $x+y-12=0$ tuwrısı $x^{2}+y^{2}-2y=0$ sheńberge salıstırǵanda qanday jaylasqan? & sırtında jaylasqan \\
\hline
9. & $\left| \overline{a} \right|=8, \left| \overline{b} \right|=5, \alpha=60^{0}$ bolsa, $( \overline{a}\overline{b} )$ ni tabıń. & $20$ \\
\hline
10. & $2x+3y-6=0$ tuwrınıń teńlemesin kesindilerde berilgen teńleme túrinde kórsetiń. & $\frac{x}{3} + \frac{ y }{ 2 } =  1$ \\
\hline
\end{tabular}

\vspace{1cm}

\begin{tabular}{lll}
Tuwrı juwaplar sanı: \underline{\hspace{1.5cm}} & 
Bahası: \underline{\hspace{1.5cm}} & 
Imtixan alıwshınıń qolı: \underline{\hspace{2cm}} \\
\end{tabular}

\egroup

\newpage


\textbf{87-variant}\\

\bgroup
\def\arraystretch{1.6} % 1 is the default, change whatever you need

\begin{tabular}{|m{5.7cm}|m{9.5cm}|}
\hline
Familiyası hám atı & \\
\hline
Fakulteti  & \\
\hline
Toparı hám tálim baǵdarı  & \\
\hline
\end{tabular}

\vspace{1cm}

\begin{tabular}{|m{0.7cm}|m{10cm}|m{4cm}|}
\hline
№ & Soraw & Juwap \\
\hline
1. & Eki vektordıń vektor kóbeymesiniń uzınlıǵın tabıw formulası? & $\left| \lbrack ab\rbrack \right|=|a||b|\sin\varphi$ \\
\hline
2. & Tegislikdegi qálegen noqatınan berilgen eki noqatqa shekemgi bolǵan aralıqlardıń ayırmasınıń modulı ózgermeytuǵın bolǵan noqatlardıń geometriyalıq ornı ne dep ataladı? & giperbola \\
\hline
3. & Eki tuwrı sızıq arasındaǵı múyeshti tabıw formulası? & $\text{tg}\varphi=\frac{k_2-k_1}{1+k_1k_2}$ \\
\hline
4. & $\frac{x^2}{a^2}-\frac{y^2}{b^2}=1$ giperbolanıń $(x_0;y_0)$ noqatındaǵı urınbasınıń teńlemesin kórsetiń. & $\frac{x_0x}{a^2}-\frac{y_0y}{b^2}=1$ \\
\hline
5. & $\overline{a}=\left\{ 4,-2,-4 \right\}$ hám $\overline{b}=\left\{ 6,-3, 2 \right\}$ vektorları berilgen, $(\overline{a}-\overline{b}) ^{2}$-? & $41$ \\
\hline
6. & $5x-y+7=0$ hám $3x+2y=0$ tuwrıları arasındaǵı múyeshni tabıń. & $\varphi=\frac{\pi}{4}$ \\
\hline
7. & $\overline{a}=\left\{ 2, 1, 0 \right\}$ hám $\overline{b}=\left\{ 1, 0,-1 \right\}$ bolsa, $\overline{a}-\overline{b}$ ni tabıń. & $\overline{a} -\overline{b} = \left\{ 1,1,1 \right\}$ \\
\hline
8. & Koordinatalar kósherleri hám $ 3x+4y-12=0 $ tuwrı sızıǵı menen shegaralanǵan úshmúyeshliktiń maydanın tabıń. & $ S=6 $ \\
\hline
9. & $x-2y+1=0$ teńlemesi menen berilgen tuwrınıń normal túrdegi teńlemesin kórsetiń. & $\frac{x}{- \sqrt{5}}+\frac{2y}{\sqrt{5}}-\frac{1}{\sqrt{5}}=0$ \\
\hline
10. & $3x-y+5=0$, $x+3y-4=0$ tuwrı sızıqları arasındaǵı múyeshti tabıń. & $90^{0}$ \\
\hline
\end{tabular}

\vspace{1cm}

\begin{tabular}{lll}
Tuwrı juwaplar sanı: \underline{\hspace{1.5cm}} & 
Bahası: \underline{\hspace{1.5cm}} & 
Imtixan alıwshınıń qolı: \underline{\hspace{2cm}} \\
\end{tabular}

\egroup

\newpage


\textbf{88-variant}\\

\bgroup
\def\arraystretch{1.6} % 1 is the default, change whatever you need

\begin{tabular}{|m{5.7cm}|m{9.5cm}|}
\hline
Familiyası hám atı & \\
\hline
Fakulteti  & \\
\hline
Toparı hám tálim baǵdarı  & \\
\hline
\end{tabular}

\vspace{1cm}

\begin{tabular}{|m{0.7cm}|m{10cm}|m{4cm}|}
\hline
№ & Soraw & Juwap \\
\hline
1. & Vektorlardıń kósherdegi proekciyasınıń formulası? & $x=|a|\cos\varphi, y=|a|\sin\varphi$ \\
\hline
2. & $Ax+By+D=0$ teńlemesi arqalı ... tegisliktiń teńlemesi berilgen? & $OZ$ kósherine parallel \\
\hline
3. & $\frac{x^2}{a^2}+\frac{y^2}{b^2}=1$ ellipstiń $(x_0;y_0)$ noqatındaǵı urınbasınıń teńlemesin tabıń. & $\frac{x_0x}{a^2}+\frac{y_0y}{b^2}=1$ \\
\hline
4. & Vektorlardı qosıw koordinatalarda qanday formula menen anıqlanadı? & $\overline{a}+\overline{b}=\{x_1+x_2;y_1+y_2\}$ \\
\hline
5. & $\overline{a}=\{5,-6, 1 \}, \overline{b}=\{-4, 3, 0 \} $, $\overline{c}=\left\{ 5,-8, 10 \right\}$ vektorları berilgen. $2{\overline{a}}^{2}+4{\overline{b}}^{2}-5{\overline{c}}^{2}$ ańlatpasınıń mánisin tabıń. & $-721$ \\
\hline
6. & $(2, 3)$ hám $(4, 3)$ noqatlarınan ótiwshi tuwrı sızıqtıń teńlemesin dúziń. & $ y-3=0$ \\
\hline
7. & $x^{2}+y^{2}-2x+4y=0$ sheńberdiń teńlemesin kanonikalıq túrdegi teńlemege alıp keliń. & $(x-1)^{2}+(y+2)^{2}=5$ \\
\hline
8. & $A(4, 3), B(7, 7)$ noqatları arasındaǵı aralıqtı tabıń. & $d(AB)=5$ \\
\hline
9. & $3x^{2}+10xy+3y^{2}-2x-14y-13=0$ teńlemesiniń tipin anıqlań. & giperbola \\
\hline
10. & $x^{2}-4y^{2}+6x+5=0$ giperbolanıń kanonikalıq teńlemesin dúziń. & $\frac{(x+3)^{2}}{4}-\frac{y^{2}}{1}=1$ \\
\hline
\end{tabular}

\vspace{1cm}

\begin{tabular}{lll}
Tuwrı juwaplar sanı: \underline{\hspace{1.5cm}} & 
Bahası: \underline{\hspace{1.5cm}} & 
Imtixan alıwshınıń qolı: \underline{\hspace{2cm}} \\
\end{tabular}

\egroup

\newpage


\textbf{89-variant}\\

\bgroup
\def\arraystretch{1.6} % 1 is the default, change whatever you need

\begin{tabular}{|m{5.7cm}|m{9.5cm}|}
\hline
Familiyası hám atı & \\
\hline
Fakulteti  & \\
\hline
Toparı hám tálim baǵdarı  & \\
\hline
\end{tabular}

\vspace{1cm}

\begin{tabular}{|m{0.7cm}|m{10cm}|m{4cm}|}
\hline
№ & Soraw & Juwap \\
\hline
1. & $OY$ kósheriniń teńlemesi? & $x=0$ \\
\hline
2. & Egerde $a=\{ x_1; y_1; z_1\}, b=\{ x_2, y_2; z_2\}$ bolsa, vektor kóbeymeniń koordinatalarda ańlatılıwı qanday boladı? &  $\lbrack ab\rbrack=\{y_1z_2-y_2z_1; z_1x_2-z_2x_1; x_1y_2-x_2y_1\}$ \\
\hline
3. & $A_1x+B_1y+C_1z+D_1=0$ hám $Ax_2+By_2+Cz_2+D_2=0$ tegislikleri perpendikulyar bolıwı shárti & $A_1\cdot A_2+B_1\cdot B_2+C_1\cdot C_2=0$ \\
\hline
4. & Úsh vektordıń aralas kóbeymesi ushın $(abc)=0$ teńligi orınlı bolsa ne dep ataladı? & $\overline{a}$, $\overline{b}$ hám $\overline{c}$ vektorları komplanar \\
\hline
5. & $M_{1}M_{2}$ kesindiniń ortasınıń koordinatalarınıń tabıń, eger $M_{1} (2, 3), M_{2} (4, 7)$ bolsa. & $(3,5)$ \\
\hline
6. & $x+y-3=0$ hám $2x+3y-8=0$ tuwrıları óz-ara qanday jaylasqan? & kesilisedi \\
\hline
7. & $x^{2}+y^{2}-2x+4y-20=0$ sheńberdiń $C$ orayın hám $R$ radiusın tabıń. & $C(1;-2), R=5$ \\
\hline
8. & $(x+1)^{2}+(y-2) ^{2}+(z+2) ^{2}=49$ sferanıń orayınıń koordinataların tabıń. & $(-1,2,-2)$ \\
\hline
9. & Eger $2a=16, e=\frac{5}{4}$ bolsa, fokusı abscissa kósherinde, koordinata basına salıstırǵanda simmetriyalıq jaylasqan giperbolanıń teńlemesin dúziń. & $\frac{x^{2}}{64}-\frac{y^{2}}{36}=1$ \\
\hline
10. & Eger $2b=24, 2 c=10$ bolsa, onda abscissa kósherinde koordinata basına salıstırǵanda simmetriyalıq jaylasqan fokuslarǵa iye, ellipstiń teńlemesin dúziń. & $\frac{x^{2}}{169}+\frac{y^{2}}{144}=1$ \\
\hline
\end{tabular}

\vspace{1cm}

\begin{tabular}{lll}
Tuwrı juwaplar sanı: \underline{\hspace{1.5cm}} & 
Bahası: \underline{\hspace{1.5cm}} & 
Imtixan alıwshınıń qolı: \underline{\hspace{2cm}} \\
\end{tabular}

\egroup

\newpage


\textbf{90-variant}\\

\bgroup
\def\arraystretch{1.6} % 1 is the default, change whatever you need

\begin{tabular}{|m{5.7cm}|m{9.5cm}|}
\hline
Familiyası hám atı & \\
\hline
Fakulteti  & \\
\hline
Toparı hám tálim baǵdarı  & \\
\hline
\end{tabular}

\vspace{1cm}

\begin{tabular}{|m{0.7cm}|m{10cm}|m{4cm}|}
\hline
№ & Soraw & Juwap \\
\hline
1. & $A_1x+B_1y+C_1z+D_1=0$ hám $Ax_2+By_2+Cz_2+D_2=0$ tegislikleri ústpe-úst túsiwi shárti? & $\frac{A_1}{A_2}=\frac{B_1}{B_2}=\frac{C_1}{C_2}=\frac{D_1}{D_2}$ \\
\hline
2. & Eki vektordıń skalyar kóbeymesiniń formulası? & $(ab)=|a||b|\cos\varphi$ \\
\hline
3. & $A_1x+B_1y+C_1z+D_1=0$ hám $Ax_2+By_2+Cz_2+D_2=0$ tegislikleri parallel bolıwı shárti & $\frac{A_1}{A_2}=\frac{B_1}{B_2}=\frac{C_1}{C_2}$ \\
\hline
4. & $Ax+C=0$ tuwrı sızıqtıń grafigi koordinata kósherlerine salıstırǵanda qanday jaylasqan? & $OY$ kósherine parallel \\
\hline
5. & $M_{1} (12;-1)$ hám $M_{2} (0;4)$ noqatlardıń arasındaǵı aralıqtı tabıń. & $13$ \\
\hline
6. & $x+y=0$ teńlemesi menen berilgen tuwrı sızıqtıń múyeshlik koefficientin anıqlań. & $- 1$ \\
\hline
7. & Orayı $C (-1;2)$ noqatında, $A (-2;6 )$ noqatınan ótetuǵın sheńberdiń teńlemesin dúziń. & $(x+1)^{2}+(y-2)^{2}=17$ \\
\hline
8. & $x+2=0$ keńislik qanday geometriyalıq betlikti anıqlaydı? &  $OYZ$ tegisligine parallel bolǵan tegislikti \\
\hline
9. & $\frac{x^{2}}{225}-\frac{y^{2}}{64}=-1$ giperbola fokusınıń koordinatalarınıń tabıń. & $F_{1}(0;-17), F_{2}(0;17)$ \\
\hline
10. & $9x^{2}+25y^{2}=225$ ellipsi berilgen, ellipstiń fokusların, ekscentrisitetin tabıń. & $F_1\left(-4;0 \right) , F_2\left( 4;0 \right) , e = \frac{4}{5}$ \\
\hline
\end{tabular}

\vspace{1cm}

\begin{tabular}{lll}
Tuwrı juwaplar sanı: \underline{\hspace{1.5cm}} & 
Bahası: \underline{\hspace{1.5cm}} & 
Imtixan alıwshınıń qolı: \underline{\hspace{2cm}} \\
\end{tabular}

\egroup

\newpage


\textbf{91-variant}\\

\bgroup
\def\arraystretch{1.6} % 1 is the default, change whatever you need

\begin{tabular}{|m{5.7cm}|m{9.5cm}|}
\hline
Familiyası hám atı & \\
\hline
Fakulteti  & \\
\hline
Toparı hám tálim baǵdarı  & \\
\hline
\end{tabular}

\vspace{1cm}

\begin{tabular}{|m{0.7cm}|m{10cm}|m{4cm}|}
\hline
№ & Soraw & Juwap \\
\hline
1. & Eki vektor qashan kollinear dep ataladı? & bir tuwrıda yamasa parallel tuwrıda jaylasqan bolsa \\
\hline
2. & Tuwrı múyeshli koordinatalar sisteması dep nege aytamız? & Masshtab birlikleri berilgen o'zara perpendikulyar $OX$ hám $OY$ kósherleri \\
\hline
3. & $OXY$ tegisliginiń teńlemesi? & $z=0$ \\
\hline
4. & Giperbolanıń kanonikalıq teńlemesi? & $\frac{x^2}{a^2}-\frac{y^2}{b^2}=1$ \\
\hline
5. & $A (-1;0;1),\ B (1;-1;0)$ noqatları berilgen. $\overline{BA}$ vektorın tabıń. & $\left\{ - 2;1;1 \right\}$ \\
\hline
6. & $2x+3y+4=0$ tuwrısına parallel hám $M_{0} (2;1)$ noqattan ótetuǵın tuwrınıń teńlemesin dúziń. & $2x+3y-7=0$ \\
\hline
7. & $x+y-12=0$ tuwrısı $x^{2}+y^{2}-2y=0$ sheńberge salıstırǵanda qanday jaylasqan? & sırtında jaylasqan \\
\hline
8. & $\left| \overline{a} \right|=8, \left| \overline{b} \right|=5, \alpha=60^{0}$ bolsa, $( \overline{a}\overline{b} )$ ni tabıń. & $20$ \\
\hline
9. & $2x+3y-6=0$ tuwrınıń teńlemesin kesindilerde berilgen teńleme túrinde kórsetiń. & $\frac{x}{3} + \frac{ y }{ 2 } =  1$ \\
\hline
10. & $\overline{a}=\left\{ 4,-2,-4 \right\}$ hám $\overline{b}=\left\{ 6,-3, 2 \right\}$ vektorları berilgen, $(\overline{a}-\overline{b}) ^{2}$-? & $41$ \\
\hline
\end{tabular}

\vspace{1cm}

\begin{tabular}{lll}
Tuwrı juwaplar sanı: \underline{\hspace{1.5cm}} & 
Bahası: \underline{\hspace{1.5cm}} & 
Imtixan alıwshınıń qolı: \underline{\hspace{2cm}} \\
\end{tabular}

\egroup

\newpage


\textbf{92-variant}\\

\bgroup
\def\arraystretch{1.6} % 1 is the default, change whatever you need

\begin{tabular}{|m{5.7cm}|m{9.5cm}|}
\hline
Familiyası hám atı & \\
\hline
Fakulteti  & \\
\hline
Toparı hám tálim baǵdarı  & \\
\hline
\end{tabular}

\vspace{1cm}

\begin{tabular}{|m{0.7cm}|m{10cm}|m{4cm}|}
\hline
№ & Soraw & Juwap \\
\hline
1. & Eki vektordıń vektor kóbeymesiniń uzınlıǵın tabıw formulası? & $\left| \lbrack ab\rbrack \right|=|a||b|\sin\varphi$ \\
\hline
2. & Tegislikdegi qálegen noqatınan berilgen eki noqatqa shekemgi bolǵan aralıqlardıń ayırmasınıń modulı ózgermeytuǵın bolǵan noqatlardıń geometriyalıq ornı ne dep ataladı? & giperbola \\
\hline
3. & Eki tuwrı sızıq arasındaǵı múyeshti tabıw formulası? & $\text{tg}\varphi=\frac{k_2-k_1}{1+k_1k_2}$ \\
\hline
4. & $\frac{x^2}{a^2}-\frac{y^2}{b^2}=1$ giperbolanıń $(x_0;y_0)$ noqatındaǵı urınbasınıń teńlemesin kórsetiń. & $\frac{x_0x}{a^2}-\frac{y_0y}{b^2}=1$ \\
\hline
5. & $5x-y+7=0$ hám $3x+2y=0$ tuwrıları arasındaǵı múyeshni tabıń. & $\varphi=\frac{\pi}{4}$ \\
\hline
6. & $\overline{a}=\left\{ 2, 1, 0 \right\}$ hám $\overline{b}=\left\{ 1, 0,-1 \right\}$ bolsa, $\overline{a}-\overline{b}$ ni tabıń. & $\overline{a} -\overline{b} = \left\{ 1,1,1 \right\}$ \\
\hline
7. & Koordinatalar kósherleri hám $ 3x+4y-12=0 $ tuwrı sızıǵı menen shegaralanǵan úshmúyeshliktiń maydanın tabıń. & $ S=6 $ \\
\hline
8. & $x-2y+1=0$ teńlemesi menen berilgen tuwrınıń normal túrdegi teńlemesin kórsetiń. & $\frac{x}{- \sqrt{5}}+\frac{2y}{\sqrt{5}}-\frac{1}{\sqrt{5}}=0$ \\
\hline
9. & $3x-y+5=0$, $x+3y-4=0$ tuwrı sızıqları arasındaǵı múyeshti tabıń. & $90^{0}$ \\
\hline
10. & $\overline{a}=\{5,-6, 1 \}, \overline{b}=\{-4, 3, 0 \} $, $\overline{c}=\left\{ 5,-8, 10 \right\}$ vektorları berilgen. $2{\overline{a}}^{2}+4{\overline{b}}^{2}-5{\overline{c}}^{2}$ ańlatpasınıń mánisin tabıń. & $-721$ \\
\hline
\end{tabular}

\vspace{1cm}

\begin{tabular}{lll}
Tuwrı juwaplar sanı: \underline{\hspace{1.5cm}} & 
Bahası: \underline{\hspace{1.5cm}} & 
Imtixan alıwshınıń qolı: \underline{\hspace{2cm}} \\
\end{tabular}

\egroup

\newpage


\textbf{93-variant}\\

\bgroup
\def\arraystretch{1.6} % 1 is the default, change whatever you need

\begin{tabular}{|m{5.7cm}|m{9.5cm}|}
\hline
Familiyası hám atı & \\
\hline
Fakulteti  & \\
\hline
Toparı hám tálim baǵdarı  & \\
\hline
\end{tabular}

\vspace{1cm}

\begin{tabular}{|m{0.7cm}|m{10cm}|m{4cm}|}
\hline
№ & Soraw & Juwap \\
\hline
1. & Vektorlardıń kósherdegi proekciyasınıń formulası? & $x=|a|\cos\varphi, y=|a|\sin\varphi$ \\
\hline
2. & $Ax+By+D=0$ teńlemesi arqalı ... tegisliktiń teńlemesi berilgen? & $OZ$ kósherine parallel \\
\hline
3. & $\frac{x^2}{a^2}+\frac{y^2}{b^2}=1$ ellipstiń $(x_0;y_0)$ noqatındaǵı urınbasınıń teńlemesin tabıń. & $\frac{x_0x}{a^2}+\frac{y_0y}{b^2}=1$ \\
\hline
4. & Vektorlardı qosıw koordinatalarda qanday formula menen anıqlanadı? & $\overline{a}+\overline{b}=\{x_1+x_2;y_1+y_2\}$ \\
\hline
5. & $(2, 3)$ hám $(4, 3)$ noqatlarınan ótiwshi tuwrı sızıqtıń teńlemesin dúziń. & $ y-3=0$ \\
\hline
6. & $x^{2}+y^{2}-2x+4y=0$ sheńberdiń teńlemesin kanonikalıq túrdegi teńlemege alıp keliń. & $(x-1)^{2}+(y+2)^{2}=5$ \\
\hline
7. & $A(4, 3), B(7, 7)$ noqatları arasındaǵı aralıqtı tabıń. & $d(AB)=5$ \\
\hline
8. & $3x^{2}+10xy+3y^{2}-2x-14y-13=0$ teńlemesiniń tipin anıqlań. & giperbola \\
\hline
9. & $x^{2}-4y^{2}+6x+5=0$ giperbolanıń kanonikalıq teńlemesin dúziń. & $\frac{(x+3)^{2}}{4}-\frac{y^{2}}{1}=1$ \\
\hline
10. & $M_{1}M_{2}$ kesindiniń ortasınıń koordinatalarınıń tabıń, eger $M_{1} (2, 3), M_{2} (4, 7)$ bolsa. & $(3,5)$ \\
\hline
\end{tabular}

\vspace{1cm}

\begin{tabular}{lll}
Tuwrı juwaplar sanı: \underline{\hspace{1.5cm}} & 
Bahası: \underline{\hspace{1.5cm}} & 
Imtixan alıwshınıń qolı: \underline{\hspace{2cm}} \\
\end{tabular}

\egroup

\newpage


\textbf{94-variant}\\

\bgroup
\def\arraystretch{1.6} % 1 is the default, change whatever you need

\begin{tabular}{|m{5.7cm}|m{9.5cm}|}
\hline
Familiyası hám atı & \\
\hline
Fakulteti  & \\
\hline
Toparı hám tálim baǵdarı  & \\
\hline
\end{tabular}

\vspace{1cm}

\begin{tabular}{|m{0.7cm}|m{10cm}|m{4cm}|}
\hline
№ & Soraw & Juwap \\
\hline
1. & $OY$ kósheriniń teńlemesi? & $x=0$ \\
\hline
2. & Egerde $a=\{ x_1; y_1; z_1\}, b=\{ x_2, y_2; z_2\}$ bolsa, vektor kóbeymeniń koordinatalarda ańlatılıwı qanday boladı? &  $\lbrack ab\rbrack=\{y_1z_2-y_2z_1; z_1x_2-z_2x_1; x_1y_2-x_2y_1\}$ \\
\hline
3. & $A_1x+B_1y+C_1z+D_1=0$ hám $Ax_2+By_2+Cz_2+D_2=0$ tegislikleri perpendikulyar bolıwı shárti & $A_1\cdot A_2+B_1\cdot B_2+C_1\cdot C_2=0$ \\
\hline
4. & Úsh vektordıń aralas kóbeymesi ushın $(abc)=0$ teńligi orınlı bolsa ne dep ataladı? & $\overline{a}$, $\overline{b}$ hám $\overline{c}$ vektorları komplanar \\
\hline
5. & $x+y-3=0$ hám $2x+3y-8=0$ tuwrıları óz-ara qanday jaylasqan? & kesilisedi \\
\hline
6. & $x^{2}+y^{2}-2x+4y-20=0$ sheńberdiń $C$ orayın hám $R$ radiusın tabıń. & $C(1;-2), R=5$ \\
\hline
7. & $(x+1)^{2}+(y-2) ^{2}+(z+2) ^{2}=49$ sferanıń orayınıń koordinataların tabıń. & $(-1,2,-2)$ \\
\hline
8. & Eger $2a=16, e=\frac{5}{4}$ bolsa, fokusı abscissa kósherinde, koordinata basına salıstırǵanda simmetriyalıq jaylasqan giperbolanıń teńlemesin dúziń. & $\frac{x^{2}}{64}-\frac{y^{2}}{36}=1$ \\
\hline
9. & Eger $2b=24, 2 c=10$ bolsa, onda abscissa kósherinde koordinata basına salıstırǵanda simmetriyalıq jaylasqan fokuslarǵa iye, ellipstiń teńlemesin dúziń. & $\frac{x^{2}}{169}+\frac{y^{2}}{144}=1$ \\
\hline
10. & $M_{1} (12;-1)$ hám $M_{2} (0;4)$ noqatlardıń arasındaǵı aralıqtı tabıń. & $13$ \\
\hline
\end{tabular}

\vspace{1cm}

\begin{tabular}{lll}
Tuwrı juwaplar sanı: \underline{\hspace{1.5cm}} & 
Bahası: \underline{\hspace{1.5cm}} & 
Imtixan alıwshınıń qolı: \underline{\hspace{2cm}} \\
\end{tabular}

\egroup

\newpage


\textbf{95-variant}\\

\bgroup
\def\arraystretch{1.6} % 1 is the default, change whatever you need

\begin{tabular}{|m{5.7cm}|m{9.5cm}|}
\hline
Familiyası hám atı & \\
\hline
Fakulteti  & \\
\hline
Toparı hám tálim baǵdarı  & \\
\hline
\end{tabular}

\vspace{1cm}

\begin{tabular}{|m{0.7cm}|m{10cm}|m{4cm}|}
\hline
№ & Soraw & Juwap \\
\hline
1. & $A_1x+B_1y+C_1z+D_1=0$ hám $Ax_2+By_2+Cz_2+D_2=0$ tegislikleri ústpe-úst túsiwi shárti? & $\frac{A_1}{A_2}=\frac{B_1}{B_2}=\frac{C_1}{C_2}=\frac{D_1}{D_2}$ \\
\hline
2. & Eki vektordıń skalyar kóbeymesiniń formulası? & $(ab)=|a||b|\cos\varphi$ \\
\hline
3. & $A_1x+B_1y+C_1z+D_1=0$ hám $Ax_2+By_2+Cz_2+D_2=0$ tegislikleri parallel bolıwı shárti & $\frac{A_1}{A_2}=\frac{B_1}{B_2}=\frac{C_1}{C_2}$ \\
\hline
4. & $Ax+C=0$ tuwrı sızıqtıń grafigi koordinata kósherlerine salıstırǵanda qanday jaylasqan? & $OY$ kósherine parallel \\
\hline
5. & $x+y=0$ teńlemesi menen berilgen tuwrı sızıqtıń múyeshlik koefficientin anıqlań. & $- 1$ \\
\hline
6. & Orayı $C (-1;2)$ noqatında, $A (-2;6 )$ noqatınan ótetuǵın sheńberdiń teńlemesin dúziń. & $(x+1)^{2}+(y-2)^{2}=17$ \\
\hline
7. & $x+2=0$ keńislik qanday geometriyalıq betlikti anıqlaydı? &  $OYZ$ tegisligine parallel bolǵan tegislikti \\
\hline
8. & $\frac{x^{2}}{225}-\frac{y^{2}}{64}=-1$ giperbola fokusınıń koordinatalarınıń tabıń. & $F_{1}(0;-17), F_{2}(0;17)$ \\
\hline
9. & $9x^{2}+25y^{2}=225$ ellipsi berilgen, ellipstiń fokusların, ekscentrisitetin tabıń. & $F_1\left(-4;0 \right) , F_2\left( 4;0 \right) , e = \frac{4}{5}$ \\
\hline
10. & $A (-1;0;1),\ B (1;-1;0)$ noqatları berilgen. $\overline{BA}$ vektorın tabıń. & $\left\{ - 2;1;1 \right\}$ \\
\hline
\end{tabular}

\vspace{1cm}

\begin{tabular}{lll}
Tuwrı juwaplar sanı: \underline{\hspace{1.5cm}} & 
Bahası: \underline{\hspace{1.5cm}} & 
Imtixan alıwshınıń qolı: \underline{\hspace{2cm}} \\
\end{tabular}

\egroup

\newpage


\textbf{96-variant}\\

\bgroup
\def\arraystretch{1.6} % 1 is the default, change whatever you need

\begin{tabular}{|m{5.7cm}|m{9.5cm}|}
\hline
Familiyası hám atı & \\
\hline
Fakulteti  & \\
\hline
Toparı hám tálim baǵdarı  & \\
\hline
\end{tabular}

\vspace{1cm}

\begin{tabular}{|m{0.7cm}|m{10cm}|m{4cm}|}
\hline
№ & Soraw & Juwap \\
\hline
1. & Eki vektor qashan kollinear dep ataladı? & bir tuwrıda yamasa parallel tuwrıda jaylasqan bolsa \\
\hline
2. & Tuwrı múyeshli koordinatalar sisteması dep nege aytamız? & Masshtab birlikleri berilgen o'zara perpendikulyar $OX$ hám $OY$ kósherleri \\
\hline
3. & $OXY$ tegisliginiń teńlemesi? & $z=0$ \\
\hline
4. & Giperbolanıń kanonikalıq teńlemesi? & $\frac{x^2}{a^2}-\frac{y^2}{b^2}=1$ \\
\hline
5. & $2x+3y+4=0$ tuwrısına parallel hám $M_{0} (2;1)$ noqattan ótetuǵın tuwrınıń teńlemesin dúziń. & $2x+3y-7=0$ \\
\hline
6. & $x+y-12=0$ tuwrısı $x^{2}+y^{2}-2y=0$ sheńberge salıstırǵanda qanday jaylasqan? & sırtında jaylasqan \\
\hline
7. & $\left| \overline{a} \right|=8, \left| \overline{b} \right|=5, \alpha=60^{0}$ bolsa, $( \overline{a}\overline{b} )$ ni tabıń. & $20$ \\
\hline
8. & $2x+3y-6=0$ tuwrınıń teńlemesin kesindilerde berilgen teńleme túrinde kórsetiń. & $\frac{x}{3} + \frac{ y }{ 2 } =  1$ \\
\hline
9. & $\overline{a}=\left\{ 4,-2,-4 \right\}$ hám $\overline{b}=\left\{ 6,-3, 2 \right\}$ vektorları berilgen, $(\overline{a}-\overline{b}) ^{2}$-? & $41$ \\
\hline
10. & $5x-y+7=0$ hám $3x+2y=0$ tuwrıları arasındaǵı múyeshni tabıń. & $\varphi=\frac{\pi}{4}$ \\
\hline
\end{tabular}

\vspace{1cm}

\begin{tabular}{lll}
Tuwrı juwaplar sanı: \underline{\hspace{1.5cm}} & 
Bahası: \underline{\hspace{1.5cm}} & 
Imtixan alıwshınıń qolı: \underline{\hspace{2cm}} \\
\end{tabular}

\egroup

\newpage


\textbf{97-variant}\\

\bgroup
\def\arraystretch{1.6} % 1 is the default, change whatever you need

\begin{tabular}{|m{5.7cm}|m{9.5cm}|}
\hline
Familiyası hám atı & \\
\hline
Fakulteti  & \\
\hline
Toparı hám tálim baǵdarı  & \\
\hline
\end{tabular}

\vspace{1cm}

\begin{tabular}{|m{0.7cm}|m{10cm}|m{4cm}|}
\hline
№ & Soraw & Juwap \\
\hline
1. & Eki vektordıń vektor kóbeymesiniń uzınlıǵın tabıw formulası? & $\left| \lbrack ab\rbrack \right|=|a||b|\sin\varphi$ \\
\hline
2. & Tegislikdegi qálegen noqatınan berilgen eki noqatqa shekemgi bolǵan aralıqlardıń ayırmasınıń modulı ózgermeytuǵın bolǵan noqatlardıń geometriyalıq ornı ne dep ataladı? & giperbola \\
\hline
3. & Eki tuwrı sızıq arasındaǵı múyeshti tabıw formulası? & $\text{tg}\varphi=\frac{k_2-k_1}{1+k_1k_2}$ \\
\hline
4. & $\frac{x^2}{a^2}-\frac{y^2}{b^2}=1$ giperbolanıń $(x_0;y_0)$ noqatındaǵı urınbasınıń teńlemesin kórsetiń. & $\frac{x_0x}{a^2}-\frac{y_0y}{b^2}=1$ \\
\hline
5. & $\overline{a}=\left\{ 2, 1, 0 \right\}$ hám $\overline{b}=\left\{ 1, 0,-1 \right\}$ bolsa, $\overline{a}-\overline{b}$ ni tabıń. & $\overline{a} -\overline{b} = \left\{ 1,1,1 \right\}$ \\
\hline
6. & Koordinatalar kósherleri hám $ 3x+4y-12=0 $ tuwrı sızıǵı menen shegaralanǵan úshmúyeshliktiń maydanın tabıń. & $ S=6 $ \\
\hline
7. & $x-2y+1=0$ teńlemesi menen berilgen tuwrınıń normal túrdegi teńlemesin kórsetiń. & $\frac{x}{- \sqrt{5}}+\frac{2y}{\sqrt{5}}-\frac{1}{\sqrt{5}}=0$ \\
\hline
8. & $3x-y+5=0$, $x+3y-4=0$ tuwrı sızıqları arasındaǵı múyeshti tabıń. & $90^{0}$ \\
\hline
9. & $\overline{a}=\{5,-6, 1 \}, \overline{b}=\{-4, 3, 0 \} $, $\overline{c}=\left\{ 5,-8, 10 \right\}$ vektorları berilgen. $2{\overline{a}}^{2}+4{\overline{b}}^{2}-5{\overline{c}}^{2}$ ańlatpasınıń mánisin tabıń. & $-721$ \\
\hline
10. & $(2, 3)$ hám $(4, 3)$ noqatlarınan ótiwshi tuwrı sızıqtıń teńlemesin dúziń. & $ y-3=0$ \\
\hline
\end{tabular}

\vspace{1cm}

\begin{tabular}{lll}
Tuwrı juwaplar sanı: \underline{\hspace{1.5cm}} & 
Bahası: \underline{\hspace{1.5cm}} & 
Imtixan alıwshınıń qolı: \underline{\hspace{2cm}} \\
\end{tabular}

\egroup

\newpage


\textbf{98-variant}\\

\bgroup
\def\arraystretch{1.6} % 1 is the default, change whatever you need

\begin{tabular}{|m{5.7cm}|m{9.5cm}|}
\hline
Familiyası hám atı & \\
\hline
Fakulteti  & \\
\hline
Toparı hám tálim baǵdarı  & \\
\hline
\end{tabular}

\vspace{1cm}

\begin{tabular}{|m{0.7cm}|m{10cm}|m{4cm}|}
\hline
№ & Soraw & Juwap \\
\hline
1. & Vektorlardıń kósherdegi proekciyasınıń formulası? & $x=|a|\cos\varphi, y=|a|\sin\varphi$ \\
\hline
2. & $Ax+By+D=0$ teńlemesi arqalı ... tegisliktiń teńlemesi berilgen? & $OZ$ kósherine parallel \\
\hline
3. & $\frac{x^2}{a^2}+\frac{y^2}{b^2}=1$ ellipstiń $(x_0;y_0)$ noqatındaǵı urınbasınıń teńlemesin tabıń. & $\frac{x_0x}{a^2}+\frac{y_0y}{b^2}=1$ \\
\hline
4. & Vektorlardı qosıw koordinatalarda qanday formula menen anıqlanadı? & $\overline{a}+\overline{b}=\{x_1+x_2;y_1+y_2\}$ \\
\hline
5. & $x^{2}+y^{2}-2x+4y=0$ sheńberdiń teńlemesin kanonikalıq túrdegi teńlemege alıp keliń. & $(x-1)^{2}+(y+2)^{2}=5$ \\
\hline
6. & $A(4, 3), B(7, 7)$ noqatları arasındaǵı aralıqtı tabıń. & $d(AB)=5$ \\
\hline
7. & $3x^{2}+10xy+3y^{2}-2x-14y-13=0$ teńlemesiniń tipin anıqlań. & giperbola \\
\hline
8. & $x^{2}-4y^{2}+6x+5=0$ giperbolanıń kanonikalıq teńlemesin dúziń. & $\frac{(x+3)^{2}}{4}-\frac{y^{2}}{1}=1$ \\
\hline
9. & $M_{1}M_{2}$ kesindiniń ortasınıń koordinatalarınıń tabıń, eger $M_{1} (2, 3), M_{2} (4, 7)$ bolsa. & $(3,5)$ \\
\hline
10. & $x+y-3=0$ hám $2x+3y-8=0$ tuwrıları óz-ara qanday jaylasqan? & kesilisedi \\
\hline
\end{tabular}

\vspace{1cm}

\begin{tabular}{lll}
Tuwrı juwaplar sanı: \underline{\hspace{1.5cm}} & 
Bahası: \underline{\hspace{1.5cm}} & 
Imtixan alıwshınıń qolı: \underline{\hspace{2cm}} \\
\end{tabular}

\egroup

\newpage


\textbf{99-variant}\\

\bgroup
\def\arraystretch{1.6} % 1 is the default, change whatever you need

\begin{tabular}{|m{5.7cm}|m{9.5cm}|}
\hline
Familiyası hám atı & \\
\hline
Fakulteti  & \\
\hline
Toparı hám tálim baǵdarı  & \\
\hline
\end{tabular}

\vspace{1cm}

\begin{tabular}{|m{0.7cm}|m{10cm}|m{4cm}|}
\hline
№ & Soraw & Juwap \\
\hline
1. & $OY$ kósheriniń teńlemesi? & $x=0$ \\
\hline
2. & Egerde $a=\{ x_1; y_1; z_1\}, b=\{ x_2, y_2; z_2\}$ bolsa, vektor kóbeymeniń koordinatalarda ańlatılıwı qanday boladı? &  $\lbrack ab\rbrack=\{y_1z_2-y_2z_1; z_1x_2-z_2x_1; x_1y_2-x_2y_1\}$ \\
\hline
3. & $A_1x+B_1y+C_1z+D_1=0$ hám $Ax_2+By_2+Cz_2+D_2=0$ tegislikleri perpendikulyar bolıwı shárti & $A_1\cdot A_2+B_1\cdot B_2+C_1\cdot C_2=0$ \\
\hline
4. & Úsh vektordıń aralas kóbeymesi ushın $(abc)=0$ teńligi orınlı bolsa ne dep ataladı? & $\overline{a}$, $\overline{b}$ hám $\overline{c}$ vektorları komplanar \\
\hline
5. & $x^{2}+y^{2}-2x+4y-20=0$ sheńberdiń $C$ orayın hám $R$ radiusın tabıń. & $C(1;-2), R=5$ \\
\hline
6. & $(x+1)^{2}+(y-2) ^{2}+(z+2) ^{2}=49$ sferanıń orayınıń koordinataların tabıń. & $(-1,2,-2)$ \\
\hline
7. & Eger $2a=16, e=\frac{5}{4}$ bolsa, fokusı abscissa kósherinde, koordinata basına salıstırǵanda simmetriyalıq jaylasqan giperbolanıń teńlemesin dúziń. & $\frac{x^{2}}{64}-\frac{y^{2}}{36}=1$ \\
\hline
8. & Eger $2b=24, 2 c=10$ bolsa, onda abscissa kósherinde koordinata basına salıstırǵanda simmetriyalıq jaylasqan fokuslarǵa iye, ellipstiń teńlemesin dúziń. & $\frac{x^{2}}{169}+\frac{y^{2}}{144}=1$ \\
\hline
9. & $M_{1} (12;-1)$ hám $M_{2} (0;4)$ noqatlardıń arasındaǵı aralıqtı tabıń. & $13$ \\
\hline
10. & $x+y=0$ teńlemesi menen berilgen tuwrı sızıqtıń múyeshlik koefficientin anıqlań. & $- 1$ \\
\hline
\end{tabular}

\vspace{1cm}

\begin{tabular}{lll}
Tuwrı juwaplar sanı: \underline{\hspace{1.5cm}} & 
Bahası: \underline{\hspace{1.5cm}} & 
Imtixan alıwshınıń qolı: \underline{\hspace{2cm}} \\
\end{tabular}

\egroup

\newpage


\textbf{100-variant}\\

\bgroup
\def\arraystretch{1.6} % 1 is the default, change whatever you need

\begin{tabular}{|m{5.7cm}|m{9.5cm}|}
\hline
Familiyası hám atı & \\
\hline
Fakulteti  & \\
\hline
Toparı hám tálim baǵdarı  & \\
\hline
\end{tabular}

\vspace{1cm}

\begin{tabular}{|m{0.7cm}|m{10cm}|m{4cm}|}
\hline
№ & Soraw & Juwap \\
\hline
1. & $A_1x+B_1y+C_1z+D_1=0$ hám $Ax_2+By_2+Cz_2+D_2=0$ tegislikleri ústpe-úst túsiwi shárti? & $\frac{A_1}{A_2}=\frac{B_1}{B_2}=\frac{C_1}{C_2}=\frac{D_1}{D_2}$ \\
\hline
2. & Eki vektordıń skalyar kóbeymesiniń formulası? & $(ab)=|a||b|\cos\varphi$ \\
\hline
3. & $A_1x+B_1y+C_1z+D_1=0$ hám $Ax_2+By_2+Cz_2+D_2=0$ tegislikleri parallel bolıwı shárti & $\frac{A_1}{A_2}=\frac{B_1}{B_2}=\frac{C_1}{C_2}$ \\
\hline
4. & $Ax+C=0$ tuwrı sızıqtıń grafigi koordinata kósherlerine salıstırǵanda qanday jaylasqan? & $OY$ kósherine parallel \\
\hline
5. & Orayı $C (-1;2)$ noqatında, $A (-2;6 )$ noqatınan ótetuǵın sheńberdiń teńlemesin dúziń. & $(x+1)^{2}+(y-2)^{2}=17$ \\
\hline
6. & $x+2=0$ keńislik qanday geometriyalıq betlikti anıqlaydı? &  $OYZ$ tegisligine parallel bolǵan tegislikti \\
\hline
7. & $\frac{x^{2}}{225}-\frac{y^{2}}{64}=-1$ giperbola fokusınıń koordinatalarınıń tabıń. & $F_{1}(0;-17), F_{2}(0;17)$ \\
\hline
8. & $9x^{2}+25y^{2}=225$ ellipsi berilgen, ellipstiń fokusların, ekscentrisitetin tabıń. & $F_1\left(-4;0 \right) , F_2\left( 4;0 \right) , e = \frac{4}{5}$ \\
\hline
9. & $A (-1;0;1),\ B (1;-1;0)$ noqatları berilgen. $\overline{BA}$ vektorın tabıń. & $\left\{ - 2;1;1 \right\}$ \\
\hline
10. & $2x+3y+4=0$ tuwrısına parallel hám $M_{0} (2;1)$ noqattan ótetuǵın tuwrınıń teńlemesin dúziń. & $2x+3y-7=0$ \\
\hline
\end{tabular}

\vspace{1cm}

\begin{tabular}{lll}
Tuwrı juwaplar sanı: \underline{\hspace{1.5cm}} & 
Bahası: \underline{\hspace{1.5cm}} & 
Imtixan alıwshınıń qolı: \underline{\hspace{2cm}} \\
\end{tabular}

\egroup

\newpage



\end{document}
