\documentclass{article}
\usepackage[fontsize=12pt]{fontsize}
\usepackage[utf8]{inputenc}
\usepackage[T2A]{fontenc}
\usepackage{array}
\usepackage[a4paper,
left=15mm,
top=15mm,]{geometry}
\usepackage{setspace}


\renewcommand{\baselinestretch}{1.2} 

\begin{document}

\pagenumbering{gobble}


\textbf{1-variant}\\

\bgroup
\def\arraystretch{1.6} % 1 is the default, change whatever you need

\begin{tabular}{|m{5.7cm}|m{9.5cm}|}
\hline
Familiyası hám atı & \\
\hline
Fakulteti  & \\
\hline
Toparı hám tálim baǵdarı  & \\
\hline
\end{tabular}

\vspace{0.7cm}

\begin{tabular}{|m{0.7cm}|m{10cm}|m{4cm}|}
\hline
№ & Soraw & Juwap \\
\hline
1. & Giperbolanıń kanonikalıq teńlemesi? &  \\
\hline
2. & \(\frac{x^{2}}{a^{2}} - \frac{y^{2}}{b^{2}} = 1\) giperbolanıń \((x_{0};y_{0})\) noqatındaǵı urınbasınıń teńlemesin kórsetiń. &  \\
\hline
3. & undefined &  \\
\hline
4. & undefined &  \\
\hline
5. & \(\left| \bar{a} \right| = 8, \left| \bar{b} \right| = 5, \alpha = 60^{0}\) bolsa, \(( \bar{a}\bar{b} )\) ni tabıń. &  \\
\hline
6. & \(x - 2 y + 1 = 0\) teńlemesi menen berilgen tuwrınıń normal túrdegi teńlemesin kórsetiń. &  \\
\hline
7. & Orayı \(C (- 1;2)\) noqatında, \(A (- 2;6 )\) noqatınan ótetuǵın sheńberdiń teńlemesin dúziń. &  \\
\hline
8. & \(A_{1}x + B_{1}y + C_{1}z + D_{1} = 0\) hám tegislikleri ústpe-úst túsiwi ushın qaysı shárt orınlı bolıwı kerek? &  \\
\hline
9. & \(\frac{x^{2}}{225} - \frac{y^{2}}{64} = - 1\) giperbola fokusınıń koordinatalarınıń tabıń. &  \\
\hline
10. & \(x^{2} - 4 y^{2} + 6 x + 5 = 0\) giperbolanıń kanonikalıq teńlemesin dúziń. & \\
\hline
\end{tabular}

\vspace{0.7cm}

\begin{tabular}{lll}
Tuwrı juwaplar sanı: \underline{\hspace{1cm}} & 
Bahası: \underline{\hspace{1cm}} & 
Imtixan alıwshınıń qolı: \underline{\hspace{2cm}} \\
\end{tabular}

\egroup

\newpage


\textbf{2-variant}\\

\bgroup
\def\arraystretch{1.6} % 1 is the default, change whatever you need

\begin{tabular}{|m{5.7cm}|m{9.5cm}|}
\hline
Familiyası hám atı & \\
\hline
Fakulteti  & \\
\hline
Toparı hám tálim baǵdarı  & \\
\hline
\end{tabular}

\vspace{0.7cm}

\begin{tabular}{|m{0.7cm}|m{10cm}|m{4cm}|}
\hline
№ & Soraw & Juwap \\
\hline
1. & Vektorlardı qosıw koordinatalarda qanday formula menen anıqlanadı? &  \\
\hline
2. & undefined &  \\
\hline
3. & undefined &  \\
\hline
4. & undefined &  \\
\hline
5. & \(\overline{a} = \{5,- 6, 1 \}, \overline{b} = \{ - 4, 3, 0 \} \), \(\overline{c} = \left\{ 5,- 8, 10 \right\}\) vektorları berilgen. \(2{\bar{a}}^{2} + 4{\bar{b}}^{2} - 5{\bar{c}}^{2}\) ańlatpasınıń mánisin tabıń. &  \\
\hline
6. & \(3 x - y + 5 = 0, x + 3 y - 4 = 0\) tuwrı sızıqları arasındaǵı múyeshti tabıń. &  \\
\hline
7. & \(x^{2} + y^{2} - 2 x + 4 y - 20 = 0\) sheńberdiń \(C\) orayın hám \(R\) radiusın tabıń. &  \\
\hline
8. & \(A_{1}x + B_{1}y + C_{1}z + D_{1} = 0\) hám \(Ax + By + Cz + D = 0\) tegislikleri perpendikulyar bolıwı ushın qaysı shárt orınlı bolıwı kerek? &  \\
\hline
9. & Eger \(2 a = 16, e = \frac{5}{4}\) bolsa, fokusı abscissa kósherinde, koordinata basına salıstırǵanda simmetriyalıq jaylasqan giperbolanıń teńlemesin dúziń. &  \\
\hline
10. & \(9 x^{2} + 25 y^{2} = 225\) ellipsi berilgen, ellipstiń fokusların, ekscentrisitetin tabıń. & \\
\hline
\end{tabular}

\vspace{0.7cm}

\begin{tabular}{lll}
Tuwrı juwaplar sanı: \underline{\hspace{1cm}} & 
Bahası: \underline{\hspace{1cm}} & 
Imtixan alıwshınıń qolı: \underline{\hspace{2cm}} \\
\end{tabular}

\egroup

\newpage


\textbf{3-variant}\\

\bgroup
\def\arraystretch{1.6} % 1 is the default, change whatever you need

\begin{tabular}{|m{5.7cm}|m{9.5cm}|}
\hline
Familiyası hám atı & \\
\hline
Fakulteti  & \\
\hline
Toparı hám tálim baǵdarı  & \\
\hline
\end{tabular}

\vspace{0.7cm}

\begin{tabular}{|m{0.7cm}|m{10cm}|m{4cm}|}
\hline
№ & Soraw & Juwap \\
\hline
1. & Eki vektordıń vektor kóbeymesiniń uzınlıǵın tabıw formulası? &  \\
\hline
2. & undefined &  \\
\hline
3. & undefined &  \\
\hline
4. & undefined &  \\
\hline
5. & \(M_{1} (12; - 1)\) hám \(M_{2} (0;4)\) noqatlardıń arasındaǵı aralıqtı tabıń. &  \\
\hline
6. & \(x + y - 3 = 0\) hám \(2 x + 3 y - 8 = 0\) tuwrıları óz-ara qanday jaylasqan? &  \\
\hline
7. & \(x^{2} + y^{2} - 2 x + 4 y = 0\) sheńberdiń teńlemesin kanonikalıq túrdegi teńlemege alıp keliń. &  \\
\hline
8. & \((x + 1) ^{2} + (y - 2) ^{2} + (z + 2) ^{2} = 49\) sferanıń orayınıń koordinataların tabıń. &  \\
\hline
9. & \(3 x^{2} + 10 xy + 3 y^{2} - 2 x - 14 y - 13 = 0\) teńlemesiniń tipin anıqlań. &  \\
\hline
10. & Eger \(2 b = 24, 2 c = 10\) bolsa, onda abscissa kósherinde koordinata basına salıstırǵanda simmetriyalıq jaylasqan fokuslarǵa iye, ellipstiń teńlemesin dúziń. & \\
\hline
\end{tabular}

\vspace{0.7cm}

\begin{tabular}{lll}
Tuwrı juwaplar sanı: \underline{\hspace{1cm}} & 
Bahası: \underline{\hspace{1cm}} & 
Imtixan alıwshınıń qolı: \underline{\hspace{2cm}} \\
\end{tabular}

\egroup

\newpage


\textbf{4-variant}\\

\bgroup
\def\arraystretch{1.6} % 1 is the default, change whatever you need

\begin{tabular}{|m{5.7cm}|m{9.5cm}|}
\hline
Familiyası hám atı & \\
\hline
Fakulteti  & \\
\hline
Toparı hám tálim baǵdarı  & \\
\hline
\end{tabular}

\vspace{0.7cm}

\begin{tabular}{|m{0.7cm}|m{10cm}|m{4cm}|}
\hline
№ & Soraw & Juwap \\
\hline
1. & Tegislikdegi qálegen noqattan berilgen eki noqatqa shekemgi bolǵan aralıqlardıń ayırmasınıń modulı ózgermeytuǵın bolǵan noqatlardıń geometriyalıq ornı ne dep ataladı? &  \\
\hline
2. & undefined &  \\
\hline
3. & undefined &  \\
\hline
4. & undefined &  \\
\hline
5. & \(\bar{a} = \left\{ 2, 1, 0 \right\}\) hám \(\bar{b} = \left\{ 1, 0,- 1 \right\}\) bolsa, \(\bar{a} - \bar{b}\) ni tabıń. &  \\
\hline
6. & \(2 x + 3 y + 4 = 0\) tuwrısına parallel hám \(M_{0} (2;1)\) noqattan ótetuǵın tuwrınıń teńlemesin dúziń. &  \\
\hline
7. & \(x + y - 12 = 0\) tuwrısı \(x^{2} + y^{2} - 2 y = 0\) sheńberge salıstırǵanda qanday jaylasqan? &  \\
\hline
8. & \(x + 2 = 0\) keńislik qanday geometriyalıq betlikti anıqlaydı? &  \\
\hline
9. & \(\frac{x^{2}}{a^{2}} + \frac{y^{2}}{b^{2}} = 1\) ellipstiń \((x_{0};y_{0})\) noqatındaǵı urınbasınıń teńlemesin tabıń. &  \\
\hline
10. & \(A (- 1;0;1),\ B (1; - 1;0)\) noqatları berilgen. \(\bar{BA}\) vektorın tabıń. & \\
\hline
\end{tabular}

\vspace{0.7cm}

\begin{tabular}{lll}
Tuwrı juwaplar sanı: \underline{\hspace{1cm}} & 
Bahası: \underline{\hspace{1cm}} & 
Imtixan alıwshınıń qolı: \underline{\hspace{2cm}} \\
\end{tabular}

\egroup

\newpage


\textbf{5-variant}\\

\bgroup
\def\arraystretch{1.6} % 1 is the default, change whatever you need

\begin{tabular}{|m{5.7cm}|m{9.5cm}|}
\hline
Familiyası hám atı & \\
\hline
Fakulteti  & \\
\hline
Toparı hám tálim baǵdarı  & \\
\hline
\end{tabular}

\vspace{0.7cm}

\begin{tabular}{|m{0.7cm}|m{10cm}|m{4cm}|}
\hline
№ & Soraw & Juwap \\
\hline
1. & Eki vektor qashan kollinear dep ataladı? &  \\
\hline
2. & undefined &  \\
\hline
3. & undefined &  \\
\hline
4. & undefined &  \\
\hline
5. & $(2, 3)$ hám $(4, 3)$ noqatlarınan ótiwshi tuwrı sızıqtıń teńlemesin dúziń. &  \\
\hline
6. & \(A_{1}x + B_{1}y + C_{1}z + D_{1} = 0\) hám \(Ax + By + Cz + D = 0\) tegislikleri parallel bolıwı ushın qaysı shárt orınlı bolıwı kerek? &  \\
\hline
7. & \(\bar{a} = \left\{ 4,- 2,- 4 \right\}\) hám \(\bar{b} = \left\{ 6,- 3, 2 \right\}\) vektorları berilgen, \((\bar{a} - \bar{b}) ^{2}\)-? &  \\
\hline
8. & \(2 x + 3 y - 6 = 0\) tuwrınıń teńlemesin kesindilerde berilgen teńleme túrinde kórsetiń. &  \\
\hline
9. & \(A (4, 3), B (7, 7)\) noqatları arasındaǵı aralıqtı tabıń. &  \\
\hline
10. & \(M_{1}M_{2}\) kesindiniń ortasınıń koordinatalarınıń tabıń, eger \(M_{1} (2, 3), M_{2} (4, 7)\) bolsa. & \\
\hline
\end{tabular}

\vspace{0.7cm}

\begin{tabular}{lll}
Tuwrı juwaplar sanı: \underline{\hspace{1cm}} & 
Bahası: \underline{\hspace{1cm}} & 
Imtixan alıwshınıń qolı: \underline{\hspace{2cm}} \\
\end{tabular}

\egroup

\newpage


\textbf{6-variant}\\

\bgroup
\def\arraystretch{1.6} % 1 is the default, change whatever you need

\begin{tabular}{|m{5.7cm}|m{9.5cm}|}
\hline
Familiyası hám atı & \\
\hline
Fakulteti  & \\
\hline
Toparı hám tálim baǵdarı  & \\
\hline
\end{tabular}

\vspace{0.7cm}

\begin{tabular}{|m{0.7cm}|m{10cm}|m{4cm}|}
\hline
№ & Soraw & Juwap \\
\hline
1. & Egerde \(a = \{ x_{1}; y_{1}; z_{1}\}, b = \{ x_{2}; y_{2}; z_{2}\}\) bolsa, vektor kóbeymeniń koordinatalarda ańlatılıwı qanday boladı? &  \\
\hline
2. & undefined &  \\
\hline
3. & undefined &  \\
\hline
4. & undefined &  \\
\hline
5. & \(x + y = 0\) teńlemesi menen berilgen tuwrı sızıqtıń múyeshlik koefficientin anıqlań. &  \\
\hline
6. & \(5 x - y + 7 = 0\) hám \(3 x + 2 y = 0\) tuwrıları arasındaǵı múyeshni tabıń. &  \\
\hline
7. & Koordinatalar kósherleri hám \( 3 x + 4 y - 12 = 0 \) tuwrı sızıǵı menen shegaralanǵan úshmúyeshliktiń maydanın tabıń. &  \\
\hline
8. & \(\left| \bar{a} \right| = 8, \left| \bar{b} \right| = 5, \alpha = 60^{0}\) bolsa, \(( \bar{a}\bar{b} )\) ni tabıń. &  \\
\hline
9. & \(x - 2 y + 1 = 0\) teńlemesi menen berilgen tuwrınıń normal túrdegi teńlemesin kórsetiń. &  \\
\hline
10. & Orayı \(C (- 1;2)\) noqatında, \(A (- 2;6 )\) noqatınan ótetuǵın sheńberdiń teńlemesin dúziń. & \\
\hline
\end{tabular}

\vspace{0.7cm}

\begin{tabular}{lll}
Tuwrı juwaplar sanı: \underline{\hspace{1cm}} & 
Bahası: \underline{\hspace{1cm}} & 
Imtixan alıwshınıń qolı: \underline{\hspace{2cm}} \\
\end{tabular}

\egroup

\newpage


\textbf{7-variant}\\

\bgroup
\def\arraystretch{1.6} % 1 is the default, change whatever you need

\begin{tabular}{|m{5.7cm}|m{9.5cm}|}
\hline
Familiyası hám atı & \\
\hline
Fakulteti  & \\
\hline
Toparı hám tálim baǵdarı  & \\
\hline
\end{tabular}

\vspace{0.7cm}

\begin{tabular}{|m{0.7cm}|m{10cm}|m{4cm}|}
\hline
№ & Soraw & Juwap \\
\hline
1. & \(Ax + C = 0\) tuwrı sızıqtıń grafigi koordinata kósherlerine salıstırǵanda qanday jaylasqan? &  \\
\hline
2. & undefined &  \\
\hline
3. & undefined &  \\
\hline
4. & undefined &  \\
\hline
5. & \(A_{1}x + B_{1}y + C_{1}z + D_{1} = 0\) hám tegislikleri ústpe-úst túsiwi ushın qaysı shárt orınlı bolıwı kerek? &  \\
\hline
6. & \(\frac{x^{2}}{225} - \frac{y^{2}}{64} = - 1\) giperbola fokusınıń koordinatalarınıń tabıń. &  \\
\hline
7. & \(x^{2} - 4 y^{2} + 6 x + 5 = 0\) giperbolanıń kanonikalıq teńlemesin dúziń. &  \\
\hline
8. & \(\overline{a} = \{5,- 6, 1 \}, \overline{b} = \{ - 4, 3, 0 \} \), \(\overline{c} = \left\{ 5,- 8, 10 \right\}\) vektorları berilgen. \(2{\bar{a}}^{2} + 4{\bar{b}}^{2} - 5{\bar{c}}^{2}\) ańlatpasınıń mánisin tabıń. &  \\
\hline
9. & \(3 x - y + 5 = 0, x + 3 y - 4 = 0\) tuwrı sızıqları arasındaǵı múyeshti tabıń. &  \\
\hline
10. & \(x^{2} + y^{2} - 2 x + 4 y - 20 = 0\) sheńberdiń \(C\) orayın hám \(R\) radiusın tabıń. & \\
\hline
\end{tabular}

\vspace{0.7cm}

\begin{tabular}{lll}
Tuwrı juwaplar sanı: \underline{\hspace{1cm}} & 
Bahası: \underline{\hspace{1cm}} & 
Imtixan alıwshınıń qolı: \underline{\hspace{2cm}} \\
\end{tabular}

\egroup

\newpage


\textbf{8-variant}\\

\bgroup
\def\arraystretch{1.6} % 1 is the default, change whatever you need

\begin{tabular}{|m{5.7cm}|m{9.5cm}|}
\hline
Familiyası hám atı & \\
\hline
Fakulteti  & \\
\hline
Toparı hám tálim baǵdarı  & \\
\hline
\end{tabular}

\vspace{0.7cm}

\begin{tabular}{|m{0.7cm}|m{10cm}|m{4cm}|}
\hline
№ & Soraw & Juwap \\
\hline
1. & Úsh vektordıń aralas kóbeymesi ushın \((abc) = 0\) teńligi orınlı bolsa ne dep ataladı? &  \\
\hline
2. & undefined &  \\
\hline
3. & undefined &  \\
\hline
4. & undefined &  \\
\hline
5. & \(A_{1}x + B_{1}y + C_{1}z + D_{1} = 0\) hám \(Ax + By + Cz + D = 0\) tegislikleri perpendikulyar bolıwı ushın qaysı shárt orınlı bolıwı kerek? &  \\
\hline
6. & Eger \(2 a = 16, e = \frac{5}{4}\) bolsa, fokusı abscissa kósherinde, koordinata basına salıstırǵanda simmetriyalıq jaylasqan giperbolanıń teńlemesin dúziń. &  \\
\hline
7. & \(9 x^{2} + 25 y^{2} = 225\) ellipsi berilgen, ellipstiń fokusların, ekscentrisitetin tabıń. &  \\
\hline
8. & \(M_{1} (12; - 1)\) hám \(M_{2} (0;4)\) noqatlardıń arasındaǵı aralıqtı tabıń. &  \\
\hline
9. & \(x + y - 3 = 0\) hám \(2 x + 3 y - 8 = 0\) tuwrıları óz-ara qanday jaylasqan? &  \\
\hline
10. & \(x^{2} + y^{2} - 2 x + 4 y = 0\) sheńberdiń teńlemesin kanonikalıq túrdegi teńlemege alıp keliń. & \\
\hline
\end{tabular}

\vspace{0.7cm}

\begin{tabular}{lll}
Tuwrı juwaplar sanı: \underline{\hspace{1cm}} & 
Bahası: \underline{\hspace{1cm}} & 
Imtixan alıwshınıń qolı: \underline{\hspace{2cm}} \\
\end{tabular}

\egroup

\newpage


\textbf{9-variant}\\

\bgroup
\def\arraystretch{1.6} % 1 is the default, change whatever you need

\begin{tabular}{|m{5.7cm}|m{9.5cm}|}
\hline
Familiyası hám atı & \\
\hline
Fakulteti  & \\
\hline
Toparı hám tálim baǵdarı  & \\
\hline
\end{tabular}

\vspace{0.7cm}

\begin{tabular}{|m{0.7cm}|m{10cm}|m{4cm}|}
\hline
№ & Soraw & Juwap \\
\hline
1. & Tuwrı múyeshli koordinatalar sisteması dep nege aytamız? &  \\
\hline
2. & undefined &  \\
\hline
3. & undefined &  \\
\hline
4. & undefined &  \\
\hline
5. & \((x + 1) ^{2} + (y - 2) ^{2} + (z + 2) ^{2} = 49\) sferanıń orayınıń koordinataların tabıń. &  \\
\hline
6. & \(3 x^{2} + 10 xy + 3 y^{2} - 2 x - 14 y - 13 = 0\) teńlemesiniń tipin anıqlań. &  \\
\hline
7. & Eger \(2 b = 24, 2 c = 10\) bolsa, onda abscissa kósherinde koordinata basına salıstırǵanda simmetriyalıq jaylasqan fokuslarǵa iye, ellipstiń teńlemesin dúziń. &  \\
\hline
8. & \(\bar{a} = \left\{ 2, 1, 0 \right\}\) hám \(\bar{b} = \left\{ 1, 0,- 1 \right\}\) bolsa, \(\bar{a} - \bar{b}\) ni tabıń. &  \\
\hline
9. & \(2 x + 3 y + 4 = 0\) tuwrısına parallel hám \(M_{0} (2;1)\) noqattan ótetuǵın tuwrınıń teńlemesin dúziń. &  \\
\hline
10. & \(x + y - 12 = 0\) tuwrısı \(x^{2} + y^{2} - 2 y = 0\) sheńberge salıstırǵanda qanday jaylasqan? & \\
\hline
\end{tabular}

\vspace{0.7cm}

\begin{tabular}{lll}
Tuwrı juwaplar sanı: \underline{\hspace{1cm}} & 
Bahası: \underline{\hspace{1cm}} & 
Imtixan alıwshınıń qolı: \underline{\hspace{2cm}} \\
\end{tabular}

\egroup

\newpage


\textbf{10-variant}\\

\bgroup
\def\arraystretch{1.6} % 1 is the default, change whatever you need

\begin{tabular}{|m{5.7cm}|m{9.5cm}|}
\hline
Familiyası hám atı & \\
\hline
Fakulteti  & \\
\hline
Toparı hám tálim baǵdarı  & \\
\hline
\end{tabular}

\vspace{0.7cm}

\begin{tabular}{|m{0.7cm}|m{10cm}|m{4cm}|}
\hline
№ & Soraw & Juwap \\
\hline
1. & Vektorlardıń kósherdegi proekciyasınıń formulası? &  \\
\hline
2. & undefined &  \\
\hline
3. & undefined &  \\
\hline
4. & undefined &  \\
\hline
5. & \(x + 2 = 0\) keńislik qanday geometriyalıq betlikti anıqlaydı? &  \\
\hline
6. & \(\frac{x^{2}}{a^{2}} + \frac{y^{2}}{b^{2}} = 1\) ellipstiń \((x_{0};y_{0})\) noqatındaǵı urınbasınıń teńlemesin tabıń. &  \\
\hline
7. & \(A (- 1;0;1),\ B (1; - 1;0)\) noqatları berilgen. \(\bar{BA}\) vektorın tabıń. &  \\
\hline
8. & $(2, 3)$ hám $(4, 3)$ noqatlarınan ótiwshi tuwrı sızıqtıń teńlemesin dúziń. &  \\
\hline
9. & \(A_{1}x + B_{1}y + C_{1}z + D_{1} = 0\) hám \(Ax + By + Cz + D = 0\) tegislikleri parallel bolıwı ushın qaysı shárt orınlı bolıwı kerek? &  \\
\hline
10. & \(\bar{a} = \left\{ 4,- 2,- 4 \right\}\) hám \(\bar{b} = \left\{ 6,- 3, 2 \right\}\) vektorları berilgen, \((\bar{a} - \bar{b}) ^{2}\)-? & \\
\hline
\end{tabular}

\vspace{0.7cm}

\begin{tabular}{lll}
Tuwrı juwaplar sanı: \underline{\hspace{1cm}} & 
Bahası: \underline{\hspace{1cm}} & 
Imtixan alıwshınıń qolı: \underline{\hspace{2cm}} \\
\end{tabular}

\egroup

\newpage


\textbf{11-variant}\\

\bgroup
\def\arraystretch{1.6} % 1 is the default, change whatever you need

\begin{tabular}{|m{5.7cm}|m{9.5cm}|}
\hline
Familiyası hám atı & \\
\hline
Fakulteti  & \\
\hline
Toparı hám tálim baǵdarı  & \\
\hline
\end{tabular}

\vspace{0.7cm}

\begin{tabular}{|m{0.7cm}|m{10cm}|m{4cm}|}
\hline
№ & Soraw & Juwap \\
\hline
1. & Eki vektordıń skalyar kóbeymesiniń formulası? &  \\
\hline
2. & undefined &  \\
\hline
3. & undefined &  \\
\hline
4. & undefined &  \\
\hline
5. & \(2 x + 3 y - 6 = 0\) tuwrınıń teńlemesin kesindilerde berilgen teńleme túrinde kórsetiń. &  \\
\hline
6. & \(A (4, 3), B (7, 7)\) noqatları arasındaǵı aralıqtı tabıń. &  \\
\hline
7. & \(M_{1}M_{2}\) kesindiniń ortasınıń koordinatalarınıń tabıń, eger \(M_{1} (2, 3), M_{2} (4, 7)\) bolsa. &  \\
\hline
8. & \(x + y = 0\) teńlemesi menen berilgen tuwrı sızıqtıń múyeshlik koefficientin anıqlań. &  \\
\hline
9. & \(5 x - y + 7 = 0\) hám \(3 x + 2 y = 0\) tuwrıları arasındaǵı múyeshni tabıń. &  \\
\hline
10. & Koordinatalar kósherleri hám \( 3 x + 4 y - 12 = 0 \) tuwrı sızıǵı menen shegaralanǵan úshmúyeshliktiń maydanın tabıń. & \\
\hline
\end{tabular}

\vspace{0.7cm}

\begin{tabular}{lll}
Tuwrı juwaplar sanı: \underline{\hspace{1cm}} & 
Bahası: \underline{\hspace{1cm}} & 
Imtixan alıwshınıń qolı: \underline{\hspace{2cm}} \\
\end{tabular}

\egroup

\newpage


\textbf{12-variant}\\

\bgroup
\def\arraystretch{1.6} % 1 is the default, change whatever you need

\begin{tabular}{|m{5.7cm}|m{9.5cm}|}
\hline
Familiyası hám atı & \\
\hline
Fakulteti  & \\
\hline
Toparı hám tálim baǵdarı  & \\
\hline
\end{tabular}

\vspace{0.7cm}

\begin{tabular}{|m{0.7cm}|m{10cm}|m{4cm}|}
\hline
№ & Soraw & Juwap \\
\hline
1. & \(OXY\) tegisliginiń teńlemesi? &  \\
\hline
2. & undefined &  \\
\hline
3. & undefined &  \\
\hline
4. & undefined &  \\
\hline
5. & \(\left| \bar{a} \right| = 8, \left| \bar{b} \right| = 5, \alpha = 60^{0}\) bolsa, \(( \bar{a}\bar{b} )\) ni tabıń. &  \\
\hline
6. & \(x - 2 y + 1 = 0\) teńlemesi menen berilgen tuwrınıń normal túrdegi teńlemesin kórsetiń. &  \\
\hline
7. & Orayı \(C (- 1;2)\) noqatında, \(A (- 2;6 )\) noqatınan ótetuǵın sheńberdiń teńlemesin dúziń. &  \\
\hline
8. & \(A_{1}x + B_{1}y + C_{1}z + D_{1} = 0\) hám tegislikleri ústpe-úst túsiwi ushın qaysı shárt orınlı bolıwı kerek? &  \\
\hline
9. & \(\frac{x^{2}}{225} - \frac{y^{2}}{64} = - 1\) giperbola fokusınıń koordinatalarınıń tabıń. &  \\
\hline
10. & \(x^{2} - 4 y^{2} + 6 x + 5 = 0\) giperbolanıń kanonikalıq teńlemesin dúziń. & \\
\hline
\end{tabular}

\vspace{0.7cm}

\begin{tabular}{lll}
Tuwrı juwaplar sanı: \underline{\hspace{1cm}} & 
Bahası: \underline{\hspace{1cm}} & 
Imtixan alıwshınıń qolı: \underline{\hspace{2cm}} \\
\end{tabular}

\egroup

\newpage


\textbf{13-variant}\\

\bgroup
\def\arraystretch{1.6} % 1 is the default, change whatever you need

\begin{tabular}{|m{5.7cm}|m{9.5cm}|}
\hline
Familiyası hám atı & \\
\hline
Fakulteti  & \\
\hline
Toparı hám tálim baǵdarı  & \\
\hline
\end{tabular}

\vspace{0.7cm}

\begin{tabular}{|m{0.7cm}|m{10cm}|m{4cm}|}
\hline
№ & Soraw & Juwap \\
\hline
1. & Eki tuwrı sızıq arasındaǵı múyeshti tabıw formulası? &  \\
\hline
2. & undefined &  \\
\hline
3. & undefined &  \\
\hline
4. & undefined &  \\
\hline
5. & \(\overline{a} = \{5,- 6, 1 \}, \overline{b} = \{ - 4, 3, 0 \} \), \(\overline{c} = \left\{ 5,- 8, 10 \right\}\) vektorları berilgen. \(2{\bar{a}}^{2} + 4{\bar{b}}^{2} - 5{\bar{c}}^{2}\) ańlatpasınıń mánisin tabıń. &  \\
\hline
6. & \(3 x - y + 5 = 0, x + 3 y - 4 = 0\) tuwrı sızıqları arasındaǵı múyeshti tabıń. &  \\
\hline
7. & \(x^{2} + y^{2} - 2 x + 4 y - 20 = 0\) sheńberdiń \(C\) orayın hám \(R\) radiusın tabıń. &  \\
\hline
8. & \(A_{1}x + B_{1}y + C_{1}z + D_{1} = 0\) hám \(Ax + By + Cz + D = 0\) tegislikleri perpendikulyar bolıwı ushın qaysı shárt orınlı bolıwı kerek? &  \\
\hline
9. & Eger \(2 a = 16, e = \frac{5}{4}\) bolsa, fokusı abscissa kósherinde, koordinata basına salıstırǵanda simmetriyalıq jaylasqan giperbolanıń teńlemesin dúziń. &  \\
\hline
10. & \(9 x^{2} + 25 y^{2} = 225\) ellipsi berilgen, ellipstiń fokusların, ekscentrisitetin tabıń. & \\
\hline
\end{tabular}

\vspace{0.7cm}

\begin{tabular}{lll}
Tuwrı juwaplar sanı: \underline{\hspace{1cm}} & 
Bahası: \underline{\hspace{1cm}} & 
Imtixan alıwshınıń qolı: \underline{\hspace{2cm}} \\
\end{tabular}

\egroup

\newpage


\textbf{14-variant}\\

\bgroup
\def\arraystretch{1.6} % 1 is the default, change whatever you need

\begin{tabular}{|m{5.7cm}|m{9.5cm}|}
\hline
Familiyası hám atı & \\
\hline
Fakulteti  & \\
\hline
Toparı hám tálim baǵdarı  & \\
\hline
\end{tabular}

\vspace{0.7cm}

\begin{tabular}{|m{0.7cm}|m{10cm}|m{4cm}|}
\hline
№ & Soraw & Juwap \\
\hline
1. & \(OY\) kósheriniń teńlemesi? &  \\
\hline
2. & undefined &  \\
\hline
3. & undefined &  \\
\hline
4. & undefined &  \\
\hline
5. & \(M_{1} (12; - 1)\) hám \(M_{2} (0;4)\) noqatlardıń arasındaǵı aralıqtı tabıń. &  \\
\hline
6. & \(x + y - 3 = 0\) hám \(2 x + 3 y - 8 = 0\) tuwrıları óz-ara qanday jaylasqan? &  \\
\hline
7. & \(x^{2} + y^{2} - 2 x + 4 y = 0\) sheńberdiń teńlemesin kanonikalıq túrdegi teńlemege alıp keliń. &  \\
\hline
8. & \((x + 1) ^{2} + (y - 2) ^{2} + (z + 2) ^{2} = 49\) sferanıń orayınıń koordinataların tabıń. &  \\
\hline
9. & \(3 x^{2} + 10 xy + 3 y^{2} - 2 x - 14 y - 13 = 0\) teńlemesiniń tipin anıqlań. &  \\
\hline
10. & Eger \(2 b = 24, 2 c = 10\) bolsa, onda abscissa kósherinde koordinata basına salıstırǵanda simmetriyalıq jaylasqan fokuslarǵa iye, ellipstiń teńlemesin dúziń. & \\
\hline
\end{tabular}

\vspace{0.7cm}

\begin{tabular}{lll}
Tuwrı juwaplar sanı: \underline{\hspace{1cm}} & 
Bahası: \underline{\hspace{1cm}} & 
Imtixan alıwshınıń qolı: \underline{\hspace{2cm}} \\
\end{tabular}

\egroup

\newpage


\textbf{15-variant}\\

\bgroup
\def\arraystretch{1.6} % 1 is the default, change whatever you need

\begin{tabular}{|m{5.7cm}|m{9.5cm}|}
\hline
Familiyası hám atı & \\
\hline
Fakulteti  & \\
\hline
Toparı hám tálim baǵdarı  & \\
\hline
\end{tabular}

\vspace{0.7cm}

\begin{tabular}{|m{0.7cm}|m{10cm}|m{4cm}|}
\hline
№ & Soraw & Juwap \\
\hline
1. & \(Ax + By + D = 0\) teńlemesi arqalı ... tegisliktiń teńlemesi berilgen? &  \\
\hline
2. & undefined &  \\
\hline
3. & undefined &  \\
\hline
4. & undefined &  \\
\hline
5. & \(\bar{a} = \left\{ 2, 1, 0 \right\}\) hám \(\bar{b} = \left\{ 1, 0,- 1 \right\}\) bolsa, \(\bar{a} - \bar{b}\) ni tabıń. &  \\
\hline
6. & \(2 x + 3 y + 4 = 0\) tuwrısına parallel hám \(M_{0} (2;1)\) noqattan ótetuǵın tuwrınıń teńlemesin dúziń. &  \\
\hline
7. & \(x + y - 12 = 0\) tuwrısı \(x^{2} + y^{2} - 2 y = 0\) sheńberge salıstırǵanda qanday jaylasqan? &  \\
\hline
8. & \(x + 2 = 0\) keńislik qanday geometriyalıq betlikti anıqlaydı? &  \\
\hline
9. & \(\frac{x^{2}}{a^{2}} + \frac{y^{2}}{b^{2}} = 1\) ellipstiń \((x_{0};y_{0})\) noqatındaǵı urınbasınıń teńlemesin tabıń. &  \\
\hline
10. & \(A (- 1;0;1),\ B (1; - 1;0)\) noqatları berilgen. \(\bar{BA}\) vektorın tabıń. & \\
\hline
\end{tabular}

\vspace{0.7cm}

\begin{tabular}{lll}
Tuwrı juwaplar sanı: \underline{\hspace{1cm}} & 
Bahası: \underline{\hspace{1cm}} & 
Imtixan alıwshınıń qolı: \underline{\hspace{2cm}} \\
\end{tabular}

\egroup

\newpage


\textbf{16-variant}\\

\bgroup
\def\arraystretch{1.6} % 1 is the default, change whatever you need

\begin{tabular}{|m{5.7cm}|m{9.5cm}|}
\hline
Familiyası hám atı & \\
\hline
Fakulteti  & \\
\hline
Toparı hám tálim baǵdarı  & \\
\hline
\end{tabular}

\vspace{0.7cm}

\begin{tabular}{|m{0.7cm}|m{10cm}|m{4cm}|}
\hline
№ & Soraw & Juwap \\
\hline
1. & Vektorlardı qosıw tómendegi qaysı qásiyetke iye emes? &  \\
\hline
2. & undefined &  \\
\hline
3. & undefined &  \\
\hline
4. & undefined &  \\
\hline
5. & $(2, 3)$ hám $(4, 3)$ noqatlarınan ótiwshi tuwrı sızıqtıń teńlemesin dúziń. &  \\
\hline
6. & \(A_{1}x + B_{1}y + C_{1}z + D_{1} = 0\) hám \(Ax + By + Cz + D = 0\) tegislikleri parallel bolıwı ushın qaysı shárt orınlı bolıwı kerek? &  \\
\hline
7. & \(\bar{a} = \left\{ 4,- 2,- 4 \right\}\) hám \(\bar{b} = \left\{ 6,- 3, 2 \right\}\) vektorları berilgen, \((\bar{a} - \bar{b}) ^{2}\)-? &  \\
\hline
8. & \(2 x + 3 y - 6 = 0\) tuwrınıń teńlemesin kesindilerde berilgen teńleme túrinde kórsetiń. &  \\
\hline
9. & \(A (4, 3), B (7, 7)\) noqatları arasındaǵı aralıqtı tabıń. &  \\
\hline
10. & \(M_{1}M_{2}\) kesindiniń ortasınıń koordinatalarınıń tabıń, eger \(M_{1} (2, 3), M_{2} (4, 7)\) bolsa. & \\
\hline
\end{tabular}

\vspace{0.7cm}

\begin{tabular}{lll}
Tuwrı juwaplar sanı: \underline{\hspace{1cm}} & 
Bahası: \underline{\hspace{1cm}} & 
Imtixan alıwshınıń qolı: \underline{\hspace{2cm}} \\
\end{tabular}

\egroup

\newpage


\textbf{17-variant}\\

\bgroup
\def\arraystretch{1.6} % 1 is the default, change whatever you need

\begin{tabular}{|m{5.7cm}|m{9.5cm}|}
\hline
Familiyası hám atı & \\
\hline
Fakulteti  & \\
\hline
Toparı hám tálim baǵdarı  & \\
\hline
\end{tabular}

\vspace{0.7cm}

\begin{tabular}{|m{0.7cm}|m{10cm}|m{4cm}|}
\hline
№ & Soraw & Juwap \\
\hline
1. & Giperbolanıń kanonikalıq teńlemesi? &  \\
\hline
2. & undefined &  \\
\hline
3. & undefined &  \\
\hline
4. & undefined &  \\
\hline
5. & \(x + y = 0\) teńlemesi menen berilgen tuwrı sızıqtıń múyeshlik koefficientin anıqlań. &  \\
\hline
6. & \(5 x - y + 7 = 0\) hám \(3 x + 2 y = 0\) tuwrıları arasındaǵı múyeshni tabıń. &  \\
\hline
7. & Koordinatalar kósherleri hám \( 3 x + 4 y - 12 = 0 \) tuwrı sızıǵı menen shegaralanǵan úshmúyeshliktiń maydanın tabıń. &  \\
\hline
8. & \(\left| \bar{a} \right| = 8, \left| \bar{b} \right| = 5, \alpha = 60^{0}\) bolsa, \(( \bar{a}\bar{b} )\) ni tabıń. &  \\
\hline
9. & \(x - 2 y + 1 = 0\) teńlemesi menen berilgen tuwrınıń normal túrdegi teńlemesin kórsetiń. &  \\
\hline
10. & Orayı \(C (- 1;2)\) noqatında, \(A (- 2;6 )\) noqatınan ótetuǵın sheńberdiń teńlemesin dúziń. & \\
\hline
\end{tabular}

\vspace{0.7cm}

\begin{tabular}{lll}
Tuwrı juwaplar sanı: \underline{\hspace{1cm}} & 
Bahası: \underline{\hspace{1cm}} & 
Imtixan alıwshınıń qolı: \underline{\hspace{2cm}} \\
\end{tabular}

\egroup

\newpage


\textbf{18-variant}\\

\bgroup
\def\arraystretch{1.6} % 1 is the default, change whatever you need

\begin{tabular}{|m{5.7cm}|m{9.5cm}|}
\hline
Familiyası hám atı & \\
\hline
Fakulteti  & \\
\hline
Toparı hám tálim baǵdarı  & \\
\hline
\end{tabular}

\vspace{0.7cm}

\begin{tabular}{|m{0.7cm}|m{10cm}|m{4cm}|}
\hline
№ & Soraw & Juwap \\
\hline
1. & \(\frac{x^{2}}{a^{2}} - \frac{y^{2}}{b^{2}} = 1\) giperbolanıń \((x_{0};y_{0})\) noqatındaǵı urınbasınıń teńlemesin kórsetiń. &  \\
\hline
2. & undefined &  \\
\hline
3. & undefined &  \\
\hline
4. & undefined &  \\
\hline
5. & \(A_{1}x + B_{1}y + C_{1}z + D_{1} = 0\) hám tegislikleri ústpe-úst túsiwi ushın qaysı shárt orınlı bolıwı kerek? &  \\
\hline
6. & \(\frac{x^{2}}{225} - \frac{y^{2}}{64} = - 1\) giperbola fokusınıń koordinatalarınıń tabıń. &  \\
\hline
7. & \(x^{2} - 4 y^{2} + 6 x + 5 = 0\) giperbolanıń kanonikalıq teńlemesin dúziń. &  \\
\hline
8. & \(\overline{a} = \{5,- 6, 1 \}, \overline{b} = \{ - 4, 3, 0 \} \), \(\overline{c} = \left\{ 5,- 8, 10 \right\}\) vektorları berilgen. \(2{\bar{a}}^{2} + 4{\bar{b}}^{2} - 5{\bar{c}}^{2}\) ańlatpasınıń mánisin tabıń. &  \\
\hline
9. & \(3 x - y + 5 = 0, x + 3 y - 4 = 0\) tuwrı sızıqları arasındaǵı múyeshti tabıń. &  \\
\hline
10. & \(x^{2} + y^{2} - 2 x + 4 y - 20 = 0\) sheńberdiń \(C\) orayın hám \(R\) radiusın tabıń. & \\
\hline
\end{tabular}

\vspace{0.7cm}

\begin{tabular}{lll}
Tuwrı juwaplar sanı: \underline{\hspace{1cm}} & 
Bahası: \underline{\hspace{1cm}} & 
Imtixan alıwshınıń qolı: \underline{\hspace{2cm}} \\
\end{tabular}

\egroup

\newpage


\textbf{19-variant}\\

\bgroup
\def\arraystretch{1.6} % 1 is the default, change whatever you need

\begin{tabular}{|m{5.7cm}|m{9.5cm}|}
\hline
Familiyası hám atı & \\
\hline
Fakulteti  & \\
\hline
Toparı hám tálim baǵdarı  & \\
\hline
\end{tabular}

\vspace{0.7cm}

\begin{tabular}{|m{0.7cm}|m{10cm}|m{4cm}|}
\hline
№ & Soraw & Juwap \\
\hline
1. & Vektorlardı qosıw koordinatalarda qanday formula menen anıqlanadı? &  \\
\hline
2. & undefined &  \\
\hline
3. & undefined &  \\
\hline
4. & undefined &  \\
\hline
5. & \(A_{1}x + B_{1}y + C_{1}z + D_{1} = 0\) hám \(Ax + By + Cz + D = 0\) tegislikleri perpendikulyar bolıwı ushın qaysı shárt orınlı bolıwı kerek? &  \\
\hline
6. & Eger \(2 a = 16, e = \frac{5}{4}\) bolsa, fokusı abscissa kósherinde, koordinata basına salıstırǵanda simmetriyalıq jaylasqan giperbolanıń teńlemesin dúziń. &  \\
\hline
7. & \(9 x^{2} + 25 y^{2} = 225\) ellipsi berilgen, ellipstiń fokusların, ekscentrisitetin tabıń. &  \\
\hline
8. & \(M_{1} (12; - 1)\) hám \(M_{2} (0;4)\) noqatlardıń arasındaǵı aralıqtı tabıń. &  \\
\hline
9. & \(x + y - 3 = 0\) hám \(2 x + 3 y - 8 = 0\) tuwrıları óz-ara qanday jaylasqan? &  \\
\hline
10. & \(x^{2} + y^{2} - 2 x + 4 y = 0\) sheńberdiń teńlemesin kanonikalıq túrdegi teńlemege alıp keliń. & \\
\hline
\end{tabular}

\vspace{0.7cm}

\begin{tabular}{lll}
Tuwrı juwaplar sanı: \underline{\hspace{1cm}} & 
Bahası: \underline{\hspace{1cm}} & 
Imtixan alıwshınıń qolı: \underline{\hspace{2cm}} \\
\end{tabular}

\egroup

\newpage


\textbf{20-variant}\\

\bgroup
\def\arraystretch{1.6} % 1 is the default, change whatever you need

\begin{tabular}{|m{5.7cm}|m{9.5cm}|}
\hline
Familiyası hám atı & \\
\hline
Fakulteti  & \\
\hline
Toparı hám tálim baǵdarı  & \\
\hline
\end{tabular}

\vspace{0.7cm}

\begin{tabular}{|m{0.7cm}|m{10cm}|m{4cm}|}
\hline
№ & Soraw & Juwap \\
\hline
1. & Eki vektordıń vektor kóbeymesiniń uzınlıǵın tabıw formulası? &  \\
\hline
2. & undefined &  \\
\hline
3. & undefined &  \\
\hline
4. & undefined &  \\
\hline
5. & \((x + 1) ^{2} + (y - 2) ^{2} + (z + 2) ^{2} = 49\) sferanıń orayınıń koordinataların tabıń. &  \\
\hline
6. & \(3 x^{2} + 10 xy + 3 y^{2} - 2 x - 14 y - 13 = 0\) teńlemesiniń tipin anıqlań. &  \\
\hline
7. & Eger \(2 b = 24, 2 c = 10\) bolsa, onda abscissa kósherinde koordinata basına salıstırǵanda simmetriyalıq jaylasqan fokuslarǵa iye, ellipstiń teńlemesin dúziń. &  \\
\hline
8. & \(\bar{a} = \left\{ 2, 1, 0 \right\}\) hám \(\bar{b} = \left\{ 1, 0,- 1 \right\}\) bolsa, \(\bar{a} - \bar{b}\) ni tabıń. &  \\
\hline
9. & \(2 x + 3 y + 4 = 0\) tuwrısına parallel hám \(M_{0} (2;1)\) noqattan ótetuǵın tuwrınıń teńlemesin dúziń. &  \\
\hline
10. & \(x + y - 12 = 0\) tuwrısı \(x^{2} + y^{2} - 2 y = 0\) sheńberge salıstırǵanda qanday jaylasqan? & \\
\hline
\end{tabular}

\vspace{0.7cm}

\begin{tabular}{lll}
Tuwrı juwaplar sanı: \underline{\hspace{1cm}} & 
Bahası: \underline{\hspace{1cm}} & 
Imtixan alıwshınıń qolı: \underline{\hspace{2cm}} \\
\end{tabular}

\egroup

\newpage


\textbf{21-variant}\\

\bgroup
\def\arraystretch{1.6} % 1 is the default, change whatever you need

\begin{tabular}{|m{5.7cm}|m{9.5cm}|}
\hline
Familiyası hám atı & \\
\hline
Fakulteti  & \\
\hline
Toparı hám tálim baǵdarı  & \\
\hline
\end{tabular}

\vspace{0.7cm}

\begin{tabular}{|m{0.7cm}|m{10cm}|m{4cm}|}
\hline
№ & Soraw & Juwap \\
\hline
1. & Tegislikdegi qálegen noqattan berilgen eki noqatqa shekemgi bolǵan aralıqlardıń ayırmasınıń modulı ózgermeytuǵın bolǵan noqatlardıń geometriyalıq ornı ne dep ataladı? &  \\
\hline
2. & undefined &  \\
\hline
3. & undefined &  \\
\hline
4. & undefined &  \\
\hline
5. & \(x + 2 = 0\) keńislik qanday geometriyalıq betlikti anıqlaydı? &  \\
\hline
6. & \(\frac{x^{2}}{a^{2}} + \frac{y^{2}}{b^{2}} = 1\) ellipstiń \((x_{0};y_{0})\) noqatındaǵı urınbasınıń teńlemesin tabıń. &  \\
\hline
7. & \(A (- 1;0;1),\ B (1; - 1;0)\) noqatları berilgen. \(\bar{BA}\) vektorın tabıń. &  \\
\hline
8. & $(2, 3)$ hám $(4, 3)$ noqatlarınan ótiwshi tuwrı sızıqtıń teńlemesin dúziń. &  \\
\hline
9. & \(A_{1}x + B_{1}y + C_{1}z + D_{1} = 0\) hám \(Ax + By + Cz + D = 0\) tegislikleri parallel bolıwı ushın qaysı shárt orınlı bolıwı kerek? &  \\
\hline
10. & \(\bar{a} = \left\{ 4,- 2,- 4 \right\}\) hám \(\bar{b} = \left\{ 6,- 3, 2 \right\}\) vektorları berilgen, \((\bar{a} - \bar{b}) ^{2}\)-? & \\
\hline
\end{tabular}

\vspace{0.7cm}

\begin{tabular}{lll}
Tuwrı juwaplar sanı: \underline{\hspace{1cm}} & 
Bahası: \underline{\hspace{1cm}} & 
Imtixan alıwshınıń qolı: \underline{\hspace{2cm}} \\
\end{tabular}

\egroup

\newpage


\textbf{22-variant}\\

\bgroup
\def\arraystretch{1.6} % 1 is the default, change whatever you need

\begin{tabular}{|m{5.7cm}|m{9.5cm}|}
\hline
Familiyası hám atı & \\
\hline
Fakulteti  & \\
\hline
Toparı hám tálim baǵdarı  & \\
\hline
\end{tabular}

\vspace{0.7cm}

\begin{tabular}{|m{0.7cm}|m{10cm}|m{4cm}|}
\hline
№ & Soraw & Juwap \\
\hline
1. & Eki vektor qashan kollinear dep ataladı? &  \\
\hline
2. & undefined &  \\
\hline
3. & undefined &  \\
\hline
4. & undefined &  \\
\hline
5. & \(2 x + 3 y - 6 = 0\) tuwrınıń teńlemesin kesindilerde berilgen teńleme túrinde kórsetiń. &  \\
\hline
6. & \(A (4, 3), B (7, 7)\) noqatları arasındaǵı aralıqtı tabıń. &  \\
\hline
7. & \(M_{1}M_{2}\) kesindiniń ortasınıń koordinatalarınıń tabıń, eger \(M_{1} (2, 3), M_{2} (4, 7)\) bolsa. &  \\
\hline
8. & \(x + y = 0\) teńlemesi menen berilgen tuwrı sızıqtıń múyeshlik koefficientin anıqlań. &  \\
\hline
9. & \(5 x - y + 7 = 0\) hám \(3 x + 2 y = 0\) tuwrıları arasındaǵı múyeshni tabıń. &  \\
\hline
10. & Koordinatalar kósherleri hám \( 3 x + 4 y - 12 = 0 \) tuwrı sızıǵı menen shegaralanǵan úshmúyeshliktiń maydanın tabıń. & \\
\hline
\end{tabular}

\vspace{0.7cm}

\begin{tabular}{lll}
Tuwrı juwaplar sanı: \underline{\hspace{1cm}} & 
Bahası: \underline{\hspace{1cm}} & 
Imtixan alıwshınıń qolı: \underline{\hspace{2cm}} \\
\end{tabular}

\egroup

\newpage


\textbf{23-variant}\\

\bgroup
\def\arraystretch{1.6} % 1 is the default, change whatever you need

\begin{tabular}{|m{5.7cm}|m{9.5cm}|}
\hline
Familiyası hám atı & \\
\hline
Fakulteti  & \\
\hline
Toparı hám tálim baǵdarı  & \\
\hline
\end{tabular}

\vspace{0.7cm}

\begin{tabular}{|m{0.7cm}|m{10cm}|m{4cm}|}
\hline
№ & Soraw & Juwap \\
\hline
1. & Egerde \(a = \{ x_{1}; y_{1}; z_{1}\}, b = \{ x_{2}; y_{2}; z_{2}\}\) bolsa, vektor kóbeymeniń koordinatalarda ańlatılıwı qanday boladı? &  \\
\hline
2. & undefined &  \\
\hline
3. & undefined &  \\
\hline
4. & undefined &  \\
\hline
5. & \(\left| \bar{a} \right| = 8, \left| \bar{b} \right| = 5, \alpha = 60^{0}\) bolsa, \(( \bar{a}\bar{b} )\) ni tabıń. &  \\
\hline
6. & \(x - 2 y + 1 = 0\) teńlemesi menen berilgen tuwrınıń normal túrdegi teńlemesin kórsetiń. &  \\
\hline
7. & Orayı \(C (- 1;2)\) noqatında, \(A (- 2;6 )\) noqatınan ótetuǵın sheńberdiń teńlemesin dúziń. &  \\
\hline
8. & \(A_{1}x + B_{1}y + C_{1}z + D_{1} = 0\) hám tegislikleri ústpe-úst túsiwi ushın qaysı shárt orınlı bolıwı kerek? &  \\
\hline
9. & \(\frac{x^{2}}{225} - \frac{y^{2}}{64} = - 1\) giperbola fokusınıń koordinatalarınıń tabıń. &  \\
\hline
10. & \(x^{2} - 4 y^{2} + 6 x + 5 = 0\) giperbolanıń kanonikalıq teńlemesin dúziń. & \\
\hline
\end{tabular}

\vspace{0.7cm}

\begin{tabular}{lll}
Tuwrı juwaplar sanı: \underline{\hspace{1cm}} & 
Bahası: \underline{\hspace{1cm}} & 
Imtixan alıwshınıń qolı: \underline{\hspace{2cm}} \\
\end{tabular}

\egroup

\newpage


\textbf{24-variant}\\

\bgroup
\def\arraystretch{1.6} % 1 is the default, change whatever you need

\begin{tabular}{|m{5.7cm}|m{9.5cm}|}
\hline
Familiyası hám atı & \\
\hline
Fakulteti  & \\
\hline
Toparı hám tálim baǵdarı  & \\
\hline
\end{tabular}

\vspace{0.7cm}

\begin{tabular}{|m{0.7cm}|m{10cm}|m{4cm}|}
\hline
№ & Soraw & Juwap \\
\hline
1. & \(Ax + C = 0\) tuwrı sızıqtıń grafigi koordinata kósherlerine salıstırǵanda qanday jaylasqan? &  \\
\hline
2. & undefined &  \\
\hline
3. & undefined &  \\
\hline
4. & undefined &  \\
\hline
5. & \(\overline{a} = \{5,- 6, 1 \}, \overline{b} = \{ - 4, 3, 0 \} \), \(\overline{c} = \left\{ 5,- 8, 10 \right\}\) vektorları berilgen. \(2{\bar{a}}^{2} + 4{\bar{b}}^{2} - 5{\bar{c}}^{2}\) ańlatpasınıń mánisin tabıń. &  \\
\hline
6. & \(3 x - y + 5 = 0, x + 3 y - 4 = 0\) tuwrı sızıqları arasındaǵı múyeshti tabıń. &  \\
\hline
7. & \(x^{2} + y^{2} - 2 x + 4 y - 20 = 0\) sheńberdiń \(C\) orayın hám \(R\) radiusın tabıń. &  \\
\hline
8. & \(A_{1}x + B_{1}y + C_{1}z + D_{1} = 0\) hám \(Ax + By + Cz + D = 0\) tegislikleri perpendikulyar bolıwı ushın qaysı shárt orınlı bolıwı kerek? &  \\
\hline
9. & Eger \(2 a = 16, e = \frac{5}{4}\) bolsa, fokusı abscissa kósherinde, koordinata basına salıstırǵanda simmetriyalıq jaylasqan giperbolanıń teńlemesin dúziń. &  \\
\hline
10. & \(9 x^{2} + 25 y^{2} = 225\) ellipsi berilgen, ellipstiń fokusların, ekscentrisitetin tabıń. & \\
\hline
\end{tabular}

\vspace{0.7cm}

\begin{tabular}{lll}
Tuwrı juwaplar sanı: \underline{\hspace{1cm}} & 
Bahası: \underline{\hspace{1cm}} & 
Imtixan alıwshınıń qolı: \underline{\hspace{2cm}} \\
\end{tabular}

\egroup

\newpage


\textbf{25-variant}\\

\bgroup
\def\arraystretch{1.6} % 1 is the default, change whatever you need

\begin{tabular}{|m{5.7cm}|m{9.5cm}|}
\hline
Familiyası hám atı & \\
\hline
Fakulteti  & \\
\hline
Toparı hám tálim baǵdarı  & \\
\hline
\end{tabular}

\vspace{0.7cm}

\begin{tabular}{|m{0.7cm}|m{10cm}|m{4cm}|}
\hline
№ & Soraw & Juwap \\
\hline
1. & Úsh vektordıń aralas kóbeymesi ushın \((abc) = 0\) teńligi orınlı bolsa ne dep ataladı? &  \\
\hline
2. & undefined &  \\
\hline
3. & undefined &  \\
\hline
4. & undefined &  \\
\hline
5. & \(M_{1} (12; - 1)\) hám \(M_{2} (0;4)\) noqatlardıń arasındaǵı aralıqtı tabıń. &  \\
\hline
6. & \(x + y - 3 = 0\) hám \(2 x + 3 y - 8 = 0\) tuwrıları óz-ara qanday jaylasqan? &  \\
\hline
7. & \(x^{2} + y^{2} - 2 x + 4 y = 0\) sheńberdiń teńlemesin kanonikalıq túrdegi teńlemege alıp keliń. &  \\
\hline
8. & \((x + 1) ^{2} + (y - 2) ^{2} + (z + 2) ^{2} = 49\) sferanıń orayınıń koordinataların tabıń. &  \\
\hline
9. & \(3 x^{2} + 10 xy + 3 y^{2} - 2 x - 14 y - 13 = 0\) teńlemesiniń tipin anıqlań. &  \\
\hline
10. & Eger \(2 b = 24, 2 c = 10\) bolsa, onda abscissa kósherinde koordinata basına salıstırǵanda simmetriyalıq jaylasqan fokuslarǵa iye, ellipstiń teńlemesin dúziń. & \\
\hline
\end{tabular}

\vspace{0.7cm}

\begin{tabular}{lll}
Tuwrı juwaplar sanı: \underline{\hspace{1cm}} & 
Bahası: \underline{\hspace{1cm}} & 
Imtixan alıwshınıń qolı: \underline{\hspace{2cm}} \\
\end{tabular}

\egroup

\newpage


\textbf{26-variant}\\

\bgroup
\def\arraystretch{1.6} % 1 is the default, change whatever you need

\begin{tabular}{|m{5.7cm}|m{9.5cm}|}
\hline
Familiyası hám atı & \\
\hline
Fakulteti  & \\
\hline
Toparı hám tálim baǵdarı  & \\
\hline
\end{tabular}

\vspace{0.7cm}

\begin{tabular}{|m{0.7cm}|m{10cm}|m{4cm}|}
\hline
№ & Soraw & Juwap \\
\hline
1. & Tuwrı múyeshli koordinatalar sisteması dep nege aytamız? &  \\
\hline
2. & undefined &  \\
\hline
3. & undefined &  \\
\hline
4. & undefined &  \\
\hline
5. & \(\bar{a} = \left\{ 2, 1, 0 \right\}\) hám \(\bar{b} = \left\{ 1, 0,- 1 \right\}\) bolsa, \(\bar{a} - \bar{b}\) ni tabıń. &  \\
\hline
6. & \(2 x + 3 y + 4 = 0\) tuwrısına parallel hám \(M_{0} (2;1)\) noqattan ótetuǵın tuwrınıń teńlemesin dúziń. &  \\
\hline
7. & \(x + y - 12 = 0\) tuwrısı \(x^{2} + y^{2} - 2 y = 0\) sheńberge salıstırǵanda qanday jaylasqan? &  \\
\hline
8. & \(x + 2 = 0\) keńislik qanday geometriyalıq betlikti anıqlaydı? &  \\
\hline
9. & \(\frac{x^{2}}{a^{2}} + \frac{y^{2}}{b^{2}} = 1\) ellipstiń \((x_{0};y_{0})\) noqatındaǵı urınbasınıń teńlemesin tabıń. &  \\
\hline
10. & \(A (- 1;0;1),\ B (1; - 1;0)\) noqatları berilgen. \(\bar{BA}\) vektorın tabıń. & \\
\hline
\end{tabular}

\vspace{0.7cm}

\begin{tabular}{lll}
Tuwrı juwaplar sanı: \underline{\hspace{1cm}} & 
Bahası: \underline{\hspace{1cm}} & 
Imtixan alıwshınıń qolı: \underline{\hspace{2cm}} \\
\end{tabular}

\egroup

\newpage


\textbf{27-variant}\\

\bgroup
\def\arraystretch{1.6} % 1 is the default, change whatever you need

\begin{tabular}{|m{5.7cm}|m{9.5cm}|}
\hline
Familiyası hám atı & \\
\hline
Fakulteti  & \\
\hline
Toparı hám tálim baǵdarı  & \\
\hline
\end{tabular}

\vspace{0.7cm}

\begin{tabular}{|m{0.7cm}|m{10cm}|m{4cm}|}
\hline
№ & Soraw & Juwap \\
\hline
1. & Vektorlardıń kósherdegi proekciyasınıń formulası? &  \\
\hline
2. & undefined &  \\
\hline
3. & undefined &  \\
\hline
4. & undefined &  \\
\hline
5. & $(2, 3)$ hám $(4, 3)$ noqatlarınan ótiwshi tuwrı sızıqtıń teńlemesin dúziń. &  \\
\hline
6. & \(A_{1}x + B_{1}y + C_{1}z + D_{1} = 0\) hám \(Ax + By + Cz + D = 0\) tegislikleri parallel bolıwı ushın qaysı shárt orınlı bolıwı kerek? &  \\
\hline
7. & \(\bar{a} = \left\{ 4,- 2,- 4 \right\}\) hám \(\bar{b} = \left\{ 6,- 3, 2 \right\}\) vektorları berilgen, \((\bar{a} - \bar{b}) ^{2}\)-? &  \\
\hline
8. & \(2 x + 3 y - 6 = 0\) tuwrınıń teńlemesin kesindilerde berilgen teńleme túrinde kórsetiń. &  \\
\hline
9. & \(A (4, 3), B (7, 7)\) noqatları arasındaǵı aralıqtı tabıń. &  \\
\hline
10. & \(M_{1}M_{2}\) kesindiniń ortasınıń koordinatalarınıń tabıń, eger \(M_{1} (2, 3), M_{2} (4, 7)\) bolsa. & \\
\hline
\end{tabular}

\vspace{0.7cm}

\begin{tabular}{lll}
Tuwrı juwaplar sanı: \underline{\hspace{1cm}} & 
Bahası: \underline{\hspace{1cm}} & 
Imtixan alıwshınıń qolı: \underline{\hspace{2cm}} \\
\end{tabular}

\egroup

\newpage


\textbf{28-variant}\\

\bgroup
\def\arraystretch{1.6} % 1 is the default, change whatever you need

\begin{tabular}{|m{5.7cm}|m{9.5cm}|}
\hline
Familiyası hám atı & \\
\hline
Fakulteti  & \\
\hline
Toparı hám tálim baǵdarı  & \\
\hline
\end{tabular}

\vspace{0.7cm}

\begin{tabular}{|m{0.7cm}|m{10cm}|m{4cm}|}
\hline
№ & Soraw & Juwap \\
\hline
1. & Eki vektordıń skalyar kóbeymesiniń formulası? &  \\
\hline
2. & undefined &  \\
\hline
3. & undefined &  \\
\hline
4. & undefined &  \\
\hline
5. & \(x + y = 0\) teńlemesi menen berilgen tuwrı sızıqtıń múyeshlik koefficientin anıqlań. &  \\
\hline
6. & \(5 x - y + 7 = 0\) hám \(3 x + 2 y = 0\) tuwrıları arasındaǵı múyeshni tabıń. &  \\
\hline
7. & Koordinatalar kósherleri hám \( 3 x + 4 y - 12 = 0 \) tuwrı sızıǵı menen shegaralanǵan úshmúyeshliktiń maydanın tabıń. &  \\
\hline
8. & \(\left| \bar{a} \right| = 8, \left| \bar{b} \right| = 5, \alpha = 60^{0}\) bolsa, \(( \bar{a}\bar{b} )\) ni tabıń. &  \\
\hline
9. & \(x - 2 y + 1 = 0\) teńlemesi menen berilgen tuwrınıń normal túrdegi teńlemesin kórsetiń. &  \\
\hline
10. & Orayı \(C (- 1;2)\) noqatında, \(A (- 2;6 )\) noqatınan ótetuǵın sheńberdiń teńlemesin dúziń. & \\
\hline
\end{tabular}

\vspace{0.7cm}

\begin{tabular}{lll}
Tuwrı juwaplar sanı: \underline{\hspace{1cm}} & 
Bahası: \underline{\hspace{1cm}} & 
Imtixan alıwshınıń qolı: \underline{\hspace{2cm}} \\
\end{tabular}

\egroup

\newpage


\textbf{29-variant}\\

\bgroup
\def\arraystretch{1.6} % 1 is the default, change whatever you need

\begin{tabular}{|m{5.7cm}|m{9.5cm}|}
\hline
Familiyası hám atı & \\
\hline
Fakulteti  & \\
\hline
Toparı hám tálim baǵdarı  & \\
\hline
\end{tabular}

\vspace{0.7cm}

\begin{tabular}{|m{0.7cm}|m{10cm}|m{4cm}|}
\hline
№ & Soraw & Juwap \\
\hline
1. & \(OXY\) tegisliginiń teńlemesi? &  \\
\hline
2. & undefined &  \\
\hline
3. & undefined &  \\
\hline
4. & undefined &  \\
\hline
5. & \(A_{1}x + B_{1}y + C_{1}z + D_{1} = 0\) hám tegislikleri ústpe-úst túsiwi ushın qaysı shárt orınlı bolıwı kerek? &  \\
\hline
6. & \(\frac{x^{2}}{225} - \frac{y^{2}}{64} = - 1\) giperbola fokusınıń koordinatalarınıń tabıń. &  \\
\hline
7. & \(x^{2} - 4 y^{2} + 6 x + 5 = 0\) giperbolanıń kanonikalıq teńlemesin dúziń. &  \\
\hline
8. & \(\overline{a} = \{5,- 6, 1 \}, \overline{b} = \{ - 4, 3, 0 \} \), \(\overline{c} = \left\{ 5,- 8, 10 \right\}\) vektorları berilgen. \(2{\bar{a}}^{2} + 4{\bar{b}}^{2} - 5{\bar{c}}^{2}\) ańlatpasınıń mánisin tabıń. &  \\
\hline
9. & \(3 x - y + 5 = 0, x + 3 y - 4 = 0\) tuwrı sızıqları arasındaǵı múyeshti tabıń. &  \\
\hline
10. & \(x^{2} + y^{2} - 2 x + 4 y - 20 = 0\) sheńberdiń \(C\) orayın hám \(R\) radiusın tabıń. & \\
\hline
\end{tabular}

\vspace{0.7cm}

\begin{tabular}{lll}
Tuwrı juwaplar sanı: \underline{\hspace{1cm}} & 
Bahası: \underline{\hspace{1cm}} & 
Imtixan alıwshınıń qolı: \underline{\hspace{2cm}} \\
\end{tabular}

\egroup

\newpage


\textbf{30-variant}\\

\bgroup
\def\arraystretch{1.6} % 1 is the default, change whatever you need

\begin{tabular}{|m{5.7cm}|m{9.5cm}|}
\hline
Familiyası hám atı & \\
\hline
Fakulteti  & \\
\hline
Toparı hám tálim baǵdarı  & \\
\hline
\end{tabular}

\vspace{0.7cm}

\begin{tabular}{|m{0.7cm}|m{10cm}|m{4cm}|}
\hline
№ & Soraw & Juwap \\
\hline
1. & Eki tuwrı sızıq arasındaǵı múyeshti tabıw formulası? &  \\
\hline
2. & undefined &  \\
\hline
3. & undefined &  \\
\hline
4. & undefined &  \\
\hline
5. & \(A_{1}x + B_{1}y + C_{1}z + D_{1} = 0\) hám \(Ax + By + Cz + D = 0\) tegislikleri perpendikulyar bolıwı ushın qaysı shárt orınlı bolıwı kerek? &  \\
\hline
6. & Eger \(2 a = 16, e = \frac{5}{4}\) bolsa, fokusı abscissa kósherinde, koordinata basına salıstırǵanda simmetriyalıq jaylasqan giperbolanıń teńlemesin dúziń. &  \\
\hline
7. & \(9 x^{2} + 25 y^{2} = 225\) ellipsi berilgen, ellipstiń fokusların, ekscentrisitetin tabıń. &  \\
\hline
8. & \(M_{1} (12; - 1)\) hám \(M_{2} (0;4)\) noqatlardıń arasındaǵı aralıqtı tabıń. &  \\
\hline
9. & \(x + y - 3 = 0\) hám \(2 x + 3 y - 8 = 0\) tuwrıları óz-ara qanday jaylasqan? &  \\
\hline
10. & \(x^{2} + y^{2} - 2 x + 4 y = 0\) sheńberdiń teńlemesin kanonikalıq túrdegi teńlemege alıp keliń. & \\
\hline
\end{tabular}

\vspace{0.7cm}

\begin{tabular}{lll}
Tuwrı juwaplar sanı: \underline{\hspace{1cm}} & 
Bahası: \underline{\hspace{1cm}} & 
Imtixan alıwshınıń qolı: \underline{\hspace{2cm}} \\
\end{tabular}

\egroup

\newpage


\textbf{31-variant}\\

\bgroup
\def\arraystretch{1.6} % 1 is the default, change whatever you need

\begin{tabular}{|m{5.7cm}|m{9.5cm}|}
\hline
Familiyası hám atı & \\
\hline
Fakulteti  & \\
\hline
Toparı hám tálim baǵdarı  & \\
\hline
\end{tabular}

\vspace{0.7cm}

\begin{tabular}{|m{0.7cm}|m{10cm}|m{4cm}|}
\hline
№ & Soraw & Juwap \\
\hline
1. & \(OY\) kósheriniń teńlemesi? &  \\
\hline
2. & undefined &  \\
\hline
3. & undefined &  \\
\hline
4. & undefined &  \\
\hline
5. & \((x + 1) ^{2} + (y - 2) ^{2} + (z + 2) ^{2} = 49\) sferanıń orayınıń koordinataların tabıń. &  \\
\hline
6. & \(3 x^{2} + 10 xy + 3 y^{2} - 2 x - 14 y - 13 = 0\) teńlemesiniń tipin anıqlań. &  \\
\hline
7. & Eger \(2 b = 24, 2 c = 10\) bolsa, onda abscissa kósherinde koordinata basına salıstırǵanda simmetriyalıq jaylasqan fokuslarǵa iye, ellipstiń teńlemesin dúziń. &  \\
\hline
8. & \(\bar{a} = \left\{ 2, 1, 0 \right\}\) hám \(\bar{b} = \left\{ 1, 0,- 1 \right\}\) bolsa, \(\bar{a} - \bar{b}\) ni tabıń. &  \\
\hline
9. & \(2 x + 3 y + 4 = 0\) tuwrısına parallel hám \(M_{0} (2;1)\) noqattan ótetuǵın tuwrınıń teńlemesin dúziń. &  \\
\hline
10. & \(x + y - 12 = 0\) tuwrısı \(x^{2} + y^{2} - 2 y = 0\) sheńberge salıstırǵanda qanday jaylasqan? & \\
\hline
\end{tabular}

\vspace{0.7cm}

\begin{tabular}{lll}
Tuwrı juwaplar sanı: \underline{\hspace{1cm}} & 
Bahası: \underline{\hspace{1cm}} & 
Imtixan alıwshınıń qolı: \underline{\hspace{2cm}} \\
\end{tabular}

\egroup

\newpage


\textbf{32-variant}\\

\bgroup
\def\arraystretch{1.6} % 1 is the default, change whatever you need

\begin{tabular}{|m{5.7cm}|m{9.5cm}|}
\hline
Familiyası hám atı & \\
\hline
Fakulteti  & \\
\hline
Toparı hám tálim baǵdarı  & \\
\hline
\end{tabular}

\vspace{0.7cm}

\begin{tabular}{|m{0.7cm}|m{10cm}|m{4cm}|}
\hline
№ & Soraw & Juwap \\
\hline
1. & \(Ax + By + D = 0\) teńlemesi arqalı ... tegisliktiń teńlemesi berilgen? &  \\
\hline
2. & undefined &  \\
\hline
3. & undefined &  \\
\hline
4. & undefined &  \\
\hline
5. & \(x + 2 = 0\) keńislik qanday geometriyalıq betlikti anıqlaydı? &  \\
\hline
6. & \(\frac{x^{2}}{a^{2}} + \frac{y^{2}}{b^{2}} = 1\) ellipstiń \((x_{0};y_{0})\) noqatındaǵı urınbasınıń teńlemesin tabıń. &  \\
\hline
7. & \(A (- 1;0;1),\ B (1; - 1;0)\) noqatları berilgen. \(\bar{BA}\) vektorın tabıń. &  \\
\hline
8. & $(2, 3)$ hám $(4, 3)$ noqatlarınan ótiwshi tuwrı sızıqtıń teńlemesin dúziń. &  \\
\hline
9. & \(A_{1}x + B_{1}y + C_{1}z + D_{1} = 0\) hám \(Ax + By + Cz + D = 0\) tegislikleri parallel bolıwı ushın qaysı shárt orınlı bolıwı kerek? &  \\
\hline
10. & \(\bar{a} = \left\{ 4,- 2,- 4 \right\}\) hám \(\bar{b} = \left\{ 6,- 3, 2 \right\}\) vektorları berilgen, \((\bar{a} - \bar{b}) ^{2}\)-? & \\
\hline
\end{tabular}

\vspace{0.7cm}

\begin{tabular}{lll}
Tuwrı juwaplar sanı: \underline{\hspace{1cm}} & 
Bahası: \underline{\hspace{1cm}} & 
Imtixan alıwshınıń qolı: \underline{\hspace{2cm}} \\
\end{tabular}

\egroup

\newpage


\textbf{33-variant}\\

\bgroup
\def\arraystretch{1.6} % 1 is the default, change whatever you need

\begin{tabular}{|m{5.7cm}|m{9.5cm}|}
\hline
Familiyası hám atı & \\
\hline
Fakulteti  & \\
\hline
Toparı hám tálim baǵdarı  & \\
\hline
\end{tabular}

\vspace{0.7cm}

\begin{tabular}{|m{0.7cm}|m{10cm}|m{4cm}|}
\hline
№ & Soraw & Juwap \\
\hline
1. & Vektorlardı qosıw tómendegi qaysı qásiyetke iye emes? &  \\
\hline
2. & undefined &  \\
\hline
3. & undefined &  \\
\hline
4. & undefined &  \\
\hline
5. & \(2 x + 3 y - 6 = 0\) tuwrınıń teńlemesin kesindilerde berilgen teńleme túrinde kórsetiń. &  \\
\hline
6. & \(A (4, 3), B (7, 7)\) noqatları arasındaǵı aralıqtı tabıń. &  \\
\hline
7. & \(M_{1}M_{2}\) kesindiniń ortasınıń koordinatalarınıń tabıń, eger \(M_{1} (2, 3), M_{2} (4, 7)\) bolsa. &  \\
\hline
8. & \(x + y = 0\) teńlemesi menen berilgen tuwrı sızıqtıń múyeshlik koefficientin anıqlań. &  \\
\hline
9. & \(5 x - y + 7 = 0\) hám \(3 x + 2 y = 0\) tuwrıları arasındaǵı múyeshni tabıń. &  \\
\hline
10. & Koordinatalar kósherleri hám \( 3 x + 4 y - 12 = 0 \) tuwrı sızıǵı menen shegaralanǵan úshmúyeshliktiń maydanın tabıń. & \\
\hline
\end{tabular}

\vspace{0.7cm}

\begin{tabular}{lll}
Tuwrı juwaplar sanı: \underline{\hspace{1cm}} & 
Bahası: \underline{\hspace{1cm}} & 
Imtixan alıwshınıń qolı: \underline{\hspace{2cm}} \\
\end{tabular}

\egroup

\newpage


\textbf{34-variant}\\

\bgroup
\def\arraystretch{1.6} % 1 is the default, change whatever you need

\begin{tabular}{|m{5.7cm}|m{9.5cm}|}
\hline
Familiyası hám atı & \\
\hline
Fakulteti  & \\
\hline
Toparı hám tálim baǵdarı  & \\
\hline
\end{tabular}

\vspace{0.7cm}

\begin{tabular}{|m{0.7cm}|m{10cm}|m{4cm}|}
\hline
№ & Soraw & Juwap \\
\hline
1. & Giperbolanıń kanonikalıq teńlemesi? &  \\
\hline
2. & undefined &  \\
\hline
3. & undefined &  \\
\hline
4. & undefined &  \\
\hline
5. & \(\left| \bar{a} \right| = 8, \left| \bar{b} \right| = 5, \alpha = 60^{0}\) bolsa, \(( \bar{a}\bar{b} )\) ni tabıń. &  \\
\hline
6. & \(x - 2 y + 1 = 0\) teńlemesi menen berilgen tuwrınıń normal túrdegi teńlemesin kórsetiń. &  \\
\hline
7. & Orayı \(C (- 1;2)\) noqatında, \(A (- 2;6 )\) noqatınan ótetuǵın sheńberdiń teńlemesin dúziń. &  \\
\hline
8. & \(A_{1}x + B_{1}y + C_{1}z + D_{1} = 0\) hám tegislikleri ústpe-úst túsiwi ushın qaysı shárt orınlı bolıwı kerek? &  \\
\hline
9. & \(\frac{x^{2}}{225} - \frac{y^{2}}{64} = - 1\) giperbola fokusınıń koordinatalarınıń tabıń. &  \\
\hline
10. & \(x^{2} - 4 y^{2} + 6 x + 5 = 0\) giperbolanıń kanonikalıq teńlemesin dúziń. & \\
\hline
\end{tabular}

\vspace{0.7cm}

\begin{tabular}{lll}
Tuwrı juwaplar sanı: \underline{\hspace{1cm}} & 
Bahası: \underline{\hspace{1cm}} & 
Imtixan alıwshınıń qolı: \underline{\hspace{2cm}} \\
\end{tabular}

\egroup

\newpage


\textbf{35-variant}\\

\bgroup
\def\arraystretch{1.6} % 1 is the default, change whatever you need

\begin{tabular}{|m{5.7cm}|m{9.5cm}|}
\hline
Familiyası hám atı & \\
\hline
Fakulteti  & \\
\hline
Toparı hám tálim baǵdarı  & \\
\hline
\end{tabular}

\vspace{0.7cm}

\begin{tabular}{|m{0.7cm}|m{10cm}|m{4cm}|}
\hline
№ & Soraw & Juwap \\
\hline
1. & \(\frac{x^{2}}{a^{2}} - \frac{y^{2}}{b^{2}} = 1\) giperbolanıń \((x_{0};y_{0})\) noqatındaǵı urınbasınıń teńlemesin kórsetiń. &  \\
\hline
2. & undefined &  \\
\hline
3. & undefined &  \\
\hline
4. & undefined &  \\
\hline
5. & \(\overline{a} = \{5,- 6, 1 \}, \overline{b} = \{ - 4, 3, 0 \} \), \(\overline{c} = \left\{ 5,- 8, 10 \right\}\) vektorları berilgen. \(2{\bar{a}}^{2} + 4{\bar{b}}^{2} - 5{\bar{c}}^{2}\) ańlatpasınıń mánisin tabıń. &  \\
\hline
6. & \(3 x - y + 5 = 0, x + 3 y - 4 = 0\) tuwrı sızıqları arasındaǵı múyeshti tabıń. &  \\
\hline
7. & \(x^{2} + y^{2} - 2 x + 4 y - 20 = 0\) sheńberdiń \(C\) orayın hám \(R\) radiusın tabıń. &  \\
\hline
8. & \(A_{1}x + B_{1}y + C_{1}z + D_{1} = 0\) hám \(Ax + By + Cz + D = 0\) tegislikleri perpendikulyar bolıwı ushın qaysı shárt orınlı bolıwı kerek? &  \\
\hline
9. & Eger \(2 a = 16, e = \frac{5}{4}\) bolsa, fokusı abscissa kósherinde, koordinata basına salıstırǵanda simmetriyalıq jaylasqan giperbolanıń teńlemesin dúziń. &  \\
\hline
10. & \(9 x^{2} + 25 y^{2} = 225\) ellipsi berilgen, ellipstiń fokusların, ekscentrisitetin tabıń. & \\
\hline
\end{tabular}

\vspace{0.7cm}

\begin{tabular}{lll}
Tuwrı juwaplar sanı: \underline{\hspace{1cm}} & 
Bahası: \underline{\hspace{1cm}} & 
Imtixan alıwshınıń qolı: \underline{\hspace{2cm}} \\
\end{tabular}

\egroup

\newpage


\textbf{36-variant}\\

\bgroup
\def\arraystretch{1.6} % 1 is the default, change whatever you need

\begin{tabular}{|m{5.7cm}|m{9.5cm}|}
\hline
Familiyası hám atı & \\
\hline
Fakulteti  & \\
\hline
Toparı hám tálim baǵdarı  & \\
\hline
\end{tabular}

\vspace{0.7cm}

\begin{tabular}{|m{0.7cm}|m{10cm}|m{4cm}|}
\hline
№ & Soraw & Juwap \\
\hline
1. & Vektorlardı qosıw koordinatalarda qanday formula menen anıqlanadı? &  \\
\hline
2. & undefined &  \\
\hline
3. & undefined &  \\
\hline
4. & undefined &  \\
\hline
5. & \(M_{1} (12; - 1)\) hám \(M_{2} (0;4)\) noqatlardıń arasındaǵı aralıqtı tabıń. &  \\
\hline
6. & \(x + y - 3 = 0\) hám \(2 x + 3 y - 8 = 0\) tuwrıları óz-ara qanday jaylasqan? &  \\
\hline
7. & \(x^{2} + y^{2} - 2 x + 4 y = 0\) sheńberdiń teńlemesin kanonikalıq túrdegi teńlemege alıp keliń. &  \\
\hline
8. & \((x + 1) ^{2} + (y - 2) ^{2} + (z + 2) ^{2} = 49\) sferanıń orayınıń koordinataların tabıń. &  \\
\hline
9. & \(3 x^{2} + 10 xy + 3 y^{2} - 2 x - 14 y - 13 = 0\) teńlemesiniń tipin anıqlań. &  \\
\hline
10. & Eger \(2 b = 24, 2 c = 10\) bolsa, onda abscissa kósherinde koordinata basına salıstırǵanda simmetriyalıq jaylasqan fokuslarǵa iye, ellipstiń teńlemesin dúziń. & \\
\hline
\end{tabular}

\vspace{0.7cm}

\begin{tabular}{lll}
Tuwrı juwaplar sanı: \underline{\hspace{1cm}} & 
Bahası: \underline{\hspace{1cm}} & 
Imtixan alıwshınıń qolı: \underline{\hspace{2cm}} \\
\end{tabular}

\egroup

\newpage


\textbf{37-variant}\\

\bgroup
\def\arraystretch{1.6} % 1 is the default, change whatever you need

\begin{tabular}{|m{5.7cm}|m{9.5cm}|}
\hline
Familiyası hám atı & \\
\hline
Fakulteti  & \\
\hline
Toparı hám tálim baǵdarı  & \\
\hline
\end{tabular}

\vspace{0.7cm}

\begin{tabular}{|m{0.7cm}|m{10cm}|m{4cm}|}
\hline
№ & Soraw & Juwap \\
\hline
1. & Eki vektordıń vektor kóbeymesiniń uzınlıǵın tabıw formulası? &  \\
\hline
2. & undefined &  \\
\hline
3. & undefined &  \\
\hline
4. & undefined &  \\
\hline
5. & \(\bar{a} = \left\{ 2, 1, 0 \right\}\) hám \(\bar{b} = \left\{ 1, 0,- 1 \right\}\) bolsa, \(\bar{a} - \bar{b}\) ni tabıń. &  \\
\hline
6. & \(2 x + 3 y + 4 = 0\) tuwrısına parallel hám \(M_{0} (2;1)\) noqattan ótetuǵın tuwrınıń teńlemesin dúziń. &  \\
\hline
7. & \(x + y - 12 = 0\) tuwrısı \(x^{2} + y^{2} - 2 y = 0\) sheńberge salıstırǵanda qanday jaylasqan? &  \\
\hline
8. & \(x + 2 = 0\) keńislik qanday geometriyalıq betlikti anıqlaydı? &  \\
\hline
9. & \(\frac{x^{2}}{a^{2}} + \frac{y^{2}}{b^{2}} = 1\) ellipstiń \((x_{0};y_{0})\) noqatındaǵı urınbasınıń teńlemesin tabıń. &  \\
\hline
10. & \(A (- 1;0;1),\ B (1; - 1;0)\) noqatları berilgen. \(\bar{BA}\) vektorın tabıń. & \\
\hline
\end{tabular}

\vspace{0.7cm}

\begin{tabular}{lll}
Tuwrı juwaplar sanı: \underline{\hspace{1cm}} & 
Bahası: \underline{\hspace{1cm}} & 
Imtixan alıwshınıń qolı: \underline{\hspace{2cm}} \\
\end{tabular}

\egroup

\newpage


\textbf{38-variant}\\

\bgroup
\def\arraystretch{1.6} % 1 is the default, change whatever you need

\begin{tabular}{|m{5.7cm}|m{9.5cm}|}
\hline
Familiyası hám atı & \\
\hline
Fakulteti  & \\
\hline
Toparı hám tálim baǵdarı  & \\
\hline
\end{tabular}

\vspace{0.7cm}

\begin{tabular}{|m{0.7cm}|m{10cm}|m{4cm}|}
\hline
№ & Soraw & Juwap \\
\hline
1. & Tegislikdegi qálegen noqattan berilgen eki noqatqa shekemgi bolǵan aralıqlardıń ayırmasınıń modulı ózgermeytuǵın bolǵan noqatlardıń geometriyalıq ornı ne dep ataladı? &  \\
\hline
2. & undefined &  \\
\hline
3. & undefined &  \\
\hline
4. & undefined &  \\
\hline
5. & $(2, 3)$ hám $(4, 3)$ noqatlarınan ótiwshi tuwrı sızıqtıń teńlemesin dúziń. &  \\
\hline
6. & \(A_{1}x + B_{1}y + C_{1}z + D_{1} = 0\) hám \(Ax + By + Cz + D = 0\) tegislikleri parallel bolıwı ushın qaysı shárt orınlı bolıwı kerek? &  \\
\hline
7. & \(\bar{a} = \left\{ 4,- 2,- 4 \right\}\) hám \(\bar{b} = \left\{ 6,- 3, 2 \right\}\) vektorları berilgen, \((\bar{a} - \bar{b}) ^{2}\)-? &  \\
\hline
8. & \(2 x + 3 y - 6 = 0\) tuwrınıń teńlemesin kesindilerde berilgen teńleme túrinde kórsetiń. &  \\
\hline
9. & \(A (4, 3), B (7, 7)\) noqatları arasındaǵı aralıqtı tabıń. &  \\
\hline
10. & \(M_{1}M_{2}\) kesindiniń ortasınıń koordinatalarınıń tabıń, eger \(M_{1} (2, 3), M_{2} (4, 7)\) bolsa. & \\
\hline
\end{tabular}

\vspace{0.7cm}

\begin{tabular}{lll}
Tuwrı juwaplar sanı: \underline{\hspace{1cm}} & 
Bahası: \underline{\hspace{1cm}} & 
Imtixan alıwshınıń qolı: \underline{\hspace{2cm}} \\
\end{tabular}

\egroup

\newpage


\textbf{39-variant}\\

\bgroup
\def\arraystretch{1.6} % 1 is the default, change whatever you need

\begin{tabular}{|m{5.7cm}|m{9.5cm}|}
\hline
Familiyası hám atı & \\
\hline
Fakulteti  & \\
\hline
Toparı hám tálim baǵdarı  & \\
\hline
\end{tabular}

\vspace{0.7cm}

\begin{tabular}{|m{0.7cm}|m{10cm}|m{4cm}|}
\hline
№ & Soraw & Juwap \\
\hline
1. & Eki vektor qashan kollinear dep ataladı? &  \\
\hline
2. & undefined &  \\
\hline
3. & undefined &  \\
\hline
4. & undefined &  \\
\hline
5. & \(x + y = 0\) teńlemesi menen berilgen tuwrı sızıqtıń múyeshlik koefficientin anıqlań. &  \\
\hline
6. & \(5 x - y + 7 = 0\) hám \(3 x + 2 y = 0\) tuwrıları arasındaǵı múyeshni tabıń. &  \\
\hline
7. & Koordinatalar kósherleri hám \( 3 x + 4 y - 12 = 0 \) tuwrı sızıǵı menen shegaralanǵan úshmúyeshliktiń maydanın tabıń. &  \\
\hline
8. & \(\left| \bar{a} \right| = 8, \left| \bar{b} \right| = 5, \alpha = 60^{0}\) bolsa, \(( \bar{a}\bar{b} )\) ni tabıń. &  \\
\hline
9. & \(x - 2 y + 1 = 0\) teńlemesi menen berilgen tuwrınıń normal túrdegi teńlemesin kórsetiń. &  \\
\hline
10. & Orayı \(C (- 1;2)\) noqatında, \(A (- 2;6 )\) noqatınan ótetuǵın sheńberdiń teńlemesin dúziń. & \\
\hline
\end{tabular}

\vspace{0.7cm}

\begin{tabular}{lll}
Tuwrı juwaplar sanı: \underline{\hspace{1cm}} & 
Bahası: \underline{\hspace{1cm}} & 
Imtixan alıwshınıń qolı: \underline{\hspace{2cm}} \\
\end{tabular}

\egroup

\newpage


\textbf{40-variant}\\

\bgroup
\def\arraystretch{1.6} % 1 is the default, change whatever you need

\begin{tabular}{|m{5.7cm}|m{9.5cm}|}
\hline
Familiyası hám atı & \\
\hline
Fakulteti  & \\
\hline
Toparı hám tálim baǵdarı  & \\
\hline
\end{tabular}

\vspace{0.7cm}

\begin{tabular}{|m{0.7cm}|m{10cm}|m{4cm}|}
\hline
№ & Soraw & Juwap \\
\hline
1. & Egerde \(a = \{ x_{1}; y_{1}; z_{1}\}, b = \{ x_{2}; y_{2}; z_{2}\}\) bolsa, vektor kóbeymeniń koordinatalarda ańlatılıwı qanday boladı? &  \\
\hline
2. & undefined &  \\
\hline
3. & undefined &  \\
\hline
4. & undefined &  \\
\hline
5. & \(A_{1}x + B_{1}y + C_{1}z + D_{1} = 0\) hám tegislikleri ústpe-úst túsiwi ushın qaysı shárt orınlı bolıwı kerek? &  \\
\hline
6. & \(\frac{x^{2}}{225} - \frac{y^{2}}{64} = - 1\) giperbola fokusınıń koordinatalarınıń tabıń. &  \\
\hline
7. & \(x^{2} - 4 y^{2} + 6 x + 5 = 0\) giperbolanıń kanonikalıq teńlemesin dúziń. &  \\
\hline
8. & \(\overline{a} = \{5,- 6, 1 \}, \overline{b} = \{ - 4, 3, 0 \} \), \(\overline{c} = \left\{ 5,- 8, 10 \right\}\) vektorları berilgen. \(2{\bar{a}}^{2} + 4{\bar{b}}^{2} - 5{\bar{c}}^{2}\) ańlatpasınıń mánisin tabıń. &  \\
\hline
9. & \(3 x - y + 5 = 0, x + 3 y - 4 = 0\) tuwrı sızıqları arasındaǵı múyeshti tabıń. &  \\
\hline
10. & \(x^{2} + y^{2} - 2 x + 4 y - 20 = 0\) sheńberdiń \(C\) orayın hám \(R\) radiusın tabıń. & \\
\hline
\end{tabular}

\vspace{0.7cm}

\begin{tabular}{lll}
Tuwrı juwaplar sanı: \underline{\hspace{1cm}} & 
Bahası: \underline{\hspace{1cm}} & 
Imtixan alıwshınıń qolı: \underline{\hspace{2cm}} \\
\end{tabular}

\egroup

\newpage


\textbf{41-variant}\\

\bgroup
\def\arraystretch{1.6} % 1 is the default, change whatever you need

\begin{tabular}{|m{5.7cm}|m{9.5cm}|}
\hline
Familiyası hám atı & \\
\hline
Fakulteti  & \\
\hline
Toparı hám tálim baǵdarı  & \\
\hline
\end{tabular}

\vspace{0.7cm}

\begin{tabular}{|m{0.7cm}|m{10cm}|m{4cm}|}
\hline
№ & Soraw & Juwap \\
\hline
1. & \(Ax + C = 0\) tuwrı sızıqtıń grafigi koordinata kósherlerine salıstırǵanda qanday jaylasqan? &  \\
\hline
2. & undefined &  \\
\hline
3. & undefined &  \\
\hline
4. & undefined &  \\
\hline
5. & \(A_{1}x + B_{1}y + C_{1}z + D_{1} = 0\) hám \(Ax + By + Cz + D = 0\) tegislikleri perpendikulyar bolıwı ushın qaysı shárt orınlı bolıwı kerek? &  \\
\hline
6. & Eger \(2 a = 16, e = \frac{5}{4}\) bolsa, fokusı abscissa kósherinde, koordinata basına salıstırǵanda simmetriyalıq jaylasqan giperbolanıń teńlemesin dúziń. &  \\
\hline
7. & \(9 x^{2} + 25 y^{2} = 225\) ellipsi berilgen, ellipstiń fokusların, ekscentrisitetin tabıń. &  \\
\hline
8. & \(M_{1} (12; - 1)\) hám \(M_{2} (0;4)\) noqatlardıń arasındaǵı aralıqtı tabıń. &  \\
\hline
9. & \(x + y - 3 = 0\) hám \(2 x + 3 y - 8 = 0\) tuwrıları óz-ara qanday jaylasqan? &  \\
\hline
10. & \(x^{2} + y^{2} - 2 x + 4 y = 0\) sheńberdiń teńlemesin kanonikalıq túrdegi teńlemege alıp keliń. & \\
\hline
\end{tabular}

\vspace{0.7cm}

\begin{tabular}{lll}
Tuwrı juwaplar sanı: \underline{\hspace{1cm}} & 
Bahası: \underline{\hspace{1cm}} & 
Imtixan alıwshınıń qolı: \underline{\hspace{2cm}} \\
\end{tabular}

\egroup

\newpage


\textbf{42-variant}\\

\bgroup
\def\arraystretch{1.6} % 1 is the default, change whatever you need

\begin{tabular}{|m{5.7cm}|m{9.5cm}|}
\hline
Familiyası hám atı & \\
\hline
Fakulteti  & \\
\hline
Toparı hám tálim baǵdarı  & \\
\hline
\end{tabular}

\vspace{0.7cm}

\begin{tabular}{|m{0.7cm}|m{10cm}|m{4cm}|}
\hline
№ & Soraw & Juwap \\
\hline
1. & Úsh vektordıń aralas kóbeymesi ushın \((abc) = 0\) teńligi orınlı bolsa ne dep ataladı? &  \\
\hline
2. & undefined &  \\
\hline
3. & undefined &  \\
\hline
4. & undefined &  \\
\hline
5. & \((x + 1) ^{2} + (y - 2) ^{2} + (z + 2) ^{2} = 49\) sferanıń orayınıń koordinataların tabıń. &  \\
\hline
6. & \(3 x^{2} + 10 xy + 3 y^{2} - 2 x - 14 y - 13 = 0\) teńlemesiniń tipin anıqlań. &  \\
\hline
7. & Eger \(2 b = 24, 2 c = 10\) bolsa, onda abscissa kósherinde koordinata basına salıstırǵanda simmetriyalıq jaylasqan fokuslarǵa iye, ellipstiń teńlemesin dúziń. &  \\
\hline
8. & \(\bar{a} = \left\{ 2, 1, 0 \right\}\) hám \(\bar{b} = \left\{ 1, 0,- 1 \right\}\) bolsa, \(\bar{a} - \bar{b}\) ni tabıń. &  \\
\hline
9. & \(2 x + 3 y + 4 = 0\) tuwrısına parallel hám \(M_{0} (2;1)\) noqattan ótetuǵın tuwrınıń teńlemesin dúziń. &  \\
\hline
10. & \(x + y - 12 = 0\) tuwrısı \(x^{2} + y^{2} - 2 y = 0\) sheńberge salıstırǵanda qanday jaylasqan? & \\
\hline
\end{tabular}

\vspace{0.7cm}

\begin{tabular}{lll}
Tuwrı juwaplar sanı: \underline{\hspace{1cm}} & 
Bahası: \underline{\hspace{1cm}} & 
Imtixan alıwshınıń qolı: \underline{\hspace{2cm}} \\
\end{tabular}

\egroup

\newpage


\textbf{43-variant}\\

\bgroup
\def\arraystretch{1.6} % 1 is the default, change whatever you need

\begin{tabular}{|m{5.7cm}|m{9.5cm}|}
\hline
Familiyası hám atı & \\
\hline
Fakulteti  & \\
\hline
Toparı hám tálim baǵdarı  & \\
\hline
\end{tabular}

\vspace{0.7cm}

\begin{tabular}{|m{0.7cm}|m{10cm}|m{4cm}|}
\hline
№ & Soraw & Juwap \\
\hline
1. & Tuwrı múyeshli koordinatalar sisteması dep nege aytamız? &  \\
\hline
2. & undefined &  \\
\hline
3. & undefined &  \\
\hline
4. & undefined &  \\
\hline
5. & \(x + 2 = 0\) keńislik qanday geometriyalıq betlikti anıqlaydı? &  \\
\hline
6. & \(\frac{x^{2}}{a^{2}} + \frac{y^{2}}{b^{2}} = 1\) ellipstiń \((x_{0};y_{0})\) noqatındaǵı urınbasınıń teńlemesin tabıń. &  \\
\hline
7. & \(A (- 1;0;1),\ B (1; - 1;0)\) noqatları berilgen. \(\bar{BA}\) vektorın tabıń. &  \\
\hline
8. & $(2, 3)$ hám $(4, 3)$ noqatlarınan ótiwshi tuwrı sızıqtıń teńlemesin dúziń. &  \\
\hline
9. & \(A_{1}x + B_{1}y + C_{1}z + D_{1} = 0\) hám \(Ax + By + Cz + D = 0\) tegislikleri parallel bolıwı ushın qaysı shárt orınlı bolıwı kerek? &  \\
\hline
10. & \(\bar{a} = \left\{ 4,- 2,- 4 \right\}\) hám \(\bar{b} = \left\{ 6,- 3, 2 \right\}\) vektorları berilgen, \((\bar{a} - \bar{b}) ^{2}\)-? & \\
\hline
\end{tabular}

\vspace{0.7cm}

\begin{tabular}{lll}
Tuwrı juwaplar sanı: \underline{\hspace{1cm}} & 
Bahası: \underline{\hspace{1cm}} & 
Imtixan alıwshınıń qolı: \underline{\hspace{2cm}} \\
\end{tabular}

\egroup

\newpage


\textbf{44-variant}\\

\bgroup
\def\arraystretch{1.6} % 1 is the default, change whatever you need

\begin{tabular}{|m{5.7cm}|m{9.5cm}|}
\hline
Familiyası hám atı & \\
\hline
Fakulteti  & \\
\hline
Toparı hám tálim baǵdarı  & \\
\hline
\end{tabular}

\vspace{0.7cm}

\begin{tabular}{|m{0.7cm}|m{10cm}|m{4cm}|}
\hline
№ & Soraw & Juwap \\
\hline
1. & Vektorlardıń kósherdegi proekciyasınıń formulası? &  \\
\hline
2. & undefined &  \\
\hline
3. & undefined &  \\
\hline
4. & undefined &  \\
\hline
5. & \(2 x + 3 y - 6 = 0\) tuwrınıń teńlemesin kesindilerde berilgen teńleme túrinde kórsetiń. &  \\
\hline
6. & \(A (4, 3), B (7, 7)\) noqatları arasındaǵı aralıqtı tabıń. &  \\
\hline
7. & \(M_{1}M_{2}\) kesindiniń ortasınıń koordinatalarınıń tabıń, eger \(M_{1} (2, 3), M_{2} (4, 7)\) bolsa. &  \\
\hline
8. & \(x + y = 0\) teńlemesi menen berilgen tuwrı sızıqtıń múyeshlik koefficientin anıqlań. &  \\
\hline
9. & \(5 x - y + 7 = 0\) hám \(3 x + 2 y = 0\) tuwrıları arasındaǵı múyeshni tabıń. &  \\
\hline
10. & Koordinatalar kósherleri hám \( 3 x + 4 y - 12 = 0 \) tuwrı sızıǵı menen shegaralanǵan úshmúyeshliktiń maydanın tabıń. & \\
\hline
\end{tabular}

\vspace{0.7cm}

\begin{tabular}{lll}
Tuwrı juwaplar sanı: \underline{\hspace{1cm}} & 
Bahası: \underline{\hspace{1cm}} & 
Imtixan alıwshınıń qolı: \underline{\hspace{2cm}} \\
\end{tabular}

\egroup

\newpage


\textbf{45-variant}\\

\bgroup
\def\arraystretch{1.6} % 1 is the default, change whatever you need

\begin{tabular}{|m{5.7cm}|m{9.5cm}|}
\hline
Familiyası hám atı & \\
\hline
Fakulteti  & \\
\hline
Toparı hám tálim baǵdarı  & \\
\hline
\end{tabular}

\vspace{0.7cm}

\begin{tabular}{|m{0.7cm}|m{10cm}|m{4cm}|}
\hline
№ & Soraw & Juwap \\
\hline
1. & Eki vektordıń skalyar kóbeymesiniń formulası? &  \\
\hline
2. & undefined &  \\
\hline
3. & undefined &  \\
\hline
4. & undefined &  \\
\hline
5. & \(\left| \bar{a} \right| = 8, \left| \bar{b} \right| = 5, \alpha = 60^{0}\) bolsa, \(( \bar{a}\bar{b} )\) ni tabıń. &  \\
\hline
6. & \(x - 2 y + 1 = 0\) teńlemesi menen berilgen tuwrınıń normal túrdegi teńlemesin kórsetiń. &  \\
\hline
7. & Orayı \(C (- 1;2)\) noqatında, \(A (- 2;6 )\) noqatınan ótetuǵın sheńberdiń teńlemesin dúziń. &  \\
\hline
8. & \(A_{1}x + B_{1}y + C_{1}z + D_{1} = 0\) hám tegislikleri ústpe-úst túsiwi ushın qaysı shárt orınlı bolıwı kerek? &  \\
\hline
9. & \(\frac{x^{2}}{225} - \frac{y^{2}}{64} = - 1\) giperbola fokusınıń koordinatalarınıń tabıń. &  \\
\hline
10. & \(x^{2} - 4 y^{2} + 6 x + 5 = 0\) giperbolanıń kanonikalıq teńlemesin dúziń. & \\
\hline
\end{tabular}

\vspace{0.7cm}

\begin{tabular}{lll}
Tuwrı juwaplar sanı: \underline{\hspace{1cm}} & 
Bahası: \underline{\hspace{1cm}} & 
Imtixan alıwshınıń qolı: \underline{\hspace{2cm}} \\
\end{tabular}

\egroup

\newpage


\textbf{46-variant}\\

\bgroup
\def\arraystretch{1.6} % 1 is the default, change whatever you need

\begin{tabular}{|m{5.7cm}|m{9.5cm}|}
\hline
Familiyası hám atı & \\
\hline
Fakulteti  & \\
\hline
Toparı hám tálim baǵdarı  & \\
\hline
\end{tabular}

\vspace{0.7cm}

\begin{tabular}{|m{0.7cm}|m{10cm}|m{4cm}|}
\hline
№ & Soraw & Juwap \\
\hline
1. & \(OXY\) tegisliginiń teńlemesi? &  \\
\hline
2. & undefined &  \\
\hline
3. & undefined &  \\
\hline
4. & undefined &  \\
\hline
5. & \(\overline{a} = \{5,- 6, 1 \}, \overline{b} = \{ - 4, 3, 0 \} \), \(\overline{c} = \left\{ 5,- 8, 10 \right\}\) vektorları berilgen. \(2{\bar{a}}^{2} + 4{\bar{b}}^{2} - 5{\bar{c}}^{2}\) ańlatpasınıń mánisin tabıń. &  \\
\hline
6. & \(3 x - y + 5 = 0, x + 3 y - 4 = 0\) tuwrı sızıqları arasındaǵı múyeshti tabıń. &  \\
\hline
7. & \(x^{2} + y^{2} - 2 x + 4 y - 20 = 0\) sheńberdiń \(C\) orayın hám \(R\) radiusın tabıń. &  \\
\hline
8. & \(A_{1}x + B_{1}y + C_{1}z + D_{1} = 0\) hám \(Ax + By + Cz + D = 0\) tegislikleri perpendikulyar bolıwı ushın qaysı shárt orınlı bolıwı kerek? &  \\
\hline
9. & Eger \(2 a = 16, e = \frac{5}{4}\) bolsa, fokusı abscissa kósherinde, koordinata basına salıstırǵanda simmetriyalıq jaylasqan giperbolanıń teńlemesin dúziń. &  \\
\hline
10. & \(9 x^{2} + 25 y^{2} = 225\) ellipsi berilgen, ellipstiń fokusların, ekscentrisitetin tabıń. & \\
\hline
\end{tabular}

\vspace{0.7cm}

\begin{tabular}{lll}
Tuwrı juwaplar sanı: \underline{\hspace{1cm}} & 
Bahası: \underline{\hspace{1cm}} & 
Imtixan alıwshınıń qolı: \underline{\hspace{2cm}} \\
\end{tabular}

\egroup

\newpage


\textbf{47-variant}\\

\bgroup
\def\arraystretch{1.6} % 1 is the default, change whatever you need

\begin{tabular}{|m{5.7cm}|m{9.5cm}|}
\hline
Familiyası hám atı & \\
\hline
Fakulteti  & \\
\hline
Toparı hám tálim baǵdarı  & \\
\hline
\end{tabular}

\vspace{0.7cm}

\begin{tabular}{|m{0.7cm}|m{10cm}|m{4cm}|}
\hline
№ & Soraw & Juwap \\
\hline
1. & Eki tuwrı sızıq arasındaǵı múyeshti tabıw formulası? &  \\
\hline
2. & undefined &  \\
\hline
3. & undefined &  \\
\hline
4. & undefined &  \\
\hline
5. & \(M_{1} (12; - 1)\) hám \(M_{2} (0;4)\) noqatlardıń arasındaǵı aralıqtı tabıń. &  \\
\hline
6. & \(x + y - 3 = 0\) hám \(2 x + 3 y - 8 = 0\) tuwrıları óz-ara qanday jaylasqan? &  \\
\hline
7. & \(x^{2} + y^{2} - 2 x + 4 y = 0\) sheńberdiń teńlemesin kanonikalıq túrdegi teńlemege alıp keliń. &  \\
\hline
8. & \((x + 1) ^{2} + (y - 2) ^{2} + (z + 2) ^{2} = 49\) sferanıń orayınıń koordinataların tabıń. &  \\
\hline
9. & \(3 x^{2} + 10 xy + 3 y^{2} - 2 x - 14 y - 13 = 0\) teńlemesiniń tipin anıqlań. &  \\
\hline
10. & Eger \(2 b = 24, 2 c = 10\) bolsa, onda abscissa kósherinde koordinata basına salıstırǵanda simmetriyalıq jaylasqan fokuslarǵa iye, ellipstiń teńlemesin dúziń. & \\
\hline
\end{tabular}

\vspace{0.7cm}

\begin{tabular}{lll}
Tuwrı juwaplar sanı: \underline{\hspace{1cm}} & 
Bahası: \underline{\hspace{1cm}} & 
Imtixan alıwshınıń qolı: \underline{\hspace{2cm}} \\
\end{tabular}

\egroup

\newpage


\textbf{48-variant}\\

\bgroup
\def\arraystretch{1.6} % 1 is the default, change whatever you need

\begin{tabular}{|m{5.7cm}|m{9.5cm}|}
\hline
Familiyası hám atı & \\
\hline
Fakulteti  & \\
\hline
Toparı hám tálim baǵdarı  & \\
\hline
\end{tabular}

\vspace{0.7cm}

\begin{tabular}{|m{0.7cm}|m{10cm}|m{4cm}|}
\hline
№ & Soraw & Juwap \\
\hline
1. & \(OY\) kósheriniń teńlemesi? &  \\
\hline
2. & undefined &  \\
\hline
3. & undefined &  \\
\hline
4. & undefined &  \\
\hline
5. & \(\bar{a} = \left\{ 2, 1, 0 \right\}\) hám \(\bar{b} = \left\{ 1, 0,- 1 \right\}\) bolsa, \(\bar{a} - \bar{b}\) ni tabıń. &  \\
\hline
6. & \(2 x + 3 y + 4 = 0\) tuwrısına parallel hám \(M_{0} (2;1)\) noqattan ótetuǵın tuwrınıń teńlemesin dúziń. &  \\
\hline
7. & \(x + y - 12 = 0\) tuwrısı \(x^{2} + y^{2} - 2 y = 0\) sheńberge salıstırǵanda qanday jaylasqan? &  \\
\hline
8. & \(x + 2 = 0\) keńislik qanday geometriyalıq betlikti anıqlaydı? &  \\
\hline
9. & \(\frac{x^{2}}{a^{2}} + \frac{y^{2}}{b^{2}} = 1\) ellipstiń \((x_{0};y_{0})\) noqatındaǵı urınbasınıń teńlemesin tabıń. &  \\
\hline
10. & \(A (- 1;0;1),\ B (1; - 1;0)\) noqatları berilgen. \(\bar{BA}\) vektorın tabıń. & \\
\hline
\end{tabular}

\vspace{0.7cm}

\begin{tabular}{lll}
Tuwrı juwaplar sanı: \underline{\hspace{1cm}} & 
Bahası: \underline{\hspace{1cm}} & 
Imtixan alıwshınıń qolı: \underline{\hspace{2cm}} \\
\end{tabular}

\egroup

\newpage


\textbf{49-variant}\\

\bgroup
\def\arraystretch{1.6} % 1 is the default, change whatever you need

\begin{tabular}{|m{5.7cm}|m{9.5cm}|}
\hline
Familiyası hám atı & \\
\hline
Fakulteti  & \\
\hline
Toparı hám tálim baǵdarı  & \\
\hline
\end{tabular}

\vspace{0.7cm}

\begin{tabular}{|m{0.7cm}|m{10cm}|m{4cm}|}
\hline
№ & Soraw & Juwap \\
\hline
1. & \(Ax + By + D = 0\) teńlemesi arqalı ... tegisliktiń teńlemesi berilgen? &  \\
\hline
2. & undefined &  \\
\hline
3. & undefined &  \\
\hline
4. & undefined &  \\
\hline
5. & $(2, 3)$ hám $(4, 3)$ noqatlarınan ótiwshi tuwrı sızıqtıń teńlemesin dúziń. &  \\
\hline
6. & \(A_{1}x + B_{1}y + C_{1}z + D_{1} = 0\) hám \(Ax + By + Cz + D = 0\) tegislikleri parallel bolıwı ushın qaysı shárt orınlı bolıwı kerek? &  \\
\hline
7. & \(\bar{a} = \left\{ 4,- 2,- 4 \right\}\) hám \(\bar{b} = \left\{ 6,- 3, 2 \right\}\) vektorları berilgen, \((\bar{a} - \bar{b}) ^{2}\)-? &  \\
\hline
8. & \(2 x + 3 y - 6 = 0\) tuwrınıń teńlemesin kesindilerde berilgen teńleme túrinde kórsetiń. &  \\
\hline
9. & \(A (4, 3), B (7, 7)\) noqatları arasındaǵı aralıqtı tabıń. &  \\
\hline
10. & \(M_{1}M_{2}\) kesindiniń ortasınıń koordinatalarınıń tabıń, eger \(M_{1} (2, 3), M_{2} (4, 7)\) bolsa. & \\
\hline
\end{tabular}

\vspace{0.7cm}

\begin{tabular}{lll}
Tuwrı juwaplar sanı: \underline{\hspace{1cm}} & 
Bahası: \underline{\hspace{1cm}} & 
Imtixan alıwshınıń qolı: \underline{\hspace{2cm}} \\
\end{tabular}

\egroup

\newpage


\textbf{50-variant}\\

\bgroup
\def\arraystretch{1.6} % 1 is the default, change whatever you need

\begin{tabular}{|m{5.7cm}|m{9.5cm}|}
\hline
Familiyası hám atı & \\
\hline
Fakulteti  & \\
\hline
Toparı hám tálim baǵdarı  & \\
\hline
\end{tabular}

\vspace{0.7cm}

\begin{tabular}{|m{0.7cm}|m{10cm}|m{4cm}|}
\hline
№ & Soraw & Juwap \\
\hline
1. & Vektorlardı qosıw tómendegi qaysı qásiyetke iye emes? &  \\
\hline
2. & undefined &  \\
\hline
3. & undefined &  \\
\hline
4. & undefined &  \\
\hline
5. & \(x + y = 0\) teńlemesi menen berilgen tuwrı sızıqtıń múyeshlik koefficientin anıqlań. &  \\
\hline
6. & \(5 x - y + 7 = 0\) hám \(3 x + 2 y = 0\) tuwrıları arasındaǵı múyeshni tabıń. &  \\
\hline
7. & Koordinatalar kósherleri hám \( 3 x + 4 y - 12 = 0 \) tuwrı sızıǵı menen shegaralanǵan úshmúyeshliktiń maydanın tabıń. &  \\
\hline
8. & \(\left| \bar{a} \right| = 8, \left| \bar{b} \right| = 5, \alpha = 60^{0}\) bolsa, \(( \bar{a}\bar{b} )\) ni tabıń. &  \\
\hline
9. & \(x - 2 y + 1 = 0\) teńlemesi menen berilgen tuwrınıń normal túrdegi teńlemesin kórsetiń. &  \\
\hline
10. & Orayı \(C (- 1;2)\) noqatında, \(A (- 2;6 )\) noqatınan ótetuǵın sheńberdiń teńlemesin dúziń. & \\
\hline
\end{tabular}

\vspace{0.7cm}

\begin{tabular}{lll}
Tuwrı juwaplar sanı: \underline{\hspace{1cm}} & 
Bahası: \underline{\hspace{1cm}} & 
Imtixan alıwshınıń qolı: \underline{\hspace{2cm}} \\
\end{tabular}

\egroup

\newpage


\textbf{51-variant}\\

\bgroup
\def\arraystretch{1.6} % 1 is the default, change whatever you need

\begin{tabular}{|m{5.7cm}|m{9.5cm}|}
\hline
Familiyası hám atı & \\
\hline
Fakulteti  & \\
\hline
Toparı hám tálim baǵdarı  & \\
\hline
\end{tabular}

\vspace{0.7cm}

\begin{tabular}{|m{0.7cm}|m{10cm}|m{4cm}|}
\hline
№ & Soraw & Juwap \\
\hline
1. & Giperbolanıń kanonikalıq teńlemesi? &  \\
\hline
2. & undefined &  \\
\hline
3. & undefined &  \\
\hline
4. & undefined &  \\
\hline
5. & \(A_{1}x + B_{1}y + C_{1}z + D_{1} = 0\) hám tegislikleri ústpe-úst túsiwi ushın qaysı shárt orınlı bolıwı kerek? &  \\
\hline
6. & \(\frac{x^{2}}{225} - \frac{y^{2}}{64} = - 1\) giperbola fokusınıń koordinatalarınıń tabıń. &  \\
\hline
7. & \(x^{2} - 4 y^{2} + 6 x + 5 = 0\) giperbolanıń kanonikalıq teńlemesin dúziń. &  \\
\hline
8. & \(\overline{a} = \{5,- 6, 1 \}, \overline{b} = \{ - 4, 3, 0 \} \), \(\overline{c} = \left\{ 5,- 8, 10 \right\}\) vektorları berilgen. \(2{\bar{a}}^{2} + 4{\bar{b}}^{2} - 5{\bar{c}}^{2}\) ańlatpasınıń mánisin tabıń. &  \\
\hline
9. & \(3 x - y + 5 = 0, x + 3 y - 4 = 0\) tuwrı sızıqları arasındaǵı múyeshti tabıń. &  \\
\hline
10. & \(x^{2} + y^{2} - 2 x + 4 y - 20 = 0\) sheńberdiń \(C\) orayın hám \(R\) radiusın tabıń. & \\
\hline
\end{tabular}

\vspace{0.7cm}

\begin{tabular}{lll}
Tuwrı juwaplar sanı: \underline{\hspace{1cm}} & 
Bahası: \underline{\hspace{1cm}} & 
Imtixan alıwshınıń qolı: \underline{\hspace{2cm}} \\
\end{tabular}

\egroup

\newpage


\textbf{52-variant}\\

\bgroup
\def\arraystretch{1.6} % 1 is the default, change whatever you need

\begin{tabular}{|m{5.7cm}|m{9.5cm}|}
\hline
Familiyası hám atı & \\
\hline
Fakulteti  & \\
\hline
Toparı hám tálim baǵdarı  & \\
\hline
\end{tabular}

\vspace{0.7cm}

\begin{tabular}{|m{0.7cm}|m{10cm}|m{4cm}|}
\hline
№ & Soraw & Juwap \\
\hline
1. & \(\frac{x^{2}}{a^{2}} - \frac{y^{2}}{b^{2}} = 1\) giperbolanıń \((x_{0};y_{0})\) noqatındaǵı urınbasınıń teńlemesin kórsetiń. &  \\
\hline
2. & undefined &  \\
\hline
3. & undefined &  \\
\hline
4. & undefined &  \\
\hline
5. & \(A_{1}x + B_{1}y + C_{1}z + D_{1} = 0\) hám \(Ax + By + Cz + D = 0\) tegislikleri perpendikulyar bolıwı ushın qaysı shárt orınlı bolıwı kerek? &  \\
\hline
6. & Eger \(2 a = 16, e = \frac{5}{4}\) bolsa, fokusı abscissa kósherinde, koordinata basına salıstırǵanda simmetriyalıq jaylasqan giperbolanıń teńlemesin dúziń. &  \\
\hline
7. & \(9 x^{2} + 25 y^{2} = 225\) ellipsi berilgen, ellipstiń fokusların, ekscentrisitetin tabıń. &  \\
\hline
8. & \(M_{1} (12; - 1)\) hám \(M_{2} (0;4)\) noqatlardıń arasındaǵı aralıqtı tabıń. &  \\
\hline
9. & \(x + y - 3 = 0\) hám \(2 x + 3 y - 8 = 0\) tuwrıları óz-ara qanday jaylasqan? &  \\
\hline
10. & \(x^{2} + y^{2} - 2 x + 4 y = 0\) sheńberdiń teńlemesin kanonikalıq túrdegi teńlemege alıp keliń. & \\
\hline
\end{tabular}

\vspace{0.7cm}

\begin{tabular}{lll}
Tuwrı juwaplar sanı: \underline{\hspace{1cm}} & 
Bahası: \underline{\hspace{1cm}} & 
Imtixan alıwshınıń qolı: \underline{\hspace{2cm}} \\
\end{tabular}

\egroup

\newpage


\textbf{53-variant}\\

\bgroup
\def\arraystretch{1.6} % 1 is the default, change whatever you need

\begin{tabular}{|m{5.7cm}|m{9.5cm}|}
\hline
Familiyası hám atı & \\
\hline
Fakulteti  & \\
\hline
Toparı hám tálim baǵdarı  & \\
\hline
\end{tabular}

\vspace{0.7cm}

\begin{tabular}{|m{0.7cm}|m{10cm}|m{4cm}|}
\hline
№ & Soraw & Juwap \\
\hline
1. & Vektorlardı qosıw koordinatalarda qanday formula menen anıqlanadı? &  \\
\hline
2. & undefined &  \\
\hline
3. & undefined &  \\
\hline
4. & undefined &  \\
\hline
5. & \((x + 1) ^{2} + (y - 2) ^{2} + (z + 2) ^{2} = 49\) sferanıń orayınıń koordinataların tabıń. &  \\
\hline
6. & \(3 x^{2} + 10 xy + 3 y^{2} - 2 x - 14 y - 13 = 0\) teńlemesiniń tipin anıqlań. &  \\
\hline
7. & Eger \(2 b = 24, 2 c = 10\) bolsa, onda abscissa kósherinde koordinata basına salıstırǵanda simmetriyalıq jaylasqan fokuslarǵa iye, ellipstiń teńlemesin dúziń. &  \\
\hline
8. & \(\bar{a} = \left\{ 2, 1, 0 \right\}\) hám \(\bar{b} = \left\{ 1, 0,- 1 \right\}\) bolsa, \(\bar{a} - \bar{b}\) ni tabıń. &  \\
\hline
9. & \(2 x + 3 y + 4 = 0\) tuwrısına parallel hám \(M_{0} (2;1)\) noqattan ótetuǵın tuwrınıń teńlemesin dúziń. &  \\
\hline
10. & \(x + y - 12 = 0\) tuwrısı \(x^{2} + y^{2} - 2 y = 0\) sheńberge salıstırǵanda qanday jaylasqan? & \\
\hline
\end{tabular}

\vspace{0.7cm}

\begin{tabular}{lll}
Tuwrı juwaplar sanı: \underline{\hspace{1cm}} & 
Bahası: \underline{\hspace{1cm}} & 
Imtixan alıwshınıń qolı: \underline{\hspace{2cm}} \\
\end{tabular}

\egroup

\newpage


\textbf{54-variant}\\

\bgroup
\def\arraystretch{1.6} % 1 is the default, change whatever you need

\begin{tabular}{|m{5.7cm}|m{9.5cm}|}
\hline
Familiyası hám atı & \\
\hline
Fakulteti  & \\
\hline
Toparı hám tálim baǵdarı  & \\
\hline
\end{tabular}

\vspace{0.7cm}

\begin{tabular}{|m{0.7cm}|m{10cm}|m{4cm}|}
\hline
№ & Soraw & Juwap \\
\hline
1. & Eki vektordıń vektor kóbeymesiniń uzınlıǵın tabıw formulası? &  \\
\hline
2. & undefined &  \\
\hline
3. & undefined &  \\
\hline
4. & undefined &  \\
\hline
5. & \(x + 2 = 0\) keńislik qanday geometriyalıq betlikti anıqlaydı? &  \\
\hline
6. & \(\frac{x^{2}}{a^{2}} + \frac{y^{2}}{b^{2}} = 1\) ellipstiń \((x_{0};y_{0})\) noqatındaǵı urınbasınıń teńlemesin tabıń. &  \\
\hline
7. & \(A (- 1;0;1),\ B (1; - 1;0)\) noqatları berilgen. \(\bar{BA}\) vektorın tabıń. &  \\
\hline
8. & $(2, 3)$ hám $(4, 3)$ noqatlarınan ótiwshi tuwrı sızıqtıń teńlemesin dúziń. &  \\
\hline
9. & \(A_{1}x + B_{1}y + C_{1}z + D_{1} = 0\) hám \(Ax + By + Cz + D = 0\) tegislikleri parallel bolıwı ushın qaysı shárt orınlı bolıwı kerek? &  \\
\hline
10. & \(\bar{a} = \left\{ 4,- 2,- 4 \right\}\) hám \(\bar{b} = \left\{ 6,- 3, 2 \right\}\) vektorları berilgen, \((\bar{a} - \bar{b}) ^{2}\)-? & \\
\hline
\end{tabular}

\vspace{0.7cm}

\begin{tabular}{lll}
Tuwrı juwaplar sanı: \underline{\hspace{1cm}} & 
Bahası: \underline{\hspace{1cm}} & 
Imtixan alıwshınıń qolı: \underline{\hspace{2cm}} \\
\end{tabular}

\egroup

\newpage


\textbf{55-variant}\\

\bgroup
\def\arraystretch{1.6} % 1 is the default, change whatever you need

\begin{tabular}{|m{5.7cm}|m{9.5cm}|}
\hline
Familiyası hám atı & \\
\hline
Fakulteti  & \\
\hline
Toparı hám tálim baǵdarı  & \\
\hline
\end{tabular}

\vspace{0.7cm}

\begin{tabular}{|m{0.7cm}|m{10cm}|m{4cm}|}
\hline
№ & Soraw & Juwap \\
\hline
1. & Tegislikdegi qálegen noqattan berilgen eki noqatqa shekemgi bolǵan aralıqlardıń ayırmasınıń modulı ózgermeytuǵın bolǵan noqatlardıń geometriyalıq ornı ne dep ataladı? &  \\
\hline
2. & undefined &  \\
\hline
3. & undefined &  \\
\hline
4. & undefined &  \\
\hline
5. & \(2 x + 3 y - 6 = 0\) tuwrınıń teńlemesin kesindilerde berilgen teńleme túrinde kórsetiń. &  \\
\hline
6. & \(A (4, 3), B (7, 7)\) noqatları arasındaǵı aralıqtı tabıń. &  \\
\hline
7. & \(M_{1}M_{2}\) kesindiniń ortasınıń koordinatalarınıń tabıń, eger \(M_{1} (2, 3), M_{2} (4, 7)\) bolsa. &  \\
\hline
8. & \(x + y = 0\) teńlemesi menen berilgen tuwrı sızıqtıń múyeshlik koefficientin anıqlań. &  \\
\hline
9. & \(5 x - y + 7 = 0\) hám \(3 x + 2 y = 0\) tuwrıları arasındaǵı múyeshni tabıń. &  \\
\hline
10. & Koordinatalar kósherleri hám \( 3 x + 4 y - 12 = 0 \) tuwrı sızıǵı menen shegaralanǵan úshmúyeshliktiń maydanın tabıń. & \\
\hline
\end{tabular}

\vspace{0.7cm}

\begin{tabular}{lll}
Tuwrı juwaplar sanı: \underline{\hspace{1cm}} & 
Bahası: \underline{\hspace{1cm}} & 
Imtixan alıwshınıń qolı: \underline{\hspace{2cm}} \\
\end{tabular}

\egroup

\newpage


\textbf{56-variant}\\

\bgroup
\def\arraystretch{1.6} % 1 is the default, change whatever you need

\begin{tabular}{|m{5.7cm}|m{9.5cm}|}
\hline
Familiyası hám atı & \\
\hline
Fakulteti  & \\
\hline
Toparı hám tálim baǵdarı  & \\
\hline
\end{tabular}

\vspace{0.7cm}

\begin{tabular}{|m{0.7cm}|m{10cm}|m{4cm}|}
\hline
№ & Soraw & Juwap \\
\hline
1. & Eki vektor qashan kollinear dep ataladı? &  \\
\hline
2. & undefined &  \\
\hline
3. & undefined &  \\
\hline
4. & undefined &  \\
\hline
5. & \(\left| \bar{a} \right| = 8, \left| \bar{b} \right| = 5, \alpha = 60^{0}\) bolsa, \(( \bar{a}\bar{b} )\) ni tabıń. &  \\
\hline
6. & \(x - 2 y + 1 = 0\) teńlemesi menen berilgen tuwrınıń normal túrdegi teńlemesin kórsetiń. &  \\
\hline
7. & Orayı \(C (- 1;2)\) noqatında, \(A (- 2;6 )\) noqatınan ótetuǵın sheńberdiń teńlemesin dúziń. &  \\
\hline
8. & \(A_{1}x + B_{1}y + C_{1}z + D_{1} = 0\) hám tegislikleri ústpe-úst túsiwi ushın qaysı shárt orınlı bolıwı kerek? &  \\
\hline
9. & \(\frac{x^{2}}{225} - \frac{y^{2}}{64} = - 1\) giperbola fokusınıń koordinatalarınıń tabıń. &  \\
\hline
10. & \(x^{2} - 4 y^{2} + 6 x + 5 = 0\) giperbolanıń kanonikalıq teńlemesin dúziń. & \\
\hline
\end{tabular}

\vspace{0.7cm}

\begin{tabular}{lll}
Tuwrı juwaplar sanı: \underline{\hspace{1cm}} & 
Bahası: \underline{\hspace{1cm}} & 
Imtixan alıwshınıń qolı: \underline{\hspace{2cm}} \\
\end{tabular}

\egroup

\newpage


\textbf{57-variant}\\

\bgroup
\def\arraystretch{1.6} % 1 is the default, change whatever you need

\begin{tabular}{|m{5.7cm}|m{9.5cm}|}
\hline
Familiyası hám atı & \\
\hline
Fakulteti  & \\
\hline
Toparı hám tálim baǵdarı  & \\
\hline
\end{tabular}

\vspace{0.7cm}

\begin{tabular}{|m{0.7cm}|m{10cm}|m{4cm}|}
\hline
№ & Soraw & Juwap \\
\hline
1. & Egerde \(a = \{ x_{1}; y_{1}; z_{1}\}, b = \{ x_{2}; y_{2}; z_{2}\}\) bolsa, vektor kóbeymeniń koordinatalarda ańlatılıwı qanday boladı? &  \\
\hline
2. & undefined &  \\
\hline
3. & undefined &  \\
\hline
4. & undefined &  \\
\hline
5. & \(\overline{a} = \{5,- 6, 1 \}, \overline{b} = \{ - 4, 3, 0 \} \), \(\overline{c} = \left\{ 5,- 8, 10 \right\}\) vektorları berilgen. \(2{\bar{a}}^{2} + 4{\bar{b}}^{2} - 5{\bar{c}}^{2}\) ańlatpasınıń mánisin tabıń. &  \\
\hline
6. & \(3 x - y + 5 = 0, x + 3 y - 4 = 0\) tuwrı sızıqları arasındaǵı múyeshti tabıń. &  \\
\hline
7. & \(x^{2} + y^{2} - 2 x + 4 y - 20 = 0\) sheńberdiń \(C\) orayın hám \(R\) radiusın tabıń. &  \\
\hline
8. & \(A_{1}x + B_{1}y + C_{1}z + D_{1} = 0\) hám \(Ax + By + Cz + D = 0\) tegislikleri perpendikulyar bolıwı ushın qaysı shárt orınlı bolıwı kerek? &  \\
\hline
9. & Eger \(2 a = 16, e = \frac{5}{4}\) bolsa, fokusı abscissa kósherinde, koordinata basına salıstırǵanda simmetriyalıq jaylasqan giperbolanıń teńlemesin dúziń. &  \\
\hline
10. & \(9 x^{2} + 25 y^{2} = 225\) ellipsi berilgen, ellipstiń fokusların, ekscentrisitetin tabıń. & \\
\hline
\end{tabular}

\vspace{0.7cm}

\begin{tabular}{lll}
Tuwrı juwaplar sanı: \underline{\hspace{1cm}} & 
Bahası: \underline{\hspace{1cm}} & 
Imtixan alıwshınıń qolı: \underline{\hspace{2cm}} \\
\end{tabular}

\egroup

\newpage


\textbf{58-variant}\\

\bgroup
\def\arraystretch{1.6} % 1 is the default, change whatever you need

\begin{tabular}{|m{5.7cm}|m{9.5cm}|}
\hline
Familiyası hám atı & \\
\hline
Fakulteti  & \\
\hline
Toparı hám tálim baǵdarı  & \\
\hline
\end{tabular}

\vspace{0.7cm}

\begin{tabular}{|m{0.7cm}|m{10cm}|m{4cm}|}
\hline
№ & Soraw & Juwap \\
\hline
1. & \(Ax + C = 0\) tuwrı sızıqtıń grafigi koordinata kósherlerine salıstırǵanda qanday jaylasqan? &  \\
\hline
2. & undefined &  \\
\hline
3. & undefined &  \\
\hline
4. & undefined &  \\
\hline
5. & \(M_{1} (12; - 1)\) hám \(M_{2} (0;4)\) noqatlardıń arasındaǵı aralıqtı tabıń. &  \\
\hline
6. & \(x + y - 3 = 0\) hám \(2 x + 3 y - 8 = 0\) tuwrıları óz-ara qanday jaylasqan? &  \\
\hline
7. & \(x^{2} + y^{2} - 2 x + 4 y = 0\) sheńberdiń teńlemesin kanonikalıq túrdegi teńlemege alıp keliń. &  \\
\hline
8. & \((x + 1) ^{2} + (y - 2) ^{2} + (z + 2) ^{2} = 49\) sferanıń orayınıń koordinataların tabıń. &  \\
\hline
9. & \(3 x^{2} + 10 xy + 3 y^{2} - 2 x - 14 y - 13 = 0\) teńlemesiniń tipin anıqlań. &  \\
\hline
10. & Eger \(2 b = 24, 2 c = 10\) bolsa, onda abscissa kósherinde koordinata basına salıstırǵanda simmetriyalıq jaylasqan fokuslarǵa iye, ellipstiń teńlemesin dúziń. & \\
\hline
\end{tabular}

\vspace{0.7cm}

\begin{tabular}{lll}
Tuwrı juwaplar sanı: \underline{\hspace{1cm}} & 
Bahası: \underline{\hspace{1cm}} & 
Imtixan alıwshınıń qolı: \underline{\hspace{2cm}} \\
\end{tabular}

\egroup

\newpage


\textbf{59-variant}\\

\bgroup
\def\arraystretch{1.6} % 1 is the default, change whatever you need

\begin{tabular}{|m{5.7cm}|m{9.5cm}|}
\hline
Familiyası hám atı & \\
\hline
Fakulteti  & \\
\hline
Toparı hám tálim baǵdarı  & \\
\hline
\end{tabular}

\vspace{0.7cm}

\begin{tabular}{|m{0.7cm}|m{10cm}|m{4cm}|}
\hline
№ & Soraw & Juwap \\
\hline
1. & Úsh vektordıń aralas kóbeymesi ushın \((abc) = 0\) teńligi orınlı bolsa ne dep ataladı? &  \\
\hline
2. & undefined &  \\
\hline
3. & undefined &  \\
\hline
4. & undefined &  \\
\hline
5. & \(\bar{a} = \left\{ 2, 1, 0 \right\}\) hám \(\bar{b} = \left\{ 1, 0,- 1 \right\}\) bolsa, \(\bar{a} - \bar{b}\) ni tabıń. &  \\
\hline
6. & \(2 x + 3 y + 4 = 0\) tuwrısına parallel hám \(M_{0} (2;1)\) noqattan ótetuǵın tuwrınıń teńlemesin dúziń. &  \\
\hline
7. & \(x + y - 12 = 0\) tuwrısı \(x^{2} + y^{2} - 2 y = 0\) sheńberge salıstırǵanda qanday jaylasqan? &  \\
\hline
8. & \(x + 2 = 0\) keńislik qanday geometriyalıq betlikti anıqlaydı? &  \\
\hline
9. & \(\frac{x^{2}}{a^{2}} + \frac{y^{2}}{b^{2}} = 1\) ellipstiń \((x_{0};y_{0})\) noqatındaǵı urınbasınıń teńlemesin tabıń. &  \\
\hline
10. & \(A (- 1;0;1),\ B (1; - 1;0)\) noqatları berilgen. \(\bar{BA}\) vektorın tabıń. & \\
\hline
\end{tabular}

\vspace{0.7cm}

\begin{tabular}{lll}
Tuwrı juwaplar sanı: \underline{\hspace{1cm}} & 
Bahası: \underline{\hspace{1cm}} & 
Imtixan alıwshınıń qolı: \underline{\hspace{2cm}} \\
\end{tabular}

\egroup

\newpage


\textbf{60-variant}\\

\bgroup
\def\arraystretch{1.6} % 1 is the default, change whatever you need

\begin{tabular}{|m{5.7cm}|m{9.5cm}|}
\hline
Familiyası hám atı & \\
\hline
Fakulteti  & \\
\hline
Toparı hám tálim baǵdarı  & \\
\hline
\end{tabular}

\vspace{0.7cm}

\begin{tabular}{|m{0.7cm}|m{10cm}|m{4cm}|}
\hline
№ & Soraw & Juwap \\
\hline
1. & Tuwrı múyeshli koordinatalar sisteması dep nege aytamız? &  \\
\hline
2. & undefined &  \\
\hline
3. & undefined &  \\
\hline
4. & undefined &  \\
\hline
5. & $(2, 3)$ hám $(4, 3)$ noqatlarınan ótiwshi tuwrı sızıqtıń teńlemesin dúziń. &  \\
\hline
6. & \(A_{1}x + B_{1}y + C_{1}z + D_{1} = 0\) hám \(Ax + By + Cz + D = 0\) tegislikleri parallel bolıwı ushın qaysı shárt orınlı bolıwı kerek? &  \\
\hline
7. & \(\bar{a} = \left\{ 4,- 2,- 4 \right\}\) hám \(\bar{b} = \left\{ 6,- 3, 2 \right\}\) vektorları berilgen, \((\bar{a} - \bar{b}) ^{2}\)-? &  \\
\hline
8. & \(2 x + 3 y - 6 = 0\) tuwrınıń teńlemesin kesindilerde berilgen teńleme túrinde kórsetiń. &  \\
\hline
9. & \(A (4, 3), B (7, 7)\) noqatları arasındaǵı aralıqtı tabıń. &  \\
\hline
10. & \(M_{1}M_{2}\) kesindiniń ortasınıń koordinatalarınıń tabıń, eger \(M_{1} (2, 3), M_{2} (4, 7)\) bolsa. & \\
\hline
\end{tabular}

\vspace{0.7cm}

\begin{tabular}{lll}
Tuwrı juwaplar sanı: \underline{\hspace{1cm}} & 
Bahası: \underline{\hspace{1cm}} & 
Imtixan alıwshınıń qolı: \underline{\hspace{2cm}} \\
\end{tabular}

\egroup

\newpage


\textbf{61-variant}\\

\bgroup
\def\arraystretch{1.6} % 1 is the default, change whatever you need

\begin{tabular}{|m{5.7cm}|m{9.5cm}|}
\hline
Familiyası hám atı & \\
\hline
Fakulteti  & \\
\hline
Toparı hám tálim baǵdarı  & \\
\hline
\end{tabular}

\vspace{0.7cm}

\begin{tabular}{|m{0.7cm}|m{10cm}|m{4cm}|}
\hline
№ & Soraw & Juwap \\
\hline
1. & Vektorlardıń kósherdegi proekciyasınıń formulası? &  \\
\hline
2. & undefined &  \\
\hline
3. & undefined &  \\
\hline
4. & undefined &  \\
\hline
5. & \(x + y = 0\) teńlemesi menen berilgen tuwrı sızıqtıń múyeshlik koefficientin anıqlań. &  \\
\hline
6. & \(5 x - y + 7 = 0\) hám \(3 x + 2 y = 0\) tuwrıları arasındaǵı múyeshni tabıń. &  \\
\hline
7. & Koordinatalar kósherleri hám \( 3 x + 4 y - 12 = 0 \) tuwrı sızıǵı menen shegaralanǵan úshmúyeshliktiń maydanın tabıń. &  \\
\hline
8. & \(\left| \bar{a} \right| = 8, \left| \bar{b} \right| = 5, \alpha = 60^{0}\) bolsa, \(( \bar{a}\bar{b} )\) ni tabıń. &  \\
\hline
9. & \(x - 2 y + 1 = 0\) teńlemesi menen berilgen tuwrınıń normal túrdegi teńlemesin kórsetiń. &  \\
\hline
10. & Orayı \(C (- 1;2)\) noqatında, \(A (- 2;6 )\) noqatınan ótetuǵın sheńberdiń teńlemesin dúziń. & \\
\hline
\end{tabular}

\vspace{0.7cm}

\begin{tabular}{lll}
Tuwrı juwaplar sanı: \underline{\hspace{1cm}} & 
Bahası: \underline{\hspace{1cm}} & 
Imtixan alıwshınıń qolı: \underline{\hspace{2cm}} \\
\end{tabular}

\egroup

\newpage


\textbf{62-variant}\\

\bgroup
\def\arraystretch{1.6} % 1 is the default, change whatever you need

\begin{tabular}{|m{5.7cm}|m{9.5cm}|}
\hline
Familiyası hám atı & \\
\hline
Fakulteti  & \\
\hline
Toparı hám tálim baǵdarı  & \\
\hline
\end{tabular}

\vspace{0.7cm}

\begin{tabular}{|m{0.7cm}|m{10cm}|m{4cm}|}
\hline
№ & Soraw & Juwap \\
\hline
1. & Eki vektordıń skalyar kóbeymesiniń formulası? &  \\
\hline
2. & undefined &  \\
\hline
3. & undefined &  \\
\hline
4. & undefined &  \\
\hline
5. & \(A_{1}x + B_{1}y + C_{1}z + D_{1} = 0\) hám tegislikleri ústpe-úst túsiwi ushın qaysı shárt orınlı bolıwı kerek? &  \\
\hline
6. & \(\frac{x^{2}}{225} - \frac{y^{2}}{64} = - 1\) giperbola fokusınıń koordinatalarınıń tabıń. &  \\
\hline
7. & \(x^{2} - 4 y^{2} + 6 x + 5 = 0\) giperbolanıń kanonikalıq teńlemesin dúziń. &  \\
\hline
8. & \(\overline{a} = \{5,- 6, 1 \}, \overline{b} = \{ - 4, 3, 0 \} \), \(\overline{c} = \left\{ 5,- 8, 10 \right\}\) vektorları berilgen. \(2{\bar{a}}^{2} + 4{\bar{b}}^{2} - 5{\bar{c}}^{2}\) ańlatpasınıń mánisin tabıń. &  \\
\hline
9. & \(3 x - y + 5 = 0, x + 3 y - 4 = 0\) tuwrı sızıqları arasındaǵı múyeshti tabıń. &  \\
\hline
10. & \(x^{2} + y^{2} - 2 x + 4 y - 20 = 0\) sheńberdiń \(C\) orayın hám \(R\) radiusın tabıń. & \\
\hline
\end{tabular}

\vspace{0.7cm}

\begin{tabular}{lll}
Tuwrı juwaplar sanı: \underline{\hspace{1cm}} & 
Bahası: \underline{\hspace{1cm}} & 
Imtixan alıwshınıń qolı: \underline{\hspace{2cm}} \\
\end{tabular}

\egroup

\newpage


\textbf{63-variant}\\

\bgroup
\def\arraystretch{1.6} % 1 is the default, change whatever you need

\begin{tabular}{|m{5.7cm}|m{9.5cm}|}
\hline
Familiyası hám atı & \\
\hline
Fakulteti  & \\
\hline
Toparı hám tálim baǵdarı  & \\
\hline
\end{tabular}

\vspace{0.7cm}

\begin{tabular}{|m{0.7cm}|m{10cm}|m{4cm}|}
\hline
№ & Soraw & Juwap \\
\hline
1. & \(OXY\) tegisliginiń teńlemesi? &  \\
\hline
2. & undefined &  \\
\hline
3. & undefined &  \\
\hline
4. & undefined &  \\
\hline
5. & \(A_{1}x + B_{1}y + C_{1}z + D_{1} = 0\) hám \(Ax + By + Cz + D = 0\) tegislikleri perpendikulyar bolıwı ushın qaysı shárt orınlı bolıwı kerek? &  \\
\hline
6. & Eger \(2 a = 16, e = \frac{5}{4}\) bolsa, fokusı abscissa kósherinde, koordinata basına salıstırǵanda simmetriyalıq jaylasqan giperbolanıń teńlemesin dúziń. &  \\
\hline
7. & \(9 x^{2} + 25 y^{2} = 225\) ellipsi berilgen, ellipstiń fokusların, ekscentrisitetin tabıń. &  \\
\hline
8. & \(M_{1} (12; - 1)\) hám \(M_{2} (0;4)\) noqatlardıń arasındaǵı aralıqtı tabıń. &  \\
\hline
9. & \(x + y - 3 = 0\) hám \(2 x + 3 y - 8 = 0\) tuwrıları óz-ara qanday jaylasqan? &  \\
\hline
10. & \(x^{2} + y^{2} - 2 x + 4 y = 0\) sheńberdiń teńlemesin kanonikalıq túrdegi teńlemege alıp keliń. & \\
\hline
\end{tabular}

\vspace{0.7cm}

\begin{tabular}{lll}
Tuwrı juwaplar sanı: \underline{\hspace{1cm}} & 
Bahası: \underline{\hspace{1cm}} & 
Imtixan alıwshınıń qolı: \underline{\hspace{2cm}} \\
\end{tabular}

\egroup

\newpage


\textbf{64-variant}\\

\bgroup
\def\arraystretch{1.6} % 1 is the default, change whatever you need

\begin{tabular}{|m{5.7cm}|m{9.5cm}|}
\hline
Familiyası hám atı & \\
\hline
Fakulteti  & \\
\hline
Toparı hám tálim baǵdarı  & \\
\hline
\end{tabular}

\vspace{0.7cm}

\begin{tabular}{|m{0.7cm}|m{10cm}|m{4cm}|}
\hline
№ & Soraw & Juwap \\
\hline
1. & Eki tuwrı sızıq arasındaǵı múyeshti tabıw formulası? &  \\
\hline
2. & undefined &  \\
\hline
3. & undefined &  \\
\hline
4. & undefined &  \\
\hline
5. & \((x + 1) ^{2} + (y - 2) ^{2} + (z + 2) ^{2} = 49\) sferanıń orayınıń koordinataların tabıń. &  \\
\hline
6. & \(3 x^{2} + 10 xy + 3 y^{2} - 2 x - 14 y - 13 = 0\) teńlemesiniń tipin anıqlań. &  \\
\hline
7. & Eger \(2 b = 24, 2 c = 10\) bolsa, onda abscissa kósherinde koordinata basına salıstırǵanda simmetriyalıq jaylasqan fokuslarǵa iye, ellipstiń teńlemesin dúziń. &  \\
\hline
8. & \(\bar{a} = \left\{ 2, 1, 0 \right\}\) hám \(\bar{b} = \left\{ 1, 0,- 1 \right\}\) bolsa, \(\bar{a} - \bar{b}\) ni tabıń. &  \\
\hline
9. & \(2 x + 3 y + 4 = 0\) tuwrısına parallel hám \(M_{0} (2;1)\) noqattan ótetuǵın tuwrınıń teńlemesin dúziń. &  \\
\hline
10. & \(x + y - 12 = 0\) tuwrısı \(x^{2} + y^{2} - 2 y = 0\) sheńberge salıstırǵanda qanday jaylasqan? & \\
\hline
\end{tabular}

\vspace{0.7cm}

\begin{tabular}{lll}
Tuwrı juwaplar sanı: \underline{\hspace{1cm}} & 
Bahası: \underline{\hspace{1cm}} & 
Imtixan alıwshınıń qolı: \underline{\hspace{2cm}} \\
\end{tabular}

\egroup

\newpage


\textbf{65-variant}\\

\bgroup
\def\arraystretch{1.6} % 1 is the default, change whatever you need

\begin{tabular}{|m{5.7cm}|m{9.5cm}|}
\hline
Familiyası hám atı & \\
\hline
Fakulteti  & \\
\hline
Toparı hám tálim baǵdarı  & \\
\hline
\end{tabular}

\vspace{0.7cm}

\begin{tabular}{|m{0.7cm}|m{10cm}|m{4cm}|}
\hline
№ & Soraw & Juwap \\
\hline
1. & \(OY\) kósheriniń teńlemesi? &  \\
\hline
2. & undefined &  \\
\hline
3. & undefined &  \\
\hline
4. & undefined &  \\
\hline
5. & \(x + 2 = 0\) keńislik qanday geometriyalıq betlikti anıqlaydı? &  \\
\hline
6. & \(\frac{x^{2}}{a^{2}} + \frac{y^{2}}{b^{2}} = 1\) ellipstiń \((x_{0};y_{0})\) noqatındaǵı urınbasınıń teńlemesin tabıń. &  \\
\hline
7. & \(A (- 1;0;1),\ B (1; - 1;0)\) noqatları berilgen. \(\bar{BA}\) vektorın tabıń. &  \\
\hline
8. & $(2, 3)$ hám $(4, 3)$ noqatlarınan ótiwshi tuwrı sızıqtıń teńlemesin dúziń. &  \\
\hline
9. & \(A_{1}x + B_{1}y + C_{1}z + D_{1} = 0\) hám \(Ax + By + Cz + D = 0\) tegislikleri parallel bolıwı ushın qaysı shárt orınlı bolıwı kerek? &  \\
\hline
10. & \(\bar{a} = \left\{ 4,- 2,- 4 \right\}\) hám \(\bar{b} = \left\{ 6,- 3, 2 \right\}\) vektorları berilgen, \((\bar{a} - \bar{b}) ^{2}\)-? & \\
\hline
\end{tabular}

\vspace{0.7cm}

\begin{tabular}{lll}
Tuwrı juwaplar sanı: \underline{\hspace{1cm}} & 
Bahası: \underline{\hspace{1cm}} & 
Imtixan alıwshınıń qolı: \underline{\hspace{2cm}} \\
\end{tabular}

\egroup

\newpage


\textbf{66-variant}\\

\bgroup
\def\arraystretch{1.6} % 1 is the default, change whatever you need

\begin{tabular}{|m{5.7cm}|m{9.5cm}|}
\hline
Familiyası hám atı & \\
\hline
Fakulteti  & \\
\hline
Toparı hám tálim baǵdarı  & \\
\hline
\end{tabular}

\vspace{0.7cm}

\begin{tabular}{|m{0.7cm}|m{10cm}|m{4cm}|}
\hline
№ & Soraw & Juwap \\
\hline
1. & \(Ax + By + D = 0\) teńlemesi arqalı ... tegisliktiń teńlemesi berilgen? &  \\
\hline
2. & undefined &  \\
\hline
3. & undefined &  \\
\hline
4. & undefined &  \\
\hline
5. & \(2 x + 3 y - 6 = 0\) tuwrınıń teńlemesin kesindilerde berilgen teńleme túrinde kórsetiń. &  \\
\hline
6. & \(A (4, 3), B (7, 7)\) noqatları arasındaǵı aralıqtı tabıń. &  \\
\hline
7. & \(M_{1}M_{2}\) kesindiniń ortasınıń koordinatalarınıń tabıń, eger \(M_{1} (2, 3), M_{2} (4, 7)\) bolsa. &  \\
\hline
8. & \(x + y = 0\) teńlemesi menen berilgen tuwrı sızıqtıń múyeshlik koefficientin anıqlań. &  \\
\hline
9. & \(5 x - y + 7 = 0\) hám \(3 x + 2 y = 0\) tuwrıları arasındaǵı múyeshni tabıń. &  \\
\hline
10. & Koordinatalar kósherleri hám \( 3 x + 4 y - 12 = 0 \) tuwrı sızıǵı menen shegaralanǵan úshmúyeshliktiń maydanın tabıń. & \\
\hline
\end{tabular}

\vspace{0.7cm}

\begin{tabular}{lll}
Tuwrı juwaplar sanı: \underline{\hspace{1cm}} & 
Bahası: \underline{\hspace{1cm}} & 
Imtixan alıwshınıń qolı: \underline{\hspace{2cm}} \\
\end{tabular}

\egroup

\newpage


\textbf{67-variant}\\

\bgroup
\def\arraystretch{1.6} % 1 is the default, change whatever you need

\begin{tabular}{|m{5.7cm}|m{9.5cm}|}
\hline
Familiyası hám atı & \\
\hline
Fakulteti  & \\
\hline
Toparı hám tálim baǵdarı  & \\
\hline
\end{tabular}

\vspace{0.7cm}

\begin{tabular}{|m{0.7cm}|m{10cm}|m{4cm}|}
\hline
№ & Soraw & Juwap \\
\hline
1. & Vektorlardı qosıw tómendegi qaysı qásiyetke iye emes? &  \\
\hline
2. & undefined &  \\
\hline
3. & undefined &  \\
\hline
4. & undefined &  \\
\hline
5. & \(\left| \bar{a} \right| = 8, \left| \bar{b} \right| = 5, \alpha = 60^{0}\) bolsa, \(( \bar{a}\bar{b} )\) ni tabıń. &  \\
\hline
6. & \(x - 2 y + 1 = 0\) teńlemesi menen berilgen tuwrınıń normal túrdegi teńlemesin kórsetiń. &  \\
\hline
7. & Orayı \(C (- 1;2)\) noqatında, \(A (- 2;6 )\) noqatınan ótetuǵın sheńberdiń teńlemesin dúziń. &  \\
\hline
8. & \(A_{1}x + B_{1}y + C_{1}z + D_{1} = 0\) hám tegislikleri ústpe-úst túsiwi ushın qaysı shárt orınlı bolıwı kerek? &  \\
\hline
9. & \(\frac{x^{2}}{225} - \frac{y^{2}}{64} = - 1\) giperbola fokusınıń koordinatalarınıń tabıń. &  \\
\hline
10. & \(x^{2} - 4 y^{2} + 6 x + 5 = 0\) giperbolanıń kanonikalıq teńlemesin dúziń. & \\
\hline
\end{tabular}

\vspace{0.7cm}

\begin{tabular}{lll}
Tuwrı juwaplar sanı: \underline{\hspace{1cm}} & 
Bahası: \underline{\hspace{1cm}} & 
Imtixan alıwshınıń qolı: \underline{\hspace{2cm}} \\
\end{tabular}

\egroup

\newpage


\textbf{68-variant}\\

\bgroup
\def\arraystretch{1.6} % 1 is the default, change whatever you need

\begin{tabular}{|m{5.7cm}|m{9.5cm}|}
\hline
Familiyası hám atı & \\
\hline
Fakulteti  & \\
\hline
Toparı hám tálim baǵdarı  & \\
\hline
\end{tabular}

\vspace{0.7cm}

\begin{tabular}{|m{0.7cm}|m{10cm}|m{4cm}|}
\hline
№ & Soraw & Juwap \\
\hline
1. & Giperbolanıń kanonikalıq teńlemesi? &  \\
\hline
2. & undefined &  \\
\hline
3. & undefined &  \\
\hline
4. & undefined &  \\
\hline
5. & \(\overline{a} = \{5,- 6, 1 \}, \overline{b} = \{ - 4, 3, 0 \} \), \(\overline{c} = \left\{ 5,- 8, 10 \right\}\) vektorları berilgen. \(2{\bar{a}}^{2} + 4{\bar{b}}^{2} - 5{\bar{c}}^{2}\) ańlatpasınıń mánisin tabıń. &  \\
\hline
6. & \(3 x - y + 5 = 0, x + 3 y - 4 = 0\) tuwrı sızıqları arasındaǵı múyeshti tabıń. &  \\
\hline
7. & \(x^{2} + y^{2} - 2 x + 4 y - 20 = 0\) sheńberdiń \(C\) orayın hám \(R\) radiusın tabıń. &  \\
\hline
8. & \(A_{1}x + B_{1}y + C_{1}z + D_{1} = 0\) hám \(Ax + By + Cz + D = 0\) tegislikleri perpendikulyar bolıwı ushın qaysı shárt orınlı bolıwı kerek? &  \\
\hline
9. & Eger \(2 a = 16, e = \frac{5}{4}\) bolsa, fokusı abscissa kósherinde, koordinata basına salıstırǵanda simmetriyalıq jaylasqan giperbolanıń teńlemesin dúziń. &  \\
\hline
10. & \(9 x^{2} + 25 y^{2} = 225\) ellipsi berilgen, ellipstiń fokusların, ekscentrisitetin tabıń. & \\
\hline
\end{tabular}

\vspace{0.7cm}

\begin{tabular}{lll}
Tuwrı juwaplar sanı: \underline{\hspace{1cm}} & 
Bahası: \underline{\hspace{1cm}} & 
Imtixan alıwshınıń qolı: \underline{\hspace{2cm}} \\
\end{tabular}

\egroup

\newpage


\textbf{69-variant}\\

\bgroup
\def\arraystretch{1.6} % 1 is the default, change whatever you need

\begin{tabular}{|m{5.7cm}|m{9.5cm}|}
\hline
Familiyası hám atı & \\
\hline
Fakulteti  & \\
\hline
Toparı hám tálim baǵdarı  & \\
\hline
\end{tabular}

\vspace{0.7cm}

\begin{tabular}{|m{0.7cm}|m{10cm}|m{4cm}|}
\hline
№ & Soraw & Juwap \\
\hline
1. & \(\frac{x^{2}}{a^{2}} - \frac{y^{2}}{b^{2}} = 1\) giperbolanıń \((x_{0};y_{0})\) noqatındaǵı urınbasınıń teńlemesin kórsetiń. &  \\
\hline
2. & undefined &  \\
\hline
3. & undefined &  \\
\hline
4. & undefined &  \\
\hline
5. & \(M_{1} (12; - 1)\) hám \(M_{2} (0;4)\) noqatlardıń arasındaǵı aralıqtı tabıń. &  \\
\hline
6. & \(x + y - 3 = 0\) hám \(2 x + 3 y - 8 = 0\) tuwrıları óz-ara qanday jaylasqan? &  \\
\hline
7. & \(x^{2} + y^{2} - 2 x + 4 y = 0\) sheńberdiń teńlemesin kanonikalıq túrdegi teńlemege alıp keliń. &  \\
\hline
8. & \((x + 1) ^{2} + (y - 2) ^{2} + (z + 2) ^{2} = 49\) sferanıń orayınıń koordinataların tabıń. &  \\
\hline
9. & \(3 x^{2} + 10 xy + 3 y^{2} - 2 x - 14 y - 13 = 0\) teńlemesiniń tipin anıqlań. &  \\
\hline
10. & Eger \(2 b = 24, 2 c = 10\) bolsa, onda abscissa kósherinde koordinata basına salıstırǵanda simmetriyalıq jaylasqan fokuslarǵa iye, ellipstiń teńlemesin dúziń. & \\
\hline
\end{tabular}

\vspace{0.7cm}

\begin{tabular}{lll}
Tuwrı juwaplar sanı: \underline{\hspace{1cm}} & 
Bahası: \underline{\hspace{1cm}} & 
Imtixan alıwshınıń qolı: \underline{\hspace{2cm}} \\
\end{tabular}

\egroup

\newpage


\textbf{70-variant}\\

\bgroup
\def\arraystretch{1.6} % 1 is the default, change whatever you need

\begin{tabular}{|m{5.7cm}|m{9.5cm}|}
\hline
Familiyası hám atı & \\
\hline
Fakulteti  & \\
\hline
Toparı hám tálim baǵdarı  & \\
\hline
\end{tabular}

\vspace{0.7cm}

\begin{tabular}{|m{0.7cm}|m{10cm}|m{4cm}|}
\hline
№ & Soraw & Juwap \\
\hline
1. & Vektorlardı qosıw koordinatalarda qanday formula menen anıqlanadı? &  \\
\hline
2. & undefined &  \\
\hline
3. & undefined &  \\
\hline
4. & undefined &  \\
\hline
5. & \(\bar{a} = \left\{ 2, 1, 0 \right\}\) hám \(\bar{b} = \left\{ 1, 0,- 1 \right\}\) bolsa, \(\bar{a} - \bar{b}\) ni tabıń. &  \\
\hline
6. & \(2 x + 3 y + 4 = 0\) tuwrısına parallel hám \(M_{0} (2;1)\) noqattan ótetuǵın tuwrınıń teńlemesin dúziń. &  \\
\hline
7. & \(x + y - 12 = 0\) tuwrısı \(x^{2} + y^{2} - 2 y = 0\) sheńberge salıstırǵanda qanday jaylasqan? &  \\
\hline
8. & \(x + 2 = 0\) keńislik qanday geometriyalıq betlikti anıqlaydı? &  \\
\hline
9. & \(\frac{x^{2}}{a^{2}} + \frac{y^{2}}{b^{2}} = 1\) ellipstiń \((x_{0};y_{0})\) noqatındaǵı urınbasınıń teńlemesin tabıń. &  \\
\hline
10. & \(A (- 1;0;1),\ B (1; - 1;0)\) noqatları berilgen. \(\bar{BA}\) vektorın tabıń. & \\
\hline
\end{tabular}

\vspace{0.7cm}

\begin{tabular}{lll}
Tuwrı juwaplar sanı: \underline{\hspace{1cm}} & 
Bahası: \underline{\hspace{1cm}} & 
Imtixan alıwshınıń qolı: \underline{\hspace{2cm}} \\
\end{tabular}

\egroup

\newpage


\textbf{71-variant}\\

\bgroup
\def\arraystretch{1.6} % 1 is the default, change whatever you need

\begin{tabular}{|m{5.7cm}|m{9.5cm}|}
\hline
Familiyası hám atı & \\
\hline
Fakulteti  & \\
\hline
Toparı hám tálim baǵdarı  & \\
\hline
\end{tabular}

\vspace{0.7cm}

\begin{tabular}{|m{0.7cm}|m{10cm}|m{4cm}|}
\hline
№ & Soraw & Juwap \\
\hline
1. & Eki vektordıń vektor kóbeymesiniń uzınlıǵın tabıw formulası? &  \\
\hline
2. & undefined &  \\
\hline
3. & undefined &  \\
\hline
4. & undefined &  \\
\hline
5. & $(2, 3)$ hám $(4, 3)$ noqatlarınan ótiwshi tuwrı sızıqtıń teńlemesin dúziń. &  \\
\hline
6. & \(A_{1}x + B_{1}y + C_{1}z + D_{1} = 0\) hám \(Ax + By + Cz + D = 0\) tegislikleri parallel bolıwı ushın qaysı shárt orınlı bolıwı kerek? &  \\
\hline
7. & \(\bar{a} = \left\{ 4,- 2,- 4 \right\}\) hám \(\bar{b} = \left\{ 6,- 3, 2 \right\}\) vektorları berilgen, \((\bar{a} - \bar{b}) ^{2}\)-? &  \\
\hline
8. & \(2 x + 3 y - 6 = 0\) tuwrınıń teńlemesin kesindilerde berilgen teńleme túrinde kórsetiń. &  \\
\hline
9. & \(A (4, 3), B (7, 7)\) noqatları arasındaǵı aralıqtı tabıń. &  \\
\hline
10. & \(M_{1}M_{2}\) kesindiniń ortasınıń koordinatalarınıń tabıń, eger \(M_{1} (2, 3), M_{2} (4, 7)\) bolsa. & \\
\hline
\end{tabular}

\vspace{0.7cm}

\begin{tabular}{lll}
Tuwrı juwaplar sanı: \underline{\hspace{1cm}} & 
Bahası: \underline{\hspace{1cm}} & 
Imtixan alıwshınıń qolı: \underline{\hspace{2cm}} \\
\end{tabular}

\egroup

\newpage


\textbf{72-variant}\\

\bgroup
\def\arraystretch{1.6} % 1 is the default, change whatever you need

\begin{tabular}{|m{5.7cm}|m{9.5cm}|}
\hline
Familiyası hám atı & \\
\hline
Fakulteti  & \\
\hline
Toparı hám tálim baǵdarı  & \\
\hline
\end{tabular}

\vspace{0.7cm}

\begin{tabular}{|m{0.7cm}|m{10cm}|m{4cm}|}
\hline
№ & Soraw & Juwap \\
\hline
1. & Tegislikdegi qálegen noqattan berilgen eki noqatqa shekemgi bolǵan aralıqlardıń ayırmasınıń modulı ózgermeytuǵın bolǵan noqatlardıń geometriyalıq ornı ne dep ataladı? &  \\
\hline
2. & undefined &  \\
\hline
3. & undefined &  \\
\hline
4. & undefined &  \\
\hline
5. & \(x + y = 0\) teńlemesi menen berilgen tuwrı sızıqtıń múyeshlik koefficientin anıqlań. &  \\
\hline
6. & \(5 x - y + 7 = 0\) hám \(3 x + 2 y = 0\) tuwrıları arasındaǵı múyeshni tabıń. &  \\
\hline
7. & Koordinatalar kósherleri hám \( 3 x + 4 y - 12 = 0 \) tuwrı sızıǵı menen shegaralanǵan úshmúyeshliktiń maydanın tabıń. &  \\
\hline
8. & \(\left| \bar{a} \right| = 8, \left| \bar{b} \right| = 5, \alpha = 60^{0}\) bolsa, \(( \bar{a}\bar{b} )\) ni tabıń. &  \\
\hline
9. & \(x - 2 y + 1 = 0\) teńlemesi menen berilgen tuwrınıń normal túrdegi teńlemesin kórsetiń. &  \\
\hline
10. & Orayı \(C (- 1;2)\) noqatında, \(A (- 2;6 )\) noqatınan ótetuǵın sheńberdiń teńlemesin dúziń. & \\
\hline
\end{tabular}

\vspace{0.7cm}

\begin{tabular}{lll}
Tuwrı juwaplar sanı: \underline{\hspace{1cm}} & 
Bahası: \underline{\hspace{1cm}} & 
Imtixan alıwshınıń qolı: \underline{\hspace{2cm}} \\
\end{tabular}

\egroup

\newpage


\textbf{73-variant}\\

\bgroup
\def\arraystretch{1.6} % 1 is the default, change whatever you need

\begin{tabular}{|m{5.7cm}|m{9.5cm}|}
\hline
Familiyası hám atı & \\
\hline
Fakulteti  & \\
\hline
Toparı hám tálim baǵdarı  & \\
\hline
\end{tabular}

\vspace{0.7cm}

\begin{tabular}{|m{0.7cm}|m{10cm}|m{4cm}|}
\hline
№ & Soraw & Juwap \\
\hline
1. & Eki vektor qashan kollinear dep ataladı? &  \\
\hline
2. & undefined &  \\
\hline
3. & undefined &  \\
\hline
4. & undefined &  \\
\hline
5. & \(A_{1}x + B_{1}y + C_{1}z + D_{1} = 0\) hám tegislikleri ústpe-úst túsiwi ushın qaysı shárt orınlı bolıwı kerek? &  \\
\hline
6. & \(\frac{x^{2}}{225} - \frac{y^{2}}{64} = - 1\) giperbola fokusınıń koordinatalarınıń tabıń. &  \\
\hline
7. & \(x^{2} - 4 y^{2} + 6 x + 5 = 0\) giperbolanıń kanonikalıq teńlemesin dúziń. &  \\
\hline
8. & \(\overline{a} = \{5,- 6, 1 \}, \overline{b} = \{ - 4, 3, 0 \} \), \(\overline{c} = \left\{ 5,- 8, 10 \right\}\) vektorları berilgen. \(2{\bar{a}}^{2} + 4{\bar{b}}^{2} - 5{\bar{c}}^{2}\) ańlatpasınıń mánisin tabıń. &  \\
\hline
9. & \(3 x - y + 5 = 0, x + 3 y - 4 = 0\) tuwrı sızıqları arasındaǵı múyeshti tabıń. &  \\
\hline
10. & \(x^{2} + y^{2} - 2 x + 4 y - 20 = 0\) sheńberdiń \(C\) orayın hám \(R\) radiusın tabıń. & \\
\hline
\end{tabular}

\vspace{0.7cm}

\begin{tabular}{lll}
Tuwrı juwaplar sanı: \underline{\hspace{1cm}} & 
Bahası: \underline{\hspace{1cm}} & 
Imtixan alıwshınıń qolı: \underline{\hspace{2cm}} \\
\end{tabular}

\egroup

\newpage


\textbf{74-variant}\\

\bgroup
\def\arraystretch{1.6} % 1 is the default, change whatever you need

\begin{tabular}{|m{5.7cm}|m{9.5cm}|}
\hline
Familiyası hám atı & \\
\hline
Fakulteti  & \\
\hline
Toparı hám tálim baǵdarı  & \\
\hline
\end{tabular}

\vspace{0.7cm}

\begin{tabular}{|m{0.7cm}|m{10cm}|m{4cm}|}
\hline
№ & Soraw & Juwap \\
\hline
1. & Egerde \(a = \{ x_{1}; y_{1}; z_{1}\}, b = \{ x_{2}; y_{2}; z_{2}\}\) bolsa, vektor kóbeymeniń koordinatalarda ańlatılıwı qanday boladı? &  \\
\hline
2. & undefined &  \\
\hline
3. & undefined &  \\
\hline
4. & undefined &  \\
\hline
5. & \(A_{1}x + B_{1}y + C_{1}z + D_{1} = 0\) hám \(Ax + By + Cz + D = 0\) tegislikleri perpendikulyar bolıwı ushın qaysı shárt orınlı bolıwı kerek? &  \\
\hline
6. & Eger \(2 a = 16, e = \frac{5}{4}\) bolsa, fokusı abscissa kósherinde, koordinata basına salıstırǵanda simmetriyalıq jaylasqan giperbolanıń teńlemesin dúziń. &  \\
\hline
7. & \(9 x^{2} + 25 y^{2} = 225\) ellipsi berilgen, ellipstiń fokusların, ekscentrisitetin tabıń. &  \\
\hline
8. & \(M_{1} (12; - 1)\) hám \(M_{2} (0;4)\) noqatlardıń arasındaǵı aralıqtı tabıń. &  \\
\hline
9. & \(x + y - 3 = 0\) hám \(2 x + 3 y - 8 = 0\) tuwrıları óz-ara qanday jaylasqan? &  \\
\hline
10. & \(x^{2} + y^{2} - 2 x + 4 y = 0\) sheńberdiń teńlemesin kanonikalıq túrdegi teńlemege alıp keliń. & \\
\hline
\end{tabular}

\vspace{0.7cm}

\begin{tabular}{lll}
Tuwrı juwaplar sanı: \underline{\hspace{1cm}} & 
Bahası: \underline{\hspace{1cm}} & 
Imtixan alıwshınıń qolı: \underline{\hspace{2cm}} \\
\end{tabular}

\egroup

\newpage


\textbf{75-variant}\\

\bgroup
\def\arraystretch{1.6} % 1 is the default, change whatever you need

\begin{tabular}{|m{5.7cm}|m{9.5cm}|}
\hline
Familiyası hám atı & \\
\hline
Fakulteti  & \\
\hline
Toparı hám tálim baǵdarı  & \\
\hline
\end{tabular}

\vspace{0.7cm}

\begin{tabular}{|m{0.7cm}|m{10cm}|m{4cm}|}
\hline
№ & Soraw & Juwap \\
\hline
1. & \(Ax + C = 0\) tuwrı sızıqtıń grafigi koordinata kósherlerine salıstırǵanda qanday jaylasqan? &  \\
\hline
2. & undefined &  \\
\hline
3. & undefined &  \\
\hline
4. & undefined &  \\
\hline
5. & \((x + 1) ^{2} + (y - 2) ^{2} + (z + 2) ^{2} = 49\) sferanıń orayınıń koordinataların tabıń. &  \\
\hline
6. & \(3 x^{2} + 10 xy + 3 y^{2} - 2 x - 14 y - 13 = 0\) teńlemesiniń tipin anıqlań. &  \\
\hline
7. & Eger \(2 b = 24, 2 c = 10\) bolsa, onda abscissa kósherinde koordinata basına salıstırǵanda simmetriyalıq jaylasqan fokuslarǵa iye, ellipstiń teńlemesin dúziń. &  \\
\hline
8. & \(\bar{a} = \left\{ 2, 1, 0 \right\}\) hám \(\bar{b} = \left\{ 1, 0,- 1 \right\}\) bolsa, \(\bar{a} - \bar{b}\) ni tabıń. &  \\
\hline
9. & \(2 x + 3 y + 4 = 0\) tuwrısına parallel hám \(M_{0} (2;1)\) noqattan ótetuǵın tuwrınıń teńlemesin dúziń. &  \\
\hline
10. & \(x + y - 12 = 0\) tuwrısı \(x^{2} + y^{2} - 2 y = 0\) sheńberge salıstırǵanda qanday jaylasqan? & \\
\hline
\end{tabular}

\vspace{0.7cm}

\begin{tabular}{lll}
Tuwrı juwaplar sanı: \underline{\hspace{1cm}} & 
Bahası: \underline{\hspace{1cm}} & 
Imtixan alıwshınıń qolı: \underline{\hspace{2cm}} \\
\end{tabular}

\egroup

\newpage


\textbf{76-variant}\\

\bgroup
\def\arraystretch{1.6} % 1 is the default, change whatever you need

\begin{tabular}{|m{5.7cm}|m{9.5cm}|}
\hline
Familiyası hám atı & \\
\hline
Fakulteti  & \\
\hline
Toparı hám tálim baǵdarı  & \\
\hline
\end{tabular}

\vspace{0.7cm}

\begin{tabular}{|m{0.7cm}|m{10cm}|m{4cm}|}
\hline
№ & Soraw & Juwap \\
\hline
1. & Úsh vektordıń aralas kóbeymesi ushın \((abc) = 0\) teńligi orınlı bolsa ne dep ataladı? &  \\
\hline
2. & undefined &  \\
\hline
3. & undefined &  \\
\hline
4. & undefined &  \\
\hline
5. & \(x + 2 = 0\) keńislik qanday geometriyalıq betlikti anıqlaydı? &  \\
\hline
6. & \(\frac{x^{2}}{a^{2}} + \frac{y^{2}}{b^{2}} = 1\) ellipstiń \((x_{0};y_{0})\) noqatındaǵı urınbasınıń teńlemesin tabıń. &  \\
\hline
7. & \(A (- 1;0;1),\ B (1; - 1;0)\) noqatları berilgen. \(\bar{BA}\) vektorın tabıń. &  \\
\hline
8. & $(2, 3)$ hám $(4, 3)$ noqatlarınan ótiwshi tuwrı sızıqtıń teńlemesin dúziń. &  \\
\hline
9. & \(A_{1}x + B_{1}y + C_{1}z + D_{1} = 0\) hám \(Ax + By + Cz + D = 0\) tegislikleri parallel bolıwı ushın qaysı shárt orınlı bolıwı kerek? &  \\
\hline
10. & \(\bar{a} = \left\{ 4,- 2,- 4 \right\}\) hám \(\bar{b} = \left\{ 6,- 3, 2 \right\}\) vektorları berilgen, \((\bar{a} - \bar{b}) ^{2}\)-? & \\
\hline
\end{tabular}

\vspace{0.7cm}

\begin{tabular}{lll}
Tuwrı juwaplar sanı: \underline{\hspace{1cm}} & 
Bahası: \underline{\hspace{1cm}} & 
Imtixan alıwshınıń qolı: \underline{\hspace{2cm}} \\
\end{tabular}

\egroup

\newpage


\textbf{77-variant}\\

\bgroup
\def\arraystretch{1.6} % 1 is the default, change whatever you need

\begin{tabular}{|m{5.7cm}|m{9.5cm}|}
\hline
Familiyası hám atı & \\
\hline
Fakulteti  & \\
\hline
Toparı hám tálim baǵdarı  & \\
\hline
\end{tabular}

\vspace{0.7cm}

\begin{tabular}{|m{0.7cm}|m{10cm}|m{4cm}|}
\hline
№ & Soraw & Juwap \\
\hline
1. & Tuwrı múyeshli koordinatalar sisteması dep nege aytamız? &  \\
\hline
2. & undefined &  \\
\hline
3. & undefined &  \\
\hline
4. & undefined &  \\
\hline
5. & \(2 x + 3 y - 6 = 0\) tuwrınıń teńlemesin kesindilerde berilgen teńleme túrinde kórsetiń. &  \\
\hline
6. & \(A (4, 3), B (7, 7)\) noqatları arasındaǵı aralıqtı tabıń. &  \\
\hline
7. & \(M_{1}M_{2}\) kesindiniń ortasınıń koordinatalarınıń tabıń, eger \(M_{1} (2, 3), M_{2} (4, 7)\) bolsa. &  \\
\hline
8. & \(x + y = 0\) teńlemesi menen berilgen tuwrı sızıqtıń múyeshlik koefficientin anıqlań. &  \\
\hline
9. & \(5 x - y + 7 = 0\) hám \(3 x + 2 y = 0\) tuwrıları arasındaǵı múyeshni tabıń. &  \\
\hline
10. & Koordinatalar kósherleri hám \( 3 x + 4 y - 12 = 0 \) tuwrı sızıǵı menen shegaralanǵan úshmúyeshliktiń maydanın tabıń. & \\
\hline
\end{tabular}

\vspace{0.7cm}

\begin{tabular}{lll}
Tuwrı juwaplar sanı: \underline{\hspace{1cm}} & 
Bahası: \underline{\hspace{1cm}} & 
Imtixan alıwshınıń qolı: \underline{\hspace{2cm}} \\
\end{tabular}

\egroup

\newpage


\textbf{78-variant}\\

\bgroup
\def\arraystretch{1.6} % 1 is the default, change whatever you need

\begin{tabular}{|m{5.7cm}|m{9.5cm}|}
\hline
Familiyası hám atı & \\
\hline
Fakulteti  & \\
\hline
Toparı hám tálim baǵdarı  & \\
\hline
\end{tabular}

\vspace{0.7cm}

\begin{tabular}{|m{0.7cm}|m{10cm}|m{4cm}|}
\hline
№ & Soraw & Juwap \\
\hline
1. & Vektorlardıń kósherdegi proekciyasınıń formulası? &  \\
\hline
2. & undefined &  \\
\hline
3. & undefined &  \\
\hline
4. & undefined &  \\
\hline
5. & \(\left| \bar{a} \right| = 8, \left| \bar{b} \right| = 5, \alpha = 60^{0}\) bolsa, \(( \bar{a}\bar{b} )\) ni tabıń. &  \\
\hline
6. & \(x - 2 y + 1 = 0\) teńlemesi menen berilgen tuwrınıń normal túrdegi teńlemesin kórsetiń. &  \\
\hline
7. & Orayı \(C (- 1;2)\) noqatında, \(A (- 2;6 )\) noqatınan ótetuǵın sheńberdiń teńlemesin dúziń. &  \\
\hline
8. & \(A_{1}x + B_{1}y + C_{1}z + D_{1} = 0\) hám tegislikleri ústpe-úst túsiwi ushın qaysı shárt orınlı bolıwı kerek? &  \\
\hline
9. & \(\frac{x^{2}}{225} - \frac{y^{2}}{64} = - 1\) giperbola fokusınıń koordinatalarınıń tabıń. &  \\
\hline
10. & \(x^{2} - 4 y^{2} + 6 x + 5 = 0\) giperbolanıń kanonikalıq teńlemesin dúziń. & \\
\hline
\end{tabular}

\vspace{0.7cm}

\begin{tabular}{lll}
Tuwrı juwaplar sanı: \underline{\hspace{1cm}} & 
Bahası: \underline{\hspace{1cm}} & 
Imtixan alıwshınıń qolı: \underline{\hspace{2cm}} \\
\end{tabular}

\egroup

\newpage


\textbf{79-variant}\\

\bgroup
\def\arraystretch{1.6} % 1 is the default, change whatever you need

\begin{tabular}{|m{5.7cm}|m{9.5cm}|}
\hline
Familiyası hám atı & \\
\hline
Fakulteti  & \\
\hline
Toparı hám tálim baǵdarı  & \\
\hline
\end{tabular}

\vspace{0.7cm}

\begin{tabular}{|m{0.7cm}|m{10cm}|m{4cm}|}
\hline
№ & Soraw & Juwap \\
\hline
1. & Eki vektordıń skalyar kóbeymesiniń formulası? &  \\
\hline
2. & undefined &  \\
\hline
3. & undefined &  \\
\hline
4. & undefined &  \\
\hline
5. & \(\overline{a} = \{5,- 6, 1 \}, \overline{b} = \{ - 4, 3, 0 \} \), \(\overline{c} = \left\{ 5,- 8, 10 \right\}\) vektorları berilgen. \(2{\bar{a}}^{2} + 4{\bar{b}}^{2} - 5{\bar{c}}^{2}\) ańlatpasınıń mánisin tabıń. &  \\
\hline
6. & \(3 x - y + 5 = 0, x + 3 y - 4 = 0\) tuwrı sızıqları arasındaǵı múyeshti tabıń. &  \\
\hline
7. & \(x^{2} + y^{2} - 2 x + 4 y - 20 = 0\) sheńberdiń \(C\) orayın hám \(R\) radiusın tabıń. &  \\
\hline
8. & \(A_{1}x + B_{1}y + C_{1}z + D_{1} = 0\) hám \(Ax + By + Cz + D = 0\) tegislikleri perpendikulyar bolıwı ushın qaysı shárt orınlı bolıwı kerek? &  \\
\hline
9. & Eger \(2 a = 16, e = \frac{5}{4}\) bolsa, fokusı abscissa kósherinde, koordinata basına salıstırǵanda simmetriyalıq jaylasqan giperbolanıń teńlemesin dúziń. &  \\
\hline
10. & \(9 x^{2} + 25 y^{2} = 225\) ellipsi berilgen, ellipstiń fokusların, ekscentrisitetin tabıń. & \\
\hline
\end{tabular}

\vspace{0.7cm}

\begin{tabular}{lll}
Tuwrı juwaplar sanı: \underline{\hspace{1cm}} & 
Bahası: \underline{\hspace{1cm}} & 
Imtixan alıwshınıń qolı: \underline{\hspace{2cm}} \\
\end{tabular}

\egroup

\newpage


\textbf{80-variant}\\

\bgroup
\def\arraystretch{1.6} % 1 is the default, change whatever you need

\begin{tabular}{|m{5.7cm}|m{9.5cm}|}
\hline
Familiyası hám atı & \\
\hline
Fakulteti  & \\
\hline
Toparı hám tálim baǵdarı  & \\
\hline
\end{tabular}

\vspace{0.7cm}

\begin{tabular}{|m{0.7cm}|m{10cm}|m{4cm}|}
\hline
№ & Soraw & Juwap \\
\hline
1. & \(OXY\) tegisliginiń teńlemesi? &  \\
\hline
2. & undefined &  \\
\hline
3. & undefined &  \\
\hline
4. & undefined &  \\
\hline
5. & \(M_{1} (12; - 1)\) hám \(M_{2} (0;4)\) noqatlardıń arasındaǵı aralıqtı tabıń. &  \\
\hline
6. & \(x + y - 3 = 0\) hám \(2 x + 3 y - 8 = 0\) tuwrıları óz-ara qanday jaylasqan? &  \\
\hline
7. & \(x^{2} + y^{2} - 2 x + 4 y = 0\) sheńberdiń teńlemesin kanonikalıq túrdegi teńlemege alıp keliń. &  \\
\hline
8. & \((x + 1) ^{2} + (y - 2) ^{2} + (z + 2) ^{2} = 49\) sferanıń orayınıń koordinataların tabıń. &  \\
\hline
9. & \(3 x^{2} + 10 xy + 3 y^{2} - 2 x - 14 y - 13 = 0\) teńlemesiniń tipin anıqlań. &  \\
\hline
10. & Eger \(2 b = 24, 2 c = 10\) bolsa, onda abscissa kósherinde koordinata basına salıstırǵanda simmetriyalıq jaylasqan fokuslarǵa iye, ellipstiń teńlemesin dúziń. & \\
\hline
\end{tabular}

\vspace{0.7cm}

\begin{tabular}{lll}
Tuwrı juwaplar sanı: \underline{\hspace{1cm}} & 
Bahası: \underline{\hspace{1cm}} & 
Imtixan alıwshınıń qolı: \underline{\hspace{2cm}} \\
\end{tabular}

\egroup

\newpage


\textbf{81-variant}\\

\bgroup
\def\arraystretch{1.6} % 1 is the default, change whatever you need

\begin{tabular}{|m{5.7cm}|m{9.5cm}|}
\hline
Familiyası hám atı & \\
\hline
Fakulteti  & \\
\hline
Toparı hám tálim baǵdarı  & \\
\hline
\end{tabular}

\vspace{0.7cm}

\begin{tabular}{|m{0.7cm}|m{10cm}|m{4cm}|}
\hline
№ & Soraw & Juwap \\
\hline
1. & Eki tuwrı sızıq arasındaǵı múyeshti tabıw formulası? &  \\
\hline
2. & undefined &  \\
\hline
3. & undefined &  \\
\hline
4. & undefined &  \\
\hline
5. & \(\bar{a} = \left\{ 2, 1, 0 \right\}\) hám \(\bar{b} = \left\{ 1, 0,- 1 \right\}\) bolsa, \(\bar{a} - \bar{b}\) ni tabıń. &  \\
\hline
6. & \(2 x + 3 y + 4 = 0\) tuwrısına parallel hám \(M_{0} (2;1)\) noqattan ótetuǵın tuwrınıń teńlemesin dúziń. &  \\
\hline
7. & \(x + y - 12 = 0\) tuwrısı \(x^{2} + y^{2} - 2 y = 0\) sheńberge salıstırǵanda qanday jaylasqan? &  \\
\hline
8. & \(x + 2 = 0\) keńislik qanday geometriyalıq betlikti anıqlaydı? &  \\
\hline
9. & \(\frac{x^{2}}{a^{2}} + \frac{y^{2}}{b^{2}} = 1\) ellipstiń \((x_{0};y_{0})\) noqatındaǵı urınbasınıń teńlemesin tabıń. &  \\
\hline
10. & \(A (- 1;0;1),\ B (1; - 1;0)\) noqatları berilgen. \(\bar{BA}\) vektorın tabıń. & \\
\hline
\end{tabular}

\vspace{0.7cm}

\begin{tabular}{lll}
Tuwrı juwaplar sanı: \underline{\hspace{1cm}} & 
Bahası: \underline{\hspace{1cm}} & 
Imtixan alıwshınıń qolı: \underline{\hspace{2cm}} \\
\end{tabular}

\egroup

\newpage


\textbf{82-variant}\\

\bgroup
\def\arraystretch{1.6} % 1 is the default, change whatever you need

\begin{tabular}{|m{5.7cm}|m{9.5cm}|}
\hline
Familiyası hám atı & \\
\hline
Fakulteti  & \\
\hline
Toparı hám tálim baǵdarı  & \\
\hline
\end{tabular}

\vspace{0.7cm}

\begin{tabular}{|m{0.7cm}|m{10cm}|m{4cm}|}
\hline
№ & Soraw & Juwap \\
\hline
1. & \(OY\) kósheriniń teńlemesi? &  \\
\hline
2. & undefined &  \\
\hline
3. & undefined &  \\
\hline
4. & undefined &  \\
\hline
5. & $(2, 3)$ hám $(4, 3)$ noqatlarınan ótiwshi tuwrı sızıqtıń teńlemesin dúziń. &  \\
\hline
6. & \(A_{1}x + B_{1}y + C_{1}z + D_{1} = 0\) hám \(Ax + By + Cz + D = 0\) tegislikleri parallel bolıwı ushın qaysı shárt orınlı bolıwı kerek? &  \\
\hline
7. & \(\bar{a} = \left\{ 4,- 2,- 4 \right\}\) hám \(\bar{b} = \left\{ 6,- 3, 2 \right\}\) vektorları berilgen, \((\bar{a} - \bar{b}) ^{2}\)-? &  \\
\hline
8. & \(2 x + 3 y - 6 = 0\) tuwrınıń teńlemesin kesindilerde berilgen teńleme túrinde kórsetiń. &  \\
\hline
9. & \(A (4, 3), B (7, 7)\) noqatları arasındaǵı aralıqtı tabıń. &  \\
\hline
10. & \(M_{1}M_{2}\) kesindiniń ortasınıń koordinatalarınıń tabıń, eger \(M_{1} (2, 3), M_{2} (4, 7)\) bolsa. & \\
\hline
\end{tabular}

\vspace{0.7cm}

\begin{tabular}{lll}
Tuwrı juwaplar sanı: \underline{\hspace{1cm}} & 
Bahası: \underline{\hspace{1cm}} & 
Imtixan alıwshınıń qolı: \underline{\hspace{2cm}} \\
\end{tabular}

\egroup

\newpage


\textbf{83-variant}\\

\bgroup
\def\arraystretch{1.6} % 1 is the default, change whatever you need

\begin{tabular}{|m{5.7cm}|m{9.5cm}|}
\hline
Familiyası hám atı & \\
\hline
Fakulteti  & \\
\hline
Toparı hám tálim baǵdarı  & \\
\hline
\end{tabular}

\vspace{0.7cm}

\begin{tabular}{|m{0.7cm}|m{10cm}|m{4cm}|}
\hline
№ & Soraw & Juwap \\
\hline
1. & \(Ax + By + D = 0\) teńlemesi arqalı ... tegisliktiń teńlemesi berilgen? &  \\
\hline
2. & undefined &  \\
\hline
3. & undefined &  \\
\hline
4. & undefined &  \\
\hline
5. & \(x + y = 0\) teńlemesi menen berilgen tuwrı sızıqtıń múyeshlik koefficientin anıqlań. &  \\
\hline
6. & \(5 x - y + 7 = 0\) hám \(3 x + 2 y = 0\) tuwrıları arasındaǵı múyeshni tabıń. &  \\
\hline
7. & Koordinatalar kósherleri hám \( 3 x + 4 y - 12 = 0 \) tuwrı sızıǵı menen shegaralanǵan úshmúyeshliktiń maydanın tabıń. &  \\
\hline
8. & \(\left| \bar{a} \right| = 8, \left| \bar{b} \right| = 5, \alpha = 60^{0}\) bolsa, \(( \bar{a}\bar{b} )\) ni tabıń. &  \\
\hline
9. & \(x - 2 y + 1 = 0\) teńlemesi menen berilgen tuwrınıń normal túrdegi teńlemesin kórsetiń. &  \\
\hline
10. & Orayı \(C (- 1;2)\) noqatında, \(A (- 2;6 )\) noqatınan ótetuǵın sheńberdiń teńlemesin dúziń. & \\
\hline
\end{tabular}

\vspace{0.7cm}

\begin{tabular}{lll}
Tuwrı juwaplar sanı: \underline{\hspace{1cm}} & 
Bahası: \underline{\hspace{1cm}} & 
Imtixan alıwshınıń qolı: \underline{\hspace{2cm}} \\
\end{tabular}

\egroup

\newpage


\textbf{84-variant}\\

\bgroup
\def\arraystretch{1.6} % 1 is the default, change whatever you need

\begin{tabular}{|m{5.7cm}|m{9.5cm}|}
\hline
Familiyası hám atı & \\
\hline
Fakulteti  & \\
\hline
Toparı hám tálim baǵdarı  & \\
\hline
\end{tabular}

\vspace{0.7cm}

\begin{tabular}{|m{0.7cm}|m{10cm}|m{4cm}|}
\hline
№ & Soraw & Juwap \\
\hline
1. & Vektorlardı qosıw tómendegi qaysı qásiyetke iye emes? &  \\
\hline
2. & undefined &  \\
\hline
3. & undefined &  \\
\hline
4. & undefined &  \\
\hline
5. & \(A_{1}x + B_{1}y + C_{1}z + D_{1} = 0\) hám tegislikleri ústpe-úst túsiwi ushın qaysı shárt orınlı bolıwı kerek? &  \\
\hline
6. & \(\frac{x^{2}}{225} - \frac{y^{2}}{64} = - 1\) giperbola fokusınıń koordinatalarınıń tabıń. &  \\
\hline
7. & \(x^{2} - 4 y^{2} + 6 x + 5 = 0\) giperbolanıń kanonikalıq teńlemesin dúziń. &  \\
\hline
8. & \(\overline{a} = \{5,- 6, 1 \}, \overline{b} = \{ - 4, 3, 0 \} \), \(\overline{c} = \left\{ 5,- 8, 10 \right\}\) vektorları berilgen. \(2{\bar{a}}^{2} + 4{\bar{b}}^{2} - 5{\bar{c}}^{2}\) ańlatpasınıń mánisin tabıń. &  \\
\hline
9. & \(3 x - y + 5 = 0, x + 3 y - 4 = 0\) tuwrı sızıqları arasındaǵı múyeshti tabıń. &  \\
\hline
10. & \(x^{2} + y^{2} - 2 x + 4 y - 20 = 0\) sheńberdiń \(C\) orayın hám \(R\) radiusın tabıń. & \\
\hline
\end{tabular}

\vspace{0.7cm}

\begin{tabular}{lll}
Tuwrı juwaplar sanı: \underline{\hspace{1cm}} & 
Bahası: \underline{\hspace{1cm}} & 
Imtixan alıwshınıń qolı: \underline{\hspace{2cm}} \\
\end{tabular}

\egroup

\newpage


\textbf{85-variant}\\

\bgroup
\def\arraystretch{1.6} % 1 is the default, change whatever you need

\begin{tabular}{|m{5.7cm}|m{9.5cm}|}
\hline
Familiyası hám atı & \\
\hline
Fakulteti  & \\
\hline
Toparı hám tálim baǵdarı  & \\
\hline
\end{tabular}

\vspace{0.7cm}

\begin{tabular}{|m{0.7cm}|m{10cm}|m{4cm}|}
\hline
№ & Soraw & Juwap \\
\hline
1. & Giperbolanıń kanonikalıq teńlemesi? &  \\
\hline
2. & undefined &  \\
\hline
3. & undefined &  \\
\hline
4. & undefined &  \\
\hline
5. & \(A_{1}x + B_{1}y + C_{1}z + D_{1} = 0\) hám \(Ax + By + Cz + D = 0\) tegislikleri perpendikulyar bolıwı ushın qaysı shárt orınlı bolıwı kerek? &  \\
\hline
6. & Eger \(2 a = 16, e = \frac{5}{4}\) bolsa, fokusı abscissa kósherinde, koordinata basına salıstırǵanda simmetriyalıq jaylasqan giperbolanıń teńlemesin dúziń. &  \\
\hline
7. & \(9 x^{2} + 25 y^{2} = 225\) ellipsi berilgen, ellipstiń fokusların, ekscentrisitetin tabıń. &  \\
\hline
8. & \(M_{1} (12; - 1)\) hám \(M_{2} (0;4)\) noqatlardıń arasındaǵı aralıqtı tabıń. &  \\
\hline
9. & \(x + y - 3 = 0\) hám \(2 x + 3 y - 8 = 0\) tuwrıları óz-ara qanday jaylasqan? &  \\
\hline
10. & \(x^{2} + y^{2} - 2 x + 4 y = 0\) sheńberdiń teńlemesin kanonikalıq túrdegi teńlemege alıp keliń. & \\
\hline
\end{tabular}

\vspace{0.7cm}

\begin{tabular}{lll}
Tuwrı juwaplar sanı: \underline{\hspace{1cm}} & 
Bahası: \underline{\hspace{1cm}} & 
Imtixan alıwshınıń qolı: \underline{\hspace{2cm}} \\
\end{tabular}

\egroup

\newpage


\textbf{86-variant}\\

\bgroup
\def\arraystretch{1.6} % 1 is the default, change whatever you need

\begin{tabular}{|m{5.7cm}|m{9.5cm}|}
\hline
Familiyası hám atı & \\
\hline
Fakulteti  & \\
\hline
Toparı hám tálim baǵdarı  & \\
\hline
\end{tabular}

\vspace{0.7cm}

\begin{tabular}{|m{0.7cm}|m{10cm}|m{4cm}|}
\hline
№ & Soraw & Juwap \\
\hline
1. & \(\frac{x^{2}}{a^{2}} - \frac{y^{2}}{b^{2}} = 1\) giperbolanıń \((x_{0};y_{0})\) noqatındaǵı urınbasınıń teńlemesin kórsetiń. &  \\
\hline
2. & undefined &  \\
\hline
3. & undefined &  \\
\hline
4. & undefined &  \\
\hline
5. & \((x + 1) ^{2} + (y - 2) ^{2} + (z + 2) ^{2} = 49\) sferanıń orayınıń koordinataların tabıń. &  \\
\hline
6. & \(3 x^{2} + 10 xy + 3 y^{2} - 2 x - 14 y - 13 = 0\) teńlemesiniń tipin anıqlań. &  \\
\hline
7. & Eger \(2 b = 24, 2 c = 10\) bolsa, onda abscissa kósherinde koordinata basına salıstırǵanda simmetriyalıq jaylasqan fokuslarǵa iye, ellipstiń teńlemesin dúziń. &  \\
\hline
8. & \(\bar{a} = \left\{ 2, 1, 0 \right\}\) hám \(\bar{b} = \left\{ 1, 0,- 1 \right\}\) bolsa, \(\bar{a} - \bar{b}\) ni tabıń. &  \\
\hline
9. & \(2 x + 3 y + 4 = 0\) tuwrısına parallel hám \(M_{0} (2;1)\) noqattan ótetuǵın tuwrınıń teńlemesin dúziń. &  \\
\hline
10. & \(x + y - 12 = 0\) tuwrısı \(x^{2} + y^{2} - 2 y = 0\) sheńberge salıstırǵanda qanday jaylasqan? & \\
\hline
\end{tabular}

\vspace{0.7cm}

\begin{tabular}{lll}
Tuwrı juwaplar sanı: \underline{\hspace{1cm}} & 
Bahası: \underline{\hspace{1cm}} & 
Imtixan alıwshınıń qolı: \underline{\hspace{2cm}} \\
\end{tabular}

\egroup

\newpage


\textbf{87-variant}\\

\bgroup
\def\arraystretch{1.6} % 1 is the default, change whatever you need

\begin{tabular}{|m{5.7cm}|m{9.5cm}|}
\hline
Familiyası hám atı & \\
\hline
Fakulteti  & \\
\hline
Toparı hám tálim baǵdarı  & \\
\hline
\end{tabular}

\vspace{0.7cm}

\begin{tabular}{|m{0.7cm}|m{10cm}|m{4cm}|}
\hline
№ & Soraw & Juwap \\
\hline
1. & Vektorlardı qosıw koordinatalarda qanday formula menen anıqlanadı? &  \\
\hline
2. & undefined &  \\
\hline
3. & undefined &  \\
\hline
4. & undefined &  \\
\hline
5. & \(x + 2 = 0\) keńislik qanday geometriyalıq betlikti anıqlaydı? &  \\
\hline
6. & \(\frac{x^{2}}{a^{2}} + \frac{y^{2}}{b^{2}} = 1\) ellipstiń \((x_{0};y_{0})\) noqatındaǵı urınbasınıń teńlemesin tabıń. &  \\
\hline
7. & \(A (- 1;0;1),\ B (1; - 1;0)\) noqatları berilgen. \(\bar{BA}\) vektorın tabıń. &  \\
\hline
8. & $(2, 3)$ hám $(4, 3)$ noqatlarınan ótiwshi tuwrı sızıqtıń teńlemesin dúziń. &  \\
\hline
9. & \(A_{1}x + B_{1}y + C_{1}z + D_{1} = 0\) hám \(Ax + By + Cz + D = 0\) tegislikleri parallel bolıwı ushın qaysı shárt orınlı bolıwı kerek? &  \\
\hline
10. & \(\bar{a} = \left\{ 4,- 2,- 4 \right\}\) hám \(\bar{b} = \left\{ 6,- 3, 2 \right\}\) vektorları berilgen, \((\bar{a} - \bar{b}) ^{2}\)-? & \\
\hline
\end{tabular}

\vspace{0.7cm}

\begin{tabular}{lll}
Tuwrı juwaplar sanı: \underline{\hspace{1cm}} & 
Bahası: \underline{\hspace{1cm}} & 
Imtixan alıwshınıń qolı: \underline{\hspace{2cm}} \\
\end{tabular}

\egroup

\newpage


\textbf{88-variant}\\

\bgroup
\def\arraystretch{1.6} % 1 is the default, change whatever you need

\begin{tabular}{|m{5.7cm}|m{9.5cm}|}
\hline
Familiyası hám atı & \\
\hline
Fakulteti  & \\
\hline
Toparı hám tálim baǵdarı  & \\
\hline
\end{tabular}

\vspace{0.7cm}

\begin{tabular}{|m{0.7cm}|m{10cm}|m{4cm}|}
\hline
№ & Soraw & Juwap \\
\hline
1. & Eki vektordıń vektor kóbeymesiniń uzınlıǵın tabıw formulası? &  \\
\hline
2. & undefined &  \\
\hline
3. & undefined &  \\
\hline
4. & undefined &  \\
\hline
5. & \(2 x + 3 y - 6 = 0\) tuwrınıń teńlemesin kesindilerde berilgen teńleme túrinde kórsetiń. &  \\
\hline
6. & \(A (4, 3), B (7, 7)\) noqatları arasındaǵı aralıqtı tabıń. &  \\
\hline
7. & \(M_{1}M_{2}\) kesindiniń ortasınıń koordinatalarınıń tabıń, eger \(M_{1} (2, 3), M_{2} (4, 7)\) bolsa. &  \\
\hline
8. & \(x + y = 0\) teńlemesi menen berilgen tuwrı sızıqtıń múyeshlik koefficientin anıqlań. &  \\
\hline
9. & \(5 x - y + 7 = 0\) hám \(3 x + 2 y = 0\) tuwrıları arasındaǵı múyeshni tabıń. &  \\
\hline
10. & Koordinatalar kósherleri hám \( 3 x + 4 y - 12 = 0 \) tuwrı sızıǵı menen shegaralanǵan úshmúyeshliktiń maydanın tabıń. & \\
\hline
\end{tabular}

\vspace{0.7cm}

\begin{tabular}{lll}
Tuwrı juwaplar sanı: \underline{\hspace{1cm}} & 
Bahası: \underline{\hspace{1cm}} & 
Imtixan alıwshınıń qolı: \underline{\hspace{2cm}} \\
\end{tabular}

\egroup

\newpage


\textbf{89-variant}\\

\bgroup
\def\arraystretch{1.6} % 1 is the default, change whatever you need

\begin{tabular}{|m{5.7cm}|m{9.5cm}|}
\hline
Familiyası hám atı & \\
\hline
Fakulteti  & \\
\hline
Toparı hám tálim baǵdarı  & \\
\hline
\end{tabular}

\vspace{0.7cm}

\begin{tabular}{|m{0.7cm}|m{10cm}|m{4cm}|}
\hline
№ & Soraw & Juwap \\
\hline
1. & Tegislikdegi qálegen noqattan berilgen eki noqatqa shekemgi bolǵan aralıqlardıń ayırmasınıń modulı ózgermeytuǵın bolǵan noqatlardıń geometriyalıq ornı ne dep ataladı? &  \\
\hline
2. & undefined &  \\
\hline
3. & undefined &  \\
\hline
4. & undefined &  \\
\hline
5. & \(\left| \bar{a} \right| = 8, \left| \bar{b} \right| = 5, \alpha = 60^{0}\) bolsa, \(( \bar{a}\bar{b} )\) ni tabıń. &  \\
\hline
6. & \(x - 2 y + 1 = 0\) teńlemesi menen berilgen tuwrınıń normal túrdegi teńlemesin kórsetiń. &  \\
\hline
7. & Orayı \(C (- 1;2)\) noqatında, \(A (- 2;6 )\) noqatınan ótetuǵın sheńberdiń teńlemesin dúziń. &  \\
\hline
8. & \(A_{1}x + B_{1}y + C_{1}z + D_{1} = 0\) hám tegislikleri ústpe-úst túsiwi ushın qaysı shárt orınlı bolıwı kerek? &  \\
\hline
9. & \(\frac{x^{2}}{225} - \frac{y^{2}}{64} = - 1\) giperbola fokusınıń koordinatalarınıń tabıń. &  \\
\hline
10. & \(x^{2} - 4 y^{2} + 6 x + 5 = 0\) giperbolanıń kanonikalıq teńlemesin dúziń. & \\
\hline
\end{tabular}

\vspace{0.7cm}

\begin{tabular}{lll}
Tuwrı juwaplar sanı: \underline{\hspace{1cm}} & 
Bahası: \underline{\hspace{1cm}} & 
Imtixan alıwshınıń qolı: \underline{\hspace{2cm}} \\
\end{tabular}

\egroup

\newpage


\textbf{90-variant}\\

\bgroup
\def\arraystretch{1.6} % 1 is the default, change whatever you need

\begin{tabular}{|m{5.7cm}|m{9.5cm}|}
\hline
Familiyası hám atı & \\
\hline
Fakulteti  & \\
\hline
Toparı hám tálim baǵdarı  & \\
\hline
\end{tabular}

\vspace{0.7cm}

\begin{tabular}{|m{0.7cm}|m{10cm}|m{4cm}|}
\hline
№ & Soraw & Juwap \\
\hline
1. & Eki vektor qashan kollinear dep ataladı? &  \\
\hline
2. & undefined &  \\
\hline
3. & undefined &  \\
\hline
4. & undefined &  \\
\hline
5. & \(\overline{a} = \{5,- 6, 1 \}, \overline{b} = \{ - 4, 3, 0 \} \), \(\overline{c} = \left\{ 5,- 8, 10 \right\}\) vektorları berilgen. \(2{\bar{a}}^{2} + 4{\bar{b}}^{2} - 5{\bar{c}}^{2}\) ańlatpasınıń mánisin tabıń. &  \\
\hline
6. & \(3 x - y + 5 = 0, x + 3 y - 4 = 0\) tuwrı sızıqları arasındaǵı múyeshti tabıń. &  \\
\hline
7. & \(x^{2} + y^{2} - 2 x + 4 y - 20 = 0\) sheńberdiń \(C\) orayın hám \(R\) radiusın tabıń. &  \\
\hline
8. & \(A_{1}x + B_{1}y + C_{1}z + D_{1} = 0\) hám \(Ax + By + Cz + D = 0\) tegislikleri perpendikulyar bolıwı ushın qaysı shárt orınlı bolıwı kerek? &  \\
\hline
9. & Eger \(2 a = 16, e = \frac{5}{4}\) bolsa, fokusı abscissa kósherinde, koordinata basına salıstırǵanda simmetriyalıq jaylasqan giperbolanıń teńlemesin dúziń. &  \\
\hline
10. & \(9 x^{2} + 25 y^{2} = 225\) ellipsi berilgen, ellipstiń fokusların, ekscentrisitetin tabıń. & \\
\hline
\end{tabular}

\vspace{0.7cm}

\begin{tabular}{lll}
Tuwrı juwaplar sanı: \underline{\hspace{1cm}} & 
Bahası: \underline{\hspace{1cm}} & 
Imtixan alıwshınıń qolı: \underline{\hspace{2cm}} \\
\end{tabular}

\egroup

\newpage


\textbf{91-variant}\\

\bgroup
\def\arraystretch{1.6} % 1 is the default, change whatever you need

\begin{tabular}{|m{5.7cm}|m{9.5cm}|}
\hline
Familiyası hám atı & \\
\hline
Fakulteti  & \\
\hline
Toparı hám tálim baǵdarı  & \\
\hline
\end{tabular}

\vspace{0.7cm}

\begin{tabular}{|m{0.7cm}|m{10cm}|m{4cm}|}
\hline
№ & Soraw & Juwap \\
\hline
1. & Egerde \(a = \{ x_{1}; y_{1}; z_{1}\}, b = \{ x_{2}; y_{2}; z_{2}\}\) bolsa, vektor kóbeymeniń koordinatalarda ańlatılıwı qanday boladı? &  \\
\hline
2. & undefined &  \\
\hline
3. & undefined &  \\
\hline
4. & undefined &  \\
\hline
5. & \(M_{1} (12; - 1)\) hám \(M_{2} (0;4)\) noqatlardıń arasındaǵı aralıqtı tabıń. &  \\
\hline
6. & \(x + y - 3 = 0\) hám \(2 x + 3 y - 8 = 0\) tuwrıları óz-ara qanday jaylasqan? &  \\
\hline
7. & \(x^{2} + y^{2} - 2 x + 4 y = 0\) sheńberdiń teńlemesin kanonikalıq túrdegi teńlemege alıp keliń. &  \\
\hline
8. & \((x + 1) ^{2} + (y - 2) ^{2} + (z + 2) ^{2} = 49\) sferanıń orayınıń koordinataların tabıń. &  \\
\hline
9. & \(3 x^{2} + 10 xy + 3 y^{2} - 2 x - 14 y - 13 = 0\) teńlemesiniń tipin anıqlań. &  \\
\hline
10. & Eger \(2 b = 24, 2 c = 10\) bolsa, onda abscissa kósherinde koordinata basına salıstırǵanda simmetriyalıq jaylasqan fokuslarǵa iye, ellipstiń teńlemesin dúziń. & \\
\hline
\end{tabular}

\vspace{0.7cm}

\begin{tabular}{lll}
Tuwrı juwaplar sanı: \underline{\hspace{1cm}} & 
Bahası: \underline{\hspace{1cm}} & 
Imtixan alıwshınıń qolı: \underline{\hspace{2cm}} \\
\end{tabular}

\egroup

\newpage


\textbf{92-variant}\\

\bgroup
\def\arraystretch{1.6} % 1 is the default, change whatever you need

\begin{tabular}{|m{5.7cm}|m{9.5cm}|}
\hline
Familiyası hám atı & \\
\hline
Fakulteti  & \\
\hline
Toparı hám tálim baǵdarı  & \\
\hline
\end{tabular}

\vspace{0.7cm}

\begin{tabular}{|m{0.7cm}|m{10cm}|m{4cm}|}
\hline
№ & Soraw & Juwap \\
\hline
1. & \(Ax + C = 0\) tuwrı sızıqtıń grafigi koordinata kósherlerine salıstırǵanda qanday jaylasqan? &  \\
\hline
2. & undefined &  \\
\hline
3. & undefined &  \\
\hline
4. & undefined &  \\
\hline
5. & \(\bar{a} = \left\{ 2, 1, 0 \right\}\) hám \(\bar{b} = \left\{ 1, 0,- 1 \right\}\) bolsa, \(\bar{a} - \bar{b}\) ni tabıń. &  \\
\hline
6. & \(2 x + 3 y + 4 = 0\) tuwrısına parallel hám \(M_{0} (2;1)\) noqattan ótetuǵın tuwrınıń teńlemesin dúziń. &  \\
\hline
7. & \(x + y - 12 = 0\) tuwrısı \(x^{2} + y^{2} - 2 y = 0\) sheńberge salıstırǵanda qanday jaylasqan? &  \\
\hline
8. & \(x + 2 = 0\) keńislik qanday geometriyalıq betlikti anıqlaydı? &  \\
\hline
9. & \(\frac{x^{2}}{a^{2}} + \frac{y^{2}}{b^{2}} = 1\) ellipstiń \((x_{0};y_{0})\) noqatındaǵı urınbasınıń teńlemesin tabıń. &  \\
\hline
10. & \(A (- 1;0;1),\ B (1; - 1;0)\) noqatları berilgen. \(\bar{BA}\) vektorın tabıń. & \\
\hline
\end{tabular}

\vspace{0.7cm}

\begin{tabular}{lll}
Tuwrı juwaplar sanı: \underline{\hspace{1cm}} & 
Bahası: \underline{\hspace{1cm}} & 
Imtixan alıwshınıń qolı: \underline{\hspace{2cm}} \\
\end{tabular}

\egroup

\newpage


\textbf{93-variant}\\

\bgroup
\def\arraystretch{1.6} % 1 is the default, change whatever you need

\begin{tabular}{|m{5.7cm}|m{9.5cm}|}
\hline
Familiyası hám atı & \\
\hline
Fakulteti  & \\
\hline
Toparı hám tálim baǵdarı  & \\
\hline
\end{tabular}

\vspace{0.7cm}

\begin{tabular}{|m{0.7cm}|m{10cm}|m{4cm}|}
\hline
№ & Soraw & Juwap \\
\hline
1. & Úsh vektordıń aralas kóbeymesi ushın \((abc) = 0\) teńligi orınlı bolsa ne dep ataladı? &  \\
\hline
2. & undefined &  \\
\hline
3. & undefined &  \\
\hline
4. & undefined &  \\
\hline
5. & $(2, 3)$ hám $(4, 3)$ noqatlarınan ótiwshi tuwrı sızıqtıń teńlemesin dúziń. &  \\
\hline
6. & \(A_{1}x + B_{1}y + C_{1}z + D_{1} = 0\) hám \(Ax + By + Cz + D = 0\) tegislikleri parallel bolıwı ushın qaysı shárt orınlı bolıwı kerek? &  \\
\hline
7. & \(\bar{a} = \left\{ 4,- 2,- 4 \right\}\) hám \(\bar{b} = \left\{ 6,- 3, 2 \right\}\) vektorları berilgen, \((\bar{a} - \bar{b}) ^{2}\)-? &  \\
\hline
8. & \(2 x + 3 y - 6 = 0\) tuwrınıń teńlemesin kesindilerde berilgen teńleme túrinde kórsetiń. &  \\
\hline
9. & \(A (4, 3), B (7, 7)\) noqatları arasındaǵı aralıqtı tabıń. &  \\
\hline
10. & \(M_{1}M_{2}\) kesindiniń ortasınıń koordinatalarınıń tabıń, eger \(M_{1} (2, 3), M_{2} (4, 7)\) bolsa. & \\
\hline
\end{tabular}

\vspace{0.7cm}

\begin{tabular}{lll}
Tuwrı juwaplar sanı: \underline{\hspace{1cm}} & 
Bahası: \underline{\hspace{1cm}} & 
Imtixan alıwshınıń qolı: \underline{\hspace{2cm}} \\
\end{tabular}

\egroup

\newpage


\textbf{94-variant}\\

\bgroup
\def\arraystretch{1.6} % 1 is the default, change whatever you need

\begin{tabular}{|m{5.7cm}|m{9.5cm}|}
\hline
Familiyası hám atı & \\
\hline
Fakulteti  & \\
\hline
Toparı hám tálim baǵdarı  & \\
\hline
\end{tabular}

\vspace{0.7cm}

\begin{tabular}{|m{0.7cm}|m{10cm}|m{4cm}|}
\hline
№ & Soraw & Juwap \\
\hline
1. & Tuwrı múyeshli koordinatalar sisteması dep nege aytamız? &  \\
\hline
2. & undefined &  \\
\hline
3. & undefined &  \\
\hline
4. & undefined &  \\
\hline
5. & \(x + y = 0\) teńlemesi menen berilgen tuwrı sızıqtıń múyeshlik koefficientin anıqlań. &  \\
\hline
6. & \(5 x - y + 7 = 0\) hám \(3 x + 2 y = 0\) tuwrıları arasındaǵı múyeshni tabıń. &  \\
\hline
7. & Koordinatalar kósherleri hám \( 3 x + 4 y - 12 = 0 \) tuwrı sızıǵı menen shegaralanǵan úshmúyeshliktiń maydanın tabıń. &  \\
\hline
8. & \(\left| \bar{a} \right| = 8, \left| \bar{b} \right| = 5, \alpha = 60^{0}\) bolsa, \(( \bar{a}\bar{b} )\) ni tabıń. &  \\
\hline
9. & \(x - 2 y + 1 = 0\) teńlemesi menen berilgen tuwrınıń normal túrdegi teńlemesin kórsetiń. &  \\
\hline
10. & Orayı \(C (- 1;2)\) noqatında, \(A (- 2;6 )\) noqatınan ótetuǵın sheńberdiń teńlemesin dúziń. & \\
\hline
\end{tabular}

\vspace{0.7cm}

\begin{tabular}{lll}
Tuwrı juwaplar sanı: \underline{\hspace{1cm}} & 
Bahası: \underline{\hspace{1cm}} & 
Imtixan alıwshınıń qolı: \underline{\hspace{2cm}} \\
\end{tabular}

\egroup

\newpage


\textbf{95-variant}\\

\bgroup
\def\arraystretch{1.6} % 1 is the default, change whatever you need

\begin{tabular}{|m{5.7cm}|m{9.5cm}|}
\hline
Familiyası hám atı & \\
\hline
Fakulteti  & \\
\hline
Toparı hám tálim baǵdarı  & \\
\hline
\end{tabular}

\vspace{0.7cm}

\begin{tabular}{|m{0.7cm}|m{10cm}|m{4cm}|}
\hline
№ & Soraw & Juwap \\
\hline
1. & Vektorlardıń kósherdegi proekciyasınıń formulası? &  \\
\hline
2. & undefined &  \\
\hline
3. & undefined &  \\
\hline
4. & undefined &  \\
\hline
5. & \(A_{1}x + B_{1}y + C_{1}z + D_{1} = 0\) hám tegislikleri ústpe-úst túsiwi ushın qaysı shárt orınlı bolıwı kerek? &  \\
\hline
6. & \(\frac{x^{2}}{225} - \frac{y^{2}}{64} = - 1\) giperbola fokusınıń koordinatalarınıń tabıń. &  \\
\hline
7. & \(x^{2} - 4 y^{2} + 6 x + 5 = 0\) giperbolanıń kanonikalıq teńlemesin dúziń. &  \\
\hline
8. & \(\overline{a} = \{5,- 6, 1 \}, \overline{b} = \{ - 4, 3, 0 \} \), \(\overline{c} = \left\{ 5,- 8, 10 \right\}\) vektorları berilgen. \(2{\bar{a}}^{2} + 4{\bar{b}}^{2} - 5{\bar{c}}^{2}\) ańlatpasınıń mánisin tabıń. &  \\
\hline
9. & \(3 x - y + 5 = 0, x + 3 y - 4 = 0\) tuwrı sızıqları arasındaǵı múyeshti tabıń. &  \\
\hline
10. & \(x^{2} + y^{2} - 2 x + 4 y - 20 = 0\) sheńberdiń \(C\) orayın hám \(R\) radiusın tabıń. & \\
\hline
\end{tabular}

\vspace{0.7cm}

\begin{tabular}{lll}
Tuwrı juwaplar sanı: \underline{\hspace{1cm}} & 
Bahası: \underline{\hspace{1cm}} & 
Imtixan alıwshınıń qolı: \underline{\hspace{2cm}} \\
\end{tabular}

\egroup

\newpage


\textbf{96-variant}\\

\bgroup
\def\arraystretch{1.6} % 1 is the default, change whatever you need

\begin{tabular}{|m{5.7cm}|m{9.5cm}|}
\hline
Familiyası hám atı & \\
\hline
Fakulteti  & \\
\hline
Toparı hám tálim baǵdarı  & \\
\hline
\end{tabular}

\vspace{0.7cm}

\begin{tabular}{|m{0.7cm}|m{10cm}|m{4cm}|}
\hline
№ & Soraw & Juwap \\
\hline
1. & Eki vektordıń skalyar kóbeymesiniń formulası? &  \\
\hline
2. & undefined &  \\
\hline
3. & undefined &  \\
\hline
4. & undefined &  \\
\hline
5. & \(A_{1}x + B_{1}y + C_{1}z + D_{1} = 0\) hám \(Ax + By + Cz + D = 0\) tegislikleri perpendikulyar bolıwı ushın qaysı shárt orınlı bolıwı kerek? &  \\
\hline
6. & Eger \(2 a = 16, e = \frac{5}{4}\) bolsa, fokusı abscissa kósherinde, koordinata basına salıstırǵanda simmetriyalıq jaylasqan giperbolanıń teńlemesin dúziń. &  \\
\hline
7. & \(9 x^{2} + 25 y^{2} = 225\) ellipsi berilgen, ellipstiń fokusların, ekscentrisitetin tabıń. &  \\
\hline
8. & \(M_{1} (12; - 1)\) hám \(M_{2} (0;4)\) noqatlardıń arasındaǵı aralıqtı tabıń. &  \\
\hline
9. & \(x + y - 3 = 0\) hám \(2 x + 3 y - 8 = 0\) tuwrıları óz-ara qanday jaylasqan? &  \\
\hline
10. & \(x^{2} + y^{2} - 2 x + 4 y = 0\) sheńberdiń teńlemesin kanonikalıq túrdegi teńlemege alıp keliń. & \\
\hline
\end{tabular}

\vspace{0.7cm}

\begin{tabular}{lll}
Tuwrı juwaplar sanı: \underline{\hspace{1cm}} & 
Bahası: \underline{\hspace{1cm}} & 
Imtixan alıwshınıń qolı: \underline{\hspace{2cm}} \\
\end{tabular}

\egroup

\newpage


\textbf{97-variant}\\

\bgroup
\def\arraystretch{1.6} % 1 is the default, change whatever you need

\begin{tabular}{|m{5.7cm}|m{9.5cm}|}
\hline
Familiyası hám atı & \\
\hline
Fakulteti  & \\
\hline
Toparı hám tálim baǵdarı  & \\
\hline
\end{tabular}

\vspace{0.7cm}

\begin{tabular}{|m{0.7cm}|m{10cm}|m{4cm}|}
\hline
№ & Soraw & Juwap \\
\hline
1. & \(OXY\) tegisliginiń teńlemesi? &  \\
\hline
2. & undefined &  \\
\hline
3. & undefined &  \\
\hline
4. & undefined &  \\
\hline
5. & \((x + 1) ^{2} + (y - 2) ^{2} + (z + 2) ^{2} = 49\) sferanıń orayınıń koordinataların tabıń. &  \\
\hline
6. & \(3 x^{2} + 10 xy + 3 y^{2} - 2 x - 14 y - 13 = 0\) teńlemesiniń tipin anıqlań. &  \\
\hline
7. & Eger \(2 b = 24, 2 c = 10\) bolsa, onda abscissa kósherinde koordinata basına salıstırǵanda simmetriyalıq jaylasqan fokuslarǵa iye, ellipstiń teńlemesin dúziń. &  \\
\hline
8. & \(\bar{a} = \left\{ 2, 1, 0 \right\}\) hám \(\bar{b} = \left\{ 1, 0,- 1 \right\}\) bolsa, \(\bar{a} - \bar{b}\) ni tabıń. &  \\
\hline
9. & \(2 x + 3 y + 4 = 0\) tuwrısına parallel hám \(M_{0} (2;1)\) noqattan ótetuǵın tuwrınıń teńlemesin dúziń. &  \\
\hline
10. & \(x + y - 12 = 0\) tuwrısı \(x^{2} + y^{2} - 2 y = 0\) sheńberge salıstırǵanda qanday jaylasqan? & \\
\hline
\end{tabular}

\vspace{0.7cm}

\begin{tabular}{lll}
Tuwrı juwaplar sanı: \underline{\hspace{1cm}} & 
Bahası: \underline{\hspace{1cm}} & 
Imtixan alıwshınıń qolı: \underline{\hspace{2cm}} \\
\end{tabular}

\egroup

\newpage


\textbf{98-variant}\\

\bgroup
\def\arraystretch{1.6} % 1 is the default, change whatever you need

\begin{tabular}{|m{5.7cm}|m{9.5cm}|}
\hline
Familiyası hám atı & \\
\hline
Fakulteti  & \\
\hline
Toparı hám tálim baǵdarı  & \\
\hline
\end{tabular}

\vspace{0.7cm}

\begin{tabular}{|m{0.7cm}|m{10cm}|m{4cm}|}
\hline
№ & Soraw & Juwap \\
\hline
1. & Eki tuwrı sızıq arasındaǵı múyeshti tabıw formulası? &  \\
\hline
2. & undefined &  \\
\hline
3. & undefined &  \\
\hline
4. & undefined &  \\
\hline
5. & \(x + 2 = 0\) keńislik qanday geometriyalıq betlikti anıqlaydı? &  \\
\hline
6. & \(\frac{x^{2}}{a^{2}} + \frac{y^{2}}{b^{2}} = 1\) ellipstiń \((x_{0};y_{0})\) noqatındaǵı urınbasınıń teńlemesin tabıń. &  \\
\hline
7. & \(A (- 1;0;1),\ B (1; - 1;0)\) noqatları berilgen. \(\bar{BA}\) vektorın tabıń. &  \\
\hline
8. & $(2, 3)$ hám $(4, 3)$ noqatlarınan ótiwshi tuwrı sızıqtıń teńlemesin dúziń. &  \\
\hline
9. & \(A_{1}x + B_{1}y + C_{1}z + D_{1} = 0\) hám \(Ax + By + Cz + D = 0\) tegislikleri parallel bolıwı ushın qaysı shárt orınlı bolıwı kerek? &  \\
\hline
10. & \(\bar{a} = \left\{ 4,- 2,- 4 \right\}\) hám \(\bar{b} = \left\{ 6,- 3, 2 \right\}\) vektorları berilgen, \((\bar{a} - \bar{b}) ^{2}\)-? & \\
\hline
\end{tabular}

\vspace{0.7cm}

\begin{tabular}{lll}
Tuwrı juwaplar sanı: \underline{\hspace{1cm}} & 
Bahası: \underline{\hspace{1cm}} & 
Imtixan alıwshınıń qolı: \underline{\hspace{2cm}} \\
\end{tabular}

\egroup

\newpage


\textbf{99-variant}\\

\bgroup
\def\arraystretch{1.6} % 1 is the default, change whatever you need

\begin{tabular}{|m{5.7cm}|m{9.5cm}|}
\hline
Familiyası hám atı & \\
\hline
Fakulteti  & \\
\hline
Toparı hám tálim baǵdarı  & \\
\hline
\end{tabular}

\vspace{0.7cm}

\begin{tabular}{|m{0.7cm}|m{10cm}|m{4cm}|}
\hline
№ & Soraw & Juwap \\
\hline
1. & \(OY\) kósheriniń teńlemesi? &  \\
\hline
2. & undefined &  \\
\hline
3. & undefined &  \\
\hline
4. & undefined &  \\
\hline
5. & \(2 x + 3 y - 6 = 0\) tuwrınıń teńlemesin kesindilerde berilgen teńleme túrinde kórsetiń. &  \\
\hline
6. & \(A (4, 3), B (7, 7)\) noqatları arasındaǵı aralıqtı tabıń. &  \\
\hline
7. & \(M_{1}M_{2}\) kesindiniń ortasınıń koordinatalarınıń tabıń, eger \(M_{1} (2, 3), M_{2} (4, 7)\) bolsa. &  \\
\hline
8. & \(x + y = 0\) teńlemesi menen berilgen tuwrı sızıqtıń múyeshlik koefficientin anıqlań. &  \\
\hline
9. & \(5 x - y + 7 = 0\) hám \(3 x + 2 y = 0\) tuwrıları arasındaǵı múyeshni tabıń. &  \\
\hline
10. & Koordinatalar kósherleri hám \( 3 x + 4 y - 12 = 0 \) tuwrı sızıǵı menen shegaralanǵan úshmúyeshliktiń maydanın tabıń. & \\
\hline
\end{tabular}

\vspace{0.7cm}

\begin{tabular}{lll}
Tuwrı juwaplar sanı: \underline{\hspace{1cm}} & 
Bahası: \underline{\hspace{1cm}} & 
Imtixan alıwshınıń qolı: \underline{\hspace{2cm}} \\
\end{tabular}

\egroup

\newpage


\textbf{100-variant}\\

\bgroup
\def\arraystretch{1.6} % 1 is the default, change whatever you need

\begin{tabular}{|m{5.7cm}|m{9.5cm}|}
\hline
Familiyası hám atı & \\
\hline
Fakulteti  & \\
\hline
Toparı hám tálim baǵdarı  & \\
\hline
\end{tabular}

\vspace{0.7cm}

\begin{tabular}{|m{0.7cm}|m{10cm}|m{4cm}|}
\hline
№ & Soraw & Juwap \\
\hline
1. & \(Ax + By + D = 0\) teńlemesi arqalı ... tegisliktiń teńlemesi berilgen? &  \\
\hline
2. & undefined &  \\
\hline
3. & undefined &  \\
\hline
4. & undefined &  \\
\hline
5. & \(\left| \bar{a} \right| = 8, \left| \bar{b} \right| = 5, \alpha = 60^{0}\) bolsa, \(( \bar{a}\bar{b} )\) ni tabıń. &  \\
\hline
6. & \(x - 2 y + 1 = 0\) teńlemesi menen berilgen tuwrınıń normal túrdegi teńlemesin kórsetiń. &  \\
\hline
7. & Orayı \(C (- 1;2)\) noqatında, \(A (- 2;6 )\) noqatınan ótetuǵın sheńberdiń teńlemesin dúziń. &  \\
\hline
8. & \(A_{1}x + B_{1}y + C_{1}z + D_{1} = 0\) hám tegislikleri ústpe-úst túsiwi ushın qaysı shárt orınlı bolıwı kerek? &  \\
\hline
9. & \(\frac{x^{2}}{225} - \frac{y^{2}}{64} = - 1\) giperbola fokusınıń koordinatalarınıń tabıń. &  \\
\hline
10. & \(x^{2} - 4 y^{2} + 6 x + 5 = 0\) giperbolanıń kanonikalıq teńlemesin dúziń. & \\
\hline
\end{tabular}

\vspace{0.7cm}

\begin{tabular}{lll}
Tuwrı juwaplar sanı: \underline{\hspace{1cm}} & 
Bahası: \underline{\hspace{1cm}} & 
Imtixan alıwshınıń qolı: \underline{\hspace{2cm}} \\
\end{tabular}

\egroup

\newpage



\end{document}
