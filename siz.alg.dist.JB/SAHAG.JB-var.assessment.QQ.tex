\documentclass{article}
\usepackage[fontsize=12pt]{fontsize}
\usepackage[utf8]{inputenc}
\usepackage[T2A]{fontenc}
\usepackage{array}
\usepackage[a4paper,
left=15mm,
top=15mm,]{geometry}
\usepackage{amsmath}
\usepackage{setspace}


\renewcommand{\baselinestretch}{1} 

\begin{document}

\pagenumbering{gobble}


\textbf{1-variant}\\

\bgroup
\def\arraystretch{1.6} % 1 is the default, change whatever you need

\begin{tabular}{|m{5.7cm}|m{9.5cm}|}
\hline
Familiyası hám atı & \\
\hline
Fakulteti  & \\
\hline
Toparı hám tálim baǵdarı  & \\
\hline
\end{tabular}

\vspace{1cm}

\begin{tabular}{|m{0.7cm}|m{10cm}|m{4cm}|}
\hline
№ & Soraw & Juwap \\
\hline
1. & Eki vektor qashan kollinear dep ataladı? &  \\
\hline
2. & Tuwrı múyeshli koordinatalar sisteması dep nege aytamız? &  \\
\hline
3. & $OXY$ tegisliginiń teńlemesi? &  \\
\hline
4. & Giperbolanıń kanonikalıq teńlemesi? &  \\
\hline
5. & $\overline{a}=\{5,-6, 1 \}, \overline{b}=\{-4, 3, 0 \} $, $\overline{c}=\left\{ 5,-8, 10 \right\}$ vektorları berilgen. $2{\overline{a}}^{2}+4{\overline{b}}^{2}-5{\overline{c}}^{2}$ ańlatpasınıń mánisin tabıń. &  \\
\hline
6. & $(2, 3)$ hám $(4, 3)$ noqatlarınan ótiwshi tuwrı sızıqtıń teńlemesin dúziń. &  \\
\hline
7. & $x^{2}+y^{2}-2x+4y=0$ sheńberdiń teńlemesin kanonikalıq túrdegi teńlemege alıp keliń. &  \\
\hline
8. & $A(4, 3), B(7, 7)$ noqatları arasındaǵı aralıqtı tabıń. &  \\
\hline
9. & $3x^{2}+10xy+3y^{2}-2x-14y-13=0$ teńlemesiniń tipin anıqlań. &  \\
\hline
10. & $x^{2}-4y^{2}+6x+5=0$ giperbolanıń kanonikalıq teńlemesin dúziń. &  \\
\hline
\end{tabular}

\vspace{1cm}

\begin{tabular}{lll}
Tuwrı juwaplar sanı: \underline{\hspace{1.5cm}} & 
Bahası: \underline{\hspace{1.5cm}} & 
Imtixan alıwshınıń qolı: \underline{\hspace{2cm}} \\
\end{tabular}

\egroup

\newpage


\textbf{2-variant}\\

\bgroup
\def\arraystretch{1.6} % 1 is the default, change whatever you need

\begin{tabular}{|m{5.7cm}|m{9.5cm}|}
\hline
Familiyası hám atı & \\
\hline
Fakulteti  & \\
\hline
Toparı hám tálim baǵdarı  & \\
\hline
\end{tabular}

\vspace{1cm}

\begin{tabular}{|m{0.7cm}|m{10cm}|m{4cm}|}
\hline
№ & Soraw & Juwap \\
\hline
1. & Eki vektordıń vektor kóbeymesiniń uzınlıǵın tabıw formulası? &  \\
\hline
2. & Tegislikdegi qálegen noqatınan berilgen eki noqatqa shekemgi bolǵan aralıqlardıń ayırmasınıń modulı ózgermeytuǵın bolǵan noqatlardıń geometriyalıq ornı ne dep ataladı? &  \\
\hline
3. & Eki tuwrı sızıq arasındaǵı múyeshti tabıw formulası? &  \\
\hline
4. & $\frac{x^2}{a^2}-\frac{y^2}{b^2}=1$ giperbolanıń $(x_0;y_0)$ noqatındaǵı urınbasınıń teńlemesin kórsetiń. &  \\
\hline
5. & $M_{1}M_{2}$ kesindiniń ortasınıń koordinatalarınıń tabıń, eger $M_{1} (2, 3), M_{2} (4, 7)$ bolsa. &  \\
\hline
6. & $x+y-3=0$ hám $2x+3y-8=0$ tuwrıları óz-ara qanday jaylasqan? &  \\
\hline
7. & $x^{2}+y^{2}-2x+4y-20=0$ sheńberdiń $C$ orayın hám $R$ radiusın tabıń. &  \\
\hline
8. & $(x+1)^{2}+(y-2) ^{2}+(z+2) ^{2}=49$ sferanıń orayınıń koordinataların tabıń. &  \\
\hline
9. & Eger $2a=16, e=\frac{5}{4}$ bolsa, fokusı abscissa kósherinde, koordinata basına salıstırǵanda simmetriyalıq jaylasqan giperbolanıń teńlemesin dúziń. &  \\
\hline
10. & Eger $2b=24, 2 c=10$ bolsa, onda abscissa kósherinde koordinata basına salıstırǵanda simmetriyalıq jaylasqan fokuslarǵa iye, ellipstiń teńlemesin dúziń. &  \\
\hline
\end{tabular}

\vspace{1cm}

\begin{tabular}{lll}
Tuwrı juwaplar sanı: \underline{\hspace{1.5cm}} & 
Bahası: \underline{\hspace{1.5cm}} & 
Imtixan alıwshınıń qolı: \underline{\hspace{2cm}} \\
\end{tabular}

\egroup

\newpage


\textbf{3-variant}\\

\bgroup
\def\arraystretch{1.6} % 1 is the default, change whatever you need

\begin{tabular}{|m{5.7cm}|m{9.5cm}|}
\hline
Familiyası hám atı & \\
\hline
Fakulteti  & \\
\hline
Toparı hám tálim baǵdarı  & \\
\hline
\end{tabular}

\vspace{1cm}

\begin{tabular}{|m{0.7cm}|m{10cm}|m{4cm}|}
\hline
№ & Soraw & Juwap \\
\hline
1. & Vektorlardıń kósherdegi proekciyasınıń formulası? &  \\
\hline
2. & $Ax+By+D=0$ teńlemesi arqalı ... tegisliktiń teńlemesi berilgen? &  \\
\hline
3. & $\frac{x^2}{a^2}+\frac{y^2}{b^2}=1$ ellipstiń $(x_0;y_0)$ noqatındaǵı urınbasınıń teńlemesin tabıń. &  \\
\hline
4. & Vektorlardı qosıw koordinatalarda qanday formula menen anıqlanadı? &  \\
\hline
5. & $M_{1} (12;-1)$ hám $M_{2} (0;4)$ noqatlardıń arasındaǵı aralıqtı tabıń. &  \\
\hline
6. & $x+y=0$ teńlemesi menen berilgen tuwrı sızıqtıń múyeshlik koefficientin anıqlań. &  \\
\hline
7. & Orayı $C (-1;2)$ noqatında, $A (-2;6 )$ noqatınan ótetuǵın sheńberdiń teńlemesin dúziń. &  \\
\hline
8. & $x+2=0$ keńislik qanday geometriyalıq betlikti anıqlaydı? &  \\
\hline
9. & $\frac{x^{2}}{225}-\frac{y^{2}}{64}=-1$ giperbola fokusınıń koordinatalarınıń tabıń. &  \\
\hline
10. & $9x^{2}+25y^{2}=225$ ellipsi berilgen, ellipstiń fokusların, ekscentrisitetin tabıń. &  \\
\hline
\end{tabular}

\vspace{1cm}

\begin{tabular}{lll}
Tuwrı juwaplar sanı: \underline{\hspace{1.5cm}} & 
Bahası: \underline{\hspace{1.5cm}} & 
Imtixan alıwshınıń qolı: \underline{\hspace{2cm}} \\
\end{tabular}

\egroup

\newpage


\textbf{4-variant}\\

\bgroup
\def\arraystretch{1.6} % 1 is the default, change whatever you need

\begin{tabular}{|m{5.7cm}|m{9.5cm}|}
\hline
Familiyası hám atı & \\
\hline
Fakulteti  & \\
\hline
Toparı hám tálim baǵdarı  & \\
\hline
\end{tabular}

\vspace{1cm}

\begin{tabular}{|m{0.7cm}|m{10cm}|m{4cm}|}
\hline
№ & Soraw & Juwap \\
\hline
1. & $OY$ kósheriniń teńlemesi? &  \\
\hline
2. & Egerde $a=\{ x_1; y_1; z_1\}, b=\{ x_2, y_2; z_2\}$ bolsa, vektor kóbeymeniń koordinatalarda ańlatılıwı qanday boladı? &  \\
\hline
3. & $A_1x+B_1y+C_1z+D_1=0$ hám $Ax_2+By_2+Cz_2+D_2=0$ tegislikleri perpendikulyar bolıwı shárti &  \\
\hline
4. & Úsh vektordıń aralas kóbeymesi ushın $(abc)=0$ teńligi orınlı bolsa ne dep ataladı? &  \\
\hline
5. & $A (-1;0;1),\ B (1;-1;0)$ noqatları berilgen. $\overline{BA}$ vektorın tabıń. &  \\
\hline
6. & $2x+3y+4=0$ tuwrısına parallel hám $M_{0} (2;1)$ noqattan ótetuǵın tuwrınıń teńlemesin dúziń. &  \\
\hline
7. & $x+y-12=0$ tuwrısı $x^{2}+y^{2}-2y=0$ sheńberge salıstırǵanda qanday jaylasqan? &  \\
\hline
8. & $\left| \overline{a} \right|=8, \left| \overline{b} \right|=5, \alpha=60^{0}$ bolsa, $( \overline{a}\overline{b} )$ ni tabıń. &  \\
\hline
9. & $2x+3y-6=0$ tuwrınıń teńlemesin kesindilerde berilgen teńleme túrinde kórsetiń. &  \\
\hline
10. & $\overline{a}=\left\{ 4,-2,-4 \right\}$ hám $\overline{b}=\left\{ 6,-3, 2 \right\}$ vektorları berilgen, $(\overline{a}-\overline{b}) ^{2}$-? &  \\
\hline
\end{tabular}

\vspace{1cm}

\begin{tabular}{lll}
Tuwrı juwaplar sanı: \underline{\hspace{1.5cm}} & 
Bahası: \underline{\hspace{1.5cm}} & 
Imtixan alıwshınıń qolı: \underline{\hspace{2cm}} \\
\end{tabular}

\egroup

\newpage


\textbf{5-variant}\\

\bgroup
\def\arraystretch{1.6} % 1 is the default, change whatever you need

\begin{tabular}{|m{5.7cm}|m{9.5cm}|}
\hline
Familiyası hám atı & \\
\hline
Fakulteti  & \\
\hline
Toparı hám tálim baǵdarı  & \\
\hline
\end{tabular}

\vspace{1cm}

\begin{tabular}{|m{0.7cm}|m{10cm}|m{4cm}|}
\hline
№ & Soraw & Juwap \\
\hline
1. & $A_1x+B_1y+C_1z+D_1=0$ hám $Ax_2+By_2+Cz_2+D_2=0$ tegislikleri ústpe-úst túsiwi shárti? &  \\
\hline
2. & Eki vektordıń skalyar kóbeymesiniń formulası? &  \\
\hline
3. & $A_1x+B_1y+C_1z+D_1=0$ hám $Ax_2+By_2+Cz_2+D_2=0$ tegislikleri parallel bolıwı shárti &  \\
\hline
4. & $Ax+C=0$ tuwrı sızıqtıń grafigi koordinata kósherlerine salıstırǵanda qanday jaylasqan? &  \\
\hline
5. & $5x-y+7=0$ hám $3x+2y=0$ tuwrıları arasındaǵı múyeshni tabıń. &  \\
\hline
6. & $\overline{a}=\left\{ 2, 1, 0 \right\}$ hám $\overline{b}=\left\{ 1, 0,-1 \right\}$ bolsa, $\overline{a}-\overline{b}$ ni tabıń. &  \\
\hline
7. & Koordinatalar kósherleri hám $ 3x+4y-12=0 $ tuwrı sızıǵı menen shegaralanǵan úshmúyeshliktiń maydanın tabıń. &  \\
\hline
8. & $x-2y+1=0$ teńlemesi menen berilgen tuwrınıń normal túrdegi teńlemesin kórsetiń. &  \\
\hline
9. & $3x-y+5=0$, $x+3y-4=0$ tuwrı sızıqları arasındaǵı múyeshti tabıń. &  \\
\hline
10. & $\overline{a}=\{5,-6, 1 \}, \overline{b}=\{-4, 3, 0 \} $, $\overline{c}=\left\{ 5,-8, 10 \right\}$ vektorları berilgen. $2{\overline{a}}^{2}+4{\overline{b}}^{2}-5{\overline{c}}^{2}$ ańlatpasınıń mánisin tabıń. &  \\
\hline
\end{tabular}

\vspace{1cm}

\begin{tabular}{lll}
Tuwrı juwaplar sanı: \underline{\hspace{1.5cm}} & 
Bahası: \underline{\hspace{1.5cm}} & 
Imtixan alıwshınıń qolı: \underline{\hspace{2cm}} \\
\end{tabular}

\egroup

\newpage


\textbf{6-variant}\\

\bgroup
\def\arraystretch{1.6} % 1 is the default, change whatever you need

\begin{tabular}{|m{5.7cm}|m{9.5cm}|}
\hline
Familiyası hám atı & \\
\hline
Fakulteti  & \\
\hline
Toparı hám tálim baǵdarı  & \\
\hline
\end{tabular}

\vspace{1cm}

\begin{tabular}{|m{0.7cm}|m{10cm}|m{4cm}|}
\hline
№ & Soraw & Juwap \\
\hline
1. & Eki vektor qashan kollinear dep ataladı? &  \\
\hline
2. & Tuwrı múyeshli koordinatalar sisteması dep nege aytamız? &  \\
\hline
3. & $OXY$ tegisliginiń teńlemesi? &  \\
\hline
4. & Giperbolanıń kanonikalıq teńlemesi? &  \\
\hline
5. & $(2, 3)$ hám $(4, 3)$ noqatlarınan ótiwshi tuwrı sızıqtıń teńlemesin dúziń. &  \\
\hline
6. & $x^{2}+y^{2}-2x+4y=0$ sheńberdiń teńlemesin kanonikalıq túrdegi teńlemege alıp keliń. &  \\
\hline
7. & $A(4, 3), B(7, 7)$ noqatları arasındaǵı aralıqtı tabıń. &  \\
\hline
8. & $3x^{2}+10xy+3y^{2}-2x-14y-13=0$ teńlemesiniń tipin anıqlań. &  \\
\hline
9. & $x^{2}-4y^{2}+6x+5=0$ giperbolanıń kanonikalıq teńlemesin dúziń. &  \\
\hline
10. & $M_{1}M_{2}$ kesindiniń ortasınıń koordinatalarınıń tabıń, eger $M_{1} (2, 3), M_{2} (4, 7)$ bolsa. &  \\
\hline
\end{tabular}

\vspace{1cm}

\begin{tabular}{lll}
Tuwrı juwaplar sanı: \underline{\hspace{1.5cm}} & 
Bahası: \underline{\hspace{1.5cm}} & 
Imtixan alıwshınıń qolı: \underline{\hspace{2cm}} \\
\end{tabular}

\egroup

\newpage


\textbf{7-variant}\\

\bgroup
\def\arraystretch{1.6} % 1 is the default, change whatever you need

\begin{tabular}{|m{5.7cm}|m{9.5cm}|}
\hline
Familiyası hám atı & \\
\hline
Fakulteti  & \\
\hline
Toparı hám tálim baǵdarı  & \\
\hline
\end{tabular}

\vspace{1cm}

\begin{tabular}{|m{0.7cm}|m{10cm}|m{4cm}|}
\hline
№ & Soraw & Juwap \\
\hline
1. & Eki vektordıń vektor kóbeymesiniń uzınlıǵın tabıw formulası? &  \\
\hline
2. & Tegislikdegi qálegen noqatınan berilgen eki noqatqa shekemgi bolǵan aralıqlardıń ayırmasınıń modulı ózgermeytuǵın bolǵan noqatlardıń geometriyalıq ornı ne dep ataladı? &  \\
\hline
3. & Eki tuwrı sızıq arasındaǵı múyeshti tabıw formulası? &  \\
\hline
4. & $\frac{x^2}{a^2}-\frac{y^2}{b^2}=1$ giperbolanıń $(x_0;y_0)$ noqatındaǵı urınbasınıń teńlemesin kórsetiń. &  \\
\hline
5. & $x+y-3=0$ hám $2x+3y-8=0$ tuwrıları óz-ara qanday jaylasqan? &  \\
\hline
6. & $x^{2}+y^{2}-2x+4y-20=0$ sheńberdiń $C$ orayın hám $R$ radiusın tabıń. &  \\
\hline
7. & $(x+1)^{2}+(y-2) ^{2}+(z+2) ^{2}=49$ sferanıń orayınıń koordinataların tabıń. &  \\
\hline
8. & Eger $2a=16, e=\frac{5}{4}$ bolsa, fokusı abscissa kósherinde, koordinata basına salıstırǵanda simmetriyalıq jaylasqan giperbolanıń teńlemesin dúziń. &  \\
\hline
9. & Eger $2b=24, 2 c=10$ bolsa, onda abscissa kósherinde koordinata basına salıstırǵanda simmetriyalıq jaylasqan fokuslarǵa iye, ellipstiń teńlemesin dúziń. &  \\
\hline
10. & $M_{1} (12;-1)$ hám $M_{2} (0;4)$ noqatlardıń arasındaǵı aralıqtı tabıń. &  \\
\hline
\end{tabular}

\vspace{1cm}

\begin{tabular}{lll}
Tuwrı juwaplar sanı: \underline{\hspace{1.5cm}} & 
Bahası: \underline{\hspace{1.5cm}} & 
Imtixan alıwshınıń qolı: \underline{\hspace{2cm}} \\
\end{tabular}

\egroup

\newpage


\textbf{8-variant}\\

\bgroup
\def\arraystretch{1.6} % 1 is the default, change whatever you need

\begin{tabular}{|m{5.7cm}|m{9.5cm}|}
\hline
Familiyası hám atı & \\
\hline
Fakulteti  & \\
\hline
Toparı hám tálim baǵdarı  & \\
\hline
\end{tabular}

\vspace{1cm}

\begin{tabular}{|m{0.7cm}|m{10cm}|m{4cm}|}
\hline
№ & Soraw & Juwap \\
\hline
1. & Vektorlardıń kósherdegi proekciyasınıń formulası? &  \\
\hline
2. & $Ax+By+D=0$ teńlemesi arqalı ... tegisliktiń teńlemesi berilgen? &  \\
\hline
3. & $\frac{x^2}{a^2}+\frac{y^2}{b^2}=1$ ellipstiń $(x_0;y_0)$ noqatındaǵı urınbasınıń teńlemesin tabıń. &  \\
\hline
4. & Vektorlardı qosıw koordinatalarda qanday formula menen anıqlanadı? &  \\
\hline
5. & $x+y=0$ teńlemesi menen berilgen tuwrı sızıqtıń múyeshlik koefficientin anıqlań. &  \\
\hline
6. & Orayı $C (-1;2)$ noqatında, $A (-2;6 )$ noqatınan ótetuǵın sheńberdiń teńlemesin dúziń. &  \\
\hline
7. & $x+2=0$ keńislik qanday geometriyalıq betlikti anıqlaydı? &  \\
\hline
8. & $\frac{x^{2}}{225}-\frac{y^{2}}{64}=-1$ giperbola fokusınıń koordinatalarınıń tabıń. &  \\
\hline
9. & $9x^{2}+25y^{2}=225$ ellipsi berilgen, ellipstiń fokusların, ekscentrisitetin tabıń. &  \\
\hline
10. & $A (-1;0;1),\ B (1;-1;0)$ noqatları berilgen. $\overline{BA}$ vektorın tabıń. &  \\
\hline
\end{tabular}

\vspace{1cm}

\begin{tabular}{lll}
Tuwrı juwaplar sanı: \underline{\hspace{1.5cm}} & 
Bahası: \underline{\hspace{1.5cm}} & 
Imtixan alıwshınıń qolı: \underline{\hspace{2cm}} \\
\end{tabular}

\egroup

\newpage


\textbf{9-variant}\\

\bgroup
\def\arraystretch{1.6} % 1 is the default, change whatever you need

\begin{tabular}{|m{5.7cm}|m{9.5cm}|}
\hline
Familiyası hám atı & \\
\hline
Fakulteti  & \\
\hline
Toparı hám tálim baǵdarı  & \\
\hline
\end{tabular}

\vspace{1cm}

\begin{tabular}{|m{0.7cm}|m{10cm}|m{4cm}|}
\hline
№ & Soraw & Juwap \\
\hline
1. & $OY$ kósheriniń teńlemesi? &  \\
\hline
2. & Egerde $a=\{ x_1; y_1; z_1\}, b=\{ x_2, y_2; z_2\}$ bolsa, vektor kóbeymeniń koordinatalarda ańlatılıwı qanday boladı? &  \\
\hline
3. & $A_1x+B_1y+C_1z+D_1=0$ hám $Ax_2+By_2+Cz_2+D_2=0$ tegislikleri perpendikulyar bolıwı shárti &  \\
\hline
4. & Úsh vektordıń aralas kóbeymesi ushın $(abc)=0$ teńligi orınlı bolsa ne dep ataladı? &  \\
\hline
5. & $2x+3y+4=0$ tuwrısına parallel hám $M_{0} (2;1)$ noqattan ótetuǵın tuwrınıń teńlemesin dúziń. &  \\
\hline
6. & $x+y-12=0$ tuwrısı $x^{2}+y^{2}-2y=0$ sheńberge salıstırǵanda qanday jaylasqan? &  \\
\hline
7. & $\left| \overline{a} \right|=8, \left| \overline{b} \right|=5, \alpha=60^{0}$ bolsa, $( \overline{a}\overline{b} )$ ni tabıń. &  \\
\hline
8. & $2x+3y-6=0$ tuwrınıń teńlemesin kesindilerde berilgen teńleme túrinde kórsetiń. &  \\
\hline
9. & $\overline{a}=\left\{ 4,-2,-4 \right\}$ hám $\overline{b}=\left\{ 6,-3, 2 \right\}$ vektorları berilgen, $(\overline{a}-\overline{b}) ^{2}$-? &  \\
\hline
10. & $5x-y+7=0$ hám $3x+2y=0$ tuwrıları arasındaǵı múyeshni tabıń. &  \\
\hline
\end{tabular}

\vspace{1cm}

\begin{tabular}{lll}
Tuwrı juwaplar sanı: \underline{\hspace{1.5cm}} & 
Bahası: \underline{\hspace{1.5cm}} & 
Imtixan alıwshınıń qolı: \underline{\hspace{2cm}} \\
\end{tabular}

\egroup

\newpage


\textbf{10-variant}\\

\bgroup
\def\arraystretch{1.6} % 1 is the default, change whatever you need

\begin{tabular}{|m{5.7cm}|m{9.5cm}|}
\hline
Familiyası hám atı & \\
\hline
Fakulteti  & \\
\hline
Toparı hám tálim baǵdarı  & \\
\hline
\end{tabular}

\vspace{1cm}

\begin{tabular}{|m{0.7cm}|m{10cm}|m{4cm}|}
\hline
№ & Soraw & Juwap \\
\hline
1. & $A_1x+B_1y+C_1z+D_1=0$ hám $Ax_2+By_2+Cz_2+D_2=0$ tegislikleri ústpe-úst túsiwi shárti? &  \\
\hline
2. & Eki vektordıń skalyar kóbeymesiniń formulası? &  \\
\hline
3. & $A_1x+B_1y+C_1z+D_1=0$ hám $Ax_2+By_2+Cz_2+D_2=0$ tegislikleri parallel bolıwı shárti &  \\
\hline
4. & $Ax+C=0$ tuwrı sızıqtıń grafigi koordinata kósherlerine salıstırǵanda qanday jaylasqan? &  \\
\hline
5. & $\overline{a}=\left\{ 2, 1, 0 \right\}$ hám $\overline{b}=\left\{ 1, 0,-1 \right\}$ bolsa, $\overline{a}-\overline{b}$ ni tabıń. &  \\
\hline
6. & Koordinatalar kósherleri hám $ 3x+4y-12=0 $ tuwrı sızıǵı menen shegaralanǵan úshmúyeshliktiń maydanın tabıń. &  \\
\hline
7. & $x-2y+1=0$ teńlemesi menen berilgen tuwrınıń normal túrdegi teńlemesin kórsetiń. &  \\
\hline
8. & $3x-y+5=0$, $x+3y-4=0$ tuwrı sızıqları arasındaǵı múyeshti tabıń. &  \\
\hline
9. & $\overline{a}=\{5,-6, 1 \}, \overline{b}=\{-4, 3, 0 \} $, $\overline{c}=\left\{ 5,-8, 10 \right\}$ vektorları berilgen. $2{\overline{a}}^{2}+4{\overline{b}}^{2}-5{\overline{c}}^{2}$ ańlatpasınıń mánisin tabıń. &  \\
\hline
10. & $(2, 3)$ hám $(4, 3)$ noqatlarınan ótiwshi tuwrı sızıqtıń teńlemesin dúziń. &  \\
\hline
\end{tabular}

\vspace{1cm}

\begin{tabular}{lll}
Tuwrı juwaplar sanı: \underline{\hspace{1.5cm}} & 
Bahası: \underline{\hspace{1.5cm}} & 
Imtixan alıwshınıń qolı: \underline{\hspace{2cm}} \\
\end{tabular}

\egroup

\newpage


\textbf{11-variant}\\

\bgroup
\def\arraystretch{1.6} % 1 is the default, change whatever you need

\begin{tabular}{|m{5.7cm}|m{9.5cm}|}
\hline
Familiyası hám atı & \\
\hline
Fakulteti  & \\
\hline
Toparı hám tálim baǵdarı  & \\
\hline
\end{tabular}

\vspace{1cm}

\begin{tabular}{|m{0.7cm}|m{10cm}|m{4cm}|}
\hline
№ & Soraw & Juwap \\
\hline
1. & Eki vektor qashan kollinear dep ataladı? &  \\
\hline
2. & Tuwrı múyeshli koordinatalar sisteması dep nege aytamız? &  \\
\hline
3. & $OXY$ tegisliginiń teńlemesi? &  \\
\hline
4. & Giperbolanıń kanonikalıq teńlemesi? &  \\
\hline
5. & $x^{2}+y^{2}-2x+4y=0$ sheńberdiń teńlemesin kanonikalıq túrdegi teńlemege alıp keliń. &  \\
\hline
6. & $A(4, 3), B(7, 7)$ noqatları arasındaǵı aralıqtı tabıń. &  \\
\hline
7. & $3x^{2}+10xy+3y^{2}-2x-14y-13=0$ teńlemesiniń tipin anıqlań. &  \\
\hline
8. & $x^{2}-4y^{2}+6x+5=0$ giperbolanıń kanonikalıq teńlemesin dúziń. &  \\
\hline
9. & $M_{1}M_{2}$ kesindiniń ortasınıń koordinatalarınıń tabıń, eger $M_{1} (2, 3), M_{2} (4, 7)$ bolsa. &  \\
\hline
10. & $x+y-3=0$ hám $2x+3y-8=0$ tuwrıları óz-ara qanday jaylasqan? &  \\
\hline
\end{tabular}

\vspace{1cm}

\begin{tabular}{lll}
Tuwrı juwaplar sanı: \underline{\hspace{1.5cm}} & 
Bahası: \underline{\hspace{1.5cm}} & 
Imtixan alıwshınıń qolı: \underline{\hspace{2cm}} \\
\end{tabular}

\egroup

\newpage


\textbf{12-variant}\\

\bgroup
\def\arraystretch{1.6} % 1 is the default, change whatever you need

\begin{tabular}{|m{5.7cm}|m{9.5cm}|}
\hline
Familiyası hám atı & \\
\hline
Fakulteti  & \\
\hline
Toparı hám tálim baǵdarı  & \\
\hline
\end{tabular}

\vspace{1cm}

\begin{tabular}{|m{0.7cm}|m{10cm}|m{4cm}|}
\hline
№ & Soraw & Juwap \\
\hline
1. & Eki vektordıń vektor kóbeymesiniń uzınlıǵın tabıw formulası? &  \\
\hline
2. & Tegislikdegi qálegen noqatınan berilgen eki noqatqa shekemgi bolǵan aralıqlardıń ayırmasınıń modulı ózgermeytuǵın bolǵan noqatlardıń geometriyalıq ornı ne dep ataladı? &  \\
\hline
3. & Eki tuwrı sızıq arasındaǵı múyeshti tabıw formulası? &  \\
\hline
4. & $\frac{x^2}{a^2}-\frac{y^2}{b^2}=1$ giperbolanıń $(x_0;y_0)$ noqatındaǵı urınbasınıń teńlemesin kórsetiń. &  \\
\hline
5. & $x^{2}+y^{2}-2x+4y-20=0$ sheńberdiń $C$ orayın hám $R$ radiusın tabıń. &  \\
\hline
6. & $(x+1)^{2}+(y-2) ^{2}+(z+2) ^{2}=49$ sferanıń orayınıń koordinataların tabıń. &  \\
\hline
7. & Eger $2a=16, e=\frac{5}{4}$ bolsa, fokusı abscissa kósherinde, koordinata basına salıstırǵanda simmetriyalıq jaylasqan giperbolanıń teńlemesin dúziń. &  \\
\hline
8. & Eger $2b=24, 2 c=10$ bolsa, onda abscissa kósherinde koordinata basına salıstırǵanda simmetriyalıq jaylasqan fokuslarǵa iye, ellipstiń teńlemesin dúziń. &  \\
\hline
9. & $M_{1} (12;-1)$ hám $M_{2} (0;4)$ noqatlardıń arasındaǵı aralıqtı tabıń. &  \\
\hline
10. & $x+y=0$ teńlemesi menen berilgen tuwrı sızıqtıń múyeshlik koefficientin anıqlań. &  \\
\hline
\end{tabular}

\vspace{1cm}

\begin{tabular}{lll}
Tuwrı juwaplar sanı: \underline{\hspace{1.5cm}} & 
Bahası: \underline{\hspace{1.5cm}} & 
Imtixan alıwshınıń qolı: \underline{\hspace{2cm}} \\
\end{tabular}

\egroup

\newpage


\textbf{13-variant}\\

\bgroup
\def\arraystretch{1.6} % 1 is the default, change whatever you need

\begin{tabular}{|m{5.7cm}|m{9.5cm}|}
\hline
Familiyası hám atı & \\
\hline
Fakulteti  & \\
\hline
Toparı hám tálim baǵdarı  & \\
\hline
\end{tabular}

\vspace{1cm}

\begin{tabular}{|m{0.7cm}|m{10cm}|m{4cm}|}
\hline
№ & Soraw & Juwap \\
\hline
1. & Vektorlardıń kósherdegi proekciyasınıń formulası? &  \\
\hline
2. & $Ax+By+D=0$ teńlemesi arqalı ... tegisliktiń teńlemesi berilgen? &  \\
\hline
3. & $\frac{x^2}{a^2}+\frac{y^2}{b^2}=1$ ellipstiń $(x_0;y_0)$ noqatındaǵı urınbasınıń teńlemesin tabıń. &  \\
\hline
4. & Vektorlardı qosıw koordinatalarda qanday formula menen anıqlanadı? &  \\
\hline
5. & Orayı $C (-1;2)$ noqatında, $A (-2;6 )$ noqatınan ótetuǵın sheńberdiń teńlemesin dúziń. &  \\
\hline
6. & $x+2=0$ keńislik qanday geometriyalıq betlikti anıqlaydı? &  \\
\hline
7. & $\frac{x^{2}}{225}-\frac{y^{2}}{64}=-1$ giperbola fokusınıń koordinatalarınıń tabıń. &  \\
\hline
8. & $9x^{2}+25y^{2}=225$ ellipsi berilgen, ellipstiń fokusların, ekscentrisitetin tabıń. &  \\
\hline
9. & $A (-1;0;1),\ B (1;-1;0)$ noqatları berilgen. $\overline{BA}$ vektorın tabıń. &  \\
\hline
10. & $2x+3y+4=0$ tuwrısına parallel hám $M_{0} (2;1)$ noqattan ótetuǵın tuwrınıń teńlemesin dúziń. &  \\
\hline
\end{tabular}

\vspace{1cm}

\begin{tabular}{lll}
Tuwrı juwaplar sanı: \underline{\hspace{1.5cm}} & 
Bahası: \underline{\hspace{1.5cm}} & 
Imtixan alıwshınıń qolı: \underline{\hspace{2cm}} \\
\end{tabular}

\egroup

\newpage


\textbf{14-variant}\\

\bgroup
\def\arraystretch{1.6} % 1 is the default, change whatever you need

\begin{tabular}{|m{5.7cm}|m{9.5cm}|}
\hline
Familiyası hám atı & \\
\hline
Fakulteti  & \\
\hline
Toparı hám tálim baǵdarı  & \\
\hline
\end{tabular}

\vspace{1cm}

\begin{tabular}{|m{0.7cm}|m{10cm}|m{4cm}|}
\hline
№ & Soraw & Juwap \\
\hline
1. & $OY$ kósheriniń teńlemesi? &  \\
\hline
2. & Egerde $a=\{ x_1; y_1; z_1\}, b=\{ x_2, y_2; z_2\}$ bolsa, vektor kóbeymeniń koordinatalarda ańlatılıwı qanday boladı? &  \\
\hline
3. & $A_1x+B_1y+C_1z+D_1=0$ hám $Ax_2+By_2+Cz_2+D_2=0$ tegislikleri perpendikulyar bolıwı shárti &  \\
\hline
4. & Úsh vektordıń aralas kóbeymesi ushın $(abc)=0$ teńligi orınlı bolsa ne dep ataladı? &  \\
\hline
5. & $x+y-12=0$ tuwrısı $x^{2}+y^{2}-2y=0$ sheńberge salıstırǵanda qanday jaylasqan? &  \\
\hline
6. & $\left| \overline{a} \right|=8, \left| \overline{b} \right|=5, \alpha=60^{0}$ bolsa, $( \overline{a}\overline{b} )$ ni tabıń. &  \\
\hline
7. & $2x+3y-6=0$ tuwrınıń teńlemesin kesindilerde berilgen teńleme túrinde kórsetiń. &  \\
\hline
8. & $\overline{a}=\left\{ 4,-2,-4 \right\}$ hám $\overline{b}=\left\{ 6,-3, 2 \right\}$ vektorları berilgen, $(\overline{a}-\overline{b}) ^{2}$-? &  \\
\hline
9. & $5x-y+7=0$ hám $3x+2y=0$ tuwrıları arasındaǵı múyeshni tabıń. &  \\
\hline
10. & $\overline{a}=\left\{ 2, 1, 0 \right\}$ hám $\overline{b}=\left\{ 1, 0,-1 \right\}$ bolsa, $\overline{a}-\overline{b}$ ni tabıń. &  \\
\hline
\end{tabular}

\vspace{1cm}

\begin{tabular}{lll}
Tuwrı juwaplar sanı: \underline{\hspace{1.5cm}} & 
Bahası: \underline{\hspace{1.5cm}} & 
Imtixan alıwshınıń qolı: \underline{\hspace{2cm}} \\
\end{tabular}

\egroup

\newpage


\textbf{15-variant}\\

\bgroup
\def\arraystretch{1.6} % 1 is the default, change whatever you need

\begin{tabular}{|m{5.7cm}|m{9.5cm}|}
\hline
Familiyası hám atı & \\
\hline
Fakulteti  & \\
\hline
Toparı hám tálim baǵdarı  & \\
\hline
\end{tabular}

\vspace{1cm}

\begin{tabular}{|m{0.7cm}|m{10cm}|m{4cm}|}
\hline
№ & Soraw & Juwap \\
\hline
1. & $A_1x+B_1y+C_1z+D_1=0$ hám $Ax_2+By_2+Cz_2+D_2=0$ tegislikleri ústpe-úst túsiwi shárti? &  \\
\hline
2. & Eki vektordıń skalyar kóbeymesiniń formulası? &  \\
\hline
3. & $A_1x+B_1y+C_1z+D_1=0$ hám $Ax_2+By_2+Cz_2+D_2=0$ tegislikleri parallel bolıwı shárti &  \\
\hline
4. & $Ax+C=0$ tuwrı sızıqtıń grafigi koordinata kósherlerine salıstırǵanda qanday jaylasqan? &  \\
\hline
5. & Koordinatalar kósherleri hám $ 3x+4y-12=0 $ tuwrı sızıǵı menen shegaralanǵan úshmúyeshliktiń maydanın tabıń. &  \\
\hline
6. & $x-2y+1=0$ teńlemesi menen berilgen tuwrınıń normal túrdegi teńlemesin kórsetiń. &  \\
\hline
7. & $3x-y+5=0$, $x+3y-4=0$ tuwrı sızıqları arasındaǵı múyeshti tabıń. &  \\
\hline
8. & $\overline{a}=\{5,-6, 1 \}, \overline{b}=\{-4, 3, 0 \} $, $\overline{c}=\left\{ 5,-8, 10 \right\}$ vektorları berilgen. $2{\overline{a}}^{2}+4{\overline{b}}^{2}-5{\overline{c}}^{2}$ ańlatpasınıń mánisin tabıń. &  \\
\hline
9. & $(2, 3)$ hám $(4, 3)$ noqatlarınan ótiwshi tuwrı sızıqtıń teńlemesin dúziń. &  \\
\hline
10. & $x^{2}+y^{2}-2x+4y=0$ sheńberdiń teńlemesin kanonikalıq túrdegi teńlemege alıp keliń. &  \\
\hline
\end{tabular}

\vspace{1cm}

\begin{tabular}{lll}
Tuwrı juwaplar sanı: \underline{\hspace{1.5cm}} & 
Bahası: \underline{\hspace{1.5cm}} & 
Imtixan alıwshınıń qolı: \underline{\hspace{2cm}} \\
\end{tabular}

\egroup

\newpage


\textbf{16-variant}\\

\bgroup
\def\arraystretch{1.6} % 1 is the default, change whatever you need

\begin{tabular}{|m{5.7cm}|m{9.5cm}|}
\hline
Familiyası hám atı & \\
\hline
Fakulteti  & \\
\hline
Toparı hám tálim baǵdarı  & \\
\hline
\end{tabular}

\vspace{1cm}

\begin{tabular}{|m{0.7cm}|m{10cm}|m{4cm}|}
\hline
№ & Soraw & Juwap \\
\hline
1. & Eki vektor qashan kollinear dep ataladı? &  \\
\hline
2. & Tuwrı múyeshli koordinatalar sisteması dep nege aytamız? &  \\
\hline
3. & $OXY$ tegisliginiń teńlemesi? &  \\
\hline
4. & Giperbolanıń kanonikalıq teńlemesi? &  \\
\hline
5. & $A(4, 3), B(7, 7)$ noqatları arasındaǵı aralıqtı tabıń. &  \\
\hline
6. & $3x^{2}+10xy+3y^{2}-2x-14y-13=0$ teńlemesiniń tipin anıqlań. &  \\
\hline
7. & $x^{2}-4y^{2}+6x+5=0$ giperbolanıń kanonikalıq teńlemesin dúziń. &  \\
\hline
8. & $M_{1}M_{2}$ kesindiniń ortasınıń koordinatalarınıń tabıń, eger $M_{1} (2, 3), M_{2} (4, 7)$ bolsa. &  \\
\hline
9. & $x+y-3=0$ hám $2x+3y-8=0$ tuwrıları óz-ara qanday jaylasqan? &  \\
\hline
10. & $x^{2}+y^{2}-2x+4y-20=0$ sheńberdiń $C$ orayın hám $R$ radiusın tabıń. &  \\
\hline
\end{tabular}

\vspace{1cm}

\begin{tabular}{lll}
Tuwrı juwaplar sanı: \underline{\hspace{1.5cm}} & 
Bahası: \underline{\hspace{1.5cm}} & 
Imtixan alıwshınıń qolı: \underline{\hspace{2cm}} \\
\end{tabular}

\egroup

\newpage


\textbf{17-variant}\\

\bgroup
\def\arraystretch{1.6} % 1 is the default, change whatever you need

\begin{tabular}{|m{5.7cm}|m{9.5cm}|}
\hline
Familiyası hám atı & \\
\hline
Fakulteti  & \\
\hline
Toparı hám tálim baǵdarı  & \\
\hline
\end{tabular}

\vspace{1cm}

\begin{tabular}{|m{0.7cm}|m{10cm}|m{4cm}|}
\hline
№ & Soraw & Juwap \\
\hline
1. & Eki vektordıń vektor kóbeymesiniń uzınlıǵın tabıw formulası? &  \\
\hline
2. & Tegislikdegi qálegen noqatınan berilgen eki noqatqa shekemgi bolǵan aralıqlardıń ayırmasınıń modulı ózgermeytuǵın bolǵan noqatlardıń geometriyalıq ornı ne dep ataladı? &  \\
\hline
3. & Eki tuwrı sızıq arasındaǵı múyeshti tabıw formulası? &  \\
\hline
4. & $\frac{x^2}{a^2}-\frac{y^2}{b^2}=1$ giperbolanıń $(x_0;y_0)$ noqatındaǵı urınbasınıń teńlemesin kórsetiń. &  \\
\hline
5. & $(x+1)^{2}+(y-2) ^{2}+(z+2) ^{2}=49$ sferanıń orayınıń koordinataların tabıń. &  \\
\hline
6. & Eger $2a=16, e=\frac{5}{4}$ bolsa, fokusı abscissa kósherinde, koordinata basına salıstırǵanda simmetriyalıq jaylasqan giperbolanıń teńlemesin dúziń. &  \\
\hline
7. & Eger $2b=24, 2 c=10$ bolsa, onda abscissa kósherinde koordinata basına salıstırǵanda simmetriyalıq jaylasqan fokuslarǵa iye, ellipstiń teńlemesin dúziń. &  \\
\hline
8. & $M_{1} (12;-1)$ hám $M_{2} (0;4)$ noqatlardıń arasındaǵı aralıqtı tabıń. &  \\
\hline
9. & $x+y=0$ teńlemesi menen berilgen tuwrı sızıqtıń múyeshlik koefficientin anıqlań. &  \\
\hline
10. & Orayı $C (-1;2)$ noqatında, $A (-2;6 )$ noqatınan ótetuǵın sheńberdiń teńlemesin dúziń. &  \\
\hline
\end{tabular}

\vspace{1cm}

\begin{tabular}{lll}
Tuwrı juwaplar sanı: \underline{\hspace{1.5cm}} & 
Bahası: \underline{\hspace{1.5cm}} & 
Imtixan alıwshınıń qolı: \underline{\hspace{2cm}} \\
\end{tabular}

\egroup

\newpage


\textbf{18-variant}\\

\bgroup
\def\arraystretch{1.6} % 1 is the default, change whatever you need

\begin{tabular}{|m{5.7cm}|m{9.5cm}|}
\hline
Familiyası hám atı & \\
\hline
Fakulteti  & \\
\hline
Toparı hám tálim baǵdarı  & \\
\hline
\end{tabular}

\vspace{1cm}

\begin{tabular}{|m{0.7cm}|m{10cm}|m{4cm}|}
\hline
№ & Soraw & Juwap \\
\hline
1. & Vektorlardıń kósherdegi proekciyasınıń formulası? &  \\
\hline
2. & $Ax+By+D=0$ teńlemesi arqalı ... tegisliktiń teńlemesi berilgen? &  \\
\hline
3. & $\frac{x^2}{a^2}+\frac{y^2}{b^2}=1$ ellipstiń $(x_0;y_0)$ noqatındaǵı urınbasınıń teńlemesin tabıń. &  \\
\hline
4. & Vektorlardı qosıw koordinatalarda qanday formula menen anıqlanadı? &  \\
\hline
5. & $x+2=0$ keńislik qanday geometriyalıq betlikti anıqlaydı? &  \\
\hline
6. & $\frac{x^{2}}{225}-\frac{y^{2}}{64}=-1$ giperbola fokusınıń koordinatalarınıń tabıń. &  \\
\hline
7. & $9x^{2}+25y^{2}=225$ ellipsi berilgen, ellipstiń fokusların, ekscentrisitetin tabıń. &  \\
\hline
8. & $A (-1;0;1),\ B (1;-1;0)$ noqatları berilgen. $\overline{BA}$ vektorın tabıń. &  \\
\hline
9. & $2x+3y+4=0$ tuwrısına parallel hám $M_{0} (2;1)$ noqattan ótetuǵın tuwrınıń teńlemesin dúziń. &  \\
\hline
10. & $x+y-12=0$ tuwrısı $x^{2}+y^{2}-2y=0$ sheńberge salıstırǵanda qanday jaylasqan? &  \\
\hline
\end{tabular}

\vspace{1cm}

\begin{tabular}{lll}
Tuwrı juwaplar sanı: \underline{\hspace{1.5cm}} & 
Bahası: \underline{\hspace{1.5cm}} & 
Imtixan alıwshınıń qolı: \underline{\hspace{2cm}} \\
\end{tabular}

\egroup

\newpage


\textbf{19-variant}\\

\bgroup
\def\arraystretch{1.6} % 1 is the default, change whatever you need

\begin{tabular}{|m{5.7cm}|m{9.5cm}|}
\hline
Familiyası hám atı & \\
\hline
Fakulteti  & \\
\hline
Toparı hám tálim baǵdarı  & \\
\hline
\end{tabular}

\vspace{1cm}

\begin{tabular}{|m{0.7cm}|m{10cm}|m{4cm}|}
\hline
№ & Soraw & Juwap \\
\hline
1. & $OY$ kósheriniń teńlemesi? &  \\
\hline
2. & Egerde $a=\{ x_1; y_1; z_1\}, b=\{ x_2, y_2; z_2\}$ bolsa, vektor kóbeymeniń koordinatalarda ańlatılıwı qanday boladı? &  \\
\hline
3. & $A_1x+B_1y+C_1z+D_1=0$ hám $Ax_2+By_2+Cz_2+D_2=0$ tegislikleri perpendikulyar bolıwı shárti &  \\
\hline
4. & Úsh vektordıń aralas kóbeymesi ushın $(abc)=0$ teńligi orınlı bolsa ne dep ataladı? &  \\
\hline
5. & $\left| \overline{a} \right|=8, \left| \overline{b} \right|=5, \alpha=60^{0}$ bolsa, $( \overline{a}\overline{b} )$ ni tabıń. &  \\
\hline
6. & $2x+3y-6=0$ tuwrınıń teńlemesin kesindilerde berilgen teńleme túrinde kórsetiń. &  \\
\hline
7. & $\overline{a}=\left\{ 4,-2,-4 \right\}$ hám $\overline{b}=\left\{ 6,-3, 2 \right\}$ vektorları berilgen, $(\overline{a}-\overline{b}) ^{2}$-? &  \\
\hline
8. & $5x-y+7=0$ hám $3x+2y=0$ tuwrıları arasındaǵı múyeshni tabıń. &  \\
\hline
9. & $\overline{a}=\left\{ 2, 1, 0 \right\}$ hám $\overline{b}=\left\{ 1, 0,-1 \right\}$ bolsa, $\overline{a}-\overline{b}$ ni tabıń. &  \\
\hline
10. & Koordinatalar kósherleri hám $ 3x+4y-12=0 $ tuwrı sızıǵı menen shegaralanǵan úshmúyeshliktiń maydanın tabıń. &  \\
\hline
\end{tabular}

\vspace{1cm}

\begin{tabular}{lll}
Tuwrı juwaplar sanı: \underline{\hspace{1.5cm}} & 
Bahası: \underline{\hspace{1.5cm}} & 
Imtixan alıwshınıń qolı: \underline{\hspace{2cm}} \\
\end{tabular}

\egroup

\newpage


\textbf{20-variant}\\

\bgroup
\def\arraystretch{1.6} % 1 is the default, change whatever you need

\begin{tabular}{|m{5.7cm}|m{9.5cm}|}
\hline
Familiyası hám atı & \\
\hline
Fakulteti  & \\
\hline
Toparı hám tálim baǵdarı  & \\
\hline
\end{tabular}

\vspace{1cm}

\begin{tabular}{|m{0.7cm}|m{10cm}|m{4cm}|}
\hline
№ & Soraw & Juwap \\
\hline
1. & $A_1x+B_1y+C_1z+D_1=0$ hám $Ax_2+By_2+Cz_2+D_2=0$ tegislikleri ústpe-úst túsiwi shárti? &  \\
\hline
2. & Eki vektordıń skalyar kóbeymesiniń formulası? &  \\
\hline
3. & $A_1x+B_1y+C_1z+D_1=0$ hám $Ax_2+By_2+Cz_2+D_2=0$ tegislikleri parallel bolıwı shárti &  \\
\hline
4. & $Ax+C=0$ tuwrı sızıqtıń grafigi koordinata kósherlerine salıstırǵanda qanday jaylasqan? &  \\
\hline
5. & $x-2y+1=0$ teńlemesi menen berilgen tuwrınıń normal túrdegi teńlemesin kórsetiń. &  \\
\hline
6. & $3x-y+5=0$, $x+3y-4=0$ tuwrı sızıqları arasındaǵı múyeshti tabıń. &  \\
\hline
7. & $\overline{a}=\{5,-6, 1 \}, \overline{b}=\{-4, 3, 0 \} $, $\overline{c}=\left\{ 5,-8, 10 \right\}$ vektorları berilgen. $2{\overline{a}}^{2}+4{\overline{b}}^{2}-5{\overline{c}}^{2}$ ańlatpasınıń mánisin tabıń. &  \\
\hline
8. & $(2, 3)$ hám $(4, 3)$ noqatlarınan ótiwshi tuwrı sızıqtıń teńlemesin dúziń. &  \\
\hline
9. & $x^{2}+y^{2}-2x+4y=0$ sheńberdiń teńlemesin kanonikalıq túrdegi teńlemege alıp keliń. &  \\
\hline
10. & $A(4, 3), B(7, 7)$ noqatları arasındaǵı aralıqtı tabıń. &  \\
\hline
\end{tabular}

\vspace{1cm}

\begin{tabular}{lll}
Tuwrı juwaplar sanı: \underline{\hspace{1.5cm}} & 
Bahası: \underline{\hspace{1.5cm}} & 
Imtixan alıwshınıń qolı: \underline{\hspace{2cm}} \\
\end{tabular}

\egroup

\newpage


\textbf{21-variant}\\

\bgroup
\def\arraystretch{1.6} % 1 is the default, change whatever you need

\begin{tabular}{|m{5.7cm}|m{9.5cm}|}
\hline
Familiyası hám atı & \\
\hline
Fakulteti  & \\
\hline
Toparı hám tálim baǵdarı  & \\
\hline
\end{tabular}

\vspace{1cm}

\begin{tabular}{|m{0.7cm}|m{10cm}|m{4cm}|}
\hline
№ & Soraw & Juwap \\
\hline
1. & Eki vektor qashan kollinear dep ataladı? &  \\
\hline
2. & Tuwrı múyeshli koordinatalar sisteması dep nege aytamız? &  \\
\hline
3. & $OXY$ tegisliginiń teńlemesi? &  \\
\hline
4. & Giperbolanıń kanonikalıq teńlemesi? &  \\
\hline
5. & $3x^{2}+10xy+3y^{2}-2x-14y-13=0$ teńlemesiniń tipin anıqlań. &  \\
\hline
6. & $x^{2}-4y^{2}+6x+5=0$ giperbolanıń kanonikalıq teńlemesin dúziń. &  \\
\hline
7. & $M_{1}M_{2}$ kesindiniń ortasınıń koordinatalarınıń tabıń, eger $M_{1} (2, 3), M_{2} (4, 7)$ bolsa. &  \\
\hline
8. & $x+y-3=0$ hám $2x+3y-8=0$ tuwrıları óz-ara qanday jaylasqan? &  \\
\hline
9. & $x^{2}+y^{2}-2x+4y-20=0$ sheńberdiń $C$ orayın hám $R$ radiusın tabıń. &  \\
\hline
10. & $(x+1)^{2}+(y-2) ^{2}+(z+2) ^{2}=49$ sferanıń orayınıń koordinataların tabıń. &  \\
\hline
\end{tabular}

\vspace{1cm}

\begin{tabular}{lll}
Tuwrı juwaplar sanı: \underline{\hspace{1.5cm}} & 
Bahası: \underline{\hspace{1.5cm}} & 
Imtixan alıwshınıń qolı: \underline{\hspace{2cm}} \\
\end{tabular}

\egroup

\newpage


\textbf{22-variant}\\

\bgroup
\def\arraystretch{1.6} % 1 is the default, change whatever you need

\begin{tabular}{|m{5.7cm}|m{9.5cm}|}
\hline
Familiyası hám atı & \\
\hline
Fakulteti  & \\
\hline
Toparı hám tálim baǵdarı  & \\
\hline
\end{tabular}

\vspace{1cm}

\begin{tabular}{|m{0.7cm}|m{10cm}|m{4cm}|}
\hline
№ & Soraw & Juwap \\
\hline
1. & Eki vektordıń vektor kóbeymesiniń uzınlıǵın tabıw formulası? &  \\
\hline
2. & Tegislikdegi qálegen noqatınan berilgen eki noqatqa shekemgi bolǵan aralıqlardıń ayırmasınıń modulı ózgermeytuǵın bolǵan noqatlardıń geometriyalıq ornı ne dep ataladı? &  \\
\hline
3. & Eki tuwrı sızıq arasındaǵı múyeshti tabıw formulası? &  \\
\hline
4. & $\frac{x^2}{a^2}-\frac{y^2}{b^2}=1$ giperbolanıń $(x_0;y_0)$ noqatındaǵı urınbasınıń teńlemesin kórsetiń. &  \\
\hline
5. & Eger $2a=16, e=\frac{5}{4}$ bolsa, fokusı abscissa kósherinde, koordinata basına salıstırǵanda simmetriyalıq jaylasqan giperbolanıń teńlemesin dúziń. &  \\
\hline
6. & Eger $2b=24, 2 c=10$ bolsa, onda abscissa kósherinde koordinata basına salıstırǵanda simmetriyalıq jaylasqan fokuslarǵa iye, ellipstiń teńlemesin dúziń. &  \\
\hline
7. & $M_{1} (12;-1)$ hám $M_{2} (0;4)$ noqatlardıń arasındaǵı aralıqtı tabıń. &  \\
\hline
8. & $x+y=0$ teńlemesi menen berilgen tuwrı sızıqtıń múyeshlik koefficientin anıqlań. &  \\
\hline
9. & Orayı $C (-1;2)$ noqatında, $A (-2;6 )$ noqatınan ótetuǵın sheńberdiń teńlemesin dúziń. &  \\
\hline
10. & $x+2=0$ keńislik qanday geometriyalıq betlikti anıqlaydı? &  \\
\hline
\end{tabular}

\vspace{1cm}

\begin{tabular}{lll}
Tuwrı juwaplar sanı: \underline{\hspace{1.5cm}} & 
Bahası: \underline{\hspace{1.5cm}} & 
Imtixan alıwshınıń qolı: \underline{\hspace{2cm}} \\
\end{tabular}

\egroup

\newpage


\textbf{23-variant}\\

\bgroup
\def\arraystretch{1.6} % 1 is the default, change whatever you need

\begin{tabular}{|m{5.7cm}|m{9.5cm}|}
\hline
Familiyası hám atı & \\
\hline
Fakulteti  & \\
\hline
Toparı hám tálim baǵdarı  & \\
\hline
\end{tabular}

\vspace{1cm}

\begin{tabular}{|m{0.7cm}|m{10cm}|m{4cm}|}
\hline
№ & Soraw & Juwap \\
\hline
1. & Vektorlardıń kósherdegi proekciyasınıń formulası? &  \\
\hline
2. & $Ax+By+D=0$ teńlemesi arqalı ... tegisliktiń teńlemesi berilgen? &  \\
\hline
3. & $\frac{x^2}{a^2}+\frac{y^2}{b^2}=1$ ellipstiń $(x_0;y_0)$ noqatındaǵı urınbasınıń teńlemesin tabıń. &  \\
\hline
4. & Vektorlardı qosıw koordinatalarda qanday formula menen anıqlanadı? &  \\
\hline
5. & $\frac{x^{2}}{225}-\frac{y^{2}}{64}=-1$ giperbola fokusınıń koordinatalarınıń tabıń. &  \\
\hline
6. & $9x^{2}+25y^{2}=225$ ellipsi berilgen, ellipstiń fokusların, ekscentrisitetin tabıń. &  \\
\hline
7. & $A (-1;0;1),\ B (1;-1;0)$ noqatları berilgen. $\overline{BA}$ vektorın tabıń. &  \\
\hline
8. & $2x+3y+4=0$ tuwrısına parallel hám $M_{0} (2;1)$ noqattan ótetuǵın tuwrınıń teńlemesin dúziń. &  \\
\hline
9. & $x+y-12=0$ tuwrısı $x^{2}+y^{2}-2y=0$ sheńberge salıstırǵanda qanday jaylasqan? &  \\
\hline
10. & $\left| \overline{a} \right|=8, \left| \overline{b} \right|=5, \alpha=60^{0}$ bolsa, $( \overline{a}\overline{b} )$ ni tabıń. &  \\
\hline
\end{tabular}

\vspace{1cm}

\begin{tabular}{lll}
Tuwrı juwaplar sanı: \underline{\hspace{1.5cm}} & 
Bahası: \underline{\hspace{1.5cm}} & 
Imtixan alıwshınıń qolı: \underline{\hspace{2cm}} \\
\end{tabular}

\egroup

\newpage


\textbf{24-variant}\\

\bgroup
\def\arraystretch{1.6} % 1 is the default, change whatever you need

\begin{tabular}{|m{5.7cm}|m{9.5cm}|}
\hline
Familiyası hám atı & \\
\hline
Fakulteti  & \\
\hline
Toparı hám tálim baǵdarı  & \\
\hline
\end{tabular}

\vspace{1cm}

\begin{tabular}{|m{0.7cm}|m{10cm}|m{4cm}|}
\hline
№ & Soraw & Juwap \\
\hline
1. & $OY$ kósheriniń teńlemesi? &  \\
\hline
2. & Egerde $a=\{ x_1; y_1; z_1\}, b=\{ x_2, y_2; z_2\}$ bolsa, vektor kóbeymeniń koordinatalarda ańlatılıwı qanday boladı? &  \\
\hline
3. & $A_1x+B_1y+C_1z+D_1=0$ hám $Ax_2+By_2+Cz_2+D_2=0$ tegislikleri perpendikulyar bolıwı shárti &  \\
\hline
4. & Úsh vektordıń aralas kóbeymesi ushın $(abc)=0$ teńligi orınlı bolsa ne dep ataladı? &  \\
\hline
5. & $2x+3y-6=0$ tuwrınıń teńlemesin kesindilerde berilgen teńleme túrinde kórsetiń. &  \\
\hline
6. & $\overline{a}=\left\{ 4,-2,-4 \right\}$ hám $\overline{b}=\left\{ 6,-3, 2 \right\}$ vektorları berilgen, $(\overline{a}-\overline{b}) ^{2}$-? &  \\
\hline
7. & $5x-y+7=0$ hám $3x+2y=0$ tuwrıları arasındaǵı múyeshni tabıń. &  \\
\hline
8. & $\overline{a}=\left\{ 2, 1, 0 \right\}$ hám $\overline{b}=\left\{ 1, 0,-1 \right\}$ bolsa, $\overline{a}-\overline{b}$ ni tabıń. &  \\
\hline
9. & Koordinatalar kósherleri hám $ 3x+4y-12=0 $ tuwrı sızıǵı menen shegaralanǵan úshmúyeshliktiń maydanın tabıń. &  \\
\hline
10. & $x-2y+1=0$ teńlemesi menen berilgen tuwrınıń normal túrdegi teńlemesin kórsetiń. &  \\
\hline
\end{tabular}

\vspace{1cm}

\begin{tabular}{lll}
Tuwrı juwaplar sanı: \underline{\hspace{1.5cm}} & 
Bahası: \underline{\hspace{1.5cm}} & 
Imtixan alıwshınıń qolı: \underline{\hspace{2cm}} \\
\end{tabular}

\egroup

\newpage


\textbf{25-variant}\\

\bgroup
\def\arraystretch{1.6} % 1 is the default, change whatever you need

\begin{tabular}{|m{5.7cm}|m{9.5cm}|}
\hline
Familiyası hám atı & \\
\hline
Fakulteti  & \\
\hline
Toparı hám tálim baǵdarı  & \\
\hline
\end{tabular}

\vspace{1cm}

\begin{tabular}{|m{0.7cm}|m{10cm}|m{4cm}|}
\hline
№ & Soraw & Juwap \\
\hline
1. & $A_1x+B_1y+C_1z+D_1=0$ hám $Ax_2+By_2+Cz_2+D_2=0$ tegislikleri ústpe-úst túsiwi shárti? &  \\
\hline
2. & Eki vektordıń skalyar kóbeymesiniń formulası? &  \\
\hline
3. & $A_1x+B_1y+C_1z+D_1=0$ hám $Ax_2+By_2+Cz_2+D_2=0$ tegislikleri parallel bolıwı shárti &  \\
\hline
4. & $Ax+C=0$ tuwrı sızıqtıń grafigi koordinata kósherlerine salıstırǵanda qanday jaylasqan? &  \\
\hline
5. & $3x-y+5=0$, $x+3y-4=0$ tuwrı sızıqları arasındaǵı múyeshti tabıń. &  \\
\hline
6. & $\overline{a}=\{5,-6, 1 \}, \overline{b}=\{-4, 3, 0 \} $, $\overline{c}=\left\{ 5,-8, 10 \right\}$ vektorları berilgen. $2{\overline{a}}^{2}+4{\overline{b}}^{2}-5{\overline{c}}^{2}$ ańlatpasınıń mánisin tabıń. &  \\
\hline
7. & $(2, 3)$ hám $(4, 3)$ noqatlarınan ótiwshi tuwrı sızıqtıń teńlemesin dúziń. &  \\
\hline
8. & $x^{2}+y^{2}-2x+4y=0$ sheńberdiń teńlemesin kanonikalıq túrdegi teńlemege alıp keliń. &  \\
\hline
9. & $A(4, 3), B(7, 7)$ noqatları arasındaǵı aralıqtı tabıń. &  \\
\hline
10. & $3x^{2}+10xy+3y^{2}-2x-14y-13=0$ teńlemesiniń tipin anıqlań. &  \\
\hline
\end{tabular}

\vspace{1cm}

\begin{tabular}{lll}
Tuwrı juwaplar sanı: \underline{\hspace{1.5cm}} & 
Bahası: \underline{\hspace{1.5cm}} & 
Imtixan alıwshınıń qolı: \underline{\hspace{2cm}} \\
\end{tabular}

\egroup

\newpage


\textbf{26-variant}\\

\bgroup
\def\arraystretch{1.6} % 1 is the default, change whatever you need

\begin{tabular}{|m{5.7cm}|m{9.5cm}|}
\hline
Familiyası hám atı & \\
\hline
Fakulteti  & \\
\hline
Toparı hám tálim baǵdarı  & \\
\hline
\end{tabular}

\vspace{1cm}

\begin{tabular}{|m{0.7cm}|m{10cm}|m{4cm}|}
\hline
№ & Soraw & Juwap \\
\hline
1. & Eki vektor qashan kollinear dep ataladı? &  \\
\hline
2. & Tuwrı múyeshli koordinatalar sisteması dep nege aytamız? &  \\
\hline
3. & $OXY$ tegisliginiń teńlemesi? &  \\
\hline
4. & Giperbolanıń kanonikalıq teńlemesi? &  \\
\hline
5. & $x^{2}-4y^{2}+6x+5=0$ giperbolanıń kanonikalıq teńlemesin dúziń. &  \\
\hline
6. & $M_{1}M_{2}$ kesindiniń ortasınıń koordinatalarınıń tabıń, eger $M_{1} (2, 3), M_{2} (4, 7)$ bolsa. &  \\
\hline
7. & $x+y-3=0$ hám $2x+3y-8=0$ tuwrıları óz-ara qanday jaylasqan? &  \\
\hline
8. & $x^{2}+y^{2}-2x+4y-20=0$ sheńberdiń $C$ orayın hám $R$ radiusın tabıń. &  \\
\hline
9. & $(x+1)^{2}+(y-2) ^{2}+(z+2) ^{2}=49$ sferanıń orayınıń koordinataların tabıń. &  \\
\hline
10. & Eger $2a=16, e=\frac{5}{4}$ bolsa, fokusı abscissa kósherinde, koordinata basına salıstırǵanda simmetriyalıq jaylasqan giperbolanıń teńlemesin dúziń. &  \\
\hline
\end{tabular}

\vspace{1cm}

\begin{tabular}{lll}
Tuwrı juwaplar sanı: \underline{\hspace{1.5cm}} & 
Bahası: \underline{\hspace{1.5cm}} & 
Imtixan alıwshınıń qolı: \underline{\hspace{2cm}} \\
\end{tabular}

\egroup

\newpage


\textbf{27-variant}\\

\bgroup
\def\arraystretch{1.6} % 1 is the default, change whatever you need

\begin{tabular}{|m{5.7cm}|m{9.5cm}|}
\hline
Familiyası hám atı & \\
\hline
Fakulteti  & \\
\hline
Toparı hám tálim baǵdarı  & \\
\hline
\end{tabular}

\vspace{1cm}

\begin{tabular}{|m{0.7cm}|m{10cm}|m{4cm}|}
\hline
№ & Soraw & Juwap \\
\hline
1. & Eki vektordıń vektor kóbeymesiniń uzınlıǵın tabıw formulası? &  \\
\hline
2. & Tegislikdegi qálegen noqatınan berilgen eki noqatqa shekemgi bolǵan aralıqlardıń ayırmasınıń modulı ózgermeytuǵın bolǵan noqatlardıń geometriyalıq ornı ne dep ataladı? &  \\
\hline
3. & Eki tuwrı sızıq arasındaǵı múyeshti tabıw formulası? &  \\
\hline
4. & $\frac{x^2}{a^2}-\frac{y^2}{b^2}=1$ giperbolanıń $(x_0;y_0)$ noqatındaǵı urınbasınıń teńlemesin kórsetiń. &  \\
\hline
5. & Eger $2b=24, 2 c=10$ bolsa, onda abscissa kósherinde koordinata basına salıstırǵanda simmetriyalıq jaylasqan fokuslarǵa iye, ellipstiń teńlemesin dúziń. &  \\
\hline
6. & $M_{1} (12;-1)$ hám $M_{2} (0;4)$ noqatlardıń arasındaǵı aralıqtı tabıń. &  \\
\hline
7. & $x+y=0$ teńlemesi menen berilgen tuwrı sızıqtıń múyeshlik koefficientin anıqlań. &  \\
\hline
8. & Orayı $C (-1;2)$ noqatında, $A (-2;6 )$ noqatınan ótetuǵın sheńberdiń teńlemesin dúziń. &  \\
\hline
9. & $x+2=0$ keńislik qanday geometriyalıq betlikti anıqlaydı? &  \\
\hline
10. & $\frac{x^{2}}{225}-\frac{y^{2}}{64}=-1$ giperbola fokusınıń koordinatalarınıń tabıń. &  \\
\hline
\end{tabular}

\vspace{1cm}

\begin{tabular}{lll}
Tuwrı juwaplar sanı: \underline{\hspace{1.5cm}} & 
Bahası: \underline{\hspace{1.5cm}} & 
Imtixan alıwshınıń qolı: \underline{\hspace{2cm}} \\
\end{tabular}

\egroup

\newpage


\textbf{28-variant}\\

\bgroup
\def\arraystretch{1.6} % 1 is the default, change whatever you need

\begin{tabular}{|m{5.7cm}|m{9.5cm}|}
\hline
Familiyası hám atı & \\
\hline
Fakulteti  & \\
\hline
Toparı hám tálim baǵdarı  & \\
\hline
\end{tabular}

\vspace{1cm}

\begin{tabular}{|m{0.7cm}|m{10cm}|m{4cm}|}
\hline
№ & Soraw & Juwap \\
\hline
1. & Vektorlardıń kósherdegi proekciyasınıń formulası? &  \\
\hline
2. & $Ax+By+D=0$ teńlemesi arqalı ... tegisliktiń teńlemesi berilgen? &  \\
\hline
3. & $\frac{x^2}{a^2}+\frac{y^2}{b^2}=1$ ellipstiń $(x_0;y_0)$ noqatındaǵı urınbasınıń teńlemesin tabıń. &  \\
\hline
4. & Vektorlardı qosıw koordinatalarda qanday formula menen anıqlanadı? &  \\
\hline
5. & $9x^{2}+25y^{2}=225$ ellipsi berilgen, ellipstiń fokusların, ekscentrisitetin tabıń. &  \\
\hline
6. & $A (-1;0;1),\ B (1;-1;0)$ noqatları berilgen. $\overline{BA}$ vektorın tabıń. &  \\
\hline
7. & $2x+3y+4=0$ tuwrısına parallel hám $M_{0} (2;1)$ noqattan ótetuǵın tuwrınıń teńlemesin dúziń. &  \\
\hline
8. & $x+y-12=0$ tuwrısı $x^{2}+y^{2}-2y=0$ sheńberge salıstırǵanda qanday jaylasqan? &  \\
\hline
9. & $\left| \overline{a} \right|=8, \left| \overline{b} \right|=5, \alpha=60^{0}$ bolsa, $( \overline{a}\overline{b} )$ ni tabıń. &  \\
\hline
10. & $2x+3y-6=0$ tuwrınıń teńlemesin kesindilerde berilgen teńleme túrinde kórsetiń. &  \\
\hline
\end{tabular}

\vspace{1cm}

\begin{tabular}{lll}
Tuwrı juwaplar sanı: \underline{\hspace{1.5cm}} & 
Bahası: \underline{\hspace{1.5cm}} & 
Imtixan alıwshınıń qolı: \underline{\hspace{2cm}} \\
\end{tabular}

\egroup

\newpage


\textbf{29-variant}\\

\bgroup
\def\arraystretch{1.6} % 1 is the default, change whatever you need

\begin{tabular}{|m{5.7cm}|m{9.5cm}|}
\hline
Familiyası hám atı & \\
\hline
Fakulteti  & \\
\hline
Toparı hám tálim baǵdarı  & \\
\hline
\end{tabular}

\vspace{1cm}

\begin{tabular}{|m{0.7cm}|m{10cm}|m{4cm}|}
\hline
№ & Soraw & Juwap \\
\hline
1. & $OY$ kósheriniń teńlemesi? &  \\
\hline
2. & Egerde $a=\{ x_1; y_1; z_1\}, b=\{ x_2, y_2; z_2\}$ bolsa, vektor kóbeymeniń koordinatalarda ańlatılıwı qanday boladı? &  \\
\hline
3. & $A_1x+B_1y+C_1z+D_1=0$ hám $Ax_2+By_2+Cz_2+D_2=0$ tegislikleri perpendikulyar bolıwı shárti &  \\
\hline
4. & Úsh vektordıń aralas kóbeymesi ushın $(abc)=0$ teńligi orınlı bolsa ne dep ataladı? &  \\
\hline
5. & $\overline{a}=\left\{ 4,-2,-4 \right\}$ hám $\overline{b}=\left\{ 6,-3, 2 \right\}$ vektorları berilgen, $(\overline{a}-\overline{b}) ^{2}$-? &  \\
\hline
6. & $5x-y+7=0$ hám $3x+2y=0$ tuwrıları arasındaǵı múyeshni tabıń. &  \\
\hline
7. & $\overline{a}=\left\{ 2, 1, 0 \right\}$ hám $\overline{b}=\left\{ 1, 0,-1 \right\}$ bolsa, $\overline{a}-\overline{b}$ ni tabıń. &  \\
\hline
8. & Koordinatalar kósherleri hám $ 3x+4y-12=0 $ tuwrı sızıǵı menen shegaralanǵan úshmúyeshliktiń maydanın tabıń. &  \\
\hline
9. & $x-2y+1=0$ teńlemesi menen berilgen tuwrınıń normal túrdegi teńlemesin kórsetiń. &  \\
\hline
10. & $3x-y+5=0$, $x+3y-4=0$ tuwrı sızıqları arasındaǵı múyeshti tabıń. &  \\
\hline
\end{tabular}

\vspace{1cm}

\begin{tabular}{lll}
Tuwrı juwaplar sanı: \underline{\hspace{1.5cm}} & 
Bahası: \underline{\hspace{1.5cm}} & 
Imtixan alıwshınıń qolı: \underline{\hspace{2cm}} \\
\end{tabular}

\egroup

\newpage


\textbf{30-variant}\\

\bgroup
\def\arraystretch{1.6} % 1 is the default, change whatever you need

\begin{tabular}{|m{5.7cm}|m{9.5cm}|}
\hline
Familiyası hám atı & \\
\hline
Fakulteti  & \\
\hline
Toparı hám tálim baǵdarı  & \\
\hline
\end{tabular}

\vspace{1cm}

\begin{tabular}{|m{0.7cm}|m{10cm}|m{4cm}|}
\hline
№ & Soraw & Juwap \\
\hline
1. & $A_1x+B_1y+C_1z+D_1=0$ hám $Ax_2+By_2+Cz_2+D_2=0$ tegislikleri ústpe-úst túsiwi shárti? &  \\
\hline
2. & Eki vektordıń skalyar kóbeymesiniń formulası? &  \\
\hline
3. & $A_1x+B_1y+C_1z+D_1=0$ hám $Ax_2+By_2+Cz_2+D_2=0$ tegislikleri parallel bolıwı shárti &  \\
\hline
4. & $Ax+C=0$ tuwrı sızıqtıń grafigi koordinata kósherlerine salıstırǵanda qanday jaylasqan? &  \\
\hline
5. & $\overline{a}=\{5,-6, 1 \}, \overline{b}=\{-4, 3, 0 \} $, $\overline{c}=\left\{ 5,-8, 10 \right\}$ vektorları berilgen. $2{\overline{a}}^{2}+4{\overline{b}}^{2}-5{\overline{c}}^{2}$ ańlatpasınıń mánisin tabıń. &  \\
\hline
6. & $(2, 3)$ hám $(4, 3)$ noqatlarınan ótiwshi tuwrı sızıqtıń teńlemesin dúziń. &  \\
\hline
7. & $x^{2}+y^{2}-2x+4y=0$ sheńberdiń teńlemesin kanonikalıq túrdegi teńlemege alıp keliń. &  \\
\hline
8. & $A(4, 3), B(7, 7)$ noqatları arasındaǵı aralıqtı tabıń. &  \\
\hline
9. & $3x^{2}+10xy+3y^{2}-2x-14y-13=0$ teńlemesiniń tipin anıqlań. &  \\
\hline
10. & $x^{2}-4y^{2}+6x+5=0$ giperbolanıń kanonikalıq teńlemesin dúziń. &  \\
\hline
\end{tabular}

\vspace{1cm}

\begin{tabular}{lll}
Tuwrı juwaplar sanı: \underline{\hspace{1.5cm}} & 
Bahası: \underline{\hspace{1.5cm}} & 
Imtixan alıwshınıń qolı: \underline{\hspace{2cm}} \\
\end{tabular}

\egroup

\newpage


\textbf{31-variant}\\

\bgroup
\def\arraystretch{1.6} % 1 is the default, change whatever you need

\begin{tabular}{|m{5.7cm}|m{9.5cm}|}
\hline
Familiyası hám atı & \\
\hline
Fakulteti  & \\
\hline
Toparı hám tálim baǵdarı  & \\
\hline
\end{tabular}

\vspace{1cm}

\begin{tabular}{|m{0.7cm}|m{10cm}|m{4cm}|}
\hline
№ & Soraw & Juwap \\
\hline
1. & Eki vektor qashan kollinear dep ataladı? &  \\
\hline
2. & Tuwrı múyeshli koordinatalar sisteması dep nege aytamız? &  \\
\hline
3. & $OXY$ tegisliginiń teńlemesi? &  \\
\hline
4. & Giperbolanıń kanonikalıq teńlemesi? &  \\
\hline
5. & $M_{1}M_{2}$ kesindiniń ortasınıń koordinatalarınıń tabıń, eger $M_{1} (2, 3), M_{2} (4, 7)$ bolsa. &  \\
\hline
6. & $x+y-3=0$ hám $2x+3y-8=0$ tuwrıları óz-ara qanday jaylasqan? &  \\
\hline
7. & $x^{2}+y^{2}-2x+4y-20=0$ sheńberdiń $C$ orayın hám $R$ radiusın tabıń. &  \\
\hline
8. & $(x+1)^{2}+(y-2) ^{2}+(z+2) ^{2}=49$ sferanıń orayınıń koordinataların tabıń. &  \\
\hline
9. & Eger $2a=16, e=\frac{5}{4}$ bolsa, fokusı abscissa kósherinde, koordinata basına salıstırǵanda simmetriyalıq jaylasqan giperbolanıń teńlemesin dúziń. &  \\
\hline
10. & Eger $2b=24, 2 c=10$ bolsa, onda abscissa kósherinde koordinata basına salıstırǵanda simmetriyalıq jaylasqan fokuslarǵa iye, ellipstiń teńlemesin dúziń. &  \\
\hline
\end{tabular}

\vspace{1cm}

\begin{tabular}{lll}
Tuwrı juwaplar sanı: \underline{\hspace{1.5cm}} & 
Bahası: \underline{\hspace{1.5cm}} & 
Imtixan alıwshınıń qolı: \underline{\hspace{2cm}} \\
\end{tabular}

\egroup

\newpage


\textbf{32-variant}\\

\bgroup
\def\arraystretch{1.6} % 1 is the default, change whatever you need

\begin{tabular}{|m{5.7cm}|m{9.5cm}|}
\hline
Familiyası hám atı & \\
\hline
Fakulteti  & \\
\hline
Toparı hám tálim baǵdarı  & \\
\hline
\end{tabular}

\vspace{1cm}

\begin{tabular}{|m{0.7cm}|m{10cm}|m{4cm}|}
\hline
№ & Soraw & Juwap \\
\hline
1. & Eki vektordıń vektor kóbeymesiniń uzınlıǵın tabıw formulası? &  \\
\hline
2. & Tegislikdegi qálegen noqatınan berilgen eki noqatqa shekemgi bolǵan aralıqlardıń ayırmasınıń modulı ózgermeytuǵın bolǵan noqatlardıń geometriyalıq ornı ne dep ataladı? &  \\
\hline
3. & Eki tuwrı sızıq arasındaǵı múyeshti tabıw formulası? &  \\
\hline
4. & $\frac{x^2}{a^2}-\frac{y^2}{b^2}=1$ giperbolanıń $(x_0;y_0)$ noqatındaǵı urınbasınıń teńlemesin kórsetiń. &  \\
\hline
5. & $M_{1} (12;-1)$ hám $M_{2} (0;4)$ noqatlardıń arasındaǵı aralıqtı tabıń. &  \\
\hline
6. & $x+y=0$ teńlemesi menen berilgen tuwrı sızıqtıń múyeshlik koefficientin anıqlań. &  \\
\hline
7. & Orayı $C (-1;2)$ noqatında, $A (-2;6 )$ noqatınan ótetuǵın sheńberdiń teńlemesin dúziń. &  \\
\hline
8. & $x+2=0$ keńislik qanday geometriyalıq betlikti anıqlaydı? &  \\
\hline
9. & $\frac{x^{2}}{225}-\frac{y^{2}}{64}=-1$ giperbola fokusınıń koordinatalarınıń tabıń. &  \\
\hline
10. & $9x^{2}+25y^{2}=225$ ellipsi berilgen, ellipstiń fokusların, ekscentrisitetin tabıń. &  \\
\hline
\end{tabular}

\vspace{1cm}

\begin{tabular}{lll}
Tuwrı juwaplar sanı: \underline{\hspace{1.5cm}} & 
Bahası: \underline{\hspace{1.5cm}} & 
Imtixan alıwshınıń qolı: \underline{\hspace{2cm}} \\
\end{tabular}

\egroup

\newpage


\textbf{33-variant}\\

\bgroup
\def\arraystretch{1.6} % 1 is the default, change whatever you need

\begin{tabular}{|m{5.7cm}|m{9.5cm}|}
\hline
Familiyası hám atı & \\
\hline
Fakulteti  & \\
\hline
Toparı hám tálim baǵdarı  & \\
\hline
\end{tabular}

\vspace{1cm}

\begin{tabular}{|m{0.7cm}|m{10cm}|m{4cm}|}
\hline
№ & Soraw & Juwap \\
\hline
1. & Vektorlardıń kósherdegi proekciyasınıń formulası? &  \\
\hline
2. & $Ax+By+D=0$ teńlemesi arqalı ... tegisliktiń teńlemesi berilgen? &  \\
\hline
3. & $\frac{x^2}{a^2}+\frac{y^2}{b^2}=1$ ellipstiń $(x_0;y_0)$ noqatındaǵı urınbasınıń teńlemesin tabıń. &  \\
\hline
4. & Vektorlardı qosıw koordinatalarda qanday formula menen anıqlanadı? &  \\
\hline
5. & $A (-1;0;1),\ B (1;-1;0)$ noqatları berilgen. $\overline{BA}$ vektorın tabıń. &  \\
\hline
6. & $2x+3y+4=0$ tuwrısına parallel hám $M_{0} (2;1)$ noqattan ótetuǵın tuwrınıń teńlemesin dúziń. &  \\
\hline
7. & $x+y-12=0$ tuwrısı $x^{2}+y^{2}-2y=0$ sheńberge salıstırǵanda qanday jaylasqan? &  \\
\hline
8. & $\left| \overline{a} \right|=8, \left| \overline{b} \right|=5, \alpha=60^{0}$ bolsa, $( \overline{a}\overline{b} )$ ni tabıń. &  \\
\hline
9. & $2x+3y-6=0$ tuwrınıń teńlemesin kesindilerde berilgen teńleme túrinde kórsetiń. &  \\
\hline
10. & $\overline{a}=\left\{ 4,-2,-4 \right\}$ hám $\overline{b}=\left\{ 6,-3, 2 \right\}$ vektorları berilgen, $(\overline{a}-\overline{b}) ^{2}$-? &  \\
\hline
\end{tabular}

\vspace{1cm}

\begin{tabular}{lll}
Tuwrı juwaplar sanı: \underline{\hspace{1.5cm}} & 
Bahası: \underline{\hspace{1.5cm}} & 
Imtixan alıwshınıń qolı: \underline{\hspace{2cm}} \\
\end{tabular}

\egroup

\newpage


\textbf{34-variant}\\

\bgroup
\def\arraystretch{1.6} % 1 is the default, change whatever you need

\begin{tabular}{|m{5.7cm}|m{9.5cm}|}
\hline
Familiyası hám atı & \\
\hline
Fakulteti  & \\
\hline
Toparı hám tálim baǵdarı  & \\
\hline
\end{tabular}

\vspace{1cm}

\begin{tabular}{|m{0.7cm}|m{10cm}|m{4cm}|}
\hline
№ & Soraw & Juwap \\
\hline
1. & $OY$ kósheriniń teńlemesi? &  \\
\hline
2. & Egerde $a=\{ x_1; y_1; z_1\}, b=\{ x_2, y_2; z_2\}$ bolsa, vektor kóbeymeniń koordinatalarda ańlatılıwı qanday boladı? &  \\
\hline
3. & $A_1x+B_1y+C_1z+D_1=0$ hám $Ax_2+By_2+Cz_2+D_2=0$ tegislikleri perpendikulyar bolıwı shárti &  \\
\hline
4. & Úsh vektordıń aralas kóbeymesi ushın $(abc)=0$ teńligi orınlı bolsa ne dep ataladı? &  \\
\hline
5. & $5x-y+7=0$ hám $3x+2y=0$ tuwrıları arasındaǵı múyeshni tabıń. &  \\
\hline
6. & $\overline{a}=\left\{ 2, 1, 0 \right\}$ hám $\overline{b}=\left\{ 1, 0,-1 \right\}$ bolsa, $\overline{a}-\overline{b}$ ni tabıń. &  \\
\hline
7. & Koordinatalar kósherleri hám $ 3x+4y-12=0 $ tuwrı sızıǵı menen shegaralanǵan úshmúyeshliktiń maydanın tabıń. &  \\
\hline
8. & $x-2y+1=0$ teńlemesi menen berilgen tuwrınıń normal túrdegi teńlemesin kórsetiń. &  \\
\hline
9. & $3x-y+5=0$, $x+3y-4=0$ tuwrı sızıqları arasındaǵı múyeshti tabıń. &  \\
\hline
10. & $\overline{a}=\{5,-6, 1 \}, \overline{b}=\{-4, 3, 0 \} $, $\overline{c}=\left\{ 5,-8, 10 \right\}$ vektorları berilgen. $2{\overline{a}}^{2}+4{\overline{b}}^{2}-5{\overline{c}}^{2}$ ańlatpasınıń mánisin tabıń. &  \\
\hline
\end{tabular}

\vspace{1cm}

\begin{tabular}{lll}
Tuwrı juwaplar sanı: \underline{\hspace{1.5cm}} & 
Bahası: \underline{\hspace{1.5cm}} & 
Imtixan alıwshınıń qolı: \underline{\hspace{2cm}} \\
\end{tabular}

\egroup

\newpage


\textbf{35-variant}\\

\bgroup
\def\arraystretch{1.6} % 1 is the default, change whatever you need

\begin{tabular}{|m{5.7cm}|m{9.5cm}|}
\hline
Familiyası hám atı & \\
\hline
Fakulteti  & \\
\hline
Toparı hám tálim baǵdarı  & \\
\hline
\end{tabular}

\vspace{1cm}

\begin{tabular}{|m{0.7cm}|m{10cm}|m{4cm}|}
\hline
№ & Soraw & Juwap \\
\hline
1. & $A_1x+B_1y+C_1z+D_1=0$ hám $Ax_2+By_2+Cz_2+D_2=0$ tegislikleri ústpe-úst túsiwi shárti? &  \\
\hline
2. & Eki vektordıń skalyar kóbeymesiniń formulası? &  \\
\hline
3. & $A_1x+B_1y+C_1z+D_1=0$ hám $Ax_2+By_2+Cz_2+D_2=0$ tegislikleri parallel bolıwı shárti &  \\
\hline
4. & $Ax+C=0$ tuwrı sızıqtıń grafigi koordinata kósherlerine salıstırǵanda qanday jaylasqan? &  \\
\hline
5. & $(2, 3)$ hám $(4, 3)$ noqatlarınan ótiwshi tuwrı sızıqtıń teńlemesin dúziń. &  \\
\hline
6. & $x^{2}+y^{2}-2x+4y=0$ sheńberdiń teńlemesin kanonikalıq túrdegi teńlemege alıp keliń. &  \\
\hline
7. & $A(4, 3), B(7, 7)$ noqatları arasındaǵı aralıqtı tabıń. &  \\
\hline
8. & $3x^{2}+10xy+3y^{2}-2x-14y-13=0$ teńlemesiniń tipin anıqlań. &  \\
\hline
9. & $x^{2}-4y^{2}+6x+5=0$ giperbolanıń kanonikalıq teńlemesin dúziń. &  \\
\hline
10. & $M_{1}M_{2}$ kesindiniń ortasınıń koordinatalarınıń tabıń, eger $M_{1} (2, 3), M_{2} (4, 7)$ bolsa. &  \\
\hline
\end{tabular}

\vspace{1cm}

\begin{tabular}{lll}
Tuwrı juwaplar sanı: \underline{\hspace{1.5cm}} & 
Bahası: \underline{\hspace{1.5cm}} & 
Imtixan alıwshınıń qolı: \underline{\hspace{2cm}} \\
\end{tabular}

\egroup

\newpage


\textbf{36-variant}\\

\bgroup
\def\arraystretch{1.6} % 1 is the default, change whatever you need

\begin{tabular}{|m{5.7cm}|m{9.5cm}|}
\hline
Familiyası hám atı & \\
\hline
Fakulteti  & \\
\hline
Toparı hám tálim baǵdarı  & \\
\hline
\end{tabular}

\vspace{1cm}

\begin{tabular}{|m{0.7cm}|m{10cm}|m{4cm}|}
\hline
№ & Soraw & Juwap \\
\hline
1. & Eki vektor qashan kollinear dep ataladı? &  \\
\hline
2. & Tuwrı múyeshli koordinatalar sisteması dep nege aytamız? &  \\
\hline
3. & $OXY$ tegisliginiń teńlemesi? &  \\
\hline
4. & Giperbolanıń kanonikalıq teńlemesi? &  \\
\hline
5. & $x+y-3=0$ hám $2x+3y-8=0$ tuwrıları óz-ara qanday jaylasqan? &  \\
\hline
6. & $x^{2}+y^{2}-2x+4y-20=0$ sheńberdiń $C$ orayın hám $R$ radiusın tabıń. &  \\
\hline
7. & $(x+1)^{2}+(y-2) ^{2}+(z+2) ^{2}=49$ sferanıń orayınıń koordinataların tabıń. &  \\
\hline
8. & Eger $2a=16, e=\frac{5}{4}$ bolsa, fokusı abscissa kósherinde, koordinata basına salıstırǵanda simmetriyalıq jaylasqan giperbolanıń teńlemesin dúziń. &  \\
\hline
9. & Eger $2b=24, 2 c=10$ bolsa, onda abscissa kósherinde koordinata basına salıstırǵanda simmetriyalıq jaylasqan fokuslarǵa iye, ellipstiń teńlemesin dúziń. &  \\
\hline
10. & $M_{1} (12;-1)$ hám $M_{2} (0;4)$ noqatlardıń arasındaǵı aralıqtı tabıń. &  \\
\hline
\end{tabular}

\vspace{1cm}

\begin{tabular}{lll}
Tuwrı juwaplar sanı: \underline{\hspace{1.5cm}} & 
Bahası: \underline{\hspace{1.5cm}} & 
Imtixan alıwshınıń qolı: \underline{\hspace{2cm}} \\
\end{tabular}

\egroup

\newpage


\textbf{37-variant}\\

\bgroup
\def\arraystretch{1.6} % 1 is the default, change whatever you need

\begin{tabular}{|m{5.7cm}|m{9.5cm}|}
\hline
Familiyası hám atı & \\
\hline
Fakulteti  & \\
\hline
Toparı hám tálim baǵdarı  & \\
\hline
\end{tabular}

\vspace{1cm}

\begin{tabular}{|m{0.7cm}|m{10cm}|m{4cm}|}
\hline
№ & Soraw & Juwap \\
\hline
1. & Eki vektordıń vektor kóbeymesiniń uzınlıǵın tabıw formulası? &  \\
\hline
2. & Tegislikdegi qálegen noqatınan berilgen eki noqatqa shekemgi bolǵan aralıqlardıń ayırmasınıń modulı ózgermeytuǵın bolǵan noqatlardıń geometriyalıq ornı ne dep ataladı? &  \\
\hline
3. & Eki tuwrı sızıq arasındaǵı múyeshti tabıw formulası? &  \\
\hline
4. & $\frac{x^2}{a^2}-\frac{y^2}{b^2}=1$ giperbolanıń $(x_0;y_0)$ noqatındaǵı urınbasınıń teńlemesin kórsetiń. &  \\
\hline
5. & $x+y=0$ teńlemesi menen berilgen tuwrı sızıqtıń múyeshlik koefficientin anıqlań. &  \\
\hline
6. & Orayı $C (-1;2)$ noqatında, $A (-2;6 )$ noqatınan ótetuǵın sheńberdiń teńlemesin dúziń. &  \\
\hline
7. & $x+2=0$ keńislik qanday geometriyalıq betlikti anıqlaydı? &  \\
\hline
8. & $\frac{x^{2}}{225}-\frac{y^{2}}{64}=-1$ giperbola fokusınıń koordinatalarınıń tabıń. &  \\
\hline
9. & $9x^{2}+25y^{2}=225$ ellipsi berilgen, ellipstiń fokusların, ekscentrisitetin tabıń. &  \\
\hline
10. & $A (-1;0;1),\ B (1;-1;0)$ noqatları berilgen. $\overline{BA}$ vektorın tabıń. &  \\
\hline
\end{tabular}

\vspace{1cm}

\begin{tabular}{lll}
Tuwrı juwaplar sanı: \underline{\hspace{1.5cm}} & 
Bahası: \underline{\hspace{1.5cm}} & 
Imtixan alıwshınıń qolı: \underline{\hspace{2cm}} \\
\end{tabular}

\egroup

\newpage


\textbf{38-variant}\\

\bgroup
\def\arraystretch{1.6} % 1 is the default, change whatever you need

\begin{tabular}{|m{5.7cm}|m{9.5cm}|}
\hline
Familiyası hám atı & \\
\hline
Fakulteti  & \\
\hline
Toparı hám tálim baǵdarı  & \\
\hline
\end{tabular}

\vspace{1cm}

\begin{tabular}{|m{0.7cm}|m{10cm}|m{4cm}|}
\hline
№ & Soraw & Juwap \\
\hline
1. & Vektorlardıń kósherdegi proekciyasınıń formulası? &  \\
\hline
2. & $Ax+By+D=0$ teńlemesi arqalı ... tegisliktiń teńlemesi berilgen? &  \\
\hline
3. & $\frac{x^2}{a^2}+\frac{y^2}{b^2}=1$ ellipstiń $(x_0;y_0)$ noqatındaǵı urınbasınıń teńlemesin tabıń. &  \\
\hline
4. & Vektorlardı qosıw koordinatalarda qanday formula menen anıqlanadı? &  \\
\hline
5. & $2x+3y+4=0$ tuwrısına parallel hám $M_{0} (2;1)$ noqattan ótetuǵın tuwrınıń teńlemesin dúziń. &  \\
\hline
6. & $x+y-12=0$ tuwrısı $x^{2}+y^{2}-2y=0$ sheńberge salıstırǵanda qanday jaylasqan? &  \\
\hline
7. & $\left| \overline{a} \right|=8, \left| \overline{b} \right|=5, \alpha=60^{0}$ bolsa, $( \overline{a}\overline{b} )$ ni tabıń. &  \\
\hline
8. & $2x+3y-6=0$ tuwrınıń teńlemesin kesindilerde berilgen teńleme túrinde kórsetiń. &  \\
\hline
9. & $\overline{a}=\left\{ 4,-2,-4 \right\}$ hám $\overline{b}=\left\{ 6,-3, 2 \right\}$ vektorları berilgen, $(\overline{a}-\overline{b}) ^{2}$-? &  \\
\hline
10. & $5x-y+7=0$ hám $3x+2y=0$ tuwrıları arasındaǵı múyeshni tabıń. &  \\
\hline
\end{tabular}

\vspace{1cm}

\begin{tabular}{lll}
Tuwrı juwaplar sanı: \underline{\hspace{1.5cm}} & 
Bahası: \underline{\hspace{1.5cm}} & 
Imtixan alıwshınıń qolı: \underline{\hspace{2cm}} \\
\end{tabular}

\egroup

\newpage


\textbf{39-variant}\\

\bgroup
\def\arraystretch{1.6} % 1 is the default, change whatever you need

\begin{tabular}{|m{5.7cm}|m{9.5cm}|}
\hline
Familiyası hám atı & \\
\hline
Fakulteti  & \\
\hline
Toparı hám tálim baǵdarı  & \\
\hline
\end{tabular}

\vspace{1cm}

\begin{tabular}{|m{0.7cm}|m{10cm}|m{4cm}|}
\hline
№ & Soraw & Juwap \\
\hline
1. & $OY$ kósheriniń teńlemesi? &  \\
\hline
2. & Egerde $a=\{ x_1; y_1; z_1\}, b=\{ x_2, y_2; z_2\}$ bolsa, vektor kóbeymeniń koordinatalarda ańlatılıwı qanday boladı? &  \\
\hline
3. & $A_1x+B_1y+C_1z+D_1=0$ hám $Ax_2+By_2+Cz_2+D_2=0$ tegislikleri perpendikulyar bolıwı shárti &  \\
\hline
4. & Úsh vektordıń aralas kóbeymesi ushın $(abc)=0$ teńligi orınlı bolsa ne dep ataladı? &  \\
\hline
5. & $\overline{a}=\left\{ 2, 1, 0 \right\}$ hám $\overline{b}=\left\{ 1, 0,-1 \right\}$ bolsa, $\overline{a}-\overline{b}$ ni tabıń. &  \\
\hline
6. & Koordinatalar kósherleri hám $ 3x+4y-12=0 $ tuwrı sızıǵı menen shegaralanǵan úshmúyeshliktiń maydanın tabıń. &  \\
\hline
7. & $x-2y+1=0$ teńlemesi menen berilgen tuwrınıń normal túrdegi teńlemesin kórsetiń. &  \\
\hline
8. & $3x-y+5=0$, $x+3y-4=0$ tuwrı sızıqları arasındaǵı múyeshti tabıń. &  \\
\hline
9. & $\overline{a}=\{5,-6, 1 \}, \overline{b}=\{-4, 3, 0 \} $, $\overline{c}=\left\{ 5,-8, 10 \right\}$ vektorları berilgen. $2{\overline{a}}^{2}+4{\overline{b}}^{2}-5{\overline{c}}^{2}$ ańlatpasınıń mánisin tabıń. &  \\
\hline
10. & $(2, 3)$ hám $(4, 3)$ noqatlarınan ótiwshi tuwrı sızıqtıń teńlemesin dúziń. &  \\
\hline
\end{tabular}

\vspace{1cm}

\begin{tabular}{lll}
Tuwrı juwaplar sanı: \underline{\hspace{1.5cm}} & 
Bahası: \underline{\hspace{1.5cm}} & 
Imtixan alıwshınıń qolı: \underline{\hspace{2cm}} \\
\end{tabular}

\egroup

\newpage


\textbf{40-variant}\\

\bgroup
\def\arraystretch{1.6} % 1 is the default, change whatever you need

\begin{tabular}{|m{5.7cm}|m{9.5cm}|}
\hline
Familiyası hám atı & \\
\hline
Fakulteti  & \\
\hline
Toparı hám tálim baǵdarı  & \\
\hline
\end{tabular}

\vspace{1cm}

\begin{tabular}{|m{0.7cm}|m{10cm}|m{4cm}|}
\hline
№ & Soraw & Juwap \\
\hline
1. & $A_1x+B_1y+C_1z+D_1=0$ hám $Ax_2+By_2+Cz_2+D_2=0$ tegislikleri ústpe-úst túsiwi shárti? &  \\
\hline
2. & Eki vektordıń skalyar kóbeymesiniń formulası? &  \\
\hline
3. & $A_1x+B_1y+C_1z+D_1=0$ hám $Ax_2+By_2+Cz_2+D_2=0$ tegislikleri parallel bolıwı shárti &  \\
\hline
4. & $Ax+C=0$ tuwrı sızıqtıń grafigi koordinata kósherlerine salıstırǵanda qanday jaylasqan? &  \\
\hline
5. & $x^{2}+y^{2}-2x+4y=0$ sheńberdiń teńlemesin kanonikalıq túrdegi teńlemege alıp keliń. &  \\
\hline
6. & $A(4, 3), B(7, 7)$ noqatları arasındaǵı aralıqtı tabıń. &  \\
\hline
7. & $3x^{2}+10xy+3y^{2}-2x-14y-13=0$ teńlemesiniń tipin anıqlań. &  \\
\hline
8. & $x^{2}-4y^{2}+6x+5=0$ giperbolanıń kanonikalıq teńlemesin dúziń. &  \\
\hline
9. & $M_{1}M_{2}$ kesindiniń ortasınıń koordinatalarınıń tabıń, eger $M_{1} (2, 3), M_{2} (4, 7)$ bolsa. &  \\
\hline
10. & $x+y-3=0$ hám $2x+3y-8=0$ tuwrıları óz-ara qanday jaylasqan? &  \\
\hline
\end{tabular}

\vspace{1cm}

\begin{tabular}{lll}
Tuwrı juwaplar sanı: \underline{\hspace{1.5cm}} & 
Bahası: \underline{\hspace{1.5cm}} & 
Imtixan alıwshınıń qolı: \underline{\hspace{2cm}} \\
\end{tabular}

\egroup

\newpage


\textbf{41-variant}\\

\bgroup
\def\arraystretch{1.6} % 1 is the default, change whatever you need

\begin{tabular}{|m{5.7cm}|m{9.5cm}|}
\hline
Familiyası hám atı & \\
\hline
Fakulteti  & \\
\hline
Toparı hám tálim baǵdarı  & \\
\hline
\end{tabular}

\vspace{1cm}

\begin{tabular}{|m{0.7cm}|m{10cm}|m{4cm}|}
\hline
№ & Soraw & Juwap \\
\hline
1. & Eki vektor qashan kollinear dep ataladı? &  \\
\hline
2. & Tuwrı múyeshli koordinatalar sisteması dep nege aytamız? &  \\
\hline
3. & $OXY$ tegisliginiń teńlemesi? &  \\
\hline
4. & Giperbolanıń kanonikalıq teńlemesi? &  \\
\hline
5. & $x^{2}+y^{2}-2x+4y-20=0$ sheńberdiń $C$ orayın hám $R$ radiusın tabıń. &  \\
\hline
6. & $(x+1)^{2}+(y-2) ^{2}+(z+2) ^{2}=49$ sferanıń orayınıń koordinataların tabıń. &  \\
\hline
7. & Eger $2a=16, e=\frac{5}{4}$ bolsa, fokusı abscissa kósherinde, koordinata basına salıstırǵanda simmetriyalıq jaylasqan giperbolanıń teńlemesin dúziń. &  \\
\hline
8. & Eger $2b=24, 2 c=10$ bolsa, onda abscissa kósherinde koordinata basına salıstırǵanda simmetriyalıq jaylasqan fokuslarǵa iye, ellipstiń teńlemesin dúziń. &  \\
\hline
9. & $M_{1} (12;-1)$ hám $M_{2} (0;4)$ noqatlardıń arasındaǵı aralıqtı tabıń. &  \\
\hline
10. & $x+y=0$ teńlemesi menen berilgen tuwrı sızıqtıń múyeshlik koefficientin anıqlań. &  \\
\hline
\end{tabular}

\vspace{1cm}

\begin{tabular}{lll}
Tuwrı juwaplar sanı: \underline{\hspace{1.5cm}} & 
Bahası: \underline{\hspace{1.5cm}} & 
Imtixan alıwshınıń qolı: \underline{\hspace{2cm}} \\
\end{tabular}

\egroup

\newpage


\textbf{42-variant}\\

\bgroup
\def\arraystretch{1.6} % 1 is the default, change whatever you need

\begin{tabular}{|m{5.7cm}|m{9.5cm}|}
\hline
Familiyası hám atı & \\
\hline
Fakulteti  & \\
\hline
Toparı hám tálim baǵdarı  & \\
\hline
\end{tabular}

\vspace{1cm}

\begin{tabular}{|m{0.7cm}|m{10cm}|m{4cm}|}
\hline
№ & Soraw & Juwap \\
\hline
1. & Eki vektordıń vektor kóbeymesiniń uzınlıǵın tabıw formulası? &  \\
\hline
2. & Tegislikdegi qálegen noqatınan berilgen eki noqatqa shekemgi bolǵan aralıqlardıń ayırmasınıń modulı ózgermeytuǵın bolǵan noqatlardıń geometriyalıq ornı ne dep ataladı? &  \\
\hline
3. & Eki tuwrı sızıq arasındaǵı múyeshti tabıw formulası? &  \\
\hline
4. & $\frac{x^2}{a^2}-\frac{y^2}{b^2}=1$ giperbolanıń $(x_0;y_0)$ noqatındaǵı urınbasınıń teńlemesin kórsetiń. &  \\
\hline
5. & Orayı $C (-1;2)$ noqatında, $A (-2;6 )$ noqatınan ótetuǵın sheńberdiń teńlemesin dúziń. &  \\
\hline
6. & $x+2=0$ keńislik qanday geometriyalıq betlikti anıqlaydı? &  \\
\hline
7. & $\frac{x^{2}}{225}-\frac{y^{2}}{64}=-1$ giperbola fokusınıń koordinatalarınıń tabıń. &  \\
\hline
8. & $9x^{2}+25y^{2}=225$ ellipsi berilgen, ellipstiń fokusların, ekscentrisitetin tabıń. &  \\
\hline
9. & $A (-1;0;1),\ B (1;-1;0)$ noqatları berilgen. $\overline{BA}$ vektorın tabıń. &  \\
\hline
10. & $2x+3y+4=0$ tuwrısına parallel hám $M_{0} (2;1)$ noqattan ótetuǵın tuwrınıń teńlemesin dúziń. &  \\
\hline
\end{tabular}

\vspace{1cm}

\begin{tabular}{lll}
Tuwrı juwaplar sanı: \underline{\hspace{1.5cm}} & 
Bahası: \underline{\hspace{1.5cm}} & 
Imtixan alıwshınıń qolı: \underline{\hspace{2cm}} \\
\end{tabular}

\egroup

\newpage


\textbf{43-variant}\\

\bgroup
\def\arraystretch{1.6} % 1 is the default, change whatever you need

\begin{tabular}{|m{5.7cm}|m{9.5cm}|}
\hline
Familiyası hám atı & \\
\hline
Fakulteti  & \\
\hline
Toparı hám tálim baǵdarı  & \\
\hline
\end{tabular}

\vspace{1cm}

\begin{tabular}{|m{0.7cm}|m{10cm}|m{4cm}|}
\hline
№ & Soraw & Juwap \\
\hline
1. & Vektorlardıń kósherdegi proekciyasınıń formulası? &  \\
\hline
2. & $Ax+By+D=0$ teńlemesi arqalı ... tegisliktiń teńlemesi berilgen? &  \\
\hline
3. & $\frac{x^2}{a^2}+\frac{y^2}{b^2}=1$ ellipstiń $(x_0;y_0)$ noqatındaǵı urınbasınıń teńlemesin tabıń. &  \\
\hline
4. & Vektorlardı qosıw koordinatalarda qanday formula menen anıqlanadı? &  \\
\hline
5. & $x+y-12=0$ tuwrısı $x^{2}+y^{2}-2y=0$ sheńberge salıstırǵanda qanday jaylasqan? &  \\
\hline
6. & $\left| \overline{a} \right|=8, \left| \overline{b} \right|=5, \alpha=60^{0}$ bolsa, $( \overline{a}\overline{b} )$ ni tabıń. &  \\
\hline
7. & $2x+3y-6=0$ tuwrınıń teńlemesin kesindilerde berilgen teńleme túrinde kórsetiń. &  \\
\hline
8. & $\overline{a}=\left\{ 4,-2,-4 \right\}$ hám $\overline{b}=\left\{ 6,-3, 2 \right\}$ vektorları berilgen, $(\overline{a}-\overline{b}) ^{2}$-? &  \\
\hline
9. & $5x-y+7=0$ hám $3x+2y=0$ tuwrıları arasındaǵı múyeshni tabıń. &  \\
\hline
10. & $\overline{a}=\left\{ 2, 1, 0 \right\}$ hám $\overline{b}=\left\{ 1, 0,-1 \right\}$ bolsa, $\overline{a}-\overline{b}$ ni tabıń. &  \\
\hline
\end{tabular}

\vspace{1cm}

\begin{tabular}{lll}
Tuwrı juwaplar sanı: \underline{\hspace{1.5cm}} & 
Bahası: \underline{\hspace{1.5cm}} & 
Imtixan alıwshınıń qolı: \underline{\hspace{2cm}} \\
\end{tabular}

\egroup

\newpage


\textbf{44-variant}\\

\bgroup
\def\arraystretch{1.6} % 1 is the default, change whatever you need

\begin{tabular}{|m{5.7cm}|m{9.5cm}|}
\hline
Familiyası hám atı & \\
\hline
Fakulteti  & \\
\hline
Toparı hám tálim baǵdarı  & \\
\hline
\end{tabular}

\vspace{1cm}

\begin{tabular}{|m{0.7cm}|m{10cm}|m{4cm}|}
\hline
№ & Soraw & Juwap \\
\hline
1. & $OY$ kósheriniń teńlemesi? &  \\
\hline
2. & Egerde $a=\{ x_1; y_1; z_1\}, b=\{ x_2, y_2; z_2\}$ bolsa, vektor kóbeymeniń koordinatalarda ańlatılıwı qanday boladı? &  \\
\hline
3. & $A_1x+B_1y+C_1z+D_1=0$ hám $Ax_2+By_2+Cz_2+D_2=0$ tegislikleri perpendikulyar bolıwı shárti &  \\
\hline
4. & Úsh vektordıń aralas kóbeymesi ushın $(abc)=0$ teńligi orınlı bolsa ne dep ataladı? &  \\
\hline
5. & Koordinatalar kósherleri hám $ 3x+4y-12=0 $ tuwrı sızıǵı menen shegaralanǵan úshmúyeshliktiń maydanın tabıń. &  \\
\hline
6. & $x-2y+1=0$ teńlemesi menen berilgen tuwrınıń normal túrdegi teńlemesin kórsetiń. &  \\
\hline
7. & $3x-y+5=0$, $x+3y-4=0$ tuwrı sızıqları arasındaǵı múyeshti tabıń. &  \\
\hline
8. & $\overline{a}=\{5,-6, 1 \}, \overline{b}=\{-4, 3, 0 \} $, $\overline{c}=\left\{ 5,-8, 10 \right\}$ vektorları berilgen. $2{\overline{a}}^{2}+4{\overline{b}}^{2}-5{\overline{c}}^{2}$ ańlatpasınıń mánisin tabıń. &  \\
\hline
9. & $(2, 3)$ hám $(4, 3)$ noqatlarınan ótiwshi tuwrı sızıqtıń teńlemesin dúziń. &  \\
\hline
10. & $x^{2}+y^{2}-2x+4y=0$ sheńberdiń teńlemesin kanonikalıq túrdegi teńlemege alıp keliń. &  \\
\hline
\end{tabular}

\vspace{1cm}

\begin{tabular}{lll}
Tuwrı juwaplar sanı: \underline{\hspace{1.5cm}} & 
Bahası: \underline{\hspace{1.5cm}} & 
Imtixan alıwshınıń qolı: \underline{\hspace{2cm}} \\
\end{tabular}

\egroup

\newpage


\textbf{45-variant}\\

\bgroup
\def\arraystretch{1.6} % 1 is the default, change whatever you need

\begin{tabular}{|m{5.7cm}|m{9.5cm}|}
\hline
Familiyası hám atı & \\
\hline
Fakulteti  & \\
\hline
Toparı hám tálim baǵdarı  & \\
\hline
\end{tabular}

\vspace{1cm}

\begin{tabular}{|m{0.7cm}|m{10cm}|m{4cm}|}
\hline
№ & Soraw & Juwap \\
\hline
1. & $A_1x+B_1y+C_1z+D_1=0$ hám $Ax_2+By_2+Cz_2+D_2=0$ tegislikleri ústpe-úst túsiwi shárti? &  \\
\hline
2. & Eki vektordıń skalyar kóbeymesiniń formulası? &  \\
\hline
3. & $A_1x+B_1y+C_1z+D_1=0$ hám $Ax_2+By_2+Cz_2+D_2=0$ tegislikleri parallel bolıwı shárti &  \\
\hline
4. & $Ax+C=0$ tuwrı sızıqtıń grafigi koordinata kósherlerine salıstırǵanda qanday jaylasqan? &  \\
\hline
5. & $A(4, 3), B(7, 7)$ noqatları arasındaǵı aralıqtı tabıń. &  \\
\hline
6. & $3x^{2}+10xy+3y^{2}-2x-14y-13=0$ teńlemesiniń tipin anıqlań. &  \\
\hline
7. & $x^{2}-4y^{2}+6x+5=0$ giperbolanıń kanonikalıq teńlemesin dúziń. &  \\
\hline
8. & $M_{1}M_{2}$ kesindiniń ortasınıń koordinatalarınıń tabıń, eger $M_{1} (2, 3), M_{2} (4, 7)$ bolsa. &  \\
\hline
9. & $x+y-3=0$ hám $2x+3y-8=0$ tuwrıları óz-ara qanday jaylasqan? &  \\
\hline
10. & $x^{2}+y^{2}-2x+4y-20=0$ sheńberdiń $C$ orayın hám $R$ radiusın tabıń. &  \\
\hline
\end{tabular}

\vspace{1cm}

\begin{tabular}{lll}
Tuwrı juwaplar sanı: \underline{\hspace{1.5cm}} & 
Bahası: \underline{\hspace{1.5cm}} & 
Imtixan alıwshınıń qolı: \underline{\hspace{2cm}} \\
\end{tabular}

\egroup

\newpage


\textbf{46-variant}\\

\bgroup
\def\arraystretch{1.6} % 1 is the default, change whatever you need

\begin{tabular}{|m{5.7cm}|m{9.5cm}|}
\hline
Familiyası hám atı & \\
\hline
Fakulteti  & \\
\hline
Toparı hám tálim baǵdarı  & \\
\hline
\end{tabular}

\vspace{1cm}

\begin{tabular}{|m{0.7cm}|m{10cm}|m{4cm}|}
\hline
№ & Soraw & Juwap \\
\hline
1. & Eki vektor qashan kollinear dep ataladı? &  \\
\hline
2. & Tuwrı múyeshli koordinatalar sisteması dep nege aytamız? &  \\
\hline
3. & $OXY$ tegisliginiń teńlemesi? &  \\
\hline
4. & Giperbolanıń kanonikalıq teńlemesi? &  \\
\hline
5. & $(x+1)^{2}+(y-2) ^{2}+(z+2) ^{2}=49$ sferanıń orayınıń koordinataların tabıń. &  \\
\hline
6. & Eger $2a=16, e=\frac{5}{4}$ bolsa, fokusı abscissa kósherinde, koordinata basına salıstırǵanda simmetriyalıq jaylasqan giperbolanıń teńlemesin dúziń. &  \\
\hline
7. & Eger $2b=24, 2 c=10$ bolsa, onda abscissa kósherinde koordinata basına salıstırǵanda simmetriyalıq jaylasqan fokuslarǵa iye, ellipstiń teńlemesin dúziń. &  \\
\hline
8. & $M_{1} (12;-1)$ hám $M_{2} (0;4)$ noqatlardıń arasındaǵı aralıqtı tabıń. &  \\
\hline
9. & $x+y=0$ teńlemesi menen berilgen tuwrı sızıqtıń múyeshlik koefficientin anıqlań. &  \\
\hline
10. & Orayı $C (-1;2)$ noqatında, $A (-2;6 )$ noqatınan ótetuǵın sheńberdiń teńlemesin dúziń. &  \\
\hline
\end{tabular}

\vspace{1cm}

\begin{tabular}{lll}
Tuwrı juwaplar sanı: \underline{\hspace{1.5cm}} & 
Bahası: \underline{\hspace{1.5cm}} & 
Imtixan alıwshınıń qolı: \underline{\hspace{2cm}} \\
\end{tabular}

\egroup

\newpage


\textbf{47-variant}\\

\bgroup
\def\arraystretch{1.6} % 1 is the default, change whatever you need

\begin{tabular}{|m{5.7cm}|m{9.5cm}|}
\hline
Familiyası hám atı & \\
\hline
Fakulteti  & \\
\hline
Toparı hám tálim baǵdarı  & \\
\hline
\end{tabular}

\vspace{1cm}

\begin{tabular}{|m{0.7cm}|m{10cm}|m{4cm}|}
\hline
№ & Soraw & Juwap \\
\hline
1. & Eki vektordıń vektor kóbeymesiniń uzınlıǵın tabıw formulası? &  \\
\hline
2. & Tegislikdegi qálegen noqatınan berilgen eki noqatqa shekemgi bolǵan aralıqlardıń ayırmasınıń modulı ózgermeytuǵın bolǵan noqatlardıń geometriyalıq ornı ne dep ataladı? &  \\
\hline
3. & Eki tuwrı sızıq arasındaǵı múyeshti tabıw formulası? &  \\
\hline
4. & $\frac{x^2}{a^2}-\frac{y^2}{b^2}=1$ giperbolanıń $(x_0;y_0)$ noqatındaǵı urınbasınıń teńlemesin kórsetiń. &  \\
\hline
5. & $x+2=0$ keńislik qanday geometriyalıq betlikti anıqlaydı? &  \\
\hline
6. & $\frac{x^{2}}{225}-\frac{y^{2}}{64}=-1$ giperbola fokusınıń koordinatalarınıń tabıń. &  \\
\hline
7. & $9x^{2}+25y^{2}=225$ ellipsi berilgen, ellipstiń fokusların, ekscentrisitetin tabıń. &  \\
\hline
8. & $A (-1;0;1),\ B (1;-1;0)$ noqatları berilgen. $\overline{BA}$ vektorın tabıń. &  \\
\hline
9. & $2x+3y+4=0$ tuwrısına parallel hám $M_{0} (2;1)$ noqattan ótetuǵın tuwrınıń teńlemesin dúziń. &  \\
\hline
10. & $x+y-12=0$ tuwrısı $x^{2}+y^{2}-2y=0$ sheńberge salıstırǵanda qanday jaylasqan? &  \\
\hline
\end{tabular}

\vspace{1cm}

\begin{tabular}{lll}
Tuwrı juwaplar sanı: \underline{\hspace{1.5cm}} & 
Bahası: \underline{\hspace{1.5cm}} & 
Imtixan alıwshınıń qolı: \underline{\hspace{2cm}} \\
\end{tabular}

\egroup

\newpage


\textbf{48-variant}\\

\bgroup
\def\arraystretch{1.6} % 1 is the default, change whatever you need

\begin{tabular}{|m{5.7cm}|m{9.5cm}|}
\hline
Familiyası hám atı & \\
\hline
Fakulteti  & \\
\hline
Toparı hám tálim baǵdarı  & \\
\hline
\end{tabular}

\vspace{1cm}

\begin{tabular}{|m{0.7cm}|m{10cm}|m{4cm}|}
\hline
№ & Soraw & Juwap \\
\hline
1. & Vektorlardıń kósherdegi proekciyasınıń formulası? &  \\
\hline
2. & $Ax+By+D=0$ teńlemesi arqalı ... tegisliktiń teńlemesi berilgen? &  \\
\hline
3. & $\frac{x^2}{a^2}+\frac{y^2}{b^2}=1$ ellipstiń $(x_0;y_0)$ noqatındaǵı urınbasınıń teńlemesin tabıń. &  \\
\hline
4. & Vektorlardı qosıw koordinatalarda qanday formula menen anıqlanadı? &  \\
\hline
5. & $\left| \overline{a} \right|=8, \left| \overline{b} \right|=5, \alpha=60^{0}$ bolsa, $( \overline{a}\overline{b} )$ ni tabıń. &  \\
\hline
6. & $2x+3y-6=0$ tuwrınıń teńlemesin kesindilerde berilgen teńleme túrinde kórsetiń. &  \\
\hline
7. & $\overline{a}=\left\{ 4,-2,-4 \right\}$ hám $\overline{b}=\left\{ 6,-3, 2 \right\}$ vektorları berilgen, $(\overline{a}-\overline{b}) ^{2}$-? &  \\
\hline
8. & $5x-y+7=0$ hám $3x+2y=0$ tuwrıları arasındaǵı múyeshni tabıń. &  \\
\hline
9. & $\overline{a}=\left\{ 2, 1, 0 \right\}$ hám $\overline{b}=\left\{ 1, 0,-1 \right\}$ bolsa, $\overline{a}-\overline{b}$ ni tabıń. &  \\
\hline
10. & Koordinatalar kósherleri hám $ 3x+4y-12=0 $ tuwrı sızıǵı menen shegaralanǵan úshmúyeshliktiń maydanın tabıń. &  \\
\hline
\end{tabular}

\vspace{1cm}

\begin{tabular}{lll}
Tuwrı juwaplar sanı: \underline{\hspace{1.5cm}} & 
Bahası: \underline{\hspace{1.5cm}} & 
Imtixan alıwshınıń qolı: \underline{\hspace{2cm}} \\
\end{tabular}

\egroup

\newpage


\textbf{49-variant}\\

\bgroup
\def\arraystretch{1.6} % 1 is the default, change whatever you need

\begin{tabular}{|m{5.7cm}|m{9.5cm}|}
\hline
Familiyası hám atı & \\
\hline
Fakulteti  & \\
\hline
Toparı hám tálim baǵdarı  & \\
\hline
\end{tabular}

\vspace{1cm}

\begin{tabular}{|m{0.7cm}|m{10cm}|m{4cm}|}
\hline
№ & Soraw & Juwap \\
\hline
1. & $OY$ kósheriniń teńlemesi? &  \\
\hline
2. & Egerde $a=\{ x_1; y_1; z_1\}, b=\{ x_2, y_2; z_2\}$ bolsa, vektor kóbeymeniń koordinatalarda ańlatılıwı qanday boladı? &  \\
\hline
3. & $A_1x+B_1y+C_1z+D_1=0$ hám $Ax_2+By_2+Cz_2+D_2=0$ tegislikleri perpendikulyar bolıwı shárti &  \\
\hline
4. & Úsh vektordıń aralas kóbeymesi ushın $(abc)=0$ teńligi orınlı bolsa ne dep ataladı? &  \\
\hline
5. & $x-2y+1=0$ teńlemesi menen berilgen tuwrınıń normal túrdegi teńlemesin kórsetiń. &  \\
\hline
6. & $3x-y+5=0$, $x+3y-4=0$ tuwrı sızıqları arasındaǵı múyeshti tabıń. &  \\
\hline
7. & $\overline{a}=\{5,-6, 1 \}, \overline{b}=\{-4, 3, 0 \} $, $\overline{c}=\left\{ 5,-8, 10 \right\}$ vektorları berilgen. $2{\overline{a}}^{2}+4{\overline{b}}^{2}-5{\overline{c}}^{2}$ ańlatpasınıń mánisin tabıń. &  \\
\hline
8. & $(2, 3)$ hám $(4, 3)$ noqatlarınan ótiwshi tuwrı sızıqtıń teńlemesin dúziń. &  \\
\hline
9. & $x^{2}+y^{2}-2x+4y=0$ sheńberdiń teńlemesin kanonikalıq túrdegi teńlemege alıp keliń. &  \\
\hline
10. & $A(4, 3), B(7, 7)$ noqatları arasındaǵı aralıqtı tabıń. &  \\
\hline
\end{tabular}

\vspace{1cm}

\begin{tabular}{lll}
Tuwrı juwaplar sanı: \underline{\hspace{1.5cm}} & 
Bahası: \underline{\hspace{1.5cm}} & 
Imtixan alıwshınıń qolı: \underline{\hspace{2cm}} \\
\end{tabular}

\egroup

\newpage


\textbf{50-variant}\\

\bgroup
\def\arraystretch{1.6} % 1 is the default, change whatever you need

\begin{tabular}{|m{5.7cm}|m{9.5cm}|}
\hline
Familiyası hám atı & \\
\hline
Fakulteti  & \\
\hline
Toparı hám tálim baǵdarı  & \\
\hline
\end{tabular}

\vspace{1cm}

\begin{tabular}{|m{0.7cm}|m{10cm}|m{4cm}|}
\hline
№ & Soraw & Juwap \\
\hline
1. & $A_1x+B_1y+C_1z+D_1=0$ hám $Ax_2+By_2+Cz_2+D_2=0$ tegislikleri ústpe-úst túsiwi shárti? &  \\
\hline
2. & Eki vektordıń skalyar kóbeymesiniń formulası? &  \\
\hline
3. & $A_1x+B_1y+C_1z+D_1=0$ hám $Ax_2+By_2+Cz_2+D_2=0$ tegislikleri parallel bolıwı shárti &  \\
\hline
4. & $Ax+C=0$ tuwrı sızıqtıń grafigi koordinata kósherlerine salıstırǵanda qanday jaylasqan? &  \\
\hline
5. & $3x^{2}+10xy+3y^{2}-2x-14y-13=0$ teńlemesiniń tipin anıqlań. &  \\
\hline
6. & $x^{2}-4y^{2}+6x+5=0$ giperbolanıń kanonikalıq teńlemesin dúziń. &  \\
\hline
7. & $M_{1}M_{2}$ kesindiniń ortasınıń koordinatalarınıń tabıń, eger $M_{1} (2, 3), M_{2} (4, 7)$ bolsa. &  \\
\hline
8. & $x+y-3=0$ hám $2x+3y-8=0$ tuwrıları óz-ara qanday jaylasqan? &  \\
\hline
9. & $x^{2}+y^{2}-2x+4y-20=0$ sheńberdiń $C$ orayın hám $R$ radiusın tabıń. &  \\
\hline
10. & $(x+1)^{2}+(y-2) ^{2}+(z+2) ^{2}=49$ sferanıń orayınıń koordinataların tabıń. &  \\
\hline
\end{tabular}

\vspace{1cm}

\begin{tabular}{lll}
Tuwrı juwaplar sanı: \underline{\hspace{1.5cm}} & 
Bahası: \underline{\hspace{1.5cm}} & 
Imtixan alıwshınıń qolı: \underline{\hspace{2cm}} \\
\end{tabular}

\egroup

\newpage


\textbf{51-variant}\\

\bgroup
\def\arraystretch{1.6} % 1 is the default, change whatever you need

\begin{tabular}{|m{5.7cm}|m{9.5cm}|}
\hline
Familiyası hám atı & \\
\hline
Fakulteti  & \\
\hline
Toparı hám tálim baǵdarı  & \\
\hline
\end{tabular}

\vspace{1cm}

\begin{tabular}{|m{0.7cm}|m{10cm}|m{4cm}|}
\hline
№ & Soraw & Juwap \\
\hline
1. & Eki vektor qashan kollinear dep ataladı? &  \\
\hline
2. & Tuwrı múyeshli koordinatalar sisteması dep nege aytamız? &  \\
\hline
3. & $OXY$ tegisliginiń teńlemesi? &  \\
\hline
4. & Giperbolanıń kanonikalıq teńlemesi? &  \\
\hline
5. & Eger $2a=16, e=\frac{5}{4}$ bolsa, fokusı abscissa kósherinde, koordinata basına salıstırǵanda simmetriyalıq jaylasqan giperbolanıń teńlemesin dúziń. &  \\
\hline
6. & Eger $2b=24, 2 c=10$ bolsa, onda abscissa kósherinde koordinata basına salıstırǵanda simmetriyalıq jaylasqan fokuslarǵa iye, ellipstiń teńlemesin dúziń. &  \\
\hline
7. & $M_{1} (12;-1)$ hám $M_{2} (0;4)$ noqatlardıń arasındaǵı aralıqtı tabıń. &  \\
\hline
8. & $x+y=0$ teńlemesi menen berilgen tuwrı sızıqtıń múyeshlik koefficientin anıqlań. &  \\
\hline
9. & Orayı $C (-1;2)$ noqatında, $A (-2;6 )$ noqatınan ótetuǵın sheńberdiń teńlemesin dúziń. &  \\
\hline
10. & $x+2=0$ keńislik qanday geometriyalıq betlikti anıqlaydı? &  \\
\hline
\end{tabular}

\vspace{1cm}

\begin{tabular}{lll}
Tuwrı juwaplar sanı: \underline{\hspace{1.5cm}} & 
Bahası: \underline{\hspace{1.5cm}} & 
Imtixan alıwshınıń qolı: \underline{\hspace{2cm}} \\
\end{tabular}

\egroup

\newpage


\textbf{52-variant}\\

\bgroup
\def\arraystretch{1.6} % 1 is the default, change whatever you need

\begin{tabular}{|m{5.7cm}|m{9.5cm}|}
\hline
Familiyası hám atı & \\
\hline
Fakulteti  & \\
\hline
Toparı hám tálim baǵdarı  & \\
\hline
\end{tabular}

\vspace{1cm}

\begin{tabular}{|m{0.7cm}|m{10cm}|m{4cm}|}
\hline
№ & Soraw & Juwap \\
\hline
1. & Eki vektordıń vektor kóbeymesiniń uzınlıǵın tabıw formulası? &  \\
\hline
2. & Tegislikdegi qálegen noqatınan berilgen eki noqatqa shekemgi bolǵan aralıqlardıń ayırmasınıń modulı ózgermeytuǵın bolǵan noqatlardıń geometriyalıq ornı ne dep ataladı? &  \\
\hline
3. & Eki tuwrı sızıq arasındaǵı múyeshti tabıw formulası? &  \\
\hline
4. & $\frac{x^2}{a^2}-\frac{y^2}{b^2}=1$ giperbolanıń $(x_0;y_0)$ noqatındaǵı urınbasınıń teńlemesin kórsetiń. &  \\
\hline
5. & $\frac{x^{2}}{225}-\frac{y^{2}}{64}=-1$ giperbola fokusınıń koordinatalarınıń tabıń. &  \\
\hline
6. & $9x^{2}+25y^{2}=225$ ellipsi berilgen, ellipstiń fokusların, ekscentrisitetin tabıń. &  \\
\hline
7. & $A (-1;0;1),\ B (1;-1;0)$ noqatları berilgen. $\overline{BA}$ vektorın tabıń. &  \\
\hline
8. & $2x+3y+4=0$ tuwrısına parallel hám $M_{0} (2;1)$ noqattan ótetuǵın tuwrınıń teńlemesin dúziń. &  \\
\hline
9. & $x+y-12=0$ tuwrısı $x^{2}+y^{2}-2y=0$ sheńberge salıstırǵanda qanday jaylasqan? &  \\
\hline
10. & $\left| \overline{a} \right|=8, \left| \overline{b} \right|=5, \alpha=60^{0}$ bolsa, $( \overline{a}\overline{b} )$ ni tabıń. &  \\
\hline
\end{tabular}

\vspace{1cm}

\begin{tabular}{lll}
Tuwrı juwaplar sanı: \underline{\hspace{1.5cm}} & 
Bahası: \underline{\hspace{1.5cm}} & 
Imtixan alıwshınıń qolı: \underline{\hspace{2cm}} \\
\end{tabular}

\egroup

\newpage


\textbf{53-variant}\\

\bgroup
\def\arraystretch{1.6} % 1 is the default, change whatever you need

\begin{tabular}{|m{5.7cm}|m{9.5cm}|}
\hline
Familiyası hám atı & \\
\hline
Fakulteti  & \\
\hline
Toparı hám tálim baǵdarı  & \\
\hline
\end{tabular}

\vspace{1cm}

\begin{tabular}{|m{0.7cm}|m{10cm}|m{4cm}|}
\hline
№ & Soraw & Juwap \\
\hline
1. & Vektorlardıń kósherdegi proekciyasınıń formulası? &  \\
\hline
2. & $Ax+By+D=0$ teńlemesi arqalı ... tegisliktiń teńlemesi berilgen? &  \\
\hline
3. & $\frac{x^2}{a^2}+\frac{y^2}{b^2}=1$ ellipstiń $(x_0;y_0)$ noqatındaǵı urınbasınıń teńlemesin tabıń. &  \\
\hline
4. & Vektorlardı qosıw koordinatalarda qanday formula menen anıqlanadı? &  \\
\hline
5. & $2x+3y-6=0$ tuwrınıń teńlemesin kesindilerde berilgen teńleme túrinde kórsetiń. &  \\
\hline
6. & $\overline{a}=\left\{ 4,-2,-4 \right\}$ hám $\overline{b}=\left\{ 6,-3, 2 \right\}$ vektorları berilgen, $(\overline{a}-\overline{b}) ^{2}$-? &  \\
\hline
7. & $5x-y+7=0$ hám $3x+2y=0$ tuwrıları arasındaǵı múyeshni tabıń. &  \\
\hline
8. & $\overline{a}=\left\{ 2, 1, 0 \right\}$ hám $\overline{b}=\left\{ 1, 0,-1 \right\}$ bolsa, $\overline{a}-\overline{b}$ ni tabıń. &  \\
\hline
9. & Koordinatalar kósherleri hám $ 3x+4y-12=0 $ tuwrı sızıǵı menen shegaralanǵan úshmúyeshliktiń maydanın tabıń. &  \\
\hline
10. & $x-2y+1=0$ teńlemesi menen berilgen tuwrınıń normal túrdegi teńlemesin kórsetiń. &  \\
\hline
\end{tabular}

\vspace{1cm}

\begin{tabular}{lll}
Tuwrı juwaplar sanı: \underline{\hspace{1.5cm}} & 
Bahası: \underline{\hspace{1.5cm}} & 
Imtixan alıwshınıń qolı: \underline{\hspace{2cm}} \\
\end{tabular}

\egroup

\newpage


\textbf{54-variant}\\

\bgroup
\def\arraystretch{1.6} % 1 is the default, change whatever you need

\begin{tabular}{|m{5.7cm}|m{9.5cm}|}
\hline
Familiyası hám atı & \\
\hline
Fakulteti  & \\
\hline
Toparı hám tálim baǵdarı  & \\
\hline
\end{tabular}

\vspace{1cm}

\begin{tabular}{|m{0.7cm}|m{10cm}|m{4cm}|}
\hline
№ & Soraw & Juwap \\
\hline
1. & $OY$ kósheriniń teńlemesi? &  \\
\hline
2. & Egerde $a=\{ x_1; y_1; z_1\}, b=\{ x_2, y_2; z_2\}$ bolsa, vektor kóbeymeniń koordinatalarda ańlatılıwı qanday boladı? &  \\
\hline
3. & $A_1x+B_1y+C_1z+D_1=0$ hám $Ax_2+By_2+Cz_2+D_2=0$ tegislikleri perpendikulyar bolıwı shárti &  \\
\hline
4. & Úsh vektordıń aralas kóbeymesi ushın $(abc)=0$ teńligi orınlı bolsa ne dep ataladı? &  \\
\hline
5. & $3x-y+5=0$, $x+3y-4=0$ tuwrı sızıqları arasındaǵı múyeshti tabıń. &  \\
\hline
6. & $\overline{a}=\{5,-6, 1 \}, \overline{b}=\{-4, 3, 0 \} $, $\overline{c}=\left\{ 5,-8, 10 \right\}$ vektorları berilgen. $2{\overline{a}}^{2}+4{\overline{b}}^{2}-5{\overline{c}}^{2}$ ańlatpasınıń mánisin tabıń. &  \\
\hline
7. & $(2, 3)$ hám $(4, 3)$ noqatlarınan ótiwshi tuwrı sızıqtıń teńlemesin dúziń. &  \\
\hline
8. & $x^{2}+y^{2}-2x+4y=0$ sheńberdiń teńlemesin kanonikalıq túrdegi teńlemege alıp keliń. &  \\
\hline
9. & $A(4, 3), B(7, 7)$ noqatları arasındaǵı aralıqtı tabıń. &  \\
\hline
10. & $3x^{2}+10xy+3y^{2}-2x-14y-13=0$ teńlemesiniń tipin anıqlań. &  \\
\hline
\end{tabular}

\vspace{1cm}

\begin{tabular}{lll}
Tuwrı juwaplar sanı: \underline{\hspace{1.5cm}} & 
Bahası: \underline{\hspace{1.5cm}} & 
Imtixan alıwshınıń qolı: \underline{\hspace{2cm}} \\
\end{tabular}

\egroup

\newpage


\textbf{55-variant}\\

\bgroup
\def\arraystretch{1.6} % 1 is the default, change whatever you need

\begin{tabular}{|m{5.7cm}|m{9.5cm}|}
\hline
Familiyası hám atı & \\
\hline
Fakulteti  & \\
\hline
Toparı hám tálim baǵdarı  & \\
\hline
\end{tabular}

\vspace{1cm}

\begin{tabular}{|m{0.7cm}|m{10cm}|m{4cm}|}
\hline
№ & Soraw & Juwap \\
\hline
1. & $A_1x+B_1y+C_1z+D_1=0$ hám $Ax_2+By_2+Cz_2+D_2=0$ tegislikleri ústpe-úst túsiwi shárti? &  \\
\hline
2. & Eki vektordıń skalyar kóbeymesiniń formulası? &  \\
\hline
3. & $A_1x+B_1y+C_1z+D_1=0$ hám $Ax_2+By_2+Cz_2+D_2=0$ tegislikleri parallel bolıwı shárti &  \\
\hline
4. & $Ax+C=0$ tuwrı sızıqtıń grafigi koordinata kósherlerine salıstırǵanda qanday jaylasqan? &  \\
\hline
5. & $x^{2}-4y^{2}+6x+5=0$ giperbolanıń kanonikalıq teńlemesin dúziń. &  \\
\hline
6. & $M_{1}M_{2}$ kesindiniń ortasınıń koordinatalarınıń tabıń, eger $M_{1} (2, 3), M_{2} (4, 7)$ bolsa. &  \\
\hline
7. & $x+y-3=0$ hám $2x+3y-8=0$ tuwrıları óz-ara qanday jaylasqan? &  \\
\hline
8. & $x^{2}+y^{2}-2x+4y-20=0$ sheńberdiń $C$ orayın hám $R$ radiusın tabıń. &  \\
\hline
9. & $(x+1)^{2}+(y-2) ^{2}+(z+2) ^{2}=49$ sferanıń orayınıń koordinataların tabıń. &  \\
\hline
10. & Eger $2a=16, e=\frac{5}{4}$ bolsa, fokusı abscissa kósherinde, koordinata basına salıstırǵanda simmetriyalıq jaylasqan giperbolanıń teńlemesin dúziń. &  \\
\hline
\end{tabular}

\vspace{1cm}

\begin{tabular}{lll}
Tuwrı juwaplar sanı: \underline{\hspace{1.5cm}} & 
Bahası: \underline{\hspace{1.5cm}} & 
Imtixan alıwshınıń qolı: \underline{\hspace{2cm}} \\
\end{tabular}

\egroup

\newpage


\textbf{56-variant}\\

\bgroup
\def\arraystretch{1.6} % 1 is the default, change whatever you need

\begin{tabular}{|m{5.7cm}|m{9.5cm}|}
\hline
Familiyası hám atı & \\
\hline
Fakulteti  & \\
\hline
Toparı hám tálim baǵdarı  & \\
\hline
\end{tabular}

\vspace{1cm}

\begin{tabular}{|m{0.7cm}|m{10cm}|m{4cm}|}
\hline
№ & Soraw & Juwap \\
\hline
1. & Eki vektor qashan kollinear dep ataladı? &  \\
\hline
2. & Tuwrı múyeshli koordinatalar sisteması dep nege aytamız? &  \\
\hline
3. & $OXY$ tegisliginiń teńlemesi? &  \\
\hline
4. & Giperbolanıń kanonikalıq teńlemesi? &  \\
\hline
5. & Eger $2b=24, 2 c=10$ bolsa, onda abscissa kósherinde koordinata basına salıstırǵanda simmetriyalıq jaylasqan fokuslarǵa iye, ellipstiń teńlemesin dúziń. &  \\
\hline
6. & $M_{1} (12;-1)$ hám $M_{2} (0;4)$ noqatlardıń arasındaǵı aralıqtı tabıń. &  \\
\hline
7. & $x+y=0$ teńlemesi menen berilgen tuwrı sızıqtıń múyeshlik koefficientin anıqlań. &  \\
\hline
8. & Orayı $C (-1;2)$ noqatında, $A (-2;6 )$ noqatınan ótetuǵın sheńberdiń teńlemesin dúziń. &  \\
\hline
9. & $x+2=0$ keńislik qanday geometriyalıq betlikti anıqlaydı? &  \\
\hline
10. & $\frac{x^{2}}{225}-\frac{y^{2}}{64}=-1$ giperbola fokusınıń koordinatalarınıń tabıń. &  \\
\hline
\end{tabular}

\vspace{1cm}

\begin{tabular}{lll}
Tuwrı juwaplar sanı: \underline{\hspace{1.5cm}} & 
Bahası: \underline{\hspace{1.5cm}} & 
Imtixan alıwshınıń qolı: \underline{\hspace{2cm}} \\
\end{tabular}

\egroup

\newpage


\textbf{57-variant}\\

\bgroup
\def\arraystretch{1.6} % 1 is the default, change whatever you need

\begin{tabular}{|m{5.7cm}|m{9.5cm}|}
\hline
Familiyası hám atı & \\
\hline
Fakulteti  & \\
\hline
Toparı hám tálim baǵdarı  & \\
\hline
\end{tabular}

\vspace{1cm}

\begin{tabular}{|m{0.7cm}|m{10cm}|m{4cm}|}
\hline
№ & Soraw & Juwap \\
\hline
1. & Eki vektordıń vektor kóbeymesiniń uzınlıǵın tabıw formulası? &  \\
\hline
2. & Tegislikdegi qálegen noqatınan berilgen eki noqatqa shekemgi bolǵan aralıqlardıń ayırmasınıń modulı ózgermeytuǵın bolǵan noqatlardıń geometriyalıq ornı ne dep ataladı? &  \\
\hline
3. & Eki tuwrı sızıq arasındaǵı múyeshti tabıw formulası? &  \\
\hline
4. & $\frac{x^2}{a^2}-\frac{y^2}{b^2}=1$ giperbolanıń $(x_0;y_0)$ noqatındaǵı urınbasınıń teńlemesin kórsetiń. &  \\
\hline
5. & $9x^{2}+25y^{2}=225$ ellipsi berilgen, ellipstiń fokusların, ekscentrisitetin tabıń. &  \\
\hline
6. & $A (-1;0;1),\ B (1;-1;0)$ noqatları berilgen. $\overline{BA}$ vektorın tabıń. &  \\
\hline
7. & $2x+3y+4=0$ tuwrısına parallel hám $M_{0} (2;1)$ noqattan ótetuǵın tuwrınıń teńlemesin dúziń. &  \\
\hline
8. & $x+y-12=0$ tuwrısı $x^{2}+y^{2}-2y=0$ sheńberge salıstırǵanda qanday jaylasqan? &  \\
\hline
9. & $\left| \overline{a} \right|=8, \left| \overline{b} \right|=5, \alpha=60^{0}$ bolsa, $( \overline{a}\overline{b} )$ ni tabıń. &  \\
\hline
10. & $2x+3y-6=0$ tuwrınıń teńlemesin kesindilerde berilgen teńleme túrinde kórsetiń. &  \\
\hline
\end{tabular}

\vspace{1cm}

\begin{tabular}{lll}
Tuwrı juwaplar sanı: \underline{\hspace{1.5cm}} & 
Bahası: \underline{\hspace{1.5cm}} & 
Imtixan alıwshınıń qolı: \underline{\hspace{2cm}} \\
\end{tabular}

\egroup

\newpage


\textbf{58-variant}\\

\bgroup
\def\arraystretch{1.6} % 1 is the default, change whatever you need

\begin{tabular}{|m{5.7cm}|m{9.5cm}|}
\hline
Familiyası hám atı & \\
\hline
Fakulteti  & \\
\hline
Toparı hám tálim baǵdarı  & \\
\hline
\end{tabular}

\vspace{1cm}

\begin{tabular}{|m{0.7cm}|m{10cm}|m{4cm}|}
\hline
№ & Soraw & Juwap \\
\hline
1. & Vektorlardıń kósherdegi proekciyasınıń formulası? &  \\
\hline
2. & $Ax+By+D=0$ teńlemesi arqalı ... tegisliktiń teńlemesi berilgen? &  \\
\hline
3. & $\frac{x^2}{a^2}+\frac{y^2}{b^2}=1$ ellipstiń $(x_0;y_0)$ noqatındaǵı urınbasınıń teńlemesin tabıń. &  \\
\hline
4. & Vektorlardı qosıw koordinatalarda qanday formula menen anıqlanadı? &  \\
\hline
5. & $\overline{a}=\left\{ 4,-2,-4 \right\}$ hám $\overline{b}=\left\{ 6,-3, 2 \right\}$ vektorları berilgen, $(\overline{a}-\overline{b}) ^{2}$-? &  \\
\hline
6. & $5x-y+7=0$ hám $3x+2y=0$ tuwrıları arasındaǵı múyeshni tabıń. &  \\
\hline
7. & $\overline{a}=\left\{ 2, 1, 0 \right\}$ hám $\overline{b}=\left\{ 1, 0,-1 \right\}$ bolsa, $\overline{a}-\overline{b}$ ni tabıń. &  \\
\hline
8. & Koordinatalar kósherleri hám $ 3x+4y-12=0 $ tuwrı sızıǵı menen shegaralanǵan úshmúyeshliktiń maydanın tabıń. &  \\
\hline
9. & $x-2y+1=0$ teńlemesi menen berilgen tuwrınıń normal túrdegi teńlemesin kórsetiń. &  \\
\hline
10. & $3x-y+5=0$, $x+3y-4=0$ tuwrı sızıqları arasındaǵı múyeshti tabıń. &  \\
\hline
\end{tabular}

\vspace{1cm}

\begin{tabular}{lll}
Tuwrı juwaplar sanı: \underline{\hspace{1.5cm}} & 
Bahası: \underline{\hspace{1.5cm}} & 
Imtixan alıwshınıń qolı: \underline{\hspace{2cm}} \\
\end{tabular}

\egroup

\newpage


\textbf{59-variant}\\

\bgroup
\def\arraystretch{1.6} % 1 is the default, change whatever you need

\begin{tabular}{|m{5.7cm}|m{9.5cm}|}
\hline
Familiyası hám atı & \\
\hline
Fakulteti  & \\
\hline
Toparı hám tálim baǵdarı  & \\
\hline
\end{tabular}

\vspace{1cm}

\begin{tabular}{|m{0.7cm}|m{10cm}|m{4cm}|}
\hline
№ & Soraw & Juwap \\
\hline
1. & $OY$ kósheriniń teńlemesi? &  \\
\hline
2. & Egerde $a=\{ x_1; y_1; z_1\}, b=\{ x_2, y_2; z_2\}$ bolsa, vektor kóbeymeniń koordinatalarda ańlatılıwı qanday boladı? &  \\
\hline
3. & $A_1x+B_1y+C_1z+D_1=0$ hám $Ax_2+By_2+Cz_2+D_2=0$ tegislikleri perpendikulyar bolıwı shárti &  \\
\hline
4. & Úsh vektordıń aralas kóbeymesi ushın $(abc)=0$ teńligi orınlı bolsa ne dep ataladı? &  \\
\hline
5. & $\overline{a}=\{5,-6, 1 \}, \overline{b}=\{-4, 3, 0 \} $, $\overline{c}=\left\{ 5,-8, 10 \right\}$ vektorları berilgen. $2{\overline{a}}^{2}+4{\overline{b}}^{2}-5{\overline{c}}^{2}$ ańlatpasınıń mánisin tabıń. &  \\
\hline
6. & $(2, 3)$ hám $(4, 3)$ noqatlarınan ótiwshi tuwrı sızıqtıń teńlemesin dúziń. &  \\
\hline
7. & $x^{2}+y^{2}-2x+4y=0$ sheńberdiń teńlemesin kanonikalıq túrdegi teńlemege alıp keliń. &  \\
\hline
8. & $A(4, 3), B(7, 7)$ noqatları arasındaǵı aralıqtı tabıń. &  \\
\hline
9. & $3x^{2}+10xy+3y^{2}-2x-14y-13=0$ teńlemesiniń tipin anıqlań. &  \\
\hline
10. & $x^{2}-4y^{2}+6x+5=0$ giperbolanıń kanonikalıq teńlemesin dúziń. &  \\
\hline
\end{tabular}

\vspace{1cm}

\begin{tabular}{lll}
Tuwrı juwaplar sanı: \underline{\hspace{1.5cm}} & 
Bahası: \underline{\hspace{1.5cm}} & 
Imtixan alıwshınıń qolı: \underline{\hspace{2cm}} \\
\end{tabular}

\egroup

\newpage


\textbf{60-variant}\\

\bgroup
\def\arraystretch{1.6} % 1 is the default, change whatever you need

\begin{tabular}{|m{5.7cm}|m{9.5cm}|}
\hline
Familiyası hám atı & \\
\hline
Fakulteti  & \\
\hline
Toparı hám tálim baǵdarı  & \\
\hline
\end{tabular}

\vspace{1cm}

\begin{tabular}{|m{0.7cm}|m{10cm}|m{4cm}|}
\hline
№ & Soraw & Juwap \\
\hline
1. & $A_1x+B_1y+C_1z+D_1=0$ hám $Ax_2+By_2+Cz_2+D_2=0$ tegislikleri ústpe-úst túsiwi shárti? &  \\
\hline
2. & Eki vektordıń skalyar kóbeymesiniń formulası? &  \\
\hline
3. & $A_1x+B_1y+C_1z+D_1=0$ hám $Ax_2+By_2+Cz_2+D_2=0$ tegislikleri parallel bolıwı shárti &  \\
\hline
4. & $Ax+C=0$ tuwrı sızıqtıń grafigi koordinata kósherlerine salıstırǵanda qanday jaylasqan? &  \\
\hline
5. & $M_{1}M_{2}$ kesindiniń ortasınıń koordinatalarınıń tabıń, eger $M_{1} (2, 3), M_{2} (4, 7)$ bolsa. &  \\
\hline
6. & $x+y-3=0$ hám $2x+3y-8=0$ tuwrıları óz-ara qanday jaylasqan? &  \\
\hline
7. & $x^{2}+y^{2}-2x+4y-20=0$ sheńberdiń $C$ orayın hám $R$ radiusın tabıń. &  \\
\hline
8. & $(x+1)^{2}+(y-2) ^{2}+(z+2) ^{2}=49$ sferanıń orayınıń koordinataların tabıń. &  \\
\hline
9. & Eger $2a=16, e=\frac{5}{4}$ bolsa, fokusı abscissa kósherinde, koordinata basına salıstırǵanda simmetriyalıq jaylasqan giperbolanıń teńlemesin dúziń. &  \\
\hline
10. & Eger $2b=24, 2 c=10$ bolsa, onda abscissa kósherinde koordinata basına salıstırǵanda simmetriyalıq jaylasqan fokuslarǵa iye, ellipstiń teńlemesin dúziń. &  \\
\hline
\end{tabular}

\vspace{1cm}

\begin{tabular}{lll}
Tuwrı juwaplar sanı: \underline{\hspace{1.5cm}} & 
Bahası: \underline{\hspace{1.5cm}} & 
Imtixan alıwshınıń qolı: \underline{\hspace{2cm}} \\
\end{tabular}

\egroup

\newpage


\textbf{61-variant}\\

\bgroup
\def\arraystretch{1.6} % 1 is the default, change whatever you need

\begin{tabular}{|m{5.7cm}|m{9.5cm}|}
\hline
Familiyası hám atı & \\
\hline
Fakulteti  & \\
\hline
Toparı hám tálim baǵdarı  & \\
\hline
\end{tabular}

\vspace{1cm}

\begin{tabular}{|m{0.7cm}|m{10cm}|m{4cm}|}
\hline
№ & Soraw & Juwap \\
\hline
1. & Eki vektor qashan kollinear dep ataladı? &  \\
\hline
2. & Tuwrı múyeshli koordinatalar sisteması dep nege aytamız? &  \\
\hline
3. & $OXY$ tegisliginiń teńlemesi? &  \\
\hline
4. & Giperbolanıń kanonikalıq teńlemesi? &  \\
\hline
5. & $M_{1} (12;-1)$ hám $M_{2} (0;4)$ noqatlardıń arasındaǵı aralıqtı tabıń. &  \\
\hline
6. & $x+y=0$ teńlemesi menen berilgen tuwrı sızıqtıń múyeshlik koefficientin anıqlań. &  \\
\hline
7. & Orayı $C (-1;2)$ noqatında, $A (-2;6 )$ noqatınan ótetuǵın sheńberdiń teńlemesin dúziń. &  \\
\hline
8. & $x+2=0$ keńislik qanday geometriyalıq betlikti anıqlaydı? &  \\
\hline
9. & $\frac{x^{2}}{225}-\frac{y^{2}}{64}=-1$ giperbola fokusınıń koordinatalarınıń tabıń. &  \\
\hline
10. & $9x^{2}+25y^{2}=225$ ellipsi berilgen, ellipstiń fokusların, ekscentrisitetin tabıń. &  \\
\hline
\end{tabular}

\vspace{1cm}

\begin{tabular}{lll}
Tuwrı juwaplar sanı: \underline{\hspace{1.5cm}} & 
Bahası: \underline{\hspace{1.5cm}} & 
Imtixan alıwshınıń qolı: \underline{\hspace{2cm}} \\
\end{tabular}

\egroup

\newpage


\textbf{62-variant}\\

\bgroup
\def\arraystretch{1.6} % 1 is the default, change whatever you need

\begin{tabular}{|m{5.7cm}|m{9.5cm}|}
\hline
Familiyası hám atı & \\
\hline
Fakulteti  & \\
\hline
Toparı hám tálim baǵdarı  & \\
\hline
\end{tabular}

\vspace{1cm}

\begin{tabular}{|m{0.7cm}|m{10cm}|m{4cm}|}
\hline
№ & Soraw & Juwap \\
\hline
1. & Eki vektordıń vektor kóbeymesiniń uzınlıǵın tabıw formulası? &  \\
\hline
2. & Tegislikdegi qálegen noqatınan berilgen eki noqatqa shekemgi bolǵan aralıqlardıń ayırmasınıń modulı ózgermeytuǵın bolǵan noqatlardıń geometriyalıq ornı ne dep ataladı? &  \\
\hline
3. & Eki tuwrı sızıq arasındaǵı múyeshti tabıw formulası? &  \\
\hline
4. & $\frac{x^2}{a^2}-\frac{y^2}{b^2}=1$ giperbolanıń $(x_0;y_0)$ noqatındaǵı urınbasınıń teńlemesin kórsetiń. &  \\
\hline
5. & $A (-1;0;1),\ B (1;-1;0)$ noqatları berilgen. $\overline{BA}$ vektorın tabıń. &  \\
\hline
6. & $2x+3y+4=0$ tuwrısına parallel hám $M_{0} (2;1)$ noqattan ótetuǵın tuwrınıń teńlemesin dúziń. &  \\
\hline
7. & $x+y-12=0$ tuwrısı $x^{2}+y^{2}-2y=0$ sheńberge salıstırǵanda qanday jaylasqan? &  \\
\hline
8. & $\left| \overline{a} \right|=8, \left| \overline{b} \right|=5, \alpha=60^{0}$ bolsa, $( \overline{a}\overline{b} )$ ni tabıń. &  \\
\hline
9. & $2x+3y-6=0$ tuwrınıń teńlemesin kesindilerde berilgen teńleme túrinde kórsetiń. &  \\
\hline
10. & $\overline{a}=\left\{ 4,-2,-4 \right\}$ hám $\overline{b}=\left\{ 6,-3, 2 \right\}$ vektorları berilgen, $(\overline{a}-\overline{b}) ^{2}$-? &  \\
\hline
\end{tabular}

\vspace{1cm}

\begin{tabular}{lll}
Tuwrı juwaplar sanı: \underline{\hspace{1.5cm}} & 
Bahası: \underline{\hspace{1.5cm}} & 
Imtixan alıwshınıń qolı: \underline{\hspace{2cm}} \\
\end{tabular}

\egroup

\newpage


\textbf{63-variant}\\

\bgroup
\def\arraystretch{1.6} % 1 is the default, change whatever you need

\begin{tabular}{|m{5.7cm}|m{9.5cm}|}
\hline
Familiyası hám atı & \\
\hline
Fakulteti  & \\
\hline
Toparı hám tálim baǵdarı  & \\
\hline
\end{tabular}

\vspace{1cm}

\begin{tabular}{|m{0.7cm}|m{10cm}|m{4cm}|}
\hline
№ & Soraw & Juwap \\
\hline
1. & Vektorlardıń kósherdegi proekciyasınıń formulası? &  \\
\hline
2. & $Ax+By+D=0$ teńlemesi arqalı ... tegisliktiń teńlemesi berilgen? &  \\
\hline
3. & $\frac{x^2}{a^2}+\frac{y^2}{b^2}=1$ ellipstiń $(x_0;y_0)$ noqatındaǵı urınbasınıń teńlemesin tabıń. &  \\
\hline
4. & Vektorlardı qosıw koordinatalarda qanday formula menen anıqlanadı? &  \\
\hline
5. & $5x-y+7=0$ hám $3x+2y=0$ tuwrıları arasındaǵı múyeshni tabıń. &  \\
\hline
6. & $\overline{a}=\left\{ 2, 1, 0 \right\}$ hám $\overline{b}=\left\{ 1, 0,-1 \right\}$ bolsa, $\overline{a}-\overline{b}$ ni tabıń. &  \\
\hline
7. & Koordinatalar kósherleri hám $ 3x+4y-12=0 $ tuwrı sızıǵı menen shegaralanǵan úshmúyeshliktiń maydanın tabıń. &  \\
\hline
8. & $x-2y+1=0$ teńlemesi menen berilgen tuwrınıń normal túrdegi teńlemesin kórsetiń. &  \\
\hline
9. & $3x-y+5=0$, $x+3y-4=0$ tuwrı sızıqları arasındaǵı múyeshti tabıń. &  \\
\hline
10. & $\overline{a}=\{5,-6, 1 \}, \overline{b}=\{-4, 3, 0 \} $, $\overline{c}=\left\{ 5,-8, 10 \right\}$ vektorları berilgen. $2{\overline{a}}^{2}+4{\overline{b}}^{2}-5{\overline{c}}^{2}$ ańlatpasınıń mánisin tabıń. &  \\
\hline
\end{tabular}

\vspace{1cm}

\begin{tabular}{lll}
Tuwrı juwaplar sanı: \underline{\hspace{1.5cm}} & 
Bahası: \underline{\hspace{1.5cm}} & 
Imtixan alıwshınıń qolı: \underline{\hspace{2cm}} \\
\end{tabular}

\egroup

\newpage


\textbf{64-variant}\\

\bgroup
\def\arraystretch{1.6} % 1 is the default, change whatever you need

\begin{tabular}{|m{5.7cm}|m{9.5cm}|}
\hline
Familiyası hám atı & \\
\hline
Fakulteti  & \\
\hline
Toparı hám tálim baǵdarı  & \\
\hline
\end{tabular}

\vspace{1cm}

\begin{tabular}{|m{0.7cm}|m{10cm}|m{4cm}|}
\hline
№ & Soraw & Juwap \\
\hline
1. & $OY$ kósheriniń teńlemesi? &  \\
\hline
2. & Egerde $a=\{ x_1; y_1; z_1\}, b=\{ x_2, y_2; z_2\}$ bolsa, vektor kóbeymeniń koordinatalarda ańlatılıwı qanday boladı? &  \\
\hline
3. & $A_1x+B_1y+C_1z+D_1=0$ hám $Ax_2+By_2+Cz_2+D_2=0$ tegislikleri perpendikulyar bolıwı shárti &  \\
\hline
4. & Úsh vektordıń aralas kóbeymesi ushın $(abc)=0$ teńligi orınlı bolsa ne dep ataladı? &  \\
\hline
5. & $(2, 3)$ hám $(4, 3)$ noqatlarınan ótiwshi tuwrı sızıqtıń teńlemesin dúziń. &  \\
\hline
6. & $x^{2}+y^{2}-2x+4y=0$ sheńberdiń teńlemesin kanonikalıq túrdegi teńlemege alıp keliń. &  \\
\hline
7. & $A(4, 3), B(7, 7)$ noqatları arasındaǵı aralıqtı tabıń. &  \\
\hline
8. & $3x^{2}+10xy+3y^{2}-2x-14y-13=0$ teńlemesiniń tipin anıqlań. &  \\
\hline
9. & $x^{2}-4y^{2}+6x+5=0$ giperbolanıń kanonikalıq teńlemesin dúziń. &  \\
\hline
10. & $M_{1}M_{2}$ kesindiniń ortasınıń koordinatalarınıń tabıń, eger $M_{1} (2, 3), M_{2} (4, 7)$ bolsa. &  \\
\hline
\end{tabular}

\vspace{1cm}

\begin{tabular}{lll}
Tuwrı juwaplar sanı: \underline{\hspace{1.5cm}} & 
Bahası: \underline{\hspace{1.5cm}} & 
Imtixan alıwshınıń qolı: \underline{\hspace{2cm}} \\
\end{tabular}

\egroup

\newpage


\textbf{65-variant}\\

\bgroup
\def\arraystretch{1.6} % 1 is the default, change whatever you need

\begin{tabular}{|m{5.7cm}|m{9.5cm}|}
\hline
Familiyası hám atı & \\
\hline
Fakulteti  & \\
\hline
Toparı hám tálim baǵdarı  & \\
\hline
\end{tabular}

\vspace{1cm}

\begin{tabular}{|m{0.7cm}|m{10cm}|m{4cm}|}
\hline
№ & Soraw & Juwap \\
\hline
1. & $A_1x+B_1y+C_1z+D_1=0$ hám $Ax_2+By_2+Cz_2+D_2=0$ tegislikleri ústpe-úst túsiwi shárti? &  \\
\hline
2. & Eki vektordıń skalyar kóbeymesiniń formulası? &  \\
\hline
3. & $A_1x+B_1y+C_1z+D_1=0$ hám $Ax_2+By_2+Cz_2+D_2=0$ tegislikleri parallel bolıwı shárti &  \\
\hline
4. & $Ax+C=0$ tuwrı sızıqtıń grafigi koordinata kósherlerine salıstırǵanda qanday jaylasqan? &  \\
\hline
5. & $x+y-3=0$ hám $2x+3y-8=0$ tuwrıları óz-ara qanday jaylasqan? &  \\
\hline
6. & $x^{2}+y^{2}-2x+4y-20=0$ sheńberdiń $C$ orayın hám $R$ radiusın tabıń. &  \\
\hline
7. & $(x+1)^{2}+(y-2) ^{2}+(z+2) ^{2}=49$ sferanıń orayınıń koordinataların tabıń. &  \\
\hline
8. & Eger $2a=16, e=\frac{5}{4}$ bolsa, fokusı abscissa kósherinde, koordinata basına salıstırǵanda simmetriyalıq jaylasqan giperbolanıń teńlemesin dúziń. &  \\
\hline
9. & Eger $2b=24, 2 c=10$ bolsa, onda abscissa kósherinde koordinata basına salıstırǵanda simmetriyalıq jaylasqan fokuslarǵa iye, ellipstiń teńlemesin dúziń. &  \\
\hline
10. & $M_{1} (12;-1)$ hám $M_{2} (0;4)$ noqatlardıń arasındaǵı aralıqtı tabıń. &  \\
\hline
\end{tabular}

\vspace{1cm}

\begin{tabular}{lll}
Tuwrı juwaplar sanı: \underline{\hspace{1.5cm}} & 
Bahası: \underline{\hspace{1.5cm}} & 
Imtixan alıwshınıń qolı: \underline{\hspace{2cm}} \\
\end{tabular}

\egroup

\newpage


\textbf{66-variant}\\

\bgroup
\def\arraystretch{1.6} % 1 is the default, change whatever you need

\begin{tabular}{|m{5.7cm}|m{9.5cm}|}
\hline
Familiyası hám atı & \\
\hline
Fakulteti  & \\
\hline
Toparı hám tálim baǵdarı  & \\
\hline
\end{tabular}

\vspace{1cm}

\begin{tabular}{|m{0.7cm}|m{10cm}|m{4cm}|}
\hline
№ & Soraw & Juwap \\
\hline
1. & Eki vektor qashan kollinear dep ataladı? &  \\
\hline
2. & Tuwrı múyeshli koordinatalar sisteması dep nege aytamız? &  \\
\hline
3. & $OXY$ tegisliginiń teńlemesi? &  \\
\hline
4. & Giperbolanıń kanonikalıq teńlemesi? &  \\
\hline
5. & $x+y=0$ teńlemesi menen berilgen tuwrı sızıqtıń múyeshlik koefficientin anıqlań. &  \\
\hline
6. & Orayı $C (-1;2)$ noqatında, $A (-2;6 )$ noqatınan ótetuǵın sheńberdiń teńlemesin dúziń. &  \\
\hline
7. & $x+2=0$ keńislik qanday geometriyalıq betlikti anıqlaydı? &  \\
\hline
8. & $\frac{x^{2}}{225}-\frac{y^{2}}{64}=-1$ giperbola fokusınıń koordinatalarınıń tabıń. &  \\
\hline
9. & $9x^{2}+25y^{2}=225$ ellipsi berilgen, ellipstiń fokusların, ekscentrisitetin tabıń. &  \\
\hline
10. & $A (-1;0;1),\ B (1;-1;0)$ noqatları berilgen. $\overline{BA}$ vektorın tabıń. &  \\
\hline
\end{tabular}

\vspace{1cm}

\begin{tabular}{lll}
Tuwrı juwaplar sanı: \underline{\hspace{1.5cm}} & 
Bahası: \underline{\hspace{1.5cm}} & 
Imtixan alıwshınıń qolı: \underline{\hspace{2cm}} \\
\end{tabular}

\egroup

\newpage


\textbf{67-variant}\\

\bgroup
\def\arraystretch{1.6} % 1 is the default, change whatever you need

\begin{tabular}{|m{5.7cm}|m{9.5cm}|}
\hline
Familiyası hám atı & \\
\hline
Fakulteti  & \\
\hline
Toparı hám tálim baǵdarı  & \\
\hline
\end{tabular}

\vspace{1cm}

\begin{tabular}{|m{0.7cm}|m{10cm}|m{4cm}|}
\hline
№ & Soraw & Juwap \\
\hline
1. & Eki vektordıń vektor kóbeymesiniń uzınlıǵın tabıw formulası? &  \\
\hline
2. & Tegislikdegi qálegen noqatınan berilgen eki noqatqa shekemgi bolǵan aralıqlardıń ayırmasınıń modulı ózgermeytuǵın bolǵan noqatlardıń geometriyalıq ornı ne dep ataladı? &  \\
\hline
3. & Eki tuwrı sızıq arasındaǵı múyeshti tabıw formulası? &  \\
\hline
4. & $\frac{x^2}{a^2}-\frac{y^2}{b^2}=1$ giperbolanıń $(x_0;y_0)$ noqatındaǵı urınbasınıń teńlemesin kórsetiń. &  \\
\hline
5. & $2x+3y+4=0$ tuwrısına parallel hám $M_{0} (2;1)$ noqattan ótetuǵın tuwrınıń teńlemesin dúziń. &  \\
\hline
6. & $x+y-12=0$ tuwrısı $x^{2}+y^{2}-2y=0$ sheńberge salıstırǵanda qanday jaylasqan? &  \\
\hline
7. & $\left| \overline{a} \right|=8, \left| \overline{b} \right|=5, \alpha=60^{0}$ bolsa, $( \overline{a}\overline{b} )$ ni tabıń. &  \\
\hline
8. & $2x+3y-6=0$ tuwrınıń teńlemesin kesindilerde berilgen teńleme túrinde kórsetiń. &  \\
\hline
9. & $\overline{a}=\left\{ 4,-2,-4 \right\}$ hám $\overline{b}=\left\{ 6,-3, 2 \right\}$ vektorları berilgen, $(\overline{a}-\overline{b}) ^{2}$-? &  \\
\hline
10. & $5x-y+7=0$ hám $3x+2y=0$ tuwrıları arasındaǵı múyeshni tabıń. &  \\
\hline
\end{tabular}

\vspace{1cm}

\begin{tabular}{lll}
Tuwrı juwaplar sanı: \underline{\hspace{1.5cm}} & 
Bahası: \underline{\hspace{1.5cm}} & 
Imtixan alıwshınıń qolı: \underline{\hspace{2cm}} \\
\end{tabular}

\egroup

\newpage


\textbf{68-variant}\\

\bgroup
\def\arraystretch{1.6} % 1 is the default, change whatever you need

\begin{tabular}{|m{5.7cm}|m{9.5cm}|}
\hline
Familiyası hám atı & \\
\hline
Fakulteti  & \\
\hline
Toparı hám tálim baǵdarı  & \\
\hline
\end{tabular}

\vspace{1cm}

\begin{tabular}{|m{0.7cm}|m{10cm}|m{4cm}|}
\hline
№ & Soraw & Juwap \\
\hline
1. & Vektorlardıń kósherdegi proekciyasınıń formulası? &  \\
\hline
2. & $Ax+By+D=0$ teńlemesi arqalı ... tegisliktiń teńlemesi berilgen? &  \\
\hline
3. & $\frac{x^2}{a^2}+\frac{y^2}{b^2}=1$ ellipstiń $(x_0;y_0)$ noqatındaǵı urınbasınıń teńlemesin tabıń. &  \\
\hline
4. & Vektorlardı qosıw koordinatalarda qanday formula menen anıqlanadı? &  \\
\hline
5. & $\overline{a}=\left\{ 2, 1, 0 \right\}$ hám $\overline{b}=\left\{ 1, 0,-1 \right\}$ bolsa, $\overline{a}-\overline{b}$ ni tabıń. &  \\
\hline
6. & Koordinatalar kósherleri hám $ 3x+4y-12=0 $ tuwrı sızıǵı menen shegaralanǵan úshmúyeshliktiń maydanın tabıń. &  \\
\hline
7. & $x-2y+1=0$ teńlemesi menen berilgen tuwrınıń normal túrdegi teńlemesin kórsetiń. &  \\
\hline
8. & $3x-y+5=0$, $x+3y-4=0$ tuwrı sızıqları arasındaǵı múyeshti tabıń. &  \\
\hline
9. & $\overline{a}=\{5,-6, 1 \}, \overline{b}=\{-4, 3, 0 \} $, $\overline{c}=\left\{ 5,-8, 10 \right\}$ vektorları berilgen. $2{\overline{a}}^{2}+4{\overline{b}}^{2}-5{\overline{c}}^{2}$ ańlatpasınıń mánisin tabıń. &  \\
\hline
10. & $(2, 3)$ hám $(4, 3)$ noqatlarınan ótiwshi tuwrı sızıqtıń teńlemesin dúziń. &  \\
\hline
\end{tabular}

\vspace{1cm}

\begin{tabular}{lll}
Tuwrı juwaplar sanı: \underline{\hspace{1.5cm}} & 
Bahası: \underline{\hspace{1.5cm}} & 
Imtixan alıwshınıń qolı: \underline{\hspace{2cm}} \\
\end{tabular}

\egroup

\newpage


\textbf{69-variant}\\

\bgroup
\def\arraystretch{1.6} % 1 is the default, change whatever you need

\begin{tabular}{|m{5.7cm}|m{9.5cm}|}
\hline
Familiyası hám atı & \\
\hline
Fakulteti  & \\
\hline
Toparı hám tálim baǵdarı  & \\
\hline
\end{tabular}

\vspace{1cm}

\begin{tabular}{|m{0.7cm}|m{10cm}|m{4cm}|}
\hline
№ & Soraw & Juwap \\
\hline
1. & $OY$ kósheriniń teńlemesi? &  \\
\hline
2. & Egerde $a=\{ x_1; y_1; z_1\}, b=\{ x_2, y_2; z_2\}$ bolsa, vektor kóbeymeniń koordinatalarda ańlatılıwı qanday boladı? &  \\
\hline
3. & $A_1x+B_1y+C_1z+D_1=0$ hám $Ax_2+By_2+Cz_2+D_2=0$ tegislikleri perpendikulyar bolıwı shárti &  \\
\hline
4. & Úsh vektordıń aralas kóbeymesi ushın $(abc)=0$ teńligi orınlı bolsa ne dep ataladı? &  \\
\hline
5. & $x^{2}+y^{2}-2x+4y=0$ sheńberdiń teńlemesin kanonikalıq túrdegi teńlemege alıp keliń. &  \\
\hline
6. & $A(4, 3), B(7, 7)$ noqatları arasındaǵı aralıqtı tabıń. &  \\
\hline
7. & $3x^{2}+10xy+3y^{2}-2x-14y-13=0$ teńlemesiniń tipin anıqlań. &  \\
\hline
8. & $x^{2}-4y^{2}+6x+5=0$ giperbolanıń kanonikalıq teńlemesin dúziń. &  \\
\hline
9. & $M_{1}M_{2}$ kesindiniń ortasınıń koordinatalarınıń tabıń, eger $M_{1} (2, 3), M_{2} (4, 7)$ bolsa. &  \\
\hline
10. & $x+y-3=0$ hám $2x+3y-8=0$ tuwrıları óz-ara qanday jaylasqan? &  \\
\hline
\end{tabular}

\vspace{1cm}

\begin{tabular}{lll}
Tuwrı juwaplar sanı: \underline{\hspace{1.5cm}} & 
Bahası: \underline{\hspace{1.5cm}} & 
Imtixan alıwshınıń qolı: \underline{\hspace{2cm}} \\
\end{tabular}

\egroup

\newpage


\textbf{70-variant}\\

\bgroup
\def\arraystretch{1.6} % 1 is the default, change whatever you need

\begin{tabular}{|m{5.7cm}|m{9.5cm}|}
\hline
Familiyası hám atı & \\
\hline
Fakulteti  & \\
\hline
Toparı hám tálim baǵdarı  & \\
\hline
\end{tabular}

\vspace{1cm}

\begin{tabular}{|m{0.7cm}|m{10cm}|m{4cm}|}
\hline
№ & Soraw & Juwap \\
\hline
1. & $A_1x+B_1y+C_1z+D_1=0$ hám $Ax_2+By_2+Cz_2+D_2=0$ tegislikleri ústpe-úst túsiwi shárti? &  \\
\hline
2. & Eki vektordıń skalyar kóbeymesiniń formulası? &  \\
\hline
3. & $A_1x+B_1y+C_1z+D_1=0$ hám $Ax_2+By_2+Cz_2+D_2=0$ tegislikleri parallel bolıwı shárti &  \\
\hline
4. & $Ax+C=0$ tuwrı sızıqtıń grafigi koordinata kósherlerine salıstırǵanda qanday jaylasqan? &  \\
\hline
5. & $x^{2}+y^{2}-2x+4y-20=0$ sheńberdiń $C$ orayın hám $R$ radiusın tabıń. &  \\
\hline
6. & $(x+1)^{2}+(y-2) ^{2}+(z+2) ^{2}=49$ sferanıń orayınıń koordinataların tabıń. &  \\
\hline
7. & Eger $2a=16, e=\frac{5}{4}$ bolsa, fokusı abscissa kósherinde, koordinata basına salıstırǵanda simmetriyalıq jaylasqan giperbolanıń teńlemesin dúziń. &  \\
\hline
8. & Eger $2b=24, 2 c=10$ bolsa, onda abscissa kósherinde koordinata basına salıstırǵanda simmetriyalıq jaylasqan fokuslarǵa iye, ellipstiń teńlemesin dúziń. &  \\
\hline
9. & $M_{1} (12;-1)$ hám $M_{2} (0;4)$ noqatlardıń arasındaǵı aralıqtı tabıń. &  \\
\hline
10. & $x+y=0$ teńlemesi menen berilgen tuwrı sızıqtıń múyeshlik koefficientin anıqlań. &  \\
\hline
\end{tabular}

\vspace{1cm}

\begin{tabular}{lll}
Tuwrı juwaplar sanı: \underline{\hspace{1.5cm}} & 
Bahası: \underline{\hspace{1.5cm}} & 
Imtixan alıwshınıń qolı: \underline{\hspace{2cm}} \\
\end{tabular}

\egroup

\newpage


\textbf{71-variant}\\

\bgroup
\def\arraystretch{1.6} % 1 is the default, change whatever you need

\begin{tabular}{|m{5.7cm}|m{9.5cm}|}
\hline
Familiyası hám atı & \\
\hline
Fakulteti  & \\
\hline
Toparı hám tálim baǵdarı  & \\
\hline
\end{tabular}

\vspace{1cm}

\begin{tabular}{|m{0.7cm}|m{10cm}|m{4cm}|}
\hline
№ & Soraw & Juwap \\
\hline
1. & Eki vektor qashan kollinear dep ataladı? &  \\
\hline
2. & Tuwrı múyeshli koordinatalar sisteması dep nege aytamız? &  \\
\hline
3. & $OXY$ tegisliginiń teńlemesi? &  \\
\hline
4. & Giperbolanıń kanonikalıq teńlemesi? &  \\
\hline
5. & Orayı $C (-1;2)$ noqatında, $A (-2;6 )$ noqatınan ótetuǵın sheńberdiń teńlemesin dúziń. &  \\
\hline
6. & $x+2=0$ keńislik qanday geometriyalıq betlikti anıqlaydı? &  \\
\hline
7. & $\frac{x^{2}}{225}-\frac{y^{2}}{64}=-1$ giperbola fokusınıń koordinatalarınıń tabıń. &  \\
\hline
8. & $9x^{2}+25y^{2}=225$ ellipsi berilgen, ellipstiń fokusların, ekscentrisitetin tabıń. &  \\
\hline
9. & $A (-1;0;1),\ B (1;-1;0)$ noqatları berilgen. $\overline{BA}$ vektorın tabıń. &  \\
\hline
10. & $2x+3y+4=0$ tuwrısına parallel hám $M_{0} (2;1)$ noqattan ótetuǵın tuwrınıń teńlemesin dúziń. &  \\
\hline
\end{tabular}

\vspace{1cm}

\begin{tabular}{lll}
Tuwrı juwaplar sanı: \underline{\hspace{1.5cm}} & 
Bahası: \underline{\hspace{1.5cm}} & 
Imtixan alıwshınıń qolı: \underline{\hspace{2cm}} \\
\end{tabular}

\egroup

\newpage


\textbf{72-variant}\\

\bgroup
\def\arraystretch{1.6} % 1 is the default, change whatever you need

\begin{tabular}{|m{5.7cm}|m{9.5cm}|}
\hline
Familiyası hám atı & \\
\hline
Fakulteti  & \\
\hline
Toparı hám tálim baǵdarı  & \\
\hline
\end{tabular}

\vspace{1cm}

\begin{tabular}{|m{0.7cm}|m{10cm}|m{4cm}|}
\hline
№ & Soraw & Juwap \\
\hline
1. & Eki vektordıń vektor kóbeymesiniń uzınlıǵın tabıw formulası? &  \\
\hline
2. & Tegislikdegi qálegen noqatınan berilgen eki noqatqa shekemgi bolǵan aralıqlardıń ayırmasınıń modulı ózgermeytuǵın bolǵan noqatlardıń geometriyalıq ornı ne dep ataladı? &  \\
\hline
3. & Eki tuwrı sızıq arasındaǵı múyeshti tabıw formulası? &  \\
\hline
4. & $\frac{x^2}{a^2}-\frac{y^2}{b^2}=1$ giperbolanıń $(x_0;y_0)$ noqatındaǵı urınbasınıń teńlemesin kórsetiń. &  \\
\hline
5. & $x+y-12=0$ tuwrısı $x^{2}+y^{2}-2y=0$ sheńberge salıstırǵanda qanday jaylasqan? &  \\
\hline
6. & $\left| \overline{a} \right|=8, \left| \overline{b} \right|=5, \alpha=60^{0}$ bolsa, $( \overline{a}\overline{b} )$ ni tabıń. &  \\
\hline
7. & $2x+3y-6=0$ tuwrınıń teńlemesin kesindilerde berilgen teńleme túrinde kórsetiń. &  \\
\hline
8. & $\overline{a}=\left\{ 4,-2,-4 \right\}$ hám $\overline{b}=\left\{ 6,-3, 2 \right\}$ vektorları berilgen, $(\overline{a}-\overline{b}) ^{2}$-? &  \\
\hline
9. & $5x-y+7=0$ hám $3x+2y=0$ tuwrıları arasındaǵı múyeshni tabıń. &  \\
\hline
10. & $\overline{a}=\left\{ 2, 1, 0 \right\}$ hám $\overline{b}=\left\{ 1, 0,-1 \right\}$ bolsa, $\overline{a}-\overline{b}$ ni tabıń. &  \\
\hline
\end{tabular}

\vspace{1cm}

\begin{tabular}{lll}
Tuwrı juwaplar sanı: \underline{\hspace{1.5cm}} & 
Bahası: \underline{\hspace{1.5cm}} & 
Imtixan alıwshınıń qolı: \underline{\hspace{2cm}} \\
\end{tabular}

\egroup

\newpage


\textbf{73-variant}\\

\bgroup
\def\arraystretch{1.6} % 1 is the default, change whatever you need

\begin{tabular}{|m{5.7cm}|m{9.5cm}|}
\hline
Familiyası hám atı & \\
\hline
Fakulteti  & \\
\hline
Toparı hám tálim baǵdarı  & \\
\hline
\end{tabular}

\vspace{1cm}

\begin{tabular}{|m{0.7cm}|m{10cm}|m{4cm}|}
\hline
№ & Soraw & Juwap \\
\hline
1. & Vektorlardıń kósherdegi proekciyasınıń formulası? &  \\
\hline
2. & $Ax+By+D=0$ teńlemesi arqalı ... tegisliktiń teńlemesi berilgen? &  \\
\hline
3. & $\frac{x^2}{a^2}+\frac{y^2}{b^2}=1$ ellipstiń $(x_0;y_0)$ noqatındaǵı urınbasınıń teńlemesin tabıń. &  \\
\hline
4. & Vektorlardı qosıw koordinatalarda qanday formula menen anıqlanadı? &  \\
\hline
5. & Koordinatalar kósherleri hám $ 3x+4y-12=0 $ tuwrı sızıǵı menen shegaralanǵan úshmúyeshliktiń maydanın tabıń. &  \\
\hline
6. & $x-2y+1=0$ teńlemesi menen berilgen tuwrınıń normal túrdegi teńlemesin kórsetiń. &  \\
\hline
7. & $3x-y+5=0$, $x+3y-4=0$ tuwrı sızıqları arasındaǵı múyeshti tabıń. &  \\
\hline
8. & $\overline{a}=\{5,-6, 1 \}, \overline{b}=\{-4, 3, 0 \} $, $\overline{c}=\left\{ 5,-8, 10 \right\}$ vektorları berilgen. $2{\overline{a}}^{2}+4{\overline{b}}^{2}-5{\overline{c}}^{2}$ ańlatpasınıń mánisin tabıń. &  \\
\hline
9. & $(2, 3)$ hám $(4, 3)$ noqatlarınan ótiwshi tuwrı sızıqtıń teńlemesin dúziń. &  \\
\hline
10. & $x^{2}+y^{2}-2x+4y=0$ sheńberdiń teńlemesin kanonikalıq túrdegi teńlemege alıp keliń. &  \\
\hline
\end{tabular}

\vspace{1cm}

\begin{tabular}{lll}
Tuwrı juwaplar sanı: \underline{\hspace{1.5cm}} & 
Bahası: \underline{\hspace{1.5cm}} & 
Imtixan alıwshınıń qolı: \underline{\hspace{2cm}} \\
\end{tabular}

\egroup

\newpage


\textbf{74-variant}\\

\bgroup
\def\arraystretch{1.6} % 1 is the default, change whatever you need

\begin{tabular}{|m{5.7cm}|m{9.5cm}|}
\hline
Familiyası hám atı & \\
\hline
Fakulteti  & \\
\hline
Toparı hám tálim baǵdarı  & \\
\hline
\end{tabular}

\vspace{1cm}

\begin{tabular}{|m{0.7cm}|m{10cm}|m{4cm}|}
\hline
№ & Soraw & Juwap \\
\hline
1. & $OY$ kósheriniń teńlemesi? &  \\
\hline
2. & Egerde $a=\{ x_1; y_1; z_1\}, b=\{ x_2, y_2; z_2\}$ bolsa, vektor kóbeymeniń koordinatalarda ańlatılıwı qanday boladı? &  \\
\hline
3. & $A_1x+B_1y+C_1z+D_1=0$ hám $Ax_2+By_2+Cz_2+D_2=0$ tegislikleri perpendikulyar bolıwı shárti &  \\
\hline
4. & Úsh vektordıń aralas kóbeymesi ushın $(abc)=0$ teńligi orınlı bolsa ne dep ataladı? &  \\
\hline
5. & $A(4, 3), B(7, 7)$ noqatları arasındaǵı aralıqtı tabıń. &  \\
\hline
6. & $3x^{2}+10xy+3y^{2}-2x-14y-13=0$ teńlemesiniń tipin anıqlań. &  \\
\hline
7. & $x^{2}-4y^{2}+6x+5=0$ giperbolanıń kanonikalıq teńlemesin dúziń. &  \\
\hline
8. & $M_{1}M_{2}$ kesindiniń ortasınıń koordinatalarınıń tabıń, eger $M_{1} (2, 3), M_{2} (4, 7)$ bolsa. &  \\
\hline
9. & $x+y-3=0$ hám $2x+3y-8=0$ tuwrıları óz-ara qanday jaylasqan? &  \\
\hline
10. & $x^{2}+y^{2}-2x+4y-20=0$ sheńberdiń $C$ orayın hám $R$ radiusın tabıń. &  \\
\hline
\end{tabular}

\vspace{1cm}

\begin{tabular}{lll}
Tuwrı juwaplar sanı: \underline{\hspace{1.5cm}} & 
Bahası: \underline{\hspace{1.5cm}} & 
Imtixan alıwshınıń qolı: \underline{\hspace{2cm}} \\
\end{tabular}

\egroup

\newpage


\textbf{75-variant}\\

\bgroup
\def\arraystretch{1.6} % 1 is the default, change whatever you need

\begin{tabular}{|m{5.7cm}|m{9.5cm}|}
\hline
Familiyası hám atı & \\
\hline
Fakulteti  & \\
\hline
Toparı hám tálim baǵdarı  & \\
\hline
\end{tabular}

\vspace{1cm}

\begin{tabular}{|m{0.7cm}|m{10cm}|m{4cm}|}
\hline
№ & Soraw & Juwap \\
\hline
1. & $A_1x+B_1y+C_1z+D_1=0$ hám $Ax_2+By_2+Cz_2+D_2=0$ tegislikleri ústpe-úst túsiwi shárti? &  \\
\hline
2. & Eki vektordıń skalyar kóbeymesiniń formulası? &  \\
\hline
3. & $A_1x+B_1y+C_1z+D_1=0$ hám $Ax_2+By_2+Cz_2+D_2=0$ tegislikleri parallel bolıwı shárti &  \\
\hline
4. & $Ax+C=0$ tuwrı sızıqtıń grafigi koordinata kósherlerine salıstırǵanda qanday jaylasqan? &  \\
\hline
5. & $(x+1)^{2}+(y-2) ^{2}+(z+2) ^{2}=49$ sferanıń orayınıń koordinataların tabıń. &  \\
\hline
6. & Eger $2a=16, e=\frac{5}{4}$ bolsa, fokusı abscissa kósherinde, koordinata basına salıstırǵanda simmetriyalıq jaylasqan giperbolanıń teńlemesin dúziń. &  \\
\hline
7. & Eger $2b=24, 2 c=10$ bolsa, onda abscissa kósherinde koordinata basına salıstırǵanda simmetriyalıq jaylasqan fokuslarǵa iye, ellipstiń teńlemesin dúziń. &  \\
\hline
8. & $M_{1} (12;-1)$ hám $M_{2} (0;4)$ noqatlardıń arasındaǵı aralıqtı tabıń. &  \\
\hline
9. & $x+y=0$ teńlemesi menen berilgen tuwrı sızıqtıń múyeshlik koefficientin anıqlań. &  \\
\hline
10. & Orayı $C (-1;2)$ noqatında, $A (-2;6 )$ noqatınan ótetuǵın sheńberdiń teńlemesin dúziń. &  \\
\hline
\end{tabular}

\vspace{1cm}

\begin{tabular}{lll}
Tuwrı juwaplar sanı: \underline{\hspace{1.5cm}} & 
Bahası: \underline{\hspace{1.5cm}} & 
Imtixan alıwshınıń qolı: \underline{\hspace{2cm}} \\
\end{tabular}

\egroup

\newpage


\textbf{76-variant}\\

\bgroup
\def\arraystretch{1.6} % 1 is the default, change whatever you need

\begin{tabular}{|m{5.7cm}|m{9.5cm}|}
\hline
Familiyası hám atı & \\
\hline
Fakulteti  & \\
\hline
Toparı hám tálim baǵdarı  & \\
\hline
\end{tabular}

\vspace{1cm}

\begin{tabular}{|m{0.7cm}|m{10cm}|m{4cm}|}
\hline
№ & Soraw & Juwap \\
\hline
1. & Eki vektor qashan kollinear dep ataladı? &  \\
\hline
2. & Tuwrı múyeshli koordinatalar sisteması dep nege aytamız? &  \\
\hline
3. & $OXY$ tegisliginiń teńlemesi? &  \\
\hline
4. & Giperbolanıń kanonikalıq teńlemesi? &  \\
\hline
5. & $x+2=0$ keńislik qanday geometriyalıq betlikti anıqlaydı? &  \\
\hline
6. & $\frac{x^{2}}{225}-\frac{y^{2}}{64}=-1$ giperbola fokusınıń koordinatalarınıń tabıń. &  \\
\hline
7. & $9x^{2}+25y^{2}=225$ ellipsi berilgen, ellipstiń fokusların, ekscentrisitetin tabıń. &  \\
\hline
8. & $A (-1;0;1),\ B (1;-1;0)$ noqatları berilgen. $\overline{BA}$ vektorın tabıń. &  \\
\hline
9. & $2x+3y+4=0$ tuwrısına parallel hám $M_{0} (2;1)$ noqattan ótetuǵın tuwrınıń teńlemesin dúziń. &  \\
\hline
10. & $x+y-12=0$ tuwrısı $x^{2}+y^{2}-2y=0$ sheńberge salıstırǵanda qanday jaylasqan? &  \\
\hline
\end{tabular}

\vspace{1cm}

\begin{tabular}{lll}
Tuwrı juwaplar sanı: \underline{\hspace{1.5cm}} & 
Bahası: \underline{\hspace{1.5cm}} & 
Imtixan alıwshınıń qolı: \underline{\hspace{2cm}} \\
\end{tabular}

\egroup

\newpage


\textbf{77-variant}\\

\bgroup
\def\arraystretch{1.6} % 1 is the default, change whatever you need

\begin{tabular}{|m{5.7cm}|m{9.5cm}|}
\hline
Familiyası hám atı & \\
\hline
Fakulteti  & \\
\hline
Toparı hám tálim baǵdarı  & \\
\hline
\end{tabular}

\vspace{1cm}

\begin{tabular}{|m{0.7cm}|m{10cm}|m{4cm}|}
\hline
№ & Soraw & Juwap \\
\hline
1. & Eki vektordıń vektor kóbeymesiniń uzınlıǵın tabıw formulası? &  \\
\hline
2. & Tegislikdegi qálegen noqatınan berilgen eki noqatqa shekemgi bolǵan aralıqlardıń ayırmasınıń modulı ózgermeytuǵın bolǵan noqatlardıń geometriyalıq ornı ne dep ataladı? &  \\
\hline
3. & Eki tuwrı sızıq arasındaǵı múyeshti tabıw formulası? &  \\
\hline
4. & $\frac{x^2}{a^2}-\frac{y^2}{b^2}=1$ giperbolanıń $(x_0;y_0)$ noqatındaǵı urınbasınıń teńlemesin kórsetiń. &  \\
\hline
5. & $\left| \overline{a} \right|=8, \left| \overline{b} \right|=5, \alpha=60^{0}$ bolsa, $( \overline{a}\overline{b} )$ ni tabıń. &  \\
\hline
6. & $2x+3y-6=0$ tuwrınıń teńlemesin kesindilerde berilgen teńleme túrinde kórsetiń. &  \\
\hline
7. & $\overline{a}=\left\{ 4,-2,-4 \right\}$ hám $\overline{b}=\left\{ 6,-3, 2 \right\}$ vektorları berilgen, $(\overline{a}-\overline{b}) ^{2}$-? &  \\
\hline
8. & $5x-y+7=0$ hám $3x+2y=0$ tuwrıları arasındaǵı múyeshni tabıń. &  \\
\hline
9. & $\overline{a}=\left\{ 2, 1, 0 \right\}$ hám $\overline{b}=\left\{ 1, 0,-1 \right\}$ bolsa, $\overline{a}-\overline{b}$ ni tabıń. &  \\
\hline
10. & Koordinatalar kósherleri hám $ 3x+4y-12=0 $ tuwrı sızıǵı menen shegaralanǵan úshmúyeshliktiń maydanın tabıń. &  \\
\hline
\end{tabular}

\vspace{1cm}

\begin{tabular}{lll}
Tuwrı juwaplar sanı: \underline{\hspace{1.5cm}} & 
Bahası: \underline{\hspace{1.5cm}} & 
Imtixan alıwshınıń qolı: \underline{\hspace{2cm}} \\
\end{tabular}

\egroup

\newpage


\textbf{78-variant}\\

\bgroup
\def\arraystretch{1.6} % 1 is the default, change whatever you need

\begin{tabular}{|m{5.7cm}|m{9.5cm}|}
\hline
Familiyası hám atı & \\
\hline
Fakulteti  & \\
\hline
Toparı hám tálim baǵdarı  & \\
\hline
\end{tabular}

\vspace{1cm}

\begin{tabular}{|m{0.7cm}|m{10cm}|m{4cm}|}
\hline
№ & Soraw & Juwap \\
\hline
1. & Vektorlardıń kósherdegi proekciyasınıń formulası? &  \\
\hline
2. & $Ax+By+D=0$ teńlemesi arqalı ... tegisliktiń teńlemesi berilgen? &  \\
\hline
3. & $\frac{x^2}{a^2}+\frac{y^2}{b^2}=1$ ellipstiń $(x_0;y_0)$ noqatındaǵı urınbasınıń teńlemesin tabıń. &  \\
\hline
4. & Vektorlardı qosıw koordinatalarda qanday formula menen anıqlanadı? &  \\
\hline
5. & $x-2y+1=0$ teńlemesi menen berilgen tuwrınıń normal túrdegi teńlemesin kórsetiń. &  \\
\hline
6. & $3x-y+5=0$, $x+3y-4=0$ tuwrı sızıqları arasındaǵı múyeshti tabıń. &  \\
\hline
7. & $\overline{a}=\{5,-6, 1 \}, \overline{b}=\{-4, 3, 0 \} $, $\overline{c}=\left\{ 5,-8, 10 \right\}$ vektorları berilgen. $2{\overline{a}}^{2}+4{\overline{b}}^{2}-5{\overline{c}}^{2}$ ańlatpasınıń mánisin tabıń. &  \\
\hline
8. & $(2, 3)$ hám $(4, 3)$ noqatlarınan ótiwshi tuwrı sızıqtıń teńlemesin dúziń. &  \\
\hline
9. & $x^{2}+y^{2}-2x+4y=0$ sheńberdiń teńlemesin kanonikalıq túrdegi teńlemege alıp keliń. &  \\
\hline
10. & $A(4, 3), B(7, 7)$ noqatları arasındaǵı aralıqtı tabıń. &  \\
\hline
\end{tabular}

\vspace{1cm}

\begin{tabular}{lll}
Tuwrı juwaplar sanı: \underline{\hspace{1.5cm}} & 
Bahası: \underline{\hspace{1.5cm}} & 
Imtixan alıwshınıń qolı: \underline{\hspace{2cm}} \\
\end{tabular}

\egroup

\newpage


\textbf{79-variant}\\

\bgroup
\def\arraystretch{1.6} % 1 is the default, change whatever you need

\begin{tabular}{|m{5.7cm}|m{9.5cm}|}
\hline
Familiyası hám atı & \\
\hline
Fakulteti  & \\
\hline
Toparı hám tálim baǵdarı  & \\
\hline
\end{tabular}

\vspace{1cm}

\begin{tabular}{|m{0.7cm}|m{10cm}|m{4cm}|}
\hline
№ & Soraw & Juwap \\
\hline
1. & $OY$ kósheriniń teńlemesi? &  \\
\hline
2. & Egerde $a=\{ x_1; y_1; z_1\}, b=\{ x_2, y_2; z_2\}$ bolsa, vektor kóbeymeniń koordinatalarda ańlatılıwı qanday boladı? &  \\
\hline
3. & $A_1x+B_1y+C_1z+D_1=0$ hám $Ax_2+By_2+Cz_2+D_2=0$ tegislikleri perpendikulyar bolıwı shárti &  \\
\hline
4. & Úsh vektordıń aralas kóbeymesi ushın $(abc)=0$ teńligi orınlı bolsa ne dep ataladı? &  \\
\hline
5. & $3x^{2}+10xy+3y^{2}-2x-14y-13=0$ teńlemesiniń tipin anıqlań. &  \\
\hline
6. & $x^{2}-4y^{2}+6x+5=0$ giperbolanıń kanonikalıq teńlemesin dúziń. &  \\
\hline
7. & $M_{1}M_{2}$ kesindiniń ortasınıń koordinatalarınıń tabıń, eger $M_{1} (2, 3), M_{2} (4, 7)$ bolsa. &  \\
\hline
8. & $x+y-3=0$ hám $2x+3y-8=0$ tuwrıları óz-ara qanday jaylasqan? &  \\
\hline
9. & $x^{2}+y^{2}-2x+4y-20=0$ sheńberdiń $C$ orayın hám $R$ radiusın tabıń. &  \\
\hline
10. & $(x+1)^{2}+(y-2) ^{2}+(z+2) ^{2}=49$ sferanıń orayınıń koordinataların tabıń. &  \\
\hline
\end{tabular}

\vspace{1cm}

\begin{tabular}{lll}
Tuwrı juwaplar sanı: \underline{\hspace{1.5cm}} & 
Bahası: \underline{\hspace{1.5cm}} & 
Imtixan alıwshınıń qolı: \underline{\hspace{2cm}} \\
\end{tabular}

\egroup

\newpage


\textbf{80-variant}\\

\bgroup
\def\arraystretch{1.6} % 1 is the default, change whatever you need

\begin{tabular}{|m{5.7cm}|m{9.5cm}|}
\hline
Familiyası hám atı & \\
\hline
Fakulteti  & \\
\hline
Toparı hám tálim baǵdarı  & \\
\hline
\end{tabular}

\vspace{1cm}

\begin{tabular}{|m{0.7cm}|m{10cm}|m{4cm}|}
\hline
№ & Soraw & Juwap \\
\hline
1. & $A_1x+B_1y+C_1z+D_1=0$ hám $Ax_2+By_2+Cz_2+D_2=0$ tegislikleri ústpe-úst túsiwi shárti? &  \\
\hline
2. & Eki vektordıń skalyar kóbeymesiniń formulası? &  \\
\hline
3. & $A_1x+B_1y+C_1z+D_1=0$ hám $Ax_2+By_2+Cz_2+D_2=0$ tegislikleri parallel bolıwı shárti &  \\
\hline
4. & $Ax+C=0$ tuwrı sızıqtıń grafigi koordinata kósherlerine salıstırǵanda qanday jaylasqan? &  \\
\hline
5. & Eger $2a=16, e=\frac{5}{4}$ bolsa, fokusı abscissa kósherinde, koordinata basına salıstırǵanda simmetriyalıq jaylasqan giperbolanıń teńlemesin dúziń. &  \\
\hline
6. & Eger $2b=24, 2 c=10$ bolsa, onda abscissa kósherinde koordinata basına salıstırǵanda simmetriyalıq jaylasqan fokuslarǵa iye, ellipstiń teńlemesin dúziń. &  \\
\hline
7. & $M_{1} (12;-1)$ hám $M_{2} (0;4)$ noqatlardıń arasındaǵı aralıqtı tabıń. &  \\
\hline
8. & $x+y=0$ teńlemesi menen berilgen tuwrı sızıqtıń múyeshlik koefficientin anıqlań. &  \\
\hline
9. & Orayı $C (-1;2)$ noqatında, $A (-2;6 )$ noqatınan ótetuǵın sheńberdiń teńlemesin dúziń. &  \\
\hline
10. & $x+2=0$ keńislik qanday geometriyalıq betlikti anıqlaydı? &  \\
\hline
\end{tabular}

\vspace{1cm}

\begin{tabular}{lll}
Tuwrı juwaplar sanı: \underline{\hspace{1.5cm}} & 
Bahası: \underline{\hspace{1.5cm}} & 
Imtixan alıwshınıń qolı: \underline{\hspace{2cm}} \\
\end{tabular}

\egroup

\newpage


\textbf{81-variant}\\

\bgroup
\def\arraystretch{1.6} % 1 is the default, change whatever you need

\begin{tabular}{|m{5.7cm}|m{9.5cm}|}
\hline
Familiyası hám atı & \\
\hline
Fakulteti  & \\
\hline
Toparı hám tálim baǵdarı  & \\
\hline
\end{tabular}

\vspace{1cm}

\begin{tabular}{|m{0.7cm}|m{10cm}|m{4cm}|}
\hline
№ & Soraw & Juwap \\
\hline
1. & Eki vektor qashan kollinear dep ataladı? &  \\
\hline
2. & Tuwrı múyeshli koordinatalar sisteması dep nege aytamız? &  \\
\hline
3. & $OXY$ tegisliginiń teńlemesi? &  \\
\hline
4. & Giperbolanıń kanonikalıq teńlemesi? &  \\
\hline
5. & $\frac{x^{2}}{225}-\frac{y^{2}}{64}=-1$ giperbola fokusınıń koordinatalarınıń tabıń. &  \\
\hline
6. & $9x^{2}+25y^{2}=225$ ellipsi berilgen, ellipstiń fokusların, ekscentrisitetin tabıń. &  \\
\hline
7. & $A (-1;0;1),\ B (1;-1;0)$ noqatları berilgen. $\overline{BA}$ vektorın tabıń. &  \\
\hline
8. & $2x+3y+4=0$ tuwrısına parallel hám $M_{0} (2;1)$ noqattan ótetuǵın tuwrınıń teńlemesin dúziń. &  \\
\hline
9. & $x+y-12=0$ tuwrısı $x^{2}+y^{2}-2y=0$ sheńberge salıstırǵanda qanday jaylasqan? &  \\
\hline
10. & $\left| \overline{a} \right|=8, \left| \overline{b} \right|=5, \alpha=60^{0}$ bolsa, $( \overline{a}\overline{b} )$ ni tabıń. &  \\
\hline
\end{tabular}

\vspace{1cm}

\begin{tabular}{lll}
Tuwrı juwaplar sanı: \underline{\hspace{1.5cm}} & 
Bahası: \underline{\hspace{1.5cm}} & 
Imtixan alıwshınıń qolı: \underline{\hspace{2cm}} \\
\end{tabular}

\egroup

\newpage


\textbf{82-variant}\\

\bgroup
\def\arraystretch{1.6} % 1 is the default, change whatever you need

\begin{tabular}{|m{5.7cm}|m{9.5cm}|}
\hline
Familiyası hám atı & \\
\hline
Fakulteti  & \\
\hline
Toparı hám tálim baǵdarı  & \\
\hline
\end{tabular}

\vspace{1cm}

\begin{tabular}{|m{0.7cm}|m{10cm}|m{4cm}|}
\hline
№ & Soraw & Juwap \\
\hline
1. & Eki vektordıń vektor kóbeymesiniń uzınlıǵın tabıw formulası? &  \\
\hline
2. & Tegislikdegi qálegen noqatınan berilgen eki noqatqa shekemgi bolǵan aralıqlardıń ayırmasınıń modulı ózgermeytuǵın bolǵan noqatlardıń geometriyalıq ornı ne dep ataladı? &  \\
\hline
3. & Eki tuwrı sızıq arasındaǵı múyeshti tabıw formulası? &  \\
\hline
4. & $\frac{x^2}{a^2}-\frac{y^2}{b^2}=1$ giperbolanıń $(x_0;y_0)$ noqatındaǵı urınbasınıń teńlemesin kórsetiń. &  \\
\hline
5. & $2x+3y-6=0$ tuwrınıń teńlemesin kesindilerde berilgen teńleme túrinde kórsetiń. &  \\
\hline
6. & $\overline{a}=\left\{ 4,-2,-4 \right\}$ hám $\overline{b}=\left\{ 6,-3, 2 \right\}$ vektorları berilgen, $(\overline{a}-\overline{b}) ^{2}$-? &  \\
\hline
7. & $5x-y+7=0$ hám $3x+2y=0$ tuwrıları arasındaǵı múyeshni tabıń. &  \\
\hline
8. & $\overline{a}=\left\{ 2, 1, 0 \right\}$ hám $\overline{b}=\left\{ 1, 0,-1 \right\}$ bolsa, $\overline{a}-\overline{b}$ ni tabıń. &  \\
\hline
9. & Koordinatalar kósherleri hám $ 3x+4y-12=0 $ tuwrı sızıǵı menen shegaralanǵan úshmúyeshliktiń maydanın tabıń. &  \\
\hline
10. & $x-2y+1=0$ teńlemesi menen berilgen tuwrınıń normal túrdegi teńlemesin kórsetiń. &  \\
\hline
\end{tabular}

\vspace{1cm}

\begin{tabular}{lll}
Tuwrı juwaplar sanı: \underline{\hspace{1.5cm}} & 
Bahası: \underline{\hspace{1.5cm}} & 
Imtixan alıwshınıń qolı: \underline{\hspace{2cm}} \\
\end{tabular}

\egroup

\newpage


\textbf{83-variant}\\

\bgroup
\def\arraystretch{1.6} % 1 is the default, change whatever you need

\begin{tabular}{|m{5.7cm}|m{9.5cm}|}
\hline
Familiyası hám atı & \\
\hline
Fakulteti  & \\
\hline
Toparı hám tálim baǵdarı  & \\
\hline
\end{tabular}

\vspace{1cm}

\begin{tabular}{|m{0.7cm}|m{10cm}|m{4cm}|}
\hline
№ & Soraw & Juwap \\
\hline
1. & Vektorlardıń kósherdegi proekciyasınıń formulası? &  \\
\hline
2. & $Ax+By+D=0$ teńlemesi arqalı ... tegisliktiń teńlemesi berilgen? &  \\
\hline
3. & $\frac{x^2}{a^2}+\frac{y^2}{b^2}=1$ ellipstiń $(x_0;y_0)$ noqatındaǵı urınbasınıń teńlemesin tabıń. &  \\
\hline
4. & Vektorlardı qosıw koordinatalarda qanday formula menen anıqlanadı? &  \\
\hline
5. & $3x-y+5=0$, $x+3y-4=0$ tuwrı sızıqları arasındaǵı múyeshti tabıń. &  \\
\hline
6. & $\overline{a}=\{5,-6, 1 \}, \overline{b}=\{-4, 3, 0 \} $, $\overline{c}=\left\{ 5,-8, 10 \right\}$ vektorları berilgen. $2{\overline{a}}^{2}+4{\overline{b}}^{2}-5{\overline{c}}^{2}$ ańlatpasınıń mánisin tabıń. &  \\
\hline
7. & $(2, 3)$ hám $(4, 3)$ noqatlarınan ótiwshi tuwrı sızıqtıń teńlemesin dúziń. &  \\
\hline
8. & $x^{2}+y^{2}-2x+4y=0$ sheńberdiń teńlemesin kanonikalıq túrdegi teńlemege alıp keliń. &  \\
\hline
9. & $A(4, 3), B(7, 7)$ noqatları arasındaǵı aralıqtı tabıń. &  \\
\hline
10. & $3x^{2}+10xy+3y^{2}-2x-14y-13=0$ teńlemesiniń tipin anıqlań. &  \\
\hline
\end{tabular}

\vspace{1cm}

\begin{tabular}{lll}
Tuwrı juwaplar sanı: \underline{\hspace{1.5cm}} & 
Bahası: \underline{\hspace{1.5cm}} & 
Imtixan alıwshınıń qolı: \underline{\hspace{2cm}} \\
\end{tabular}

\egroup

\newpage


\textbf{84-variant}\\

\bgroup
\def\arraystretch{1.6} % 1 is the default, change whatever you need

\begin{tabular}{|m{5.7cm}|m{9.5cm}|}
\hline
Familiyası hám atı & \\
\hline
Fakulteti  & \\
\hline
Toparı hám tálim baǵdarı  & \\
\hline
\end{tabular}

\vspace{1cm}

\begin{tabular}{|m{0.7cm}|m{10cm}|m{4cm}|}
\hline
№ & Soraw & Juwap \\
\hline
1. & $OY$ kósheriniń teńlemesi? &  \\
\hline
2. & Egerde $a=\{ x_1; y_1; z_1\}, b=\{ x_2, y_2; z_2\}$ bolsa, vektor kóbeymeniń koordinatalarda ańlatılıwı qanday boladı? &  \\
\hline
3. & $A_1x+B_1y+C_1z+D_1=0$ hám $Ax_2+By_2+Cz_2+D_2=0$ tegislikleri perpendikulyar bolıwı shárti &  \\
\hline
4. & Úsh vektordıń aralas kóbeymesi ushın $(abc)=0$ teńligi orınlı bolsa ne dep ataladı? &  \\
\hline
5. & $x^{2}-4y^{2}+6x+5=0$ giperbolanıń kanonikalıq teńlemesin dúziń. &  \\
\hline
6. & $M_{1}M_{2}$ kesindiniń ortasınıń koordinatalarınıń tabıń, eger $M_{1} (2, 3), M_{2} (4, 7)$ bolsa. &  \\
\hline
7. & $x+y-3=0$ hám $2x+3y-8=0$ tuwrıları óz-ara qanday jaylasqan? &  \\
\hline
8. & $x^{2}+y^{2}-2x+4y-20=0$ sheńberdiń $C$ orayın hám $R$ radiusın tabıń. &  \\
\hline
9. & $(x+1)^{2}+(y-2) ^{2}+(z+2) ^{2}=49$ sferanıń orayınıń koordinataların tabıń. &  \\
\hline
10. & Eger $2a=16, e=\frac{5}{4}$ bolsa, fokusı abscissa kósherinde, koordinata basına salıstırǵanda simmetriyalıq jaylasqan giperbolanıń teńlemesin dúziń. &  \\
\hline
\end{tabular}

\vspace{1cm}

\begin{tabular}{lll}
Tuwrı juwaplar sanı: \underline{\hspace{1.5cm}} & 
Bahası: \underline{\hspace{1.5cm}} & 
Imtixan alıwshınıń qolı: \underline{\hspace{2cm}} \\
\end{tabular}

\egroup

\newpage


\textbf{85-variant}\\

\bgroup
\def\arraystretch{1.6} % 1 is the default, change whatever you need

\begin{tabular}{|m{5.7cm}|m{9.5cm}|}
\hline
Familiyası hám atı & \\
\hline
Fakulteti  & \\
\hline
Toparı hám tálim baǵdarı  & \\
\hline
\end{tabular}

\vspace{1cm}

\begin{tabular}{|m{0.7cm}|m{10cm}|m{4cm}|}
\hline
№ & Soraw & Juwap \\
\hline
1. & $A_1x+B_1y+C_1z+D_1=0$ hám $Ax_2+By_2+Cz_2+D_2=0$ tegislikleri ústpe-úst túsiwi shárti? &  \\
\hline
2. & Eki vektordıń skalyar kóbeymesiniń formulası? &  \\
\hline
3. & $A_1x+B_1y+C_1z+D_1=0$ hám $Ax_2+By_2+Cz_2+D_2=0$ tegislikleri parallel bolıwı shárti &  \\
\hline
4. & $Ax+C=0$ tuwrı sızıqtıń grafigi koordinata kósherlerine salıstırǵanda qanday jaylasqan? &  \\
\hline
5. & Eger $2b=24, 2 c=10$ bolsa, onda abscissa kósherinde koordinata basına salıstırǵanda simmetriyalıq jaylasqan fokuslarǵa iye, ellipstiń teńlemesin dúziń. &  \\
\hline
6. & $M_{1} (12;-1)$ hám $M_{2} (0;4)$ noqatlardıń arasındaǵı aralıqtı tabıń. &  \\
\hline
7. & $x+y=0$ teńlemesi menen berilgen tuwrı sızıqtıń múyeshlik koefficientin anıqlań. &  \\
\hline
8. & Orayı $C (-1;2)$ noqatında, $A (-2;6 )$ noqatınan ótetuǵın sheńberdiń teńlemesin dúziń. &  \\
\hline
9. & $x+2=0$ keńislik qanday geometriyalıq betlikti anıqlaydı? &  \\
\hline
10. & $\frac{x^{2}}{225}-\frac{y^{2}}{64}=-1$ giperbola fokusınıń koordinatalarınıń tabıń. &  \\
\hline
\end{tabular}

\vspace{1cm}

\begin{tabular}{lll}
Tuwrı juwaplar sanı: \underline{\hspace{1.5cm}} & 
Bahası: \underline{\hspace{1.5cm}} & 
Imtixan alıwshınıń qolı: \underline{\hspace{2cm}} \\
\end{tabular}

\egroup

\newpage


\textbf{86-variant}\\

\bgroup
\def\arraystretch{1.6} % 1 is the default, change whatever you need

\begin{tabular}{|m{5.7cm}|m{9.5cm}|}
\hline
Familiyası hám atı & \\
\hline
Fakulteti  & \\
\hline
Toparı hám tálim baǵdarı  & \\
\hline
\end{tabular}

\vspace{1cm}

\begin{tabular}{|m{0.7cm}|m{10cm}|m{4cm}|}
\hline
№ & Soraw & Juwap \\
\hline
1. & Eki vektor qashan kollinear dep ataladı? &  \\
\hline
2. & Tuwrı múyeshli koordinatalar sisteması dep nege aytamız? &  \\
\hline
3. & $OXY$ tegisliginiń teńlemesi? &  \\
\hline
4. & Giperbolanıń kanonikalıq teńlemesi? &  \\
\hline
5. & $9x^{2}+25y^{2}=225$ ellipsi berilgen, ellipstiń fokusların, ekscentrisitetin tabıń. &  \\
\hline
6. & $A (-1;0;1),\ B (1;-1;0)$ noqatları berilgen. $\overline{BA}$ vektorın tabıń. &  \\
\hline
7. & $2x+3y+4=0$ tuwrısına parallel hám $M_{0} (2;1)$ noqattan ótetuǵın tuwrınıń teńlemesin dúziń. &  \\
\hline
8. & $x+y-12=0$ tuwrısı $x^{2}+y^{2}-2y=0$ sheńberge salıstırǵanda qanday jaylasqan? &  \\
\hline
9. & $\left| \overline{a} \right|=8, \left| \overline{b} \right|=5, \alpha=60^{0}$ bolsa, $( \overline{a}\overline{b} )$ ni tabıń. &  \\
\hline
10. & $2x+3y-6=0$ tuwrınıń teńlemesin kesindilerde berilgen teńleme túrinde kórsetiń. &  \\
\hline
\end{tabular}

\vspace{1cm}

\begin{tabular}{lll}
Tuwrı juwaplar sanı: \underline{\hspace{1.5cm}} & 
Bahası: \underline{\hspace{1.5cm}} & 
Imtixan alıwshınıń qolı: \underline{\hspace{2cm}} \\
\end{tabular}

\egroup

\newpage


\textbf{87-variant}\\

\bgroup
\def\arraystretch{1.6} % 1 is the default, change whatever you need

\begin{tabular}{|m{5.7cm}|m{9.5cm}|}
\hline
Familiyası hám atı & \\
\hline
Fakulteti  & \\
\hline
Toparı hám tálim baǵdarı  & \\
\hline
\end{tabular}

\vspace{1cm}

\begin{tabular}{|m{0.7cm}|m{10cm}|m{4cm}|}
\hline
№ & Soraw & Juwap \\
\hline
1. & Eki vektordıń vektor kóbeymesiniń uzınlıǵın tabıw formulası? &  \\
\hline
2. & Tegislikdegi qálegen noqatınan berilgen eki noqatqa shekemgi bolǵan aralıqlardıń ayırmasınıń modulı ózgermeytuǵın bolǵan noqatlardıń geometriyalıq ornı ne dep ataladı? &  \\
\hline
3. & Eki tuwrı sızıq arasındaǵı múyeshti tabıw formulası? &  \\
\hline
4. & $\frac{x^2}{a^2}-\frac{y^2}{b^2}=1$ giperbolanıń $(x_0;y_0)$ noqatındaǵı urınbasınıń teńlemesin kórsetiń. &  \\
\hline
5. & $\overline{a}=\left\{ 4,-2,-4 \right\}$ hám $\overline{b}=\left\{ 6,-3, 2 \right\}$ vektorları berilgen, $(\overline{a}-\overline{b}) ^{2}$-? &  \\
\hline
6. & $5x-y+7=0$ hám $3x+2y=0$ tuwrıları arasındaǵı múyeshni tabıń. &  \\
\hline
7. & $\overline{a}=\left\{ 2, 1, 0 \right\}$ hám $\overline{b}=\left\{ 1, 0,-1 \right\}$ bolsa, $\overline{a}-\overline{b}$ ni tabıń. &  \\
\hline
8. & Koordinatalar kósherleri hám $ 3x+4y-12=0 $ tuwrı sızıǵı menen shegaralanǵan úshmúyeshliktiń maydanın tabıń. &  \\
\hline
9. & $x-2y+1=0$ teńlemesi menen berilgen tuwrınıń normal túrdegi teńlemesin kórsetiń. &  \\
\hline
10. & $3x-y+5=0$, $x+3y-4=0$ tuwrı sızıqları arasındaǵı múyeshti tabıń. &  \\
\hline
\end{tabular}

\vspace{1cm}

\begin{tabular}{lll}
Tuwrı juwaplar sanı: \underline{\hspace{1.5cm}} & 
Bahası: \underline{\hspace{1.5cm}} & 
Imtixan alıwshınıń qolı: \underline{\hspace{2cm}} \\
\end{tabular}

\egroup

\newpage


\textbf{88-variant}\\

\bgroup
\def\arraystretch{1.6} % 1 is the default, change whatever you need

\begin{tabular}{|m{5.7cm}|m{9.5cm}|}
\hline
Familiyası hám atı & \\
\hline
Fakulteti  & \\
\hline
Toparı hám tálim baǵdarı  & \\
\hline
\end{tabular}

\vspace{1cm}

\begin{tabular}{|m{0.7cm}|m{10cm}|m{4cm}|}
\hline
№ & Soraw & Juwap \\
\hline
1. & Vektorlardıń kósherdegi proekciyasınıń formulası? &  \\
\hline
2. & $Ax+By+D=0$ teńlemesi arqalı ... tegisliktiń teńlemesi berilgen? &  \\
\hline
3. & $\frac{x^2}{a^2}+\frac{y^2}{b^2}=1$ ellipstiń $(x_0;y_0)$ noqatındaǵı urınbasınıń teńlemesin tabıń. &  \\
\hline
4. & Vektorlardı qosıw koordinatalarda qanday formula menen anıqlanadı? &  \\
\hline
5. & $\overline{a}=\{5,-6, 1 \}, \overline{b}=\{-4, 3, 0 \} $, $\overline{c}=\left\{ 5,-8, 10 \right\}$ vektorları berilgen. $2{\overline{a}}^{2}+4{\overline{b}}^{2}-5{\overline{c}}^{2}$ ańlatpasınıń mánisin tabıń. &  \\
\hline
6. & $(2, 3)$ hám $(4, 3)$ noqatlarınan ótiwshi tuwrı sızıqtıń teńlemesin dúziń. &  \\
\hline
7. & $x^{2}+y^{2}-2x+4y=0$ sheńberdiń teńlemesin kanonikalıq túrdegi teńlemege alıp keliń. &  \\
\hline
8. & $A(4, 3), B(7, 7)$ noqatları arasındaǵı aralıqtı tabıń. &  \\
\hline
9. & $3x^{2}+10xy+3y^{2}-2x-14y-13=0$ teńlemesiniń tipin anıqlań. &  \\
\hline
10. & $x^{2}-4y^{2}+6x+5=0$ giperbolanıń kanonikalıq teńlemesin dúziń. &  \\
\hline
\end{tabular}

\vspace{1cm}

\begin{tabular}{lll}
Tuwrı juwaplar sanı: \underline{\hspace{1.5cm}} & 
Bahası: \underline{\hspace{1.5cm}} & 
Imtixan alıwshınıń qolı: \underline{\hspace{2cm}} \\
\end{tabular}

\egroup

\newpage


\textbf{89-variant}\\

\bgroup
\def\arraystretch{1.6} % 1 is the default, change whatever you need

\begin{tabular}{|m{5.7cm}|m{9.5cm}|}
\hline
Familiyası hám atı & \\
\hline
Fakulteti  & \\
\hline
Toparı hám tálim baǵdarı  & \\
\hline
\end{tabular}

\vspace{1cm}

\begin{tabular}{|m{0.7cm}|m{10cm}|m{4cm}|}
\hline
№ & Soraw & Juwap \\
\hline
1. & $OY$ kósheriniń teńlemesi? &  \\
\hline
2. & Egerde $a=\{ x_1; y_1; z_1\}, b=\{ x_2, y_2; z_2\}$ bolsa, vektor kóbeymeniń koordinatalarda ańlatılıwı qanday boladı? &  \\
\hline
3. & $A_1x+B_1y+C_1z+D_1=0$ hám $Ax_2+By_2+Cz_2+D_2=0$ tegislikleri perpendikulyar bolıwı shárti &  \\
\hline
4. & Úsh vektordıń aralas kóbeymesi ushın $(abc)=0$ teńligi orınlı bolsa ne dep ataladı? &  \\
\hline
5. & $M_{1}M_{2}$ kesindiniń ortasınıń koordinatalarınıń tabıń, eger $M_{1} (2, 3), M_{2} (4, 7)$ bolsa. &  \\
\hline
6. & $x+y-3=0$ hám $2x+3y-8=0$ tuwrıları óz-ara qanday jaylasqan? &  \\
\hline
7. & $x^{2}+y^{2}-2x+4y-20=0$ sheńberdiń $C$ orayın hám $R$ radiusın tabıń. &  \\
\hline
8. & $(x+1)^{2}+(y-2) ^{2}+(z+2) ^{2}=49$ sferanıń orayınıń koordinataların tabıń. &  \\
\hline
9. & Eger $2a=16, e=\frac{5}{4}$ bolsa, fokusı abscissa kósherinde, koordinata basına salıstırǵanda simmetriyalıq jaylasqan giperbolanıń teńlemesin dúziń. &  \\
\hline
10. & Eger $2b=24, 2 c=10$ bolsa, onda abscissa kósherinde koordinata basına salıstırǵanda simmetriyalıq jaylasqan fokuslarǵa iye, ellipstiń teńlemesin dúziń. &  \\
\hline
\end{tabular}

\vspace{1cm}

\begin{tabular}{lll}
Tuwrı juwaplar sanı: \underline{\hspace{1.5cm}} & 
Bahası: \underline{\hspace{1.5cm}} & 
Imtixan alıwshınıń qolı: \underline{\hspace{2cm}} \\
\end{tabular}

\egroup

\newpage


\textbf{90-variant}\\

\bgroup
\def\arraystretch{1.6} % 1 is the default, change whatever you need

\begin{tabular}{|m{5.7cm}|m{9.5cm}|}
\hline
Familiyası hám atı & \\
\hline
Fakulteti  & \\
\hline
Toparı hám tálim baǵdarı  & \\
\hline
\end{tabular}

\vspace{1cm}

\begin{tabular}{|m{0.7cm}|m{10cm}|m{4cm}|}
\hline
№ & Soraw & Juwap \\
\hline
1. & $A_1x+B_1y+C_1z+D_1=0$ hám $Ax_2+By_2+Cz_2+D_2=0$ tegislikleri ústpe-úst túsiwi shárti? &  \\
\hline
2. & Eki vektordıń skalyar kóbeymesiniń formulası? &  \\
\hline
3. & $A_1x+B_1y+C_1z+D_1=0$ hám $Ax_2+By_2+Cz_2+D_2=0$ tegislikleri parallel bolıwı shárti &  \\
\hline
4. & $Ax+C=0$ tuwrı sızıqtıń grafigi koordinata kósherlerine salıstırǵanda qanday jaylasqan? &  \\
\hline
5. & $M_{1} (12;-1)$ hám $M_{2} (0;4)$ noqatlardıń arasındaǵı aralıqtı tabıń. &  \\
\hline
6. & $x+y=0$ teńlemesi menen berilgen tuwrı sızıqtıń múyeshlik koefficientin anıqlań. &  \\
\hline
7. & Orayı $C (-1;2)$ noqatında, $A (-2;6 )$ noqatınan ótetuǵın sheńberdiń teńlemesin dúziń. &  \\
\hline
8. & $x+2=0$ keńislik qanday geometriyalıq betlikti anıqlaydı? &  \\
\hline
9. & $\frac{x^{2}}{225}-\frac{y^{2}}{64}=-1$ giperbola fokusınıń koordinatalarınıń tabıń. &  \\
\hline
10. & $9x^{2}+25y^{2}=225$ ellipsi berilgen, ellipstiń fokusların, ekscentrisitetin tabıń. &  \\
\hline
\end{tabular}

\vspace{1cm}

\begin{tabular}{lll}
Tuwrı juwaplar sanı: \underline{\hspace{1.5cm}} & 
Bahası: \underline{\hspace{1.5cm}} & 
Imtixan alıwshınıń qolı: \underline{\hspace{2cm}} \\
\end{tabular}

\egroup

\newpage


\textbf{91-variant}\\

\bgroup
\def\arraystretch{1.6} % 1 is the default, change whatever you need

\begin{tabular}{|m{5.7cm}|m{9.5cm}|}
\hline
Familiyası hám atı & \\
\hline
Fakulteti  & \\
\hline
Toparı hám tálim baǵdarı  & \\
\hline
\end{tabular}

\vspace{1cm}

\begin{tabular}{|m{0.7cm}|m{10cm}|m{4cm}|}
\hline
№ & Soraw & Juwap \\
\hline
1. & Eki vektor qashan kollinear dep ataladı? &  \\
\hline
2. & Tuwrı múyeshli koordinatalar sisteması dep nege aytamız? &  \\
\hline
3. & $OXY$ tegisliginiń teńlemesi? &  \\
\hline
4. & Giperbolanıń kanonikalıq teńlemesi? &  \\
\hline
5. & $A (-1;0;1),\ B (1;-1;0)$ noqatları berilgen. $\overline{BA}$ vektorın tabıń. &  \\
\hline
6. & $2x+3y+4=0$ tuwrısına parallel hám $M_{0} (2;1)$ noqattan ótetuǵın tuwrınıń teńlemesin dúziń. &  \\
\hline
7. & $x+y-12=0$ tuwrısı $x^{2}+y^{2}-2y=0$ sheńberge salıstırǵanda qanday jaylasqan? &  \\
\hline
8. & $\left| \overline{a} \right|=8, \left| \overline{b} \right|=5, \alpha=60^{0}$ bolsa, $( \overline{a}\overline{b} )$ ni tabıń. &  \\
\hline
9. & $2x+3y-6=0$ tuwrınıń teńlemesin kesindilerde berilgen teńleme túrinde kórsetiń. &  \\
\hline
10. & $\overline{a}=\left\{ 4,-2,-4 \right\}$ hám $\overline{b}=\left\{ 6,-3, 2 \right\}$ vektorları berilgen, $(\overline{a}-\overline{b}) ^{2}$-? &  \\
\hline
\end{tabular}

\vspace{1cm}

\begin{tabular}{lll}
Tuwrı juwaplar sanı: \underline{\hspace{1.5cm}} & 
Bahası: \underline{\hspace{1.5cm}} & 
Imtixan alıwshınıń qolı: \underline{\hspace{2cm}} \\
\end{tabular}

\egroup

\newpage


\textbf{92-variant}\\

\bgroup
\def\arraystretch{1.6} % 1 is the default, change whatever you need

\begin{tabular}{|m{5.7cm}|m{9.5cm}|}
\hline
Familiyası hám atı & \\
\hline
Fakulteti  & \\
\hline
Toparı hám tálim baǵdarı  & \\
\hline
\end{tabular}

\vspace{1cm}

\begin{tabular}{|m{0.7cm}|m{10cm}|m{4cm}|}
\hline
№ & Soraw & Juwap \\
\hline
1. & Eki vektordıń vektor kóbeymesiniń uzınlıǵın tabıw formulası? &  \\
\hline
2. & Tegislikdegi qálegen noqatınan berilgen eki noqatqa shekemgi bolǵan aralıqlardıń ayırmasınıń modulı ózgermeytuǵın bolǵan noqatlardıń geometriyalıq ornı ne dep ataladı? &  \\
\hline
3. & Eki tuwrı sızıq arasındaǵı múyeshti tabıw formulası? &  \\
\hline
4. & $\frac{x^2}{a^2}-\frac{y^2}{b^2}=1$ giperbolanıń $(x_0;y_0)$ noqatındaǵı urınbasınıń teńlemesin kórsetiń. &  \\
\hline
5. & $5x-y+7=0$ hám $3x+2y=0$ tuwrıları arasındaǵı múyeshni tabıń. &  \\
\hline
6. & $\overline{a}=\left\{ 2, 1, 0 \right\}$ hám $\overline{b}=\left\{ 1, 0,-1 \right\}$ bolsa, $\overline{a}-\overline{b}$ ni tabıń. &  \\
\hline
7. & Koordinatalar kósherleri hám $ 3x+4y-12=0 $ tuwrı sızıǵı menen shegaralanǵan úshmúyeshliktiń maydanın tabıń. &  \\
\hline
8. & $x-2y+1=0$ teńlemesi menen berilgen tuwrınıń normal túrdegi teńlemesin kórsetiń. &  \\
\hline
9. & $3x-y+5=0$, $x+3y-4=0$ tuwrı sızıqları arasındaǵı múyeshti tabıń. &  \\
\hline
10. & $\overline{a}=\{5,-6, 1 \}, \overline{b}=\{-4, 3, 0 \} $, $\overline{c}=\left\{ 5,-8, 10 \right\}$ vektorları berilgen. $2{\overline{a}}^{2}+4{\overline{b}}^{2}-5{\overline{c}}^{2}$ ańlatpasınıń mánisin tabıń. &  \\
\hline
\end{tabular}

\vspace{1cm}

\begin{tabular}{lll}
Tuwrı juwaplar sanı: \underline{\hspace{1.5cm}} & 
Bahası: \underline{\hspace{1.5cm}} & 
Imtixan alıwshınıń qolı: \underline{\hspace{2cm}} \\
\end{tabular}

\egroup

\newpage


\textbf{93-variant}\\

\bgroup
\def\arraystretch{1.6} % 1 is the default, change whatever you need

\begin{tabular}{|m{5.7cm}|m{9.5cm}|}
\hline
Familiyası hám atı & \\
\hline
Fakulteti  & \\
\hline
Toparı hám tálim baǵdarı  & \\
\hline
\end{tabular}

\vspace{1cm}

\begin{tabular}{|m{0.7cm}|m{10cm}|m{4cm}|}
\hline
№ & Soraw & Juwap \\
\hline
1. & Vektorlardıń kósherdegi proekciyasınıń formulası? &  \\
\hline
2. & $Ax+By+D=0$ teńlemesi arqalı ... tegisliktiń teńlemesi berilgen? &  \\
\hline
3. & $\frac{x^2}{a^2}+\frac{y^2}{b^2}=1$ ellipstiń $(x_0;y_0)$ noqatındaǵı urınbasınıń teńlemesin tabıń. &  \\
\hline
4. & Vektorlardı qosıw koordinatalarda qanday formula menen anıqlanadı? &  \\
\hline
5. & $(2, 3)$ hám $(4, 3)$ noqatlarınan ótiwshi tuwrı sızıqtıń teńlemesin dúziń. &  \\
\hline
6. & $x^{2}+y^{2}-2x+4y=0$ sheńberdiń teńlemesin kanonikalıq túrdegi teńlemege alıp keliń. &  \\
\hline
7. & $A(4, 3), B(7, 7)$ noqatları arasındaǵı aralıqtı tabıń. &  \\
\hline
8. & $3x^{2}+10xy+3y^{2}-2x-14y-13=0$ teńlemesiniń tipin anıqlań. &  \\
\hline
9. & $x^{2}-4y^{2}+6x+5=0$ giperbolanıń kanonikalıq teńlemesin dúziń. &  \\
\hline
10. & $M_{1}M_{2}$ kesindiniń ortasınıń koordinatalarınıń tabıń, eger $M_{1} (2, 3), M_{2} (4, 7)$ bolsa. &  \\
\hline
\end{tabular}

\vspace{1cm}

\begin{tabular}{lll}
Tuwrı juwaplar sanı: \underline{\hspace{1.5cm}} & 
Bahası: \underline{\hspace{1.5cm}} & 
Imtixan alıwshınıń qolı: \underline{\hspace{2cm}} \\
\end{tabular}

\egroup

\newpage


\textbf{94-variant}\\

\bgroup
\def\arraystretch{1.6} % 1 is the default, change whatever you need

\begin{tabular}{|m{5.7cm}|m{9.5cm}|}
\hline
Familiyası hám atı & \\
\hline
Fakulteti  & \\
\hline
Toparı hám tálim baǵdarı  & \\
\hline
\end{tabular}

\vspace{1cm}

\begin{tabular}{|m{0.7cm}|m{10cm}|m{4cm}|}
\hline
№ & Soraw & Juwap \\
\hline
1. & $OY$ kósheriniń teńlemesi? &  \\
\hline
2. & Egerde $a=\{ x_1; y_1; z_1\}, b=\{ x_2, y_2; z_2\}$ bolsa, vektor kóbeymeniń koordinatalarda ańlatılıwı qanday boladı? &  \\
\hline
3. & $A_1x+B_1y+C_1z+D_1=0$ hám $Ax_2+By_2+Cz_2+D_2=0$ tegislikleri perpendikulyar bolıwı shárti &  \\
\hline
4. & Úsh vektordıń aralas kóbeymesi ushın $(abc)=0$ teńligi orınlı bolsa ne dep ataladı? &  \\
\hline
5. & $x+y-3=0$ hám $2x+3y-8=0$ tuwrıları óz-ara qanday jaylasqan? &  \\
\hline
6. & $x^{2}+y^{2}-2x+4y-20=0$ sheńberdiń $C$ orayın hám $R$ radiusın tabıń. &  \\
\hline
7. & $(x+1)^{2}+(y-2) ^{2}+(z+2) ^{2}=49$ sferanıń orayınıń koordinataların tabıń. &  \\
\hline
8. & Eger $2a=16, e=\frac{5}{4}$ bolsa, fokusı abscissa kósherinde, koordinata basına salıstırǵanda simmetriyalıq jaylasqan giperbolanıń teńlemesin dúziń. &  \\
\hline
9. & Eger $2b=24, 2 c=10$ bolsa, onda abscissa kósherinde koordinata basına salıstırǵanda simmetriyalıq jaylasqan fokuslarǵa iye, ellipstiń teńlemesin dúziń. &  \\
\hline
10. & $M_{1} (12;-1)$ hám $M_{2} (0;4)$ noqatlardıń arasındaǵı aralıqtı tabıń. &  \\
\hline
\end{tabular}

\vspace{1cm}

\begin{tabular}{lll}
Tuwrı juwaplar sanı: \underline{\hspace{1.5cm}} & 
Bahası: \underline{\hspace{1.5cm}} & 
Imtixan alıwshınıń qolı: \underline{\hspace{2cm}} \\
\end{tabular}

\egroup

\newpage


\textbf{95-variant}\\

\bgroup
\def\arraystretch{1.6} % 1 is the default, change whatever you need

\begin{tabular}{|m{5.7cm}|m{9.5cm}|}
\hline
Familiyası hám atı & \\
\hline
Fakulteti  & \\
\hline
Toparı hám tálim baǵdarı  & \\
\hline
\end{tabular}

\vspace{1cm}

\begin{tabular}{|m{0.7cm}|m{10cm}|m{4cm}|}
\hline
№ & Soraw & Juwap \\
\hline
1. & $A_1x+B_1y+C_1z+D_1=0$ hám $Ax_2+By_2+Cz_2+D_2=0$ tegislikleri ústpe-úst túsiwi shárti? &  \\
\hline
2. & Eki vektordıń skalyar kóbeymesiniń formulası? &  \\
\hline
3. & $A_1x+B_1y+C_1z+D_1=0$ hám $Ax_2+By_2+Cz_2+D_2=0$ tegislikleri parallel bolıwı shárti &  \\
\hline
4. & $Ax+C=0$ tuwrı sızıqtıń grafigi koordinata kósherlerine salıstırǵanda qanday jaylasqan? &  \\
\hline
5. & $x+y=0$ teńlemesi menen berilgen tuwrı sızıqtıń múyeshlik koefficientin anıqlań. &  \\
\hline
6. & Orayı $C (-1;2)$ noqatında, $A (-2;6 )$ noqatınan ótetuǵın sheńberdiń teńlemesin dúziń. &  \\
\hline
7. & $x+2=0$ keńislik qanday geometriyalıq betlikti anıqlaydı? &  \\
\hline
8. & $\frac{x^{2}}{225}-\frac{y^{2}}{64}=-1$ giperbola fokusınıń koordinatalarınıń tabıń. &  \\
\hline
9. & $9x^{2}+25y^{2}=225$ ellipsi berilgen, ellipstiń fokusların, ekscentrisitetin tabıń. &  \\
\hline
10. & $A (-1;0;1),\ B (1;-1;0)$ noqatları berilgen. $\overline{BA}$ vektorın tabıń. &  \\
\hline
\end{tabular}

\vspace{1cm}

\begin{tabular}{lll}
Tuwrı juwaplar sanı: \underline{\hspace{1.5cm}} & 
Bahası: \underline{\hspace{1.5cm}} & 
Imtixan alıwshınıń qolı: \underline{\hspace{2cm}} \\
\end{tabular}

\egroup

\newpage


\textbf{96-variant}\\

\bgroup
\def\arraystretch{1.6} % 1 is the default, change whatever you need

\begin{tabular}{|m{5.7cm}|m{9.5cm}|}
\hline
Familiyası hám atı & \\
\hline
Fakulteti  & \\
\hline
Toparı hám tálim baǵdarı  & \\
\hline
\end{tabular}

\vspace{1cm}

\begin{tabular}{|m{0.7cm}|m{10cm}|m{4cm}|}
\hline
№ & Soraw & Juwap \\
\hline
1. & Eki vektor qashan kollinear dep ataladı? &  \\
\hline
2. & Tuwrı múyeshli koordinatalar sisteması dep nege aytamız? &  \\
\hline
3. & $OXY$ tegisliginiń teńlemesi? &  \\
\hline
4. & Giperbolanıń kanonikalıq teńlemesi? &  \\
\hline
5. & $2x+3y+4=0$ tuwrısına parallel hám $M_{0} (2;1)$ noqattan ótetuǵın tuwrınıń teńlemesin dúziń. &  \\
\hline
6. & $x+y-12=0$ tuwrısı $x^{2}+y^{2}-2y=0$ sheńberge salıstırǵanda qanday jaylasqan? &  \\
\hline
7. & $\left| \overline{a} \right|=8, \left| \overline{b} \right|=5, \alpha=60^{0}$ bolsa, $( \overline{a}\overline{b} )$ ni tabıń. &  \\
\hline
8. & $2x+3y-6=0$ tuwrınıń teńlemesin kesindilerde berilgen teńleme túrinde kórsetiń. &  \\
\hline
9. & $\overline{a}=\left\{ 4,-2,-4 \right\}$ hám $\overline{b}=\left\{ 6,-3, 2 \right\}$ vektorları berilgen, $(\overline{a}-\overline{b}) ^{2}$-? &  \\
\hline
10. & $5x-y+7=0$ hám $3x+2y=0$ tuwrıları arasındaǵı múyeshni tabıń. &  \\
\hline
\end{tabular}

\vspace{1cm}

\begin{tabular}{lll}
Tuwrı juwaplar sanı: \underline{\hspace{1.5cm}} & 
Bahası: \underline{\hspace{1.5cm}} & 
Imtixan alıwshınıń qolı: \underline{\hspace{2cm}} \\
\end{tabular}

\egroup

\newpage


\textbf{97-variant}\\

\bgroup
\def\arraystretch{1.6} % 1 is the default, change whatever you need

\begin{tabular}{|m{5.7cm}|m{9.5cm}|}
\hline
Familiyası hám atı & \\
\hline
Fakulteti  & \\
\hline
Toparı hám tálim baǵdarı  & \\
\hline
\end{tabular}

\vspace{1cm}

\begin{tabular}{|m{0.7cm}|m{10cm}|m{4cm}|}
\hline
№ & Soraw & Juwap \\
\hline
1. & Eki vektordıń vektor kóbeymesiniń uzınlıǵın tabıw formulası? &  \\
\hline
2. & Tegislikdegi qálegen noqatınan berilgen eki noqatqa shekemgi bolǵan aralıqlardıń ayırmasınıń modulı ózgermeytuǵın bolǵan noqatlardıń geometriyalıq ornı ne dep ataladı? &  \\
\hline
3. & Eki tuwrı sızıq arasındaǵı múyeshti tabıw formulası? &  \\
\hline
4. & $\frac{x^2}{a^2}-\frac{y^2}{b^2}=1$ giperbolanıń $(x_0;y_0)$ noqatındaǵı urınbasınıń teńlemesin kórsetiń. &  \\
\hline
5. & $\overline{a}=\left\{ 2, 1, 0 \right\}$ hám $\overline{b}=\left\{ 1, 0,-1 \right\}$ bolsa, $\overline{a}-\overline{b}$ ni tabıń. &  \\
\hline
6. & Koordinatalar kósherleri hám $ 3x+4y-12=0 $ tuwrı sızıǵı menen shegaralanǵan úshmúyeshliktiń maydanın tabıń. &  \\
\hline
7. & $x-2y+1=0$ teńlemesi menen berilgen tuwrınıń normal túrdegi teńlemesin kórsetiń. &  \\
\hline
8. & $3x-y+5=0$, $x+3y-4=0$ tuwrı sızıqları arasındaǵı múyeshti tabıń. &  \\
\hline
9. & $\overline{a}=\{5,-6, 1 \}, \overline{b}=\{-4, 3, 0 \} $, $\overline{c}=\left\{ 5,-8, 10 \right\}$ vektorları berilgen. $2{\overline{a}}^{2}+4{\overline{b}}^{2}-5{\overline{c}}^{2}$ ańlatpasınıń mánisin tabıń. &  \\
\hline
10. & $(2, 3)$ hám $(4, 3)$ noqatlarınan ótiwshi tuwrı sızıqtıń teńlemesin dúziń. &  \\
\hline
\end{tabular}

\vspace{1cm}

\begin{tabular}{lll}
Tuwrı juwaplar sanı: \underline{\hspace{1.5cm}} & 
Bahası: \underline{\hspace{1.5cm}} & 
Imtixan alıwshınıń qolı: \underline{\hspace{2cm}} \\
\end{tabular}

\egroup

\newpage


\textbf{98-variant}\\

\bgroup
\def\arraystretch{1.6} % 1 is the default, change whatever you need

\begin{tabular}{|m{5.7cm}|m{9.5cm}|}
\hline
Familiyası hám atı & \\
\hline
Fakulteti  & \\
\hline
Toparı hám tálim baǵdarı  & \\
\hline
\end{tabular}

\vspace{1cm}

\begin{tabular}{|m{0.7cm}|m{10cm}|m{4cm}|}
\hline
№ & Soraw & Juwap \\
\hline
1. & Vektorlardıń kósherdegi proekciyasınıń formulası? &  \\
\hline
2. & $Ax+By+D=0$ teńlemesi arqalı ... tegisliktiń teńlemesi berilgen? &  \\
\hline
3. & $\frac{x^2}{a^2}+\frac{y^2}{b^2}=1$ ellipstiń $(x_0;y_0)$ noqatındaǵı urınbasınıń teńlemesin tabıń. &  \\
\hline
4. & Vektorlardı qosıw koordinatalarda qanday formula menen anıqlanadı? &  \\
\hline
5. & $x^{2}+y^{2}-2x+4y=0$ sheńberdiń teńlemesin kanonikalıq túrdegi teńlemege alıp keliń. &  \\
\hline
6. & $A(4, 3), B(7, 7)$ noqatları arasındaǵı aralıqtı tabıń. &  \\
\hline
7. & $3x^{2}+10xy+3y^{2}-2x-14y-13=0$ teńlemesiniń tipin anıqlań. &  \\
\hline
8. & $x^{2}-4y^{2}+6x+5=0$ giperbolanıń kanonikalıq teńlemesin dúziń. &  \\
\hline
9. & $M_{1}M_{2}$ kesindiniń ortasınıń koordinatalarınıń tabıń, eger $M_{1} (2, 3), M_{2} (4, 7)$ bolsa. &  \\
\hline
10. & $x+y-3=0$ hám $2x+3y-8=0$ tuwrıları óz-ara qanday jaylasqan? &  \\
\hline
\end{tabular}

\vspace{1cm}

\begin{tabular}{lll}
Tuwrı juwaplar sanı: \underline{\hspace{1.5cm}} & 
Bahası: \underline{\hspace{1.5cm}} & 
Imtixan alıwshınıń qolı: \underline{\hspace{2cm}} \\
\end{tabular}

\egroup

\newpage


\textbf{99-variant}\\

\bgroup
\def\arraystretch{1.6} % 1 is the default, change whatever you need

\begin{tabular}{|m{5.7cm}|m{9.5cm}|}
\hline
Familiyası hám atı & \\
\hline
Fakulteti  & \\
\hline
Toparı hám tálim baǵdarı  & \\
\hline
\end{tabular}

\vspace{1cm}

\begin{tabular}{|m{0.7cm}|m{10cm}|m{4cm}|}
\hline
№ & Soraw & Juwap \\
\hline
1. & $OY$ kósheriniń teńlemesi? &  \\
\hline
2. & Egerde $a=\{ x_1; y_1; z_1\}, b=\{ x_2, y_2; z_2\}$ bolsa, vektor kóbeymeniń koordinatalarda ańlatılıwı qanday boladı? &  \\
\hline
3. & $A_1x+B_1y+C_1z+D_1=0$ hám $Ax_2+By_2+Cz_2+D_2=0$ tegislikleri perpendikulyar bolıwı shárti &  \\
\hline
4. & Úsh vektordıń aralas kóbeymesi ushın $(abc)=0$ teńligi orınlı bolsa ne dep ataladı? &  \\
\hline
5. & $x^{2}+y^{2}-2x+4y-20=0$ sheńberdiń $C$ orayın hám $R$ radiusın tabıń. &  \\
\hline
6. & $(x+1)^{2}+(y-2) ^{2}+(z+2) ^{2}=49$ sferanıń orayınıń koordinataların tabıń. &  \\
\hline
7. & Eger $2a=16, e=\frac{5}{4}$ bolsa, fokusı abscissa kósherinde, koordinata basına salıstırǵanda simmetriyalıq jaylasqan giperbolanıń teńlemesin dúziń. &  \\
\hline
8. & Eger $2b=24, 2 c=10$ bolsa, onda abscissa kósherinde koordinata basına salıstırǵanda simmetriyalıq jaylasqan fokuslarǵa iye, ellipstiń teńlemesin dúziń. &  \\
\hline
9. & $M_{1} (12;-1)$ hám $M_{2} (0;4)$ noqatlardıń arasındaǵı aralıqtı tabıń. &  \\
\hline
10. & $x+y=0$ teńlemesi menen berilgen tuwrı sızıqtıń múyeshlik koefficientin anıqlań. &  \\
\hline
\end{tabular}

\vspace{1cm}

\begin{tabular}{lll}
Tuwrı juwaplar sanı: \underline{\hspace{1.5cm}} & 
Bahası: \underline{\hspace{1.5cm}} & 
Imtixan alıwshınıń qolı: \underline{\hspace{2cm}} \\
\end{tabular}

\egroup

\newpage


\textbf{100-variant}\\

\bgroup
\def\arraystretch{1.6} % 1 is the default, change whatever you need

\begin{tabular}{|m{5.7cm}|m{9.5cm}|}
\hline
Familiyası hám atı & \\
\hline
Fakulteti  & \\
\hline
Toparı hám tálim baǵdarı  & \\
\hline
\end{tabular}

\vspace{1cm}

\begin{tabular}{|m{0.7cm}|m{10cm}|m{4cm}|}
\hline
№ & Soraw & Juwap \\
\hline
1. & $A_1x+B_1y+C_1z+D_1=0$ hám $Ax_2+By_2+Cz_2+D_2=0$ tegislikleri ústpe-úst túsiwi shárti? &  \\
\hline
2. & Eki vektordıń skalyar kóbeymesiniń formulası? &  \\
\hline
3. & $A_1x+B_1y+C_1z+D_1=0$ hám $Ax_2+By_2+Cz_2+D_2=0$ tegislikleri parallel bolıwı shárti &  \\
\hline
4. & $Ax+C=0$ tuwrı sızıqtıń grafigi koordinata kósherlerine salıstırǵanda qanday jaylasqan? &  \\
\hline
5. & Orayı $C (-1;2)$ noqatında, $A (-2;6 )$ noqatınan ótetuǵın sheńberdiń teńlemesin dúziń. &  \\
\hline
6. & $x+2=0$ keńislik qanday geometriyalıq betlikti anıqlaydı? &  \\
\hline
7. & $\frac{x^{2}}{225}-\frac{y^{2}}{64}=-1$ giperbola fokusınıń koordinatalarınıń tabıń. &  \\
\hline
8. & $9x^{2}+25y^{2}=225$ ellipsi berilgen, ellipstiń fokusların, ekscentrisitetin tabıń. &  \\
\hline
9. & $A (-1;0;1),\ B (1;-1;0)$ noqatları berilgen. $\overline{BA}$ vektorın tabıń. &  \\
\hline
10. & $2x+3y+4=0$ tuwrısına parallel hám $M_{0} (2;1)$ noqattan ótetuǵın tuwrınıń teńlemesin dúziń. &  \\
\hline
\end{tabular}

\vspace{1cm}

\begin{tabular}{lll}
Tuwrı juwaplar sanı: \underline{\hspace{1.5cm}} & 
Bahası: \underline{\hspace{1.5cm}} & 
Imtixan alıwshınıń qolı: \underline{\hspace{2cm}} \\
\end{tabular}

\egroup

\newpage



\end{document}
