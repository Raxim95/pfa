\(\overline{a} = \{5,- 6, 1 \}, \overline{b} = \{ - 4, 3, 0 \} \), \(\overline{c} = \left\{ 5,- 8, 10 \right\}\) vektorları berilgen. \(2{\bar{a}}^{2} + 4{\bar{b}}^{2} - 5{\bar{c}}^{2}\) ańlatpasınıń mánisin tabıń.
\(\left| \bar{a} \right| = 8, \left| \bar{b} \right| = 5, \alpha = 60^{0}\) bolsa, \(( \bar{a}\bar{b} )\) ni tabıń.
\(\bar{a} = \left\{ 2, 1, 0 \right\}\) hám \(\bar{b} = \left\{ 1, 0,- 1 \right\}\) bolsa, \(\bar{a} - \bar{b}\) ni tabıń.
\(\bar{a} = \left\{ 4,- 2,- 4 \right\}\) hám \(\bar{b} = \left\{ 6,- 3, 2 \right\}\) vektorları berilgen, \((\bar{a} - \bar{b}) ^{2}\)-?
\(A (- 1;0;1),\ B (1; - 1;0)\) noqatları berilgen. \(\bar{BA}\) vektorın tabıń.
\(M_{1} (12; - 1)\) hám \(M_{2} (0;4)\) noqatlardıń arasındaǵı aralıqtı tabıń.
\(M_{1}M_{2}\) kesindiniń ortasınıń koordinatalarınıń tabıń, eger \(M_{1} (2, 3), M_{2} (4, 7)\) bolsa.

\(5 x - y + 7 = 0\) hám \(3 x + 2 y = 0\) tuwrıları arasındaǵı múyeshni tabıń.
\(2 x + 3 y + 4 = 0\) tuwrısına parallel hám \(M_{0} (2;1)\) noqattan ótetuǵın tuwrınıń teńlemesin dúziń.
\(x + y - 3 = 0\) hám \(2 x + 3 y - 8 = 0\) tuwrıları óz-ara qanday jaylasqan?
\(2 x + 3 y - 6 = 0\) tuwrınıń teńlemesin kesindilerde berilgen teńleme túrinde kórsetiń.
\(x - 2 y + 1 = 0\) teńlemesi menen berilgen tuwrınıń normal túrdegi teńlemesin kórsetiń.
$(2, 3)$ hám $(4, 3)$ noqatlarınan ótiwshi tuwrı sızıqtıń teńlemesin dúziń.
Koordinatalar kósherleri hám \( 3 x + 4 y - 12 = 0 \) tuwrı sızıǵı menen shegaralanǵan úshmúyeshliktiń maydanın tabıń.
\(x + y = 0\) teńlemesi menen berilgen tuwrı sızıqtıń múyeshlik koefficientin anıqlań.
\(3 x - y + 5 = 0, x + 3 y - 4 = 0\) tuwrı sızıqları arasındaǵı múyeshti tabıń.

\(x^{2} + y^{2} - 2 x + 4 y - 20 = 0\) sheńberdiń \(C\) orayın hám \(R\) radiusın tabıń.
Orayı \(C (- 1;2)\) noqatında, \(A (- 2;6 )\) noqatınan ótetuǵın sheńberdiń teńlemesin dúziń.
\(x + y - 12 = 0\) tuwrısı \(x^{2} + y^{2} - 2 y = 0\) sheńberge salıstırǵanda qanday jaylasqan?
\(x^{2} + y^{2} - 2 x + 4 y = 0\) sheńberdiń teńlemesin kanonikalıq túrdegi teńlemege alıp keliń.

\(A_{1}x + B_{1}y + C_{1}z + D_{1} = 0\) hám \(Ax + By + Cz + D = 0\) tegislikleri parallel bolıwı ushın qaysı shárt orınlı bolıwı kerek?
\(A_{1}x + B_{1}y + C_{1}z + D_{1} = 0\) hám \(Ax + By + Cz + D = 0\) tegislikleri perpendikulyar bolıwı ushın qaysı shárt orınlı bolıwı kerek?
\(A_{1}x + B_{1}y + C_{1}z + D_{1} = 0\) hám tegislikleri ústpe-úst túsiwi ushın qaysı shárt orınlı bolıwı kerek?
\(A (4, 3), B (7, 7)\) noqatları arasındaǵı aralıqtı tabıń.
\(x + 2 = 0\) keńislik qanday geometriyalıq betlikti anıqlaydı?
\((x + 1) ^{2} + (y - 2) ^{2} + (z + 2) ^{2} = 49\) sferanıń orayınıń koordinataların tabıń.

\(3 x^{2} + 10 xy + 3 y^{2} - 2 x - 14 y - 13 = 0\) teńlemesiniń tipin anıqlań.
\(\frac{x^{2}}{a^{2}} + \frac{y^{2}}{b^{2}} = 1\) ellipstiń \((x_{0};y_{0})\) noqatındaǵı urınbasınıń teńlemesin tabıń.
\(\frac{x^{2}}{225} - \frac{y^{2}}{64} = - 1\) giperbola fokusınıń koordinatalarınıń tabıń.
Eger \(2 a = 16, e = \frac{5}{4}\) bolsa, fokusı abscissa kósherinde, koordinata basına salıstırǵanda simmetriyalıq jaylasqan giperbolanıń teńlemesin dúziń.

\(9 x^{2} + 25 y^{2} = 225\) ellipsi berilgen, ellipstiń fokusların, ekscentrisitetin tabıń.
Eger \(2 b = 24, 2 c = 10\) bolsa, onda abscissa kósherinde koordinata basına salıstırǵanda simmetriyalıq jaylasqan fokuslarǵa iye, ellipstiń teńlemesin dúziń.
\(x^{2} - 4 y^{2} + 6 x + 5 = 0\) giperbolanıń kanonikalıq teńlemesin dúziń.