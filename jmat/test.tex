\documentclass{article}
\usepackage[utf8]{inputenc}
\usepackage{array}
\usepackage[a4paper,
  left=15mm,
  top=15mm,]{geometry}
\begin{document}

\large
\pagenumbering{gobble}


\begin{center}\textbf{1-variant}\end{center}

\bgroup
\def\arraystretch{1.5}
\begin{tabular}{ |m{6cm}|m{10cm}| }
  \hline
  Familiyası hám atı & \\
  \hline
  Fakulteti &\\
  \hline
  Toparı hám tálim baǵdarı & \\
  \hline
\end{tabular}
\egroup

\vspace{0.5cm}

\bgroup
\def\arraystretch{2}
\begin{tabular}{ |l|m{8cm}|m{7cm}| }
  \hline
  №. & Soraw & Juwap \\
  \hline
  1. & Eki ózgeriwshili funkciyanıń ekinshi tártipli aralas tuwındıları qalay belgilenedi &  \\
  \hline
  2. & Nyuton-Leybnis formulasın jazıń &  \\
  \hline
  3. & Anıq integraldı esaplawdıń Nyuton-Leybnic formulasın jazıń &  \\
  \hline
  4. & $\displaystyle\int k \cdot f(x)dx = ?$ &  \\
  \hline
  5. & Anıq emes integraldı esaplań: $\displaystyle\int \left( 10x^{4} + 7x^{6} - 3 \right)dx$. &  \\
  \hline
  6. & Integraldı esaplań: $\displaystyle\int_{1}^{\infty}{\frac{1}{x^{2}}dx}$. &  \\
  \hline
  7. & Úsh birdey korobkada aq hám qara sharlar bar. 1-korobkada 5 aq, 8 qara shar, 2-korobkada 3 aq, 4 qara shar, 3-korobkada 2 aq, 3 qara shar bar. Úsh korobkaniń birewinen tosınnan alınǵan bir shar aq bolıw itimallıǵın tabıń. &  \\
  \hline
  8. & Gruppadaǵı 20 studentten neshe túrli usıl menen 3 náwbetshini saylap alıwǵa boladı?. &  \\
  \hline
  9. & Differencial teńlemeni sheshiń: $y' + xy = 0$. &  \\
  \hline
  10. & Sanlı qatardıń baslanǵısh úsh aǵzasın jazıń: $\displaystyle\sum_{n = 1}^{\infty}\frac{n!}{2^{n}}$. &  \\
  \hline
\end{tabular}
\egroup

\vspace{1cm}

\begin{tabular}{ c c c }
Tuwrı juwaplar sanı: \underline{\hspace{2cm}} & Bahası: \underline{\hspace{2cm}} & Imtixan alıwshınıń qolı: \underline{\hspace{2cm}} \\
\end{tabular}

\newpage

\begin{center}\textbf{2-variant}\end{center}

\bgroup
\def\arraystretch{1.5}
\begin{tabular}{ |m{6cm}|m{10cm}| }
  \hline
  Familiyası hám atı & \\
  \hline
  Fakulteti &\\
  \hline
  Toparı hám tálim baǵdarı & \\
  \hline
\end{tabular}
\egroup

\vspace{0.5cm}

\bgroup
\def\arraystretch{2}
\begin{tabular}{ |l|m{8cm}|m{7cm}| }
  \hline
  №. & Soraw & Juwap \\
  \hline
  1. & Ózgeriwshini almastırıp integrallaw usılıniń formulasın jazıń. &  \\
  \hline
  2. & Eki ózgeriwshili funkciyanıń birinshi tártipli dara tuwındıları qalay belgilenedi &  \\
  \hline
  3. & $(x_{0} , y_{0})$ noqattıń $\varepsilon$ dógeregi qalay belgilenedi &  \\
  \hline
  4. & Bayes formulasın jazıń &  \\
  \hline
  5. & Integraldı esaplań: $\displaystyle\int {2^{x}dx} $. &  \\
  \hline
  6. & Esaplań: $\displaystyle\int_{1}^{2}{e^{x}dx}$. &  \\
  \hline
  7. & 50 buyımnan ibarat partiyada 3 buyım jaramsız. Tosınnan alınǵan 8 buyımnıń ishinde 1 buyımı jaramsız bolıw itimallıǵın tabıń. &  \\
  \hline
  8. & Eki kubikti bir márte taslaǵanda túsken ochkolardıń qosındısı 4 bolıw itimallıǵın tabıń. &  \\
  \hline
  9. & Differencial teńlemeniń ulıwma sheshimin tabıń: $xy' - 2y = 0$. &  \\
  \hline
  10. & Qatardıń qosındısın tabıń: $\displaystyle\sum_{n = 1}^{\infty}\frac{1}{n(n + 3)}$. &  \\
  \hline
\end{tabular}
\egroup

\vspace{1cm}

\begin{tabular}{ c c c }
Tuwrı juwaplar sanı: \underline{\hspace{2cm}} & Bahası: \underline{\hspace{2cm}} & Imtixan alıwshınıń qolı: \underline{\hspace{2cm}} \\
\end{tabular}

\newpage

\begin{center}\textbf{3-variant}\end{center}

\bgroup
\def\arraystretch{1.5}
\begin{tabular}{ |m{6cm}|m{10cm}| }
  \hline
  Familiyası hám atı & \\
  \hline
  Fakulteti &\\
  \hline
  Toparı hám tálim baǵdarı & \\
  \hline
\end{tabular}
\egroup

\vspace{0.5cm}

\bgroup
\def\arraystretch{2}
\begin{tabular}{ |l|m{8cm}|m{7cm}| }
  \hline
  №. & Soraw & Juwap \\
  \hline
  1. & Funkcianıń $(x_{0}, y_{0})$ noqattaǵı tuwındısınıń formulasın jazıń &  \\
  \hline
  2. & Isenimli waqıyanıń itimallıǵı nege teń &  \\
  \hline
  3. & $n$-dárejeli kóp aǵzalınıń uluwma kórinisi &  \\
  \hline
  4. & Eki ózgeriwshili funkciyanıń tolıq ósimi &  \\
  \hline
  5. & Integraldı esaplań: $\displaystyle\int (x - 1)^{20}dx$. &  \\
  \hline
  6. & Anıq itegraldı esaplań: $\displaystyle\int_{1}^{3}{\frac{2}{x + 1}dx}$. &  \\
  \hline
  7. & Úsh birdey korobkada aq hám qara sharlar bar. 1-korobkada 5 aq, 8 qara shar, 2-korobkada 3 aq, 4 qara shar, 3-korobkada 2 aq, 3 qara shar bar. Úsh korobkanıń birewinen tosınnan alınǵan bir shar aq bolıw itimallıǵın tabıń. &  \\
  \hline
  8. & Telefon nomerdiń aqırǵı eki cifrasın umıtıp, tosınnan nomerlerdi tere basladı. Kerekli nomerdi tabıw itimallıǵın esaplań. &  \\
  \hline
  9. & Sızıqlı differerncial teńlemeniń uluwma sheshimin tabıń $y' + y =e^{-x}$. &  \\
  \hline
  10. & Qatardıń jıyındısın esaplań: $\displaystyle\sum_{n = 1}^{\infty}\frac{1}{(2n - 1)(2n + 1)}$. &  \\
  \hline
\end{tabular}
\egroup

\vspace{1cm}

\begin{tabular}{ c c c }
Tuwrı juwaplar sanı: \underline{\hspace{2cm}} & Bahası: \underline{\hspace{2cm}} & Imtixan alıwshınıń qolı: \underline{\hspace{2cm}} \\
\end{tabular}

\newpage

\begin{center}\textbf{4-variant}\end{center}

\bgroup
\def\arraystretch{1.5}
\begin{tabular}{ |m{6cm}|m{10cm}| }
  \hline
  Familiyası hám atı & \\
  \hline
  Fakulteti &\\
  \hline
  Toparı hám tálim baǵdarı & \\
  \hline
\end{tabular}
\egroup

\vspace{0.5cm}

\bgroup
\def\arraystretch{2}
\begin{tabular}{ |l|m{8cm}|m{7cm}| }
  \hline
  №. & Soraw & Juwap \\
  \hline
  1. & Shártli itimallıq formulasın jazıń &  \\
  \hline
  2. & Funkciya qanday usıllarda beriledi &  \\
  \hline
  3. & Eki ózgeriwshili funkciyanıń ekinshi tártipli dara tuwındıları qalay belgilenedi &  \\
  \hline
  4. & Shekli additivlik aksiomasın jazıń &  \\
  \hline
  5. & Integraldı esaplań: $\displaystyle\int {\frac{1}{\sin x}dx} $. &  \\
  \hline
  6. & Anıq integraldı esaplań: $\displaystyle\int_{0}^{\frac{\pi}{2}}\cos xdx$. &  \\
  \hline
  7. & Qutıda 15 aq, 18 qara shar bar. Tosınnan alınǵan bir shar aq bolıw itimallıǵın tabıń. &  \\
  \hline
  8. & Dóngelektiń ishine kvadrat sızılǵan. Dóngelektiń ishinen tosınnan belgilengen noqattıń kvadrattıń ishinde jatıw itimallıǵın tabıń. &  \\
  \hline
  9. & Sızıqlı differencial teńlemeniń ulwma sheshimin tabıń: $y' + y =e^{x}$. &  \\
  \hline
  10. & Funkcional qatardıń jıynaqlılıq oblastın jazıń: $\ln x + \ln^{2}x + \ldots + \ln^{n}x + \ldots$. &  \\
  \hline
\end{tabular}
\egroup

\vspace{1cm}

\begin{tabular}{ c c c }
Tuwrı juwaplar sanı: \underline{\hspace{2cm}} & Bahası: \underline{\hspace{2cm}} & Imtixan alıwshınıń qolı: \underline{\hspace{2cm}} \\
\end{tabular}

\newpage

\begin{center}\textbf{5-variant}\end{center}

\bgroup
\def\arraystretch{1.5}
\begin{tabular}{ |m{6cm}|m{10cm}| }
  \hline
  Familiyası hám atı & \\
  \hline
  Fakulteti &\\
  \hline
  Toparı hám tálim baǵdarı & \\
  \hline
\end{tabular}
\egroup

\vspace{0.5cm}

\bgroup
\def\arraystretch{2}
\begin{tabular}{ |l|m{8cm}|m{7cm}| }
  \hline
  №. & Soraw & Juwap \\
  \hline
  1. & Eki ózgeriwshili funkciyanıń grafigi neden ibarat &  \\
  \hline
  2. & Gruppalaw formulasın jazıń &  \\
  \hline
  3. & Eki ózgeriwshili funkciyalar qalay belgilenedi &  \\
  \hline
  4. & Funkcionallıq qatardıń uluwma kórinisi &  \\
  \hline
  5. & Integraldı esaplań: $\displaystyle\int (x + \sin x)dx$. &  \\
  \hline
  6. & Anıq integraldı esaplań: $\displaystyle\int_{-\frac{\pi}{4}}^{0}\frac{dx}{\cos^{2}x}$. &  \\
  \hline
  7. & Ídısta 5 aq, 8 qara shar bar. Ídıstan tosınnan izbe-iz 3 shar alındı. Alınǵan sharlar aq, qara, qara degen izbe-izlikte bolıw itimallıǵın tabıń. &  \\
  \hline
  8. & "MATEMATIKA" sóziniń háripleri bólek kartochkalarǵa jazılıp jawıp aralastırılıp qoyılǵan. Barlıq kartochkalar tosınnan izbe-iz alınıp ashılıp, alınıw tártibinde stol ústine dizilgende taǵı "MATEMATIKA" sóziniń kelip shıǵıw itimallıǵın tabıń. &  \\
  \hline
  9. & Differencial teńlemeni esaplań: $yy'= 4$. &  \\
  \hline
  10. & Qatardıń qosındısın tabıń: $\displaystyle\sum_{n = 1}^{\infty}\frac{1}{n(n + 3)}$. &  \\
  \hline
\end{tabular}
\egroup

\vspace{1cm}

\begin{tabular}{ c c c }
Tuwrı juwaplar sanı: \underline{\hspace{2cm}} & Bahası: \underline{\hspace{2cm}} & Imtixan alıwshınıń qolı: \underline{\hspace{2cm}} \\
\end{tabular}

\newpage

\begin{center}\textbf{6-variant}\end{center}

\bgroup
\def\arraystretch{1.5}
\begin{tabular}{ |m{6cm}|m{10cm}| }
  \hline
  Familiyası hám atı & \\
  \hline
  Fakulteti &\\
  \hline
  Toparı hám tálim baǵdarı & \\
  \hline
\end{tabular}
\egroup

\vspace{0.5cm}

\bgroup
\def\arraystretch{2}
\begin{tabular}{ |l|m{8cm}|m{7cm}| }
  \hline
  №. & Soraw & Juwap \\
  \hline
  1. & Oń aǵzalı qatarlar ushın jıynaqlılıqtıń Dalamber belgisin jazıń &  \\
  \hline
  2. & Oń aǵzalı qatarlar ushın jıynaqlılıqtıń Koshi belgisin jazıń &  \\
  \hline
  3. & Esaplań $\displaystyle d\left( \int_{}^{}{f(x)dx} \right) = ?$ &  \\
  \hline
  4. & Ózgeriwshileri ajıralǵan differenciallıq teńlemesiniń uluwma kórinisin jazıń &  \\
  \hline
  5. & Anıq emes integraldı esaplań: $\displaystyle\int e^{x}dx$. &  \\
  \hline
  6. & Anıq emes integraldı esaplań: $\displaystyle\int(x^{2}+\frac{1}{x} + \sin x)dx$. &  \\
  \hline
  7. & Qutıda 5 aq hám 15 qara shar bar. Tosınnan alınǵan bir shardıń aq bolıw itimallıǵın tabıń. &  \\
  \hline
  8. & Tiyindi eki márte taslaǵanda, keminde bir márte san tárepi túsiw itimallıǵın tabıń. &  \\
  \hline
  9. & Differencial teńlemeniń ulıwma sheshimin tabıń: $y'=e^{x}$. &  \\
  \hline
  10. & Funkcional qatardıń jıynaqlılıq oblastın tabıń: $1 + x + \ldots + x^{n} + \ldots$. &  \\
  \hline
\end{tabular}
\egroup

\vspace{1cm}

\begin{tabular}{ c c c }
Tuwrı juwaplar sanı: \underline{\hspace{2cm}} & Bahası: \underline{\hspace{2cm}} & Imtixan alıwshınıń qolı: \underline{\hspace{2cm}} \\
\end{tabular}

\newpage

\begin{center}\textbf{7-variant}\end{center}

\bgroup
\def\arraystretch{1.5}
\begin{tabular}{ |m{6cm}|m{10cm}| }
  \hline
  Familiyası hám atı & \\
  \hline
  Fakulteti &\\
  \hline
  Toparı hám tálim baǵdarı & \\
  \hline
\end{tabular}
\egroup

\vspace{0.5cm}

\bgroup
\def\arraystretch{2}
\begin{tabular}{ |l|m{8cm}|m{7cm}| }
  \hline
  №. & Soraw & Juwap \\
  \hline
  1. & Sızıqlı differenciallıq teńleme kórinisi &  \\
  \hline
  2. & Orın almastırıw formulasın jazıń &  \\
  \hline
  3. & Funkciyanıń $(x_{0}, y_{0})$ noqattaǵı úzliksizlik shártin jazıń &  \\
  \hline
  4. & Itimmallıqtıń geometriyalıq anıqlamasınıń formulasın jazıń &  \\
  \hline
  5. & Esaplań: $\displaystyle\int \left( x^{4}-\frac{1}{x} \right)dx$. &  \\
  \hline
  6. & Anıq integraldı esaplań: $\displaystyle\int_{1}^{3}{\frac{2}{x + 1}dx}$. &  \\
  \hline
  7. & Korobkada 3 aq, 7 qara shar bar. Tosınnan úsh shar izbe-iz alındı. Izbe-iz alınǵan sharlardıń qara, qara, aq degen izbe-izlikte bolıw itimallıǵın tabıń. &  \\
  \hline
  8. & Telefon nomerdiń aqırǵı cifrasın umıtıp, tosınnan nomerlerdi tere basladı. Kerekli nomerdi tabıw itimallıǵın esaplań. &  \\
  \hline
  9. & Differencial teńlemeni sheshiń: $y' + xy = 0$. &  \\
  \hline
  10. & Qatardıń qosındısın tabıń: $\displaystyle\sum_{n = 1}^{\infty}\frac{1}{n(n + 1)}$. &  \\
  \hline
\end{tabular}
\egroup

\vspace{1cm}

\begin{tabular}{ c c c }
Tuwrı juwaplar sanı: \underline{\hspace{2cm}} & Bahası: \underline{\hspace{2cm}} & Imtixan alıwshınıń qolı: \underline{\hspace{2cm}} \\
\end{tabular}

\newpage

\begin{center}\textbf{8-variant}\end{center}

\bgroup
\def\arraystretch{1.5}
\begin{tabular}{ |m{6cm}|m{10cm}| }
  \hline
  Familiyası hám atı & \\
  \hline
  Fakulteti &\\
  \hline
  Toparı hám tálim baǵdarı & \\
  \hline
\end{tabular}
\egroup

\vspace{0.5cm}

\bgroup
\def\arraystretch{2}
\begin{tabular}{ |l|m{8cm}|m{7cm}| }
  \hline
  №. & Soraw & Juwap \\
  \hline
  1. & Sızıqlı defferencial teńlemeniń uluwma sheshimin jazıń &  \\
  \hline
  2. & Sanlı qatardıń uluwma kórinisin jazıń &  \\
  \hline
  3. & Funkcianıń $(x_{0}, y_{0})$ noqattaǵı úzliksizliginiń formulasın jazıń &  \\
  \hline
  4. & Bernulli differenciallıq teńemesin jazıń &  \\
  \hline
  5. & Anıq emes integraldı esaplań: $\displaystyle\int \frac{dx}{\cos^{2}x}$. &  \\
  \hline
  6. & Integraldı esaplań: $\displaystyle\int_{1}^{\infty}{\frac{1}{\left( x + 2 \right)^{2}}dx }$. &  \\
  \hline
  7. & Úsh birdey korobkada aq hám qara sharlar bar. 1-korobkada 5 aq, 8 qara shar, 2-korobkada 3 aq, 4 qara shar, 3-korobkada 2 aq, 3 qara shar bar. Úsh korobkaniń birewinen tosınnan alınǵan bir shar aq bolıw itimallıǵın tabıń. &  \\
  \hline
  8. & "BIOLOGIYA" sóziniń háripleri bólek kartochkalarǵa jazılıp jawıp, aralastırılıp qoyılǵan. Barlıq kartochkalar tosınnan izbe-iz alınıp ashılıp, alınıw tártibinde stol ústine dizilgende taǵı "BIOLOGIYA" sóziniń kelip shıǵıw itimallıǵın tabıń. &  \\
  \hline
  9. & Differencial teńlemeniń ulıwma sheshimin tabıń: $xy' - 2y = 0$. &  \\
  \hline
  10. & Funkcional qatardıń jaqınlasıw oblastın tabıń: $\displaystyle x + \frac{x^{2}}{2^{2}} + \ldots + \frac{x^{n}}{n^{2}} + \ldots$. &  \\
  \hline
\end{tabular}
\egroup

\vspace{1cm}

\begin{tabular}{ c c c }
Tuwrı juwaplar sanı: \underline{\hspace{2cm}} & Bahası: \underline{\hspace{2cm}} & Imtixan alıwshınıń qolı: \underline{\hspace{2cm}} \\
\end{tabular}

\newpage

\begin{center}\textbf{9-variant}\end{center}

\bgroup
\def\arraystretch{1.5}
\begin{tabular}{ |m{6cm}|m{10cm}| }
  \hline
  Familiyası hám atı & \\
  \hline
  Fakulteti &\\
  \hline
  Toparı hám tálim baǵdarı & \\
  \hline
\end{tabular}
\egroup

\vspace{0.5cm}

\bgroup
\def\arraystretch{2}
\begin{tabular}{ |l|m{8cm}|m{7cm}| }
  \hline
  №. & Soraw & Juwap \\
  \hline
  1. & Tolıq itimallıqtıń formulasın jazıń &  \\
  \hline
  2. & Eki ózgeriwshli funkciyanıń $M(x_{0} , y_{0})$ noqattaǵı úzliksizliginiń anıqlaması &  \\
  \hline
  3. & Kóp aǵzalını $(x - a)$ ǵa bólgendegi qaldıq nege teń &  \\
  \hline
  4. & Orın awıstırıw formulasın jazıń &  \\
  \hline
  5. & Racional funkciyanı integrallań: $\displaystyle\int {\frac{5}{(x - 3)(x + 2)}dx}$. &  \\
  \hline
  6. & Anıq integraldı esaplań: $\displaystyle\int_{0}^{1}{(3x^{2} + 1)dx}$. &  \\
  \hline
  7. & 50 buyımnan ibarat partiyada 3 buyım jaramsız. Tosınnan alınǵan 8 buyımnıń ishinde 1 buyımı jaramsız bolıw itimallıǵın tabıń. &  \\
  \hline
  8. & Gruppadaǵı 20 studentten neshe túrli usıl menen 3 náwbetshini saylap alıwǵa boladı?. &  \\
  \hline
  9. & Sızıqlı differerncial teńlemeniń uluwma sheshimin tabıń $y' + y =e^{-x}$. &  \\
  \hline
  10. & Sanlı qatardıń baslanǵısh úsh aǵzasın jazıń: $\displaystyle\sum_{n = 1}^{\infty}\frac{n!}{2^{n}}$. &  \\
  \hline
\end{tabular}
\egroup

\vspace{1cm}

\begin{tabular}{ c c c }
Tuwrı juwaplar sanı: \underline{\hspace{2cm}} & Bahası: \underline{\hspace{2cm}} & Imtixan alıwshınıń qolı: \underline{\hspace{2cm}} \\
\end{tabular}

\newpage

\begin{center}\textbf{10-variant}\end{center}

\bgroup
\def\arraystretch{1.5}
\begin{tabular}{ |m{6cm}|m{10cm}| }
  \hline
  Familiyası hám atı & \\
  \hline
  Fakulteti &\\
  \hline
  Toparı hám tálim baǵdarı & \\
  \hline
\end{tabular}
\egroup

\vspace{0.5cm}

\bgroup
\def\arraystretch{2}
\begin{tabular}{ |l|m{8cm}|m{7cm}| }
  \hline
  №. & Soraw & Juwap \\
  \hline
  1. & Anıq integraldı esaplawdıń Nyuton-Leybnis formulasın jazıń &  \\
  \hline
  2. & Itimallıqtıń klassikalıq anıqlamasınıń formulasın keltiriń &  \\
  \hline
  3. & Itimallıqtıń mánisler oblastın jazıń &  \\
  \hline
  4. & Bóleklep inegrallaw formulasın jazıń &  \\
  \hline
  5. & Racional funkciyanı integrallań: $\displaystyle\int {\frac{3}{(x - 1)(x + 2)}dx}$. &  \\
  \hline
  6. & Anıq integraldı esaplań: $\displaystyle\int_{2}^{4}\frac{dx}{x}$. &  \\
  \hline
  7. & Úsh birdey korobkada aq hám qara sharlar bar. 1-korobkada 5 aq, 8 qara shar, 2-korobkada 3 aq, 4 qara shar, 3-korobkada 2 aq, 3 qara shar bar. Úsh korobkanıń birewinen tosınnan alınǵan bir shar aq bolıw itimallıǵın tabıń. &  \\
  \hline
  8. & Eki kubikti bir márte taslaǵanda túsken ochkolardıń qosındısı 4 bolıw itimallıǵın tabıń. &  \\
  \hline
  9. & Sızıqlı differencial teńlemeniń ulwma sheshimin tabıń: $y' + y =e^{x}$. &  \\
  \hline
  10. & Qatardıń qosındısın tabıń: $\displaystyle\sum_{n = 1}^{\infty}\frac{1}{n(n + 3)}$. &  \\
  \hline
\end{tabular}
\egroup

\vspace{1cm}

\begin{tabular}{ c c c }
Tuwrı juwaplar sanı: \underline{\hspace{2cm}} & Bahası: \underline{\hspace{2cm}} & Imtixan alıwshınıń qolı: \underline{\hspace{2cm}} \\
\end{tabular}

\newpage

\begin{center}\textbf{11-variant}\end{center}

\bgroup
\def\arraystretch{1.5}
\begin{tabular}{ |m{6cm}|m{10cm}| }
  \hline
  Familiyası hám atı & \\
  \hline
  Fakulteti &\\
  \hline
  Toparı hám tálim baǵdarı & \\
  \hline
\end{tabular}
\egroup

\vspace{0.5cm}

\bgroup
\def\arraystretch{2}
\begin{tabular}{ |l|m{8cm}|m{7cm}| }
  \hline
  №. & Soraw & Juwap \\
  \hline
  1. & $\displaystyle\int dF(x)$ nege teń &  \\
  \hline
  2. & Sızıqlı differenciallıq teńlemeniń uluwma kórinisin jazıń &  \\
  \hline
  3. & Eki ózgeriwshili funkciyanıń ekstremumınıń zárúrli shárti &  \\
  \hline
  4. & Funkciyanıń anıqlanıw oblastı qalay belgilenedi &  \\
  \hline
  5. & Anıq emes integraldı esaplań: $\displaystyle\int \left( 10x^{4} + 7x^{6} - 3 \right)dx$. &  \\
  \hline
  6. & Anıq integraldı esaplań: $\displaystyle\int_{0}^{\pi}\sin xdx$. &  \\
  \hline
  7. & Qutıda 15 aq, 18 qara shar bar. Tosınnan alınǵan bir shar aq bolıw itimallıǵın tabıń. &  \\
  \hline
  8. & Telefon nomerdiń aqırǵı eki cifrasın umıtıp, tosınnan nomerlerdi tere basladı. Kerekli nomerdi tabıw itimallıǵın esaplań. &  \\
  \hline
  9. & Differencial teńlemeni esaplań: $yy'= 4$. &  \\
  \hline
  10. & Qatardıń jıyındısın esaplań: $\displaystyle\sum_{n = 1}^{\infty}\frac{1}{(2n - 1)(2n + 1)}$. &  \\
  \hline
\end{tabular}
\egroup

\vspace{1cm}

\begin{tabular}{ c c c }
Tuwrı juwaplar sanı: \underline{\hspace{2cm}} & Bahası: \underline{\hspace{2cm}} & Imtixan alıwshınıń qolı: \underline{\hspace{2cm}} \\
\end{tabular}

\newpage

\begin{center}\textbf{12-variant}\end{center}

\bgroup
\def\arraystretch{1.5}
\begin{tabular}{ |m{6cm}|m{10cm}| }
  \hline
  Familiyası hám atı & \\
  \hline
  Fakulteti &\\
  \hline
  Toparı hám tálim baǵdarı & \\
  \hline
\end{tabular}
\egroup

\vspace{0.5cm}

\bgroup
\def\arraystretch{2}
\begin{tabular}{ |l|m{8cm}|m{7cm}| }
  \hline
  №. & Soraw & Juwap \\
  \hline
  1. & Itimallıq keńisligin jazıń &  \\
  \hline
  2. & Eger $\displaystyle\sum_{n = 1}^{\infty}a_{n} = A, \sum_{n = 1}^{\infty}b_{n} = B$ bolsa, onda $\displaystyle\sum_{n = 1}^{\infty}\left( a_{n} - b_{n} \right)$ &  \\
  \hline
  3. & Eki ózgeriwshili funkciyanıń anıqlanıw oblastı qay jerde jaylasadı &  \\
  \hline
  4. & Múmkin emes waqıyanıń itimaıllıǵı nege teń &  \\
  \hline
  5. & Integraldı esaplań: $\displaystyle\int {2^{x}dx} $. &  \\
  \hline
  6. & Integraldı esaplań: $\displaystyle\int_{1}^{\infty}{\frac{1}{x^{2}}dx}$. &  \\
  \hline
  7. & Ídısta 5 aq, 8 qara shar bar. Ídıstan tosınnan izbe-iz 3 shar alındı. Alınǵan sharlar aq, qara, qara degen izbe-izlikte bolıw itimallıǵın tabıń. &  \\
  \hline
  8. & Dóngelektiń ishine kvadrat sızılǵan. Dóngelektiń ishinen tosınnan belgilengen noqattıń kvadrattıń ishinde jatıw itimallıǵın tabıń. &  \\
  \hline
  9. & Differencial teńlemeniń ulıwma sheshimin tabıń: $y'=e^{x}$. &  \\
  \hline
  10. & Funkcional qatardıń jıynaqlılıq oblastın jazıń: $\ln x + \ln^{2}x + \ldots + \ln^{n}x + \ldots$. &  \\
  \hline
\end{tabular}
\egroup

\vspace{1cm}

\begin{tabular}{ c c c }
Tuwrı juwaplar sanı: \underline{\hspace{2cm}} & Bahası: \underline{\hspace{2cm}} & Imtixan alıwshınıń qolı: \underline{\hspace{2cm}} \\
\end{tabular}

\newpage

\begin{center}\textbf{13-variant}\end{center}

\bgroup
\def\arraystretch{1.5}
\begin{tabular}{ |m{6cm}|m{10cm}| }
  \hline
  Familiyası hám atı & \\
  \hline
  Fakulteti &\\
  \hline
  Toparı hám tálim baǵdarı & \\
  \hline
\end{tabular}
\egroup

\vspace{0.5cm}

\bgroup
\def\arraystretch{2}
\begin{tabular}{ |l|m{8cm}|m{7cm}| }
  \hline
  №. & Soraw & Juwap \\
  \hline
  1. & Eger $\displaystyle\sum_{n = 1}^{\infty}a_{n} = A, \sum_{n = 1}^{\infty}b_{n} = B$ bolsa, onda $\displaystyle\sum_{n = 1}^{\infty}\left( a_{n} + b_{n} \right)$ &  \\
  \hline
  2. & Esaplań $\displaystyle \left( \int_{}^{}{f(x)dx} \right)^\prime = ?$ &  \\
  \hline
  3. & Eki ózgeriwshili funkciyanıń ekinshi tártipli aralas tuwındıları qalay belgilenedi &  \\
  \hline
  4. & Nyuton-Leybnis formulasın jazıń &  \\
  \hline
  5. & Integraldı esaplań: $\displaystyle\int (x - 1)^{20}dx$. &  \\
  \hline
  6. & Esaplań: $\displaystyle\int_{1}^{2}{e^{x}dx}$. &  \\
  \hline
  7. & Qutıda 5 aq hám 15 qara shar bar. Tosınnan alınǵan bir shardıń aq bolıw itimallıǵın tabıń. &  \\
  \hline
  8. & "MATEMATIKA" sóziniń háripleri bólek kartochkalarǵa jazılıp jawıp aralastırılıp qoyılǵan. Barlıq kartochkalar tosınnan izbe-iz alınıp ashılıp, alınıw tártibinde stol ústine dizilgende taǵı "MATEMATIKA" sóziniń kelip shıǵıw itimallıǵın tabıń. &  \\
  \hline
  9. & Differencial teńlemeni sheshiń: $y' + xy = 0$. &  \\
  \hline
  10. & Qatardıń qosındısın tabıń: $\displaystyle\sum_{n = 1}^{\infty}\frac{1}{n(n + 3)}$. &  \\
  \hline
\end{tabular}
\egroup

\vspace{1cm}

\begin{tabular}{ c c c }
Tuwrı juwaplar sanı: \underline{\hspace{2cm}} & Bahası: \underline{\hspace{2cm}} & Imtixan alıwshınıń qolı: \underline{\hspace{2cm}} \\
\end{tabular}

\newpage

\begin{center}\textbf{14-variant}\end{center}

\bgroup
\def\arraystretch{1.5}
\begin{tabular}{ |m{6cm}|m{10cm}| }
  \hline
  Familiyası hám atı & \\
  \hline
  Fakulteti &\\
  \hline
  Toparı hám tálim baǵdarı & \\
  \hline
\end{tabular}
\egroup

\vspace{0.5cm}

\bgroup
\def\arraystretch{2}
\begin{tabular}{ |l|m{8cm}|m{7cm}| }
  \hline
  №. & Soraw & Juwap \\
  \hline
  1. & Anıq integraldı esaplawdıń Nyuton-Leybnic formulasın jazıń &  \\
  \hline
  2. & $\displaystyle\int k \cdot f(x)dx = ?$ &  \\
  \hline
  3. & Ózgeriwshini almastırıp integrallaw usılıniń formulasın jazıń. &  \\
  \hline
  4. & Eki ózgeriwshili funkciyanıń birinshi tártipli dara tuwındıları qalay belgilenedi &  \\
  \hline
  5. & Integraldı esaplań: $\displaystyle\int {\frac{1}{\sin x}dx} $. &  \\
  \hline
  6. & Anıq itegraldı esaplań: $\displaystyle\int_{1}^{3}{\frac{2}{x + 1}dx}$. &  \\
  \hline
  7. & Korobkada 3 aq, 7 qara shar bar. Tosınnan úsh shar izbe-iz alındı. Izbe-iz alınǵan sharlardıń qara, qara, aq degen izbe-izlikte bolıw itimallıǵın tabıń. &  \\
  \hline
  8. & Tiyindi eki márte taslaǵanda, keminde bir márte san tárepi túsiw itimallıǵın tabıń. &  \\
  \hline
  9. & Differencial teńlemeniń ulıwma sheshimin tabıń: $xy' - 2y = 0$. &  \\
  \hline
  10. & Funkcional qatardıń jıynaqlılıq oblastın tabıń: $1 + x + \ldots + x^{n} + \ldots$. &  \\
  \hline
\end{tabular}
\egroup

\vspace{1cm}

\begin{tabular}{ c c c }
Tuwrı juwaplar sanı: \underline{\hspace{2cm}} & Bahası: \underline{\hspace{2cm}} & Imtixan alıwshınıń qolı: \underline{\hspace{2cm}} \\
\end{tabular}

\newpage

\begin{center}\textbf{15-variant}\end{center}

\bgroup
\def\arraystretch{1.5}
\begin{tabular}{ |m{6cm}|m{10cm}| }
  \hline
  Familiyası hám atı & \\
  \hline
  Fakulteti &\\
  \hline
  Toparı hám tálim baǵdarı & \\
  \hline
\end{tabular}
\egroup

\vspace{0.5cm}

\bgroup
\def\arraystretch{2}
\begin{tabular}{ |l|m{8cm}|m{7cm}| }
  \hline
  №. & Soraw & Juwap \\
  \hline
  1. & $(x_{0} , y_{0})$ noqattıń $\varepsilon$ dógeregi qalay belgilenedi &  \\
  \hline
  2. & Bayes formulasın jazıń &  \\
  \hline
  3. & Funkcianıń $(x_{0}, y_{0})$ noqattaǵı tuwındısınıń formulasın jazıń &  \\
  \hline
  4. & Isenimli waqıyanıń itimallıǵı nege teń &  \\
  \hline
  5. & Integraldı esaplań: $\displaystyle\int (x + \sin x)dx$. &  \\
  \hline
  6. & Anıq integraldı esaplań: $\displaystyle\int_{0}^{\frac{\pi}{2}}\cos xdx$. &  \\
  \hline
  7. & Úsh birdey korobkada aq hám qara sharlar bar. 1-korobkada 5 aq, 8 qara shar, 2-korobkada 3 aq, 4 qara shar, 3-korobkada 2 aq, 3 qara shar bar. Úsh korobkaniń birewinen tosınnan alınǵan bir shar aq bolıw itimallıǵın tabıń. &  \\
  \hline
  8. & Telefon nomerdiń aqırǵı cifrasın umıtıp, tosınnan nomerlerdi tere basladı. Kerekli nomerdi tabıw itimallıǵın esaplań. &  \\
  \hline
  9. & Sızıqlı differerncial teńlemeniń uluwma sheshimin tabıń $y' + y =e^{-x}$. &  \\
  \hline
  10. & Qatardıń qosındısın tabıń: $\displaystyle\sum_{n = 1}^{\infty}\frac{1}{n(n + 1)}$. &  \\
  \hline
\end{tabular}
\egroup

\vspace{1cm}

\begin{tabular}{ c c c }
Tuwrı juwaplar sanı: \underline{\hspace{2cm}} & Bahası: \underline{\hspace{2cm}} & Imtixan alıwshınıń qolı: \underline{\hspace{2cm}} \\
\end{tabular}

\newpage

\begin{center}\textbf{16-variant}\end{center}

\bgroup
\def\arraystretch{1.5}
\begin{tabular}{ |m{6cm}|m{10cm}| }
  \hline
  Familiyası hám atı & \\
  \hline
  Fakulteti &\\
  \hline
  Toparı hám tálim baǵdarı & \\
  \hline
\end{tabular}
\egroup

\vspace{0.5cm}

\bgroup
\def\arraystretch{2}
\begin{tabular}{ |l|m{8cm}|m{7cm}| }
  \hline
  №. & Soraw & Juwap \\
  \hline
  1. & $n$-dárejeli kóp aǵzalınıń uluwma kórinisi &  \\
  \hline
  2. & Eki ózgeriwshili funkciyanıń tolıq ósimi &  \\
  \hline
  3. & Shártli itimallıq formulasın jazıń &  \\
  \hline
  4. & Funkciya qanday usıllarda beriledi &  \\
  \hline
  5. & Anıq emes integraldı esaplań: $\displaystyle\int e^{x}dx$. &  \\
  \hline
  6. & Anıq integraldı esaplań: $\displaystyle\int_{-\frac{\pi}{4}}^{0}\frac{dx}{\cos^{2}x}$. &  \\
  \hline
  7. & 50 buyımnan ibarat partiyada 3 buyım jaramsız. Tosınnan alınǵan 8 buyımnıń ishinde 1 buyımı jaramsız bolıw itimallıǵın tabıń. &  \\
  \hline
  8. & "BIOLOGIYA" sóziniń háripleri bólek kartochkalarǵa jazılıp jawıp, aralastırılıp qoyılǵan. Barlıq kartochkalar tosınnan izbe-iz alınıp ashılıp, alınıw tártibinde stol ústine dizilgende taǵı "BIOLOGIYA" sóziniń kelip shıǵıw itimallıǵın tabıń. &  \\
  \hline
  9. & Sızıqlı differencial teńlemeniń ulwma sheshimin tabıń: $y' + y =e^{x}$. &  \\
  \hline
  10. & Funkcional qatardıń jaqınlasıw oblastın tabıń: $\displaystyle x + \frac{x^{2}}{2^{2}} + \ldots + \frac{x^{n}}{n^{2}} + \ldots$. &  \\
  \hline
\end{tabular}
\egroup

\vspace{1cm}

\begin{tabular}{ c c c }
Tuwrı juwaplar sanı: \underline{\hspace{2cm}} & Bahası: \underline{\hspace{2cm}} & Imtixan alıwshınıń qolı: \underline{\hspace{2cm}} \\
\end{tabular}

\newpage

\begin{center}\textbf{17-variant}\end{center}

\bgroup
\def\arraystretch{1.5}
\begin{tabular}{ |m{6cm}|m{10cm}| }
  \hline
  Familiyası hám atı & \\
  \hline
  Fakulteti &\\
  \hline
  Toparı hám tálim baǵdarı & \\
  \hline
\end{tabular}
\egroup

\vspace{0.5cm}

\bgroup
\def\arraystretch{2}
\begin{tabular}{ |l|m{8cm}|m{7cm}| }
  \hline
  №. & Soraw & Juwap \\
  \hline
  1. & Eki ózgeriwshili funkciyanıń ekinshi tártipli dara tuwındıları qalay belgilenedi &  \\
  \hline
  2. & Shekli additivlik aksiomasın jazıń &  \\
  \hline
  3. & Eki ózgeriwshili funkciyanıń grafigi neden ibarat &  \\
  \hline
  4. & Gruppalaw formulasın jazıń &  \\
  \hline
  5. & Esaplań: $\displaystyle\int \left( x^{4}-\frac{1}{x} \right)dx$. &  \\
  \hline
  6. & Anıq emes integraldı esaplań: $\displaystyle\int(x^{2}+\frac{1}{x} + \sin x)dx$. &  \\
  \hline
  7. & Úsh birdey korobkada aq hám qara sharlar bar. 1-korobkada 5 aq, 8 qara shar, 2-korobkada 3 aq, 4 qara shar, 3-korobkada 2 aq, 3 qara shar bar. Úsh korobkanıń birewinen tosınnan alınǵan bir shar aq bolıw itimallıǵın tabıń. &  \\
  \hline
  8. & Gruppadaǵı 20 studentten neshe túrli usıl menen 3 náwbetshini saylap alıwǵa boladı?. &  \\
  \hline
  9. & Differencial teńlemeni esaplań: $yy'= 4$. &  \\
  \hline
  10. & Sanlı qatardıń baslanǵısh úsh aǵzasın jazıń: $\displaystyle\sum_{n = 1}^{\infty}\frac{n!}{2^{n}}$. &  \\
  \hline
\end{tabular}
\egroup

\vspace{1cm}

\begin{tabular}{ c c c }
Tuwrı juwaplar sanı: \underline{\hspace{2cm}} & Bahası: \underline{\hspace{2cm}} & Imtixan alıwshınıń qolı: \underline{\hspace{2cm}} \\
\end{tabular}

\newpage

\begin{center}\textbf{18-variant}\end{center}

\bgroup
\def\arraystretch{1.5}
\begin{tabular}{ |m{6cm}|m{10cm}| }
  \hline
  Familiyası hám atı & \\
  \hline
  Fakulteti &\\
  \hline
  Toparı hám tálim baǵdarı & \\
  \hline
\end{tabular}
\egroup

\vspace{0.5cm}

\bgroup
\def\arraystretch{2}
\begin{tabular}{ |l|m{8cm}|m{7cm}| }
  \hline
  №. & Soraw & Juwap \\
  \hline
  1. & Eki ózgeriwshili funkciyalar qalay belgilenedi &  \\
  \hline
  2. & Funkcionallıq qatardıń uluwma kórinisi &  \\
  \hline
  3. & Oń aǵzalı qatarlar ushın jıynaqlılıqtıń Dalamber belgisin jazıń &  \\
  \hline
  4. & Oń aǵzalı qatarlar ushın jıynaqlılıqtıń Koshi belgisin jazıń &  \\
  \hline
  5. & Anıq emes integraldı esaplań: $\displaystyle\int \frac{dx}{\cos^{2}x}$. &  \\
  \hline
  6. & Anıq integraldı esaplań: $\displaystyle\int_{1}^{3}{\frac{2}{x + 1}dx}$. &  \\
  \hline
  7. & Qutıda 15 aq, 18 qara shar bar. Tosınnan alınǵan bir shar aq bolıw itimallıǵın tabıń. &  \\
  \hline
  8. & Eki kubikti bir márte taslaǵanda túsken ochkolardıń qosındısı 4 bolıw itimallıǵın tabıń. &  \\
  \hline
  9. & Differencial teńlemeniń ulıwma sheshimin tabıń: $y'=e^{x}$. &  \\
  \hline
  10. & Qatardıń qosındısın tabıń: $\displaystyle\sum_{n = 1}^{\infty}\frac{1}{n(n + 3)}$. &  \\
  \hline
\end{tabular}
\egroup

\vspace{1cm}

\begin{tabular}{ c c c }
Tuwrı juwaplar sanı: \underline{\hspace{2cm}} & Bahası: \underline{\hspace{2cm}} & Imtixan alıwshınıń qolı: \underline{\hspace{2cm}} \\
\end{tabular}

\newpage

\begin{center}\textbf{19-variant}\end{center}

\bgroup
\def\arraystretch{1.5}
\begin{tabular}{ |m{6cm}|m{10cm}| }
  \hline
  Familiyası hám atı & \\
  \hline
  Fakulteti &\\
  \hline
  Toparı hám tálim baǵdarı & \\
  \hline
\end{tabular}
\egroup

\vspace{0.5cm}

\bgroup
\def\arraystretch{2}
\begin{tabular}{ |l|m{8cm}|m{7cm}| }
  \hline
  №. & Soraw & Juwap \\
  \hline
  1. & Esaplań $\displaystyle d\left( \int_{}^{}{f(x)dx} \right) = ?$ &  \\
  \hline
  2. & Ózgeriwshileri ajıralǵan differenciallıq teńlemesiniń uluwma kórinisin jazıń &  \\
  \hline
  3. & Sızıqlı differenciallıq teńleme kórinisi &  \\
  \hline
  4. & Orın almastırıw formulasın jazıń &  \\
  \hline
  5. & Racional funkciyanı integrallań: $\displaystyle\int {\frac{5}{(x - 3)(x + 2)}dx}$. &  \\
  \hline
  6. & Integraldı esaplań: $\displaystyle\int_{1}^{\infty}{\frac{1}{\left( x + 2 \right)^{2}}dx }$. &  \\
  \hline
  7. & Ídısta 5 aq, 8 qara shar bar. Ídıstan tosınnan izbe-iz 3 shar alındı. Alınǵan sharlar aq, qara, qara degen izbe-izlikte bolıw itimallıǵın tabıń. &  \\
  \hline
  8. & Telefon nomerdiń aqırǵı eki cifrasın umıtıp, tosınnan nomerlerdi tere basladı. Kerekli nomerdi tabıw itimallıǵın esaplań. &  \\
  \hline
  9. & Differencial teńlemeni sheshiń: $y' + xy = 0$. &  \\
  \hline
  10. & Qatardıń jıyındısın esaplań: $\displaystyle\sum_{n = 1}^{\infty}\frac{1}{(2n - 1)(2n + 1)}$. &  \\
  \hline
\end{tabular}
\egroup

\vspace{1cm}

\begin{tabular}{ c c c }
Tuwrı juwaplar sanı: \underline{\hspace{2cm}} & Bahası: \underline{\hspace{2cm}} & Imtixan alıwshınıń qolı: \underline{\hspace{2cm}} \\
\end{tabular}

\newpage

\begin{center}\textbf{20-variant}\end{center}

\bgroup
\def\arraystretch{1.5}
\begin{tabular}{ |m{6cm}|m{10cm}| }
  \hline
  Familiyası hám atı & \\
  \hline
  Fakulteti &\\
  \hline
  Toparı hám tálim baǵdarı & \\
  \hline
\end{tabular}
\egroup

\vspace{0.5cm}

\bgroup
\def\arraystretch{2}
\begin{tabular}{ |l|m{8cm}|m{7cm}| }
  \hline
  №. & Soraw & Juwap \\
  \hline
  1. & Funkciyanıń $(x_{0}, y_{0})$ noqattaǵı úzliksizlik shártin jazıń &  \\
  \hline
  2. & Itimmallıqtıń geometriyalıq anıqlamasınıń formulasın jazıń &  \\
  \hline
  3. & Sızıqlı defferencial teńlemeniń uluwma sheshimin jazıń &  \\
  \hline
  4. & Sanlı qatardıń uluwma kórinisin jazıń &  \\
  \hline
  5. & Racional funkciyanı integrallań: $\displaystyle\int {\frac{3}{(x - 1)(x + 2)}dx}$. &  \\
  \hline
  6. & Anıq integraldı esaplań: $\displaystyle\int_{0}^{1}{(3x^{2} + 1)dx}$. &  \\
  \hline
  7. & Qutıda 5 aq hám 15 qara shar bar. Tosınnan alınǵan bir shardıń aq bolıw itimallıǵın tabıń. &  \\
  \hline
  8. & Dóngelektiń ishine kvadrat sızılǵan. Dóngelektiń ishinen tosınnan belgilengen noqattıń kvadrattıń ishinde jatıw itimallıǵın tabıń. &  \\
  \hline
  9. & Differencial teńlemeniń ulıwma sheshimin tabıń: $xy' - 2y = 0$. &  \\
  \hline
  10. & Funkcional qatardıń jıynaqlılıq oblastın jazıń: $\ln x + \ln^{2}x + \ldots + \ln^{n}x + \ldots$. &  \\
  \hline
\end{tabular}
\egroup

\vspace{1cm}

\begin{tabular}{ c c c }
Tuwrı juwaplar sanı: \underline{\hspace{2cm}} & Bahası: \underline{\hspace{2cm}} & Imtixan alıwshınıń qolı: \underline{\hspace{2cm}} \\
\end{tabular}

\newpage

\begin{center}\textbf{21-variant}\end{center}

\bgroup
\def\arraystretch{1.5}
\begin{tabular}{ |m{6cm}|m{10cm}| }
  \hline
  Familiyası hám atı & \\
  \hline
  Fakulteti &\\
  \hline
  Toparı hám tálim baǵdarı & \\
  \hline
\end{tabular}
\egroup

\vspace{0.5cm}

\bgroup
\def\arraystretch{2}
\begin{tabular}{ |l|m{8cm}|m{7cm}| }
  \hline
  №. & Soraw & Juwap \\
  \hline
  1. & Funkcianıń $(x_{0}, y_{0})$ noqattaǵı úzliksizliginiń formulasın jazıń &  \\
  \hline
  2. & Bernulli differenciallıq teńemesin jazıń &  \\
  \hline
  3. & Tolıq itimallıqtıń formulasın jazıń &  \\
  \hline
  4. & Eki ózgeriwshli funkciyanıń $M(x_{0} , y_{0})$ noqattaǵı úzliksizliginiń anıqlaması &  \\
  \hline
  5. & Anıq emes integraldı esaplań: $\displaystyle\int \left( 10x^{4} + 7x^{6} - 3 \right)dx$. &  \\
  \hline
  6. & Anıq integraldı esaplań: $\displaystyle\int_{2}^{4}\frac{dx}{x}$. &  \\
  \hline
  7. & Korobkada 3 aq, 7 qara shar bar. Tosınnan úsh shar izbe-iz alındı. Izbe-iz alınǵan sharlardıń qara, qara, aq degen izbe-izlikte bolıw itimallıǵın tabıń. &  \\
  \hline
  8. & "MATEMATIKA" sóziniń háripleri bólek kartochkalarǵa jazılıp jawıp aralastırılıp qoyılǵan. Barlıq kartochkalar tosınnan izbe-iz alınıp ashılıp, alınıw tártibinde stol ústine dizilgende taǵı "MATEMATIKA" sóziniń kelip shıǵıw itimallıǵın tabıń. &  \\
  \hline
  9. & Sızıqlı differerncial teńlemeniń uluwma sheshimin tabıń $y' + y =e^{-x}$. &  \\
  \hline
  10. & Qatardıń qosındısın tabıń: $\displaystyle\sum_{n = 1}^{\infty}\frac{1}{n(n + 3)}$. &  \\
  \hline
\end{tabular}
\egroup

\vspace{1cm}

\begin{tabular}{ c c c }
Tuwrı juwaplar sanı: \underline{\hspace{2cm}} & Bahası: \underline{\hspace{2cm}} & Imtixan alıwshınıń qolı: \underline{\hspace{2cm}} \\
\end{tabular}

\newpage

\begin{center}\textbf{22-variant}\end{center}

\bgroup
\def\arraystretch{1.5}
\begin{tabular}{ |m{6cm}|m{10cm}| }
  \hline
  Familiyası hám atı & \\
  \hline
  Fakulteti &\\
  \hline
  Toparı hám tálim baǵdarı & \\
  \hline
\end{tabular}
\egroup

\vspace{0.5cm}

\bgroup
\def\arraystretch{2}
\begin{tabular}{ |l|m{8cm}|m{7cm}| }
  \hline
  №. & Soraw & Juwap \\
  \hline
  1. & Kóp aǵzalını $(x - a)$ ǵa bólgendegi qaldıq nege teń &  \\
  \hline
  2. & Orın awıstırıw formulasın jazıń &  \\
  \hline
  3. & Anıq integraldı esaplawdıń Nyuton-Leybnis formulasın jazıń &  \\
  \hline
  4. & Itimallıqtıń klassikalıq anıqlamasınıń formulasın keltiriń &  \\
  \hline
  5. & Integraldı esaplań: $\displaystyle\int {2^{x}dx} $. &  \\
  \hline
  6. & Anıq integraldı esaplań: $\displaystyle\int_{0}^{\pi}\sin xdx$. &  \\
  \hline
  7. & Úsh birdey korobkada aq hám qara sharlar bar. 1-korobkada 5 aq, 8 qara shar, 2-korobkada 3 aq, 4 qara shar, 3-korobkada 2 aq, 3 qara shar bar. Úsh korobkaniń birewinen tosınnan alınǵan bir shar aq bolıw itimallıǵın tabıń. &  \\
  \hline
  8. & Tiyindi eki márte taslaǵanda, keminde bir márte san tárepi túsiw itimallıǵın tabıń. &  \\
  \hline
  9. & Sızıqlı differencial teńlemeniń ulwma sheshimin tabıń: $y' + y =e^{x}$. &  \\
  \hline
  10. & Funkcional qatardıń jıynaqlılıq oblastın tabıń: $1 + x + \ldots + x^{n} + \ldots$. &  \\
  \hline
\end{tabular}
\egroup

\vspace{1cm}

\begin{tabular}{ c c c }
Tuwrı juwaplar sanı: \underline{\hspace{2cm}} & Bahası: \underline{\hspace{2cm}} & Imtixan alıwshınıń qolı: \underline{\hspace{2cm}} \\
\end{tabular}

\newpage

\begin{center}\textbf{23-variant}\end{center}

\bgroup
\def\arraystretch{1.5}
\begin{tabular}{ |m{6cm}|m{10cm}| }
  \hline
  Familiyası hám atı & \\
  \hline
  Fakulteti &\\
  \hline
  Toparı hám tálim baǵdarı & \\
  \hline
\end{tabular}
\egroup

\vspace{0.5cm}

\bgroup
\def\arraystretch{2}
\begin{tabular}{ |l|m{8cm}|m{7cm}| }
  \hline
  №. & Soraw & Juwap \\
  \hline
  1. & Itimallıqtıń mánisler oblastın jazıń &  \\
  \hline
  2. & Bóleklep inegrallaw formulasın jazıń &  \\
  \hline
  3. & $\displaystyle\int dF(x)$ nege teń &  \\
  \hline
  4. & Sızıqlı differenciallıq teńlemeniń uluwma kórinisin jazıń &  \\
  \hline
  5. & Integraldı esaplań: $\displaystyle\int (x - 1)^{20}dx$. &  \\
  \hline
  6. & Integraldı esaplań: $\displaystyle\int_{1}^{\infty}{\frac{1}{x^{2}}dx}$. &  \\
  \hline
  7. & 50 buyımnan ibarat partiyada 3 buyım jaramsız. Tosınnan alınǵan 8 buyımnıń ishinde 1 buyımı jaramsız bolıw itimallıǵın tabıń. &  \\
  \hline
  8. & Telefon nomerdiń aqırǵı cifrasın umıtıp, tosınnan nomerlerdi tere basladı. Kerekli nomerdi tabıw itimallıǵın esaplań. &  \\
  \hline
  9. & Differencial teńlemeni esaplań: $yy'= 4$. &  \\
  \hline
  10. & Qatardıń qosındısın tabıń: $\displaystyle\sum_{n = 1}^{\infty}\frac{1}{n(n + 1)}$. &  \\
  \hline
\end{tabular}
\egroup

\vspace{1cm}

\begin{tabular}{ c c c }
Tuwrı juwaplar sanı: \underline{\hspace{2cm}} & Bahası: \underline{\hspace{2cm}} & Imtixan alıwshınıń qolı: \underline{\hspace{2cm}} \\
\end{tabular}

\newpage

\begin{center}\textbf{24-variant}\end{center}

\bgroup
\def\arraystretch{1.5}
\begin{tabular}{ |m{6cm}|m{10cm}| }
  \hline
  Familiyası hám atı & \\
  \hline
  Fakulteti &\\
  \hline
  Toparı hám tálim baǵdarı & \\
  \hline
\end{tabular}
\egroup

\vspace{0.5cm}

\bgroup
\def\arraystretch{2}
\begin{tabular}{ |l|m{8cm}|m{7cm}| }
  \hline
  №. & Soraw & Juwap \\
  \hline
  1. & Eki ózgeriwshili funkciyanıń ekstremumınıń zárúrli shárti &  \\
  \hline
  2. & Funkciyanıń anıqlanıw oblastı qalay belgilenedi &  \\
  \hline
  3. & Itimallıq keńisligin jazıń &  \\
  \hline
  4. & Eger $\displaystyle\sum_{n = 1}^{\infty}a_{n} = A, \sum_{n = 1}^{\infty}b_{n} = B$ bolsa, onda $\displaystyle\sum_{n = 1}^{\infty}\left( a_{n} - b_{n} \right)$ &  \\
  \hline
  5. & Integraldı esaplań: $\displaystyle\int {\frac{1}{\sin x}dx} $. &  \\
  \hline
  6. & Esaplań: $\displaystyle\int_{1}^{2}{e^{x}dx}$. &  \\
  \hline
  7. & Úsh birdey korobkada aq hám qara sharlar bar. 1-korobkada 5 aq, 8 qara shar, 2-korobkada 3 aq, 4 qara shar, 3-korobkada 2 aq, 3 qara shar bar. Úsh korobkanıń birewinen tosınnan alınǵan bir shar aq bolıw itimallıǵın tabıń. &  \\
  \hline
  8. & "BIOLOGIYA" sóziniń háripleri bólek kartochkalarǵa jazılıp jawıp, aralastırılıp qoyılǵan. Barlıq kartochkalar tosınnan izbe-iz alınıp ashılıp, alınıw tártibinde stol ústine dizilgende taǵı "BIOLOGIYA" sóziniń kelip shıǵıw itimallıǵın tabıń. &  \\
  \hline
  9. & Differencial teńlemeniń ulıwma sheshimin tabıń: $y'=e^{x}$. &  \\
  \hline
  10. & Funkcional qatardıń jaqınlasıw oblastın tabıń: $\displaystyle x + \frac{x^{2}}{2^{2}} + \ldots + \frac{x^{n}}{n^{2}} + \ldots$. &  \\
  \hline
\end{tabular}
\egroup

\vspace{1cm}

\begin{tabular}{ c c c }
Tuwrı juwaplar sanı: \underline{\hspace{2cm}} & Bahası: \underline{\hspace{2cm}} & Imtixan alıwshınıń qolı: \underline{\hspace{2cm}} \\
\end{tabular}

\newpage

\begin{center}\textbf{25-variant}\end{center}

\bgroup
\def\arraystretch{1.5}
\begin{tabular}{ |m{6cm}|m{10cm}| }
  \hline
  Familiyası hám atı & \\
  \hline
  Fakulteti &\\
  \hline
  Toparı hám tálim baǵdarı & \\
  \hline
\end{tabular}
\egroup

\vspace{0.5cm}

\bgroup
\def\arraystretch{2}
\begin{tabular}{ |l|m{8cm}|m{7cm}| }
  \hline
  №. & Soraw & Juwap \\
  \hline
  1. & Eki ózgeriwshili funkciyanıń anıqlanıw oblastı qay jerde jaylasadı &  \\
  \hline
  2. & Múmkin emes waqıyanıń itimaıllıǵı nege teń &  \\
  \hline
  3. & Eger $\displaystyle\sum_{n = 1}^{\infty}a_{n} = A, \sum_{n = 1}^{\infty}b_{n} = B$ bolsa, onda $\displaystyle\sum_{n = 1}^{\infty}\left( a_{n} + b_{n} \right)$ &  \\
  \hline
  4. & Esaplań $\displaystyle \left( \int_{}^{}{f(x)dx} \right)^\prime = ?$ &  \\
  \hline
  5. & Integraldı esaplań: $\displaystyle\int (x + \sin x)dx$. &  \\
  \hline
  6. & Anıq itegraldı esaplań: $\displaystyle\int_{1}^{3}{\frac{2}{x + 1}dx}$. &  \\
  \hline
  7. & Qutıda 15 aq, 18 qara shar bar. Tosınnan alınǵan bir shar aq bolıw itimallıǵın tabıń. &  \\
  \hline
  8. & Gruppadaǵı 20 studentten neshe túrli usıl menen 3 náwbetshini saylap alıwǵa boladı?. &  \\
  \hline
  9. & Differencial teńlemeni sheshiń: $y' + xy = 0$. &  \\
  \hline
  10. & Sanlı qatardıń baslanǵısh úsh aǵzasın jazıń: $\displaystyle\sum_{n = 1}^{\infty}\frac{n!}{2^{n}}$. &  \\
  \hline
\end{tabular}
\egroup

\vspace{1cm}

\begin{tabular}{ c c c }
Tuwrı juwaplar sanı: \underline{\hspace{2cm}} & Bahası: \underline{\hspace{2cm}} & Imtixan alıwshınıń qolı: \underline{\hspace{2cm}} \\
\end{tabular}

\newpage

\begin{center}\textbf{26-variant}\end{center}

\bgroup
\def\arraystretch{1.5}
\begin{tabular}{ |m{6cm}|m{10cm}| }
  \hline
  Familiyası hám atı & \\
  \hline
  Fakulteti &\\
  \hline
  Toparı hám tálim baǵdarı & \\
  \hline
\end{tabular}
\egroup

\vspace{0.5cm}

\bgroup
\def\arraystretch{2}
\begin{tabular}{ |l|m{8cm}|m{7cm}| }
  \hline
  №. & Soraw & Juwap \\
  \hline
  1. & Eki ózgeriwshili funkciyanıń ekinshi tártipli aralas tuwındıları qalay belgilenedi &  \\
  \hline
  2. & Nyuton-Leybnis formulasın jazıń &  \\
  \hline
  3. & Anıq integraldı esaplawdıń Nyuton-Leybnic formulasın jazıń &  \\
  \hline
  4. & $\displaystyle\int k \cdot f(x)dx = ?$ &  \\
  \hline
  5. & Anıq emes integraldı esaplań: $\displaystyle\int e^{x}dx$. &  \\
  \hline
  6. & Anıq integraldı esaplań: $\displaystyle\int_{0}^{\frac{\pi}{2}}\cos xdx$. &  \\
  \hline
  7. & Ídısta 5 aq, 8 qara shar bar. Ídıstan tosınnan izbe-iz 3 shar alındı. Alınǵan sharlar aq, qara, qara degen izbe-izlikte bolıw itimallıǵın tabıń. &  \\
  \hline
  8. & Eki kubikti bir márte taslaǵanda túsken ochkolardıń qosındısı 4 bolıw itimallıǵın tabıń. &  \\
  \hline
  9. & Differencial teńlemeniń ulıwma sheshimin tabıń: $xy' - 2y = 0$. &  \\
  \hline
  10. & Qatardıń qosındısın tabıń: $\displaystyle\sum_{n = 1}^{\infty}\frac{1}{n(n + 3)}$. &  \\
  \hline
\end{tabular}
\egroup

\vspace{1cm}

\begin{tabular}{ c c c }
Tuwrı juwaplar sanı: \underline{\hspace{2cm}} & Bahası: \underline{\hspace{2cm}} & Imtixan alıwshınıń qolı: \underline{\hspace{2cm}} \\
\end{tabular}

\newpage

\begin{center}\textbf{27-variant}\end{center}

\bgroup
\def\arraystretch{1.5}
\begin{tabular}{ |m{6cm}|m{10cm}| }
  \hline
  Familiyası hám atı & \\
  \hline
  Fakulteti &\\
  \hline
  Toparı hám tálim baǵdarı & \\
  \hline
\end{tabular}
\egroup

\vspace{0.5cm}

\bgroup
\def\arraystretch{2}
\begin{tabular}{ |l|m{8cm}|m{7cm}| }
  \hline
  №. & Soraw & Juwap \\
  \hline
  1. & Ózgeriwshini almastırıp integrallaw usılıniń formulasın jazıń. &  \\
  \hline
  2. & Eki ózgeriwshili funkciyanıń birinshi tártipli dara tuwındıları qalay belgilenedi &  \\
  \hline
  3. & $(x_{0} , y_{0})$ noqattıń $\varepsilon$ dógeregi qalay belgilenedi &  \\
  \hline
  4. & Bayes formulasın jazıń &  \\
  \hline
  5. & Esaplań: $\displaystyle\int \left( x^{4}-\frac{1}{x} \right)dx$. &  \\
  \hline
  6. & Anıq integraldı esaplań: $\displaystyle\int_{-\frac{\pi}{4}}^{0}\frac{dx}{\cos^{2}x}$. &  \\
  \hline
  7. & Qutıda 5 aq hám 15 qara shar bar. Tosınnan alınǵan bir shardıń aq bolıw itimallıǵın tabıń. &  \\
  \hline
  8. & Telefon nomerdiń aqırǵı eki cifrasın umıtıp, tosınnan nomerlerdi tere basladı. Kerekli nomerdi tabıw itimallıǵın esaplań. &  \\
  \hline
  9. & Sızıqlı differerncial teńlemeniń uluwma sheshimin tabıń $y' + y =e^{-x}$. &  \\
  \hline
  10. & Qatardıń jıyındısın esaplań: $\displaystyle\sum_{n = 1}^{\infty}\frac{1}{(2n - 1)(2n + 1)}$. &  \\
  \hline
\end{tabular}
\egroup

\vspace{1cm}

\begin{tabular}{ c c c }
Tuwrı juwaplar sanı: \underline{\hspace{2cm}} & Bahası: \underline{\hspace{2cm}} & Imtixan alıwshınıń qolı: \underline{\hspace{2cm}} \\
\end{tabular}

\newpage

\begin{center}\textbf{28-variant}\end{center}

\bgroup
\def\arraystretch{1.5}
\begin{tabular}{ |m{6cm}|m{10cm}| }
  \hline
  Familiyası hám atı & \\
  \hline
  Fakulteti &\\
  \hline
  Toparı hám tálim baǵdarı & \\
  \hline
\end{tabular}
\egroup

\vspace{0.5cm}

\bgroup
\def\arraystretch{2}
\begin{tabular}{ |l|m{8cm}|m{7cm}| }
  \hline
  №. & Soraw & Juwap \\
  \hline
  1. & Funkcianıń $(x_{0}, y_{0})$ noqattaǵı tuwındısınıń formulasın jazıń &  \\
  \hline
  2. & Isenimli waqıyanıń itimallıǵı nege teń &  \\
  \hline
  3. & $n$-dárejeli kóp aǵzalınıń uluwma kórinisi &  \\
  \hline
  4. & Eki ózgeriwshili funkciyanıń tolıq ósimi &  \\
  \hline
  5. & Anıq emes integraldı esaplań: $\displaystyle\int \frac{dx}{\cos^{2}x}$. &  \\
  \hline
  6. & Anıq emes integraldı esaplań: $\displaystyle\int(x^{2}+\frac{1}{x} + \sin x)dx$. &  \\
  \hline
  7. & Korobkada 3 aq, 7 qara shar bar. Tosınnan úsh shar izbe-iz alındı. Izbe-iz alınǵan sharlardıń qara, qara, aq degen izbe-izlikte bolıw itimallıǵın tabıń. &  \\
  \hline
  8. & Dóngelektiń ishine kvadrat sızılǵan. Dóngelektiń ishinen tosınnan belgilengen noqattıń kvadrattıń ishinde jatıw itimallıǵın tabıń. &  \\
  \hline
  9. & Sızıqlı differencial teńlemeniń ulwma sheshimin tabıń: $y' + y =e^{x}$. &  \\
  \hline
  10. & Funkcional qatardıń jıynaqlılıq oblastın jazıń: $\ln x + \ln^{2}x + \ldots + \ln^{n}x + \ldots$. &  \\
  \hline
\end{tabular}
\egroup

\vspace{1cm}

\begin{tabular}{ c c c }
Tuwrı juwaplar sanı: \underline{\hspace{2cm}} & Bahası: \underline{\hspace{2cm}} & Imtixan alıwshınıń qolı: \underline{\hspace{2cm}} \\
\end{tabular}

\newpage

\begin{center}\textbf{29-variant}\end{center}

\bgroup
\def\arraystretch{1.5}
\begin{tabular}{ |m{6cm}|m{10cm}| }
  \hline
  Familiyası hám atı & \\
  \hline
  Fakulteti &\\
  \hline
  Toparı hám tálim baǵdarı & \\
  \hline
\end{tabular}
\egroup

\vspace{0.5cm}

\bgroup
\def\arraystretch{2}
\begin{tabular}{ |l|m{8cm}|m{7cm}| }
  \hline
  №. & Soraw & Juwap \\
  \hline
  1. & Shártli itimallıq formulasın jazıń &  \\
  \hline
  2. & Funkciya qanday usıllarda beriledi &  \\
  \hline
  3. & Eki ózgeriwshili funkciyanıń ekinshi tártipli dara tuwındıları qalay belgilenedi &  \\
  \hline
  4. & Shekli additivlik aksiomasın jazıń &  \\
  \hline
  5. & Racional funkciyanı integrallań: $\displaystyle\int {\frac{5}{(x - 3)(x + 2)}dx}$. &  \\
  \hline
  6. & Anıq integraldı esaplań: $\displaystyle\int_{1}^{3}{\frac{2}{x + 1}dx}$. &  \\
  \hline
  7. & Úsh birdey korobkada aq hám qara sharlar bar. 1-korobkada 5 aq, 8 qara shar, 2-korobkada 3 aq, 4 qara shar, 3-korobkada 2 aq, 3 qara shar bar. Úsh korobkaniń birewinen tosınnan alınǵan bir shar aq bolıw itimallıǵın tabıń. &  \\
  \hline
  8. & "MATEMATIKA" sóziniń háripleri bólek kartochkalarǵa jazılıp jawıp aralastırılıp qoyılǵan. Barlıq kartochkalar tosınnan izbe-iz alınıp ashılıp, alınıw tártibinde stol ústine dizilgende taǵı "MATEMATIKA" sóziniń kelip shıǵıw itimallıǵın tabıń. &  \\
  \hline
  9. & Differencial teńlemeni esaplań: $yy'= 4$. &  \\
  \hline
  10. & Qatardıń qosındısın tabıń: $\displaystyle\sum_{n = 1}^{\infty}\frac{1}{n(n + 3)}$. &  \\
  \hline
\end{tabular}
\egroup

\vspace{1cm}

\begin{tabular}{ c c c }
Tuwrı juwaplar sanı: \underline{\hspace{2cm}} & Bahası: \underline{\hspace{2cm}} & Imtixan alıwshınıń qolı: \underline{\hspace{2cm}} \\
\end{tabular}

\newpage

\begin{center}\textbf{30-variant}\end{center}

\bgroup
\def\arraystretch{1.5}
\begin{tabular}{ |m{6cm}|m{10cm}| }
  \hline
  Familiyası hám atı & \\
  \hline
  Fakulteti &\\
  \hline
  Toparı hám tálim baǵdarı & \\
  \hline
\end{tabular}
\egroup

\vspace{0.5cm}

\bgroup
\def\arraystretch{2}
\begin{tabular}{ |l|m{8cm}|m{7cm}| }
  \hline
  №. & Soraw & Juwap \\
  \hline
  1. & Eki ózgeriwshili funkciyanıń grafigi neden ibarat &  \\
  \hline
  2. & Gruppalaw formulasın jazıń &  \\
  \hline
  3. & Eki ózgeriwshili funkciyalar qalay belgilenedi &  \\
  \hline
  4. & Funkcionallıq qatardıń uluwma kórinisi &  \\
  \hline
  5. & Racional funkciyanı integrallań: $\displaystyle\int {\frac{3}{(x - 1)(x + 2)}dx}$. &  \\
  \hline
  6. & Integraldı esaplań: $\displaystyle\int_{1}^{\infty}{\frac{1}{\left( x + 2 \right)^{2}}dx }$. &  \\
  \hline
  7. & 50 buyımnan ibarat partiyada 3 buyım jaramsız. Tosınnan alınǵan 8 buyımnıń ishinde 1 buyımı jaramsız bolıw itimallıǵın tabıń. &  \\
  \hline
  8. & Tiyindi eki márte taslaǵanda, keminde bir márte san tárepi túsiw itimallıǵın tabıń. &  \\
  \hline
  9. & Differencial teńlemeniń ulıwma sheshimin tabıń: $y'=e^{x}$. &  \\
  \hline
  10. & Funkcional qatardıń jıynaqlılıq oblastın tabıń: $1 + x + \ldots + x^{n} + \ldots$. &  \\
  \hline
\end{tabular}
\egroup

\vspace{1cm}

\begin{tabular}{ c c c }
Tuwrı juwaplar sanı: \underline{\hspace{2cm}} & Bahası: \underline{\hspace{2cm}} & Imtixan alıwshınıń qolı: \underline{\hspace{2cm}} \\
\end{tabular}

\newpage

\begin{center}\textbf{31-variant}\end{center}

\bgroup
\def\arraystretch{1.5}
\begin{tabular}{ |m{6cm}|m{10cm}| }
  \hline
  Familiyası hám atı & \\
  \hline
  Fakulteti &\\
  \hline
  Toparı hám tálim baǵdarı & \\
  \hline
\end{tabular}
\egroup

\vspace{0.5cm}

\bgroup
\def\arraystretch{2}
\begin{tabular}{ |l|m{8cm}|m{7cm}| }
  \hline
  №. & Soraw & Juwap \\
  \hline
  1. & Oń aǵzalı qatarlar ushın jıynaqlılıqtıń Dalamber belgisin jazıń &  \\
  \hline
  2. & Oń aǵzalı qatarlar ushın jıynaqlılıqtıń Koshi belgisin jazıń &  \\
  \hline
  3. & Esaplań $\displaystyle d\left( \int_{}^{}{f(x)dx} \right) = ?$ &  \\
  \hline
  4. & Ózgeriwshileri ajıralǵan differenciallıq teńlemesiniń uluwma kórinisin jazıń &  \\
  \hline
  5. & Anıq emes integraldı esaplań: $\displaystyle\int \left( 10x^{4} + 7x^{6} - 3 \right)dx$. &  \\
  \hline
  6. & Anıq integraldı esaplań: $\displaystyle\int_{0}^{1}{(3x^{2} + 1)dx}$. &  \\
  \hline
  7. & Úsh birdey korobkada aq hám qara sharlar bar. 1-korobkada 5 aq, 8 qara shar, 2-korobkada 3 aq, 4 qara shar, 3-korobkada 2 aq, 3 qara shar bar. Úsh korobkanıń birewinen tosınnan alınǵan bir shar aq bolıw itimallıǵın tabıń. &  \\
  \hline
  8. & Telefon nomerdiń aqırǵı cifrasın umıtıp, tosınnan nomerlerdi tere basladı. Kerekli nomerdi tabıw itimallıǵın esaplań. &  \\
  \hline
  9. & Differencial teńlemeni sheshiń: $y' + xy = 0$. &  \\
  \hline
  10. & Qatardıń qosındısın tabıń: $\displaystyle\sum_{n = 1}^{\infty}\frac{1}{n(n + 1)}$. &  \\
  \hline
\end{tabular}
\egroup

\vspace{1cm}

\begin{tabular}{ c c c }
Tuwrı juwaplar sanı: \underline{\hspace{2cm}} & Bahası: \underline{\hspace{2cm}} & Imtixan alıwshınıń qolı: \underline{\hspace{2cm}} \\
\end{tabular}

\newpage

\begin{center}\textbf{32-variant}\end{center}

\bgroup
\def\arraystretch{1.5}
\begin{tabular}{ |m{6cm}|m{10cm}| }
  \hline
  Familiyası hám atı & \\
  \hline
  Fakulteti &\\
  \hline
  Toparı hám tálim baǵdarı & \\
  \hline
\end{tabular}
\egroup

\vspace{0.5cm}

\bgroup
\def\arraystretch{2}
\begin{tabular}{ |l|m{8cm}|m{7cm}| }
  \hline
  №. & Soraw & Juwap \\
  \hline
  1. & Sızıqlı differenciallıq teńleme kórinisi &  \\
  \hline
  2. & Orın almastırıw formulasın jazıń &  \\
  \hline
  3. & Funkciyanıń $(x_{0}, y_{0})$ noqattaǵı úzliksizlik shártin jazıń &  \\
  \hline
  4. & Itimmallıqtıń geometriyalıq anıqlamasınıń formulasın jazıń &  \\
  \hline
  5. & Integraldı esaplań: $\displaystyle\int {2^{x}dx} $. &  \\
  \hline
  6. & Anıq integraldı esaplań: $\displaystyle\int_{2}^{4}\frac{dx}{x}$. &  \\
  \hline
  7. & Qutıda 15 aq, 18 qara shar bar. Tosınnan alınǵan bir shar aq bolıw itimallıǵın tabıń. &  \\
  \hline
  8. & "BIOLOGIYA" sóziniń háripleri bólek kartochkalarǵa jazılıp jawıp, aralastırılıp qoyılǵan. Barlıq kartochkalar tosınnan izbe-iz alınıp ashılıp, alınıw tártibinde stol ústine dizilgende taǵı "BIOLOGIYA" sóziniń kelip shıǵıw itimallıǵın tabıń. &  \\
  \hline
  9. & Differencial teńlemeniń ulıwma sheshimin tabıń: $xy' - 2y = 0$. &  \\
  \hline
  10. & Funkcional qatardıń jaqınlasıw oblastın tabıń: $\displaystyle x + \frac{x^{2}}{2^{2}} + \ldots + \frac{x^{n}}{n^{2}} + \ldots$. &  \\
  \hline
\end{tabular}
\egroup

\vspace{1cm}

\begin{tabular}{ c c c }
Tuwrı juwaplar sanı: \underline{\hspace{2cm}} & Bahası: \underline{\hspace{2cm}} & Imtixan alıwshınıń qolı: \underline{\hspace{2cm}} \\
\end{tabular}

\newpage

\begin{center}\textbf{33-variant}\end{center}

\bgroup
\def\arraystretch{1.5}
\begin{tabular}{ |m{6cm}|m{10cm}| }
  \hline
  Familiyası hám atı & \\
  \hline
  Fakulteti &\\
  \hline
  Toparı hám tálim baǵdarı & \\
  \hline
\end{tabular}
\egroup

\vspace{0.5cm}

\bgroup
\def\arraystretch{2}
\begin{tabular}{ |l|m{8cm}|m{7cm}| }
  \hline
  №. & Soraw & Juwap \\
  \hline
  1. & Sızıqlı defferencial teńlemeniń uluwma sheshimin jazıń &  \\
  \hline
  2. & Sanlı qatardıń uluwma kórinisin jazıń &  \\
  \hline
  3. & Funkcianıń $(x_{0}, y_{0})$ noqattaǵı úzliksizliginiń formulasın jazıń &  \\
  \hline
  4. & Bernulli differenciallıq teńemesin jazıń &  \\
  \hline
  5. & Integraldı esaplań: $\displaystyle\int (x - 1)^{20}dx$. &  \\
  \hline
  6. & Anıq integraldı esaplań: $\displaystyle\int_{0}^{\pi}\sin xdx$. &  \\
  \hline
  7. & Ídısta 5 aq, 8 qara shar bar. Ídıstan tosınnan izbe-iz 3 shar alındı. Alınǵan sharlar aq, qara, qara degen izbe-izlikte bolıw itimallıǵın tabıń. &  \\
  \hline
  8. & Gruppadaǵı 20 studentten neshe túrli usıl menen 3 náwbetshini saylap alıwǵa boladı?. &  \\
  \hline
  9. & Sızıqlı differerncial teńlemeniń uluwma sheshimin tabıń $y' + y =e^{-x}$. &  \\
  \hline
  10. & Sanlı qatardıń baslanǵısh úsh aǵzasın jazıń: $\displaystyle\sum_{n = 1}^{\infty}\frac{n!}{2^{n}}$. &  \\
  \hline
\end{tabular}
\egroup

\vspace{1cm}

\begin{tabular}{ c c c }
Tuwrı juwaplar sanı: \underline{\hspace{2cm}} & Bahası: \underline{\hspace{2cm}} & Imtixan alıwshınıń qolı: \underline{\hspace{2cm}} \\
\end{tabular}

\newpage

\begin{center}\textbf{34-variant}\end{center}

\bgroup
\def\arraystretch{1.5}
\begin{tabular}{ |m{6cm}|m{10cm}| }
  \hline
  Familiyası hám atı & \\
  \hline
  Fakulteti &\\
  \hline
  Toparı hám tálim baǵdarı & \\
  \hline
\end{tabular}
\egroup

\vspace{0.5cm}

\bgroup
\def\arraystretch{2}
\begin{tabular}{ |l|m{8cm}|m{7cm}| }
  \hline
  №. & Soraw & Juwap \\
  \hline
  1. & Tolıq itimallıqtıń formulasın jazıń &  \\
  \hline
  2. & Eki ózgeriwshli funkciyanıń $M(x_{0} , y_{0})$ noqattaǵı úzliksizliginiń anıqlaması &  \\
  \hline
  3. & Kóp aǵzalını $(x - a)$ ǵa bólgendegi qaldıq nege teń &  \\
  \hline
  4. & Orın awıstırıw formulasın jazıń &  \\
  \hline
  5. & Integraldı esaplań: $\displaystyle\int {\frac{1}{\sin x}dx} $. &  \\
  \hline
  6. & Integraldı esaplań: $\displaystyle\int_{1}^{\infty}{\frac{1}{x^{2}}dx}$. &  \\
  \hline
  7. & Qutıda 5 aq hám 15 qara shar bar. Tosınnan alınǵan bir shardıń aq bolıw itimallıǵın tabıń. &  \\
  \hline
  8. & Eki kubikti bir márte taslaǵanda túsken ochkolardıń qosındısı 4 bolıw itimallıǵın tabıń. &  \\
  \hline
  9. & Sızıqlı differencial teńlemeniń ulwma sheshimin tabıń: $y' + y =e^{x}$. &  \\
  \hline
  10. & Qatardıń qosındısın tabıń: $\displaystyle\sum_{n = 1}^{\infty}\frac{1}{n(n + 3)}$. &  \\
  \hline
\end{tabular}
\egroup

\vspace{1cm}

\begin{tabular}{ c c c }
Tuwrı juwaplar sanı: \underline{\hspace{2cm}} & Bahası: \underline{\hspace{2cm}} & Imtixan alıwshınıń qolı: \underline{\hspace{2cm}} \\
\end{tabular}

\newpage

\begin{center}\textbf{35-variant}\end{center}

\bgroup
\def\arraystretch{1.5}
\begin{tabular}{ |m{6cm}|m{10cm}| }
  \hline
  Familiyası hám atı & \\
  \hline
  Fakulteti &\\
  \hline
  Toparı hám tálim baǵdarı & \\
  \hline
\end{tabular}
\egroup

\vspace{0.5cm}

\bgroup
\def\arraystretch{2}
\begin{tabular}{ |l|m{8cm}|m{7cm}| }
  \hline
  №. & Soraw & Juwap \\
  \hline
  1. & Anıq integraldı esaplawdıń Nyuton-Leybnis formulasın jazıń &  \\
  \hline
  2. & Itimallıqtıń klassikalıq anıqlamasınıń formulasın keltiriń &  \\
  \hline
  3. & Itimallıqtıń mánisler oblastın jazıń &  \\
  \hline
  4. & Bóleklep inegrallaw formulasın jazıń &  \\
  \hline
  5. & Integraldı esaplań: $\displaystyle\int (x + \sin x)dx$. &  \\
  \hline
  6. & Esaplań: $\displaystyle\int_{1}^{2}{e^{x}dx}$. &  \\
  \hline
  7. & Korobkada 3 aq, 7 qara shar bar. Tosınnan úsh shar izbe-iz alındı. Izbe-iz alınǵan sharlardıń qara, qara, aq degen izbe-izlikte bolıw itimallıǵın tabıń. &  \\
  \hline
  8. & Telefon nomerdiń aqırǵı eki cifrasın umıtıp, tosınnan nomerlerdi tere basladı. Kerekli nomerdi tabıw itimallıǵın esaplań. &  \\
  \hline
  9. & Differencial teńlemeni esaplań: $yy'= 4$. &  \\
  \hline
  10. & Qatardıń jıyındısın esaplań: $\displaystyle\sum_{n = 1}^{\infty}\frac{1}{(2n - 1)(2n + 1)}$. &  \\
  \hline
\end{tabular}
\egroup

\vspace{1cm}

\begin{tabular}{ c c c }
Tuwrı juwaplar sanı: \underline{\hspace{2cm}} & Bahası: \underline{\hspace{2cm}} & Imtixan alıwshınıń qolı: \underline{\hspace{2cm}} \\
\end{tabular}

\newpage

\begin{center}\textbf{36-variant}\end{center}

\bgroup
\def\arraystretch{1.5}
\begin{tabular}{ |m{6cm}|m{10cm}| }
  \hline
  Familiyası hám atı & \\
  \hline
  Fakulteti &\\
  \hline
  Toparı hám tálim baǵdarı & \\
  \hline
\end{tabular}
\egroup

\vspace{0.5cm}

\bgroup
\def\arraystretch{2}
\begin{tabular}{ |l|m{8cm}|m{7cm}| }
  \hline
  №. & Soraw & Juwap \\
  \hline
  1. & $\displaystyle\int dF(x)$ nege teń &  \\
  \hline
  2. & Sızıqlı differenciallıq teńlemeniń uluwma kórinisin jazıń &  \\
  \hline
  3. & Eki ózgeriwshili funkciyanıń ekstremumınıń zárúrli shárti &  \\
  \hline
  4. & Funkciyanıń anıqlanıw oblastı qalay belgilenedi &  \\
  \hline
  5. & Anıq emes integraldı esaplań: $\displaystyle\int e^{x}dx$. &  \\
  \hline
  6. & Anıq itegraldı esaplań: $\displaystyle\int_{1}^{3}{\frac{2}{x + 1}dx}$. &  \\
  \hline
  7. & Úsh birdey korobkada aq hám qara sharlar bar. 1-korobkada 5 aq, 8 qara shar, 2-korobkada 3 aq, 4 qara shar, 3-korobkada 2 aq, 3 qara shar bar. Úsh korobkaniń birewinen tosınnan alınǵan bir shar aq bolıw itimallıǵın tabıń. &  \\
  \hline
  8. & Dóngelektiń ishine kvadrat sızılǵan. Dóngelektiń ishinen tosınnan belgilengen noqattıń kvadrattıń ishinde jatıw itimallıǵın tabıń. &  \\
  \hline
  9. & Differencial teńlemeniń ulıwma sheshimin tabıń: $y'=e^{x}$. &  \\
  \hline
  10. & Funkcional qatardıń jıynaqlılıq oblastın jazıń: $\ln x + \ln^{2}x + \ldots + \ln^{n}x + \ldots$. &  \\
  \hline
\end{tabular}
\egroup

\vspace{1cm}

\begin{tabular}{ c c c }
Tuwrı juwaplar sanı: \underline{\hspace{2cm}} & Bahası: \underline{\hspace{2cm}} & Imtixan alıwshınıń qolı: \underline{\hspace{2cm}} \\
\end{tabular}

\newpage

\begin{center}\textbf{37-variant}\end{center}

\bgroup
\def\arraystretch{1.5}
\begin{tabular}{ |m{6cm}|m{10cm}| }
  \hline
  Familiyası hám atı & \\
  \hline
  Fakulteti &\\
  \hline
  Toparı hám tálim baǵdarı & \\
  \hline
\end{tabular}
\egroup

\vspace{0.5cm}

\bgroup
\def\arraystretch{2}
\begin{tabular}{ |l|m{8cm}|m{7cm}| }
  \hline
  №. & Soraw & Juwap \\
  \hline
  1. & Itimallıq keńisligin jazıń &  \\
  \hline
  2. & Eger $\displaystyle\sum_{n = 1}^{\infty}a_{n} = A, \sum_{n = 1}^{\infty}b_{n} = B$ bolsa, onda $\displaystyle\sum_{n = 1}^{\infty}\left( a_{n} - b_{n} \right)$ &  \\
  \hline
  3. & Eki ózgeriwshili funkciyanıń anıqlanıw oblastı qay jerde jaylasadı &  \\
  \hline
  4. & Múmkin emes waqıyanıń itimaıllıǵı nege teń &  \\
  \hline
  5. & Esaplań: $\displaystyle\int \left( x^{4}-\frac{1}{x} \right)dx$. &  \\
  \hline
  6. & Anıq integraldı esaplań: $\displaystyle\int_{0}^{\frac{\pi}{2}}\cos xdx$. &  \\
  \hline
  7. & 50 buyımnan ibarat partiyada 3 buyım jaramsız. Tosınnan alınǵan 8 buyımnıń ishinde 1 buyımı jaramsız bolıw itimallıǵın tabıń. &  \\
  \hline
  8. & "MATEMATIKA" sóziniń háripleri bólek kartochkalarǵa jazılıp jawıp aralastırılıp qoyılǵan. Barlıq kartochkalar tosınnan izbe-iz alınıp ashılıp, alınıw tártibinde stol ústine dizilgende taǵı "MATEMATIKA" sóziniń kelip shıǵıw itimallıǵın tabıń. &  \\
  \hline
  9. & Differencial teńlemeni sheshiń: $y' + xy = 0$. &  \\
  \hline
  10. & Qatardıń qosındısın tabıń: $\displaystyle\sum_{n = 1}^{\infty}\frac{1}{n(n + 3)}$. &  \\
  \hline
\end{tabular}
\egroup

\vspace{1cm}

\begin{tabular}{ c c c }
Tuwrı juwaplar sanı: \underline{\hspace{2cm}} & Bahası: \underline{\hspace{2cm}} & Imtixan alıwshınıń qolı: \underline{\hspace{2cm}} \\
\end{tabular}

\newpage

\begin{center}\textbf{38-variant}\end{center}

\bgroup
\def\arraystretch{1.5}
\begin{tabular}{ |m{6cm}|m{10cm}| }
  \hline
  Familiyası hám atı & \\
  \hline
  Fakulteti &\\
  \hline
  Toparı hám tálim baǵdarı & \\
  \hline
\end{tabular}
\egroup

\vspace{0.5cm}

\bgroup
\def\arraystretch{2}
\begin{tabular}{ |l|m{8cm}|m{7cm}| }
  \hline
  №. & Soraw & Juwap \\
  \hline
  1. & Eger $\displaystyle\sum_{n = 1}^{\infty}a_{n} = A, \sum_{n = 1}^{\infty}b_{n} = B$ bolsa, onda $\displaystyle\sum_{n = 1}^{\infty}\left( a_{n} + b_{n} \right)$ &  \\
  \hline
  2. & Esaplań $\displaystyle \left( \int_{}^{}{f(x)dx} \right)^\prime = ?$ &  \\
  \hline
  3. & Eki ózgeriwshili funkciyanıń ekinshi tártipli aralas tuwındıları qalay belgilenedi &  \\
  \hline
  4. & Nyuton-Leybnis formulasın jazıń &  \\
  \hline
  5. & Anıq emes integraldı esaplań: $\displaystyle\int \frac{dx}{\cos^{2}x}$. &  \\
  \hline
  6. & Anıq integraldı esaplań: $\displaystyle\int_{-\frac{\pi}{4}}^{0}\frac{dx}{\cos^{2}x}$. &  \\
  \hline
  7. & Úsh birdey korobkada aq hám qara sharlar bar. 1-korobkada 5 aq, 8 qara shar, 2-korobkada 3 aq, 4 qara shar, 3-korobkada 2 aq, 3 qara shar bar. Úsh korobkanıń birewinen tosınnan alınǵan bir shar aq bolıw itimallıǵın tabıń. &  \\
  \hline
  8. & Tiyindi eki márte taslaǵanda, keminde bir márte san tárepi túsiw itimallıǵın tabıń. &  \\
  \hline
  9. & Differencial teńlemeniń ulıwma sheshimin tabıń: $xy' - 2y = 0$. &  \\
  \hline
  10. & Funkcional qatardıń jıynaqlılıq oblastın tabıń: $1 + x + \ldots + x^{n} + \ldots$. &  \\
  \hline
\end{tabular}
\egroup

\vspace{1cm}

\begin{tabular}{ c c c }
Tuwrı juwaplar sanı: \underline{\hspace{2cm}} & Bahası: \underline{\hspace{2cm}} & Imtixan alıwshınıń qolı: \underline{\hspace{2cm}} \\
\end{tabular}

\newpage

\begin{center}\textbf{39-variant}\end{center}

\bgroup
\def\arraystretch{1.5}
\begin{tabular}{ |m{6cm}|m{10cm}| }
  \hline
  Familiyası hám atı & \\
  \hline
  Fakulteti &\\
  \hline
  Toparı hám tálim baǵdarı & \\
  \hline
\end{tabular}
\egroup

\vspace{0.5cm}

\bgroup
\def\arraystretch{2}
\begin{tabular}{ |l|m{8cm}|m{7cm}| }
  \hline
  №. & Soraw & Juwap \\
  \hline
  1. & Anıq integraldı esaplawdıń Nyuton-Leybnic formulasın jazıń &  \\
  \hline
  2. & $\displaystyle\int k \cdot f(x)dx = ?$ &  \\
  \hline
  3. & Ózgeriwshini almastırıp integrallaw usılıniń formulasın jazıń. &  \\
  \hline
  4. & Eki ózgeriwshili funkciyanıń birinshi tártipli dara tuwındıları qalay belgilenedi &  \\
  \hline
  5. & Racional funkciyanı integrallań: $\displaystyle\int {\frac{5}{(x - 3)(x + 2)}dx}$. &  \\
  \hline
  6. & Anıq emes integraldı esaplań: $\displaystyle\int(x^{2}+\frac{1}{x} + \sin x)dx$. &  \\
  \hline
  7. & Qutıda 15 aq, 18 qara shar bar. Tosınnan alınǵan bir shar aq bolıw itimallıǵın tabıń. &  \\
  \hline
  8. & Telefon nomerdiń aqırǵı cifrasın umıtıp, tosınnan nomerlerdi tere basladı. Kerekli nomerdi tabıw itimallıǵın esaplań. &  \\
  \hline
  9. & Sızıqlı differerncial teńlemeniń uluwma sheshimin tabıń $y' + y =e^{-x}$. &  \\
  \hline
  10. & Qatardıń qosındısın tabıń: $\displaystyle\sum_{n = 1}^{\infty}\frac{1}{n(n + 1)}$. &  \\
  \hline
\end{tabular}
\egroup

\vspace{1cm}

\begin{tabular}{ c c c }
Tuwrı juwaplar sanı: \underline{\hspace{2cm}} & Bahası: \underline{\hspace{2cm}} & Imtixan alıwshınıń qolı: \underline{\hspace{2cm}} \\
\end{tabular}

\newpage

\begin{center}\textbf{40-variant}\end{center}

\bgroup
\def\arraystretch{1.5}
\begin{tabular}{ |m{6cm}|m{10cm}| }
  \hline
  Familiyası hám atı & \\
  \hline
  Fakulteti &\\
  \hline
  Toparı hám tálim baǵdarı & \\
  \hline
\end{tabular}
\egroup

\vspace{0.5cm}

\bgroup
\def\arraystretch{2}
\begin{tabular}{ |l|m{8cm}|m{7cm}| }
  \hline
  №. & Soraw & Juwap \\
  \hline
  1. & $(x_{0} , y_{0})$ noqattıń $\varepsilon$ dógeregi qalay belgilenedi &  \\
  \hline
  2. & Bayes formulasın jazıń &  \\
  \hline
  3. & Funkcianıń $(x_{0}, y_{0})$ noqattaǵı tuwındısınıń formulasın jazıń &  \\
  \hline
  4. & Isenimli waqıyanıń itimallıǵı nege teń &  \\
  \hline
  5. & Racional funkciyanı integrallań: $\displaystyle\int {\frac{3}{(x - 1)(x + 2)}dx}$. &  \\
  \hline
  6. & Anıq integraldı esaplań: $\displaystyle\int_{1}^{3}{\frac{2}{x + 1}dx}$. &  \\
  \hline
  7. & Ídısta 5 aq, 8 qara shar bar. Ídıstan tosınnan izbe-iz 3 shar alındı. Alınǵan sharlar aq, qara, qara degen izbe-izlikte bolıw itimallıǵın tabıń. &  \\
  \hline
  8. & "BIOLOGIYA" sóziniń háripleri bólek kartochkalarǵa jazılıp jawıp, aralastırılıp qoyılǵan. Barlıq kartochkalar tosınnan izbe-iz alınıp ashılıp, alınıw tártibinde stol ústine dizilgende taǵı "BIOLOGIYA" sóziniń kelip shıǵıw itimallıǵın tabıń. &  \\
  \hline
  9. & Sızıqlı differencial teńlemeniń ulwma sheshimin tabıń: $y' + y =e^{x}$. &  \\
  \hline
  10. & Funkcional qatardıń jaqınlasıw oblastın tabıń: $\displaystyle x + \frac{x^{2}}{2^{2}} + \ldots + \frac{x^{n}}{n^{2}} + \ldots$. &  \\
  \hline
\end{tabular}
\egroup

\vspace{1cm}

\begin{tabular}{ c c c }
Tuwrı juwaplar sanı: \underline{\hspace{2cm}} & Bahası: \underline{\hspace{2cm}} & Imtixan alıwshınıń qolı: \underline{\hspace{2cm}} \\
\end{tabular}

\newpage

\begin{center}\textbf{41-variant}\end{center}

\bgroup
\def\arraystretch{1.5}
\begin{tabular}{ |m{6cm}|m{10cm}| }
  \hline
  Familiyası hám atı & \\
  \hline
  Fakulteti &\\
  \hline
  Toparı hám tálim baǵdarı & \\
  \hline
\end{tabular}
\egroup

\vspace{0.5cm}

\bgroup
\def\arraystretch{2}
\begin{tabular}{ |l|m{8cm}|m{7cm}| }
  \hline
  №. & Soraw & Juwap \\
  \hline
  1. & $n$-dárejeli kóp aǵzalınıń uluwma kórinisi &  \\
  \hline
  2. & Eki ózgeriwshili funkciyanıń tolıq ósimi &  \\
  \hline
  3. & Shártli itimallıq formulasın jazıń &  \\
  \hline
  4. & Funkciya qanday usıllarda beriledi &  \\
  \hline
  5. & Anıq emes integraldı esaplań: $\displaystyle\int \left( 10x^{4} + 7x^{6} - 3 \right)dx$. &  \\
  \hline
  6. & Integraldı esaplań: $\displaystyle\int_{1}^{\infty}{\frac{1}{\left( x + 2 \right)^{2}}dx }$. &  \\
  \hline
  7. & Qutıda 5 aq hám 15 qara shar bar. Tosınnan alınǵan bir shardıń aq bolıw itimallıǵın tabıń. &  \\
  \hline
  8. & Gruppadaǵı 20 studentten neshe túrli usıl menen 3 náwbetshini saylap alıwǵa boladı?. &  \\
  \hline
  9. & Differencial teńlemeni esaplań: $yy'= 4$. &  \\
  \hline
  10. & Sanlı qatardıń baslanǵısh úsh aǵzasın jazıń: $\displaystyle\sum_{n = 1}^{\infty}\frac{n!}{2^{n}}$. &  \\
  \hline
\end{tabular}
\egroup

\vspace{1cm}

\begin{tabular}{ c c c }
Tuwrı juwaplar sanı: \underline{\hspace{2cm}} & Bahası: \underline{\hspace{2cm}} & Imtixan alıwshınıń qolı: \underline{\hspace{2cm}} \\
\end{tabular}

\newpage

\begin{center}\textbf{42-variant}\end{center}

\bgroup
\def\arraystretch{1.5}
\begin{tabular}{ |m{6cm}|m{10cm}| }
  \hline
  Familiyası hám atı & \\
  \hline
  Fakulteti &\\
  \hline
  Toparı hám tálim baǵdarı & \\
  \hline
\end{tabular}
\egroup

\vspace{0.5cm}

\bgroup
\def\arraystretch{2}
\begin{tabular}{ |l|m{8cm}|m{7cm}| }
  \hline
  №. & Soraw & Juwap \\
  \hline
  1. & Eki ózgeriwshili funkciyanıń ekinshi tártipli dara tuwındıları qalay belgilenedi &  \\
  \hline
  2. & Shekli additivlik aksiomasın jazıń &  \\
  \hline
  3. & Eki ózgeriwshili funkciyanıń grafigi neden ibarat &  \\
  \hline
  4. & Gruppalaw formulasın jazıń &  \\
  \hline
  5. & Integraldı esaplań: $\displaystyle\int {2^{x}dx} $. &  \\
  \hline
  6. & Anıq integraldı esaplań: $\displaystyle\int_{0}^{1}{(3x^{2} + 1)dx}$. &  \\
  \hline
  7. & Korobkada 3 aq, 7 qara shar bar. Tosınnan úsh shar izbe-iz alındı. Izbe-iz alınǵan sharlardıń qara, qara, aq degen izbe-izlikte bolıw itimallıǵın tabıń. &  \\
  \hline
  8. & Eki kubikti bir márte taslaǵanda túsken ochkolardıń qosındısı 4 bolıw itimallıǵın tabıń. &  \\
  \hline
  9. & Differencial teńlemeniń ulıwma sheshimin tabıń: $y'=e^{x}$. &  \\
  \hline
  10. & Qatardıń qosındısın tabıń: $\displaystyle\sum_{n = 1}^{\infty}\frac{1}{n(n + 3)}$. &  \\
  \hline
\end{tabular}
\egroup

\vspace{1cm}

\begin{tabular}{ c c c }
Tuwrı juwaplar sanı: \underline{\hspace{2cm}} & Bahası: \underline{\hspace{2cm}} & Imtixan alıwshınıń qolı: \underline{\hspace{2cm}} \\
\end{tabular}

\newpage

\begin{center}\textbf{43-variant}\end{center}

\bgroup
\def\arraystretch{1.5}
\begin{tabular}{ |m{6cm}|m{10cm}| }
  \hline
  Familiyası hám atı & \\
  \hline
  Fakulteti &\\
  \hline
  Toparı hám tálim baǵdarı & \\
  \hline
\end{tabular}
\egroup

\vspace{0.5cm}

\bgroup
\def\arraystretch{2}
\begin{tabular}{ |l|m{8cm}|m{7cm}| }
  \hline
  №. & Soraw & Juwap \\
  \hline
  1. & Eki ózgeriwshili funkciyalar qalay belgilenedi &  \\
  \hline
  2. & Funkcionallıq qatardıń uluwma kórinisi &  \\
  \hline
  3. & Oń aǵzalı qatarlar ushın jıynaqlılıqtıń Dalamber belgisin jazıń &  \\
  \hline
  4. & Oń aǵzalı qatarlar ushın jıynaqlılıqtıń Koshi belgisin jazıń &  \\
  \hline
  5. & Integraldı esaplań: $\displaystyle\int (x - 1)^{20}dx$. &  \\
  \hline
  6. & Anıq integraldı esaplań: $\displaystyle\int_{2}^{4}\frac{dx}{x}$. &  \\
  \hline
  7. & Úsh birdey korobkada aq hám qara sharlar bar. 1-korobkada 5 aq, 8 qara shar, 2-korobkada 3 aq, 4 qara shar, 3-korobkada 2 aq, 3 qara shar bar. Úsh korobkaniń birewinen tosınnan alınǵan bir shar aq bolıw itimallıǵın tabıń. &  \\
  \hline
  8. & Telefon nomerdiń aqırǵı eki cifrasın umıtıp, tosınnan nomerlerdi tere basladı. Kerekli nomerdi tabıw itimallıǵın esaplań. &  \\
  \hline
  9. & Differencial teńlemeni sheshiń: $y' + xy = 0$. &  \\
  \hline
  10. & Qatardıń jıyındısın esaplań: $\displaystyle\sum_{n = 1}^{\infty}\frac{1}{(2n - 1)(2n + 1)}$. &  \\
  \hline
\end{tabular}
\egroup

\vspace{1cm}

\begin{tabular}{ c c c }
Tuwrı juwaplar sanı: \underline{\hspace{2cm}} & Bahası: \underline{\hspace{2cm}} & Imtixan alıwshınıń qolı: \underline{\hspace{2cm}} \\
\end{tabular}

\newpage

\begin{center}\textbf{44-variant}\end{center}

\bgroup
\def\arraystretch{1.5}
\begin{tabular}{ |m{6cm}|m{10cm}| }
  \hline
  Familiyası hám atı & \\
  \hline
  Fakulteti &\\
  \hline
  Toparı hám tálim baǵdarı & \\
  \hline
\end{tabular}
\egroup

\vspace{0.5cm}

\bgroup
\def\arraystretch{2}
\begin{tabular}{ |l|m{8cm}|m{7cm}| }
  \hline
  №. & Soraw & Juwap \\
  \hline
  1. & Esaplań $\displaystyle d\left( \int_{}^{}{f(x)dx} \right) = ?$ &  \\
  \hline
  2. & Ózgeriwshileri ajıralǵan differenciallıq teńlemesiniń uluwma kórinisin jazıń &  \\
  \hline
  3. & Sızıqlı differenciallıq teńleme kórinisi &  \\
  \hline
  4. & Orın almastırıw formulasın jazıń &  \\
  \hline
  5. & Integraldı esaplań: $\displaystyle\int {\frac{1}{\sin x}dx} $. &  \\
  \hline
  6. & Anıq integraldı esaplań: $\displaystyle\int_{0}^{\pi}\sin xdx$. &  \\
  \hline
  7. & 50 buyımnan ibarat partiyada 3 buyım jaramsız. Tosınnan alınǵan 8 buyımnıń ishinde 1 buyımı jaramsız bolıw itimallıǵın tabıń. &  \\
  \hline
  8. & Dóngelektiń ishine kvadrat sızılǵan. Dóngelektiń ishinen tosınnan belgilengen noqattıń kvadrattıń ishinde jatıw itimallıǵın tabıń. &  \\
  \hline
  9. & Differencial teńlemeniń ulıwma sheshimin tabıń: $xy' - 2y = 0$. &  \\
  \hline
  10. & Funkcional qatardıń jıynaqlılıq oblastın jazıń: $\ln x + \ln^{2}x + \ldots + \ln^{n}x + \ldots$. &  \\
  \hline
\end{tabular}
\egroup

\vspace{1cm}

\begin{tabular}{ c c c }
Tuwrı juwaplar sanı: \underline{\hspace{2cm}} & Bahası: \underline{\hspace{2cm}} & Imtixan alıwshınıń qolı: \underline{\hspace{2cm}} \\
\end{tabular}

\newpage

\begin{center}\textbf{45-variant}\end{center}

\bgroup
\def\arraystretch{1.5}
\begin{tabular}{ |m{6cm}|m{10cm}| }
  \hline
  Familiyası hám atı & \\
  \hline
  Fakulteti &\\
  \hline
  Toparı hám tálim baǵdarı & \\
  \hline
\end{tabular}
\egroup

\vspace{0.5cm}

\bgroup
\def\arraystretch{2}
\begin{tabular}{ |l|m{8cm}|m{7cm}| }
  \hline
  №. & Soraw & Juwap \\
  \hline
  1. & Funkciyanıń $(x_{0}, y_{0})$ noqattaǵı úzliksizlik shártin jazıń &  \\
  \hline
  2. & Itimmallıqtıń geometriyalıq anıqlamasınıń formulasın jazıń &  \\
  \hline
  3. & Sızıqlı defferencial teńlemeniń uluwma sheshimin jazıń &  \\
  \hline
  4. & Sanlı qatardıń uluwma kórinisin jazıń &  \\
  \hline
  5. & Integraldı esaplań: $\displaystyle\int (x + \sin x)dx$. &  \\
  \hline
  6. & Integraldı esaplań: $\displaystyle\int_{1}^{\infty}{\frac{1}{x^{2}}dx}$. &  \\
  \hline
  7. & Úsh birdey korobkada aq hám qara sharlar bar. 1-korobkada 5 aq, 8 qara shar, 2-korobkada 3 aq, 4 qara shar, 3-korobkada 2 aq, 3 qara shar bar. Úsh korobkanıń birewinen tosınnan alınǵan bir shar aq bolıw itimallıǵın tabıń. &  \\
  \hline
  8. & "MATEMATIKA" sóziniń háripleri bólek kartochkalarǵa jazılıp jawıp aralastırılıp qoyılǵan. Barlıq kartochkalar tosınnan izbe-iz alınıp ashılıp, alınıw tártibinde stol ústine dizilgende taǵı "MATEMATIKA" sóziniń kelip shıǵıw itimallıǵın tabıń. &  \\
  \hline
  9. & Sızıqlı differerncial teńlemeniń uluwma sheshimin tabıń $y' + y =e^{-x}$. &  \\
  \hline
  10. & Qatardıń qosındısın tabıń: $\displaystyle\sum_{n = 1}^{\infty}\frac{1}{n(n + 3)}$. &  \\
  \hline
\end{tabular}
\egroup

\vspace{1cm}

\begin{tabular}{ c c c }
Tuwrı juwaplar sanı: \underline{\hspace{2cm}} & Bahası: \underline{\hspace{2cm}} & Imtixan alıwshınıń qolı: \underline{\hspace{2cm}} \\
\end{tabular}

\newpage

\begin{center}\textbf{46-variant}\end{center}

\bgroup
\def\arraystretch{1.5}
\begin{tabular}{ |m{6cm}|m{10cm}| }
  \hline
  Familiyası hám atı & \\
  \hline
  Fakulteti &\\
  \hline
  Toparı hám tálim baǵdarı & \\
  \hline
\end{tabular}
\egroup

\vspace{0.5cm}

\bgroup
\def\arraystretch{2}
\begin{tabular}{ |l|m{8cm}|m{7cm}| }
  \hline
  №. & Soraw & Juwap \\
  \hline
  1. & Funkcianıń $(x_{0}, y_{0})$ noqattaǵı úzliksizliginiń formulasın jazıń &  \\
  \hline
  2. & Bernulli differenciallıq teńemesin jazıń &  \\
  \hline
  3. & Tolıq itimallıqtıń formulasın jazıń &  \\
  \hline
  4. & Eki ózgeriwshli funkciyanıń $M(x_{0} , y_{0})$ noqattaǵı úzliksizliginiń anıqlaması &  \\
  \hline
  5. & Anıq emes integraldı esaplań: $\displaystyle\int e^{x}dx$. &  \\
  \hline
  6. & Esaplań: $\displaystyle\int_{1}^{2}{e^{x}dx}$. &  \\
  \hline
  7. & Qutıda 15 aq, 18 qara shar bar. Tosınnan alınǵan bir shar aq bolıw itimallıǵın tabıń. &  \\
  \hline
  8. & Tiyindi eki márte taslaǵanda, keminde bir márte san tárepi túsiw itimallıǵın tabıń. &  \\
  \hline
  9. & Sızıqlı differencial teńlemeniń ulwma sheshimin tabıń: $y' + y =e^{x}$. &  \\
  \hline
  10. & Funkcional qatardıń jıynaqlılıq oblastın tabıń: $1 + x + \ldots + x^{n} + \ldots$. &  \\
  \hline
\end{tabular}
\egroup

\vspace{1cm}

\begin{tabular}{ c c c }
Tuwrı juwaplar sanı: \underline{\hspace{2cm}} & Bahası: \underline{\hspace{2cm}} & Imtixan alıwshınıń qolı: \underline{\hspace{2cm}} \\
\end{tabular}

\newpage

\begin{center}\textbf{47-variant}\end{center}

\bgroup
\def\arraystretch{1.5}
\begin{tabular}{ |m{6cm}|m{10cm}| }
  \hline
  Familiyası hám atı & \\
  \hline
  Fakulteti &\\
  \hline
  Toparı hám tálim baǵdarı & \\
  \hline
\end{tabular}
\egroup

\vspace{0.5cm}

\bgroup
\def\arraystretch{2}
\begin{tabular}{ |l|m{8cm}|m{7cm}| }
  \hline
  №. & Soraw & Juwap \\
  \hline
  1. & Kóp aǵzalını $(x - a)$ ǵa bólgendegi qaldıq nege teń &  \\
  \hline
  2. & Orın awıstırıw formulasın jazıń &  \\
  \hline
  3. & Anıq integraldı esaplawdıń Nyuton-Leybnis formulasın jazıń &  \\
  \hline
  4. & Itimallıqtıń klassikalıq anıqlamasınıń formulasın keltiriń &  \\
  \hline
  5. & Esaplań: $\displaystyle\int \left( x^{4}-\frac{1}{x} \right)dx$. &  \\
  \hline
  6. & Anıq itegraldı esaplań: $\displaystyle\int_{1}^{3}{\frac{2}{x + 1}dx}$. &  \\
  \hline
  7. & Ídısta 5 aq, 8 qara shar bar. Ídıstan tosınnan izbe-iz 3 shar alındı. Alınǵan sharlar aq, qara, qara degen izbe-izlikte bolıw itimallıǵın tabıń. &  \\
  \hline
  8. & Telefon nomerdiń aqırǵı cifrasın umıtıp, tosınnan nomerlerdi tere basladı. Kerekli nomerdi tabıw itimallıǵın esaplań. &  \\
  \hline
  9. & Differencial teńlemeni esaplań: $yy'= 4$. &  \\
  \hline
  10. & Qatardıń qosındısın tabıń: $\displaystyle\sum_{n = 1}^{\infty}\frac{1}{n(n + 1)}$. &  \\
  \hline
\end{tabular}
\egroup

\vspace{1cm}

\begin{tabular}{ c c c }
Tuwrı juwaplar sanı: \underline{\hspace{2cm}} & Bahası: \underline{\hspace{2cm}} & Imtixan alıwshınıń qolı: \underline{\hspace{2cm}} \\
\end{tabular}

\newpage

\begin{center}\textbf{48-variant}\end{center}

\bgroup
\def\arraystretch{1.5}
\begin{tabular}{ |m{6cm}|m{10cm}| }
  \hline
  Familiyası hám atı & \\
  \hline
  Fakulteti &\\
  \hline
  Toparı hám tálim baǵdarı & \\
  \hline
\end{tabular}
\egroup

\vspace{0.5cm}

\bgroup
\def\arraystretch{2}
\begin{tabular}{ |l|m{8cm}|m{7cm}| }
  \hline
  №. & Soraw & Juwap \\
  \hline
  1. & Itimallıqtıń mánisler oblastın jazıń &  \\
  \hline
  2. & Bóleklep inegrallaw formulasın jazıń &  \\
  \hline
  3. & $\displaystyle\int dF(x)$ nege teń &  \\
  \hline
  4. & Sızıqlı differenciallıq teńlemeniń uluwma kórinisin jazıń &  \\
  \hline
  5. & Anıq emes integraldı esaplań: $\displaystyle\int \frac{dx}{\cos^{2}x}$. &  \\
  \hline
  6. & Anıq integraldı esaplań: $\displaystyle\int_{0}^{\frac{\pi}{2}}\cos xdx$. &  \\
  \hline
  7. & Qutıda 5 aq hám 15 qara shar bar. Tosınnan alınǵan bir shardıń aq bolıw itimallıǵın tabıń. &  \\
  \hline
  8. & "BIOLOGIYA" sóziniń háripleri bólek kartochkalarǵa jazılıp jawıp, aralastırılıp qoyılǵan. Barlıq kartochkalar tosınnan izbe-iz alınıp ashılıp, alınıw tártibinde stol ústine dizilgende taǵı "BIOLOGIYA" sóziniń kelip shıǵıw itimallıǵın tabıń. &  \\
  \hline
  9. & Differencial teńlemeniń ulıwma sheshimin tabıń: $y'=e^{x}$. &  \\
  \hline
  10. & Funkcional qatardıń jaqınlasıw oblastın tabıń: $\displaystyle x + \frac{x^{2}}{2^{2}} + \ldots + \frac{x^{n}}{n^{2}} + \ldots$. &  \\
  \hline
\end{tabular}
\egroup

\vspace{1cm}

\begin{tabular}{ c c c }
Tuwrı juwaplar sanı: \underline{\hspace{2cm}} & Bahası: \underline{\hspace{2cm}} & Imtixan alıwshınıń qolı: \underline{\hspace{2cm}} \\
\end{tabular}

\newpage

\begin{center}\textbf{49-variant}\end{center}

\bgroup
\def\arraystretch{1.5}
\begin{tabular}{ |m{6cm}|m{10cm}| }
  \hline
  Familiyası hám atı & \\
  \hline
  Fakulteti &\\
  \hline
  Toparı hám tálim baǵdarı & \\
  \hline
\end{tabular}
\egroup

\vspace{0.5cm}

\bgroup
\def\arraystretch{2}
\begin{tabular}{ |l|m{8cm}|m{7cm}| }
  \hline
  №. & Soraw & Juwap \\
  \hline
  1. & Eki ózgeriwshili funkciyanıń ekstremumınıń zárúrli shárti &  \\
  \hline
  2. & Funkciyanıń anıqlanıw oblastı qalay belgilenedi &  \\
  \hline
  3. & Itimallıq keńisligin jazıń &  \\
  \hline
  4. & Eger $\displaystyle\sum_{n = 1}^{\infty}a_{n} = A, \sum_{n = 1}^{\infty}b_{n} = B$ bolsa, onda $\displaystyle\sum_{n = 1}^{\infty}\left( a_{n} - b_{n} \right)$ &  \\
  \hline
  5. & Racional funkciyanı integrallań: $\displaystyle\int {\frac{5}{(x - 3)(x + 2)}dx}$. &  \\
  \hline
  6. & Anıq integraldı esaplań: $\displaystyle\int_{-\frac{\pi}{4}}^{0}\frac{dx}{\cos^{2}x}$. &  \\
  \hline
  7. & Korobkada 3 aq, 7 qara shar bar. Tosınnan úsh shar izbe-iz alındı. Izbe-iz alınǵan sharlardıń qara, qara, aq degen izbe-izlikte bolıw itimallıǵın tabıń. &  \\
  \hline
  8. & Gruppadaǵı 20 studentten neshe túrli usıl menen 3 náwbetshini saylap alıwǵa boladı?. &  \\
  \hline
  9. & Differencial teńlemeni sheshiń: $y' + xy = 0$. &  \\
  \hline
  10. & Sanlı qatardıń baslanǵısh úsh aǵzasın jazıń: $\displaystyle\sum_{n = 1}^{\infty}\frac{n!}{2^{n}}$. &  \\
  \hline
\end{tabular}
\egroup

\vspace{1cm}

\begin{tabular}{ c c c }
Tuwrı juwaplar sanı: \underline{\hspace{2cm}} & Bahası: \underline{\hspace{2cm}} & Imtixan alıwshınıń qolı: \underline{\hspace{2cm}} \\
\end{tabular}

\newpage

\begin{center}\textbf{50-variant}\end{center}

\bgroup
\def\arraystretch{1.5}
\begin{tabular}{ |m{6cm}|m{10cm}| }
  \hline
  Familiyası hám atı & \\
  \hline
  Fakulteti &\\
  \hline
  Toparı hám tálim baǵdarı & \\
  \hline
\end{tabular}
\egroup

\vspace{0.5cm}

\bgroup
\def\arraystretch{2}
\begin{tabular}{ |l|m{8cm}|m{7cm}| }
  \hline
  №. & Soraw & Juwap \\
  \hline
  1. & Eki ózgeriwshili funkciyanıń anıqlanıw oblastı qay jerde jaylasadı &  \\
  \hline
  2. & Múmkin emes waqıyanıń itimaıllıǵı nege teń &  \\
  \hline
  3. & Eger $\displaystyle\sum_{n = 1}^{\infty}a_{n} = A, \sum_{n = 1}^{\infty}b_{n} = B$ bolsa, onda $\displaystyle\sum_{n = 1}^{\infty}\left( a_{n} + b_{n} \right)$ &  \\
  \hline
  4. & Esaplań $\displaystyle \left( \int_{}^{}{f(x)dx} \right)^\prime = ?$ &  \\
  \hline
  5. & Racional funkciyanı integrallań: $\displaystyle\int {\frac{3}{(x - 1)(x + 2)}dx}$. &  \\
  \hline
  6. & Anıq emes integraldı esaplań: $\displaystyle\int(x^{2}+\frac{1}{x} + \sin x)dx$. &  \\
  \hline
  7. & Úsh birdey korobkada aq hám qara sharlar bar. 1-korobkada 5 aq, 8 qara shar, 2-korobkada 3 aq, 4 qara shar, 3-korobkada 2 aq, 3 qara shar bar. Úsh korobkaniń birewinen tosınnan alınǵan bir shar aq bolıw itimallıǵın tabıń. &  \\
  \hline
  8. & Eki kubikti bir márte taslaǵanda túsken ochkolardıń qosındısı 4 bolıw itimallıǵın tabıń. &  \\
  \hline
  9. & Differencial teńlemeniń ulıwma sheshimin tabıń: $xy' - 2y = 0$. &  \\
  \hline
  10. & Qatardıń qosındısın tabıń: $\displaystyle\sum_{n = 1}^{\infty}\frac{1}{n(n + 3)}$. &  \\
  \hline
\end{tabular}
\egroup

\vspace{1cm}

\begin{tabular}{ c c c }
Tuwrı juwaplar sanı: \underline{\hspace{2cm}} & Bahası: \underline{\hspace{2cm}} & Imtixan alıwshınıń qolı: \underline{\hspace{2cm}} \\
\end{tabular}

\newpage

\begin{center}\textbf{51-variant}\end{center}

\bgroup
\def\arraystretch{1.5}
\begin{tabular}{ |m{6cm}|m{10cm}| }
  \hline
  Familiyası hám atı & \\
  \hline
  Fakulteti &\\
  \hline
  Toparı hám tálim baǵdarı & \\
  \hline
\end{tabular}
\egroup

\vspace{0.5cm}

\bgroup
\def\arraystretch{2}
\begin{tabular}{ |l|m{8cm}|m{7cm}| }
  \hline
  №. & Soraw & Juwap \\
  \hline
  1. & Eki ózgeriwshili funkciyanıń ekinshi tártipli aralas tuwındıları qalay belgilenedi &  \\
  \hline
  2. & Nyuton-Leybnis formulasın jazıń &  \\
  \hline
  3. & Anıq integraldı esaplawdıń Nyuton-Leybnic formulasın jazıń &  \\
  \hline
  4. & $\displaystyle\int k \cdot f(x)dx = ?$ &  \\
  \hline
  5. & Anıq emes integraldı esaplań: $\displaystyle\int \left( 10x^{4} + 7x^{6} - 3 \right)dx$. &  \\
  \hline
  6. & Anıq integraldı esaplań: $\displaystyle\int_{1}^{3}{\frac{2}{x + 1}dx}$. &  \\
  \hline
  7. & 50 buyımnan ibarat partiyada 3 buyım jaramsız. Tosınnan alınǵan 8 buyımnıń ishinde 1 buyımı jaramsız bolıw itimallıǵın tabıń. &  \\
  \hline
  8. & Telefon nomerdiń aqırǵı eki cifrasın umıtıp, tosınnan nomerlerdi tere basladı. Kerekli nomerdi tabıw itimallıǵın esaplań. &  \\
  \hline
  9. & Sızıqlı differerncial teńlemeniń uluwma sheshimin tabıń $y' + y =e^{-x}$. &  \\
  \hline
  10. & Qatardıń jıyındısın esaplań: $\displaystyle\sum_{n = 1}^{\infty}\frac{1}{(2n - 1)(2n + 1)}$. &  \\
  \hline
\end{tabular}
\egroup

\vspace{1cm}

\begin{tabular}{ c c c }
Tuwrı juwaplar sanı: \underline{\hspace{2cm}} & Bahası: \underline{\hspace{2cm}} & Imtixan alıwshınıń qolı: \underline{\hspace{2cm}} \\
\end{tabular}

\newpage

\begin{center}\textbf{52-variant}\end{center}

\bgroup
\def\arraystretch{1.5}
\begin{tabular}{ |m{6cm}|m{10cm}| }
  \hline
  Familiyası hám atı & \\
  \hline
  Fakulteti &\\
  \hline
  Toparı hám tálim baǵdarı & \\
  \hline
\end{tabular}
\egroup

\vspace{0.5cm}

\bgroup
\def\arraystretch{2}
\begin{tabular}{ |l|m{8cm}|m{7cm}| }
  \hline
  №. & Soraw & Juwap \\
  \hline
  1. & Ózgeriwshini almastırıp integrallaw usılıniń formulasın jazıń. &  \\
  \hline
  2. & Eki ózgeriwshili funkciyanıń birinshi tártipli dara tuwındıları qalay belgilenedi &  \\
  \hline
  3. & $(x_{0} , y_{0})$ noqattıń $\varepsilon$ dógeregi qalay belgilenedi &  \\
  \hline
  4. & Bayes formulasın jazıń &  \\
  \hline
  5. & Integraldı esaplań: $\displaystyle\int {2^{x}dx} $. &  \\
  \hline
  6. & Integraldı esaplań: $\displaystyle\int_{1}^{\infty}{\frac{1}{\left( x + 2 \right)^{2}}dx }$. &  \\
  \hline
  7. & Úsh birdey korobkada aq hám qara sharlar bar. 1-korobkada 5 aq, 8 qara shar, 2-korobkada 3 aq, 4 qara shar, 3-korobkada 2 aq, 3 qara shar bar. Úsh korobkanıń birewinen tosınnan alınǵan bir shar aq bolıw itimallıǵın tabıń. &  \\
  \hline
  8. & Dóngelektiń ishine kvadrat sızılǵan. Dóngelektiń ishinen tosınnan belgilengen noqattıń kvadrattıń ishinde jatıw itimallıǵın tabıń. &  \\
  \hline
  9. & Sızıqlı differencial teńlemeniń ulwma sheshimin tabıń: $y' + y =e^{x}$. &  \\
  \hline
  10. & Funkcional qatardıń jıynaqlılıq oblastın jazıń: $\ln x + \ln^{2}x + \ldots + \ln^{n}x + \ldots$. &  \\
  \hline
\end{tabular}
\egroup

\vspace{1cm}

\begin{tabular}{ c c c }
Tuwrı juwaplar sanı: \underline{\hspace{2cm}} & Bahası: \underline{\hspace{2cm}} & Imtixan alıwshınıń qolı: \underline{\hspace{2cm}} \\
\end{tabular}

\newpage

\begin{center}\textbf{53-variant}\end{center}

\bgroup
\def\arraystretch{1.5}
\begin{tabular}{ |m{6cm}|m{10cm}| }
  \hline
  Familiyası hám atı & \\
  \hline
  Fakulteti &\\
  \hline
  Toparı hám tálim baǵdarı & \\
  \hline
\end{tabular}
\egroup

\vspace{0.5cm}

\bgroup
\def\arraystretch{2}
\begin{tabular}{ |l|m{8cm}|m{7cm}| }
  \hline
  №. & Soraw & Juwap \\
  \hline
  1. & Funkcianıń $(x_{0}, y_{0})$ noqattaǵı tuwındısınıń formulasın jazıń &  \\
  \hline
  2. & Isenimli waqıyanıń itimallıǵı nege teń &  \\
  \hline
  3. & $n$-dárejeli kóp aǵzalınıń uluwma kórinisi &  \\
  \hline
  4. & Eki ózgeriwshili funkciyanıń tolıq ósimi &  \\
  \hline
  5. & Integraldı esaplań: $\displaystyle\int (x - 1)^{20}dx$. &  \\
  \hline
  6. & Anıq integraldı esaplań: $\displaystyle\int_{0}^{1}{(3x^{2} + 1)dx}$. &  \\
  \hline
  7. & Qutıda 15 aq, 18 qara shar bar. Tosınnan alınǵan bir shar aq bolıw itimallıǵın tabıń. &  \\
  \hline
  8. & "MATEMATIKA" sóziniń háripleri bólek kartochkalarǵa jazılıp jawıp aralastırılıp qoyılǵan. Barlıq kartochkalar tosınnan izbe-iz alınıp ashılıp, alınıw tártibinde stol ústine dizilgende taǵı "MATEMATIKA" sóziniń kelip shıǵıw itimallıǵın tabıń. &  \\
  \hline
  9. & Differencial teńlemeni esaplań: $yy'= 4$. &  \\
  \hline
  10. & Qatardıń qosındısın tabıń: $\displaystyle\sum_{n = 1}^{\infty}\frac{1}{n(n + 3)}$. &  \\
  \hline
\end{tabular}
\egroup

\vspace{1cm}

\begin{tabular}{ c c c }
Tuwrı juwaplar sanı: \underline{\hspace{2cm}} & Bahası: \underline{\hspace{2cm}} & Imtixan alıwshınıń qolı: \underline{\hspace{2cm}} \\
\end{tabular}

\newpage

\begin{center}\textbf{54-variant}\end{center}

\bgroup
\def\arraystretch{1.5}
\begin{tabular}{ |m{6cm}|m{10cm}| }
  \hline
  Familiyası hám atı & \\
  \hline
  Fakulteti &\\
  \hline
  Toparı hám tálim baǵdarı & \\
  \hline
\end{tabular}
\egroup

\vspace{0.5cm}

\bgroup
\def\arraystretch{2}
\begin{tabular}{ |l|m{8cm}|m{7cm}| }
  \hline
  №. & Soraw & Juwap \\
  \hline
  1. & Shártli itimallıq formulasın jazıń &  \\
  \hline
  2. & Funkciya qanday usıllarda beriledi &  \\
  \hline
  3. & Eki ózgeriwshili funkciyanıń ekinshi tártipli dara tuwındıları qalay belgilenedi &  \\
  \hline
  4. & Shekli additivlik aksiomasın jazıń &  \\
  \hline
  5. & Integraldı esaplań: $\displaystyle\int {\frac{1}{\sin x}dx} $. &  \\
  \hline
  6. & Anıq integraldı esaplań: $\displaystyle\int_{2}^{4}\frac{dx}{x}$. &  \\
  \hline
  7. & Ídısta 5 aq, 8 qara shar bar. Ídıstan tosınnan izbe-iz 3 shar alındı. Alınǵan sharlar aq, qara, qara degen izbe-izlikte bolıw itimallıǵın tabıń. &  \\
  \hline
  8. & Tiyindi eki márte taslaǵanda, keminde bir márte san tárepi túsiw itimallıǵın tabıń. &  \\
  \hline
  9. & Differencial teńlemeniń ulıwma sheshimin tabıń: $y'=e^{x}$. &  \\
  \hline
  10. & Funkcional qatardıń jıynaqlılıq oblastın tabıń: $1 + x + \ldots + x^{n} + \ldots$. &  \\
  \hline
\end{tabular}
\egroup

\vspace{1cm}

\begin{tabular}{ c c c }
Tuwrı juwaplar sanı: \underline{\hspace{2cm}} & Bahası: \underline{\hspace{2cm}} & Imtixan alıwshınıń qolı: \underline{\hspace{2cm}} \\
\end{tabular}

\newpage

\begin{center}\textbf{55-variant}\end{center}

\bgroup
\def\arraystretch{1.5}
\begin{tabular}{ |m{6cm}|m{10cm}| }
  \hline
  Familiyası hám atı & \\
  \hline
  Fakulteti &\\
  \hline
  Toparı hám tálim baǵdarı & \\
  \hline
\end{tabular}
\egroup

\vspace{0.5cm}

\bgroup
\def\arraystretch{2}
\begin{tabular}{ |l|m{8cm}|m{7cm}| }
  \hline
  №. & Soraw & Juwap \\
  \hline
  1. & Eki ózgeriwshili funkciyanıń grafigi neden ibarat &  \\
  \hline
  2. & Gruppalaw formulasın jazıń &  \\
  \hline
  3. & Eki ózgeriwshili funkciyalar qalay belgilenedi &  \\
  \hline
  4. & Funkcionallıq qatardıń uluwma kórinisi &  \\
  \hline
  5. & Integraldı esaplań: $\displaystyle\int (x + \sin x)dx$. &  \\
  \hline
  6. & Anıq integraldı esaplań: $\displaystyle\int_{0}^{\pi}\sin xdx$. &  \\
  \hline
  7. & Qutıda 5 aq hám 15 qara shar bar. Tosınnan alınǵan bir shardıń aq bolıw itimallıǵın tabıń. &  \\
  \hline
  8. & Telefon nomerdiń aqırǵı cifrasın umıtıp, tosınnan nomerlerdi tere basladı. Kerekli nomerdi tabıw itimallıǵın esaplań. &  \\
  \hline
  9. & Differencial teńlemeni sheshiń: $y' + xy = 0$. &  \\
  \hline
  10. & Qatardıń qosındısın tabıń: $\displaystyle\sum_{n = 1}^{\infty}\frac{1}{n(n + 1)}$. &  \\
  \hline
\end{tabular}
\egroup

\vspace{1cm}

\begin{tabular}{ c c c }
Tuwrı juwaplar sanı: \underline{\hspace{2cm}} & Bahası: \underline{\hspace{2cm}} & Imtixan alıwshınıń qolı: \underline{\hspace{2cm}} \\
\end{tabular}

\newpage

\begin{center}\textbf{56-variant}\end{center}

\bgroup
\def\arraystretch{1.5}
\begin{tabular}{ |m{6cm}|m{10cm}| }
  \hline
  Familiyası hám atı & \\
  \hline
  Fakulteti &\\
  \hline
  Toparı hám tálim baǵdarı & \\
  \hline
\end{tabular}
\egroup

\vspace{0.5cm}

\bgroup
\def\arraystretch{2}
\begin{tabular}{ |l|m{8cm}|m{7cm}| }
  \hline
  №. & Soraw & Juwap \\
  \hline
  1. & Oń aǵzalı qatarlar ushın jıynaqlılıqtıń Dalamber belgisin jazıń &  \\
  \hline
  2. & Oń aǵzalı qatarlar ushın jıynaqlılıqtıń Koshi belgisin jazıń &  \\
  \hline
  3. & Esaplań $\displaystyle d\left( \int_{}^{}{f(x)dx} \right) = ?$ &  \\
  \hline
  4. & Ózgeriwshileri ajıralǵan differenciallıq teńlemesiniń uluwma kórinisin jazıń &  \\
  \hline
  5. & Anıq emes integraldı esaplań: $\displaystyle\int e^{x}dx$. &  \\
  \hline
  6. & Integraldı esaplań: $\displaystyle\int_{1}^{\infty}{\frac{1}{x^{2}}dx}$. &  \\
  \hline
  7. & Korobkada 3 aq, 7 qara shar bar. Tosınnan úsh shar izbe-iz alındı. Izbe-iz alınǵan sharlardıń qara, qara, aq degen izbe-izlikte bolıw itimallıǵın tabıń. &  \\
  \hline
  8. & "BIOLOGIYA" sóziniń háripleri bólek kartochkalarǵa jazılıp jawıp, aralastırılıp qoyılǵan. Barlıq kartochkalar tosınnan izbe-iz alınıp ashılıp, alınıw tártibinde stol ústine dizilgende taǵı "BIOLOGIYA" sóziniń kelip shıǵıw itimallıǵın tabıń. &  \\
  \hline
  9. & Differencial teńlemeniń ulıwma sheshimin tabıń: $xy' - 2y = 0$. &  \\
  \hline
  10. & Funkcional qatardıń jaqınlasıw oblastın tabıń: $\displaystyle x + \frac{x^{2}}{2^{2}} + \ldots + \frac{x^{n}}{n^{2}} + \ldots$. &  \\
  \hline
\end{tabular}
\egroup

\vspace{1cm}

\begin{tabular}{ c c c }
Tuwrı juwaplar sanı: \underline{\hspace{2cm}} & Bahası: \underline{\hspace{2cm}} & Imtixan alıwshınıń qolı: \underline{\hspace{2cm}} \\
\end{tabular}

\newpage

\begin{center}\textbf{57-variant}\end{center}

\bgroup
\def\arraystretch{1.5}
\begin{tabular}{ |m{6cm}|m{10cm}| }
  \hline
  Familiyası hám atı & \\
  \hline
  Fakulteti &\\
  \hline
  Toparı hám tálim baǵdarı & \\
  \hline
\end{tabular}
\egroup

\vspace{0.5cm}

\bgroup
\def\arraystretch{2}
\begin{tabular}{ |l|m{8cm}|m{7cm}| }
  \hline
  №. & Soraw & Juwap \\
  \hline
  1. & Sızıqlı differenciallıq teńleme kórinisi &  \\
  \hline
  2. & Orın almastırıw formulasın jazıń &  \\
  \hline
  3. & Funkciyanıń $(x_{0}, y_{0})$ noqattaǵı úzliksizlik shártin jazıń &  \\
  \hline
  4. & Itimmallıqtıń geometriyalıq anıqlamasınıń formulasın jazıń &  \\
  \hline
  5. & Esaplań: $\displaystyle\int \left( x^{4}-\frac{1}{x} \right)dx$. &  \\
  \hline
  6. & Esaplań: $\displaystyle\int_{1}^{2}{e^{x}dx}$. &  \\
  \hline
  7. & Úsh birdey korobkada aq hám qara sharlar bar. 1-korobkada 5 aq, 8 qara shar, 2-korobkada 3 aq, 4 qara shar, 3-korobkada 2 aq, 3 qara shar bar. Úsh korobkaniń birewinen tosınnan alınǵan bir shar aq bolıw itimallıǵın tabıń. &  \\
  \hline
  8. & Gruppadaǵı 20 studentten neshe túrli usıl menen 3 náwbetshini saylap alıwǵa boladı?. &  \\
  \hline
  9. & Sızıqlı differerncial teńlemeniń uluwma sheshimin tabıń $y' + y =e^{-x}$. &  \\
  \hline
  10. & Sanlı qatardıń baslanǵısh úsh aǵzasın jazıń: $\displaystyle\sum_{n = 1}^{\infty}\frac{n!}{2^{n}}$. &  \\
  \hline
\end{tabular}
\egroup

\vspace{1cm}

\begin{tabular}{ c c c }
Tuwrı juwaplar sanı: \underline{\hspace{2cm}} & Bahası: \underline{\hspace{2cm}} & Imtixan alıwshınıń qolı: \underline{\hspace{2cm}} \\
\end{tabular}

\newpage

\begin{center}\textbf{58-variant}\end{center}

\bgroup
\def\arraystretch{1.5}
\begin{tabular}{ |m{6cm}|m{10cm}| }
  \hline
  Familiyası hám atı & \\
  \hline
  Fakulteti &\\
  \hline
  Toparı hám tálim baǵdarı & \\
  \hline
\end{tabular}
\egroup

\vspace{0.5cm}

\bgroup
\def\arraystretch{2}
\begin{tabular}{ |l|m{8cm}|m{7cm}| }
  \hline
  №. & Soraw & Juwap \\
  \hline
  1. & Sızıqlı defferencial teńlemeniń uluwma sheshimin jazıń &  \\
  \hline
  2. & Sanlı qatardıń uluwma kórinisin jazıń &  \\
  \hline
  3. & Funkcianıń $(x_{0}, y_{0})$ noqattaǵı úzliksizliginiń formulasın jazıń &  \\
  \hline
  4. & Bernulli differenciallıq teńemesin jazıń &  \\
  \hline
  5. & Anıq emes integraldı esaplań: $\displaystyle\int \frac{dx}{\cos^{2}x}$. &  \\
  \hline
  6. & Anıq itegraldı esaplań: $\displaystyle\int_{1}^{3}{\frac{2}{x + 1}dx}$. &  \\
  \hline
  7. & 50 buyımnan ibarat partiyada 3 buyım jaramsız. Tosınnan alınǵan 8 buyımnıń ishinde 1 buyımı jaramsız bolıw itimallıǵın tabıń. &  \\
  \hline
  8. & Eki kubikti bir márte taslaǵanda túsken ochkolardıń qosındısı 4 bolıw itimallıǵın tabıń. &  \\
  \hline
  9. & Sızıqlı differencial teńlemeniń ulwma sheshimin tabıń: $y' + y =e^{x}$. &  \\
  \hline
  10. & Qatardıń qosındısın tabıń: $\displaystyle\sum_{n = 1}^{\infty}\frac{1}{n(n + 3)}$. &  \\
  \hline
\end{tabular}
\egroup

\vspace{1cm}

\begin{tabular}{ c c c }
Tuwrı juwaplar sanı: \underline{\hspace{2cm}} & Bahası: \underline{\hspace{2cm}} & Imtixan alıwshınıń qolı: \underline{\hspace{2cm}} \\
\end{tabular}

\newpage

\begin{center}\textbf{59-variant}\end{center}

\bgroup
\def\arraystretch{1.5}
\begin{tabular}{ |m{6cm}|m{10cm}| }
  \hline
  Familiyası hám atı & \\
  \hline
  Fakulteti &\\
  \hline
  Toparı hám tálim baǵdarı & \\
  \hline
\end{tabular}
\egroup

\vspace{0.5cm}

\bgroup
\def\arraystretch{2}
\begin{tabular}{ |l|m{8cm}|m{7cm}| }
  \hline
  №. & Soraw & Juwap \\
  \hline
  1. & Tolıq itimallıqtıń formulasın jazıń &  \\
  \hline
  2. & Eki ózgeriwshli funkciyanıń $M(x_{0} , y_{0})$ noqattaǵı úzliksizliginiń anıqlaması &  \\
  \hline
  3. & Kóp aǵzalını $(x - a)$ ǵa bólgendegi qaldıq nege teń &  \\
  \hline
  4. & Orın awıstırıw formulasın jazıń &  \\
  \hline
  5. & Racional funkciyanı integrallań: $\displaystyle\int {\frac{5}{(x - 3)(x + 2)}dx}$. &  \\
  \hline
  6. & Anıq integraldı esaplań: $\displaystyle\int_{0}^{\frac{\pi}{2}}\cos xdx$. &  \\
  \hline
  7. & Úsh birdey korobkada aq hám qara sharlar bar. 1-korobkada 5 aq, 8 qara shar, 2-korobkada 3 aq, 4 qara shar, 3-korobkada 2 aq, 3 qara shar bar. Úsh korobkanıń birewinen tosınnan alınǵan bir shar aq bolıw itimallıǵın tabıń. &  \\
  \hline
  8. & Telefon nomerdiń aqırǵı eki cifrasın umıtıp, tosınnan nomerlerdi tere basladı. Kerekli nomerdi tabıw itimallıǵın esaplań. &  \\
  \hline
  9. & Differencial teńlemeni esaplań: $yy'= 4$. &  \\
  \hline
  10. & Qatardıń jıyındısın esaplań: $\displaystyle\sum_{n = 1}^{\infty}\frac{1}{(2n - 1)(2n + 1)}$. &  \\
  \hline
\end{tabular}
\egroup

\vspace{1cm}

\begin{tabular}{ c c c }
Tuwrı juwaplar sanı: \underline{\hspace{2cm}} & Bahası: \underline{\hspace{2cm}} & Imtixan alıwshınıń qolı: \underline{\hspace{2cm}} \\
\end{tabular}

\newpage

\begin{center}\textbf{60-variant}\end{center}

\bgroup
\def\arraystretch{1.5}
\begin{tabular}{ |m{6cm}|m{10cm}| }
  \hline
  Familiyası hám atı & \\
  \hline
  Fakulteti &\\
  \hline
  Toparı hám tálim baǵdarı & \\
  \hline
\end{tabular}
\egroup

\vspace{0.5cm}

\bgroup
\def\arraystretch{2}
\begin{tabular}{ |l|m{8cm}|m{7cm}| }
  \hline
  №. & Soraw & Juwap \\
  \hline
  1. & Anıq integraldı esaplawdıń Nyuton-Leybnis formulasın jazıń &  \\
  \hline
  2. & Itimallıqtıń klassikalıq anıqlamasınıń formulasın keltiriń &  \\
  \hline
  3. & Itimallıqtıń mánisler oblastın jazıń &  \\
  \hline
  4. & Bóleklep inegrallaw formulasın jazıń &  \\
  \hline
  5. & Racional funkciyanı integrallań: $\displaystyle\int {\frac{3}{(x - 1)(x + 2)}dx}$. &  \\
  \hline
  6. & Anıq integraldı esaplań: $\displaystyle\int_{-\frac{\pi}{4}}^{0}\frac{dx}{\cos^{2}x}$. &  \\
  \hline
  7. & Qutıda 15 aq, 18 qara shar bar. Tosınnan alınǵan bir shar aq bolıw itimallıǵın tabıń. &  \\
  \hline
  8. & Dóngelektiń ishine kvadrat sızılǵan. Dóngelektiń ishinen tosınnan belgilengen noqattıń kvadrattıń ishinde jatıw itimallıǵın tabıń. &  \\
  \hline
  9. & Differencial teńlemeniń ulıwma sheshimin tabıń: $y'=e^{x}$. &  \\
  \hline
  10. & Funkcional qatardıń jıynaqlılıq oblastın jazıń: $\ln x + \ln^{2}x + \ldots + \ln^{n}x + \ldots$. &  \\
  \hline
\end{tabular}
\egroup

\vspace{1cm}

\begin{tabular}{ c c c }
Tuwrı juwaplar sanı: \underline{\hspace{2cm}} & Bahası: \underline{\hspace{2cm}} & Imtixan alıwshınıń qolı: \underline{\hspace{2cm}} \\
\end{tabular}

\newpage

\begin{center}\textbf{61-variant}\end{center}

\bgroup
\def\arraystretch{1.5}
\begin{tabular}{ |m{6cm}|m{10cm}| }
  \hline
  Familiyası hám atı & \\
  \hline
  Fakulteti &\\
  \hline
  Toparı hám tálim baǵdarı & \\
  \hline
\end{tabular}
\egroup

\vspace{0.5cm}

\bgroup
\def\arraystretch{2}
\begin{tabular}{ |l|m{8cm}|m{7cm}| }
  \hline
  №. & Soraw & Juwap \\
  \hline
  1. & $\displaystyle\int dF(x)$ nege teń &  \\
  \hline
  2. & Sızıqlı differenciallıq teńlemeniń uluwma kórinisin jazıń &  \\
  \hline
  3. & Eki ózgeriwshili funkciyanıń ekstremumınıń zárúrli shárti &  \\
  \hline
  4. & Funkciyanıń anıqlanıw oblastı qalay belgilenedi &  \\
  \hline
  5. & Anıq emes integraldı esaplań: $\displaystyle\int \left( 10x^{4} + 7x^{6} - 3 \right)dx$. &  \\
  \hline
  6. & Anıq emes integraldı esaplań: $\displaystyle\int(x^{2}+\frac{1}{x} + \sin x)dx$. &  \\
  \hline
  7. & Ídısta 5 aq, 8 qara shar bar. Ídıstan tosınnan izbe-iz 3 shar alındı. Alınǵan sharlar aq, qara, qara degen izbe-izlikte bolıw itimallıǵın tabıń. &  \\
  \hline
  8. & "MATEMATIKA" sóziniń háripleri bólek kartochkalarǵa jazılıp jawıp aralastırılıp qoyılǵan. Barlıq kartochkalar tosınnan izbe-iz alınıp ashılıp, alınıw tártibinde stol ústine dizilgende taǵı "MATEMATIKA" sóziniń kelip shıǵıw itimallıǵın tabıń. &  \\
  \hline
  9. & Differencial teńlemeni sheshiń: $y' + xy = 0$. &  \\
  \hline
  10. & Qatardıń qosındısın tabıń: $\displaystyle\sum_{n = 1}^{\infty}\frac{1}{n(n + 3)}$. &  \\
  \hline
\end{tabular}
\egroup

\vspace{1cm}

\begin{tabular}{ c c c }
Tuwrı juwaplar sanı: \underline{\hspace{2cm}} & Bahası: \underline{\hspace{2cm}} & Imtixan alıwshınıń qolı: \underline{\hspace{2cm}} \\
\end{tabular}

\newpage

\begin{center}\textbf{62-variant}\end{center}

\bgroup
\def\arraystretch{1.5}
\begin{tabular}{ |m{6cm}|m{10cm}| }
  \hline
  Familiyası hám atı & \\
  \hline
  Fakulteti &\\
  \hline
  Toparı hám tálim baǵdarı & \\
  \hline
\end{tabular}
\egroup

\vspace{0.5cm}

\bgroup
\def\arraystretch{2}
\begin{tabular}{ |l|m{8cm}|m{7cm}| }
  \hline
  №. & Soraw & Juwap \\
  \hline
  1. & Itimallıq keńisligin jazıń &  \\
  \hline
  2. & Eger $\displaystyle\sum_{n = 1}^{\infty}a_{n} = A, \sum_{n = 1}^{\infty}b_{n} = B$ bolsa, onda $\displaystyle\sum_{n = 1}^{\infty}\left( a_{n} - b_{n} \right)$ &  \\
  \hline
  3. & Eki ózgeriwshili funkciyanıń anıqlanıw oblastı qay jerde jaylasadı &  \\
  \hline
  4. & Múmkin emes waqıyanıń itimaıllıǵı nege teń &  \\
  \hline
  5. & Integraldı esaplań: $\displaystyle\int {2^{x}dx} $. &  \\
  \hline
  6. & Anıq integraldı esaplań: $\displaystyle\int_{1}^{3}{\frac{2}{x + 1}dx}$. &  \\
  \hline
  7. & Qutıda 5 aq hám 15 qara shar bar. Tosınnan alınǵan bir shardıń aq bolıw itimallıǵın tabıń. &  \\
  \hline
  8. & Tiyindi eki márte taslaǵanda, keminde bir márte san tárepi túsiw itimallıǵın tabıń. &  \\
  \hline
  9. & Differencial teńlemeniń ulıwma sheshimin tabıń: $xy' - 2y = 0$. &  \\
  \hline
  10. & Funkcional qatardıń jıynaqlılıq oblastın tabıń: $1 + x + \ldots + x^{n} + \ldots$. &  \\
  \hline
\end{tabular}
\egroup

\vspace{1cm}

\begin{tabular}{ c c c }
Tuwrı juwaplar sanı: \underline{\hspace{2cm}} & Bahası: \underline{\hspace{2cm}} & Imtixan alıwshınıń qolı: \underline{\hspace{2cm}} \\
\end{tabular}

\newpage

\begin{center}\textbf{63-variant}\end{center}

\bgroup
\def\arraystretch{1.5}
\begin{tabular}{ |m{6cm}|m{10cm}| }
  \hline
  Familiyası hám atı & \\
  \hline
  Fakulteti &\\
  \hline
  Toparı hám tálim baǵdarı & \\
  \hline
\end{tabular}
\egroup

\vspace{0.5cm}

\bgroup
\def\arraystretch{2}
\begin{tabular}{ |l|m{8cm}|m{7cm}| }
  \hline
  №. & Soraw & Juwap \\
  \hline
  1. & Eger $\displaystyle\sum_{n = 1}^{\infty}a_{n} = A, \sum_{n = 1}^{\infty}b_{n} = B$ bolsa, onda $\displaystyle\sum_{n = 1}^{\infty}\left( a_{n} + b_{n} \right)$ &  \\
  \hline
  2. & Esaplań $\displaystyle \left( \int_{}^{}{f(x)dx} \right)^\prime = ?$ &  \\
  \hline
  3. & Eki ózgeriwshili funkciyanıń ekinshi tártipli aralas tuwındıları qalay belgilenedi &  \\
  \hline
  4. & Nyuton-Leybnis formulasın jazıń &  \\
  \hline
  5. & Integraldı esaplań: $\displaystyle\int (x - 1)^{20}dx$. &  \\
  \hline
  6. & Integraldı esaplań: $\displaystyle\int_{1}^{\infty}{\frac{1}{\left( x + 2 \right)^{2}}dx }$. &  \\
  \hline
  7. & Korobkada 3 aq, 7 qara shar bar. Tosınnan úsh shar izbe-iz alındı. Izbe-iz alınǵan sharlardıń qara, qara, aq degen izbe-izlikte bolıw itimallıǵın tabıń. &  \\
  \hline
  8. & Telefon nomerdiń aqırǵı cifrasın umıtıp, tosınnan nomerlerdi tere basladı. Kerekli nomerdi tabıw itimallıǵın esaplań. &  \\
  \hline
  9. & Sızıqlı differerncial teńlemeniń uluwma sheshimin tabıń $y' + y =e^{-x}$. &  \\
  \hline
  10. & Qatardıń qosındısın tabıń: $\displaystyle\sum_{n = 1}^{\infty}\frac{1}{n(n + 1)}$. &  \\
  \hline
\end{tabular}
\egroup

\vspace{1cm}

\begin{tabular}{ c c c }
Tuwrı juwaplar sanı: \underline{\hspace{2cm}} & Bahası: \underline{\hspace{2cm}} & Imtixan alıwshınıń qolı: \underline{\hspace{2cm}} \\
\end{tabular}

\newpage

\begin{center}\textbf{64-variant}\end{center}

\bgroup
\def\arraystretch{1.5}
\begin{tabular}{ |m{6cm}|m{10cm}| }
  \hline
  Familiyası hám atı & \\
  \hline
  Fakulteti &\\
  \hline
  Toparı hám tálim baǵdarı & \\
  \hline
\end{tabular}
\egroup

\vspace{0.5cm}

\bgroup
\def\arraystretch{2}
\begin{tabular}{ |l|m{8cm}|m{7cm}| }
  \hline
  №. & Soraw & Juwap \\
  \hline
  1. & Anıq integraldı esaplawdıń Nyuton-Leybnic formulasın jazıń &  \\
  \hline
  2. & $\displaystyle\int k \cdot f(x)dx = ?$ &  \\
  \hline
  3. & Ózgeriwshini almastırıp integrallaw usılıniń formulasın jazıń. &  \\
  \hline
  4. & Eki ózgeriwshili funkciyanıń birinshi tártipli dara tuwındıları qalay belgilenedi &  \\
  \hline
  5. & Integraldı esaplań: $\displaystyle\int {\frac{1}{\sin x}dx} $. &  \\
  \hline
  6. & Anıq integraldı esaplań: $\displaystyle\int_{0}^{1}{(3x^{2} + 1)dx}$. &  \\
  \hline
  7. & Úsh birdey korobkada aq hám qara sharlar bar. 1-korobkada 5 aq, 8 qara shar, 2-korobkada 3 aq, 4 qara shar, 3-korobkada 2 aq, 3 qara shar bar. Úsh korobkaniń birewinen tosınnan alınǵan bir shar aq bolıw itimallıǵın tabıń. &  \\
  \hline
  8. & "BIOLOGIYA" sóziniń háripleri bólek kartochkalarǵa jazılıp jawıp, aralastırılıp qoyılǵan. Barlıq kartochkalar tosınnan izbe-iz alınıp ashılıp, alınıw tártibinde stol ústine dizilgende taǵı "BIOLOGIYA" sóziniń kelip shıǵıw itimallıǵın tabıń. &  \\
  \hline
  9. & Sızıqlı differencial teńlemeniń ulwma sheshimin tabıń: $y' + y =e^{x}$. &  \\
  \hline
  10. & Funkcional qatardıń jaqınlasıw oblastın tabıń: $\displaystyle x + \frac{x^{2}}{2^{2}} + \ldots + \frac{x^{n}}{n^{2}} + \ldots$. &  \\
  \hline
\end{tabular}
\egroup

\vspace{1cm}

\begin{tabular}{ c c c }
Tuwrı juwaplar sanı: \underline{\hspace{2cm}} & Bahası: \underline{\hspace{2cm}} & Imtixan alıwshınıń qolı: \underline{\hspace{2cm}} \\
\end{tabular}

\newpage

\begin{center}\textbf{65-variant}\end{center}

\bgroup
\def\arraystretch{1.5}
\begin{tabular}{ |m{6cm}|m{10cm}| }
  \hline
  Familiyası hám atı & \\
  \hline
  Fakulteti &\\
  \hline
  Toparı hám tálim baǵdarı & \\
  \hline
\end{tabular}
\egroup

\vspace{0.5cm}

\bgroup
\def\arraystretch{2}
\begin{tabular}{ |l|m{8cm}|m{7cm}| }
  \hline
  №. & Soraw & Juwap \\
  \hline
  1. & $(x_{0} , y_{0})$ noqattıń $\varepsilon$ dógeregi qalay belgilenedi &  \\
  \hline
  2. & Bayes formulasın jazıń &  \\
  \hline
  3. & Funkcianıń $(x_{0}, y_{0})$ noqattaǵı tuwındısınıń formulasın jazıń &  \\
  \hline
  4. & Isenimli waqıyanıń itimallıǵı nege teń &  \\
  \hline
  5. & Integraldı esaplań: $\displaystyle\int (x + \sin x)dx$. &  \\
  \hline
  6. & Anıq integraldı esaplań: $\displaystyle\int_{2}^{4}\frac{dx}{x}$. &  \\
  \hline
  7. & 50 buyımnan ibarat partiyada 3 buyım jaramsız. Tosınnan alınǵan 8 buyımnıń ishinde 1 buyımı jaramsız bolıw itimallıǵın tabıń. &  \\
  \hline
  8. & Gruppadaǵı 20 studentten neshe túrli usıl menen 3 náwbetshini saylap alıwǵa boladı?. &  \\
  \hline
  9. & Differencial teńlemeni esaplań: $yy'= 4$. &  \\
  \hline
  10. & Sanlı qatardıń baslanǵısh úsh aǵzasın jazıń: $\displaystyle\sum_{n = 1}^{\infty}\frac{n!}{2^{n}}$. &  \\
  \hline
\end{tabular}
\egroup

\vspace{1cm}

\begin{tabular}{ c c c }
Tuwrı juwaplar sanı: \underline{\hspace{2cm}} & Bahası: \underline{\hspace{2cm}} & Imtixan alıwshınıń qolı: \underline{\hspace{2cm}} \\
\end{tabular}

\newpage

\begin{center}\textbf{66-variant}\end{center}

\bgroup
\def\arraystretch{1.5}
\begin{tabular}{ |m{6cm}|m{10cm}| }
  \hline
  Familiyası hám atı & \\
  \hline
  Fakulteti &\\
  \hline
  Toparı hám tálim baǵdarı & \\
  \hline
\end{tabular}
\egroup

\vspace{0.5cm}

\bgroup
\def\arraystretch{2}
\begin{tabular}{ |l|m{8cm}|m{7cm}| }
  \hline
  №. & Soraw & Juwap \\
  \hline
  1. & $n$-dárejeli kóp aǵzalınıń uluwma kórinisi &  \\
  \hline
  2. & Eki ózgeriwshili funkciyanıń tolıq ósimi &  \\
  \hline
  3. & Shártli itimallıq formulasın jazıń &  \\
  \hline
  4. & Funkciya qanday usıllarda beriledi &  \\
  \hline
  5. & Anıq emes integraldı esaplań: $\displaystyle\int e^{x}dx$. &  \\
  \hline
  6. & Anıq integraldı esaplań: $\displaystyle\int_{0}^{\pi}\sin xdx$. &  \\
  \hline
  7. & Úsh birdey korobkada aq hám qara sharlar bar. 1-korobkada 5 aq, 8 qara shar, 2-korobkada 3 aq, 4 qara shar, 3-korobkada 2 aq, 3 qara shar bar. Úsh korobkanıń birewinen tosınnan alınǵan bir shar aq bolıw itimallıǵın tabıń. &  \\
  \hline
  8. & Eki kubikti bir márte taslaǵanda túsken ochkolardıń qosındısı 4 bolıw itimallıǵın tabıń. &  \\
  \hline
  9. & Differencial teńlemeniń ulıwma sheshimin tabıń: $y'=e^{x}$. &  \\
  \hline
  10. & Qatardıń qosındısın tabıń: $\displaystyle\sum_{n = 1}^{\infty}\frac{1}{n(n + 3)}$. &  \\
  \hline
\end{tabular}
\egroup

\vspace{1cm}

\begin{tabular}{ c c c }
Tuwrı juwaplar sanı: \underline{\hspace{2cm}} & Bahası: \underline{\hspace{2cm}} & Imtixan alıwshınıń qolı: \underline{\hspace{2cm}} \\
\end{tabular}

\newpage

\begin{center}\textbf{67-variant}\end{center}

\bgroup
\def\arraystretch{1.5}
\begin{tabular}{ |m{6cm}|m{10cm}| }
  \hline
  Familiyası hám atı & \\
  \hline
  Fakulteti &\\
  \hline
  Toparı hám tálim baǵdarı & \\
  \hline
\end{tabular}
\egroup

\vspace{0.5cm}

\bgroup
\def\arraystretch{2}
\begin{tabular}{ |l|m{8cm}|m{7cm}| }
  \hline
  №. & Soraw & Juwap \\
  \hline
  1. & Eki ózgeriwshili funkciyanıń ekinshi tártipli dara tuwındıları qalay belgilenedi &  \\
  \hline
  2. & Shekli additivlik aksiomasın jazıń &  \\
  \hline
  3. & Eki ózgeriwshili funkciyanıń grafigi neden ibarat &  \\
  \hline
  4. & Gruppalaw formulasın jazıń &  \\
  \hline
  5. & Esaplań: $\displaystyle\int \left( x^{4}-\frac{1}{x} \right)dx$. &  \\
  \hline
  6. & Integraldı esaplań: $\displaystyle\int_{1}^{\infty}{\frac{1}{x^{2}}dx}$. &  \\
  \hline
  7. & Qutıda 15 aq, 18 qara shar bar. Tosınnan alınǵan bir shar aq bolıw itimallıǵın tabıń. &  \\
  \hline
  8. & Telefon nomerdiń aqırǵı eki cifrasın umıtıp, tosınnan nomerlerdi tere basladı. Kerekli nomerdi tabıw itimallıǵın esaplań. &  \\
  \hline
  9. & Differencial teńlemeni sheshiń: $y' + xy = 0$. &  \\
  \hline
  10. & Qatardıń jıyındısın esaplań: $\displaystyle\sum_{n = 1}^{\infty}\frac{1}{(2n - 1)(2n + 1)}$. &  \\
  \hline
\end{tabular}
\egroup

\vspace{1cm}

\begin{tabular}{ c c c }
Tuwrı juwaplar sanı: \underline{\hspace{2cm}} & Bahası: \underline{\hspace{2cm}} & Imtixan alıwshınıń qolı: \underline{\hspace{2cm}} \\
\end{tabular}

\newpage

\begin{center}\textbf{68-variant}\end{center}

\bgroup
\def\arraystretch{1.5}
\begin{tabular}{ |m{6cm}|m{10cm}| }
  \hline
  Familiyası hám atı & \\
  \hline
  Fakulteti &\\
  \hline
  Toparı hám tálim baǵdarı & \\
  \hline
\end{tabular}
\egroup

\vspace{0.5cm}

\bgroup
\def\arraystretch{2}
\begin{tabular}{ |l|m{8cm}|m{7cm}| }
  \hline
  №. & Soraw & Juwap \\
  \hline
  1. & Eki ózgeriwshili funkciyalar qalay belgilenedi &  \\
  \hline
  2. & Funkcionallıq qatardıń uluwma kórinisi &  \\
  \hline
  3. & Oń aǵzalı qatarlar ushın jıynaqlılıqtıń Dalamber belgisin jazıń &  \\
  \hline
  4. & Oń aǵzalı qatarlar ushın jıynaqlılıqtıń Koshi belgisin jazıń &  \\
  \hline
  5. & Anıq emes integraldı esaplań: $\displaystyle\int \frac{dx}{\cos^{2}x}$. &  \\
  \hline
  6. & Esaplań: $\displaystyle\int_{1}^{2}{e^{x}dx}$. &  \\
  \hline
  7. & Ídısta 5 aq, 8 qara shar bar. Ídıstan tosınnan izbe-iz 3 shar alındı. Alınǵan sharlar aq, qara, qara degen izbe-izlikte bolıw itimallıǵın tabıń. &  \\
  \hline
  8. & Dóngelektiń ishine kvadrat sızılǵan. Dóngelektiń ishinen tosınnan belgilengen noqattıń kvadrattıń ishinde jatıw itimallıǵın tabıń. &  \\
  \hline
  9. & Differencial teńlemeniń ulıwma sheshimin tabıń: $xy' - 2y = 0$. &  \\
  \hline
  10. & Funkcional qatardıń jıynaqlılıq oblastın jazıń: $\ln x + \ln^{2}x + \ldots + \ln^{n}x + \ldots$. &  \\
  \hline
\end{tabular}
\egroup

\vspace{1cm}

\begin{tabular}{ c c c }
Tuwrı juwaplar sanı: \underline{\hspace{2cm}} & Bahası: \underline{\hspace{2cm}} & Imtixan alıwshınıń qolı: \underline{\hspace{2cm}} \\
\end{tabular}

\newpage

\begin{center}\textbf{69-variant}\end{center}

\bgroup
\def\arraystretch{1.5}
\begin{tabular}{ |m{6cm}|m{10cm}| }
  \hline
  Familiyası hám atı & \\
  \hline
  Fakulteti &\\
  \hline
  Toparı hám tálim baǵdarı & \\
  \hline
\end{tabular}
\egroup

\vspace{0.5cm}

\bgroup
\def\arraystretch{2}
\begin{tabular}{ |l|m{8cm}|m{7cm}| }
  \hline
  №. & Soraw & Juwap \\
  \hline
  1. & Esaplań $\displaystyle d\left( \int_{}^{}{f(x)dx} \right) = ?$ &  \\
  \hline
  2. & Ózgeriwshileri ajıralǵan differenciallıq teńlemesiniń uluwma kórinisin jazıń &  \\
  \hline
  3. & Sızıqlı differenciallıq teńleme kórinisi &  \\
  \hline
  4. & Orın almastırıw formulasın jazıń &  \\
  \hline
  5. & Racional funkciyanı integrallań: $\displaystyle\int {\frac{5}{(x - 3)(x + 2)}dx}$. &  \\
  \hline
  6. & Anıq itegraldı esaplań: $\displaystyle\int_{1}^{3}{\frac{2}{x + 1}dx}$. &  \\
  \hline
  7. & Qutıda 5 aq hám 15 qara shar bar. Tosınnan alınǵan bir shardıń aq bolıw itimallıǵın tabıń. &  \\
  \hline
  8. & "MATEMATIKA" sóziniń háripleri bólek kartochkalarǵa jazılıp jawıp aralastırılıp qoyılǵan. Barlıq kartochkalar tosınnan izbe-iz alınıp ashılıp, alınıw tártibinde stol ústine dizilgende taǵı "MATEMATIKA" sóziniń kelip shıǵıw itimallıǵın tabıń. &  \\
  \hline
  9. & Sızıqlı differerncial teńlemeniń uluwma sheshimin tabıń $y' + y =e^{-x}$. &  \\
  \hline
  10. & Qatardıń qosındısın tabıń: $\displaystyle\sum_{n = 1}^{\infty}\frac{1}{n(n + 3)}$. &  \\
  \hline
\end{tabular}
\egroup

\vspace{1cm}

\begin{tabular}{ c c c }
Tuwrı juwaplar sanı: \underline{\hspace{2cm}} & Bahası: \underline{\hspace{2cm}} & Imtixan alıwshınıń qolı: \underline{\hspace{2cm}} \\
\end{tabular}

\newpage

\begin{center}\textbf{70-variant}\end{center}

\bgroup
\def\arraystretch{1.5}
\begin{tabular}{ |m{6cm}|m{10cm}| }
  \hline
  Familiyası hám atı & \\
  \hline
  Fakulteti &\\
  \hline
  Toparı hám tálim baǵdarı & \\
  \hline
\end{tabular}
\egroup

\vspace{0.5cm}

\bgroup
\def\arraystretch{2}
\begin{tabular}{ |l|m{8cm}|m{7cm}| }
  \hline
  №. & Soraw & Juwap \\
  \hline
  1. & Funkciyanıń $(x_{0}, y_{0})$ noqattaǵı úzliksizlik shártin jazıń &  \\
  \hline
  2. & Itimmallıqtıń geometriyalıq anıqlamasınıń formulasın jazıń &  \\
  \hline
  3. & Sızıqlı defferencial teńlemeniń uluwma sheshimin jazıń &  \\
  \hline
  4. & Sanlı qatardıń uluwma kórinisin jazıń &  \\
  \hline
  5. & Racional funkciyanı integrallań: $\displaystyle\int {\frac{3}{(x - 1)(x + 2)}dx}$. &  \\
  \hline
  6. & Anıq integraldı esaplań: $\displaystyle\int_{0}^{\frac{\pi}{2}}\cos xdx$. &  \\
  \hline
  7. & Korobkada 3 aq, 7 qara shar bar. Tosınnan úsh shar izbe-iz alındı. Izbe-iz alınǵan sharlardıń qara, qara, aq degen izbe-izlikte bolıw itimallıǵın tabıń. &  \\
  \hline
  8. & Tiyindi eki márte taslaǵanda, keminde bir márte san tárepi túsiw itimallıǵın tabıń. &  \\
  \hline
  9. & Sızıqlı differencial teńlemeniń ulwma sheshimin tabıń: $y' + y =e^{x}$. &  \\
  \hline
  10. & Funkcional qatardıń jıynaqlılıq oblastın tabıń: $1 + x + \ldots + x^{n} + \ldots$. &  \\
  \hline
\end{tabular}
\egroup

\vspace{1cm}

\begin{tabular}{ c c c }
Tuwrı juwaplar sanı: \underline{\hspace{2cm}} & Bahası: \underline{\hspace{2cm}} & Imtixan alıwshınıń qolı: \underline{\hspace{2cm}} \\
\end{tabular}

\newpage

\begin{center}\textbf{71-variant}\end{center}

\bgroup
\def\arraystretch{1.5}
\begin{tabular}{ |m{6cm}|m{10cm}| }
  \hline
  Familiyası hám atı & \\
  \hline
  Fakulteti &\\
  \hline
  Toparı hám tálim baǵdarı & \\
  \hline
\end{tabular}
\egroup

\vspace{0.5cm}

\bgroup
\def\arraystretch{2}
\begin{tabular}{ |l|m{8cm}|m{7cm}| }
  \hline
  №. & Soraw & Juwap \\
  \hline
  1. & Funkcianıń $(x_{0}, y_{0})$ noqattaǵı úzliksizliginiń formulasın jazıń &  \\
  \hline
  2. & Bernulli differenciallıq teńemesin jazıń &  \\
  \hline
  3. & Tolıq itimallıqtıń formulasın jazıń &  \\
  \hline
  4. & Eki ózgeriwshli funkciyanıń $M(x_{0} , y_{0})$ noqattaǵı úzliksizliginiń anıqlaması &  \\
  \hline
  5. & Anıq emes integraldı esaplań: $\displaystyle\int \left( 10x^{4} + 7x^{6} - 3 \right)dx$. &  \\
  \hline
  6. & Anıq integraldı esaplań: $\displaystyle\int_{-\frac{\pi}{4}}^{0}\frac{dx}{\cos^{2}x}$. &  \\
  \hline
  7. & Úsh birdey korobkada aq hám qara sharlar bar. 1-korobkada 5 aq, 8 qara shar, 2-korobkada 3 aq, 4 qara shar, 3-korobkada 2 aq, 3 qara shar bar. Úsh korobkaniń birewinen tosınnan alınǵan bir shar aq bolıw itimallıǵın tabıń. &  \\
  \hline
  8. & Telefon nomerdiń aqırǵı cifrasın umıtıp, tosınnan nomerlerdi tere basladı. Kerekli nomerdi tabıw itimallıǵın esaplań. &  \\
  \hline
  9. & Differencial teńlemeni esaplań: $yy'= 4$. &  \\
  \hline
  10. & Qatardıń qosındısın tabıń: $\displaystyle\sum_{n = 1}^{\infty}\frac{1}{n(n + 1)}$. &  \\
  \hline
\end{tabular}
\egroup

\vspace{1cm}

\begin{tabular}{ c c c }
Tuwrı juwaplar sanı: \underline{\hspace{2cm}} & Bahası: \underline{\hspace{2cm}} & Imtixan alıwshınıń qolı: \underline{\hspace{2cm}} \\
\end{tabular}

\newpage

\begin{center}\textbf{72-variant}\end{center}

\bgroup
\def\arraystretch{1.5}
\begin{tabular}{ |m{6cm}|m{10cm}| }
  \hline
  Familiyası hám atı & \\
  \hline
  Fakulteti &\\
  \hline
  Toparı hám tálim baǵdarı & \\
  \hline
\end{tabular}
\egroup

\vspace{0.5cm}

\bgroup
\def\arraystretch{2}
\begin{tabular}{ |l|m{8cm}|m{7cm}| }
  \hline
  №. & Soraw & Juwap \\
  \hline
  1. & Kóp aǵzalını $(x - a)$ ǵa bólgendegi qaldıq nege teń &  \\
  \hline
  2. & Orın awıstırıw formulasın jazıń &  \\
  \hline
  3. & Anıq integraldı esaplawdıń Nyuton-Leybnis formulasın jazıń &  \\
  \hline
  4. & Itimallıqtıń klassikalıq anıqlamasınıń formulasın keltiriń &  \\
  \hline
  5. & Integraldı esaplań: $\displaystyle\int {2^{x}dx} $. &  \\
  \hline
  6. & Anıq emes integraldı esaplań: $\displaystyle\int(x^{2}+\frac{1}{x} + \sin x)dx$. &  \\
  \hline
  7. & 50 buyımnan ibarat partiyada 3 buyım jaramsız. Tosınnan alınǵan 8 buyımnıń ishinde 1 buyımı jaramsız bolıw itimallıǵın tabıń. &  \\
  \hline
  8. & "BIOLOGIYA" sóziniń háripleri bólek kartochkalarǵa jazılıp jawıp, aralastırılıp qoyılǵan. Barlıq kartochkalar tosınnan izbe-iz alınıp ashılıp, alınıw tártibinde stol ústine dizilgende taǵı "BIOLOGIYA" sóziniń kelip shıǵıw itimallıǵın tabıń. &  \\
  \hline
  9. & Differencial teńlemeniń ulıwma sheshimin tabıń: $y'=e^{x}$. &  \\
  \hline
  10. & Funkcional qatardıń jaqınlasıw oblastın tabıń: $\displaystyle x + \frac{x^{2}}{2^{2}} + \ldots + \frac{x^{n}}{n^{2}} + \ldots$. &  \\
  \hline
\end{tabular}
\egroup

\vspace{1cm}

\begin{tabular}{ c c c }
Tuwrı juwaplar sanı: \underline{\hspace{2cm}} & Bahası: \underline{\hspace{2cm}} & Imtixan alıwshınıń qolı: \underline{\hspace{2cm}} \\
\end{tabular}

\newpage

\begin{center}\textbf{73-variant}\end{center}

\bgroup
\def\arraystretch{1.5}
\begin{tabular}{ |m{6cm}|m{10cm}| }
  \hline
  Familiyası hám atı & \\
  \hline
  Fakulteti &\\
  \hline
  Toparı hám tálim baǵdarı & \\
  \hline
\end{tabular}
\egroup

\vspace{0.5cm}

\bgroup
\def\arraystretch{2}
\begin{tabular}{ |l|m{8cm}|m{7cm}| }
  \hline
  №. & Soraw & Juwap \\
  \hline
  1. & Itimallıqtıń mánisler oblastın jazıń &  \\
  \hline
  2. & Bóleklep inegrallaw formulasın jazıń &  \\
  \hline
  3. & $\displaystyle\int dF(x)$ nege teń &  \\
  \hline
  4. & Sızıqlı differenciallıq teńlemeniń uluwma kórinisin jazıń &  \\
  \hline
  5. & Integraldı esaplań: $\displaystyle\int (x - 1)^{20}dx$. &  \\
  \hline
  6. & Anıq integraldı esaplań: $\displaystyle\int_{1}^{3}{\frac{2}{x + 1}dx}$. &  \\
  \hline
  7. & Úsh birdey korobkada aq hám qara sharlar bar. 1-korobkada 5 aq, 8 qara shar, 2-korobkada 3 aq, 4 qara shar, 3-korobkada 2 aq, 3 qara shar bar. Úsh korobkanıń birewinen tosınnan alınǵan bir shar aq bolıw itimallıǵın tabıń. &  \\
  \hline
  8. & Gruppadaǵı 20 studentten neshe túrli usıl menen 3 náwbetshini saylap alıwǵa boladı?. &  \\
  \hline
  9. & Differencial teńlemeni sheshiń: $y' + xy = 0$. &  \\
  \hline
  10. & Sanlı qatardıń baslanǵısh úsh aǵzasın jazıń: $\displaystyle\sum_{n = 1}^{\infty}\frac{n!}{2^{n}}$. &  \\
  \hline
\end{tabular}
\egroup

\vspace{1cm}

\begin{tabular}{ c c c }
Tuwrı juwaplar sanı: \underline{\hspace{2cm}} & Bahası: \underline{\hspace{2cm}} & Imtixan alıwshınıń qolı: \underline{\hspace{2cm}} \\
\end{tabular}

\newpage

\begin{center}\textbf{74-variant}\end{center}

\bgroup
\def\arraystretch{1.5}
\begin{tabular}{ |m{6cm}|m{10cm}| }
  \hline
  Familiyası hám atı & \\
  \hline
  Fakulteti &\\
  \hline
  Toparı hám tálim baǵdarı & \\
  \hline
\end{tabular}
\egroup

\vspace{0.5cm}

\bgroup
\def\arraystretch{2}
\begin{tabular}{ |l|m{8cm}|m{7cm}| }
  \hline
  №. & Soraw & Juwap \\
  \hline
  1. & Eki ózgeriwshili funkciyanıń ekstremumınıń zárúrli shárti &  \\
  \hline
  2. & Funkciyanıń anıqlanıw oblastı qalay belgilenedi &  \\
  \hline
  3. & Itimallıq keńisligin jazıń &  \\
  \hline
  4. & Eger $\displaystyle\sum_{n = 1}^{\infty}a_{n} = A, \sum_{n = 1}^{\infty}b_{n} = B$ bolsa, onda $\displaystyle\sum_{n = 1}^{\infty}\left( a_{n} - b_{n} \right)$ &  \\
  \hline
  5. & Integraldı esaplań: $\displaystyle\int {\frac{1}{\sin x}dx} $. &  \\
  \hline
  6. & Integraldı esaplań: $\displaystyle\int_{1}^{\infty}{\frac{1}{\left( x + 2 \right)^{2}}dx }$. &  \\
  \hline
  7. & Qutıda 15 aq, 18 qara shar bar. Tosınnan alınǵan bir shar aq bolıw itimallıǵın tabıń. &  \\
  \hline
  8. & Eki kubikti bir márte taslaǵanda túsken ochkolardıń qosındısı 4 bolıw itimallıǵın tabıń. &  \\
  \hline
  9. & Differencial teńlemeniń ulıwma sheshimin tabıń: $xy' - 2y = 0$. &  \\
  \hline
  10. & Qatardıń qosındısın tabıń: $\displaystyle\sum_{n = 1}^{\infty}\frac{1}{n(n + 3)}$. &  \\
  \hline
\end{tabular}
\egroup

\vspace{1cm}

\begin{tabular}{ c c c }
Tuwrı juwaplar sanı: \underline{\hspace{2cm}} & Bahası: \underline{\hspace{2cm}} & Imtixan alıwshınıń qolı: \underline{\hspace{2cm}} \\
\end{tabular}

\newpage

\begin{center}\textbf{75-variant}\end{center}

\bgroup
\def\arraystretch{1.5}
\begin{tabular}{ |m{6cm}|m{10cm}| }
  \hline
  Familiyası hám atı & \\
  \hline
  Fakulteti &\\
  \hline
  Toparı hám tálim baǵdarı & \\
  \hline
\end{tabular}
\egroup

\vspace{0.5cm}

\bgroup
\def\arraystretch{2}
\begin{tabular}{ |l|m{8cm}|m{7cm}| }
  \hline
  №. & Soraw & Juwap \\
  \hline
  1. & Eki ózgeriwshili funkciyanıń anıqlanıw oblastı qay jerde jaylasadı &  \\
  \hline
  2. & Múmkin emes waqıyanıń itimaıllıǵı nege teń &  \\
  \hline
  3. & Eger $\displaystyle\sum_{n = 1}^{\infty}a_{n} = A, \sum_{n = 1}^{\infty}b_{n} = B$ bolsa, onda $\displaystyle\sum_{n = 1}^{\infty}\left( a_{n} + b_{n} \right)$ &  \\
  \hline
  4. & Esaplań $\displaystyle \left( \int_{}^{}{f(x)dx} \right)^\prime = ?$ &  \\
  \hline
  5. & Integraldı esaplań: $\displaystyle\int (x + \sin x)dx$. &  \\
  \hline
  6. & Anıq integraldı esaplań: $\displaystyle\int_{0}^{1}{(3x^{2} + 1)dx}$. &  \\
  \hline
  7. & Ídısta 5 aq, 8 qara shar bar. Ídıstan tosınnan izbe-iz 3 shar alındı. Alınǵan sharlar aq, qara, qara degen izbe-izlikte bolıw itimallıǵın tabıń. &  \\
  \hline
  8. & Telefon nomerdiń aqırǵı eki cifrasın umıtıp, tosınnan nomerlerdi tere basladı. Kerekli nomerdi tabıw itimallıǵın esaplań. &  \\
  \hline
  9. & Sızıqlı differerncial teńlemeniń uluwma sheshimin tabıń $y' + y =e^{-x}$. &  \\
  \hline
  10. & Qatardıń jıyındısın esaplań: $\displaystyle\sum_{n = 1}^{\infty}\frac{1}{(2n - 1)(2n + 1)}$. &  \\
  \hline
\end{tabular}
\egroup

\vspace{1cm}

\begin{tabular}{ c c c }
Tuwrı juwaplar sanı: \underline{\hspace{2cm}} & Bahası: \underline{\hspace{2cm}} & Imtixan alıwshınıń qolı: \underline{\hspace{2cm}} \\
\end{tabular}

\newpage

\begin{center}\textbf{76-variant}\end{center}

\bgroup
\def\arraystretch{1.5}
\begin{tabular}{ |m{6cm}|m{10cm}| }
  \hline
  Familiyası hám atı & \\
  \hline
  Fakulteti &\\
  \hline
  Toparı hám tálim baǵdarı & \\
  \hline
\end{tabular}
\egroup

\vspace{0.5cm}

\bgroup
\def\arraystretch{2}
\begin{tabular}{ |l|m{8cm}|m{7cm}| }
  \hline
  №. & Soraw & Juwap \\
  \hline
  1. & Eki ózgeriwshili funkciyanıń ekinshi tártipli aralas tuwındıları qalay belgilenedi &  \\
  \hline
  2. & Nyuton-Leybnis formulasın jazıń &  \\
  \hline
  3. & Anıq integraldı esaplawdıń Nyuton-Leybnic formulasın jazıń &  \\
  \hline
  4. & $\displaystyle\int k \cdot f(x)dx = ?$ &  \\
  \hline
  5. & Anıq emes integraldı esaplań: $\displaystyle\int e^{x}dx$. &  \\
  \hline
  6. & Anıq integraldı esaplań: $\displaystyle\int_{2}^{4}\frac{dx}{x}$. &  \\
  \hline
  7. & Qutıda 5 aq hám 15 qara shar bar. Tosınnan alınǵan bir shardıń aq bolıw itimallıǵın tabıń. &  \\
  \hline
  8. & Dóngelektiń ishine kvadrat sızılǵan. Dóngelektiń ishinen tosınnan belgilengen noqattıń kvadrattıń ishinde jatıw itimallıǵın tabıń. &  \\
  \hline
  9. & Sızıqlı differencial teńlemeniń ulwma sheshimin tabıń: $y' + y =e^{x}$. &  \\
  \hline
  10. & Funkcional qatardıń jıynaqlılıq oblastın jazıń: $\ln x + \ln^{2}x + \ldots + \ln^{n}x + \ldots$. &  \\
  \hline
\end{tabular}
\egroup

\vspace{1cm}

\begin{tabular}{ c c c }
Tuwrı juwaplar sanı: \underline{\hspace{2cm}} & Bahası: \underline{\hspace{2cm}} & Imtixan alıwshınıń qolı: \underline{\hspace{2cm}} \\
\end{tabular}

\newpage

\begin{center}\textbf{77-variant}\end{center}

\bgroup
\def\arraystretch{1.5}
\begin{tabular}{ |m{6cm}|m{10cm}| }
  \hline
  Familiyası hám atı & \\
  \hline
  Fakulteti &\\
  \hline
  Toparı hám tálim baǵdarı & \\
  \hline
\end{tabular}
\egroup

\vspace{0.5cm}

\bgroup
\def\arraystretch{2}
\begin{tabular}{ |l|m{8cm}|m{7cm}| }
  \hline
  №. & Soraw & Juwap \\
  \hline
  1. & Ózgeriwshini almastırıp integrallaw usılıniń formulasın jazıń. &  \\
  \hline
  2. & Eki ózgeriwshili funkciyanıń birinshi tártipli dara tuwındıları qalay belgilenedi &  \\
  \hline
  3. & $(x_{0} , y_{0})$ noqattıń $\varepsilon$ dógeregi qalay belgilenedi &  \\
  \hline
  4. & Bayes formulasın jazıń &  \\
  \hline
  5. & Esaplań: $\displaystyle\int \left( x^{4}-\frac{1}{x} \right)dx$. &  \\
  \hline
  6. & Anıq integraldı esaplań: $\displaystyle\int_{0}^{\pi}\sin xdx$. &  \\
  \hline
  7. & Korobkada 3 aq, 7 qara shar bar. Tosınnan úsh shar izbe-iz alındı. Izbe-iz alınǵan sharlardıń qara, qara, aq degen izbe-izlikte bolıw itimallıǵın tabıń. &  \\
  \hline
  8. & "MATEMATIKA" sóziniń háripleri bólek kartochkalarǵa jazılıp jawıp aralastırılıp qoyılǵan. Barlıq kartochkalar tosınnan izbe-iz alınıp ashılıp, alınıw tártibinde stol ústine dizilgende taǵı "MATEMATIKA" sóziniń kelip shıǵıw itimallıǵın tabıń. &  \\
  \hline
  9. & Differencial teńlemeni esaplań: $yy'= 4$. &  \\
  \hline
  10. & Qatardıń qosındısın tabıń: $\displaystyle\sum_{n = 1}^{\infty}\frac{1}{n(n + 3)}$. &  \\
  \hline
\end{tabular}
\egroup

\vspace{1cm}

\begin{tabular}{ c c c }
Tuwrı juwaplar sanı: \underline{\hspace{2cm}} & Bahası: \underline{\hspace{2cm}} & Imtixan alıwshınıń qolı: \underline{\hspace{2cm}} \\
\end{tabular}

\newpage

\begin{center}\textbf{78-variant}\end{center}

\bgroup
\def\arraystretch{1.5}
\begin{tabular}{ |m{6cm}|m{10cm}| }
  \hline
  Familiyası hám atı & \\
  \hline
  Fakulteti &\\
  \hline
  Toparı hám tálim baǵdarı & \\
  \hline
\end{tabular}
\egroup

\vspace{0.5cm}

\bgroup
\def\arraystretch{2}
\begin{tabular}{ |l|m{8cm}|m{7cm}| }
  \hline
  №. & Soraw & Juwap \\
  \hline
  1. & Funkcianıń $(x_{0}, y_{0})$ noqattaǵı tuwındısınıń formulasın jazıń &  \\
  \hline
  2. & Isenimli waqıyanıń itimallıǵı nege teń &  \\
  \hline
  3. & $n$-dárejeli kóp aǵzalınıń uluwma kórinisi &  \\
  \hline
  4. & Eki ózgeriwshili funkciyanıń tolıq ósimi &  \\
  \hline
  5. & Anıq emes integraldı esaplań: $\displaystyle\int \frac{dx}{\cos^{2}x}$. &  \\
  \hline
  6. & Integraldı esaplań: $\displaystyle\int_{1}^{\infty}{\frac{1}{x^{2}}dx}$. &  \\
  \hline
  7. & Úsh birdey korobkada aq hám qara sharlar bar. 1-korobkada 5 aq, 8 qara shar, 2-korobkada 3 aq, 4 qara shar, 3-korobkada 2 aq, 3 qara shar bar. Úsh korobkaniń birewinen tosınnan alınǵan bir shar aq bolıw itimallıǵın tabıń. &  \\
  \hline
  8. & Tiyindi eki márte taslaǵanda, keminde bir márte san tárepi túsiw itimallıǵın tabıń. &  \\
  \hline
  9. & Differencial teńlemeniń ulıwma sheshimin tabıń: $y'=e^{x}$. &  \\
  \hline
  10. & Funkcional qatardıń jıynaqlılıq oblastın tabıń: $1 + x + \ldots + x^{n} + \ldots$. &  \\
  \hline
\end{tabular}
\egroup

\vspace{1cm}

\begin{tabular}{ c c c }
Tuwrı juwaplar sanı: \underline{\hspace{2cm}} & Bahası: \underline{\hspace{2cm}} & Imtixan alıwshınıń qolı: \underline{\hspace{2cm}} \\
\end{tabular}

\newpage

\begin{center}\textbf{79-variant}\end{center}

\bgroup
\def\arraystretch{1.5}
\begin{tabular}{ |m{6cm}|m{10cm}| }
  \hline
  Familiyası hám atı & \\
  \hline
  Fakulteti &\\
  \hline
  Toparı hám tálim baǵdarı & \\
  \hline
\end{tabular}
\egroup

\vspace{0.5cm}

\bgroup
\def\arraystretch{2}
\begin{tabular}{ |l|m{8cm}|m{7cm}| }
  \hline
  №. & Soraw & Juwap \\
  \hline
  1. & Shártli itimallıq formulasın jazıń &  \\
  \hline
  2. & Funkciya qanday usıllarda beriledi &  \\
  \hline
  3. & Eki ózgeriwshili funkciyanıń ekinshi tártipli dara tuwındıları qalay belgilenedi &  \\
  \hline
  4. & Shekli additivlik aksiomasın jazıń &  \\
  \hline
  5. & Racional funkciyanı integrallań: $\displaystyle\int {\frac{5}{(x - 3)(x + 2)}dx}$. &  \\
  \hline
  6. & Esaplań: $\displaystyle\int_{1}^{2}{e^{x}dx}$. &  \\
  \hline
  7. & 50 buyımnan ibarat partiyada 3 buyım jaramsız. Tosınnan alınǵan 8 buyımnıń ishinde 1 buyımı jaramsız bolıw itimallıǵın tabıń. &  \\
  \hline
  8. & Telefon nomerdiń aqırǵı cifrasın umıtıp, tosınnan nomerlerdi tere basladı. Kerekli nomerdi tabıw itimallıǵın esaplań. &  \\
  \hline
  9. & Differencial teńlemeni sheshiń: $y' + xy = 0$. &  \\
  \hline
  10. & Qatardıń qosındısın tabıń: $\displaystyle\sum_{n = 1}^{\infty}\frac{1}{n(n + 1)}$. &  \\
  \hline
\end{tabular}
\egroup

\vspace{1cm}

\begin{tabular}{ c c c }
Tuwrı juwaplar sanı: \underline{\hspace{2cm}} & Bahası: \underline{\hspace{2cm}} & Imtixan alıwshınıń qolı: \underline{\hspace{2cm}} \\
\end{tabular}

\newpage

\begin{center}\textbf{80-variant}\end{center}

\bgroup
\def\arraystretch{1.5}
\begin{tabular}{ |m{6cm}|m{10cm}| }
  \hline
  Familiyası hám atı & \\
  \hline
  Fakulteti &\\
  \hline
  Toparı hám tálim baǵdarı & \\
  \hline
\end{tabular}
\egroup

\vspace{0.5cm}

\bgroup
\def\arraystretch{2}
\begin{tabular}{ |l|m{8cm}|m{7cm}| }
  \hline
  №. & Soraw & Juwap \\
  \hline
  1. & Eki ózgeriwshili funkciyanıń grafigi neden ibarat &  \\
  \hline
  2. & Gruppalaw formulasın jazıń &  \\
  \hline
  3. & Eki ózgeriwshili funkciyalar qalay belgilenedi &  \\
  \hline
  4. & Funkcionallıq qatardıń uluwma kórinisi &  \\
  \hline
  5. & Racional funkciyanı integrallań: $\displaystyle\int {\frac{3}{(x - 1)(x + 2)}dx}$. &  \\
  \hline
  6. & Anıq itegraldı esaplań: $\displaystyle\int_{1}^{3}{\frac{2}{x + 1}dx}$. &  \\
  \hline
  7. & Úsh birdey korobkada aq hám qara sharlar bar. 1-korobkada 5 aq, 8 qara shar, 2-korobkada 3 aq, 4 qara shar, 3-korobkada 2 aq, 3 qara shar bar. Úsh korobkanıń birewinen tosınnan alınǵan bir shar aq bolıw itimallıǵın tabıń. &  \\
  \hline
  8. & "BIOLOGIYA" sóziniń háripleri bólek kartochkalarǵa jazılıp jawıp, aralastırılıp qoyılǵan. Barlıq kartochkalar tosınnan izbe-iz alınıp ashılıp, alınıw tártibinde stol ústine dizilgende taǵı "BIOLOGIYA" sóziniń kelip shıǵıw itimallıǵın tabıń. &  \\
  \hline
  9. & Differencial teńlemeniń ulıwma sheshimin tabıń: $xy' - 2y = 0$. &  \\
  \hline
  10. & Funkcional qatardıń jaqınlasıw oblastın tabıń: $\displaystyle x + \frac{x^{2}}{2^{2}} + \ldots + \frac{x^{n}}{n^{2}} + \ldots$. &  \\
  \hline
\end{tabular}
\egroup

\vspace{1cm}

\begin{tabular}{ c c c }
Tuwrı juwaplar sanı: \underline{\hspace{2cm}} & Bahası: \underline{\hspace{2cm}} & Imtixan alıwshınıń qolı: \underline{\hspace{2cm}} \\
\end{tabular}

\newpage

\begin{center}\textbf{81-variant}\end{center}

\bgroup
\def\arraystretch{1.5}
\begin{tabular}{ |m{6cm}|m{10cm}| }
  \hline
  Familiyası hám atı & \\
  \hline
  Fakulteti &\\
  \hline
  Toparı hám tálim baǵdarı & \\
  \hline
\end{tabular}
\egroup

\vspace{0.5cm}

\bgroup
\def\arraystretch{2}
\begin{tabular}{ |l|m{8cm}|m{7cm}| }
  \hline
  №. & Soraw & Juwap \\
  \hline
  1. & Oń aǵzalı qatarlar ushın jıynaqlılıqtıń Dalamber belgisin jazıń &  \\
  \hline
  2. & Oń aǵzalı qatarlar ushın jıynaqlılıqtıń Koshi belgisin jazıń &  \\
  \hline
  3. & Esaplań $\displaystyle d\left( \int_{}^{}{f(x)dx} \right) = ?$ &  \\
  \hline
  4. & Ózgeriwshileri ajıralǵan differenciallıq teńlemesiniń uluwma kórinisin jazıń &  \\
  \hline
  5. & Anıq emes integraldı esaplań: $\displaystyle\int \left( 10x^{4} + 7x^{6} - 3 \right)dx$. &  \\
  \hline
  6. & Anıq integraldı esaplań: $\displaystyle\int_{0}^{\frac{\pi}{2}}\cos xdx$. &  \\
  \hline
  7. & Qutıda 15 aq, 18 qara shar bar. Tosınnan alınǵan bir shar aq bolıw itimallıǵın tabıń. &  \\
  \hline
  8. & Gruppadaǵı 20 studentten neshe túrli usıl menen 3 náwbetshini saylap alıwǵa boladı?. &  \\
  \hline
  9. & Sızıqlı differerncial teńlemeniń uluwma sheshimin tabıń $y' + y =e^{-x}$. &  \\
  \hline
  10. & Sanlı qatardıń baslanǵısh úsh aǵzasın jazıń: $\displaystyle\sum_{n = 1}^{\infty}\frac{n!}{2^{n}}$. &  \\
  \hline
\end{tabular}
\egroup

\vspace{1cm}

\begin{tabular}{ c c c }
Tuwrı juwaplar sanı: \underline{\hspace{2cm}} & Bahası: \underline{\hspace{2cm}} & Imtixan alıwshınıń qolı: \underline{\hspace{2cm}} \\
\end{tabular}

\newpage

\begin{center}\textbf{82-variant}\end{center}

\bgroup
\def\arraystretch{1.5}
\begin{tabular}{ |m{6cm}|m{10cm}| }
  \hline
  Familiyası hám atı & \\
  \hline
  Fakulteti &\\
  \hline
  Toparı hám tálim baǵdarı & \\
  \hline
\end{tabular}
\egroup

\vspace{0.5cm}

\bgroup
\def\arraystretch{2}
\begin{tabular}{ |l|m{8cm}|m{7cm}| }
  \hline
  №. & Soraw & Juwap \\
  \hline
  1. & Sızıqlı differenciallıq teńleme kórinisi &  \\
  \hline
  2. & Orın almastırıw formulasın jazıń &  \\
  \hline
  3. & Funkciyanıń $(x_{0}, y_{0})$ noqattaǵı úzliksizlik shártin jazıń &  \\
  \hline
  4. & Itimmallıqtıń geometriyalıq anıqlamasınıń formulasın jazıń &  \\
  \hline
  5. & Integraldı esaplań: $\displaystyle\int {2^{x}dx} $. &  \\
  \hline
  6. & Anıq integraldı esaplań: $\displaystyle\int_{-\frac{\pi}{4}}^{0}\frac{dx}{\cos^{2}x}$. &  \\
  \hline
  7. & Ídısta 5 aq, 8 qara shar bar. Ídıstan tosınnan izbe-iz 3 shar alındı. Alınǵan sharlar aq, qara, qara degen izbe-izlikte bolıw itimallıǵın tabıń. &  \\
  \hline
  8. & Eki kubikti bir márte taslaǵanda túsken ochkolardıń qosındısı 4 bolıw itimallıǵın tabıń. &  \\
  \hline
  9. & Sızıqlı differencial teńlemeniń ulwma sheshimin tabıń: $y' + y =e^{x}$. &  \\
  \hline
  10. & Qatardıń qosındısın tabıń: $\displaystyle\sum_{n = 1}^{\infty}\frac{1}{n(n + 3)}$. &  \\
  \hline
\end{tabular}
\egroup

\vspace{1cm}

\begin{tabular}{ c c c }
Tuwrı juwaplar sanı: \underline{\hspace{2cm}} & Bahası: \underline{\hspace{2cm}} & Imtixan alıwshınıń qolı: \underline{\hspace{2cm}} \\
\end{tabular}

\newpage

\begin{center}\textbf{83-variant}\end{center}

\bgroup
\def\arraystretch{1.5}
\begin{tabular}{ |m{6cm}|m{10cm}| }
  \hline
  Familiyası hám atı & \\
  \hline
  Fakulteti &\\
  \hline
  Toparı hám tálim baǵdarı & \\
  \hline
\end{tabular}
\egroup

\vspace{0.5cm}

\bgroup
\def\arraystretch{2}
\begin{tabular}{ |l|m{8cm}|m{7cm}| }
  \hline
  №. & Soraw & Juwap \\
  \hline
  1. & Sızıqlı defferencial teńlemeniń uluwma sheshimin jazıń &  \\
  \hline
  2. & Sanlı qatardıń uluwma kórinisin jazıń &  \\
  \hline
  3. & Funkcianıń $(x_{0}, y_{0})$ noqattaǵı úzliksizliginiń formulasın jazıń &  \\
  \hline
  4. & Bernulli differenciallıq teńemesin jazıń &  \\
  \hline
  5. & Integraldı esaplań: $\displaystyle\int (x - 1)^{20}dx$. &  \\
  \hline
  6. & Anıq emes integraldı esaplań: $\displaystyle\int(x^{2}+\frac{1}{x} + \sin x)dx$. &  \\
  \hline
  7. & Qutıda 5 aq hám 15 qara shar bar. Tosınnan alınǵan bir shardıń aq bolıw itimallıǵın tabıń. &  \\
  \hline
  8. & Telefon nomerdiń aqırǵı eki cifrasın umıtıp, tosınnan nomerlerdi tere basladı. Kerekli nomerdi tabıw itimallıǵın esaplań. &  \\
  \hline
  9. & Differencial teńlemeni esaplań: $yy'= 4$. &  \\
  \hline
  10. & Qatardıń jıyındısın esaplań: $\displaystyle\sum_{n = 1}^{\infty}\frac{1}{(2n - 1)(2n + 1)}$. &  \\
  \hline
\end{tabular}
\egroup

\vspace{1cm}

\begin{tabular}{ c c c }
Tuwrı juwaplar sanı: \underline{\hspace{2cm}} & Bahası: \underline{\hspace{2cm}} & Imtixan alıwshınıń qolı: \underline{\hspace{2cm}} \\
\end{tabular}

\newpage

\begin{center}\textbf{84-variant}\end{center}

\bgroup
\def\arraystretch{1.5}
\begin{tabular}{ |m{6cm}|m{10cm}| }
  \hline
  Familiyası hám atı & \\
  \hline
  Fakulteti &\\
  \hline
  Toparı hám tálim baǵdarı & \\
  \hline
\end{tabular}
\egroup

\vspace{0.5cm}

\bgroup
\def\arraystretch{2}
\begin{tabular}{ |l|m{8cm}|m{7cm}| }
  \hline
  №. & Soraw & Juwap \\
  \hline
  1. & Tolıq itimallıqtıń formulasın jazıń &  \\
  \hline
  2. & Eki ózgeriwshli funkciyanıń $M(x_{0} , y_{0})$ noqattaǵı úzliksizliginiń anıqlaması &  \\
  \hline
  3. & Kóp aǵzalını $(x - a)$ ǵa bólgendegi qaldıq nege teń &  \\
  \hline
  4. & Orın awıstırıw formulasın jazıń &  \\
  \hline
  5. & Integraldı esaplań: $\displaystyle\int {\frac{1}{\sin x}dx} $. &  \\
  \hline
  6. & Anıq integraldı esaplań: $\displaystyle\int_{1}^{3}{\frac{2}{x + 1}dx}$. &  \\
  \hline
  7. & Korobkada 3 aq, 7 qara shar bar. Tosınnan úsh shar izbe-iz alındı. Izbe-iz alınǵan sharlardıń qara, qara, aq degen izbe-izlikte bolıw itimallıǵın tabıń. &  \\
  \hline
  8. & Dóngelektiń ishine kvadrat sızılǵan. Dóngelektiń ishinen tosınnan belgilengen noqattıń kvadrattıń ishinde jatıw itimallıǵın tabıń. &  \\
  \hline
  9. & Differencial teńlemeniń ulıwma sheshimin tabıń: $y'=e^{x}$. &  \\
  \hline
  10. & Funkcional qatardıń jıynaqlılıq oblastın jazıń: $\ln x + \ln^{2}x + \ldots + \ln^{n}x + \ldots$. &  \\
  \hline
\end{tabular}
\egroup

\vspace{1cm}

\begin{tabular}{ c c c }
Tuwrı juwaplar sanı: \underline{\hspace{2cm}} & Bahası: \underline{\hspace{2cm}} & Imtixan alıwshınıń qolı: \underline{\hspace{2cm}} \\
\end{tabular}

\newpage

\begin{center}\textbf{85-variant}\end{center}

\bgroup
\def\arraystretch{1.5}
\begin{tabular}{ |m{6cm}|m{10cm}| }
  \hline
  Familiyası hám atı & \\
  \hline
  Fakulteti &\\
  \hline
  Toparı hám tálim baǵdarı & \\
  \hline
\end{tabular}
\egroup

\vspace{0.5cm}

\bgroup
\def\arraystretch{2}
\begin{tabular}{ |l|m{8cm}|m{7cm}| }
  \hline
  №. & Soraw & Juwap \\
  \hline
  1. & Anıq integraldı esaplawdıń Nyuton-Leybnis formulasın jazıń &  \\
  \hline
  2. & Itimallıqtıń klassikalıq anıqlamasınıń formulasın keltiriń &  \\
  \hline
  3. & Itimallıqtıń mánisler oblastın jazıń &  \\
  \hline
  4. & Bóleklep inegrallaw formulasın jazıń &  \\
  \hline
  5. & Integraldı esaplań: $\displaystyle\int (x + \sin x)dx$. &  \\
  \hline
  6. & Integraldı esaplań: $\displaystyle\int_{1}^{\infty}{\frac{1}{\left( x + 2 \right)^{2}}dx }$. &  \\
  \hline
  7. & Úsh birdey korobkada aq hám qara sharlar bar. 1-korobkada 5 aq, 8 qara shar, 2-korobkada 3 aq, 4 qara shar, 3-korobkada 2 aq, 3 qara shar bar. Úsh korobkaniń birewinen tosınnan alınǵan bir shar aq bolıw itimallıǵın tabıń. &  \\
  \hline
  8. & "MATEMATIKA" sóziniń háripleri bólek kartochkalarǵa jazılıp jawıp aralastırılıp qoyılǵan. Barlıq kartochkalar tosınnan izbe-iz alınıp ashılıp, alınıw tártibinde stol ústine dizilgende taǵı "MATEMATIKA" sóziniń kelip shıǵıw itimallıǵın tabıń. &  \\
  \hline
  9. & Differencial teńlemeni sheshiń: $y' + xy = 0$. &  \\
  \hline
  10. & Qatardıń qosındısın tabıń: $\displaystyle\sum_{n = 1}^{\infty}\frac{1}{n(n + 3)}$. &  \\
  \hline
\end{tabular}
\egroup

\vspace{1cm}

\begin{tabular}{ c c c }
Tuwrı juwaplar sanı: \underline{\hspace{2cm}} & Bahası: \underline{\hspace{2cm}} & Imtixan alıwshınıń qolı: \underline{\hspace{2cm}} \\
\end{tabular}

\newpage

\begin{center}\textbf{86-variant}\end{center}

\bgroup
\def\arraystretch{1.5}
\begin{tabular}{ |m{6cm}|m{10cm}| }
  \hline
  Familiyası hám atı & \\
  \hline
  Fakulteti &\\
  \hline
  Toparı hám tálim baǵdarı & \\
  \hline
\end{tabular}
\egroup

\vspace{0.5cm}

\bgroup
\def\arraystretch{2}
\begin{tabular}{ |l|m{8cm}|m{7cm}| }
  \hline
  №. & Soraw & Juwap \\
  \hline
  1. & $\displaystyle\int dF(x)$ nege teń &  \\
  \hline
  2. & Sızıqlı differenciallıq teńlemeniń uluwma kórinisin jazıń &  \\
  \hline
  3. & Eki ózgeriwshili funkciyanıń ekstremumınıń zárúrli shárti &  \\
  \hline
  4. & Funkciyanıń anıqlanıw oblastı qalay belgilenedi &  \\
  \hline
  5. & Anıq emes integraldı esaplań: $\displaystyle\int e^{x}dx$. &  \\
  \hline
  6. & Anıq integraldı esaplań: $\displaystyle\int_{0}^{1}{(3x^{2} + 1)dx}$. &  \\
  \hline
  7. & 50 buyımnan ibarat partiyada 3 buyım jaramsız. Tosınnan alınǵan 8 buyımnıń ishinde 1 buyımı jaramsız bolıw itimallıǵın tabıń. &  \\
  \hline
  8. & Tiyindi eki márte taslaǵanda, keminde bir márte san tárepi túsiw itimallıǵın tabıń. &  \\
  \hline
  9. & Differencial teńlemeniń ulıwma sheshimin tabıń: $xy' - 2y = 0$. &  \\
  \hline
  10. & Funkcional qatardıń jıynaqlılıq oblastın tabıń: $1 + x + \ldots + x^{n} + \ldots$. &  \\
  \hline
\end{tabular}
\egroup

\vspace{1cm}

\begin{tabular}{ c c c }
Tuwrı juwaplar sanı: \underline{\hspace{2cm}} & Bahası: \underline{\hspace{2cm}} & Imtixan alıwshınıń qolı: \underline{\hspace{2cm}} \\
\end{tabular}

\newpage

\begin{center}\textbf{87-variant}\end{center}

\bgroup
\def\arraystretch{1.5}
\begin{tabular}{ |m{6cm}|m{10cm}| }
  \hline
  Familiyası hám atı & \\
  \hline
  Fakulteti &\\
  \hline
  Toparı hám tálim baǵdarı & \\
  \hline
\end{tabular}
\egroup

\vspace{0.5cm}

\bgroup
\def\arraystretch{2}
\begin{tabular}{ |l|m{8cm}|m{7cm}| }
  \hline
  №. & Soraw & Juwap \\
  \hline
  1. & Itimallıq keńisligin jazıń &  \\
  \hline
  2. & Eger $\displaystyle\sum_{n = 1}^{\infty}a_{n} = A, \sum_{n = 1}^{\infty}b_{n} = B$ bolsa, onda $\displaystyle\sum_{n = 1}^{\infty}\left( a_{n} - b_{n} \right)$ &  \\
  \hline
  3. & Eki ózgeriwshili funkciyanıń anıqlanıw oblastı qay jerde jaylasadı &  \\
  \hline
  4. & Múmkin emes waqıyanıń itimaıllıǵı nege teń &  \\
  \hline
  5. & Esaplań: $\displaystyle\int \left( x^{4}-\frac{1}{x} \right)dx$. &  \\
  \hline
  6. & Anıq integraldı esaplań: $\displaystyle\int_{2}^{4}\frac{dx}{x}$. &  \\
  \hline
  7. & Úsh birdey korobkada aq hám qara sharlar bar. 1-korobkada 5 aq, 8 qara shar, 2-korobkada 3 aq, 4 qara shar, 3-korobkada 2 aq, 3 qara shar bar. Úsh korobkanıń birewinen tosınnan alınǵan bir shar aq bolıw itimallıǵın tabıń. &  \\
  \hline
  8. & Telefon nomerdiń aqırǵı cifrasın umıtıp, tosınnan nomerlerdi tere basladı. Kerekli nomerdi tabıw itimallıǵın esaplań. &  \\
  \hline
  9. & Sızıqlı differerncial teńlemeniń uluwma sheshimin tabıń $y' + y =e^{-x}$. &  \\
  \hline
  10. & Qatardıń qosındısın tabıń: $\displaystyle\sum_{n = 1}^{\infty}\frac{1}{n(n + 1)}$. &  \\
  \hline
\end{tabular}
\egroup

\vspace{1cm}

\begin{tabular}{ c c c }
Tuwrı juwaplar sanı: \underline{\hspace{2cm}} & Bahası: \underline{\hspace{2cm}} & Imtixan alıwshınıń qolı: \underline{\hspace{2cm}} \\
\end{tabular}

\newpage

\begin{center}\textbf{88-variant}\end{center}

\bgroup
\def\arraystretch{1.5}
\begin{tabular}{ |m{6cm}|m{10cm}| }
  \hline
  Familiyası hám atı & \\
  \hline
  Fakulteti &\\
  \hline
  Toparı hám tálim baǵdarı & \\
  \hline
\end{tabular}
\egroup

\vspace{0.5cm}

\bgroup
\def\arraystretch{2}
\begin{tabular}{ |l|m{8cm}|m{7cm}| }
  \hline
  №. & Soraw & Juwap \\
  \hline
  1. & Eger $\displaystyle\sum_{n = 1}^{\infty}a_{n} = A, \sum_{n = 1}^{\infty}b_{n} = B$ bolsa, onda $\displaystyle\sum_{n = 1}^{\infty}\left( a_{n} + b_{n} \right)$ &  \\
  \hline
  2. & Esaplań $\displaystyle \left( \int_{}^{}{f(x)dx} \right)^\prime = ?$ &  \\
  \hline
  3. & Eki ózgeriwshili funkciyanıń ekinshi tártipli aralas tuwındıları qalay belgilenedi &  \\
  \hline
  4. & Nyuton-Leybnis formulasın jazıń &  \\
  \hline
  5. & Anıq emes integraldı esaplań: $\displaystyle\int \frac{dx}{\cos^{2}x}$. &  \\
  \hline
  6. & Anıq integraldı esaplań: $\displaystyle\int_{0}^{\pi}\sin xdx$. &  \\
  \hline
  7. & Qutıda 15 aq, 18 qara shar bar. Tosınnan alınǵan bir shar aq bolıw itimallıǵın tabıń. &  \\
  \hline
  8. & "BIOLOGIYA" sóziniń háripleri bólek kartochkalarǵa jazılıp jawıp, aralastırılıp qoyılǵan. Barlıq kartochkalar tosınnan izbe-iz alınıp ashılıp, alınıw tártibinde stol ústine dizilgende taǵı "BIOLOGIYA" sóziniń kelip shıǵıw itimallıǵın tabıń. &  \\
  \hline
  9. & Sızıqlı differencial teńlemeniń ulwma sheshimin tabıń: $y' + y =e^{x}$. &  \\
  \hline
  10. & Funkcional qatardıń jaqınlasıw oblastın tabıń: $\displaystyle x + \frac{x^{2}}{2^{2}} + \ldots + \frac{x^{n}}{n^{2}} + \ldots$. &  \\
  \hline
\end{tabular}
\egroup

\vspace{1cm}

\begin{tabular}{ c c c }
Tuwrı juwaplar sanı: \underline{\hspace{2cm}} & Bahası: \underline{\hspace{2cm}} & Imtixan alıwshınıń qolı: \underline{\hspace{2cm}} \\
\end{tabular}

\newpage

\begin{center}\textbf{89-variant}\end{center}

\bgroup
\def\arraystretch{1.5}
\begin{tabular}{ |m{6cm}|m{10cm}| }
  \hline
  Familiyası hám atı & \\
  \hline
  Fakulteti &\\
  \hline
  Toparı hám tálim baǵdarı & \\
  \hline
\end{tabular}
\egroup

\vspace{0.5cm}

\bgroup
\def\arraystretch{2}
\begin{tabular}{ |l|m{8cm}|m{7cm}| }
  \hline
  №. & Soraw & Juwap \\
  \hline
  1. & Anıq integraldı esaplawdıń Nyuton-Leybnic formulasın jazıń &  \\
  \hline
  2. & $\displaystyle\int k \cdot f(x)dx = ?$ &  \\
  \hline
  3. & Ózgeriwshini almastırıp integrallaw usılıniń formulasın jazıń. &  \\
  \hline
  4. & Eki ózgeriwshili funkciyanıń birinshi tártipli dara tuwındıları qalay belgilenedi &  \\
  \hline
  5. & Racional funkciyanı integrallań: $\displaystyle\int {\frac{5}{(x - 3)(x + 2)}dx}$. &  \\
  \hline
  6. & Integraldı esaplań: $\displaystyle\int_{1}^{\infty}{\frac{1}{x^{2}}dx}$. &  \\
  \hline
  7. & Ídısta 5 aq, 8 qara shar bar. Ídıstan tosınnan izbe-iz 3 shar alındı. Alınǵan sharlar aq, qara, qara degen izbe-izlikte bolıw itimallıǵın tabıń. &  \\
  \hline
  8. & Gruppadaǵı 20 studentten neshe túrli usıl menen 3 náwbetshini saylap alıwǵa boladı?. &  \\
  \hline
  9. & Differencial teńlemeni esaplań: $yy'= 4$. &  \\
  \hline
  10. & Sanlı qatardıń baslanǵısh úsh aǵzasın jazıń: $\displaystyle\sum_{n = 1}^{\infty}\frac{n!}{2^{n}}$. &  \\
  \hline
\end{tabular}
\egroup

\vspace{1cm}

\begin{tabular}{ c c c }
Tuwrı juwaplar sanı: \underline{\hspace{2cm}} & Bahası: \underline{\hspace{2cm}} & Imtixan alıwshınıń qolı: \underline{\hspace{2cm}} \\
\end{tabular}

\newpage

\begin{center}\textbf{90-variant}\end{center}

\bgroup
\def\arraystretch{1.5}
\begin{tabular}{ |m{6cm}|m{10cm}| }
  \hline
  Familiyası hám atı & \\
  \hline
  Fakulteti &\\
  \hline
  Toparı hám tálim baǵdarı & \\
  \hline
\end{tabular}
\egroup

\vspace{0.5cm}

\bgroup
\def\arraystretch{2}
\begin{tabular}{ |l|m{8cm}|m{7cm}| }
  \hline
  №. & Soraw & Juwap \\
  \hline
  1. & $(x_{0} , y_{0})$ noqattıń $\varepsilon$ dógeregi qalay belgilenedi &  \\
  \hline
  2. & Bayes formulasın jazıń &  \\
  \hline
  3. & Funkcianıń $(x_{0}, y_{0})$ noqattaǵı tuwındısınıń formulasın jazıń &  \\
  \hline
  4. & Isenimli waqıyanıń itimallıǵı nege teń &  \\
  \hline
  5. & Racional funkciyanı integrallań: $\displaystyle\int {\frac{3}{(x - 1)(x + 2)}dx}$. &  \\
  \hline
  6. & Esaplań: $\displaystyle\int_{1}^{2}{e^{x}dx}$. &  \\
  \hline
  7. & Qutıda 5 aq hám 15 qara shar bar. Tosınnan alınǵan bir shardıń aq bolıw itimallıǵın tabıń. &  \\
  \hline
  8. & Eki kubikti bir márte taslaǵanda túsken ochkolardıń qosındısı 4 bolıw itimallıǵın tabıń. &  \\
  \hline
  9. & Differencial teńlemeniń ulıwma sheshimin tabıń: $y'=e^{x}$. &  \\
  \hline
  10. & Qatardıń qosındısın tabıń: $\displaystyle\sum_{n = 1}^{\infty}\frac{1}{n(n + 3)}$. &  \\
  \hline
\end{tabular}
\egroup

\vspace{1cm}

\begin{tabular}{ c c c }
Tuwrı juwaplar sanı: \underline{\hspace{2cm}} & Bahası: \underline{\hspace{2cm}} & Imtixan alıwshınıń qolı: \underline{\hspace{2cm}} \\
\end{tabular}

\newpage

\begin{center}\textbf{91-variant}\end{center}

\bgroup
\def\arraystretch{1.5}
\begin{tabular}{ |m{6cm}|m{10cm}| }
  \hline
  Familiyası hám atı & \\
  \hline
  Fakulteti &\\
  \hline
  Toparı hám tálim baǵdarı & \\
  \hline
\end{tabular}
\egroup

\vspace{0.5cm}

\bgroup
\def\arraystretch{2}
\begin{tabular}{ |l|m{8cm}|m{7cm}| }
  \hline
  №. & Soraw & Juwap \\
  \hline
  1. & $n$-dárejeli kóp aǵzalınıń uluwma kórinisi &  \\
  \hline
  2. & Eki ózgeriwshili funkciyanıń tolıq ósimi &  \\
  \hline
  3. & Shártli itimallıq formulasın jazıń &  \\
  \hline
  4. & Funkciya qanday usıllarda beriledi &  \\
  \hline
  5. & Anıq emes integraldı esaplań: $\displaystyle\int \left( 10x^{4} + 7x^{6} - 3 \right)dx$. &  \\
  \hline
  6. & Anıq itegraldı esaplań: $\displaystyle\int_{1}^{3}{\frac{2}{x + 1}dx}$. &  \\
  \hline
  7. & Korobkada 3 aq, 7 qara shar bar. Tosınnan úsh shar izbe-iz alındı. Izbe-iz alınǵan sharlardıń qara, qara, aq degen izbe-izlikte bolıw itimallıǵın tabıń. &  \\
  \hline
  8. & Telefon nomerdiń aqırǵı eki cifrasın umıtıp, tosınnan nomerlerdi tere basladı. Kerekli nomerdi tabıw itimallıǵın esaplań. &  \\
  \hline
  9. & Differencial teńlemeni sheshiń: $y' + xy = 0$. &  \\
  \hline
  10. & Qatardıń jıyındısın esaplań: $\displaystyle\sum_{n = 1}^{\infty}\frac{1}{(2n - 1)(2n + 1)}$. &  \\
  \hline
\end{tabular}
\egroup

\vspace{1cm}

\begin{tabular}{ c c c }
Tuwrı juwaplar sanı: \underline{\hspace{2cm}} & Bahası: \underline{\hspace{2cm}} & Imtixan alıwshınıń qolı: \underline{\hspace{2cm}} \\
\end{tabular}

\newpage

\begin{center}\textbf{92-variant}\end{center}

\bgroup
\def\arraystretch{1.5}
\begin{tabular}{ |m{6cm}|m{10cm}| }
  \hline
  Familiyası hám atı & \\
  \hline
  Fakulteti &\\
  \hline
  Toparı hám tálim baǵdarı & \\
  \hline
\end{tabular}
\egroup

\vspace{0.5cm}

\bgroup
\def\arraystretch{2}
\begin{tabular}{ |l|m{8cm}|m{7cm}| }
  \hline
  №. & Soraw & Juwap \\
  \hline
  1. & Eki ózgeriwshili funkciyanıń ekinshi tártipli dara tuwındıları qalay belgilenedi &  \\
  \hline
  2. & Shekli additivlik aksiomasın jazıń &  \\
  \hline
  3. & Eki ózgeriwshili funkciyanıń grafigi neden ibarat &  \\
  \hline
  4. & Gruppalaw formulasın jazıń &  \\
  \hline
  5. & Integraldı esaplań: $\displaystyle\int {2^{x}dx} $. &  \\
  \hline
  6. & Anıq integraldı esaplań: $\displaystyle\int_{0}^{\frac{\pi}{2}}\cos xdx$. &  \\
  \hline
  7. & Úsh birdey korobkada aq hám qara sharlar bar. 1-korobkada 5 aq, 8 qara shar, 2-korobkada 3 aq, 4 qara shar, 3-korobkada 2 aq, 3 qara shar bar. Úsh korobkaniń birewinen tosınnan alınǵan bir shar aq bolıw itimallıǵın tabıń. &  \\
  \hline
  8. & Dóngelektiń ishine kvadrat sızılǵan. Dóngelektiń ishinen tosınnan belgilengen noqattıń kvadrattıń ishinde jatıw itimallıǵın tabıń. &  \\
  \hline
  9. & Differencial teńlemeniń ulıwma sheshimin tabıń: $xy' - 2y = 0$. &  \\
  \hline
  10. & Funkcional qatardıń jıynaqlılıq oblastın jazıń: $\ln x + \ln^{2}x + \ldots + \ln^{n}x + \ldots$. &  \\
  \hline
\end{tabular}
\egroup

\vspace{1cm}

\begin{tabular}{ c c c }
Tuwrı juwaplar sanı: \underline{\hspace{2cm}} & Bahası: \underline{\hspace{2cm}} & Imtixan alıwshınıń qolı: \underline{\hspace{2cm}} \\
\end{tabular}

\newpage

\begin{center}\textbf{93-variant}\end{center}

\bgroup
\def\arraystretch{1.5}
\begin{tabular}{ |m{6cm}|m{10cm}| }
  \hline
  Familiyası hám atı & \\
  \hline
  Fakulteti &\\
  \hline
  Toparı hám tálim baǵdarı & \\
  \hline
\end{tabular}
\egroup

\vspace{0.5cm}

\bgroup
\def\arraystretch{2}
\begin{tabular}{ |l|m{8cm}|m{7cm}| }
  \hline
  №. & Soraw & Juwap \\
  \hline
  1. & Eki ózgeriwshili funkciyalar qalay belgilenedi &  \\
  \hline
  2. & Funkcionallıq qatardıń uluwma kórinisi &  \\
  \hline
  3. & Oń aǵzalı qatarlar ushın jıynaqlılıqtıń Dalamber belgisin jazıń &  \\
  \hline
  4. & Oń aǵzalı qatarlar ushın jıynaqlılıqtıń Koshi belgisin jazıń &  \\
  \hline
  5. & Integraldı esaplań: $\displaystyle\int (x - 1)^{20}dx$. &  \\
  \hline
  6. & Anıq integraldı esaplań: $\displaystyle\int_{-\frac{\pi}{4}}^{0}\frac{dx}{\cos^{2}x}$. &  \\
  \hline
  7. & 50 buyımnan ibarat partiyada 3 buyım jaramsız. Tosınnan alınǵan 8 buyımnıń ishinde 1 buyımı jaramsız bolıw itimallıǵın tabıń. &  \\
  \hline
  8. & "MATEMATIKA" sóziniń háripleri bólek kartochkalarǵa jazılıp jawıp aralastırılıp qoyılǵan. Barlıq kartochkalar tosınnan izbe-iz alınıp ashılıp, alınıw tártibinde stol ústine dizilgende taǵı "MATEMATIKA" sóziniń kelip shıǵıw itimallıǵın tabıń. &  \\
  \hline
  9. & Sızıqlı differerncial teńlemeniń uluwma sheshimin tabıń $y' + y =e^{-x}$. &  \\
  \hline
  10. & Qatardıń qosındısın tabıń: $\displaystyle\sum_{n = 1}^{\infty}\frac{1}{n(n + 3)}$. &  \\
  \hline
\end{tabular}
\egroup

\vspace{1cm}

\begin{tabular}{ c c c }
Tuwrı juwaplar sanı: \underline{\hspace{2cm}} & Bahası: \underline{\hspace{2cm}} & Imtixan alıwshınıń qolı: \underline{\hspace{2cm}} \\
\end{tabular}

\newpage

\begin{center}\textbf{94-variant}\end{center}

\bgroup
\def\arraystretch{1.5}
\begin{tabular}{ |m{6cm}|m{10cm}| }
  \hline
  Familiyası hám atı & \\
  \hline
  Fakulteti &\\
  \hline
  Toparı hám tálim baǵdarı & \\
  \hline
\end{tabular}
\egroup

\vspace{0.5cm}

\bgroup
\def\arraystretch{2}
\begin{tabular}{ |l|m{8cm}|m{7cm}| }
  \hline
  №. & Soraw & Juwap \\
  \hline
  1. & Esaplań $\displaystyle d\left( \int_{}^{}{f(x)dx} \right) = ?$ &  \\
  \hline
  2. & Ózgeriwshileri ajıralǵan differenciallıq teńlemesiniń uluwma kórinisin jazıń &  \\
  \hline
  3. & Sızıqlı differenciallıq teńleme kórinisi &  \\
  \hline
  4. & Orın almastırıw formulasın jazıń &  \\
  \hline
  5. & Integraldı esaplań: $\displaystyle\int {\frac{1}{\sin x}dx} $. &  \\
  \hline
  6. & Anıq emes integraldı esaplań: $\displaystyle\int(x^{2}+\frac{1}{x} + \sin x)dx$. &  \\
  \hline
  7. & Úsh birdey korobkada aq hám qara sharlar bar. 1-korobkada 5 aq, 8 qara shar, 2-korobkada 3 aq, 4 qara shar, 3-korobkada 2 aq, 3 qara shar bar. Úsh korobkanıń birewinen tosınnan alınǵan bir shar aq bolıw itimallıǵın tabıń. &  \\
  \hline
  8. & Tiyindi eki márte taslaǵanda, keminde bir márte san tárepi túsiw itimallıǵın tabıń. &  \\
  \hline
  9. & Sızıqlı differencial teńlemeniń ulwma sheshimin tabıń: $y' + y =e^{x}$. &  \\
  \hline
  10. & Funkcional qatardıń jıynaqlılıq oblastın tabıń: $1 + x + \ldots + x^{n} + \ldots$. &  \\
  \hline
\end{tabular}
\egroup

\vspace{1cm}

\begin{tabular}{ c c c }
Tuwrı juwaplar sanı: \underline{\hspace{2cm}} & Bahası: \underline{\hspace{2cm}} & Imtixan alıwshınıń qolı: \underline{\hspace{2cm}} \\
\end{tabular}

\newpage

\begin{center}\textbf{95-variant}\end{center}

\bgroup
\def\arraystretch{1.5}
\begin{tabular}{ |m{6cm}|m{10cm}| }
  \hline
  Familiyası hám atı & \\
  \hline
  Fakulteti &\\
  \hline
  Toparı hám tálim baǵdarı & \\
  \hline
\end{tabular}
\egroup

\vspace{0.5cm}

\bgroup
\def\arraystretch{2}
\begin{tabular}{ |l|m{8cm}|m{7cm}| }
  \hline
  №. & Soraw & Juwap \\
  \hline
  1. & Funkciyanıń $(x_{0}, y_{0})$ noqattaǵı úzliksizlik shártin jazıń &  \\
  \hline
  2. & Itimmallıqtıń geometriyalıq anıqlamasınıń formulasın jazıń &  \\
  \hline
  3. & Sızıqlı defferencial teńlemeniń uluwma sheshimin jazıń &  \\
  \hline
  4. & Sanlı qatardıń uluwma kórinisin jazıń &  \\
  \hline
  5. & Integraldı esaplań: $\displaystyle\int (x + \sin x)dx$. &  \\
  \hline
  6. & Anıq integraldı esaplań: $\displaystyle\int_{1}^{3}{\frac{2}{x + 1}dx}$. &  \\
  \hline
  7. & Qutıda 15 aq, 18 qara shar bar. Tosınnan alınǵan bir shar aq bolıw itimallıǵın tabıń. &  \\
  \hline
  8. & Telefon nomerdiń aqırǵı cifrasın umıtıp, tosınnan nomerlerdi tere basladı. Kerekli nomerdi tabıw itimallıǵın esaplań. &  \\
  \hline
  9. & Differencial teńlemeni esaplań: $yy'= 4$. &  \\
  \hline
  10. & Qatardıń qosındısın tabıń: $\displaystyle\sum_{n = 1}^{\infty}\frac{1}{n(n + 1)}$. &  \\
  \hline
\end{tabular}
\egroup

\vspace{1cm}

\begin{tabular}{ c c c }
Tuwrı juwaplar sanı: \underline{\hspace{2cm}} & Bahası: \underline{\hspace{2cm}} & Imtixan alıwshınıń qolı: \underline{\hspace{2cm}} \\
\end{tabular}

\newpage

\begin{center}\textbf{96-variant}\end{center}

\bgroup
\def\arraystretch{1.5}
\begin{tabular}{ |m{6cm}|m{10cm}| }
  \hline
  Familiyası hám atı & \\
  \hline
  Fakulteti &\\
  \hline
  Toparı hám tálim baǵdarı & \\
  \hline
\end{tabular}
\egroup

\vspace{0.5cm}

\bgroup
\def\arraystretch{2}
\begin{tabular}{ |l|m{8cm}|m{7cm}| }
  \hline
  №. & Soraw & Juwap \\
  \hline
  1. & Funkcianıń $(x_{0}, y_{0})$ noqattaǵı úzliksizliginiń formulasın jazıń &  \\
  \hline
  2. & Bernulli differenciallıq teńemesin jazıń &  \\
  \hline
  3. & Tolıq itimallıqtıń formulasın jazıń &  \\
  \hline
  4. & Eki ózgeriwshli funkciyanıń $M(x_{0} , y_{0})$ noqattaǵı úzliksizliginiń anıqlaması &  \\
  \hline
  5. & Anıq emes integraldı esaplań: $\displaystyle\int e^{x}dx$. &  \\
  \hline
  6. & Integraldı esaplań: $\displaystyle\int_{1}^{\infty}{\frac{1}{\left( x + 2 \right)^{2}}dx }$. &  \\
  \hline
  7. & Ídısta 5 aq, 8 qara shar bar. Ídıstan tosınnan izbe-iz 3 shar alındı. Alınǵan sharlar aq, qara, qara degen izbe-izlikte bolıw itimallıǵın tabıń. &  \\
  \hline
  8. & "BIOLOGIYA" sóziniń háripleri bólek kartochkalarǵa jazılıp jawıp, aralastırılıp qoyılǵan. Barlıq kartochkalar tosınnan izbe-iz alınıp ashılıp, alınıw tártibinde stol ústine dizilgende taǵı "BIOLOGIYA" sóziniń kelip shıǵıw itimallıǵın tabıń. &  \\
  \hline
  9. & Differencial teńlemeniń ulıwma sheshimin tabıń: $y'=e^{x}$. &  \\
  \hline
  10. & Funkcional qatardıń jaqınlasıw oblastın tabıń: $\displaystyle x + \frac{x^{2}}{2^{2}} + \ldots + \frac{x^{n}}{n^{2}} + \ldots$. &  \\
  \hline
\end{tabular}
\egroup

\vspace{1cm}

\begin{tabular}{ c c c }
Tuwrı juwaplar sanı: \underline{\hspace{2cm}} & Bahası: \underline{\hspace{2cm}} & Imtixan alıwshınıń qolı: \underline{\hspace{2cm}} \\
\end{tabular}

\newpage

\begin{center}\textbf{97-variant}\end{center}

\bgroup
\def\arraystretch{1.5}
\begin{tabular}{ |m{6cm}|m{10cm}| }
  \hline
  Familiyası hám atı & \\
  \hline
  Fakulteti &\\
  \hline
  Toparı hám tálim baǵdarı & \\
  \hline
\end{tabular}
\egroup

\vspace{0.5cm}

\bgroup
\def\arraystretch{2}
\begin{tabular}{ |l|m{8cm}|m{7cm}| }
  \hline
  №. & Soraw & Juwap \\
  \hline
  1. & Kóp aǵzalını $(x - a)$ ǵa bólgendegi qaldıq nege teń &  \\
  \hline
  2. & Orın awıstırıw formulasın jazıń &  \\
  \hline
  3. & Anıq integraldı esaplawdıń Nyuton-Leybnis formulasın jazıń &  \\
  \hline
  4. & Itimallıqtıń klassikalıq anıqlamasınıń formulasın keltiriń &  \\
  \hline
  5. & Esaplań: $\displaystyle\int \left( x^{4}-\frac{1}{x} \right)dx$. &  \\
  \hline
  6. & Anıq integraldı esaplań: $\displaystyle\int_{0}^{1}{(3x^{2} + 1)dx}$. &  \\
  \hline
  7. & Qutıda 5 aq hám 15 qara shar bar. Tosınnan alınǵan bir shardıń aq bolıw itimallıǵın tabıń. &  \\
  \hline
  8. & Gruppadaǵı 20 studentten neshe túrli usıl menen 3 náwbetshini saylap alıwǵa boladı?. &  \\
  \hline
  9. & Differencial teńlemeni sheshiń: $y' + xy = 0$. &  \\
  \hline
  10. & Sanlı qatardıń baslanǵısh úsh aǵzasın jazıń: $\displaystyle\sum_{n = 1}^{\infty}\frac{n!}{2^{n}}$. &  \\
  \hline
\end{tabular}
\egroup

\vspace{1cm}

\begin{tabular}{ c c c }
Tuwrı juwaplar sanı: \underline{\hspace{2cm}} & Bahası: \underline{\hspace{2cm}} & Imtixan alıwshınıń qolı: \underline{\hspace{2cm}} \\
\end{tabular}

\newpage

\begin{center}\textbf{98-variant}\end{center}

\bgroup
\def\arraystretch{1.5}
\begin{tabular}{ |m{6cm}|m{10cm}| }
  \hline
  Familiyası hám atı & \\
  \hline
  Fakulteti &\\
  \hline
  Toparı hám tálim baǵdarı & \\
  \hline
\end{tabular}
\egroup

\vspace{0.5cm}

\bgroup
\def\arraystretch{2}
\begin{tabular}{ |l|m{8cm}|m{7cm}| }
  \hline
  №. & Soraw & Juwap \\
  \hline
  1. & Itimallıqtıń mánisler oblastın jazıń &  \\
  \hline
  2. & Bóleklep inegrallaw formulasın jazıń &  \\
  \hline
  3. & $\displaystyle\int dF(x)$ nege teń &  \\
  \hline
  4. & Sızıqlı differenciallıq teńlemeniń uluwma kórinisin jazıń &  \\
  \hline
  5. & Anıq emes integraldı esaplań: $\displaystyle\int \frac{dx}{\cos^{2}x}$. &  \\
  \hline
  6. & Anıq integraldı esaplań: $\displaystyle\int_{2}^{4}\frac{dx}{x}$. &  \\
  \hline
  7. & Korobkada 3 aq, 7 qara shar bar. Tosınnan úsh shar izbe-iz alındı. Izbe-iz alınǵan sharlardıń qara, qara, aq degen izbe-izlikte bolıw itimallıǵın tabıń. &  \\
  \hline
  8. & Eki kubikti bir márte taslaǵanda túsken ochkolardıń qosındısı 4 bolıw itimallıǵın tabıń. &  \\
  \hline
  9. & Differencial teńlemeniń ulıwma sheshimin tabıń: $xy' - 2y = 0$. &  \\
  \hline
  10. & Qatardıń qosındısın tabıń: $\displaystyle\sum_{n = 1}^{\infty}\frac{1}{n(n + 3)}$. &  \\
  \hline
\end{tabular}
\egroup

\vspace{1cm}

\begin{tabular}{ c c c }
Tuwrı juwaplar sanı: \underline{\hspace{2cm}} & Bahası: \underline{\hspace{2cm}} & Imtixan alıwshınıń qolı: \underline{\hspace{2cm}} \\
\end{tabular}

\newpage

\begin{center}\textbf{99-variant}\end{center}

\bgroup
\def\arraystretch{1.5}
\begin{tabular}{ |m{6cm}|m{10cm}| }
  \hline
  Familiyası hám atı & \\
  \hline
  Fakulteti &\\
  \hline
  Toparı hám tálim baǵdarı & \\
  \hline
\end{tabular}
\egroup

\vspace{0.5cm}

\bgroup
\def\arraystretch{2}
\begin{tabular}{ |l|m{8cm}|m{7cm}| }
  \hline
  №. & Soraw & Juwap \\
  \hline
  1. & Eki ózgeriwshili funkciyanıń ekstremumınıń zárúrli shárti &  \\
  \hline
  2. & Funkciyanıń anıqlanıw oblastı qalay belgilenedi &  \\
  \hline
  3. & Itimallıq keńisligin jazıń &  \\
  \hline
  4. & Eger $\displaystyle\sum_{n = 1}^{\infty}a_{n} = A, \sum_{n = 1}^{\infty}b_{n} = B$ bolsa, onda $\displaystyle\sum_{n = 1}^{\infty}\left( a_{n} - b_{n} \right)$ &  \\
  \hline
  5. & Racional funkciyanı integrallań: $\displaystyle\int {\frac{5}{(x - 3)(x + 2)}dx}$. &  \\
  \hline
  6. & Anıq integraldı esaplań: $\displaystyle\int_{0}^{\pi}\sin xdx$. &  \\
  \hline
  7. & Úsh birdey korobkada aq hám qara sharlar bar. 1-korobkada 5 aq, 8 qara shar, 2-korobkada 3 aq, 4 qara shar, 3-korobkada 2 aq, 3 qara shar bar. Úsh korobkaniń birewinen tosınnan alınǵan bir shar aq bolıw itimallıǵın tabıń. &  \\
  \hline
  8. & Telefon nomerdiń aqırǵı eki cifrasın umıtıp, tosınnan nomerlerdi tere basladı. Kerekli nomerdi tabıw itimallıǵın esaplań. &  \\
  \hline
  9. & Sızıqlı differerncial teńlemeniń uluwma sheshimin tabıń $y' + y =e^{-x}$. &  \\
  \hline
  10. & Qatardıń jıyındısın esaplań: $\displaystyle\sum_{n = 1}^{\infty}\frac{1}{(2n - 1)(2n + 1)}$. &  \\
  \hline
\end{tabular}
\egroup

\vspace{1cm}

\begin{tabular}{ c c c }
Tuwrı juwaplar sanı: \underline{\hspace{2cm}} & Bahası: \underline{\hspace{2cm}} & Imtixan alıwshınıń qolı: \underline{\hspace{2cm}} \\
\end{tabular}

\newpage

\begin{center}\textbf{100-variant}\end{center}

\bgroup
\def\arraystretch{1.5}
\begin{tabular}{ |m{6cm}|m{10cm}| }
  \hline
  Familiyası hám atı & \\
  \hline
  Fakulteti &\\
  \hline
  Toparı hám tálim baǵdarı & \\
  \hline
\end{tabular}
\egroup

\vspace{0.5cm}

\bgroup
\def\arraystretch{2}
\begin{tabular}{ |l|m{8cm}|m{7cm}| }
  \hline
  №. & Soraw & Juwap \\
  \hline
  1. & Eki ózgeriwshili funkciyanıń anıqlanıw oblastı qay jerde jaylasadı &  \\
  \hline
  2. & Múmkin emes waqıyanıń itimaıllıǵı nege teń &  \\
  \hline
  3. & Eger $\displaystyle\sum_{n = 1}^{\infty}a_{n} = A, \sum_{n = 1}^{\infty}b_{n} = B$ bolsa, onda $\displaystyle\sum_{n = 1}^{\infty}\left( a_{n} + b_{n} \right)$ &  \\
  \hline
  4. & Esaplań $\displaystyle \left( \int_{}^{}{f(x)dx} \right)^\prime = ?$ &  \\
  \hline
  5. & Racional funkciyanı integrallań: $\displaystyle\int {\frac{3}{(x - 1)(x + 2)}dx}$. &  \\
  \hline
  6. & Integraldı esaplań: $\displaystyle\int_{1}^{\infty}{\frac{1}{x^{2}}dx}$. &  \\
  \hline
  7. & 50 buyımnan ibarat partiyada 3 buyım jaramsız. Tosınnan alınǵan 8 buyımnıń ishinde 1 buyımı jaramsız bolıw itimallıǵın tabıń. &  \\
  \hline
  8. & Dóngelektiń ishine kvadrat sızılǵan. Dóngelektiń ishinen tosınnan belgilengen noqattıń kvadrattıń ishinde jatıw itimallıǵın tabıń. &  \\
  \hline
  9. & Sızıqlı differencial teńlemeniń ulwma sheshimin tabıń: $y' + y =e^{x}$. &  \\
  \hline
  10. & Funkcional qatardıń jıynaqlılıq oblastın jazıń: $\ln x + \ln^{2}x + \ldots + \ln^{n}x + \ldots$. &  \\
  \hline
\end{tabular}
\egroup

\vspace{1cm}

\begin{tabular}{ c c c }
Tuwrı juwaplar sanı: \underline{\hspace{2cm}} & Bahası: \underline{\hspace{2cm}} & Imtixan alıwshınıń qolı: \underline{\hspace{2cm}} \\
\end{tabular}

\newpage
\end{document}