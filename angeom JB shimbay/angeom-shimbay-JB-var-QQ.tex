\documentclass{article}
  \usepackage[utf8]{inputenc}
  \usepackage{array}
  \usepackage[a4paper,
  left=15mm,
  top=15mm,]{geometry}
  \usepackage{setspace}
  
  \renewcommand{\baselinestretch}{1.1} 
  
\begin{document}

\large
\pagenumbering{gobble}


\begin{tabular}{m{17cm}}
\textbf{1-variant}\\
1. Parabolanıń urınbasınıń teńlemesi (parabola, tuwrı, urınıw noqatı, urınba teńlemesi).\\

2. ETIS-tıń invariantları (ETIS-tıń ulıwma teńlemesi, túrlendiriw, ETIS invariantları ).\\

3. Sheńberdiń $C$ orayı hám $R$ radiusın tabıń: $x^2+y^2+4 x-2 y+5=0$.\\

4. $x^{2} - y^{2} = 27$ giperbolasına $4x + 2y - 7 = 0$ tuwrısına parallel bolǵan urınbanıń teńlemesin tabıń.  \\

5. Giperbolanıń ekscentrisiteti $\varepsilon = \frac{13}{12}$, fokusı $F(0;13)$ noqatında hám sáykes direktrisası $13y - 144 = 0$ teńlemesi menen berilgen bolsa, giperbolanıń teńlemesin dúziń.  
\end{tabular}
\vspace{1cm}


\begin{tabular}{m{17cm}}
\textbf{2-variant}\\
1. Eki gewekli giperboloid. Kanonikalıq teńlemesi (giperbolanı simmetriya kósheri átirapında aylandırıwdan alınǵan betlik).\\

2. Giperbolanıń urınbasınıń teńlemesi (giperbolaǵa berilgen noqatta júrgizilgen urınba teńlemesi).\\

3. Uchı koordinata basında jaylasqan hám $Oy$ kósherine qarata oń táreptegi yarım tegislikte jaylasqan parabolanıń teńlemesin dúziń: parametri $p=3$.\\

4. $41x^{2} + 24xy + 9y^{2} + 24x + 18y - 36 = 0$ ETİS tipin anıqlań hám orayların tabıń koordinata kósherlerin túrlendirmey qanday sızıqtı anıqlaytuǵının kórsetiń yarım kósherlerin tabıń.  \\

5. $2x^{2} + 3y^{2} + 8x - 6y + 11 = 0$ teńlemesin ápiwaylastırıń qanday geometriyalıq obrazdı anıqlaytuǵının tabıń hám grafigin jasań.  
\end{tabular}
\vspace{1cm}


\begin{tabular}{m{17cm}}
\textbf{3-variant}\\
1. ETIS-tıń ulıwma teńlemesin klassifikatsiyalaw (ETIS-tıń ulıwma teńlemesi, ETIS-tıń ulıwma teńlemesin ápiwaylastırıw, klassifikatsiyalaw).\\

2. Ekinshi tártipli betliktiń ulıwma teńlemesi. Orayın anıqlaw formulası.\\

3. Giperbola teńlemesi berilgen: $\frac{x^{2}}{25}-\frac{y^{2}}{144}=1$. Onıń polyar teńlemesin dúziń.\\

4. Ellips $3x^{2} + 4y^{2} - 12 = 0$ teńlemesi menen berilgen. Onıń kósherleriniń uzınlıqların, fokuslarınıń koordinataların hám ekscentrisitetin tabıń.  \\

5. $A(\frac{10}{3};\frac{5}{3})$ noqattan $\frac{x^{2}}{20} + \frac{y^{2}}{5} = 1$ ellipsine júrgizilgen urınbalardıń teńlemesin dúziń.  
\end{tabular}
\vspace{1cm}


\begin{tabular}{m{17cm}}
\textbf{4-variant}\\
1. Parabola hám onıń kanonikalıq teńlemesi (anıqlaması, fokusı, direktrisası, kanonikalıq teńlemesi).\\

2. ETIS-tıń orayın anıqlaw forması (ETIS-tıń ulıwma teńlemesi, orayın anıqlaw forması).\\

3. Tipin anıqlań: $25 x^{2}-20 xy+4 y^{2}-12 x+20 y-17=0$.\\

4. $y^{2} = 3x$ parabolası menen $\frac{x^{2}}{100} + \frac{y^{2}}{225} = 1$ ellipsiniń kesilisiw noqatların tabıń.  \\

5. $\frac{x^{2}}{100} + \frac{y^{2}}{36} = 1$ ellipsiniń oń jaqtaǵı fokusınan 14 ge teń aralıqta bolǵan noqattı tabıń.  
\end{tabular}
\vspace{1cm}


\begin{tabular}{m{17cm}}
\textbf{5-variant}\\
1. Giperbolalıq paraboloydtıń tuwrı sızıqlı jasawshıları (Giperbolalıq paraboloydtı jasawshı tuwrı sızıqlar dástesi).\\

2. Ellipstiń urınbasınıń teńlemesi (ellips, tuwrı, urınıw tochka, urınba teńlemesi).\\

3. Sheńber teńlemesin dúziń: sheńber $A (2;6 ) $ noqatınan ótedi hám orayı $C (-1;2) $ noqatında jaylasqan .\\

4. $\rho = \frac{6}{1 - cos\theta}$ polyar teńlemesi menen qanday sızıq berilgenin anıqlań.  \\

5. Úlken kósheri 26 ǵa, fokusları $F( - 10;0)$, $F(14;0)$ noqatlarında jaylasqan ellipstiń teńlemesin dúziń.  
\end{tabular}
\vspace{1cm}


\begin{tabular}{m{17cm}}
\textbf{6-variant}\\
1. ETIS-tıń ulıwma teńlemesin koordinata kósherlerin burıw arqalı ápiwaylastırıń (ETIS-tıń ulıwma teńlemeleri, koordinata kósherin burıw formulası, teńlemeni kanonik túrge alıp keliw).\\

2. Ellipsoida. Kanonikalıq teńlemesi (ellipsti simmetriya kósheri dogereginde aylandırıwdan alınǵan betlik, kanonikalıq teńlemesi).\\

3. Uchı koordinata basında jaylasqan hám $Oy$ kósherine qarata shep táreptegi yarım tegislikte jaylasqan parabolanıń teńlemesin dúziń: parametri $p=0,5$.\\

4. $\frac{x^{2}}{4} - \frac{y^{2}}{5} = 1$, giperbolanıń $3x - 2y = 0$ tuwrı sızıǵına parallel bolǵan urınbasınıń teńlemesin dúziń.  \\

5. $14x^{2} + 24xy + 21y^{2} - 4x + 18y - 139 = 0$ iymek sızıǵınıń tipin anıqlań, eger oraylı iymek sızıq bolsa orayınıń koordinataların tabıń.  
\end{tabular}
\vspace{1cm}


\begin{tabular}{m{17cm}}
\textbf{7-variant}\\
1. Giperbola. Kanonikalıq teńlemesi (fokuslar, kósherler, direktrisalar, giperbola, ekscentrisitet, kanonikalıq teńlemesi).\\

2. ETIS-tıń ulıwma teńlemesin koordinata basın parallel kóshiriw arqalı ápiwayılastırıń (ETIS- tıń ulıwma teńlemesin parallel kóshiriw formulası).\\

3. Giperbola teńlemesi berilgen: $\frac{x^{2}}{16}-\frac{y^{2}}{9}=1$. Onıń polyar teńlemesin dúziń.\\

4. Koordinata kósherlerin túrlendirmey ETİS teńlemesin ápiwaylastırıń, yarım kósherlerin tabıń $4x^{2} - 4xy + 9y^{2} - 26x - 18y + 3 = 0$.\\

5. $\frac{x^{2}}{2} + \frac{y^{2}}{3} = 1$, ellipsin $x + y - 2 = 0$ noqatınan júrgizilgen urınbalarınıń teńlemesin dúziń.  
\end{tabular}
\vspace{1cm}


\begin{tabular}{m{17cm}}
\textbf{8-variant}\\
1. Betliktiń kanonikalıq teńlemeleri. Betlik haqqında túsinik. (Betliktiń anıqlaması, formulaları, kósher, baǵıtlawshı tuwrılar).\\

2. Parabolanıń polyar koordinatalardaǵı teńlemesi (polyar koordinata sistemasında parabolanıń teńlemesi).\\

3. Tipin anıqlań: $4 x^2+9 y^2-40 x+36 y+100=0$.\\

4. $x^{2} - 4y^{2} = 16$ giperbola berilgen. Onıń ekscentrisitetin, fokuslarınıń koordinataların tabıń hám asimptotalarınıń teńlemelerin dúziń.\\

5. $y^{2} = 20x$ parabolasınıń abscissası 7 ge teń bolǵan $M$ noqatınıń fokal radiusın tabıń hám fokal radiusı jatqan tuwrınıń teńlemesin dúziń.  
\end{tabular}
\vspace{1cm}


\begin{tabular}{m{17cm}}
\textbf{9-variant}\\
1. ETIS -tiń ulıwma teńlemesin ápiwaylastırıw (ETIS -tiń ulıwma teńlemesi, koordinata sistemasın túrlendirip ETIS ulıwma teńlemesin ápiwaylastırıw).\\

2. Cilindrlik betlikler (jasawshı tuwrı sızıq, baǵıtlawshı iymek sızıq, cilindrlik betlik).\\

3. Sheńber teńlemesin dúziń: orayı $C (6 ;-8) $ noqatında jaylasqan hám koordinata basınan ótedi.\\

4. $3x + 4y - 12 = 0$ tuwrı sızıǵı hám $y^{2} = - 9x$ parabolasınıń kesilisiw noqatların tabıń.  \\

5. Fokusı $F( - 1; - 4)$noqatında bolǵan, sáykes direktrissası $x - 2 = 0$ teńlemesi menen berilgen $A( - 3; - 5)$ noqatınan ótiwshi ellipstiń teńlemesin dúziń.  
\end{tabular}
\vspace{1cm}


\begin{tabular}{m{17cm}}
\textbf{10-variant}\\
1. Koordinata sistemasın túrlendiriw (birlik vektorlar, kósherler, parallel kóshiriw, koordinata kósherlerin burıw).\\

2. Betlik haqqında túsinik (tuwrı, iymek sızıq, betliktiń anıqlamaları hám formulaları).\\

3. Fokusları abscissa kósherinde hám koordinata basına qarata simmetriyalıq jaylasqan ellipstiń teńlemesin dúziń: fokusları arasındaǵı aralıq $2 c=6$ hám ekscentrisitet $\varepsilon=3/5$.\\

4. $\rho = \frac{144}{13 - 5cos\theta}$ ellipsti anıqlaytuǵının kórsetiń hám onıń yarım kósherlerin anıqlań.\\

5. $4x^{2} + 24xy + 11y^{2} + 64x + 42y + 51 = 0$ iymek sızıǵınıń tipin anıqlań eger orayı bar bolsa, onıń orayınıń koordinataların tabıń hám koordinata basın orayǵa parallel kóshiriw ámelin orınlań.  
\end{tabular}
\vspace{1cm}


\begin{tabular}{m{17cm}}
\textbf{11-variant}\\
1. Giperbolanıń polyar koordinatadaǵı teńlemesi (Polyar múyeshi, polyar radiusi giperbolanıń polyar teńlemesi).\\

2. Ellipslik paraboloid (parabola, kósher, ellipslik paraboloid).\\

3. Polyar teńlemesi menen berilgen iymek sızıqtıń tipin anıqlań: $\rho=\frac{5}{1-\frac{1}{2}\cos\theta}$.\\

4. $\frac{x^{2}}{4} - \frac{y^{2}}{5} = 1$ giperbolaǵa $3x - 2y = 0$ tuwrısına parallel bolǵan urınbanıń teńlemesin dúziń.  \\

5. $\frac{x^{2}}{3} - \frac{y^{2}}{5} = 1$ giperbolasına $P(1; - 5)$ noqatında júrgizilgen urınbalardıń teńlemesin dúziń.
\end{tabular}
\vspace{1cm}


\begin{tabular}{m{17cm}}
\textbf{12-variant}\\
1. Ellipstiń polyar koordinatalardaǵı teńlemesi (polyar koordinatalar sistemasında ellipstiń teńlemesi).\\

2. Bir gewekli giperboloid. Kanonikalıq teńlemesi (giperbolanı simmetriya kósheri átirapında aylandırıwdan alınǵan betlik).\\

3. Tipin anıqlań: $4 x^{2}-y^{2}+8 x-2 y+3=0$.\\

4. ETİS-tıń ulıwma teńlemesin koordinata sistemasın túrlendirmey ápiwaylastırıń, tipin anıqlań, obrazı qanday sızıqtı anıqlaytuǵının kórsetiń. $7x^{2} - 8xy + y^{2} - 16x - 2y - 51 = 0$  \\

5. $y^{2} = 20x$ parabolasınıń $M$ noqatın tabıń, eger onıń abscissası 7 ge teń bolsa, fokal radiusın hám fokal radius jaylasqan tuwrını anıqlań.
\end{tabular}
\vspace{1cm}


\begin{tabular}{m{17cm}}
\textbf{13-variant}\\
1. Ellips hám onıń kanonikalıq teńlemesi (anıqlaması, fokuslar, ellipstiń kanonikalıq teńlemesi, ekscentrisiteti, direktrisaları).\\

2. Ekinshi tártipli aylanba betlikler (koordinata sisteması, tegislik, vektor iymek sızıq, aylanba betlik).\\

3. Sheńber teńlemesin dúziń: orayı $C (2;-3) $ noqatında jaylasqan hám radiusı $R=7$ ge teń.\\

4. $3x + 10y - 25 = 0$ tuwrı menen $\frac{x^{2}}{25} + \frac{y^{2}}{4} = 1$ ellipstiń kesilisiw noqatların tabıń.\\

5. Eger waqıttıń qálegen momentinde $M(x;y)$ noqat $5x - 16 = 0$ tuwrı sızıqqa qaraǵanda $A(5;0)$ noqattan 1,25 márte uzaqlıqta jaylasqan. Usı $M(x;y)$ noqattıń háreketiniń teńlemesin dúziń.  
\end{tabular}
\vspace{1cm}


\begin{tabular}{m{17cm}}
\textbf{14-variant}\\
1. Parabolanıń urınbasınıń teńlemesi (parabola, tuwrı, urınıw noqatı, urınba teńlemesi).\\

2. ETIS-tıń invariantları (ETIS-tıń ulıwma teńlemesi, túrlendiriw, ETIS invariantları ).\\

3. Uchı koordinata basında jaylasqan hám $Ox$ kósherine qarata tómengi yarım tegislikte jaylasqan parabolanıń teńlemesin dúziń: parametri $p=3$.\\

4. $\rho = \frac{10}{2 - cos\theta}$ polyar teńlemesi menen qanday sızıq berilgenin anıqlań.  \\

5. $4x^{2} - 4xy + y^{2} - 2x - 14y + 7 = 0$ ETİS teńlemesin ápiwayı túrge alıp keliń, tipin anıqlań, qanday geometriyalıq obrazdı anıqlaytuǵının kórsetiń, sızılmasın góne hám taza koordinatalar sistemasına qarata jasań.  
\end{tabular}
\vspace{1cm}


\begin{tabular}{m{17cm}}
\textbf{15-variant}\\
1. Eki gewekli giperboloid. Kanonikalıq teńlemesi (giperbolanı simmetriya kósheri átirapında aylandırıwdan alınǵan betlik).\\

2. Giperbolanıń urınbasınıń teńlemesi (giperbolaǵa berilgen noqatta júrgizilgen urınba teńlemesi).\\

3. Polyar teńlemesi menen berilgen iymek sızıqtıń tipin anıqlań: $\rho=\frac{10}{1-\frac{3}{2}\cos\theta}$.\\

4. $y^{2} = 12x$ paraborolasına $3x - 2y + 30 = 0$ tuwrı sızıǵına parallel bolǵan urınbanıń teńlemesin dúziń.  \\

5. $\frac{x^{2}}{3} - \frac{y^{2}}{5} = 1$, giperbolasına $P(4;2)$ noqatınan júrgizilgen urınbalardıń teńlemesin dúziń.  
\end{tabular}
\vspace{1cm}


\begin{tabular}{m{17cm}}
\textbf{16-variant}\\
1. ETIS-tıń ulıwma teńlemesin klassifikatsiyalaw (ETIS-tıń ulıwma teńlemesi, ETIS-tıń ulıwma teńlemesin ápiwaylastırıw, klassifikatsiyalaw).\\

2. Ekinshi tártipli betliktiń ulıwma teńlemesi. Orayın anıqlaw formulası.\\

3. Berilgen sızıqlardıń oraylıq ekenligin kórsetiń hám orayın tabıń: $9 x^{2}-4 xy-7 y^{2}-12=0$.\\

4. Koordinata kósherlerin túrlendirmey ETİS teńlemesin ápiwaylastırıń, yarım kósherlerin tabıń $41x^{2} + 2xy + 9y^{2} - 26x - 18y + 3 = 0$.  \\

5. $M(2; - \frac{5}{3})$ noqatı $\frac{x^{2}}{9} + \frac{y^{2}}{5} = 1$ ellipsinde jaylasqan. $M$ noqatınıń fokal radiusları jatıwshı tuwrı sızıq teńlemelerin dúziń.  
\end{tabular}
\vspace{1cm}


\begin{tabular}{m{17cm}}
\textbf{17-variant}\\
1. Parabola hám onıń kanonikalıq teńlemesi (anıqlaması, fokusı, direktrisası, kanonikalıq teńlemesi).\\

2. ETIS-tıń orayın anıqlaw forması (ETIS-tıń ulıwma teńlemesi, orayın anıqlaw forması).\\

3. Sheńberdiń $C$ orayı hám $R$ radiusın tabıń: $x^2+y^2+6 x-4 y+14=0$.\\

4. $\rho = \frac{5}{3 - 4cos\theta}$ teńlemesi menen qanday sızıq berilgenin hám yarım kósherlerin tabıń.  \\

5. Eger qálegen waqıt momentinde $M(x;y)$ noqat $A(8;4)$ noqattan hám ordinata kósherinen birdey aralıqta jaylassa, $M(x;y)$ noqatınıń háreket etiw troektoriyasınıń teńlemesin dúziń.  
\end{tabular}
\vspace{1cm}


\begin{tabular}{m{17cm}}
\textbf{18-variant}\\
1. Giperbolalıq paraboloydtıń tuwrı sızıqlı jasawshıları (Giperbolalıq paraboloydtı jasawshı tuwrı sızıqlar dástesi).\\

2. Ellipstiń urınbasınıń teńlemesi (ellips, tuwrı, urınıw tochka, urınba teńlemesi).\\

3. Fokusları abscissa kósherinde hám koordinata basına qarata simmetriyalıq jaylasqan ellipstiń teńlemesin dúziń: direktrisaları arasındaǵı aralıq $5$ hám fokusları arasındaǵı aralıq $2 c=4$.\\

4. $\frac{x^{2}}{20} - \frac{y^{2}}{5} = 1$ giperbolasına $4x + 3y - 7 = 0$ tuwrısına perpendikulyar bolǵan urınbanıń teńlemesin dúziń.  \\

5. $2x^{2} + 3y^{2} + 8x - 6y + 11 = 0$ teńlemesi menen qanday tiptegi sızıq berilgenin anıqlań hám onıń teńlemesin ápiwaylastırıń hám grafigin jasań.  
\end{tabular}
\vspace{1cm}


\begin{tabular}{m{17cm}}
\textbf{19-variant}\\
1. ETIS-tıń ulıwma teńlemesin koordinata kósherlerin burıw arqalı ápiwaylastırıń (ETIS-tıń ulıwma teńlemeleri, koordinata kósherin burıw formulası, teńlemeni kanonik túrge alıp keliw).\\

2. Ellipsoida. Kanonikalıq teńlemesi (ellipsti simmetriya kósheri dogereginde aylandırıwdan alınǵan betlik, kanonikalıq teńlemesi).\\

3. Polyar teńlemesi menen berilgen iymek sızıqtıń tipin anıqlań: $\rho=\frac{6}{1-\cos 0}$.\\

4. Koordinata kósherlerin túrlendirmey ETİS teńlemesin ápiwaylastırıń, qanday geometriyalıq obrazdı anıqlaytuǵının kórsetiń $4x^{2} - 4xy + y^{2} + 4x - 2y + 1 = 0$.  \\

5. $\frac{x^{2}}{25} + \frac{y^{2}}{16} = 1$, ellipsine $C(10; - 8)$ noqatınan júrgizilgen urınbalarınıń teńlemesin dúziń.  
\end{tabular}
\vspace{1cm}


\begin{tabular}{m{17cm}}
\textbf{20-variant}\\
1. Giperbola. Kanonikalıq teńlemesi (fokuslar, kósherler, direktrisalar, giperbola, ekscentrisitet, kanonikalıq teńlemesi).\\

2. ETIS-tıń ulıwma teńlemesin koordinata basın parallel kóshiriw arqalı ápiwayılastırıń (ETIS- tıń ulıwma teńlemesin parallel kóshiriw formulası).\\

3. Tipin anıqlań: $9 x^{2}+4 y^{2}+18 x-8 y+49=0$.\\

4. $\frac{x^{2}}{4} - \frac{y^{2}}{5} = 1$ giperbolasına $3x + 2y = 0$ tuwrı sızıǵına perpendikulyar bolǵan urınba tuwrınıń teńlemesin dúziń.\\

5. Fokusı $F(2; - 1)$ noqatında jaylasqan, sáykes direktrisası $x - y - 1 = 0$ teńlemesi menen berilgen parabolanıń teńlemesin dúziń.  
\end{tabular}
\vspace{1cm}


\begin{tabular}{m{17cm}}
\textbf{21-variant}\\
1. Betliktiń kanonikalıq teńlemeleri. Betlik haqqında túsinik. (Betliktiń anıqlaması, formulaları, kósher, baǵıtlawshı tuwrılar).\\

2. Parabolanıń polyar koordinatalardaǵı teńlemesi (polyar koordinata sistemasında parabolanıń teńlemesi).\\

3. Sheńber teńlemesin dúziń: $A (1;1) $, $B (1;-1) $ hám $C (2;0) $ noqatlardan ótedi.\\

4. Koordinata kósherlerin túrlendirmey ETİS ulıwma teńlemesin ápiwaylastırıń, yarım kósherlerin tabıń: $13x^{2} + 18xy + 37y^{2} - 26x - 18y + 3 = 0$.  \\

5. $2x^{2} + 10xy + 12y^{2} - 7x + 18y - 15 = 0$ ETİS teńlemesin ápiwayı túrge alıp keliń, tipin anıqlań, qanday geometriyalıq obrazdı anıqlaytuǵının kórsetiń, sızılmasın góne hám taza koordinatalar sistemasına qarata jasań  
\end{tabular}
\vspace{1cm}


\begin{tabular}{m{17cm}}
\textbf{22-variant}\\
1. ETIS -tiń ulıwma teńlemesin ápiwaylastırıw (ETIS -tiń ulıwma teńlemesi, koordinata sistemasın túrlendirip ETIS ulıwma teńlemesin ápiwaylastırıw).\\

2. Cilindrlik betlikler (jasawshı tuwrı sızıq, baǵıtlawshı iymek sızıq, cilindrlik betlik).\\

3. Fokusları abscissa kósherinde hám koordinata basına qarata simmetriyalıq jaylasqan ellipstiń teńlemesin dúziń: yarım oqları 5 hám 2.\\

4. $x^{2} + 4y^{2} = 25$ ellipsi menen $4x - 2y + 23 = 0$ tuwrı sızıǵına parallel bolǵan urınba tuwrı sızıqtıń teńlemesin dúziń.  \\

5. Fokuslari $F(3;4), F(-3;-4)$ noqatlarında jaylasqan direktrisaları orasıdaǵı aralıq 3,6 ǵa teń bolǵan giperbolanıń teńlemesin dúziń.  
\end{tabular}
\vspace{1cm}


\begin{tabular}{m{17cm}}
\textbf{23-variant}\\
1. Koordinata sistemasın túrlendiriw (birlik vektorlar, kósherler, parallel kóshiriw, koordinata kósherlerin burıw).\\

2. Betlik haqqında túsinik (tuwrı, iymek sızıq, betliktiń anıqlamaları hám formulaları).\\

3. Polyar teńlemesi menen berilgen iymek sızıqtıń tipin anıqlań: $\rho=\frac{12}{2-\cos\theta}$.\\

4. $2x + 2y - 3 = 0$ tuwrısına perpendikulyar bolıp $x^{2} = 16y$ parabolasına urınıwshı tuwrınıń teńlemesin dúziń.  \\

5. $32x^{2} + 52xy - 7y^{2} + 180 = 0$ ETİS teńlemesin ápiwayı túrge alıp keliń, tipin anıqlań, qanday geometriyalıq obrazdı anıqlaytuǵının kórsetiń, sızılmasın góne hám taza koordinatalar sistemasına qarata jasań.  
\end{tabular}
\vspace{1cm}


\begin{tabular}{m{17cm}}
\textbf{24-variant}\\
1. Giperbolanıń polyar koordinatadaǵı teńlemesi (Polyar múyeshi, polyar radiusi giperbolanıń polyar teńlemesi).\\

2. Ellipslik paraboloid (parabola, kósher, ellipslik paraboloid).\\

3. Berilgen sızıqlardıń oraylıq ekenligin kórsetiń hám orayın tabıń: $3 x^{2}+5 xy+y^{2}-8 x-11 y-7=0$.\\

4. $2x + 2y - 3 = 0$ tuwrısına parallel bolıp $\frac{x^{2}}{16} + \frac{y^{2}}{64} = 1$ giperbolasına urınıwshı tuwrınıń teńlemesin dúziń.  \\

5. Fokusı $F( - 1; - 4)$ noqatında jaylasqan, sáykes direktrisası $x - 2 = 0$ teńlemesi menen berilgen, $A( - 3; - 5)$ noqatınan ótiwshi ellipstiń teńlemesin dúziń.  
\end{tabular}
\vspace{1cm}


\begin{tabular}{m{17cm}}
\textbf{25-variant}\\
1. Ellipstiń polyar koordinatalardaǵı teńlemesi (polyar koordinatalar sistemasında ellipstiń teńlemesi).\\

2. Bir gewekli giperboloid. Kanonikalıq teńlemesi (giperbolanı simmetriya kósheri átirapında aylandırıwdan alınǵan betlik).\\

3. Sheńber teńlemesin dúziń: orayı $C (1;-1) $ noqatında jaylasqan hám $5 x-12 y+9 -0$ tuwrı sızıǵına urınadı .\\

4. $y^{2} = 3x$ parabolası menen $\frac{x^{2}}{100} + \frac{y^{2}}{225} = 1$ ellipsiniń kesilisiw noqatların tabıń.  \\

5. $2x^{2} + 3y^{2} + 8x - 6y + 11 = 0$ teńlemesin ápiwaylastırıń qanday geometriyalıq obrazdı anıqlaytuǵının tabıń hám grafigin jasań.
\end{tabular}
\vspace{1cm}


\begin{tabular}{m{17cm}}
\textbf{26-variant}\\
1. Ellips hám onıń kanonikalıq teńlemesi (anıqlaması, fokuslar, ellipstiń kanonikalıq teńlemesi, ekscentrisiteti, direktrisaları).\\

2. Ekinshi tártipli aylanba betlikler (koordinata sisteması, tegislik, vektor iymek sızıq, aylanba betlik).\\

3. Fokusları abscissa kósherinde hám koordinata basına qarata simmetriyalıq jaylasqan ellipstiń teńlemesin dúziń: úlken kósheri $8$, direktrisaları arasındaǵı aralıq $16$.\\

4. $\frac{x^{2}}{16} - \frac{y^{2}}{64} = 1$, giperbolasına berilgen $10x - 3y + 9 = 0$ tuwrı sızıǵına parallel bolǵan urınbanıń teńlemesin dúziń.  \\

5. Tóbesi $A(-4;0)$ noqatında, al, direktrisası $y - 2 = 0$ tuwrı sızıq bolǵan parabolanıń teńlemesin dúziń.
\end{tabular}
\vspace{1cm}


\begin{tabular}{m{17cm}}
\textbf{27-variant}\\
1. Parabolanıń urınbasınıń teńlemesi (parabola, tuwrı, urınıw noqatı, urınba teńlemesi).\\

2. ETIS-tıń invariantları (ETIS-tıń ulıwma teńlemesi, túrlendiriw, ETIS invariantları ).\\

3. Polyar teńlemesi menen berilgen iymek sızıqtıń tipin anıqlań: $\rho=\frac{1}{3-3\cos\theta}$.\\

4. $41x^{2} + 24xy + 9y^{2} + 24x + 18y - 36 = 0$ ETİS tipin anıqlań hám orayların tabıń koordinata kósherlerin túrlendirmey qanday sızıqtı anıqlaytuǵının kórsetiń yarım kósherlerin tabıń.  \\

5. $4x^{2} - 4xy + y^{2} - 6x + 8y + 13 = 0$ ETİS-ǵı orayǵa iyeme? Orayǵa iye bolsa orayın anıqlań: jalǵız orayǵa iyeme-?, sheksiz orayǵa iyeme-?  
\end{tabular}
\vspace{1cm}


\begin{tabular}{m{17cm}}
\textbf{28-variant}\\
1. Eki gewekli giperboloid. Kanonikalıq teńlemesi (giperbolanı simmetriya kósheri átirapında aylandırıwdan alınǵan betlik).\\

2. Giperbolanıń urınbasınıń teńlemesi (giperbolaǵa berilgen noqatta júrgizilgen urınba teńlemesi).\\

3. Tipin anıqlań: $5 x^{2}+14 xy+11 y^{2}+12 x-7 y+19=0$.\\

4. Ellips $3x^{2} + 4y^{2} - 12 = 0$ teńlemesi menen berilgen. Onıń kósherleriniń uzınlıqların, fokuslarınıń koordinataların hám ekscentrisitetin tabıń.  \\

5. Fokusı $F(7;2)$ noqatında jaylasqan, sáykes direktrisası $x - 5 = 0$ teńlemesi menen berilgen parabolanıń teńlemesin dúziń.  
\end{tabular}
\vspace{1cm}


\begin{tabular}{m{17cm}}
\textbf{29-variant}\\
1. ETIS-tıń ulıwma teńlemesin klassifikatsiyalaw (ETIS-tıń ulıwma teńlemesi, ETIS-tıń ulıwma teńlemesin ápiwaylastırıw, klassifikatsiyalaw).\\

2. Ekinshi tártipli betliktiń ulıwma teńlemesi. Orayın anıqlaw formulası.\\

3. Sheńber teńlemesin dúziń: $M_1 (-1;5) $, $M_2 (-2;-2) $ i $M_3 (5;5) $ noqatlardan ótedi.\\

4. $3x + 4y - 12 = 0$ tuwrı sızıǵı hám $y^{2} = - 9x$ parabolasınıń kesilisiw noqatların tabıń.  \\

5. $32x^{2} + 52xy - 9y^{2} + 180 = 0$ ETİS teńlemesin ápiwaylastırıń, tipin anıqlań, qanday geometriyalıq obrazdı anıqlaytuǵının kórsetiń, sızılmasın sızıń.  
\end{tabular}
\vspace{1cm}


\begin{tabular}{m{17cm}}
\textbf{30-variant}\\
1. Parabola hám onıń kanonikalıq teńlemesi (anıqlaması, fokusı, direktrisası, kanonikalıq teńlemesi).\\

2. ETIS-tıń orayın anıqlaw forması (ETIS-tıń ulıwma teńlemesi, orayın anıqlaw forması).\\

3. Fokusları abscissa kósherinde hám koordinata basına qarata simmetriyalıq jaylasqan giperbolanıń teńlemesin dúziń: fokusları arasındaǵı aralıǵı $2 c=10$ hám kósheri $2 b=8$.\\

4. $\rho = \frac{6}{1 - cos\theta}$ polyar teńlemesi menen qanday sızıq berilgenin anıqlań.  \\

5. $16x^{2} - 9y^{2} - 64x - 54y - 161 = 0$ teńlemesi giperbolanıń teńlemesi ekenin anıqlań hám onıń orayı $C$, yarım kósherleri, ekscentrisitetin, asimptotalarınıń teńlemelerin dúziń.  
\end{tabular}
\vspace{1cm}


\begin{tabular}{m{17cm}}
\textbf{31-variant}\\
1. Giperbolalıq paraboloydtıń tuwrı sızıqlı jasawshıları (Giperbolalıq paraboloydtı jasawshı tuwrı sızıqlar dástesi).\\

2. Ellipstiń urınbasınıń teńlemesi (ellips, tuwrı, urınıw tochka, urınba teńlemesi).\\

3. Parabola teńlemesi berilgen: $y^2=6 x$. Onıń polyar teńlemesin dúziń.\\

4. $x^{2} - y^{2} = 27$ giperbolasına $4x + 2y - 7 = 0$ tuwrısına parallel bolǵan urınbanıń teńlemesin tabıń.  \\

5. Giperbolanıń ekscentrisiteti $\varepsilon = \frac{13}{12}$, fokusı $F(0;13)$ noqatında hám sáykes direktrisası $13y - 144 = 0$ teńlemesi menen berilgen bolsa, giperbolanıń teńlemesin dúziń.  
\end{tabular}
\vspace{1cm}


\begin{tabular}{m{17cm}}
\textbf{32-variant}\\
1. ETIS-tıń ulıwma teńlemesin koordinata kósherlerin burıw arqalı ápiwaylastırıń (ETIS-tıń ulıwma teńlemeleri, koordinata kósherin burıw formulası, teńlemeni kanonik túrge alıp keliw).\\

2. Ellipsoida. Kanonikalıq teńlemesi (ellipsti simmetriya kósheri dogereginde aylandırıwdan alınǵan betlik, kanonikalıq teńlemesi).\\

3. Tipin anıqlań: $x^{2}-4 xy+4 y^{2}+7 x-12=0$.\\

4. Koordinata kósherlerin túrlendirmey ETİS teńlemesin ápiwaylastırıń, yarım kósherlerin tabıń $4x^{2} - 4xy + 9y^{2} - 26x - 18y + 3 = 0$.\\

5. $2x^{2} + 3y^{2} + 8x - 6y + 11 = 0$ teńlemesin ápiwaylastırıń qanday geometriyalıq obrazdı anıqlaytuǵının tabıń hám grafigin jasań.  
\end{tabular}
\vspace{1cm}


\begin{tabular}{m{17cm}}
\textbf{33-variant}\\
1. Giperbola. Kanonikalıq teńlemesi (fokuslar, kósherler, direktrisalar, giperbola, ekscentrisitet, kanonikalıq teńlemesi).\\

2. ETIS-tıń ulıwma teńlemesin koordinata basın parallel kóshiriw arqalı ápiwayılastırıń (ETIS- tıń ulıwma teńlemesin parallel kóshiriw formulası).\\

3. Sheńberdiń $C$ orayı hám $R$ radiusın tabıń: $x^2+y^2-2 x+4 y-14=0$.\\

4. $x^{2} - 4y^{2} = 16$ giperbola berilgen. Onıń ekscentrisitetin, fokuslarınıń koordinataların tabıń hám asimptotalarınıń teńlemelerin dúziń.\\

5. $A(\frac{10}{3};\frac{5}{3})$ noqattan $\frac{x^{2}}{20} + \frac{y^{2}}{5} = 1$ ellipsine júrgizilgen urınbalardıń teńlemesin dúziń.  
\end{tabular}
\vspace{1cm}


\begin{tabular}{m{17cm}}
\textbf{34-variant}\\
1. Betliktiń kanonikalıq teńlemeleri. Betlik haqqında túsinik. (Betliktiń anıqlaması, formulaları, kósher, baǵıtlawshı tuwrılar).\\

2. Parabolanıń polyar koordinatalardaǵı teńlemesi (polyar koordinata sistemasında parabolanıń teńlemesi).\\

3. Fokusları abscissa kósherinde hám koordinata basına qarata simmetriyalıq jaylasqan giperbolanıń teńlemesin dúziń: oqları $2 a=10$ hám $2 b=8$.\\

4. $3x + 10y - 25 = 0$ tuwrı menen $\frac{x^{2}}{25} + \frac{y^{2}}{4} = 1$ ellipstiń kesilisiw noqatların tabıń.\\

5. $\frac{x^{2}}{100} + \frac{y^{2}}{36} = 1$ ellipsiniń oń jaqtaǵı fokusınan 14 ge teń aralıqta bolǵan noqattı tabıń.  
\end{tabular}
\vspace{1cm}


\begin{tabular}{m{17cm}}
\textbf{35-variant}\\
1. ETIS -tiń ulıwma teńlemesin ápiwaylastırıw (ETIS -tiń ulıwma teńlemesi, koordinata sistemasın túrlendirip ETIS ulıwma teńlemesin ápiwaylastırıw).\\

2. Cilindrlik betlikler (jasawshı tuwrı sızıq, baǵıtlawshı iymek sızıq, cilindrlik betlik).\\

3. Polyar teńlemesi menen berilgen iymek sızıqtıń tipin anıqlań: $\rho=\frac{5}{3-4\cos\theta}$.\\

4. $\rho = \frac{144}{13 - 5cos\theta}$ ellipsti anıqlaytuǵının kórsetiń hám onıń yarım kósherlerin anıqlań.\\

5. Úlken kósheri 26 ǵa, fokusları $F( - 10;0)$, $F(14;0)$ noqatlarında jaylasqan ellipstiń teńlemesin dúziń.  
\end{tabular}
\vspace{1cm}


\begin{tabular}{m{17cm}}
\textbf{36-variant}\\
1. Koordinata sistemasın túrlendiriw (birlik vektorlar, kósherler, parallel kóshiriw, koordinata kósherlerin burıw).\\

2. Betlik haqqında túsinik (tuwrı, iymek sızıq, betliktiń anıqlamaları hám formulaları).\\

3. Tipin anıqlań: $9 x^{2}-16 y^{2}-54 x-64 y-127=0$.\\

4. $\frac{x^{2}}{4} - \frac{y^{2}}{5} = 1$, giperbolanıń $3x - 2y = 0$ tuwrı sızıǵına parallel bolǵan urınbasınıń teńlemesin dúziń.  \\

5. $14x^{2} + 24xy + 21y^{2} - 4x + 18y - 139 = 0$ iymek sızıǵınıń tipin anıqlań, eger oraylı iymek sızıq bolsa orayınıń koordinataların tabıń.  
\end{tabular}
\vspace{1cm}


\begin{tabular}{m{17cm}}
\textbf{37-variant}\\
1. Giperbolanıń polyar koordinatadaǵı teńlemesi (Polyar múyeshi, polyar radiusi giperbolanıń polyar teńlemesi).\\

2. Ellipslik paraboloid (parabola, kósher, ellipslik paraboloid).\\

3. Sheńberdiń $C$ orayı hám $R$ radiusın tabıń: $x^2+y^2-2 x+4 y-20=0$.\\

4. ETİS-tıń ulıwma teńlemesin koordinata sistemasın túrlendirmey ápiwaylastırıń, tipin anıqlań, obrazı qanday sızıqtı anıqlaytuǵının kórsetiń. $7x^{2} - 8xy + y^{2} - 16x - 2y - 51 = 0$  \\

5. $\frac{x^{2}}{2} + \frac{y^{2}}{3} = 1$, ellipsin $x + y - 2 = 0$ noqatınan júrgizilgen urınbalarınıń teńlemesin dúziń.  
\end{tabular}
\vspace{1cm}


\begin{tabular}{m{17cm}}
\textbf{38-variant}\\
1. Ellipstiń polyar koordinatalardaǵı teńlemesi (polyar koordinatalar sistemasında ellipstiń teńlemesi).\\

2. Bir gewekli giperboloid. Kanonikalıq teńlemesi (giperbolanı simmetriya kósheri átirapında aylandırıwdan alınǵan betlik).\\

3. Uchı koordinata basında jaylasqan hám $Ox$ kósherine qarata joqarı yarım tegislikte jaylasqan parabolanıń teńlemesin dúziń: parametri $p=1/4$.\\

4. $y^{2} = 3x$ parabolası menen $\frac{x^{2}}{100} + \frac{y^{2}}{225} = 1$ ellipsiniń kesilisiw noqatların tabıń.  \\

5. $y^{2} = 20x$ parabolasınıń abscissası 7 ge teń bolǵan $M$ noqatınıń fokal radiusın tabıń hám fokal radiusı jatqan tuwrınıń teńlemesin dúziń.  
\end{tabular}
\vspace{1cm}


\begin{tabular}{m{17cm}}
\textbf{39-variant}\\
1. Ellips hám onıń kanonikalıq teńlemesi (anıqlaması, fokuslar, ellipstiń kanonikalıq teńlemesi, ekscentrisiteti, direktrisaları).\\

2. Ekinshi tártipli aylanba betlikler (koordinata sisteması, tegislik, vektor iymek sızıq, aylanba betlik).\\

3. Ellips teńlemesi berilgen: $\frac{x^2}{25}+\frac{y^2}{16}=1$. Onıń polyar teńlemesin dúziń.\\

4. $\rho = \frac{10}{2 - cos\theta}$ polyar teńlemesi menen qanday sızıq berilgenin anıqlań.  \\

5. Fokusı $F( - 1; - 4)$noqatında bolǵan, sáykes direktrissası $x - 2 = 0$ teńlemesi menen berilgen $A( - 3; - 5)$ noqatınan ótiwshi ellipstiń teńlemesin dúziń.  
\end{tabular}
\vspace{1cm}


\begin{tabular}{m{17cm}}
\textbf{40-variant}\\
1. Parabolanıń urınbasınıń teńlemesi (parabola, tuwrı, urınıw noqatı, urınba teńlemesi).\\

2. ETIS-tıń invariantları (ETIS-tıń ulıwma teńlemesi, túrlendiriw, ETIS invariantları ).\\

3. Berilgen sızıqlardıń oraylıq ekenligin kórsetiń hám orayın tabıń: $2 x^{2}-6 xy+5 y^{2}+22 x-36 y+11=0$.\\

4. $\frac{x^{2}}{4} - \frac{y^{2}}{5} = 1$ giperbolaǵa $3x - 2y = 0$ tuwrısına parallel bolǵan urınbanıń teńlemesin dúziń.  \\

5. $4x^{2} + 24xy + 11y^{2} + 64x + 42y + 51 = 0$ iymek sızıǵınıń tipin anıqlań eger orayı bar bolsa, onıń orayınıń koordinataların tabıń hám koordinata basın orayǵa parallel kóshiriw ámelin orınlań.  
\end{tabular}
\vspace{1cm}


\begin{tabular}{m{17cm}}
\textbf{41-variant}\\
1. Eki gewekli giperboloid. Kanonikalıq teńlemesi (giperbolanı simmetriya kósheri átirapında aylandırıwdan alınǵan betlik).\\

2. Giperbolanıń urınbasınıń teńlemesi (giperbolaǵa berilgen noqatta júrgizilgen urınba teńlemesi).\\

3. Sheńber teńlemesin dúziń: orayı koordinata basında jaylasqan hám $3 x-4 y+20=0$ tuwrı sızıǵına urınadı.\\

4. Koordinata kósherlerin túrlendirmey ETİS teńlemesin ápiwaylastırıń, yarım kósherlerin tabıń $41x^{2} + 2xy + 9y^{2} - 26x - 18y + 3 = 0$.  \\

5. $\frac{x^{2}}{3} - \frac{y^{2}}{5} = 1$ giperbolasına $P(1; - 5)$ noqatında júrgizilgen urınbalardıń teńlemesin dúziń.
\end{tabular}
\vspace{1cm}


\begin{tabular}{m{17cm}}
\textbf{42-variant}\\
1. ETIS-tıń ulıwma teńlemesin klassifikatsiyalaw (ETIS-tıń ulıwma teńlemesi, ETIS-tıń ulıwma teńlemesin ápiwaylastırıw, klassifikatsiyalaw).\\

2. Ekinshi tártipli betliktiń ulıwma teńlemesi. Orayın anıqlaw formulası.\\

3. Fokusları abscissa kósherinde hám koordinata basına qarata simmetriyalıq jaylasqan ellipstiń teńlemesin dúziń: kishi kósheri $24$, fokusları arasındaǵı aralıq $2 c=10$.\\

4. $\rho = \frac{5}{3 - 4cos\theta}$ teńlemesi menen qanday sızıq berilgenin hám yarım kósherlerin tabıń.  \\

5. $y^{2} = 20x$ parabolasınıń $M$ noqatın tabıń, eger onıń abscissası 7 ge teń bolsa, fokal radiusın hám fokal radius jaylasqan tuwrını anıqlań.
\end{tabular}
\vspace{1cm}


\begin{tabular}{m{17cm}}
\textbf{43-variant}\\
1. Parabola hám onıń kanonikalıq teńlemesi (anıqlaması, fokusı, direktrisası, kanonikalıq teńlemesi).\\

2. ETIS-tıń orayın anıqlaw forması (ETIS-tıń ulıwma teńlemesi, orayın anıqlaw forması).\\

3. Tipin anıqlań: $3 x^{2}-8 xy+7 y^{2}+8 x-15 y+20=0$.\\

4. $y^{2} = 12x$ paraborolasına $3x - 2y + 30 = 0$ tuwrı sızıǵına parallel bolǵan urınbanıń teńlemesin dúziń.  \\

5. Eger waqıttıń qálegen momentinde $M(x;y)$ noqat $5x - 16 = 0$ tuwrı sızıqqa qaraǵanda $A(5;0)$ noqattan 1,25 márte uzaqlıqta jaylasqan. Usı $M(x;y)$ noqattıń háreketiniń teńlemesin dúziń.  
\end{tabular}
\vspace{1cm}


\begin{tabular}{m{17cm}}
\textbf{44-variant}\\
1. Giperbolalıq paraboloydtıń tuwrı sızıqlı jasawshıları (Giperbolalıq paraboloydtı jasawshı tuwrı sızıqlar dástesi).\\

2. Ellipstiń urınbasınıń teńlemesi (ellips, tuwrı, urınıw tochka, urınba teńlemesi).\\

3. Sheńber teńlemesin dúziń: $A (3;1) $ hám $B (-1;3) $ noqatlardan ótedi, orayı $3 x-y-2=0$ tuwrı sızıǵında jaylasqan .\\

4. Koordinata kósherlerin túrlendirmey ETİS teńlemesin ápiwaylastırıń, qanday geometriyalıq obrazdı anıqlaytuǵının kórsetiń $4x^{2} - 4xy + y^{2} + 4x - 2y + 1 = 0$.  \\

5. $4x^{2} - 4xy + y^{2} - 2x - 14y + 7 = 0$ ETİS teńlemesin ápiwayı túrge alıp keliń, tipin anıqlań, qanday geometriyalıq obrazdı anıqlaytuǵının kórsetiń, sızılmasın góne hám taza koordinatalar sistemasına qarata jasań.  
\end{tabular}
\vspace{1cm}


\begin{tabular}{m{17cm}}
\textbf{45-variant}\\
1. ETIS-tıń ulıwma teńlemesin koordinata kósherlerin burıw arqalı ápiwaylastırıń (ETIS-tıń ulıwma teńlemeleri, koordinata kósherin burıw formulası, teńlemeni kanonik túrge alıp keliw).\\

2. Ellipsoida. Kanonikalıq teńlemesi (ellipsti simmetriya kósheri dogereginde aylandırıwdan alınǵan betlik, kanonikalıq teńlemesi).\\

3. Fokusları abscissa kósherinde hám koordinata basına qarata simmetriyalıq jaylasqan giperbolanıń teńlemesin dúziń: direktrisaları arasındaǵı aralıq $32/5$ hám kósheri $2 b=6$.\\

4. $\frac{x^{2}}{20} - \frac{y^{2}}{5} = 1$ giperbolasına $4x + 3y - 7 = 0$ tuwrısına perpendikulyar bolǵan urınbanıń teńlemesin dúziń.  \\

5. $\frac{x^{2}}{3} - \frac{y^{2}}{5} = 1$, giperbolasına $P(4;2)$ noqatınan júrgizilgen urınbalardıń teńlemesin dúziń.  
\end{tabular}
\vspace{1cm}


\begin{tabular}{m{17cm}}
\textbf{46-variant}\\
1. Giperbola. Kanonikalıq teńlemesi (fokuslar, kósherler, direktrisalar, giperbola, ekscentrisitet, kanonikalıq teńlemesi).\\

2. ETIS-tıń ulıwma teńlemesin koordinata basın parallel kóshiriw arqalı ápiwayılastırıń (ETIS- tıń ulıwma teńlemesin parallel kóshiriw formulası).\\

3. Tipin anıqlań: $3 x^{2}-2 xy-3 y^{2}+12 y-15=0$.\\

4. Koordinata kósherlerin túrlendirmey ETİS ulıwma teńlemesin ápiwaylastırıń, yarım kósherlerin tabıń: $13x^{2} + 18xy + 37y^{2} - 26x - 18y + 3 = 0$.  \\

5. $M(2; - \frac{5}{3})$ noqatı $\frac{x^{2}}{9} + \frac{y^{2}}{5} = 1$ ellipsinde jaylasqan. $M$ noqatınıń fokal radiusları jatıwshı tuwrı sızıq teńlemelerin dúziń.  
\end{tabular}
\vspace{1cm}


\begin{tabular}{m{17cm}}
\textbf{47-variant}\\
1. Betliktiń kanonikalıq teńlemeleri. Betlik haqqında túsinik. (Betliktiń anıqlaması, formulaları, kósher, baǵıtlawshı tuwrılar).\\

2. Parabolanıń polyar koordinatalardaǵı teńlemesi (polyar koordinata sistemasında parabolanıń teńlemesi).\\

3. Sheńber teńlemesin dúziń: sheńber diametriniń ushları $A (3;2) $ hám $B (-1;6 ) $ noqatlarında jaylasqan.\\

4. $\frac{x^{2}}{4} - \frac{y^{2}}{5} = 1$ giperbolasına $3x + 2y = 0$ tuwrı sızıǵına perpendikulyar bolǵan urınba tuwrınıń teńlemesin dúziń.\\

5. Eger qálegen waqıt momentinde $M(x;y)$ noqat $A(8;4)$ noqattan hám ordinata kósherinen birdey aralıqta jaylassa, $M(x;y)$ noqatınıń háreket etiw troektoriyasınıń teńlemesin dúziń.  
\end{tabular}
\vspace{1cm}


\begin{tabular}{m{17cm}}
\textbf{48-variant}\\
1. ETIS -tiń ulıwma teńlemesin ápiwaylastırıw (ETIS -tiń ulıwma teńlemesi, koordinata sistemasın túrlendirip ETIS ulıwma teńlemesin ápiwaylastırıw).\\

2. Cilindrlik betlikler (jasawshı tuwrı sızıq, baǵıtlawshı iymek sızıq, cilindrlik betlik).\\

3. Fokusları abscissa kósherinde hám koordinata basına qarata simmetriyalıq jaylasqan giperbolanıń teńlemesin dúziń: direktrisaları arasındaǵı aralıq $228/13$ hám fokusları arasındaǵı aralıq $2 c=26$.\\

4. $x^{2} + 4y^{2} = 25$ ellipsi menen $4x - 2y + 23 = 0$ tuwrı sızıǵına parallel bolǵan urınba tuwrı sızıqtıń teńlemesin dúziń.  \\

5. $2x^{2} + 3y^{2} + 8x - 6y + 11 = 0$ teńlemesi menen qanday tiptegi sızıq berilgenin anıqlań hám onıń teńlemesin ápiwaylastırıń hám grafigin jasań.  
\end{tabular}
\vspace{1cm}


\begin{tabular}{m{17cm}}
\textbf{49-variant}\\
1. Koordinata sistemasın túrlendiriw (birlik vektorlar, kósherler, parallel kóshiriw, koordinata kósherlerin burıw).\\

2. Betlik haqqında túsinik (tuwrı, iymek sızıq, betliktiń anıqlamaları hám formulaları).\\

3. Tipin anıqlań: $2 x^{2}+3 y^{2}+8 x-6 y+11=0$.\\

4. $2x + 2y - 3 = 0$ tuwrısına perpendikulyar bolıp $x^{2} = 16y$ parabolasına urınıwshı tuwrınıń teńlemesin dúziń.  \\

5. $\frac{x^{2}}{25} + \frac{y^{2}}{16} = 1$, ellipsine $C(10; - 8)$ noqatınan júrgizilgen urınbalarınıń teńlemesin dúziń.  
\end{tabular}
\vspace{1cm}


\begin{tabular}{m{17cm}}
\textbf{50-variant}\\
1. Giperbolanıń polyar koordinatadaǵı teńlemesi (Polyar múyeshi, polyar radiusi giperbolanıń polyar teńlemesi).\\

2. Ellipslik paraboloid (parabola, kósher, ellipslik paraboloid).\\

3. Sheńber teńlemesin dúziń: orayı koordinata basında jaylasqan hám radiusı $R=3$ ge teń.\\

4. $3x + 4y - 12 = 0$ tuwrı sızıǵı hám $y^{2} = - 9x$ parabolasınıń kesilisiw noqatların tabıń.  \\

5. Fokusı $F(2; - 1)$ noqatında jaylasqan, sáykes direktrisası $x - y - 1 = 0$ teńlemesi menen berilgen parabolanıń teńlemesin dúziń.  
\end{tabular}
\vspace{1cm}


\begin{tabular}{m{17cm}}
\textbf{51-variant}\\
1. Ellipstiń polyar koordinatalardaǵı teńlemesi (polyar koordinatalar sistemasında ellipstiń teńlemesi).\\

2. Bir gewekli giperboloid. Kanonikalıq teńlemesi (giperbolanı simmetriya kósheri átirapında aylandırıwdan alınǵan betlik).\\

3. Fokusları abscissa kósherinde hám koordinata basına qarata simmetriyalıq jaylasqan ellipstiń teńlemesin dúziń: kishi kósheri $6$, direktrisaları arasındaǵı aralıq $13$.\\

4. $2x + 2y - 3 = 0$ tuwrısına parallel bolıp $\frac{x^{2}}{16} + \frac{y^{2}}{64} = 1$ giperbolasına urınıwshı tuwrınıń teńlemesin dúziń.  \\

5. $2x^{2} + 10xy + 12y^{2} - 7x + 18y - 15 = 0$ ETİS teńlemesin ápiwayı túrge alıp keliń, tipin anıqlań, qanday geometriyalıq obrazdı anıqlaytuǵının kórsetiń, sızılmasın góne hám taza koordinatalar sistemasına qarata jasań  
\end{tabular}
\vspace{1cm}


\begin{tabular}{m{17cm}}
\textbf{52-variant}\\
1. Ellips hám onıń kanonikalıq teńlemesi (anıqlaması, fokuslar, ellipstiń kanonikalıq teńlemesi, ekscentrisiteti, direktrisaları).\\

2. Ekinshi tártipli aylanba betlikler (koordinata sisteması, tegislik, vektor iymek sızıq, aylanba betlik).\\

3. Tipin anıqlań: $2 x^{2}+10 xy+12 y^{2}-7 x+18 y-15=0$.\\

4. $41x^{2} + 24xy + 9y^{2} + 24x + 18y - 36 = 0$ ETİS tipin anıqlań hám orayların tabıń koordinata kósherlerin túrlendirmey qanday sızıqtı anıqlaytuǵının kórsetiń yarım kósherlerin tabıń.  \\

5. Fokuslari $F(3;4), F(-3;-4)$ noqatlarında jaylasqan direktrisaları orasıdaǵı aralıq 3,6 ǵa teń bolǵan giperbolanıń teńlemesin dúziń.  
\end{tabular}
\vspace{1cm}


\begin{tabular}{m{17cm}}
\textbf{53-variant}\\
1. Parabolanıń urınbasınıń teńlemesi (parabola, tuwrı, urınıw noqatı, urınba teńlemesi).\\

2. ETIS-tıń invariantları (ETIS-tıń ulıwma teńlemesi, túrlendiriw, ETIS invariantları ).\\

3. Fokusları abscissa kósherinde hám koordinata basına qarata simmetriyalıq jaylasqan ellipstiń teńlemesin dúziń: úlken kósheri $10$, fokusları arasındaǵı aralıq $2 c=8$.\\

4. Ellips $3x^{2} + 4y^{2} - 12 = 0$ teńlemesi menen berilgen. Onıń kósherleriniń uzınlıqların, fokuslarınıń koordinataların hám ekscentrisitetin tabıń.  \\

5. $32x^{2} + 52xy - 7y^{2} + 180 = 0$ ETİS teńlemesin ápiwayı túrge alıp keliń, tipin anıqlań, qanday geometriyalıq obrazdı anıqlaytuǵının kórsetiń, sızılmasın góne hám taza koordinatalar sistemasına qarata jasań.  
\end{tabular}
\vspace{1cm}


\begin{tabular}{m{17cm}}
\textbf{54-variant}\\
1. Eki gewekli giperboloid. Kanonikalıq teńlemesi (giperbolanı simmetriya kósheri átirapında aylandırıwdan alınǵan betlik).\\

2. Giperbolanıń urınbasınıń teńlemesi (giperbolaǵa berilgen noqatta júrgizilgen urınba teńlemesi).\\

3. Berilgen sızıqlardıń oraylıq ekenligin kórsetiń hám orayın tabıń: $5 x^{2}+4 xy+2 y^{2}+20 x+20 y-18=0$.\\

4. $3x + 10y - 25 = 0$ tuwrı menen $\frac{x^{2}}{25} + \frac{y^{2}}{4} = 1$ ellipstiń kesilisiw noqatların tabıń.\\

5. Fokusı $F( - 1; - 4)$ noqatında jaylasqan, sáykes direktrisası $x - 2 = 0$ teńlemesi menen berilgen, $A( - 3; - 5)$ noqatınan ótiwshi ellipstiń teńlemesin dúziń.  
\end{tabular}
\vspace{1cm}


\begin{tabular}{m{17cm}}
\textbf{55-variant}\\
1. ETIS-tıń ulıwma teńlemesin klassifikatsiyalaw (ETIS-tıń ulıwma teńlemesi, ETIS-tıń ulıwma teńlemesin ápiwaylastırıw, klassifikatsiyalaw).\\

2. Ekinshi tártipli betliktiń ulıwma teńlemesi. Orayın anıqlaw formulası.\\

3. Fokusları abscissa kósherinde hám koordinata basına qarata simmetriyalıq jaylasqan giperbolanıń teńlemesin dúziń: asimptotalar teńlemeleri $y=\pm \frac{4}{3}x$ hám fokusları arasındaǵı aralıq $2 c=20$.\\

4. $\rho = \frac{6}{1 - cos\theta}$ polyar teńlemesi menen qanday sızıq berilgenin anıqlań.  \\

5. $2x^{2} + 3y^{2} + 8x - 6y + 11 = 0$ teńlemesin ápiwaylastırıń qanday geometriyalıq obrazdı anıqlaytuǵının tabıń hám grafigin jasań.
\end{tabular}
\vspace{1cm}


\begin{tabular}{m{17cm}}
\textbf{56-variant}\\
1. Parabola hám onıń kanonikalıq teńlemesi (anıqlaması, fokusı, direktrisası, kanonikalıq teńlemesi).\\

2. ETIS-tıń orayın anıqlaw forması (ETIS-tıń ulıwma teńlemesi, orayın anıqlaw forması).\\

3. Fokusları abscissa kósherinde hám koordinata basına qarata simmetriyalıq jaylasqan giperbolanıń teńlemesin dúziń: direktrisaları arasındaǵı aralıq $8/3$ hám ekscentrisitet $\varepsilon=3/2$.\\

4. $\frac{x^{2}}{16} - \frac{y^{2}}{64} = 1$, giperbolasına berilgen $10x - 3y + 9 = 0$ tuwrı sızıǵına parallel bolǵan urınbanıń teńlemesin dúziń.  \\

5. Tóbesi $A(-4;0)$ noqatında, al, direktrisası $y - 2 = 0$ tuwrı sızıq bolǵan parabolanıń teńlemesin dúziń.
\end{tabular}
\vspace{1cm}


\begin{tabular}{m{17cm}}
\textbf{57-variant}\\
1. Giperbolalıq paraboloydtıń tuwrı sızıqlı jasawshıları (Giperbolalıq paraboloydtı jasawshı tuwrı sızıqlar dástesi).\\

2. Ellipstiń urınbasınıń teńlemesi (ellips, tuwrı, urınıw tochka, urınba teńlemesi).\\

3. Fokusları abscissa kósherinde hám koordinata basına qarata simmetriyalıq jaylasqan giperbolanıń teńlemesin dúziń: fokusları arasındaǵı aralıq $2 c=6$ hám ekscentrisitet $\varepsilon=3/2$.\\

4. Koordinata kósherlerin túrlendirmey ETİS teńlemesin ápiwaylastırıń, yarım kósherlerin tabıń $4x^{2} - 4xy + 9y^{2} - 26x - 18y + 3 = 0$.\\

5. $4x^{2} - 4xy + y^{2} - 6x + 8y + 13 = 0$ ETİS-ǵı orayǵa iyeme? Orayǵa iye bolsa orayın anıqlań: jalǵız orayǵa iyeme-?, sheksiz orayǵa iyeme-?  
\end{tabular}
\vspace{1cm}


\begin{tabular}{m{17cm}}
\textbf{58-variant}\\
1. ETIS-tıń ulıwma teńlemesin koordinata kósherlerin burıw arqalı ápiwaylastırıń (ETIS-tıń ulıwma teńlemeleri, koordinata kósherin burıw formulası, teńlemeni kanonik túrge alıp keliw).\\

2. Ellipsoida. Kanonikalıq teńlemesi (ellipsti simmetriya kósheri dogereginde aylandırıwdan alınǵan betlik, kanonikalıq teńlemesi).\\

3. Fokusları abscissa kósherinde hám koordinata basına qarata simmetriyalıq jaylasqan ellipstiń teńlemesin dúziń: úlken kósheri $20$, ekscentrisitet $\varepsilon=3/5$.\\

4. $x^{2} - 4y^{2} = 16$ giperbola berilgen. Onıń ekscentrisitetin, fokuslarınıń koordinataların tabıń hám asimptotalarınıń teńlemelerin dúziń.\\

5. Fokusı $F(7;2)$ noqatında jaylasqan, sáykes direktrisası $x - 5 = 0$ teńlemesi menen berilgen parabolanıń teńlemesin dúziń.  
\end{tabular}
\vspace{1cm}


\begin{tabular}{m{17cm}}
\textbf{59-variant}\\
1. Giperbola. Kanonikalıq teńlemesi (fokuslar, kósherler, direktrisalar, giperbola, ekscentrisitet, kanonikalıq teńlemesi).\\

2. ETIS-tıń ulıwma teńlemesin koordinata basın parallel kóshiriw arqalı ápiwayılastırıń (ETIS- tıń ulıwma teńlemesin parallel kóshiriw formulası).\\

3. Fokusları abscissa kósherinde hám koordinata basına qarata simmetriyalıq jaylasqan ellipstiń teńlemesin dúziń: kishi kósheri $10$, ekscentrisitet $\varepsilon=12/13$.\\

4. $y^{2} = 3x$ parabolası menen $\frac{x^{2}}{100} + \frac{y^{2}}{225} = 1$ ellipsiniń kesilisiw noqatların tabıń.  \\

5. $32x^{2} + 52xy - 9y^{2} + 180 = 0$ ETİS teńlemesin ápiwaylastırıń, tipin anıqlań, qanday geometriyalıq obrazdı anıqlaytuǵının kórsetiń, sızılmasın sızıń.  
\end{tabular}
\vspace{1cm}


\begin{tabular}{m{17cm}}
\textbf{60-variant}\\
1. Betliktiń kanonikalıq teńlemeleri. Betlik haqqında túsinik. (Betliktiń anıqlaması, formulaları, kósher, baǵıtlawshı tuwrılar).\\

2. Parabolanıń polyar koordinatalardaǵı teńlemesi (polyar koordinata sistemasında parabolanıń teńlemesi).\\

3. Fokusları abscissa kósherinde hám koordinata basına qarata simmetriyalıq jaylasqan giperbolanıń teńlemesin dúziń: úlken kósheri $2 a=16$ hám ekscentrisitet $\varepsilon=5/4$.\\

4. $\rho = \frac{144}{13 - 5cos\theta}$ ellipsti anıqlaytuǵının kórsetiń hám onıń yarım kósherlerin anıqlań.\\

5. $16x^{2} - 9y^{2} - 64x - 54y - 161 = 0$ teńlemesi giperbolanıń teńlemesi ekenin anıqlań hám onıń orayı $C$, yarım kósherleri, ekscentrisitetin, asimptotalarınıń teńlemelerin dúziń.  
\end{tabular}
\vspace{1cm}


\begin{tabular}{m{17cm}}
\textbf{61-variant}\\
1. ETIS -tiń ulıwma teńlemesin ápiwaylastırıw (ETIS -tiń ulıwma teńlemesi, koordinata sistemasın túrlendirip ETIS ulıwma teńlemesin ápiwaylastırıw).\\

2. Cilindrlik betlikler (jasawshı tuwrı sızıq, baǵıtlawshı iymek sızıq, cilindrlik betlik).\\

3. Sheńberdiń $C$ orayı hám $R$ radiusın tabıń: $x^2+y^2+4 x-2 y+5=0$.\\

4. $x^{2} - y^{2} = 27$ giperbolasına $4x + 2y - 7 = 0$ tuwrısına parallel bolǵan urınbanıń teńlemesin tabıń.  \\

5. Giperbolanıń ekscentrisiteti $\varepsilon = \frac{13}{12}$, fokusı $F(0;13)$ noqatında hám sáykes direktrisası $13y - 144 = 0$ teńlemesi menen berilgen bolsa, giperbolanıń teńlemesin dúziń.  
\end{tabular}
\vspace{1cm}


\begin{tabular}{m{17cm}}
\textbf{62-variant}\\
1. Koordinata sistemasın túrlendiriw (birlik vektorlar, kósherler, parallel kóshiriw, koordinata kósherlerin burıw).\\

2. Betlik haqqında túsinik (tuwrı, iymek sızıq, betliktiń anıqlamaları hám formulaları).\\

3. Uchı koordinata basında jaylasqan hám $Oy$ kósherine qarata oń táreptegi yarım tegislikte jaylasqan parabolanıń teńlemesin dúziń: parametri $p=3$.\\

4. ETİS-tıń ulıwma teńlemesin koordinata sistemasın túrlendirmey ápiwaylastırıń, tipin anıqlań, obrazı qanday sızıqtı anıqlaytuǵının kórsetiń. $7x^{2} - 8xy + y^{2} - 16x - 2y - 51 = 0$  \\

5. $2x^{2} + 3y^{2} + 8x - 6y + 11 = 0$ teńlemesin ápiwaylastırıń qanday geometriyalıq obrazdı anıqlaytuǵının tabıń hám grafigin jasań.  
\end{tabular}
\vspace{1cm}


\begin{tabular}{m{17cm}}
\textbf{63-variant}\\
1. Giperbolanıń polyar koordinatadaǵı teńlemesi (Polyar múyeshi, polyar radiusi giperbolanıń polyar teńlemesi).\\

2. Ellipslik paraboloid (parabola, kósher, ellipslik paraboloid).\\

3. Giperbola teńlemesi berilgen: $\frac{x^{2}}{25}-\frac{y^{2}}{144}=1$. Onıń polyar teńlemesin dúziń.\\

4. $3x + 4y - 12 = 0$ tuwrı sızıǵı hám $y^{2} = - 9x$ parabolasınıń kesilisiw noqatların tabıń.  \\

5. $A(\frac{10}{3};\frac{5}{3})$ noqattan $\frac{x^{2}}{20} + \frac{y^{2}}{5} = 1$ ellipsine júrgizilgen urınbalardıń teńlemesin dúziń.  
\end{tabular}
\vspace{1cm}


\begin{tabular}{m{17cm}}
\textbf{64-variant}\\
1. Ellipstiń polyar koordinatalardaǵı teńlemesi (polyar koordinatalar sistemasında ellipstiń teńlemesi).\\

2. Bir gewekli giperboloid. Kanonikalıq teńlemesi (giperbolanı simmetriya kósheri átirapında aylandırıwdan alınǵan betlik).\\

3. Tipin anıqlań: $25 x^{2}-20 xy+4 y^{2}-12 x+20 y-17=0$.\\

4. $\rho = \frac{10}{2 - cos\theta}$ polyar teńlemesi menen qanday sızıq berilgenin anıqlań.  \\

5. $\frac{x^{2}}{100} + \frac{y^{2}}{36} = 1$ ellipsiniń oń jaqtaǵı fokusınan 14 ge teń aralıqta bolǵan noqattı tabıń.  
\end{tabular}
\vspace{1cm}


\begin{tabular}{m{17cm}}
\textbf{65-variant}\\
1. Ellips hám onıń kanonikalıq teńlemesi (anıqlaması, fokuslar, ellipstiń kanonikalıq teńlemesi, ekscentrisiteti, direktrisaları).\\

2. Ekinshi tártipli aylanba betlikler (koordinata sisteması, tegislik, vektor iymek sızıq, aylanba betlik).\\

3. Sheńber teńlemesin dúziń: sheńber $A (2;6 ) $ noqatınan ótedi hám orayı $C (-1;2) $ noqatında jaylasqan .\\

4. $\frac{x^{2}}{4} - \frac{y^{2}}{5} = 1$, giperbolanıń $3x - 2y = 0$ tuwrı sızıǵına parallel bolǵan urınbasınıń teńlemesin dúziń.  \\

5. Úlken kósheri 26 ǵa, fokusları $F( - 10;0)$, $F(14;0)$ noqatlarında jaylasqan ellipstiń teńlemesin dúziń.  
\end{tabular}
\vspace{1cm}


\begin{tabular}{m{17cm}}
\textbf{66-variant}\\
1. Parabolanıń urınbasınıń teńlemesi (parabola, tuwrı, urınıw noqatı, urınba teńlemesi).\\

2. ETIS-tıń invariantları (ETIS-tıń ulıwma teńlemesi, túrlendiriw, ETIS invariantları ).\\

3. Uchı koordinata basında jaylasqan hám $Oy$ kósherine qarata shep táreptegi yarım tegislikte jaylasqan parabolanıń teńlemesin dúziń: parametri $p=0,5$.\\

4. Koordinata kósherlerin túrlendirmey ETİS teńlemesin ápiwaylastırıń, yarım kósherlerin tabıń $41x^{2} + 2xy + 9y^{2} - 26x - 18y + 3 = 0$.  \\

5. $14x^{2} + 24xy + 21y^{2} - 4x + 18y - 139 = 0$ iymek sızıǵınıń tipin anıqlań, eger oraylı iymek sızıq bolsa orayınıń koordinataların tabıń.  
\end{tabular}
\vspace{1cm}


\begin{tabular}{m{17cm}}
\textbf{67-variant}\\
1. Eki gewekli giperboloid. Kanonikalıq teńlemesi (giperbolanı simmetriya kósheri átirapında aylandırıwdan alınǵan betlik).\\

2. Giperbolanıń urınbasınıń teńlemesi (giperbolaǵa berilgen noqatta júrgizilgen urınba teńlemesi).\\

3. Giperbola teńlemesi berilgen: $\frac{x^{2}}{16}-\frac{y^{2}}{9}=1$. Onıń polyar teńlemesin dúziń.\\

4. $\rho = \frac{5}{3 - 4cos\theta}$ teńlemesi menen qanday sızıq berilgenin hám yarım kósherlerin tabıń.  \\

5. $\frac{x^{2}}{2} + \frac{y^{2}}{3} = 1$, ellipsin $x + y - 2 = 0$ noqatınan júrgizilgen urınbalarınıń teńlemesin dúziń.  
\end{tabular}
\vspace{1cm}


\begin{tabular}{m{17cm}}
\textbf{68-variant}\\
1. ETIS-tıń ulıwma teńlemesin klassifikatsiyalaw (ETIS-tıń ulıwma teńlemesi, ETIS-tıń ulıwma teńlemesin ápiwaylastırıw, klassifikatsiyalaw).\\

2. Ekinshi tártipli betliktiń ulıwma teńlemesi. Orayın anıqlaw formulası.\\

3. Tipin anıqlań: $4 x^2+9 y^2-40 x+36 y+100=0$.\\

4. $\frac{x^{2}}{4} - \frac{y^{2}}{5} = 1$ giperbolaǵa $3x - 2y = 0$ tuwrısına parallel bolǵan urınbanıń teńlemesin dúziń.  \\

5. $y^{2} = 20x$ parabolasınıń abscissası 7 ge teń bolǵan $M$ noqatınıń fokal radiusın tabıń hám fokal radiusı jatqan tuwrınıń teńlemesin dúziń.  
\end{tabular}
\vspace{1cm}


\begin{tabular}{m{17cm}}
\textbf{69-variant}\\
1. Parabola hám onıń kanonikalıq teńlemesi (anıqlaması, fokusı, direktrisası, kanonikalıq teńlemesi).\\

2. ETIS-tıń orayın anıqlaw forması (ETIS-tıń ulıwma teńlemesi, orayın anıqlaw forması).\\

3. Sheńber teńlemesin dúziń: orayı $C (6 ;-8) $ noqatında jaylasqan hám koordinata basınan ótedi.\\

4. Koordinata kósherlerin túrlendirmey ETİS teńlemesin ápiwaylastırıń, qanday geometriyalıq obrazdı anıqlaytuǵının kórsetiń $4x^{2} - 4xy + y^{2} + 4x - 2y + 1 = 0$.  \\

5. Fokusı $F( - 1; - 4)$noqatında bolǵan, sáykes direktrissası $x - 2 = 0$ teńlemesi menen berilgen $A( - 3; - 5)$ noqatınan ótiwshi ellipstiń teńlemesin dúziń.  
\end{tabular}
\vspace{1cm}


\begin{tabular}{m{17cm}}
\textbf{70-variant}\\
1. Giperbolalıq paraboloydtıń tuwrı sızıqlı jasawshıları (Giperbolalıq paraboloydtı jasawshı tuwrı sızıqlar dástesi).\\

2. Ellipstiń urınbasınıń teńlemesi (ellips, tuwrı, urınıw tochka, urınba teńlemesi).\\

3. Fokusları abscissa kósherinde hám koordinata basına qarata simmetriyalıq jaylasqan ellipstiń teńlemesin dúziń: fokusları arasındaǵı aralıq $2 c=6$ hám ekscentrisitet $\varepsilon=3/5$.\\

4. $y^{2} = 12x$ paraborolasına $3x - 2y + 30 = 0$ tuwrı sızıǵına parallel bolǵan urınbanıń teńlemesin dúziń.  \\

5. $4x^{2} + 24xy + 11y^{2} + 64x + 42y + 51 = 0$ iymek sızıǵınıń tipin anıqlań eger orayı bar bolsa, onıń orayınıń koordinataların tabıń hám koordinata basın orayǵa parallel kóshiriw ámelin orınlań.  
\end{tabular}
\vspace{1cm}


\begin{tabular}{m{17cm}}
\textbf{71-variant}\\
1. ETIS-tıń ulıwma teńlemesin koordinata kósherlerin burıw arqalı ápiwaylastırıń (ETIS-tıń ulıwma teńlemeleri, koordinata kósherin burıw formulası, teńlemeni kanonik túrge alıp keliw).\\

2. Ellipsoida. Kanonikalıq teńlemesi (ellipsti simmetriya kósheri dogereginde aylandırıwdan alınǵan betlik, kanonikalıq teńlemesi).\\

3. Polyar teńlemesi menen berilgen iymek sızıqtıń tipin anıqlań: $\rho=\frac{5}{1-\frac{1}{2}\cos\theta}$.\\

4. Koordinata kósherlerin túrlendirmey ETİS ulıwma teńlemesin ápiwaylastırıń, yarım kósherlerin tabıń: $13x^{2} + 18xy + 37y^{2} - 26x - 18y + 3 = 0$.  \\

5. $\frac{x^{2}}{3} - \frac{y^{2}}{5} = 1$ giperbolasına $P(1; - 5)$ noqatında júrgizilgen urınbalardıń teńlemesin dúziń.
\end{tabular}
\vspace{1cm}


\begin{tabular}{m{17cm}}
\textbf{72-variant}\\
1. Giperbola. Kanonikalıq teńlemesi (fokuslar, kósherler, direktrisalar, giperbola, ekscentrisitet, kanonikalıq teńlemesi).\\

2. ETIS-tıń ulıwma teńlemesin koordinata basın parallel kóshiriw arqalı ápiwayılastırıń (ETIS- tıń ulıwma teńlemesin parallel kóshiriw formulası).\\

3. Tipin anıqlań: $4 x^{2}-y^{2}+8 x-2 y+3=0$.\\

4. $\frac{x^{2}}{20} - \frac{y^{2}}{5} = 1$ giperbolasına $4x + 3y - 7 = 0$ tuwrısına perpendikulyar bolǵan urınbanıń teńlemesin dúziń.  \\

5. $y^{2} = 20x$ parabolasınıń $M$ noqatın tabıń, eger onıń abscissası 7 ge teń bolsa, fokal radiusın hám fokal radius jaylasqan tuwrını anıqlań.
\end{tabular}
\vspace{1cm}


\begin{tabular}{m{17cm}}
\textbf{73-variant}\\
1. Betliktiń kanonikalıq teńlemeleri. Betlik haqqında túsinik. (Betliktiń anıqlaması, formulaları, kósher, baǵıtlawshı tuwrılar).\\

2. Parabolanıń polyar koordinatalardaǵı teńlemesi (polyar koordinata sistemasında parabolanıń teńlemesi).\\

3. Sheńber teńlemesin dúziń: orayı $C (2;-3) $ noqatında jaylasqan hám radiusı $R=7$ ge teń.\\

4. $\frac{x^{2}}{4} - \frac{y^{2}}{5} = 1$ giperbolasına $3x + 2y = 0$ tuwrı sızıǵına perpendikulyar bolǵan urınba tuwrınıń teńlemesin dúziń.\\

5. Eger waqıttıń qálegen momentinde $M(x;y)$ noqat $5x - 16 = 0$ tuwrı sızıqqa qaraǵanda $A(5;0)$ noqattan 1,25 márte uzaqlıqta jaylasqan. Usı $M(x;y)$ noqattıń háreketiniń teńlemesin dúziń.  
\end{tabular}
\vspace{1cm}


\begin{tabular}{m{17cm}}
\textbf{74-variant}\\
1. ETIS -tiń ulıwma teńlemesin ápiwaylastırıw (ETIS -tiń ulıwma teńlemesi, koordinata sistemasın túrlendirip ETIS ulıwma teńlemesin ápiwaylastırıw).\\

2. Cilindrlik betlikler (jasawshı tuwrı sızıq, baǵıtlawshı iymek sızıq, cilindrlik betlik).\\

3. Uchı koordinata basında jaylasqan hám $Ox$ kósherine qarata tómengi yarım tegislikte jaylasqan parabolanıń teńlemesin dúziń: parametri $p=3$.\\

4. $x^{2} + 4y^{2} = 25$ ellipsi menen $4x - 2y + 23 = 0$ tuwrı sızıǵına parallel bolǵan urınba tuwrı sızıqtıń teńlemesin dúziń.  \\

5. $4x^{2} - 4xy + y^{2} - 2x - 14y + 7 = 0$ ETİS teńlemesin ápiwayı túrge alıp keliń, tipin anıqlań, qanday geometriyalıq obrazdı anıqlaytuǵının kórsetiń, sızılmasın góne hám taza koordinatalar sistemasına qarata jasań.  
\end{tabular}
\vspace{1cm}


\begin{tabular}{m{17cm}}
\textbf{75-variant}\\
1. Koordinata sistemasın túrlendiriw (birlik vektorlar, kósherler, parallel kóshiriw, koordinata kósherlerin burıw).\\

2. Betlik haqqında túsinik (tuwrı, iymek sızıq, betliktiń anıqlamaları hám formulaları).\\

3. Polyar teńlemesi menen berilgen iymek sızıqtıń tipin anıqlań: $\rho=\frac{10}{1-\frac{3}{2}\cos\theta}$.\\

4. $3x + 10y - 25 = 0$ tuwrı menen $\frac{x^{2}}{25} + \frac{y^{2}}{4} = 1$ ellipstiń kesilisiw noqatların tabıń.\\

5. $\frac{x^{2}}{3} - \frac{y^{2}}{5} = 1$, giperbolasına $P(4;2)$ noqatınan júrgizilgen urınbalardıń teńlemesin dúziń.  
\end{tabular}
\vspace{1cm}


\begin{tabular}{m{17cm}}
\textbf{76-variant}\\
1. Giperbolanıń polyar koordinatadaǵı teńlemesi (Polyar múyeshi, polyar radiusi giperbolanıń polyar teńlemesi).\\

2. Ellipslik paraboloid (parabola, kósher, ellipslik paraboloid).\\

3. Berilgen sızıqlardıń oraylıq ekenligin kórsetiń hám orayın tabıń: $9 x^{2}-4 xy-7 y^{2}-12=0$.\\

4. $2x + 2y - 3 = 0$ tuwrısına perpendikulyar bolıp $x^{2} = 16y$ parabolasına urınıwshı tuwrınıń teńlemesin dúziń.  \\

5. $M(2; - \frac{5}{3})$ noqatı $\frac{x^{2}}{9} + \frac{y^{2}}{5} = 1$ ellipsinde jaylasqan. $M$ noqatınıń fokal radiusları jatıwshı tuwrı sızıq teńlemelerin dúziń.  
\end{tabular}
\vspace{1cm}


\begin{tabular}{m{17cm}}
\textbf{77-variant}\\
1. Ellipstiń polyar koordinatalardaǵı teńlemesi (polyar koordinatalar sistemasında ellipstiń teńlemesi).\\

2. Bir gewekli giperboloid. Kanonikalıq teńlemesi (giperbolanı simmetriya kósheri átirapında aylandırıwdan alınǵan betlik).\\

3. Sheńberdiń $C$ orayı hám $R$ radiusın tabıń: $x^2+y^2+6 x-4 y+14=0$.\\

4. $41x^{2} + 24xy + 9y^{2} + 24x + 18y - 36 = 0$ ETİS tipin anıqlań hám orayların tabıń koordinata kósherlerin túrlendirmey qanday sızıqtı anıqlaytuǵının kórsetiń yarım kósherlerin tabıń.  \\

5. Eger qálegen waqıt momentinde $M(x;y)$ noqat $A(8;4)$ noqattan hám ordinata kósherinen birdey aralıqta jaylassa, $M(x;y)$ noqatınıń háreket etiw troektoriyasınıń teńlemesin dúziń.  
\end{tabular}
\vspace{1cm}


\begin{tabular}{m{17cm}}
\textbf{78-variant}\\
1. Ellips hám onıń kanonikalıq teńlemesi (anıqlaması, fokuslar, ellipstiń kanonikalıq teńlemesi, ekscentrisiteti, direktrisaları).\\

2. Ekinshi tártipli aylanba betlikler (koordinata sisteması, tegislik, vektor iymek sızıq, aylanba betlik).\\

3. Fokusları abscissa kósherinde hám koordinata basına qarata simmetriyalıq jaylasqan ellipstiń teńlemesin dúziń: direktrisaları arasındaǵı aralıq $5$ hám fokusları arasındaǵı aralıq $2 c=4$.\\

4. Ellips $3x^{2} + 4y^{2} - 12 = 0$ teńlemesi menen berilgen. Onıń kósherleriniń uzınlıqların, fokuslarınıń koordinataların hám ekscentrisitetin tabıń.  \\

5. $2x^{2} + 3y^{2} + 8x - 6y + 11 = 0$ teńlemesi menen qanday tiptegi sızıq berilgenin anıqlań hám onıń teńlemesin ápiwaylastırıń hám grafigin jasań.  
\end{tabular}
\vspace{1cm}


\begin{tabular}{m{17cm}}
\textbf{79-variant}\\
1. Parabolanıń urınbasınıń teńlemesi (parabola, tuwrı, urınıw noqatı, urınba teńlemesi).\\

2. ETIS-tıń invariantları (ETIS-tıń ulıwma teńlemesi, túrlendiriw, ETIS invariantları ).\\

3. Polyar teńlemesi menen berilgen iymek sızıqtıń tipin anıqlań: $\rho=\frac{6}{1-\cos 0}$.\\

4. $y^{2} = 3x$ parabolası menen $\frac{x^{2}}{100} + \frac{y^{2}}{225} = 1$ ellipsiniń kesilisiw noqatların tabıń.  \\

5. $\frac{x^{2}}{25} + \frac{y^{2}}{16} = 1$, ellipsine $C(10; - 8)$ noqatınan júrgizilgen urınbalarınıń teńlemesin dúziń.  
\end{tabular}
\vspace{1cm}


\begin{tabular}{m{17cm}}
\textbf{80-variant}\\
1. Eki gewekli giperboloid. Kanonikalıq teńlemesi (giperbolanı simmetriya kósheri átirapında aylandırıwdan alınǵan betlik).\\

2. Giperbolanıń urınbasınıń teńlemesi (giperbolaǵa berilgen noqatta júrgizilgen urınba teńlemesi).\\

3. Tipin anıqlań: $9 x^{2}+4 y^{2}+18 x-8 y+49=0$.\\

4. $\rho = \frac{6}{1 - cos\theta}$ polyar teńlemesi menen qanday sızıq berilgenin anıqlań.  \\

5. Fokusı $F(2; - 1)$ noqatında jaylasqan, sáykes direktrisası $x - y - 1 = 0$ teńlemesi menen berilgen parabolanıń teńlemesin dúziń.  
\end{tabular}
\vspace{1cm}


\begin{tabular}{m{17cm}}
\textbf{81-variant}\\
1. ETIS-tıń ulıwma teńlemesin klassifikatsiyalaw (ETIS-tıń ulıwma teńlemesi, ETIS-tıń ulıwma teńlemesin ápiwaylastırıw, klassifikatsiyalaw).\\

2. Ekinshi tártipli betliktiń ulıwma teńlemesi. Orayın anıqlaw formulası.\\

3. Sheńber teńlemesin dúziń: $A (1;1) $, $B (1;-1) $ hám $C (2;0) $ noqatlardan ótedi.\\

4. $2x + 2y - 3 = 0$ tuwrısına parallel bolıp $\frac{x^{2}}{16} + \frac{y^{2}}{64} = 1$ giperbolasına urınıwshı tuwrınıń teńlemesin dúziń.  \\

5. $2x^{2} + 10xy + 12y^{2} - 7x + 18y - 15 = 0$ ETİS teńlemesin ápiwayı túrge alıp keliń, tipin anıqlań, qanday geometriyalıq obrazdı anıqlaytuǵının kórsetiń, sızılmasın góne hám taza koordinatalar sistemasına qarata jasań  
\end{tabular}
\vspace{1cm}


\begin{tabular}{m{17cm}}
\textbf{82-variant}\\
1. Parabola hám onıń kanonikalıq teńlemesi (anıqlaması, fokusı, direktrisası, kanonikalıq teńlemesi).\\

2. ETIS-tıń orayın anıqlaw forması (ETIS-tıń ulıwma teńlemesi, orayın anıqlaw forması).\\

3. Fokusları abscissa kósherinde hám koordinata basına qarata simmetriyalıq jaylasqan ellipstiń teńlemesin dúziń: yarım oqları 5 hám 2.\\

4. Koordinata kósherlerin túrlendirmey ETİS teńlemesin ápiwaylastırıń, yarım kósherlerin tabıń $4x^{2} - 4xy + 9y^{2} - 26x - 18y + 3 = 0$.\\

5. Fokuslari $F(3;4), F(-3;-4)$ noqatlarında jaylasqan direktrisaları orasıdaǵı aralıq 3,6 ǵa teń bolǵan giperbolanıń teńlemesin dúziń.  
\end{tabular}
\vspace{1cm}


\begin{tabular}{m{17cm}}
\textbf{83-variant}\\
1. Giperbolalıq paraboloydtıń tuwrı sızıqlı jasawshıları (Giperbolalıq paraboloydtı jasawshı tuwrı sızıqlar dástesi).\\

2. Ellipstiń urınbasınıń teńlemesi (ellips, tuwrı, urınıw tochka, urınba teńlemesi).\\

3. Polyar teńlemesi menen berilgen iymek sızıqtıń tipin anıqlań: $\rho=\frac{12}{2-\cos\theta}$.\\

4. $x^{2} - 4y^{2} = 16$ giperbola berilgen. Onıń ekscentrisitetin, fokuslarınıń koordinataların tabıń hám asimptotalarınıń teńlemelerin dúziń.\\

5. $32x^{2} + 52xy - 7y^{2} + 180 = 0$ ETİS teńlemesin ápiwayı túrge alıp keliń, tipin anıqlań, qanday geometriyalıq obrazdı anıqlaytuǵının kórsetiń, sızılmasın góne hám taza koordinatalar sistemasına qarata jasań.  
\end{tabular}
\vspace{1cm}


\begin{tabular}{m{17cm}}
\textbf{84-variant}\\
1. ETIS-tıń ulıwma teńlemesin koordinata kósherlerin burıw arqalı ápiwaylastırıń (ETIS-tıń ulıwma teńlemeleri, koordinata kósherin burıw formulası, teńlemeni kanonik túrge alıp keliw).\\

2. Ellipsoida. Kanonikalıq teńlemesi (ellipsti simmetriya kósheri dogereginde aylandırıwdan alınǵan betlik, kanonikalıq teńlemesi).\\

3. Berilgen sızıqlardıń oraylıq ekenligin kórsetiń hám orayın tabıń: $3 x^{2}+5 xy+y^{2}-8 x-11 y-7=0$.\\

4. $3x + 4y - 12 = 0$ tuwrı sızıǵı hám $y^{2} = - 9x$ parabolasınıń kesilisiw noqatların tabıń.  \\

5. Fokusı $F( - 1; - 4)$ noqatında jaylasqan, sáykes direktrisası $x - 2 = 0$ teńlemesi menen berilgen, $A( - 3; - 5)$ noqatınan ótiwshi ellipstiń teńlemesin dúziń.  
\end{tabular}
\vspace{1cm}


\begin{tabular}{m{17cm}}
\textbf{85-variant}\\
1. Giperbola. Kanonikalıq teńlemesi (fokuslar, kósherler, direktrisalar, giperbola, ekscentrisitet, kanonikalıq teńlemesi).\\

2. ETIS-tıń ulıwma teńlemesin koordinata basın parallel kóshiriw arqalı ápiwayılastırıń (ETIS- tıń ulıwma teńlemesin parallel kóshiriw formulası).\\

3. Sheńber teńlemesin dúziń: orayı $C (1;-1) $ noqatında jaylasqan hám $5 x-12 y+9 -0$ tuwrı sızıǵına urınadı .\\

4. $\rho = \frac{144}{13 - 5cos\theta}$ ellipsti anıqlaytuǵının kórsetiń hám onıń yarım kósherlerin anıqlań.\\

5. $2x^{2} + 3y^{2} + 8x - 6y + 11 = 0$ teńlemesin ápiwaylastırıń qanday geometriyalıq obrazdı anıqlaytuǵının tabıń hám grafigin jasań.
\end{tabular}
\vspace{1cm}


\begin{tabular}{m{17cm}}
\textbf{86-variant}\\
1. Betliktiń kanonikalıq teńlemeleri. Betlik haqqında túsinik. (Betliktiń anıqlaması, formulaları, kósher, baǵıtlawshı tuwrılar).\\

2. Parabolanıń polyar koordinatalardaǵı teńlemesi (polyar koordinata sistemasında parabolanıń teńlemesi).\\

3. Fokusları abscissa kósherinde hám koordinata basına qarata simmetriyalıq jaylasqan ellipstiń teńlemesin dúziń: úlken kósheri $8$, direktrisaları arasındaǵı aralıq $16$.\\

4. $\frac{x^{2}}{16} - \frac{y^{2}}{64} = 1$, giperbolasına berilgen $10x - 3y + 9 = 0$ tuwrı sızıǵına parallel bolǵan urınbanıń teńlemesin dúziń.  \\

5. Tóbesi $A(-4;0)$ noqatında, al, direktrisası $y - 2 = 0$ tuwrı sızıq bolǵan parabolanıń teńlemesin dúziń.
\end{tabular}
\vspace{1cm}


\begin{tabular}{m{17cm}}
\textbf{87-variant}\\
1. ETIS -tiń ulıwma teńlemesin ápiwaylastırıw (ETIS -tiń ulıwma teńlemesi, koordinata sistemasın túrlendirip ETIS ulıwma teńlemesin ápiwaylastırıw).\\

2. Cilindrlik betlikler (jasawshı tuwrı sızıq, baǵıtlawshı iymek sızıq, cilindrlik betlik).\\

3. Polyar teńlemesi menen berilgen iymek sızıqtıń tipin anıqlań: $\rho=\frac{1}{3-3\cos\theta}$.\\

4. ETİS-tıń ulıwma teńlemesin koordinata sistemasın túrlendirmey ápiwaylastırıń, tipin anıqlań, obrazı qanday sızıqtı anıqlaytuǵının kórsetiń. $7x^{2} - 8xy + y^{2} - 16x - 2y - 51 = 0$  \\

5. $4x^{2} - 4xy + y^{2} - 6x + 8y + 13 = 0$ ETİS-ǵı orayǵa iyeme? Orayǵa iye bolsa orayın anıqlań: jalǵız orayǵa iyeme-?, sheksiz orayǵa iyeme-?  
\end{tabular}
\vspace{1cm}


\begin{tabular}{m{17cm}}
\textbf{88-variant}\\
1. Koordinata sistemasın túrlendiriw (birlik vektorlar, kósherler, parallel kóshiriw, koordinata kósherlerin burıw).\\

2. Betlik haqqında túsinik (tuwrı, iymek sızıq, betliktiń anıqlamaları hám formulaları).\\

3. Tipin anıqlań: $5 x^{2}+14 xy+11 y^{2}+12 x-7 y+19=0$.\\

4. $3x + 10y - 25 = 0$ tuwrı menen $\frac{x^{2}}{25} + \frac{y^{2}}{4} = 1$ ellipstiń kesilisiw noqatların tabıń.\\

5. Fokusı $F(7;2)$ noqatında jaylasqan, sáykes direktrisası $x - 5 = 0$ teńlemesi menen berilgen parabolanıń teńlemesin dúziń.  
\end{tabular}
\vspace{1cm}


\begin{tabular}{m{17cm}}
\textbf{89-variant}\\
1. Giperbolanıń polyar koordinatadaǵı teńlemesi (Polyar múyeshi, polyar radiusi giperbolanıń polyar teńlemesi).\\

2. Ellipslik paraboloid (parabola, kósher, ellipslik paraboloid).\\

3. Sheńber teńlemesin dúziń: $M_1 (-1;5) $, $M_2 (-2;-2) $ i $M_3 (5;5) $ noqatlardan ótedi.\\

4. $\rho = \frac{10}{2 - cos\theta}$ polyar teńlemesi menen qanday sızıq berilgenin anıqlań.  \\

5. $32x^{2} + 52xy - 9y^{2} + 180 = 0$ ETİS teńlemesin ápiwaylastırıń, tipin anıqlań, qanday geometriyalıq obrazdı anıqlaytuǵının kórsetiń, sızılmasın sızıń.  
\end{tabular}
\vspace{1cm}


\begin{tabular}{m{17cm}}
\textbf{90-variant}\\
1. Ellipstiń polyar koordinatalardaǵı teńlemesi (polyar koordinatalar sistemasında ellipstiń teńlemesi).\\

2. Bir gewekli giperboloid. Kanonikalıq teńlemesi (giperbolanı simmetriya kósheri átirapında aylandırıwdan alınǵan betlik).\\

3. Fokusları abscissa kósherinde hám koordinata basına qarata simmetriyalıq jaylasqan giperbolanıń teńlemesin dúziń: fokusları arasındaǵı aralıǵı $2 c=10$ hám kósheri $2 b=8$.\\

4. $x^{2} - y^{2} = 27$ giperbolasına $4x + 2y - 7 = 0$ tuwrısına parallel bolǵan urınbanıń teńlemesin tabıń.  \\

5. $16x^{2} - 9y^{2} - 64x - 54y - 161 = 0$ teńlemesi giperbolanıń teńlemesi ekenin anıqlań hám onıń orayı $C$, yarım kósherleri, ekscentrisitetin, asimptotalarınıń teńlemelerin dúziń.  
\end{tabular}
\vspace{1cm}


\begin{tabular}{m{17cm}}
\textbf{91-variant}\\
1. Ellips hám onıń kanonikalıq teńlemesi (anıqlaması, fokuslar, ellipstiń kanonikalıq teńlemesi, ekscentrisiteti, direktrisaları).\\

2. Ekinshi tártipli aylanba betlikler (koordinata sisteması, tegislik, vektor iymek sızıq, aylanba betlik).\\

3. Parabola teńlemesi berilgen: $y^2=6 x$. Onıń polyar teńlemesin dúziń.\\

4. Koordinata kósherlerin túrlendirmey ETİS teńlemesin ápiwaylastırıń, yarım kósherlerin tabıń $41x^{2} + 2xy + 9y^{2} - 26x - 18y + 3 = 0$.  \\

5. Giperbolanıń ekscentrisiteti $\varepsilon = \frac{13}{12}$, fokusı $F(0;13)$ noqatında hám sáykes direktrisası $13y - 144 = 0$ teńlemesi menen berilgen bolsa, giperbolanıń teńlemesin dúziń.  
\end{tabular}
\vspace{1cm}


\begin{tabular}{m{17cm}}
\textbf{92-variant}\\
1. Parabolanıń urınbasınıń teńlemesi (parabola, tuwrı, urınıw noqatı, urınba teńlemesi).\\

2. ETIS-tıń invariantları (ETIS-tıń ulıwma teńlemesi, túrlendiriw, ETIS invariantları ).\\

3. Tipin anıqlań: $x^{2}-4 xy+4 y^{2}+7 x-12=0$.\\

4. $\rho = \frac{5}{3 - 4cos\theta}$ teńlemesi menen qanday sızıq berilgenin hám yarım kósherlerin tabıń.  \\

5. $2x^{2} + 3y^{2} + 8x - 6y + 11 = 0$ teńlemesin ápiwaylastırıń qanday geometriyalıq obrazdı anıqlaytuǵının tabıń hám grafigin jasań.  
\end{tabular}
\vspace{1cm}


\begin{tabular}{m{17cm}}
\textbf{93-variant}\\
1. Eki gewekli giperboloid. Kanonikalıq teńlemesi (giperbolanı simmetriya kósheri átirapında aylandırıwdan alınǵan betlik).\\

2. Giperbolanıń urınbasınıń teńlemesi (giperbolaǵa berilgen noqatta júrgizilgen urınba teńlemesi).\\

3. Sheńberdiń $C$ orayı hám $R$ radiusın tabıń: $x^2+y^2-2 x+4 y-14=0$.\\

4. $\frac{x^{2}}{4} - \frac{y^{2}}{5} = 1$, giperbolanıń $3x - 2y = 0$ tuwrı sızıǵına parallel bolǵan urınbasınıń teńlemesin dúziń.  \\

5. $A(\frac{10}{3};\frac{5}{3})$ noqattan $\frac{x^{2}}{20} + \frac{y^{2}}{5} = 1$ ellipsine júrgizilgen urınbalardıń teńlemesin dúziń.  
\end{tabular}
\vspace{1cm}


\begin{tabular}{m{17cm}}
\textbf{94-variant}\\
1. ETIS-tıń ulıwma teńlemesin klassifikatsiyalaw (ETIS-tıń ulıwma teńlemesi, ETIS-tıń ulıwma teńlemesin ápiwaylastırıw, klassifikatsiyalaw).\\

2. Ekinshi tártipli betliktiń ulıwma teńlemesi. Orayın anıqlaw formulası.\\

3. Fokusları abscissa kósherinde hám koordinata basına qarata simmetriyalıq jaylasqan giperbolanıń teńlemesin dúziń: oqları $2 a=10$ hám $2 b=8$.\\

4. Koordinata kósherlerin túrlendirmey ETİS teńlemesin ápiwaylastırıń, qanday geometriyalıq obrazdı anıqlaytuǵının kórsetiń $4x^{2} - 4xy + y^{2} + 4x - 2y + 1 = 0$.  \\

5. $\frac{x^{2}}{100} + \frac{y^{2}}{36} = 1$ ellipsiniń oń jaqtaǵı fokusınan 14 ge teń aralıqta bolǵan noqattı tabıń.  
\end{tabular}
\vspace{1cm}


\begin{tabular}{m{17cm}}
\textbf{95-variant}\\
1. Parabola hám onıń kanonikalıq teńlemesi (anıqlaması, fokusı, direktrisası, kanonikalıq teńlemesi).\\

2. ETIS-tıń orayın anıqlaw forması (ETIS-tıń ulıwma teńlemesi, orayın anıqlaw forması).\\

3. Polyar teńlemesi menen berilgen iymek sızıqtıń tipin anıqlań: $\rho=\frac{5}{3-4\cos\theta}$.\\

4. $\frac{x^{2}}{4} - \frac{y^{2}}{5} = 1$ giperbolaǵa $3x - 2y = 0$ tuwrısına parallel bolǵan urınbanıń teńlemesin dúziń.  \\

5. Úlken kósheri 26 ǵa, fokusları $F( - 10;0)$, $F(14;0)$ noqatlarında jaylasqan ellipstiń teńlemesin dúziń.  
\end{tabular}
\vspace{1cm}


\begin{tabular}{m{17cm}}
\textbf{96-variant}\\
1. Giperbolalıq paraboloydtıń tuwrı sızıqlı jasawshıları (Giperbolalıq paraboloydtı jasawshı tuwrı sızıqlar dástesi).\\

2. Ellipstiń urınbasınıń teńlemesi (ellips, tuwrı, urınıw tochka, urınba teńlemesi).\\

3. Tipin anıqlań: $9 x^{2}-16 y^{2}-54 x-64 y-127=0$.\\

4. Koordinata kósherlerin túrlendirmey ETİS ulıwma teńlemesin ápiwaylastırıń, yarım kósherlerin tabıń: $13x^{2} + 18xy + 37y^{2} - 26x - 18y + 3 = 0$.  \\

5. $14x^{2} + 24xy + 21y^{2} - 4x + 18y - 139 = 0$ iymek sızıǵınıń tipin anıqlań, eger oraylı iymek sızıq bolsa orayınıń koordinataların tabıń.  
\end{tabular}
\vspace{1cm}


\begin{tabular}{m{17cm}}
\textbf{97-variant}\\
1. ETIS-tıń ulıwma teńlemesin koordinata kósherlerin burıw arqalı ápiwaylastırıń (ETIS-tıń ulıwma teńlemeleri, koordinata kósherin burıw formulası, teńlemeni kanonik túrge alıp keliw).\\

2. Ellipsoida. Kanonikalıq teńlemesi (ellipsti simmetriya kósheri dogereginde aylandırıwdan alınǵan betlik, kanonikalıq teńlemesi).\\

3. Sheńberdiń $C$ orayı hám $R$ radiusın tabıń: $x^2+y^2-2 x+4 y-20=0$.\\

4. $y^{2} = 12x$ paraborolasına $3x - 2y + 30 = 0$ tuwrı sızıǵına parallel bolǵan urınbanıń teńlemesin dúziń.  \\

5. $\frac{x^{2}}{2} + \frac{y^{2}}{3} = 1$, ellipsin $x + y - 2 = 0$ noqatınan júrgizilgen urınbalarınıń teńlemesin dúziń.  
\end{tabular}
\vspace{1cm}


\begin{tabular}{m{17cm}}
\textbf{98-variant}\\
1. Giperbola. Kanonikalıq teńlemesi (fokuslar, kósherler, direktrisalar, giperbola, ekscentrisitet, kanonikalıq teńlemesi).\\

2. ETIS-tıń ulıwma teńlemesin koordinata basın parallel kóshiriw arqalı ápiwayılastırıń (ETIS- tıń ulıwma teńlemesin parallel kóshiriw formulası).\\

3. Uchı koordinata basında jaylasqan hám $Ox$ kósherine qarata joqarı yarım tegislikte jaylasqan parabolanıń teńlemesin dúziń: parametri $p=1/4$.\\

4. $\frac{x^{2}}{20} - \frac{y^{2}}{5} = 1$ giperbolasına $4x + 3y - 7 = 0$ tuwrısına perpendikulyar bolǵan urınbanıń teńlemesin dúziń.  \\

5. $y^{2} = 20x$ parabolasınıń abscissası 7 ge teń bolǵan $M$ noqatınıń fokal radiusın tabıń hám fokal radiusı jatqan tuwrınıń teńlemesin dúziń.  
\end{tabular}
\vspace{1cm}


\begin{tabular}{m{17cm}}
\textbf{99-variant}\\
1. Betliktiń kanonikalıq teńlemeleri. Betlik haqqında túsinik. (Betliktiń anıqlaması, formulaları, kósher, baǵıtlawshı tuwrılar).\\

2. Parabolanıń polyar koordinatalardaǵı teńlemesi (polyar koordinata sistemasında parabolanıń teńlemesi).\\

3. Ellips teńlemesi berilgen: $\frac{x^2}{25}+\frac{y^2}{16}=1$. Onıń polyar teńlemesin dúziń.\\

4. $\frac{x^{2}}{4} - \frac{y^{2}}{5} = 1$ giperbolasına $3x + 2y = 0$ tuwrı sızıǵına perpendikulyar bolǵan urınba tuwrınıń teńlemesin dúziń.\\

5. Fokusı $F( - 1; - 4)$noqatında bolǵan, sáykes direktrissası $x - 2 = 0$ teńlemesi menen berilgen $A( - 3; - 5)$ noqatınan ótiwshi ellipstiń teńlemesin dúziń.  
\end{tabular}
\vspace{1cm}


\begin{tabular}{m{17cm}}
\textbf{100-variant}\\
1. ETIS -tiń ulıwma teńlemesin ápiwaylastırıw (ETIS -tiń ulıwma teńlemesi, koordinata sistemasın túrlendirip ETIS ulıwma teńlemesin ápiwaylastırıw).\\

2. Cilindrlik betlikler (jasawshı tuwrı sızıq, baǵıtlawshı iymek sızıq, cilindrlik betlik).\\

3. Berilgen sızıqlardıń oraylıq ekenligin kórsetiń hám orayın tabıń: $2 x^{2}-6 xy+5 y^{2}+22 x-36 y+11=0$.\\

4. $y^{2} = 3x$ parabolası menen $\frac{x^{2}}{100} + \frac{y^{2}}{225} = 1$ ellipsiniń kesilisiw noqatların tabıń.  \\

5. $4x^{2} + 24xy + 11y^{2} + 64x + 42y + 51 = 0$ iymek sızıǵınıń tipin anıqlań eger orayı bar bolsa, onıń orayınıń koordinataların tabıń hám koordinata basın orayǵa parallel kóshiriw ámelin orınlań.  
\end{tabular}
\vspace{1cm}



\end{document}
