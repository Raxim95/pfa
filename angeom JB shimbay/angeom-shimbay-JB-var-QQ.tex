\documentclass{article}
  \usepackage[utf8]{inputenc}
  \usepackage[T2A]{fontenc}
  \usepackage{array}
  \usepackage[a4paper,
  left=15mm,
  top=15mm,]{geometry}
  \usepackage{setspace}
  
  \renewcommand{\baselinestretch}{1.1} 
  
\begin{document}

\large
\pagenumbering{gobble}


\begin{tabular}{m{17cm}}
\textbf{1-variant}\\
1. Koordinata sistemasın túrlendiriw (birlik vektorlar, kósherler, parallel kóshiriw, koordinata kósherlerin burıw).\\

2. ETIS-tıń ulıwma teńlemesin klassifikatsiyalaw (ETIS-tıń ulıwma teńlemesi, ETIS-tıń ulıwma teńlemesin ápiwaylastırıw, klassifikatsiyalaw).\\

3. Sheńber teńlemesin dúziń: orayı koordinata basında jaylasqan hám radiusı $R=3$ ge teń.\\

4. $\frac{x^{2}}{20} - \frac{y^{2}}{5} = 1$ giperbolasına $4x + 3y - 7 = 0$ tuwrısına perpendikulyar bolǵan urınbanıń teńlemesin dúziń.  \\

5. Fokusı $F(2; - 1)$ noqatında jaylasqan, sáykes direktrisası $x - y - 1 = 0$ teńlemesi menen berilgen parabolanıń teńlemesin dúziń.  
\end{tabular}
\vspace{1cm}


\begin{tabular}{m{17cm}}
\textbf{2-variant}\\
1. Giperbolalıq paraboloydtıń tuwrı sızıqlı jasawshıları (Giperbolalıq paraboloydtı jasawshı tuwrı sızıqlar dástesi).\\

2. Giperbolanıń polyar koordinatadaǵı teńlemesi (Polyar múyeshi, polyar radiusi giperbolanıń polyar teńlemesi).\\

3. Fokusları abscissa kósherinde hám koordinata basına qarata simmetriyalıq jaylasqan giperbolanıń teńlemesin dúziń: direktrisaları arasındaǵı aralıq $228/13$ hám fokusları arasındaǵı aralıq $2 c=26$.\\

4. Koordinata kósherlerin túrlendirmey ETİS ulıwma teńlemesin ápiwaylastırıń, yarım kósherlerin tabıń: $13x^{2} + 18xy + 37y^{2} - 26x - 18y + 3 = 0$.  \\

5. $32x^{2} + 52xy - 9y^{2} + 180 = 0$ ETİS teńlemesin ápiwaylastırıń, tipin anıqlań, qanday geometriyalıq obrazdı anıqlaytuǵının kórsetiń, sızılmasın sızıń.  
\end{tabular}
\vspace{1cm}


\begin{tabular}{m{17cm}}
\textbf{3-variant}\\
1. ETIS-tıń orayın anıqlaw forması (ETIS-tıń ulıwma teńlemesi, orayın anıqlaw forması).\\

2. Ekinshi tártipli betliktiń ulıwma teńlemesi. Orayın anıqlaw formulası.\\

3. Polyar teńlemesi menen berilgen iymek sızıqtıń tipin anıqlań: $\rho=\frac{10}{1-\frac{3}{2}\cos\theta}$.\\

4. $x^{2} - 4y^{2} = 16$ giperbola berilgen. Onıń ekscentrisitetin, fokuslarınıń koordinataların tabıń hám asimptotalarınıń teńlemelerin dúziń.\\

5. $A(\frac{10}{3};\frac{5}{3})$ noqattan $\frac{x^{2}}{20} + \frac{y^{2}}{5} = 1$ ellipsine júrgizilgen urınbalardıń teńlemesin dúziń.  
\end{tabular}
\vspace{1cm}


\begin{tabular}{m{17cm}}
\textbf{4-variant}\\
1. Parabola hám onıń kanonikalıq teńlemesi (anıqlaması, fokusı, direktrisası, kanonikalıq teńlemesi).\\

2. ETIS-tıń ulıwma teńlemesin koordinata kósherlerin burıw arqalı ápiwaylastırıń (ETIS-tıń ulıwma teńlemeleri, koordinata kósherin burıw formulası, teńlemeni kanonik túrge alıp keliw).\\

3. Tipin anıqlań: $4 x^{2}-y^{2}+8 x-2 y+3=0$.\\

4. $y^{2} = 3x$ parabolası menen $\frac{x^{2}}{100} + \frac{y^{2}}{225} = 1$ ellipsiniń kesilisiw noqatların tabıń.  \\

5. $M(2; - \frac{5}{3})$ noqatı $\frac{x^{2}}{9} + \frac{y^{2}}{5} = 1$ ellipsinde jaylasqan. $M$ noqatınıń fokal radiusları jatıwshı tuwrı sızıq teńlemelerin dúziń.  
\end{tabular}
\vspace{1cm}


\begin{tabular}{m{17cm}}
\textbf{5-variant}\\
1. Eki gewekli giperboloid. Kanonikalıq teńlemesi (giperbolanı simmetriya kósheri átirapında aylandırıwdan alınǵan betlik).\\

2. Ellipstiń urınbasınıń teńlemesi (ellips, tuwrı, urınıw tochka, urınba teńlemesi).\\

3. Sheńber teńlemesin dúziń: orayı $C (1;-1) $ noqatında jaylasqan hám $5 x-12 y+9 -0$ tuwrı sızıǵına urınadı .\\

4. $\rho = \frac{6}{1 - cos\theta}$ polyar teńlemesi menen qanday sızıq berilgenin anıqlań.  \\

5. Eger waqıttıń qálegen momentinde $M(x;y)$ noqat $5x - 16 = 0$ tuwrı sızıqqa qaraǵanda $A(5;0)$ noqattan 1,25 márte uzaqlıqta jaylasqan. Usı $M(x;y)$ noqattıń háreketiniń teńlemesin dúziń.  
\end{tabular}
\vspace{1cm}


\begin{tabular}{m{17cm}}
\textbf{6-variant}\\
1. ETIS -tiń ulıwma teńlemesin ápiwaylastırıw (ETIS -tiń ulıwma teńlemesi, koordinata sistemasın túrlendirip ETIS ulıwma teńlemesin ápiwaylastırıw).\\

2. Ellipsoida. Kanonikalıq teńlemesi (ellipsti simmetriya kósheri dogereginde aylandırıwdan alınǵan betlik, kanonikalıq teńlemesi).\\

3. Uchı koordinata basında jaylasqan hám $Oy$ kósherine qarata oń táreptegi yarım tegislikte jaylasqan parabolanıń teńlemesin dúziń: parametri $p=3$.\\

4. $2x + 2y - 3 = 0$ tuwrısına parallel bolıp $\frac{x^{2}}{16} + \frac{y^{2}}{64} = 1$ giperbolasına urınıwshı tuwrınıń teńlemesin dúziń.  \\

5. $4x^{2} - 4xy + y^{2} - 2x - 14y + 7 = 0$ ETİS teńlemesin ápiwayı túrge alıp keliń, tipin anıqlań, qanday geometriyalıq obrazdı anıqlaytuǵının kórsetiń, sızılmasın góne hám taza koordinatalar sistemasına qarata jasań.  
\end{tabular}
\vspace{1cm}


\begin{tabular}{m{17cm}}
\textbf{7-variant}\\
1. Ellips hám onıń kanonikalıq teńlemesi (anıqlaması, fokuslar, ellipstiń kanonikalıq teńlemesi, ekscentrisiteti, direktrisaları).\\

2. ETIS-tıń ulıwma teńlemesin koordinata basın parallel kóshiriw arqalı ápiwayılastırıń (ETIS- tıń ulıwma teńlemesin parallel kóshiriw formulası).\\

3. Ellips teńlemesi berilgen: $\frac{x^2}{25}+\frac{y^2}{16}=1$. Onıń polyar teńlemesin dúziń.\\

4. $41x^{2} + 24xy + 9y^{2} + 24x + 18y - 36 = 0$ ETİS tipin anıqlań hám orayların tabıń koordinata kósherlerin túrlendirmey qanday sızıqtı anıqlaytuǵının kórsetiń yarım kósherlerin tabıń.  \\

5. $\frac{x^{2}}{3} - \frac{y^{2}}{5} = 1$ giperbolasına $P(1; - 5)$ noqatında júrgizilgen urınbalardıń teńlemesin dúziń.
\end{tabular}
\vspace{1cm}


\begin{tabular}{m{17cm}}
\textbf{8-variant}\\
1. Ellipslik paraboloid (parabola, kósher, ellipslik paraboloid).\\

2. Giperbolanıń urınbasınıń teńlemesi (giperbolaǵa berilgen noqatta júrgizilgen urınba teńlemesi).\\

3. Tipin anıqlań: $x^{2}-4 xy+4 y^{2}+7 x-12=0$.\\

4. Ellips $3x^{2} + 4y^{2} - 12 = 0$ teńlemesi menen berilgen. Onıń kósherleriniń uzınlıqların, fokuslarınıń koordinataların hám ekscentrisitetin tabıń.  \\

5. $y^{2} = 20x$ parabolasınıń $M$ noqatın tabıń, eger onıń abscissası 7 ge teń bolsa, fokal radiusın hám fokal radius jaylasqan tuwrını anıqlań.
\end{tabular}
\vspace{1cm}


\begin{tabular}{m{17cm}}
\textbf{9-variant}\\
1. ETIS-tıń invariantları (ETIS-tıń ulıwma teńlemesi, túrlendiriw, ETIS invariantları ).\\

2. Ekinshi tártipli aylanba betlikler (koordinata sisteması, tegislik, vektor iymek sızıq, aylanba betlik).\\

3. Sheńberdiń $C$ orayı hám $R$ radiusın tabıń: $x^2+y^2+6 x-4 y+14=0$.\\

4. $3x + 4y - 12 = 0$ tuwrı sızıǵı hám $y^{2} = - 9x$ parabolasınıń kesilisiw noqatların tabıń.  \\

5. Fokuslari $F(3;4), F(-3;-4)$ noqatlarında jaylasqan direktrisaları orasıdaǵı aralıq 3,6 ǵa teń bolǵan giperbolanıń teńlemesin dúziń.  
\end{tabular}
\vspace{1cm}


\begin{tabular}{m{17cm}}
\textbf{10-variant}\\
1. Parabolanıń polyar koordinatalardaǵı teńlemesi (polyar koordinata sistemasında parabolanıń teńlemesi).\\

2. Bir gewekli giperboloid. Kanonikalıq teńlemesi (giperbolanı simmetriya kósheri átirapında aylandırıwdan alınǵan betlik).\\

3. Uchı koordinata basında jaylasqan hám $Ox$ kósherine qarata tómengi yarım tegislikte jaylasqan parabolanıń teńlemesin dúziń: parametri $p=3$.\\

4. $\rho = \frac{5}{3 - 4cos\theta}$ teńlemesi menen qanday sızıq berilgenin hám yarım kósherlerin tabıń.  \\

5. $14x^{2} + 24xy + 21y^{2} - 4x + 18y - 139 = 0$ iymek sızıǵınıń tipin anıqlań, eger oraylı iymek sızıq bolsa orayınıń koordinataların tabıń.  
\end{tabular}
\vspace{1cm}


\begin{tabular}{m{17cm}}
\textbf{11-variant}\\
1. Ellipstiń polyar koordinatalardaǵı teńlemesi (polyar koordinatalar sistemasında ellipstiń teńlemesi).\\

2. Cilindrlik betlikler (jasawshı tuwrı sızıq, baǵıtlawshı iymek sızıq, cilindrlik betlik).\\

3. Polyar teńlemesi menen berilgen iymek sızıqtıń tipin anıqlań: $\rho=\frac{5}{3-4\cos\theta}$.\\

4. $y^{2} = 12x$ paraborolasına $3x - 2y + 30 = 0$ tuwrı sızıǵına parallel bolǵan urınbanıń teńlemesin dúziń.  \\

5. $\frac{x^{2}}{2} + \frac{y^{2}}{3} = 1$, ellipsin $x + y - 2 = 0$ noqatınan júrgizilgen urınbalarınıń teńlemesin dúziń.  
\end{tabular}
\vspace{1cm}


\begin{tabular}{m{17cm}}
\textbf{12-variant}\\
1. Giperbola. Kanonikalıq teńlemesi (fokuslar, kósherler, direktrisalar, giperbola, ekscentrisitet, kanonikalıq teńlemesi).\\

2. Betliktiń kanonikalıq teńlemeleri. Betlik haqqında túsinik. (Betliktiń anıqlaması, formulaları, kósher, baǵıtlawshı tuwrılar).\\

3. Tipin anıqlań: $4 x^2+9 y^2-40 x+36 y+100=0$.\\

4. ETİS-tıń ulıwma teńlemesin koordinata sistemasın túrlendirmey ápiwaylastırıń, tipin anıqlań, obrazı qanday sızıqtı anıqlaytuǵının kórsetiń. $7x^{2} - 8xy + y^{2} - 16x - 2y - 51 = 0$  \\

5. $\frac{x^{2}}{100} + \frac{y^{2}}{36} = 1$ ellipsiniń oń jaqtaǵı fokusınan 14 ge teń aralıqta bolǵan noqattı tabıń.  
\end{tabular}
\vspace{1cm}


\begin{tabular}{m{17cm}}
\textbf{13-variant}\\
1. Parabolanıń urınbasınıń teńlemesi (parabola, tuwrı, urınıw noqatı, urınba teńlemesi).\\

2. Betlik haqqında túsinik (tuwrı, iymek sızıq, betliktiń anıqlamaları hám formulaları).\\

3. Sheńberdiń $C$ orayı hám $R$ radiusın tabıń: $x^2+y^2-2 x+4 y-20=0$.\\

4. $3x + 10y - 25 = 0$ tuwrı menen $\frac{x^{2}}{25} + \frac{y^{2}}{4} = 1$ ellipstiń kesilisiw noqatların tabıń.\\

5. Giperbolanıń ekscentrisiteti $\varepsilon = \frac{13}{12}$, fokusı $F(0;13)$ noqatında hám sáykes direktrisası $13y - 144 = 0$ teńlemesi menen berilgen bolsa, giperbolanıń teńlemesin dúziń.  
\end{tabular}
\vspace{1cm}


\begin{tabular}{m{17cm}}
\textbf{14-variant}\\
1. Koordinata sistemasın túrlendiriw (birlik vektorlar, kósherler, parallel kóshiriw, koordinata kósherlerin burıw).\\

2. ETIS-tıń ulıwma teńlemesin klassifikatsiyalaw (ETIS-tıń ulıwma teńlemesi, ETIS-tıń ulıwma teńlemesin ápiwaylastırıw, klassifikatsiyalaw).\\

3. Fokusları abscissa kósherinde hám koordinata basına qarata simmetriyalıq jaylasqan ellipstiń teńlemesin dúziń: fokusları arasındaǵı aralıq $2 c=6$ hám ekscentrisitet $\varepsilon=3/5$.\\

4. $\rho = \frac{10}{2 - cos\theta}$ polyar teńlemesi menen qanday sızıq berilgenin anıqlań.  \\

5. $2x^{2} + 3y^{2} + 8x - 6y + 11 = 0$ teńlemesin ápiwaylastırıń qanday geometriyalıq obrazdı anıqlaytuǵının tabıń hám grafigin jasań.  
\end{tabular}
\vspace{1cm}


\begin{tabular}{m{17cm}}
\textbf{15-variant}\\
1. Giperbolalıq paraboloydtıń tuwrı sızıqlı jasawshıları (Giperbolalıq paraboloydtı jasawshı tuwrı sızıqlar dástesi).\\

2. Giperbolanıń polyar koordinatadaǵı teńlemesi (Polyar múyeshi, polyar radiusi giperbolanıń polyar teńlemesi).\\

3. Polyar teńlemesi menen berilgen iymek sızıqtıń tipin anıqlań: $\rho=\frac{12}{2-\cos\theta}$.\\

4. $2x + 2y - 3 = 0$ tuwrısına perpendikulyar bolıp $x^{2} = 16y$ parabolasına urınıwshı tuwrınıń teńlemesin dúziń.  \\

5. $\frac{x^{2}}{3} - \frac{y^{2}}{5} = 1$, giperbolasına $P(4;2)$ noqatınan júrgizilgen urınbalardıń teńlemesin dúziń.  
\end{tabular}
\vspace{1cm}


\begin{tabular}{m{17cm}}
\textbf{16-variant}\\
1. ETIS-tıń orayın anıqlaw forması (ETIS-tıń ulıwma teńlemesi, orayın anıqlaw forması).\\

2. Ekinshi tártipli betliktiń ulıwma teńlemesi. Orayın anıqlaw formulası.\\

3. Berilgen sızıqlardıń oraylıq ekenligin kórsetiń hám orayın tabıń: $9 x^{2}-4 xy-7 y^{2}-12=0$.\\

4. Koordinata kósherlerin túrlendirmey ETİS teńlemesin ápiwaylastırıń, yarım kósherlerin tabıń $41x^{2} + 2xy + 9y^{2} - 26x - 18y + 3 = 0$.  \\

5. $y^{2} = 20x$ parabolasınıń abscissası 7 ge teń bolǵan $M$ noqatınıń fokal radiusın tabıń hám fokal radiusı jatqan tuwrınıń teńlemesin dúziń.  
\end{tabular}
\vspace{1cm}


\begin{tabular}{m{17cm}}
\textbf{17-variant}\\
1. Parabola hám onıń kanonikalıq teńlemesi (anıqlaması, fokusı, direktrisası, kanonikalıq teńlemesi).\\

2. ETIS-tıń ulıwma teńlemesin koordinata kósherlerin burıw arqalı ápiwaylastırıń (ETIS-tıń ulıwma teńlemeleri, koordinata kósherin burıw formulası, teńlemeni kanonik túrge alıp keliw).\\

3. Sheńber teńlemesin dúziń: sheńber diametriniń ushları $A (3;2) $ hám $B (-1;6 ) $ noqatlarında jaylasqan.\\

4. $\rho = \frac{144}{13 - 5cos\theta}$ ellipsti anıqlaytuǵının kórsetiń hám onıń yarım kósherlerin anıqlań.\\

5. Fokusı $F( - 1; - 4)$noqatında bolǵan, sáykes direktrissası $x - 2 = 0$ teńlemesi menen berilgen $A( - 3; - 5)$ noqatınan ótiwshi ellipstiń teńlemesin dúziń.  
\end{tabular}
\vspace{1cm}


\begin{tabular}{m{17cm}}
\textbf{18-variant}\\
1. Eki gewekli giperboloid. Kanonikalıq teńlemesi (giperbolanı simmetriya kósheri átirapında aylandırıwdan alınǵan betlik).\\

2. Ellipstiń urınbasınıń teńlemesi (ellips, tuwrı, urınıw tochka, urınba teńlemesi).\\

3. Fokusları abscissa kósherinde hám koordinata basına qarata simmetriyalıq jaylasqan ellipstiń teńlemesin dúziń: kishi kósheri $10$, ekscentrisitet $\varepsilon=12/13$.\\

4. $x^{2} + 4y^{2} = 25$ ellipsi menen $4x - 2y + 23 = 0$ tuwrı sızıǵına parallel bolǵan urınba tuwrı sızıqtıń teńlemesin dúziń.  \\

5. $2x^{2} + 10xy + 12y^{2} - 7x + 18y - 15 = 0$ ETİS teńlemesin ápiwayı túrge alıp keliń, tipin anıqlań, qanday geometriyalıq obrazdı anıqlaytuǵının kórsetiń, sızılmasın góne hám taza koordinatalar sistemasına qarata jasań  
\end{tabular}
\vspace{1cm}


\begin{tabular}{m{17cm}}
\textbf{19-variant}\\
1. ETIS -tiń ulıwma teńlemesin ápiwaylastırıw (ETIS -tiń ulıwma teńlemesi, koordinata sistemasın túrlendirip ETIS ulıwma teńlemesin ápiwaylastırıw).\\

2. Ellipsoida. Kanonikalıq teńlemesi (ellipsti simmetriya kósheri dogereginde aylandırıwdan alınǵan betlik, kanonikalıq teńlemesi).\\

3. Giperbola teńlemesi berilgen: $\frac{x^{2}}{16}-\frac{y^{2}}{9}=1$. Onıń polyar teńlemesin dúziń.\\

4. Koordinata kósherlerin túrlendirmey ETİS teńlemesin ápiwaylastırıń, qanday geometriyalıq obrazdı anıqlaytuǵının kórsetiń $4x^{2} - 4xy + y^{2} + 4x - 2y + 1 = 0$.  \\

5. $\frac{x^{2}}{25} + \frac{y^{2}}{16} = 1$, ellipsine $C(10; - 8)$ noqatınan júrgizilgen urınbalarınıń teńlemesin dúziń.  
\end{tabular}
\vspace{1cm}


\begin{tabular}{m{17cm}}
\textbf{20-variant}\\
1. Ellips hám onıń kanonikalıq teńlemesi (anıqlaması, fokuslar, ellipstiń kanonikalıq teńlemesi, ekscentrisiteti, direktrisaları).\\

2. ETIS-tıń ulıwma teńlemesin koordinata basın parallel kóshiriw arqalı ápiwayılastırıń (ETIS- tıń ulıwma teńlemesin parallel kóshiriw formulası).\\

3. Berilgen sızıqlardıń oraylıq ekenligin kórsetiń hám orayın tabıń: $3 x^{2}+5 xy+y^{2}-8 x-11 y-7=0$.\\

4. $\frac{x^{2}}{4} - \frac{y^{2}}{5} = 1$ giperbolaǵa $3x - 2y = 0$ tuwrısına parallel bolǵan urınbanıń teńlemesin dúziń.  \\

5. Eger qálegen waqıt momentinde $M(x;y)$ noqat $A(8;4)$ noqattan hám ordinata kósherinen birdey aralıqta jaylassa, $M(x;y)$ noqatınıń háreket etiw troektoriyasınıń teńlemesin dúziń.  
\end{tabular}
\vspace{1cm}


\begin{tabular}{m{17cm}}
\textbf{21-variant}\\
1. Ellipslik paraboloid (parabola, kósher, ellipslik paraboloid).\\

2. Giperbolanıń urınbasınıń teńlemesi (giperbolaǵa berilgen noqatta júrgizilgen urınba teńlemesi).\\

3. Sheńber teńlemesin dúziń: sheńber $A (2;6 ) $ noqatınan ótedi hám orayı $C (-1;2) $ noqatında jaylasqan .\\

4. Koordinata kósherlerin túrlendirmey ETİS teńlemesin ápiwaylastırıń, yarım kósherlerin tabıń $4x^{2} - 4xy + 9y^{2} - 26x - 18y + 3 = 0$.\\

5. $16x^{2} - 9y^{2} - 64x - 54y - 161 = 0$ teńlemesi giperbolanıń teńlemesi ekenin anıqlań hám onıń orayı $C$, yarım kósherleri, ekscentrisitetin, asimptotalarınıń teńlemelerin dúziń.  
\end{tabular}
\vspace{1cm}


\begin{tabular}{m{17cm}}
\textbf{22-variant}\\
1. ETIS-tıń invariantları (ETIS-tıń ulıwma teńlemesi, túrlendiriw, ETIS invariantları ).\\

2. Ekinshi tártipli aylanba betlikler (koordinata sisteması, tegislik, vektor iymek sızıq, aylanba betlik).\\

3. Fokusları abscissa kósherinde hám koordinata basına qarata simmetriyalıq jaylasqan giperbolanıń teńlemesin dúziń: direktrisaları arasındaǵı aralıq $32/5$ hám kósheri $2 b=6$.\\

4. $\frac{x^{2}}{4} - \frac{y^{2}}{5} = 1$ giperbolasına $3x + 2y = 0$ tuwrı sızıǵına perpendikulyar bolǵan urınba tuwrınıń teńlemesin dúziń.\\

5. Tóbesi $A(-4;0)$ noqatında, al, direktrisası $y - 2 = 0$ tuwrı sızıq bolǵan parabolanıń teńlemesin dúziń.
\end{tabular}
\vspace{1cm}


\begin{tabular}{m{17cm}}
\textbf{23-variant}\\
1. Parabolanıń polyar koordinatalardaǵı teńlemesi (polyar koordinata sistemasında parabolanıń teńlemesi).\\

2. Bir gewekli giperboloid. Kanonikalıq teńlemesi (giperbolanı simmetriya kósheri átirapında aylandırıwdan alınǵan betlik).\\

3. Polyar teńlemesi menen berilgen iymek sızıqtıń tipin anıqlań: $\rho=\frac{5}{1-\frac{1}{2}\cos\theta}$.\\

4. $\frac{x^{2}}{4} - \frac{y^{2}}{5} = 1$, giperbolanıń $3x - 2y = 0$ tuwrı sızıǵına parallel bolǵan urınbasınıń teńlemesin dúziń.  \\

5. $4x^{2} + 24xy + 11y^{2} + 64x + 42y + 51 = 0$ iymek sızıǵınıń tipin anıqlań eger orayı bar bolsa, onıń orayınıń koordinataların tabıń hám koordinata basın orayǵa parallel kóshiriw ámelin orınlań.  
\end{tabular}
\vspace{1cm}


\begin{tabular}{m{17cm}}
\textbf{24-variant}\\
1. Ellipstiń polyar koordinatalardaǵı teńlemesi (polyar koordinatalar sistemasında ellipstiń teńlemesi).\\

2. Cilindrlik betlikler (jasawshı tuwrı sızıq, baǵıtlawshı iymek sızıq, cilindrlik betlik).\\

3. Tipin anıqlań: $2 x^{2}+3 y^{2}+8 x-6 y+11=0$.\\

4. $\frac{x^{2}}{16} - \frac{y^{2}}{64} = 1$, giperbolasına berilgen $10x - 3y + 9 = 0$ tuwrı sızıǵına parallel bolǵan urınbanıń teńlemesin dúziń.  \\

5. Fokusı $F(7;2)$ noqatında jaylasqan, sáykes direktrisası $x - 5 = 0$ teńlemesi menen berilgen parabolanıń teńlemesin dúziń.  
\end{tabular}
\vspace{1cm}


\begin{tabular}{m{17cm}}
\textbf{25-variant}\\
1. Giperbola. Kanonikalıq teńlemesi (fokuslar, kósherler, direktrisalar, giperbola, ekscentrisitet, kanonikalıq teńlemesi).\\

2. Betliktiń kanonikalıq teńlemeleri. Betlik haqqında túsinik. (Betliktiń anıqlaması, formulaları, kósher, baǵıtlawshı tuwrılar).\\

3. Sheńber teńlemesin dúziń: orayı $C (2;-3) $ noqatında jaylasqan hám radiusı $R=7$ ge teń.\\

4. $x^{2} - 4y^{2} = 16$ giperbola berilgen. Onıń ekscentrisitetin, fokuslarınıń koordinataların tabıń hám asimptotalarınıń teńlemelerin dúziń.\\

5. $2x^{2} + 3y^{2} + 8x - 6y + 11 = 0$ teńlemesin ápiwaylastırıń qanday geometriyalıq obrazdı anıqlaytuǵının tabıń hám grafigin jasań.
\end{tabular}
\vspace{1cm}


\begin{tabular}{m{17cm}}
\textbf{26-variant}\\
1. Parabolanıń urınbasınıń teńlemesi (parabola, tuwrı, urınıw noqatı, urınba teńlemesi).\\

2. Betlik haqqında túsinik (tuwrı, iymek sızıq, betliktiń anıqlamaları hám formulaları).\\

3. Fokusları abscissa kósherinde hám koordinata basına qarata simmetriyalıq jaylasqan giperbolanıń teńlemesin dúziń: oqları $2 a=10$ hám $2 b=8$.\\

4. $x^{2} - y^{2} = 27$ giperbolasına $4x + 2y - 7 = 0$ tuwrısına parallel bolǵan urınbanıń teńlemesin tabıń.  \\

5. Fokusı $F( - 1; - 4)$ noqatında jaylasqan, sáykes direktrisası $x - 2 = 0$ teńlemesi menen berilgen, $A( - 3; - 5)$ noqatınan ótiwshi ellipstiń teńlemesin dúziń.  
\end{tabular}
\vspace{1cm}


\begin{tabular}{m{17cm}}
\textbf{27-variant}\\
1. Koordinata sistemasın túrlendiriw (birlik vektorlar, kósherler, parallel kóshiriw, koordinata kósherlerin burıw).\\

2. ETIS-tıń ulıwma teńlemesin klassifikatsiyalaw (ETIS-tıń ulıwma teńlemesi, ETIS-tıń ulıwma teńlemesin ápiwaylastırıw, klassifikatsiyalaw).\\

3. Polyar teńlemesi menen berilgen iymek sızıqtıń tipin anıqlań: $\rho=\frac{6}{1-\cos 0}$.\\

4. Koordinata kósherlerin túrlendirmey ETİS ulıwma teńlemesin ápiwaylastırıń, yarım kósherlerin tabıń: $13x^{2} + 18xy + 37y^{2} - 26x - 18y + 3 = 0$.  \\

5. $32x^{2} + 52xy - 7y^{2} + 180 = 0$ ETİS teńlemesin ápiwayı túrge alıp keliń, tipin anıqlań, qanday geometriyalıq obrazdı anıqlaytuǵının kórsetiń, sızılmasın góne hám taza koordinatalar sistemasına qarata jasań.  
\end{tabular}
\vspace{1cm}


\begin{tabular}{m{17cm}}
\textbf{28-variant}\\
1. Giperbolalıq paraboloydtıń tuwrı sızıqlı jasawshıları (Giperbolalıq paraboloydtı jasawshı tuwrı sızıqlar dástesi).\\

2. Giperbolanıń polyar koordinatadaǵı teńlemesi (Polyar múyeshi, polyar radiusi giperbolanıń polyar teńlemesi).\\

3. Tipin anıqlań: $9 x^{2}+4 y^{2}+18 x-8 y+49=0$.\\

4. Ellips $3x^{2} + 4y^{2} - 12 = 0$ teńlemesi menen berilgen. Onıń kósherleriniń uzınlıqların, fokuslarınıń koordinataların hám ekscentrisitetin tabıń.  \\

5. Úlken kósheri 26 ǵa, fokusları $F( - 10;0)$, $F(14;0)$ noqatlarında jaylasqan ellipstiń teńlemesin dúziń.  
\end{tabular}
\vspace{1cm}


\begin{tabular}{m{17cm}}
\textbf{29-variant}\\
1. ETIS-tıń orayın anıqlaw forması (ETIS-tıń ulıwma teńlemesi, orayın anıqlaw forması).\\

2. Ekinshi tártipli betliktiń ulıwma teńlemesi. Orayın anıqlaw formulası.\\

3. Sheńber teńlemesin dúziń: orayı koordinata basında jaylasqan hám $3 x-4 y+20=0$ tuwrı sızıǵına urınadı.\\

4. $y^{2} = 3x$ parabolası menen $\frac{x^{2}}{100} + \frac{y^{2}}{225} = 1$ ellipsiniń kesilisiw noqatların tabıń.  \\

5. $4x^{2} - 4xy + y^{2} - 6x + 8y + 13 = 0$ ETİS-ǵı orayǵa iyeme? Orayǵa iye bolsa orayın anıqlań: jalǵız orayǵa iyeme-?, sheksiz orayǵa iyeme-?  
\end{tabular}
\vspace{1cm}


\begin{tabular}{m{17cm}}
\textbf{30-variant}\\
1. Parabola hám onıń kanonikalıq teńlemesi (anıqlaması, fokusı, direktrisası, kanonikalıq teńlemesi).\\

2. ETIS-tıń ulıwma teńlemesin koordinata kósherlerin burıw arqalı ápiwaylastırıń (ETIS-tıń ulıwma teńlemeleri, koordinata kósherin burıw formulası, teńlemeni kanonik túrge alıp keliw).\\

3. Fokusları abscissa kósherinde hám koordinata basına qarata simmetriyalıq jaylasqan ellipstiń teńlemesin dúziń: direktrisaları arasındaǵı aralıq $5$ hám fokusları arasındaǵı aralıq $2 c=4$.\\

4. $\rho = \frac{6}{1 - cos\theta}$ polyar teńlemesi menen qanday sızıq berilgenin anıqlań.  \\

5. $2x^{2} + 3y^{2} + 8x - 6y + 11 = 0$ teńlemesi menen qanday tiptegi sızıq berilgenin anıqlań hám onıń teńlemesin ápiwaylastırıń hám grafigin jasań.  
\end{tabular}
\vspace{1cm}


\begin{tabular}{m{17cm}}
\textbf{31-variant}\\
1. Eki gewekli giperboloid. Kanonikalıq teńlemesi (giperbolanı simmetriya kósheri átirapında aylandırıwdan alınǵan betlik).\\

2. Ellipstiń urınbasınıń teńlemesi (ellips, tuwrı, urınıw tochka, urınba teńlemesi).\\

3. Parabola teńlemesi berilgen: $y^2=6 x$. Onıń polyar teńlemesin dúziń.\\

4. $\frac{x^{2}}{20} - \frac{y^{2}}{5} = 1$ giperbolasına $4x + 3y - 7 = 0$ tuwrısına perpendikulyar bolǵan urınbanıń teńlemesin dúziń.  \\

5. Fokusı $F(2; - 1)$ noqatında jaylasqan, sáykes direktrisası $x - y - 1 = 0$ teńlemesi menen berilgen parabolanıń teńlemesin dúziń.  
\end{tabular}
\vspace{1cm}


\begin{tabular}{m{17cm}}
\textbf{32-variant}\\
1. ETIS -tiń ulıwma teńlemesin ápiwaylastırıw (ETIS -tiń ulıwma teńlemesi, koordinata sistemasın túrlendirip ETIS ulıwma teńlemesin ápiwaylastırıw).\\

2. Ellipsoida. Kanonikalıq teńlemesi (ellipsti simmetriya kósheri dogereginde aylandırıwdan alınǵan betlik, kanonikalıq teńlemesi).\\

3. Tipin anıqlań: $9 x^{2}-16 y^{2}-54 x-64 y-127=0$.\\

4. $41x^{2} + 24xy + 9y^{2} + 24x + 18y - 36 = 0$ ETİS tipin anıqlań hám orayların tabıń koordinata kósherlerin túrlendirmey qanday sızıqtı anıqlaytuǵının kórsetiń yarım kósherlerin tabıń.  \\

5. $32x^{2} + 52xy - 9y^{2} + 180 = 0$ ETİS teńlemesin ápiwaylastırıń, tipin anıqlań, qanday geometriyalıq obrazdı anıqlaytuǵının kórsetiń, sızılmasın sızıń.  
\end{tabular}
\vspace{1cm}


\begin{tabular}{m{17cm}}
\textbf{33-variant}\\
1. Ellips hám onıń kanonikalıq teńlemesi (anıqlaması, fokuslar, ellipstiń kanonikalıq teńlemesi, ekscentrisiteti, direktrisaları).\\

2. ETIS-tıń ulıwma teńlemesin koordinata basın parallel kóshiriw arqalı ápiwayılastırıń (ETIS- tıń ulıwma teńlemesin parallel kóshiriw formulası).\\

3. Sheńberdiń $C$ orayı hám $R$ radiusın tabıń: $x^2+y^2-2 x+4 y-14=0$.\\

4. $x^{2} - 4y^{2} = 16$ giperbola berilgen. Onıń ekscentrisitetin, fokuslarınıń koordinataların tabıń hám asimptotalarınıń teńlemelerin dúziń.\\

5. $A(\frac{10}{3};\frac{5}{3})$ noqattan $\frac{x^{2}}{20} + \frac{y^{2}}{5} = 1$ ellipsine júrgizilgen urınbalardıń teńlemesin dúziń.  
\end{tabular}
\vspace{1cm}


\begin{tabular}{m{17cm}}
\textbf{34-variant}\\
1. Ellipslik paraboloid (parabola, kósher, ellipslik paraboloid).\\

2. Giperbolanıń urınbasınıń teńlemesi (giperbolaǵa berilgen noqatta júrgizilgen urınba teńlemesi).\\

3. Fokusları abscissa kósherinde hám koordinata basına qarata simmetriyalıq jaylasqan giperbolanıń teńlemesin dúziń: direktrisaları arasındaǵı aralıq $8/3$ hám ekscentrisitet $\varepsilon=3/2$.\\

4. $3x + 4y - 12 = 0$ tuwrı sızıǵı hám $y^{2} = - 9x$ parabolasınıń kesilisiw noqatların tabıń.  \\

5. $M(2; - \frac{5}{3})$ noqatı $\frac{x^{2}}{9} + \frac{y^{2}}{5} = 1$ ellipsinde jaylasqan. $M$ noqatınıń fokal radiusları jatıwshı tuwrı sızıq teńlemelerin dúziń.  
\end{tabular}
\vspace{1cm}


\begin{tabular}{m{17cm}}
\textbf{35-variant}\\
1. ETIS-tıń invariantları (ETIS-tıń ulıwma teńlemesi, túrlendiriw, ETIS invariantları ).\\

2. Ekinshi tártipli aylanba betlikler (koordinata sisteması, tegislik, vektor iymek sızıq, aylanba betlik).\\

3. Polyar teńlemesi menen berilgen iymek sızıqtıń tipin anıqlań: $\rho=\frac{1}{3-3\cos\theta}$.\\

4. $\rho = \frac{5}{3 - 4cos\theta}$ teńlemesi menen qanday sızıq berilgenin hám yarım kósherlerin tabıń.  \\

5. Eger waqıttıń qálegen momentinde $M(x;y)$ noqat $5x - 16 = 0$ tuwrı sızıqqa qaraǵanda $A(5;0)$ noqattan 1,25 márte uzaqlıqta jaylasqan. Usı $M(x;y)$ noqattıń háreketiniń teńlemesin dúziń.  
\end{tabular}
\vspace{1cm}


\begin{tabular}{m{17cm}}
\textbf{36-variant}\\
1. Parabolanıń polyar koordinatalardaǵı teńlemesi (polyar koordinata sistemasında parabolanıń teńlemesi).\\

2. Bir gewekli giperboloid. Kanonikalıq teńlemesi (giperbolanı simmetriya kósheri átirapında aylandırıwdan alınǵan betlik).\\

3. Tipin anıqlań: $3 x^{2}-8 xy+7 y^{2}+8 x-15 y+20=0$.\\

4. $2x + 2y - 3 = 0$ tuwrısına parallel bolıp $\frac{x^{2}}{16} + \frac{y^{2}}{64} = 1$ giperbolasına urınıwshı tuwrınıń teńlemesin dúziń.  \\

5. $4x^{2} - 4xy + y^{2} - 2x - 14y + 7 = 0$ ETİS teńlemesin ápiwayı túrge alıp keliń, tipin anıqlań, qanday geometriyalıq obrazdı anıqlaytuǵının kórsetiń, sızılmasın góne hám taza koordinatalar sistemasına qarata jasań.  
\end{tabular}
\vspace{1cm}


\begin{tabular}{m{17cm}}
\textbf{37-variant}\\
1. Ellipstiń polyar koordinatalardaǵı teńlemesi (polyar koordinatalar sistemasında ellipstiń teńlemesi).\\

2. Cilindrlik betlikler (jasawshı tuwrı sızıq, baǵıtlawshı iymek sızıq, cilindrlik betlik).\\

3. Sheńber teńlemesin dúziń: orayı $C (6 ;-8) $ noqatında jaylasqan hám koordinata basınan ótedi.\\

4. ETİS-tıń ulıwma teńlemesin koordinata sistemasın túrlendirmey ápiwaylastırıń, tipin anıqlań, obrazı qanday sızıqtı anıqlaytuǵının kórsetiń. $7x^{2} - 8xy + y^{2} - 16x - 2y - 51 = 0$  \\

5. $\frac{x^{2}}{3} - \frac{y^{2}}{5} = 1$ giperbolasına $P(1; - 5)$ noqatında júrgizilgen urınbalardıń teńlemesin dúziń.
\end{tabular}
\vspace{1cm}


\begin{tabular}{m{17cm}}
\textbf{38-variant}\\
1. Giperbola. Kanonikalıq teńlemesi (fokuslar, kósherler, direktrisalar, giperbola, ekscentrisitet, kanonikalıq teńlemesi).\\

2. Betliktiń kanonikalıq teńlemeleri. Betlik haqqında túsinik. (Betliktiń anıqlaması, formulaları, kósher, baǵıtlawshı tuwrılar).\\

3. Uchı koordinata basında jaylasqan hám $Oy$ kósherine qarata shep táreptegi yarım tegislikte jaylasqan parabolanıń teńlemesin dúziń: parametri $p=0,5$.\\

4. $3x + 10y - 25 = 0$ tuwrı menen $\frac{x^{2}}{25} + \frac{y^{2}}{4} = 1$ ellipstiń kesilisiw noqatların tabıń.\\

5. $y^{2} = 20x$ parabolasınıń $M$ noqatın tabıń, eger onıń abscissası 7 ge teń bolsa, fokal radiusın hám fokal radius jaylasqan tuwrını anıqlań.
\end{tabular}
\vspace{1cm}


\begin{tabular}{m{17cm}}
\textbf{39-variant}\\
1. Parabolanıń urınbasınıń teńlemesi (parabola, tuwrı, urınıw noqatı, urınba teńlemesi).\\

2. Betlik haqqında túsinik (tuwrı, iymek sızıq, betliktiń anıqlamaları hám formulaları).\\

3. Giperbola teńlemesi berilgen: $\frac{x^{2}}{25}-\frac{y^{2}}{144}=1$. Onıń polyar teńlemesin dúziń.\\

4. $\rho = \frac{10}{2 - cos\theta}$ polyar teńlemesi menen qanday sızıq berilgenin anıqlań.  \\

5. Fokuslari $F(3;4), F(-3;-4)$ noqatlarında jaylasqan direktrisaları orasıdaǵı aralıq 3,6 ǵa teń bolǵan giperbolanıń teńlemesin dúziń.  
\end{tabular}
\vspace{1cm}


\begin{tabular}{m{17cm}}
\textbf{40-variant}\\
1. Koordinata sistemasın túrlendiriw (birlik vektorlar, kósherler, parallel kóshiriw, koordinata kósherlerin burıw).\\

2. ETIS-tıń ulıwma teńlemesin klassifikatsiyalaw (ETIS-tıń ulıwma teńlemesi, ETIS-tıń ulıwma teńlemesin ápiwaylastırıw, klassifikatsiyalaw).\\

3. Tipin anıqlań: $5 x^{2}+14 xy+11 y^{2}+12 x-7 y+19=0$.\\

4. $y^{2} = 12x$ paraborolasına $3x - 2y + 30 = 0$ tuwrı sızıǵına parallel bolǵan urınbanıń teńlemesin dúziń.  \\

5. $14x^{2} + 24xy + 21y^{2} - 4x + 18y - 139 = 0$ iymek sızıǵınıń tipin anıqlań, eger oraylı iymek sızıq bolsa orayınıń koordinataların tabıń.  
\end{tabular}
\vspace{1cm}


\begin{tabular}{m{17cm}}
\textbf{41-variant}\\
1. Giperbolalıq paraboloydtıń tuwrı sızıqlı jasawshıları (Giperbolalıq paraboloydtı jasawshı tuwrı sızıqlar dástesi).\\

2. Giperbolanıń polyar koordinatadaǵı teńlemesi (Polyar múyeshi, polyar radiusi giperbolanıń polyar teńlemesi).\\

3. Sheńber teńlemesin dúziń: $A (3;1) $ hám $B (-1;3) $ noqatlardan ótedi, orayı $3 x-y-2=0$ tuwrı sızıǵında jaylasqan .\\

4. Koordinata kósherlerin túrlendirmey ETİS teńlemesin ápiwaylastırıń, yarım kósherlerin tabıń $41x^{2} + 2xy + 9y^{2} - 26x - 18y + 3 = 0$.  \\

5. $\frac{x^{2}}{2} + \frac{y^{2}}{3} = 1$, ellipsin $x + y - 2 = 0$ noqatınan júrgizilgen urınbalarınıń teńlemesin dúziń.  
\end{tabular}
\vspace{1cm}


\begin{tabular}{m{17cm}}
\textbf{42-variant}\\
1. ETIS-tıń orayın anıqlaw forması (ETIS-tıń ulıwma teńlemesi, orayın anıqlaw forması).\\

2. Ekinshi tártipli betliktiń ulıwma teńlemesi. Orayın anıqlaw formulası.\\

3. Uchı koordinata basında jaylasqan hám $Ox$ kósherine qarata joqarı yarım tegislikte jaylasqan parabolanıń teńlemesin dúziń: parametri $p=1/4$.\\

4. $\rho = \frac{144}{13 - 5cos\theta}$ ellipsti anıqlaytuǵının kórsetiń hám onıń yarım kósherlerin anıqlań.\\

5. $\frac{x^{2}}{100} + \frac{y^{2}}{36} = 1$ ellipsiniń oń jaqtaǵı fokusınan 14 ge teń aralıqta bolǵan noqattı tabıń.  
\end{tabular}
\vspace{1cm}


\begin{tabular}{m{17cm}}
\textbf{43-variant}\\
1. Parabola hám onıń kanonikalıq teńlemesi (anıqlaması, fokusı, direktrisası, kanonikalıq teńlemesi).\\

2. ETIS-tıń ulıwma teńlemesin koordinata kósherlerin burıw arqalı ápiwaylastırıń (ETIS-tıń ulıwma teńlemeleri, koordinata kósherin burıw formulası, teńlemeni kanonik túrge alıp keliw).\\

3. Tipin anıqlań: $3 x^{2}-2 xy-3 y^{2}+12 y-15=0$.\\

4. $2x + 2y - 3 = 0$ tuwrısına perpendikulyar bolıp $x^{2} = 16y$ parabolasına urınıwshı tuwrınıń teńlemesin dúziń.  \\

5. Giperbolanıń ekscentrisiteti $\varepsilon = \frac{13}{12}$, fokusı $F(0;13)$ noqatında hám sáykes direktrisası $13y - 144 = 0$ teńlemesi menen berilgen bolsa, giperbolanıń teńlemesin dúziń.  
\end{tabular}
\vspace{1cm}


\begin{tabular}{m{17cm}}
\textbf{44-variant}\\
1. Eki gewekli giperboloid. Kanonikalıq teńlemesi (giperbolanı simmetriya kósheri átirapında aylandırıwdan alınǵan betlik).\\

2. Ellipstiń urınbasınıń teńlemesi (ellips, tuwrı, urınıw tochka, urınba teńlemesi).\\

3. Sheńberdiń $C$ orayı hám $R$ radiusın tabıń: $x^2+y^2+4 x-2 y+5=0$.\\

4. Koordinata kósherlerin túrlendirmey ETİS teńlemesin ápiwaylastırıń, qanday geometriyalıq obrazdı anıqlaytuǵının kórsetiń $4x^{2} - 4xy + y^{2} + 4x - 2y + 1 = 0$.  \\

5. $2x^{2} + 3y^{2} + 8x - 6y + 11 = 0$ teńlemesin ápiwaylastırıń qanday geometriyalıq obrazdı anıqlaytuǵının tabıń hám grafigin jasań.  
\end{tabular}
\vspace{1cm}


\begin{tabular}{m{17cm}}
\textbf{45-variant}\\
1. ETIS -tiń ulıwma teńlemesin ápiwaylastırıw (ETIS -tiń ulıwma teńlemesi, koordinata sistemasın túrlendirip ETIS ulıwma teńlemesin ápiwaylastırıw).\\

2. Ellipsoida. Kanonikalıq teńlemesi (ellipsti simmetriya kósheri dogereginde aylandırıwdan alınǵan betlik, kanonikalıq teńlemesi).\\

3. Fokusları abscissa kósherinde hám koordinata basına qarata simmetriyalıq jaylasqan giperbolanıń teńlemesin dúziń: úlken kósheri $2 a=16$ hám ekscentrisitet $\varepsilon=5/4$.\\

4. $x^{2} + 4y^{2} = 25$ ellipsi menen $4x - 2y + 23 = 0$ tuwrı sızıǵına parallel bolǵan urınba tuwrı sızıqtıń teńlemesin dúziń.  \\

5. $\frac{x^{2}}{3} - \frac{y^{2}}{5} = 1$, giperbolasına $P(4;2)$ noqatınan júrgizilgen urınbalardıń teńlemesin dúziń.  
\end{tabular}
\vspace{1cm}


\begin{tabular}{m{17cm}}
\textbf{46-variant}\\
1. Ellips hám onıń kanonikalıq teńlemesi (anıqlaması, fokuslar, ellipstiń kanonikalıq teńlemesi, ekscentrisiteti, direktrisaları).\\

2. ETIS-tıń ulıwma teńlemesin koordinata basın parallel kóshiriw arqalı ápiwayılastırıń (ETIS- tıń ulıwma teńlemesin parallel kóshiriw formulası).\\

3. Tipin anıqlań: $2 x^{2}+10 xy+12 y^{2}-7 x+18 y-15=0$.\\

4. Koordinata kósherlerin túrlendirmey ETİS teńlemesin ápiwaylastırıń, yarım kósherlerin tabıń $4x^{2} - 4xy + 9y^{2} - 26x - 18y + 3 = 0$.\\

5. $y^{2} = 20x$ parabolasınıń abscissası 7 ge teń bolǵan $M$ noqatınıń fokal radiusın tabıń hám fokal radiusı jatqan tuwrınıń teńlemesin dúziń.  
\end{tabular}
\vspace{1cm}


\begin{tabular}{m{17cm}}
\textbf{47-variant}\\
1. Ellipslik paraboloid (parabola, kósher, ellipslik paraboloid).\\

2. Giperbolanıń urınbasınıń teńlemesi (giperbolaǵa berilgen noqatta júrgizilgen urınba teńlemesi).\\

3. Fokusları abscissa kósherinde hám koordinata basına qarata simmetriyalıq jaylasqan ellipstiń teńlemesin dúziń: úlken kósheri $10$, fokusları arasındaǵı aralıq $2 c=8$.\\

4. $\frac{x^{2}}{4} - \frac{y^{2}}{5} = 1$ giperbolaǵa $3x - 2y = 0$ tuwrısına parallel bolǵan urınbanıń teńlemesin dúziń.  \\

5. Fokusı $F( - 1; - 4)$noqatında bolǵan, sáykes direktrissası $x - 2 = 0$ teńlemesi menen berilgen $A( - 3; - 5)$ noqatınan ótiwshi ellipstiń teńlemesin dúziń.  
\end{tabular}
\vspace{1cm}


\begin{tabular}{m{17cm}}
\textbf{48-variant}\\
1. ETIS-tıń invariantları (ETIS-tıń ulıwma teńlemesi, túrlendiriw, ETIS invariantları ).\\

2. Ekinshi tártipli aylanba betlikler (koordinata sisteması, tegislik, vektor iymek sızıq, aylanba betlik).\\

3. Tipin anıqlań: $25 x^{2}-20 xy+4 y^{2}-12 x+20 y-17=0$.\\

4. $\frac{x^{2}}{4} - \frac{y^{2}}{5} = 1$ giperbolasına $3x + 2y = 0$ tuwrı sızıǵına perpendikulyar bolǵan urınba tuwrınıń teńlemesin dúziń.\\

5. $2x^{2} + 10xy + 12y^{2} - 7x + 18y - 15 = 0$ ETİS teńlemesin ápiwayı túrge alıp keliń, tipin anıqlań, qanday geometriyalıq obrazdı anıqlaytuǵının kórsetiń, sızılmasın góne hám taza koordinatalar sistemasına qarata jasań  
\end{tabular}
\vspace{1cm}


\begin{tabular}{m{17cm}}
\textbf{49-variant}\\
1. Parabolanıń polyar koordinatalardaǵı teńlemesi (polyar koordinata sistemasında parabolanıń teńlemesi).\\

2. Bir gewekli giperboloid. Kanonikalıq teńlemesi (giperbolanı simmetriya kósheri átirapında aylandırıwdan alınǵan betlik).\\

3. Fokusları abscissa kósherinde hám koordinata basına qarata simmetriyalıq jaylasqan giperbolanıń teńlemesin dúziń: fokusları arasındaǵı aralıq $2 c=6$ hám ekscentrisitet $\varepsilon=3/2$.\\

4. $\frac{x^{2}}{4} - \frac{y^{2}}{5} = 1$, giperbolanıń $3x - 2y = 0$ tuwrı sızıǵına parallel bolǵan urınbasınıń teńlemesin dúziń.  \\

5. $\frac{x^{2}}{25} + \frac{y^{2}}{16} = 1$, ellipsine $C(10; - 8)$ noqatınan júrgizilgen urınbalarınıń teńlemesin dúziń.  
\end{tabular}
\vspace{1cm}


\begin{tabular}{m{17cm}}
\textbf{50-variant}\\
1. Ellipstiń polyar koordinatalardaǵı teńlemesi (polyar koordinatalar sistemasında ellipstiń teńlemesi).\\

2. Cilindrlik betlikler (jasawshı tuwrı sızıq, baǵıtlawshı iymek sızıq, cilindrlik betlik).\\

3. Berilgen sızıqlardıń oraylıq ekenligin kórsetiń hám orayın tabıń: $5 x^{2}+4 xy+2 y^{2}+20 x+20 y-18=0$.\\

4. Ellips $3x^{2} + 4y^{2} - 12 = 0$ teńlemesi menen berilgen. Onıń kósherleriniń uzınlıqların, fokuslarınıń koordinataların hám ekscentrisitetin tabıń.  \\

5. Eger qálegen waqıt momentinde $M(x;y)$ noqat $A(8;4)$ noqattan hám ordinata kósherinen birdey aralıqta jaylassa, $M(x;y)$ noqatınıń háreket etiw troektoriyasınıń teńlemesin dúziń.  
\end{tabular}
\vspace{1cm}


\begin{tabular}{m{17cm}}
\textbf{51-variant}\\
1. Giperbola. Kanonikalıq teńlemesi (fokuslar, kósherler, direktrisalar, giperbola, ekscentrisitet, kanonikalıq teńlemesi).\\

2. Betliktiń kanonikalıq teńlemeleri. Betlik haqqında túsinik. (Betliktiń anıqlaması, formulaları, kósher, baǵıtlawshı tuwrılar).\\

3. Fokusları abscissa kósherinde hám koordinata basına qarata simmetriyalıq jaylasqan ellipstiń teńlemesin dúziń: úlken kósheri $8$, direktrisaları arasındaǵı aralıq $16$.\\

4. $\frac{x^{2}}{16} - \frac{y^{2}}{64} = 1$, giperbolasına berilgen $10x - 3y + 9 = 0$ tuwrı sızıǵına parallel bolǵan urınbanıń teńlemesin dúziń.  \\

5. $16x^{2} - 9y^{2} - 64x - 54y - 161 = 0$ teńlemesi giperbolanıń teńlemesi ekenin anıqlań hám onıń orayı $C$, yarım kósherleri, ekscentrisitetin, asimptotalarınıń teńlemelerin dúziń.  
\end{tabular}
\vspace{1cm}


\begin{tabular}{m{17cm}}
\textbf{52-variant}\\
1. Parabolanıń urınbasınıń teńlemesi (parabola, tuwrı, urınıw noqatı, urınba teńlemesi).\\

2. Betlik haqqında túsinik (tuwrı, iymek sızıq, betliktiń anıqlamaları hám formulaları).\\

3. Berilgen sızıqlardıń oraylıq ekenligin kórsetiń hám orayın tabıń: $2 x^{2}-6 xy+5 y^{2}+22 x-36 y+11=0$.\\

4. Koordinata kósherlerin túrlendirmey ETİS ulıwma teńlemesin ápiwaylastırıń, yarım kósherlerin tabıń: $13x^{2} + 18xy + 37y^{2} - 26x - 18y + 3 = 0$.  \\

5. Tóbesi $A(-4;0)$ noqatında, al, direktrisası $y - 2 = 0$ tuwrı sızıq bolǵan parabolanıń teńlemesin dúziń.
\end{tabular}
\vspace{1cm}


\begin{tabular}{m{17cm}}
\textbf{53-variant}\\
1. Koordinata sistemasın túrlendiriw (birlik vektorlar, kósherler, parallel kóshiriw, koordinata kósherlerin burıw).\\

2. ETIS-tıń ulıwma teńlemesin klassifikatsiyalaw (ETIS-tıń ulıwma teńlemesi, ETIS-tıń ulıwma teńlemesin ápiwaylastırıw, klassifikatsiyalaw).\\

3. Fokusları abscissa kósherinde hám koordinata basına qarata simmetriyalıq jaylasqan giperbolanıń teńlemesin dúziń: asimptotalar teńlemeleri $y=\pm \frac{4}{3}x$ hám fokusları arasındaǵı aralıq $2 c=20$.\\

4. $x^{2} - 4y^{2} = 16$ giperbola berilgen. Onıń ekscentrisitetin, fokuslarınıń koordinataların tabıń hám asimptotalarınıń teńlemelerin dúziń.\\

5. $4x^{2} + 24xy + 11y^{2} + 64x + 42y + 51 = 0$ iymek sızıǵınıń tipin anıqlań eger orayı bar bolsa, onıń orayınıń koordinataların tabıń hám koordinata basın orayǵa parallel kóshiriw ámelin orınlań.  
\end{tabular}
\vspace{1cm}


\begin{tabular}{m{17cm}}
\textbf{54-variant}\\
1. Giperbolalıq paraboloydtıń tuwrı sızıqlı jasawshıları (Giperbolalıq paraboloydtı jasawshı tuwrı sızıqlar dástesi).\\

2. Giperbolanıń polyar koordinatadaǵı teńlemesi (Polyar múyeshi, polyar radiusi giperbolanıń polyar teńlemesi).\\

3. Fokusları abscissa kósherinde hám koordinata basına qarata simmetriyalıq jaylasqan ellipstiń teńlemesin dúziń: kishi kósheri $24$, fokusları arasındaǵı aralıq $2 c=10$.\\

4. $y^{2} = 3x$ parabolası menen $\frac{x^{2}}{100} + \frac{y^{2}}{225} = 1$ ellipsiniń kesilisiw noqatların tabıń.  \\

5. Fokusı $F(7;2)$ noqatında jaylasqan, sáykes direktrisası $x - 5 = 0$ teńlemesi menen berilgen parabolanıń teńlemesin dúziń.  
\end{tabular}
\vspace{1cm}


\begin{tabular}{m{17cm}}
\textbf{55-variant}\\
1. ETIS-tıń orayın anıqlaw forması (ETIS-tıń ulıwma teńlemesi, orayın anıqlaw forması).\\

2. Ekinshi tártipli betliktiń ulıwma teńlemesi. Orayın anıqlaw formulası.\\

3. Fokusları abscissa kósherinde hám koordinata basına qarata simmetriyalıq jaylasqan ellipstiń teńlemesin dúziń: yarım oqları 5 hám 2.\\

4. $\rho = \frac{6}{1 - cos\theta}$ polyar teńlemesi menen qanday sızıq berilgenin anıqlań.  \\

5. $2x^{2} + 3y^{2} + 8x - 6y + 11 = 0$ teńlemesin ápiwaylastırıń qanday geometriyalıq obrazdı anıqlaytuǵının tabıń hám grafigin jasań.
\end{tabular}
\vspace{1cm}


\begin{tabular}{m{17cm}}
\textbf{56-variant}\\
1. Parabola hám onıń kanonikalıq teńlemesi (anıqlaması, fokusı, direktrisası, kanonikalıq teńlemesi).\\

2. ETIS-tıń ulıwma teńlemesin koordinata kósherlerin burıw arqalı ápiwaylastırıń (ETIS-tıń ulıwma teńlemeleri, koordinata kósherin burıw formulası, teńlemeni kanonik túrge alıp keliw).\\

3. Fokusları abscissa kósherinde hám koordinata basına qarata simmetriyalıq jaylasqan ellipstiń teńlemesin dúziń: kishi kósheri $6$, direktrisaları arasındaǵı aralıq $13$.\\

4. $x^{2} - y^{2} = 27$ giperbolasına $4x + 2y - 7 = 0$ tuwrısına parallel bolǵan urınbanıń teńlemesin tabıń.  \\

5. Fokusı $F( - 1; - 4)$ noqatında jaylasqan, sáykes direktrisası $x - 2 = 0$ teńlemesi menen berilgen, $A( - 3; - 5)$ noqatınan ótiwshi ellipstiń teńlemesin dúziń.  
\end{tabular}
\vspace{1cm}


\begin{tabular}{m{17cm}}
\textbf{57-variant}\\
1. Eki gewekli giperboloid. Kanonikalıq teńlemesi (giperbolanı simmetriya kósheri átirapında aylandırıwdan alınǵan betlik).\\

2. Ellipstiń urınbasınıń teńlemesi (ellips, tuwrı, urınıw tochka, urınba teńlemesi).\\

3. Fokusları abscissa kósherinde hám koordinata basına qarata simmetriyalıq jaylasqan giperbolanıń teńlemesin dúziń: fokusları arasındaǵı aralıǵı $2 c=10$ hám kósheri $2 b=8$.\\

4. $41x^{2} + 24xy + 9y^{2} + 24x + 18y - 36 = 0$ ETİS tipin anıqlań hám orayların tabıń koordinata kósherlerin túrlendirmey qanday sızıqtı anıqlaytuǵının kórsetiń yarım kósherlerin tabıń.  \\

5. $32x^{2} + 52xy - 7y^{2} + 180 = 0$ ETİS teńlemesin ápiwayı túrge alıp keliń, tipin anıqlań, qanday geometriyalıq obrazdı anıqlaytuǵının kórsetiń, sızılmasın góne hám taza koordinatalar sistemasına qarata jasań.  
\end{tabular}
\vspace{1cm}


\begin{tabular}{m{17cm}}
\textbf{58-variant}\\
1. ETIS -tiń ulıwma teńlemesin ápiwaylastırıw (ETIS -tiń ulıwma teńlemesi, koordinata sistemasın túrlendirip ETIS ulıwma teńlemesin ápiwaylastırıw).\\

2. Ellipsoida. Kanonikalıq teńlemesi (ellipsti simmetriya kósheri dogereginde aylandırıwdan alınǵan betlik, kanonikalıq teńlemesi).\\

3. Fokusları abscissa kósherinde hám koordinata basına qarata simmetriyalıq jaylasqan ellipstiń teńlemesin dúziń: úlken kósheri $20$, ekscentrisitet $\varepsilon=3/5$.\\

4. Ellips $3x^{2} + 4y^{2} - 12 = 0$ teńlemesi menen berilgen. Onıń kósherleriniń uzınlıqların, fokuslarınıń koordinataların hám ekscentrisitetin tabıń.  \\

5. Úlken kósheri 26 ǵa, fokusları $F( - 10;0)$, $F(14;0)$ noqatlarında jaylasqan ellipstiń teńlemesin dúziń.  
\end{tabular}
\vspace{1cm}


\begin{tabular}{m{17cm}}
\textbf{59-variant}\\
1. Ellips hám onıń kanonikalıq teńlemesi (anıqlaması, fokuslar, ellipstiń kanonikalıq teńlemesi, ekscentrisiteti, direktrisaları).\\

2. ETIS-tıń ulıwma teńlemesin koordinata basın parallel kóshiriw arqalı ápiwayılastırıń (ETIS- tıń ulıwma teńlemesin parallel kóshiriw formulası).\\

3. Sheńber teńlemesin dúziń: orayı koordinata basında jaylasqan hám radiusı $R=3$ ge teń.\\

4. $3x + 4y - 12 = 0$ tuwrı sızıǵı hám $y^{2} = - 9x$ parabolasınıń kesilisiw noqatların tabıń.  \\

5. $4x^{2} - 4xy + y^{2} - 6x + 8y + 13 = 0$ ETİS-ǵı orayǵa iyeme? Orayǵa iye bolsa orayın anıqlań: jalǵız orayǵa iyeme-?, sheksiz orayǵa iyeme-?  
\end{tabular}
\vspace{1cm}


\begin{tabular}{m{17cm}}
\textbf{60-variant}\\
1. Ellipslik paraboloid (parabola, kósher, ellipslik paraboloid).\\

2. Giperbolanıń urınbasınıń teńlemesi (giperbolaǵa berilgen noqatta júrgizilgen urınba teńlemesi).\\

3. Fokusları abscissa kósherinde hám koordinata basına qarata simmetriyalıq jaylasqan giperbolanıń teńlemesin dúziń: direktrisaları arasındaǵı aralıq $228/13$ hám fokusları arasındaǵı aralıq $2 c=26$.\\

4. $\rho = \frac{5}{3 - 4cos\theta}$ teńlemesi menen qanday sızıq berilgenin hám yarım kósherlerin tabıń.  \\

5. $2x^{2} + 3y^{2} + 8x - 6y + 11 = 0$ teńlemesi menen qanday tiptegi sızıq berilgenin anıqlań hám onıń teńlemesin ápiwaylastırıń hám grafigin jasań.  
\end{tabular}
\vspace{1cm}


\begin{tabular}{m{17cm}}
\textbf{61-variant}\\
1. ETIS-tıń invariantları (ETIS-tıń ulıwma teńlemesi, túrlendiriw, ETIS invariantları ).\\

2. Ekinshi tártipli aylanba betlikler (koordinata sisteması, tegislik, vektor iymek sızıq, aylanba betlik).\\

3. Polyar teńlemesi menen berilgen iymek sızıqtıń tipin anıqlań: $\rho=\frac{10}{1-\frac{3}{2}\cos\theta}$.\\

4. $\frac{x^{2}}{20} - \frac{y^{2}}{5} = 1$ giperbolasına $4x + 3y - 7 = 0$ tuwrısına perpendikulyar bolǵan urınbanıń teńlemesin dúziń.  \\

5. Fokusı $F(2; - 1)$ noqatında jaylasqan, sáykes direktrisası $x - y - 1 = 0$ teńlemesi menen berilgen parabolanıń teńlemesin dúziń.  
\end{tabular}
\vspace{1cm}


\begin{tabular}{m{17cm}}
\textbf{62-variant}\\
1. Parabolanıń polyar koordinatalardaǵı teńlemesi (polyar koordinata sistemasında parabolanıń teńlemesi).\\

2. Bir gewekli giperboloid. Kanonikalıq teńlemesi (giperbolanı simmetriya kósheri átirapında aylandırıwdan alınǵan betlik).\\

3. Tipin anıqlań: $4 x^{2}-y^{2}+8 x-2 y+3=0$.\\

4. ETİS-tıń ulıwma teńlemesin koordinata sistemasın túrlendirmey ápiwaylastırıń, tipin anıqlań, obrazı qanday sızıqtı anıqlaytuǵının kórsetiń. $7x^{2} - 8xy + y^{2} - 16x - 2y - 51 = 0$  \\

5. $32x^{2} + 52xy - 9y^{2} + 180 = 0$ ETİS teńlemesin ápiwaylastırıń, tipin anıqlań, qanday geometriyalıq obrazdı anıqlaytuǵının kórsetiń, sızılmasın sızıń.  
\end{tabular}
\vspace{1cm}


\begin{tabular}{m{17cm}}
\textbf{63-variant}\\
1. Ellipstiń polyar koordinatalardaǵı teńlemesi (polyar koordinatalar sistemasında ellipstiń teńlemesi).\\

2. Cilindrlik betlikler (jasawshı tuwrı sızıq, baǵıtlawshı iymek sızıq, cilindrlik betlik).\\

3. Sheńber teńlemesin dúziń: orayı $C (1;-1) $ noqatında jaylasqan hám $5 x-12 y+9 -0$ tuwrı sızıǵına urınadı .\\

4. $3x + 10y - 25 = 0$ tuwrı menen $\frac{x^{2}}{25} + \frac{y^{2}}{4} = 1$ ellipstiń kesilisiw noqatların tabıń.\\

5. $A(\frac{10}{3};\frac{5}{3})$ noqattan $\frac{x^{2}}{20} + \frac{y^{2}}{5} = 1$ ellipsine júrgizilgen urınbalardıń teńlemesin dúziń.  
\end{tabular}
\vspace{1cm}


\begin{tabular}{m{17cm}}
\textbf{64-variant}\\
1. Giperbola. Kanonikalıq teńlemesi (fokuslar, kósherler, direktrisalar, giperbola, ekscentrisitet, kanonikalıq teńlemesi).\\

2. Betliktiń kanonikalıq teńlemeleri. Betlik haqqında túsinik. (Betliktiń anıqlaması, formulaları, kósher, baǵıtlawshı tuwrılar).\\

3. Uchı koordinata basında jaylasqan hám $Oy$ kósherine qarata oń táreptegi yarım tegislikte jaylasqan parabolanıń teńlemesin dúziń: parametri $p=3$.\\

4. $\rho = \frac{10}{2 - cos\theta}$ polyar teńlemesi menen qanday sızıq berilgenin anıqlań.  \\

5. $M(2; - \frac{5}{3})$ noqatı $\frac{x^{2}}{9} + \frac{y^{2}}{5} = 1$ ellipsinde jaylasqan. $M$ noqatınıń fokal radiusları jatıwshı tuwrı sızıq teńlemelerin dúziń.  
\end{tabular}
\vspace{1cm}


\begin{tabular}{m{17cm}}
\textbf{65-variant}\\
1. Parabolanıń urınbasınıń teńlemesi (parabola, tuwrı, urınıw noqatı, urınba teńlemesi).\\

2. Betlik haqqında túsinik (tuwrı, iymek sızıq, betliktiń anıqlamaları hám formulaları).\\

3. Ellips teńlemesi berilgen: $\frac{x^2}{25}+\frac{y^2}{16}=1$. Onıń polyar teńlemesin dúziń.\\

4. $2x + 2y - 3 = 0$ tuwrısına parallel bolıp $\frac{x^{2}}{16} + \frac{y^{2}}{64} = 1$ giperbolasına urınıwshı tuwrınıń teńlemesin dúziń.  \\

5. Eger waqıttıń qálegen momentinde $M(x;y)$ noqat $5x - 16 = 0$ tuwrı sızıqqa qaraǵanda $A(5;0)$ noqattan 1,25 márte uzaqlıqta jaylasqan. Usı $M(x;y)$ noqattıń háreketiniń teńlemesin dúziń.  
\end{tabular}
\vspace{1cm}


\begin{tabular}{m{17cm}}
\textbf{66-variant}\\
1. Koordinata sistemasın túrlendiriw (birlik vektorlar, kósherler, parallel kóshiriw, koordinata kósherlerin burıw).\\

2. ETIS-tıń ulıwma teńlemesin klassifikatsiyalaw (ETIS-tıń ulıwma teńlemesi, ETIS-tıń ulıwma teńlemesin ápiwaylastırıw, klassifikatsiyalaw).\\

3. Tipin anıqlań: $x^{2}-4 xy+4 y^{2}+7 x-12=0$.\\

4. Koordinata kósherlerin túrlendirmey ETİS teńlemesin ápiwaylastırıń, yarım kósherlerin tabıń $41x^{2} + 2xy + 9y^{2} - 26x - 18y + 3 = 0$.  \\

5. $4x^{2} - 4xy + y^{2} - 2x - 14y + 7 = 0$ ETİS teńlemesin ápiwayı túrge alıp keliń, tipin anıqlań, qanday geometriyalıq obrazdı anıqlaytuǵının kórsetiń, sızılmasın góne hám taza koordinatalar sistemasına qarata jasań.  
\end{tabular}
\vspace{1cm}


\begin{tabular}{m{17cm}}
\textbf{67-variant}\\
1. Giperbolalıq paraboloydtıń tuwrı sızıqlı jasawshıları (Giperbolalıq paraboloydtı jasawshı tuwrı sızıqlar dástesi).\\

2. Giperbolanıń polyar koordinatadaǵı teńlemesi (Polyar múyeshi, polyar radiusi giperbolanıń polyar teńlemesi).\\

3. Sheńberdiń $C$ orayı hám $R$ radiusın tabıń: $x^2+y^2+6 x-4 y+14=0$.\\

4. $\rho = \frac{144}{13 - 5cos\theta}$ ellipsti anıqlaytuǵının kórsetiń hám onıń yarım kósherlerin anıqlań.\\

5. $\frac{x^{2}}{3} - \frac{y^{2}}{5} = 1$ giperbolasına $P(1; - 5)$ noqatında júrgizilgen urınbalardıń teńlemesin dúziń.
\end{tabular}
\vspace{1cm}


\begin{tabular}{m{17cm}}
\textbf{68-variant}\\
1. ETIS-tıń orayın anıqlaw forması (ETIS-tıń ulıwma teńlemesi, orayın anıqlaw forması).\\

2. Ekinshi tártipli betliktiń ulıwma teńlemesi. Orayın anıqlaw formulası.\\

3. Uchı koordinata basında jaylasqan hám $Ox$ kósherine qarata tómengi yarım tegislikte jaylasqan parabolanıń teńlemesin dúziń: parametri $p=3$.\\

4. $y^{2} = 12x$ paraborolasına $3x - 2y + 30 = 0$ tuwrı sızıǵına parallel bolǵan urınbanıń teńlemesin dúziń.  \\

5. $y^{2} = 20x$ parabolasınıń $M$ noqatın tabıń, eger onıń abscissası 7 ge teń bolsa, fokal radiusın hám fokal radius jaylasqan tuwrını anıqlań.
\end{tabular}
\vspace{1cm}


\begin{tabular}{m{17cm}}
\textbf{69-variant}\\
1. Parabola hám onıń kanonikalıq teńlemesi (anıqlaması, fokusı, direktrisası, kanonikalıq teńlemesi).\\

2. ETIS-tıń ulıwma teńlemesin koordinata kósherlerin burıw arqalı ápiwaylastırıń (ETIS-tıń ulıwma teńlemeleri, koordinata kósherin burıw formulası, teńlemeni kanonik túrge alıp keliw).\\

3. Polyar teńlemesi menen berilgen iymek sızıqtıń tipin anıqlań: $\rho=\frac{5}{3-4\cos\theta}$.\\

4. Koordinata kósherlerin túrlendirmey ETİS teńlemesin ápiwaylastırıń, qanday geometriyalıq obrazdı anıqlaytuǵının kórsetiń $4x^{2} - 4xy + y^{2} + 4x - 2y + 1 = 0$.  \\

5. Fokuslari $F(3;4), F(-3;-4)$ noqatlarında jaylasqan direktrisaları orasıdaǵı aralıq 3,6 ǵa teń bolǵan giperbolanıń teńlemesin dúziń.  
\end{tabular}
\vspace{1cm}


\begin{tabular}{m{17cm}}
\textbf{70-variant}\\
1. Eki gewekli giperboloid. Kanonikalıq teńlemesi (giperbolanı simmetriya kósheri átirapında aylandırıwdan alınǵan betlik).\\

2. Ellipstiń urınbasınıń teńlemesi (ellips, tuwrı, urınıw tochka, urınba teńlemesi).\\

3. Tipin anıqlań: $4 x^2+9 y^2-40 x+36 y+100=0$.\\

4. $2x + 2y - 3 = 0$ tuwrısına perpendikulyar bolıp $x^{2} = 16y$ parabolasına urınıwshı tuwrınıń teńlemesin dúziń.  \\

5. $14x^{2} + 24xy + 21y^{2} - 4x + 18y - 139 = 0$ iymek sızıǵınıń tipin anıqlań, eger oraylı iymek sızıq bolsa orayınıń koordinataların tabıń.  
\end{tabular}
\vspace{1cm}


\begin{tabular}{m{17cm}}
\textbf{71-variant}\\
1. ETIS -tiń ulıwma teńlemesin ápiwaylastırıw (ETIS -tiń ulıwma teńlemesi, koordinata sistemasın túrlendirip ETIS ulıwma teńlemesin ápiwaylastırıw).\\

2. Ellipsoida. Kanonikalıq teńlemesi (ellipsti simmetriya kósheri dogereginde aylandırıwdan alınǵan betlik, kanonikalıq teńlemesi).\\

3. Sheńberdiń $C$ orayı hám $R$ radiusın tabıń: $x^2+y^2-2 x+4 y-20=0$.\\

4. Koordinata kósherlerin túrlendirmey ETİS teńlemesin ápiwaylastırıń, yarım kósherlerin tabıń $4x^{2} - 4xy + 9y^{2} - 26x - 18y + 3 = 0$.\\

5. $\frac{x^{2}}{2} + \frac{y^{2}}{3} = 1$, ellipsin $x + y - 2 = 0$ noqatınan júrgizilgen urınbalarınıń teńlemesin dúziń.  
\end{tabular}
\vspace{1cm}


\begin{tabular}{m{17cm}}
\textbf{72-variant}\\
1. Ellips hám onıń kanonikalıq teńlemesi (anıqlaması, fokuslar, ellipstiń kanonikalıq teńlemesi, ekscentrisiteti, direktrisaları).\\

2. ETIS-tıń ulıwma teńlemesin koordinata basın parallel kóshiriw arqalı ápiwayılastırıń (ETIS- tıń ulıwma teńlemesin parallel kóshiriw formulası).\\

3. Fokusları abscissa kósherinde hám koordinata basına qarata simmetriyalıq jaylasqan ellipstiń teńlemesin dúziń: fokusları arasındaǵı aralıq $2 c=6$ hám ekscentrisitet $\varepsilon=3/5$.\\

4. $x^{2} + 4y^{2} = 25$ ellipsi menen $4x - 2y + 23 = 0$ tuwrı sızıǵına parallel bolǵan urınba tuwrı sızıqtıń teńlemesin dúziń.  \\

5. $\frac{x^{2}}{100} + \frac{y^{2}}{36} = 1$ ellipsiniń oń jaqtaǵı fokusınan 14 ge teń aralıqta bolǵan noqattı tabıń.  
\end{tabular}
\vspace{1cm}


\begin{tabular}{m{17cm}}
\textbf{73-variant}\\
1. Ellipslik paraboloid (parabola, kósher, ellipslik paraboloid).\\

2. Giperbolanıń urınbasınıń teńlemesi (giperbolaǵa berilgen noqatta júrgizilgen urınba teńlemesi).\\

3. Polyar teńlemesi menen berilgen iymek sızıqtıń tipin anıqlań: $\rho=\frac{12}{2-\cos\theta}$.\\

4. $\frac{x^{2}}{4} - \frac{y^{2}}{5} = 1$ giperbolaǵa $3x - 2y = 0$ tuwrısına parallel bolǵan urınbanıń teńlemesin dúziń.  \\

5. Giperbolanıń ekscentrisiteti $\varepsilon = \frac{13}{12}$, fokusı $F(0;13)$ noqatında hám sáykes direktrisası $13y - 144 = 0$ teńlemesi menen berilgen bolsa, giperbolanıń teńlemesin dúziń.  
\end{tabular}
\vspace{1cm}


\begin{tabular}{m{17cm}}
\textbf{74-variant}\\
1. ETIS-tıń invariantları (ETIS-tıń ulıwma teńlemesi, túrlendiriw, ETIS invariantları ).\\

2. Ekinshi tártipli aylanba betlikler (koordinata sisteması, tegislik, vektor iymek sızıq, aylanba betlik).\\

3. Berilgen sızıqlardıń oraylıq ekenligin kórsetiń hám orayın tabıń: $9 x^{2}-4 xy-7 y^{2}-12=0$.\\

4. $\frac{x^{2}}{4} - \frac{y^{2}}{5} = 1$ giperbolasına $3x + 2y = 0$ tuwrı sızıǵına perpendikulyar bolǵan urınba tuwrınıń teńlemesin dúziń.\\

5. $2x^{2} + 3y^{2} + 8x - 6y + 11 = 0$ teńlemesin ápiwaylastırıń qanday geometriyalıq obrazdı anıqlaytuǵının tabıń hám grafigin jasań.  
\end{tabular}
\vspace{1cm}


\begin{tabular}{m{17cm}}
\textbf{75-variant}\\
1. Parabolanıń polyar koordinatalardaǵı teńlemesi (polyar koordinata sistemasında parabolanıń teńlemesi).\\

2. Bir gewekli giperboloid. Kanonikalıq teńlemesi (giperbolanı simmetriya kósheri átirapında aylandırıwdan alınǵan betlik).\\

3. Sheńber teńlemesin dúziń: sheńber diametriniń ushları $A (3;2) $ hám $B (-1;6 ) $ noqatlarında jaylasqan.\\

4. $x^{2} - 4y^{2} = 16$ giperbola berilgen. Onıń ekscentrisitetin, fokuslarınıń koordinataların tabıń hám asimptotalarınıń teńlemelerin dúziń.\\

5. $\frac{x^{2}}{3} - \frac{y^{2}}{5} = 1$, giperbolasına $P(4;2)$ noqatınan júrgizilgen urınbalardıń teńlemesin dúziń.  
\end{tabular}
\vspace{1cm}


\begin{tabular}{m{17cm}}
\textbf{76-variant}\\
1. Ellipstiń polyar koordinatalardaǵı teńlemesi (polyar koordinatalar sistemasında ellipstiń teńlemesi).\\

2. Cilindrlik betlikler (jasawshı tuwrı sızıq, baǵıtlawshı iymek sızıq, cilindrlik betlik).\\

3. Fokusları abscissa kósherinde hám koordinata basına qarata simmetriyalıq jaylasqan ellipstiń teńlemesin dúziń: kishi kósheri $10$, ekscentrisitet $\varepsilon=12/13$.\\

4. $\frac{x^{2}}{4} - \frac{y^{2}}{5} = 1$, giperbolanıń $3x - 2y = 0$ tuwrı sızıǵına parallel bolǵan urınbasınıń teńlemesin dúziń.  \\

5. $y^{2} = 20x$ parabolasınıń abscissası 7 ge teń bolǵan $M$ noqatınıń fokal radiusın tabıń hám fokal radiusı jatqan tuwrınıń teńlemesin dúziń.  
\end{tabular}
\vspace{1cm}


\begin{tabular}{m{17cm}}
\textbf{77-variant}\\
1. Giperbola. Kanonikalıq teńlemesi (fokuslar, kósherler, direktrisalar, giperbola, ekscentrisitet, kanonikalıq teńlemesi).\\

2. Betliktiń kanonikalıq teńlemeleri. Betlik haqqında túsinik. (Betliktiń anıqlaması, formulaları, kósher, baǵıtlawshı tuwrılar).\\

3. Giperbola teńlemesi berilgen: $\frac{x^{2}}{16}-\frac{y^{2}}{9}=1$. Onıń polyar teńlemesin dúziń.\\

4. Koordinata kósherlerin túrlendirmey ETİS ulıwma teńlemesin ápiwaylastırıń, yarım kósherlerin tabıń: $13x^{2} + 18xy + 37y^{2} - 26x - 18y + 3 = 0$.  \\

5. Fokusı $F( - 1; - 4)$noqatında bolǵan, sáykes direktrissası $x - 2 = 0$ teńlemesi menen berilgen $A( - 3; - 5)$ noqatınan ótiwshi ellipstiń teńlemesin dúziń.  
\end{tabular}
\vspace{1cm}


\begin{tabular}{m{17cm}}
\textbf{78-variant}\\
1. Parabolanıń urınbasınıń teńlemesi (parabola, tuwrı, urınıw noqatı, urınba teńlemesi).\\

2. Betlik haqqında túsinik (tuwrı, iymek sızıq, betliktiń anıqlamaları hám formulaları).\\

3. Berilgen sızıqlardıń oraylıq ekenligin kórsetiń hám orayın tabıń: $3 x^{2}+5 xy+y^{2}-8 x-11 y-7=0$.\\

4. Ellips $3x^{2} + 4y^{2} - 12 = 0$ teńlemesi menen berilgen. Onıń kósherleriniń uzınlıqların, fokuslarınıń koordinataların hám ekscentrisitetin tabıń.  \\

5. $2x^{2} + 10xy + 12y^{2} - 7x + 18y - 15 = 0$ ETİS teńlemesin ápiwayı túrge alıp keliń, tipin anıqlań, qanday geometriyalıq obrazdı anıqlaytuǵının kórsetiń, sızılmasın góne hám taza koordinatalar sistemasına qarata jasań  
\end{tabular}
\vspace{1cm}


\begin{tabular}{m{17cm}}
\textbf{79-variant}\\
1. Koordinata sistemasın túrlendiriw (birlik vektorlar, kósherler, parallel kóshiriw, koordinata kósherlerin burıw).\\

2. ETIS-tıń ulıwma teńlemesin klassifikatsiyalaw (ETIS-tıń ulıwma teńlemesi, ETIS-tıń ulıwma teńlemesin ápiwaylastırıw, klassifikatsiyalaw).\\

3. Sheńber teńlemesin dúziń: sheńber $A (2;6 ) $ noqatınan ótedi hám orayı $C (-1;2) $ noqatında jaylasqan .\\

4. $y^{2} = 3x$ parabolası menen $\frac{x^{2}}{100} + \frac{y^{2}}{225} = 1$ ellipsiniń kesilisiw noqatların tabıń.  \\

5. $\frac{x^{2}}{25} + \frac{y^{2}}{16} = 1$, ellipsine $C(10; - 8)$ noqatınan júrgizilgen urınbalarınıń teńlemesin dúziń.  
\end{tabular}
\vspace{1cm}


\begin{tabular}{m{17cm}}
\textbf{80-variant}\\
1. Giperbolalıq paraboloydtıń tuwrı sızıqlı jasawshıları (Giperbolalıq paraboloydtı jasawshı tuwrı sızıqlar dástesi).\\

2. Giperbolanıń polyar koordinatadaǵı teńlemesi (Polyar múyeshi, polyar radiusi giperbolanıń polyar teńlemesi).\\

3. Fokusları abscissa kósherinde hám koordinata basına qarata simmetriyalıq jaylasqan giperbolanıń teńlemesin dúziń: direktrisaları arasındaǵı aralıq $32/5$ hám kósheri $2 b=6$.\\

4. $\rho = \frac{6}{1 - cos\theta}$ polyar teńlemesi menen qanday sızıq berilgenin anıqlań.  \\

5. Eger qálegen waqıt momentinde $M(x;y)$ noqat $A(8;4)$ noqattan hám ordinata kósherinen birdey aralıqta jaylassa, $M(x;y)$ noqatınıń háreket etiw troektoriyasınıń teńlemesin dúziń.  
\end{tabular}
\vspace{1cm}


\begin{tabular}{m{17cm}}
\textbf{81-variant}\\
1. ETIS-tıń orayın anıqlaw forması (ETIS-tıń ulıwma teńlemesi, orayın anıqlaw forması).\\

2. Ekinshi tártipli betliktiń ulıwma teńlemesi. Orayın anıqlaw formulası.\\

3. Polyar teńlemesi menen berilgen iymek sızıqtıń tipin anıqlań: $\rho=\frac{5}{1-\frac{1}{2}\cos\theta}$.\\

4. $\frac{x^{2}}{16} - \frac{y^{2}}{64} = 1$, giperbolasına berilgen $10x - 3y + 9 = 0$ tuwrı sızıǵına parallel bolǵan urınbanıń teńlemesin dúziń.  \\

5. $16x^{2} - 9y^{2} - 64x - 54y - 161 = 0$ teńlemesi giperbolanıń teńlemesi ekenin anıqlań hám onıń orayı $C$, yarım kósherleri, ekscentrisitetin, asimptotalarınıń teńlemelerin dúziń.  
\end{tabular}
\vspace{1cm}


\begin{tabular}{m{17cm}}
\textbf{82-variant}\\
1. Parabola hám onıń kanonikalıq teńlemesi (anıqlaması, fokusı, direktrisası, kanonikalıq teńlemesi).\\

2. ETIS-tıń ulıwma teńlemesin koordinata kósherlerin burıw arqalı ápiwaylastırıń (ETIS-tıń ulıwma teńlemeleri, koordinata kósherin burıw formulası, teńlemeni kanonik túrge alıp keliw).\\

3. Tipin anıqlań: $2 x^{2}+3 y^{2}+8 x-6 y+11=0$.\\

4. $41x^{2} + 24xy + 9y^{2} + 24x + 18y - 36 = 0$ ETİS tipin anıqlań hám orayların tabıń koordinata kósherlerin túrlendirmey qanday sızıqtı anıqlaytuǵının kórsetiń yarım kósherlerin tabıń.  \\

5. Tóbesi $A(-4;0)$ noqatında, al, direktrisası $y - 2 = 0$ tuwrı sızıq bolǵan parabolanıń teńlemesin dúziń.
\end{tabular}
\vspace{1cm}


\begin{tabular}{m{17cm}}
\textbf{83-variant}\\
1. Eki gewekli giperboloid. Kanonikalıq teńlemesi (giperbolanı simmetriya kósheri átirapında aylandırıwdan alınǵan betlik).\\

2. Ellipstiń urınbasınıń teńlemesi (ellips, tuwrı, urınıw tochka, urınba teńlemesi).\\

3. Sheńber teńlemesin dúziń: orayı $C (2;-3) $ noqatında jaylasqan hám radiusı $R=7$ ge teń.\\

4. $x^{2} - 4y^{2} = 16$ giperbola berilgen. Onıń ekscentrisitetin, fokuslarınıń koordinataların tabıń hám asimptotalarınıń teńlemelerin dúziń.\\

5. $4x^{2} + 24xy + 11y^{2} + 64x + 42y + 51 = 0$ iymek sızıǵınıń tipin anıqlań eger orayı bar bolsa, onıń orayınıń koordinataların tabıń hám koordinata basın orayǵa parallel kóshiriw ámelin orınlań.  
\end{tabular}
\vspace{1cm}


\begin{tabular}{m{17cm}}
\textbf{84-variant}\\
1. ETIS -tiń ulıwma teńlemesin ápiwaylastırıw (ETIS -tiń ulıwma teńlemesi, koordinata sistemasın túrlendirip ETIS ulıwma teńlemesin ápiwaylastırıw).\\

2. Ellipsoida. Kanonikalıq teńlemesi (ellipsti simmetriya kósheri dogereginde aylandırıwdan alınǵan betlik, kanonikalıq teńlemesi).\\

3. Fokusları abscissa kósherinde hám koordinata basına qarata simmetriyalıq jaylasqan giperbolanıń teńlemesin dúziń: oqları $2 a=10$ hám $2 b=8$.\\

4. $3x + 4y - 12 = 0$ tuwrı sızıǵı hám $y^{2} = - 9x$ parabolasınıń kesilisiw noqatların tabıń.  \\

5. Fokusı $F(7;2)$ noqatında jaylasqan, sáykes direktrisası $x - 5 = 0$ teńlemesi menen berilgen parabolanıń teńlemesin dúziń.  
\end{tabular}
\vspace{1cm}


\begin{tabular}{m{17cm}}
\textbf{85-variant}\\
1. Ellips hám onıń kanonikalıq teńlemesi (anıqlaması, fokuslar, ellipstiń kanonikalıq teńlemesi, ekscentrisiteti, direktrisaları).\\

2. ETIS-tıń ulıwma teńlemesin koordinata basın parallel kóshiriw arqalı ápiwayılastırıń (ETIS- tıń ulıwma teńlemesin parallel kóshiriw formulası).\\

3. Polyar teńlemesi menen berilgen iymek sızıqtıń tipin anıqlań: $\rho=\frac{6}{1-\cos 0}$.\\

4. $\rho = \frac{5}{3 - 4cos\theta}$ teńlemesi menen qanday sızıq berilgenin hám yarım kósherlerin tabıń.  \\

5. $2x^{2} + 3y^{2} + 8x - 6y + 11 = 0$ teńlemesin ápiwaylastırıń qanday geometriyalıq obrazdı anıqlaytuǵının tabıń hám grafigin jasań.
\end{tabular}
\vspace{1cm}


\begin{tabular}{m{17cm}}
\textbf{86-variant}\\
1. Ellipslik paraboloid (parabola, kósher, ellipslik paraboloid).\\

2. Giperbolanıń urınbasınıń teńlemesi (giperbolaǵa berilgen noqatta júrgizilgen urınba teńlemesi).\\

3. Tipin anıqlań: $9 x^{2}+4 y^{2}+18 x-8 y+49=0$.\\

4. $x^{2} - y^{2} = 27$ giperbolasına $4x + 2y - 7 = 0$ tuwrısına parallel bolǵan urınbanıń teńlemesin tabıń.  \\

5. Fokusı $F( - 1; - 4)$ noqatında jaylasqan, sáykes direktrisası $x - 2 = 0$ teńlemesi menen berilgen, $A( - 3; - 5)$ noqatınan ótiwshi ellipstiń teńlemesin dúziń.  
\end{tabular}
\vspace{1cm}


\begin{tabular}{m{17cm}}
\textbf{87-variant}\\
1. ETIS-tıń invariantları (ETIS-tıń ulıwma teńlemesi, túrlendiriw, ETIS invariantları ).\\

2. Ekinshi tártipli aylanba betlikler (koordinata sisteması, tegislik, vektor iymek sızıq, aylanba betlik).\\

3. Sheńber teńlemesin dúziń: orayı koordinata basında jaylasqan hám $3 x-4 y+20=0$ tuwrı sızıǵına urınadı.\\

4. ETİS-tıń ulıwma teńlemesin koordinata sistemasın túrlendirmey ápiwaylastırıń, tipin anıqlań, obrazı qanday sızıqtı anıqlaytuǵının kórsetiń. $7x^{2} - 8xy + y^{2} - 16x - 2y - 51 = 0$  \\

5. $32x^{2} + 52xy - 7y^{2} + 180 = 0$ ETİS teńlemesin ápiwayı túrge alıp keliń, tipin anıqlań, qanday geometriyalıq obrazdı anıqlaytuǵının kórsetiń, sızılmasın góne hám taza koordinatalar sistemasına qarata jasań.  
\end{tabular}
\vspace{1cm}


\begin{tabular}{m{17cm}}
\textbf{88-variant}\\
1. Parabolanıń polyar koordinatalardaǵı teńlemesi (polyar koordinata sistemasında parabolanıń teńlemesi).\\

2. Bir gewekli giperboloid. Kanonikalıq teńlemesi (giperbolanı simmetriya kósheri átirapında aylandırıwdan alınǵan betlik).\\

3. Fokusları abscissa kósherinde hám koordinata basına qarata simmetriyalıq jaylasqan ellipstiń teńlemesin dúziń: direktrisaları arasındaǵı aralıq $5$ hám fokusları arasındaǵı aralıq $2 c=4$.\\

4. $3x + 10y - 25 = 0$ tuwrı menen $\frac{x^{2}}{25} + \frac{y^{2}}{4} = 1$ ellipstiń kesilisiw noqatların tabıń.\\

5. Úlken kósheri 26 ǵa, fokusları $F( - 10;0)$, $F(14;0)$ noqatlarında jaylasqan ellipstiń teńlemesin dúziń.  
\end{tabular}
\vspace{1cm}


\begin{tabular}{m{17cm}}
\textbf{89-variant}\\
1. Ellipstiń polyar koordinatalardaǵı teńlemesi (polyar koordinatalar sistemasında ellipstiń teńlemesi).\\

2. Cilindrlik betlikler (jasawshı tuwrı sızıq, baǵıtlawshı iymek sızıq, cilindrlik betlik).\\

3. Parabola teńlemesi berilgen: $y^2=6 x$. Onıń polyar teńlemesin dúziń.\\

4. $\rho = \frac{10}{2 - cos\theta}$ polyar teńlemesi menen qanday sızıq berilgenin anıqlań.  \\

5. $4x^{2} - 4xy + y^{2} - 6x + 8y + 13 = 0$ ETİS-ǵı orayǵa iyeme? Orayǵa iye bolsa orayın anıqlań: jalǵız orayǵa iyeme-?, sheksiz orayǵa iyeme-?  
\end{tabular}
\vspace{1cm}


\begin{tabular}{m{17cm}}
\textbf{90-variant}\\
1. Giperbola. Kanonikalıq teńlemesi (fokuslar, kósherler, direktrisalar, giperbola, ekscentrisitet, kanonikalıq teńlemesi).\\

2. Betliktiń kanonikalıq teńlemeleri. Betlik haqqında túsinik. (Betliktiń anıqlaması, formulaları, kósher, baǵıtlawshı tuwrılar).\\

3. Tipin anıqlań: $9 x^{2}-16 y^{2}-54 x-64 y-127=0$.\\

4. $\frac{x^{2}}{20} - \frac{y^{2}}{5} = 1$ giperbolasına $4x + 3y - 7 = 0$ tuwrısına perpendikulyar bolǵan urınbanıń teńlemesin dúziń.  \\

5. $2x^{2} + 3y^{2} + 8x - 6y + 11 = 0$ teńlemesi menen qanday tiptegi sızıq berilgenin anıqlań hám onıń teńlemesin ápiwaylastırıń hám grafigin jasań.  
\end{tabular}
\vspace{1cm}


\begin{tabular}{m{17cm}}
\textbf{91-variant}\\
1. Parabolanıń urınbasınıń teńlemesi (parabola, tuwrı, urınıw noqatı, urınba teńlemesi).\\

2. Betlik haqqında túsinik (tuwrı, iymek sızıq, betliktiń anıqlamaları hám formulaları).\\

3. Sheńberdiń $C$ orayı hám $R$ radiusın tabıń: $x^2+y^2-2 x+4 y-14=0$.\\

4. Koordinata kósherlerin túrlendirmey ETİS teńlemesin ápiwaylastırıń, yarım kósherlerin tabıń $41x^{2} + 2xy + 9y^{2} - 26x - 18y + 3 = 0$.  \\

5. Fokusı $F(2; - 1)$ noqatında jaylasqan, sáykes direktrisası $x - y - 1 = 0$ teńlemesi menen berilgen parabolanıń teńlemesin dúziń.  
\end{tabular}
\vspace{1cm}


\begin{tabular}{m{17cm}}
\textbf{92-variant}\\
1. Koordinata sistemasın túrlendiriw (birlik vektorlar, kósherler, parallel kóshiriw, koordinata kósherlerin burıw).\\

2. ETIS-tıń ulıwma teńlemesin klassifikatsiyalaw (ETIS-tıń ulıwma teńlemesi, ETIS-tıń ulıwma teńlemesin ápiwaylastırıw, klassifikatsiyalaw).\\

3. Fokusları abscissa kósherinde hám koordinata basına qarata simmetriyalıq jaylasqan giperbolanıń teńlemesin dúziń: direktrisaları arasındaǵı aralıq $8/3$ hám ekscentrisitet $\varepsilon=3/2$.\\

4. $\rho = \frac{144}{13 - 5cos\theta}$ ellipsti anıqlaytuǵının kórsetiń hám onıń yarım kósherlerin anıqlań.\\

5. $32x^{2} + 52xy - 9y^{2} + 180 = 0$ ETİS teńlemesin ápiwaylastırıń, tipin anıqlań, qanday geometriyalıq obrazdı anıqlaytuǵının kórsetiń, sızılmasın sızıń.  
\end{tabular}
\vspace{1cm}


\begin{tabular}{m{17cm}}
\textbf{93-variant}\\
1. Giperbolalıq paraboloydtıń tuwrı sızıqlı jasawshıları (Giperbolalıq paraboloydtı jasawshı tuwrı sızıqlar dástesi).\\

2. Giperbolanıń polyar koordinatadaǵı teńlemesi (Polyar múyeshi, polyar radiusi giperbolanıń polyar teńlemesi).\\

3. Polyar teńlemesi menen berilgen iymek sızıqtıń tipin anıqlań: $\rho=\frac{1}{3-3\cos\theta}$.\\

4. $2x + 2y - 3 = 0$ tuwrısına parallel bolıp $\frac{x^{2}}{16} + \frac{y^{2}}{64} = 1$ giperbolasına urınıwshı tuwrınıń teńlemesin dúziń.  \\

5. $A(\frac{10}{3};\frac{5}{3})$ noqattan $\frac{x^{2}}{20} + \frac{y^{2}}{5} = 1$ ellipsine júrgizilgen urınbalardıń teńlemesin dúziń.  
\end{tabular}
\vspace{1cm}


\begin{tabular}{m{17cm}}
\textbf{94-variant}\\
1. ETIS-tıń orayın anıqlaw forması (ETIS-tıń ulıwma teńlemesi, orayın anıqlaw forması).\\

2. Ekinshi tártipli betliktiń ulıwma teńlemesi. Orayın anıqlaw formulası.\\

3. Tipin anıqlań: $3 x^{2}-8 xy+7 y^{2}+8 x-15 y+20=0$.\\

4. Koordinata kósherlerin túrlendirmey ETİS teńlemesin ápiwaylastırıń, qanday geometriyalıq obrazdı anıqlaytuǵının kórsetiń $4x^{2} - 4xy + y^{2} + 4x - 2y + 1 = 0$.  \\

5. $M(2; - \frac{5}{3})$ noqatı $\frac{x^{2}}{9} + \frac{y^{2}}{5} = 1$ ellipsinde jaylasqan. $M$ noqatınıń fokal radiusları jatıwshı tuwrı sızıq teńlemelerin dúziń.  
\end{tabular}
\vspace{1cm}


\begin{tabular}{m{17cm}}
\textbf{95-variant}\\
1. Parabola hám onıń kanonikalıq teńlemesi (anıqlaması, fokusı, direktrisası, kanonikalıq teńlemesi).\\

2. ETIS-tıń ulıwma teńlemesin koordinata kósherlerin burıw arqalı ápiwaylastırıń (ETIS-tıń ulıwma teńlemeleri, koordinata kósherin burıw formulası, teńlemeni kanonik túrge alıp keliw).\\

3. Sheńber teńlemesin dúziń: orayı $C (6 ;-8) $ noqatında jaylasqan hám koordinata basınan ótedi.\\

4. $y^{2} = 12x$ paraborolasına $3x - 2y + 30 = 0$ tuwrı sızıǵına parallel bolǵan urınbanıń teńlemesin dúziń.  \\

5. Eger waqıttıń qálegen momentinde $M(x;y)$ noqat $5x - 16 = 0$ tuwrı sızıqqa qaraǵanda $A(5;0)$ noqattan 1,25 márte uzaqlıqta jaylasqan. Usı $M(x;y)$ noqattıń háreketiniń teńlemesin dúziń.  
\end{tabular}
\vspace{1cm}


\begin{tabular}{m{17cm}}
\textbf{96-variant}\\
1. Eki gewekli giperboloid. Kanonikalıq teńlemesi (giperbolanı simmetriya kósheri átirapında aylandırıwdan alınǵan betlik).\\

2. Ellipstiń urınbasınıń teńlemesi (ellips, tuwrı, urınıw tochka, urınba teńlemesi).\\

3. Uchı koordinata basında jaylasqan hám $Oy$ kósherine qarata shep táreptegi yarım tegislikte jaylasqan parabolanıń teńlemesin dúziń: parametri $p=0,5$.\\

4. Koordinata kósherlerin túrlendirmey ETİS teńlemesin ápiwaylastırıń, yarım kósherlerin tabıń $4x^{2} - 4xy + 9y^{2} - 26x - 18y + 3 = 0$.\\

5. $4x^{2} - 4xy + y^{2} - 2x - 14y + 7 = 0$ ETİS teńlemesin ápiwayı túrge alıp keliń, tipin anıqlań, qanday geometriyalıq obrazdı anıqlaytuǵının kórsetiń, sızılmasın góne hám taza koordinatalar sistemasına qarata jasań.  
\end{tabular}
\vspace{1cm}


\begin{tabular}{m{17cm}}
\textbf{97-variant}\\
1. ETIS -tiń ulıwma teńlemesin ápiwaylastırıw (ETIS -tiń ulıwma teńlemesi, koordinata sistemasın túrlendirip ETIS ulıwma teńlemesin ápiwaylastırıw).\\

2. Ellipsoida. Kanonikalıq teńlemesi (ellipsti simmetriya kósheri dogereginde aylandırıwdan alınǵan betlik, kanonikalıq teńlemesi).\\

3. Giperbola teńlemesi berilgen: $\frac{x^{2}}{25}-\frac{y^{2}}{144}=1$. Onıń polyar teńlemesin dúziń.\\

4. $2x + 2y - 3 = 0$ tuwrısına perpendikulyar bolıp $x^{2} = 16y$ parabolasına urınıwshı tuwrınıń teńlemesin dúziń.  \\

5. $\frac{x^{2}}{3} - \frac{y^{2}}{5} = 1$ giperbolasına $P(1; - 5)$ noqatında júrgizilgen urınbalardıń teńlemesin dúziń.
\end{tabular}
\vspace{1cm}


\begin{tabular}{m{17cm}}
\textbf{98-variant}\\
1. Ellips hám onıń kanonikalıq teńlemesi (anıqlaması, fokuslar, ellipstiń kanonikalıq teńlemesi, ekscentrisiteti, direktrisaları).\\

2. ETIS-tıń ulıwma teńlemesin koordinata basın parallel kóshiriw arqalı ápiwayılastırıń (ETIS- tıń ulıwma teńlemesin parallel kóshiriw formulası).\\

3. Tipin anıqlań: $5 x^{2}+14 xy+11 y^{2}+12 x-7 y+19=0$.\\

4. $x^{2} + 4y^{2} = 25$ ellipsi menen $4x - 2y + 23 = 0$ tuwrı sızıǵına parallel bolǵan urınba tuwrı sızıqtıń teńlemesin dúziń.  \\

5. $y^{2} = 20x$ parabolasınıń $M$ noqatın tabıń, eger onıń abscissası 7 ge teń bolsa, fokal radiusın hám fokal radius jaylasqan tuwrını anıqlań.
\end{tabular}
\vspace{1cm}


\begin{tabular}{m{17cm}}
\textbf{99-variant}\\
1. Ellipslik paraboloid (parabola, kósher, ellipslik paraboloid).\\

2. Giperbolanıń urınbasınıń teńlemesi (giperbolaǵa berilgen noqatta júrgizilgen urınba teńlemesi).\\

3. Sheńber teńlemesin dúziń: $A (3;1) $ hám $B (-1;3) $ noqatlardan ótedi, orayı $3 x-y-2=0$ tuwrı sızıǵında jaylasqan .\\

4. $\frac{x^{2}}{4} - \frac{y^{2}}{5} = 1$ giperbolaǵa $3x - 2y = 0$ tuwrısına parallel bolǵan urınbanıń teńlemesin dúziń.  \\

5. Fokuslari $F(3;4), F(-3;-4)$ noqatlarında jaylasqan direktrisaları orasıdaǵı aralıq 3,6 ǵa teń bolǵan giperbolanıń teńlemesin dúziń.  
\end{tabular}
\vspace{1cm}


\begin{tabular}{m{17cm}}
\textbf{100-variant}\\
1. ETIS-tıń invariantları (ETIS-tıń ulıwma teńlemesi, túrlendiriw, ETIS invariantları ).\\

2. Ekinshi tártipli aylanba betlikler (koordinata sisteması, tegislik, vektor iymek sızıq, aylanba betlik).\\

3. Uchı koordinata basında jaylasqan hám $Ox$ kósherine qarata joqarı yarım tegislikte jaylasqan parabolanıń teńlemesin dúziń: parametri $p=1/4$.\\

4. Ellips $3x^{2} + 4y^{2} - 12 = 0$ teńlemesi menen berilgen. Onıń kósherleriniń uzınlıqların, fokuslarınıń koordinataların hám ekscentrisitetin tabıń.  \\

5. $14x^{2} + 24xy + 21y^{2} - 4x + 18y - 139 = 0$ iymek sızıǵınıń tipin anıqlań, eger oraylı iymek sızıq bolsa orayınıń koordinataların tabıń.  
\end{tabular}
\vspace{1cm}



\end{document}
