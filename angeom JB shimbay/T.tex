Ellips hám onıń kanonikalıq teńlemesi (anıqlaması, fokuslar, ellepstiń kanonikalıq teńlemesi, ekscentrisiteti, direktrisaları).
Ellipstiń polyar koordinatalardaǵı teńlemesi (polyar koordinatalar sistemasında ellipstiń teńlemesi).
Ellipstiń urınbasınıń teńlemesi (ellips, tuwrı, urınıw tochka, urınba teńlemesi).
Giperbola. Kanonikalıq teńlemesi (fokuslar, kósherler, direktrisalar, giperbola, ekscentrisitet, kanonikalıq teńlemesi).
Giperbolanıń urınbasınıń teńlemesi (giperbolaǵa berilgen noqatta júrgizilgen urınba teńlemesi).
Giperbolanıń polyar koordinatadaǵı teńlemesi (Polyar múyeshi, polyar radiusi giperbolanıń polyar teńlemesi).
Parabolanıń urınbasınıń teńlemesi (parabola, tuwrı, urınıw noqatı, urınba teńlemesi).
Parabolanıń polyar koordinatalardaǵı teńlemesi (polyar koordinata sistemasında parabolanıń teńlemesi).
Parabola hám onıń kanonikalıq teńlemesi (anıqlaması, fokusı, direktrisası, kanonikalıq teńlemesi).
Koordinata sistemasın túrlendiriw (birlik vektorlar, kósherler, parallel kóshiriw, koordinata kósherlerin burıw).

ETIS-tıń ulıwma teńlemesin klassifikatsiyalaw (ETIS-tıń ulıwma teńlemesi, ETIS-tıń ulıwma teńlemesin ápiwaylastırıw, klassifikatsiyalaw).
ETIS -tiń ulıwma teńlemesin ápiwaylastırıw (ETIS -tiń ulıwma teńlemesi, koordinata sistemasın túrlendirip ETIS ulıwma teńlemesin ápiwaylastırıw).
ETIS-tıń orayın anıqlaw forması (ETIS-tıń ulıwma teńlemesi, orayın anıqlaw forması).
ETIS-tıń ulıwma teńlemesin koordinata kósherlerin burıw arqalı ápiwaylastırıń (ETIS-tıń ulıwma teńlemeleri, koordinata kósherin burıw formulası, teńlemeni kanonik túrge alıp keliw).
ETIS-tıń ulıwma teńlemesin koordinata basın parallel kóshiriw arqalı ápiwayılastırıń (ETIS- tıń ulıwma teńlemesin parallel kóshiriw formulası).
ETIS-tıń invariantları (ETIS-tıń ulıwma teńlemesi, túrlendiriw, ETIS invariantları ).

Betlik haqqında túsinik (tuwrı, iymek sızıq, betliktiń anıqlamaları hám formulaları).
Cilindrlik betlikler (jasawshı tuwrı sızıq, baǵıtlawshı iymek sızıq, cilindrlik betlik).
Ellipsoida. Kanonikalıq teńlemesi (ellipsti simmetriya kósheri dogereginde aylandırıwdan alınǵan betlik, kanonikalıq teńlemesi).
Bir gewekli giperboloid. Kanonikalıq teńlemesi (giperbolanı simmetriya kósheri átirapında aylandırıwdan alınǵan betlik).
Eki gewekli giperboloid. Kanonikalıq teńlemesi (giperbolanı simmetriya kósheri átirapında aylandırıwdan alınǵan betlik).
Giperbolalıq paraboloydtıń tuwrı sızıqlı jasawshıları (Giperbolalıq paraboloydtı jasawshı tuwrı sızıqlar dástesi).
Ekinshi tártipli betliktiń ulıwma teńlemesi. Orayın anıqlaw formulası.
Ekinshi tártipli aylanba betlikler (koordinata sisteması, tegislik, vektor iymek sızıq, aylanba betlik).
Ellipslik paraboloid (parabola, kósher, ellepslik paraboloid).
Betliktiń kanonikalıq teńlemeleri. Betlik haqqında túsinik. (Betliktiń anıqlaması, formulaları, kósher, baǵıtlawshı tuwrılar).
