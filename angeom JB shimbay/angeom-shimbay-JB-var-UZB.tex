\documentclass{article}
  \usepackage[utf8]{inputenc}
  \usepackage[T2A]{fontenc}
  \usepackage{array}
  \usepackage[a4paper,
  left=15mm,
  top=15mm,]{geometry}
  \usepackage{setspace}
  
  \renewcommand{\baselinestretch}{1.1} 
  
\begin{document}

\large
\pagenumbering{gobble}


\begin{tabular}{m{17cm}}
\textbf{1-variant}\\
1. Ellipsning polyar koordinatalardagi tenglamasi (polyar koordinatalar sistemasida ellipsning tenglamasi).\\

2. ITECH-ning umumiy tenglamasini koordinata boshin parallel ko'chirish bilan soddalastiring (ITECH-ning umumiy tenglamasini parallel ko'chirish formulasi).\\

3. Aylana tenglamasini tuzing: markazi $C(1;-1)$ nuqtada joylashgan va $5x-12y+9-0$ to'g'ri chiziqga urinadi.\\

4. $\frac{x^{2}}{16} - \frac{y^{2}}{64} = 1$ giperbolasiga berilgan $10x - 3y + 9 = 0$ to'g'ri chizig'iga parallel bo'lgan urinmasining tenglamasini tuzing.  \\

5. Agar xohlagan vaqt momentida $M(x;y)$ nuqta $A(8;4)$ nuqtasidan va ordinata o'qidan birxil masofada joylashsa, $M(x;y)$ nuqtaning harakat troektoriyasining tenglamasini tuzing.  
\end{tabular}
\vspace{1cm}


\begin{tabular}{m{17cm}}
\textbf{2-variant}\\
1. Elliptik paraboloid (parabola, o'q, elliptik paraboloid).\\

2. Ellipsning urinmasining tenglamasi (ellips, to'g'ri chiziq urinish nuqtasi, urinma tenglamasi).\\

3. Fokuslari abssissa o'qida va koordinata boshiga nisbatan simmetrik joylashgan ellipsning tenglamasini tuzing: kichik o'qi $10$, ekssentrisitet $\varepsilon=12/13$.\\

4. ITECH ning umumiy tenglamasini koordinata sistemasini almashtirmasdan soddalashtiring, tipini aniqlang, obrazi qanday chiziq ekanligini ko'rsating: $7x^{2} - 8xy + y^{2} - 16x - 2y - 51 = 0$\\

5. $16x^{2} - 9y^{2} - 64x - 54y - 161 = 0$ tenglamasi giperbolaning tenglamasi ekanligini ko'rsating va uning markazi $C$ ni, yarim o'qlarini, ekssentrisitetini toping, asimptotalarining tenglamalarini tuzing.  
\end{tabular}
\vspace{1cm}


\begin{tabular}{m{17cm}}
\textbf{3-variant}\\
1. ITECH-ning umumiy tenglamasini klassifikatsiyalash (ITECH-ning umumiy tenglamasi, ITECH-ning umumiy tenglamasini soddalashtirish, klassifikatsiyalash).\\

2. Ikkinshi tartibli aylanma sirtlar (koordinata sistemasi, tekislik, vektor egri chiziq, aylanma sirt).\\

3. Giperbola tenglamasi berilgan: $\frac{x^{2}}{25}-\frac{y^{2}}{144}=1$. Uning qutb tenglamasini tuzing.\\

4. $x^{2} - 4y^{2} = 16$ giperbola berilgan. Uning ekssentrisitetini, fokuslarining koordinatalarini toping va asimptotalarining tenglamalarini tuzing.\\

5. $\frac{x^{2}}{25} + \frac{y^{2}}{16} = 1$, ellipsiga $C(10; - 8)$ nuqtadan yurgizilgan urinmalarining tenglamasini tuzing.  
\end{tabular}
\vspace{1cm}


\begin{tabular}{m{17cm}}
\textbf{4-variant}\\
1. Parabolaning urinmasining tenglamasi (parabola, to'g'ri chiziq urinish nuqtasi, urinma tenglamasi).\\

2. ITECH-ning umumiy tenglamasini soddalashtirish (ITECH-ning umumiy tenglamasi, koordinata sistemasin almashtirish ITECH umumiy tenglamasini soddalashtirish).\\

3. Tipini aniqlang: $25x^{2}-20xy+4y^{2}-12x+20y-17=0$.\\

4. $3x + 4y - 12 = 0$ to'g'ri chizig'i bilan $y^{2} = - 9x$ parabolasining kesishish nuqtalarini toping.  \\

5. $y^{2} = 20x$ parabolasining $M$ nuqtasini toping, agar uning abssissasi 7 ga teng bo'lsa, fokal radiusini va fokal radiusi joylashgan to'g'rini aniqlang.
\end{tabular}
\vspace{1cm}


\begin{tabular}{m{17cm}}
\textbf{5-variant}\\
1. Silindrlik sirtlar (yasovchi to'g'ri chiziq, yo'naltiruvchi egri chiziq, silindrlik sirt).\\

2. Giperbola. Kanonik tenglamasi (fokuslar, o'qlar, direktrisalar, giperbola, ekstsentrisitet, kanonik tenglamasi).\\

3. Aylana tenglamasini tuzing: markazi $C(6;-8)$ nuqtada joylashgan va koordinata boshidan o'tadi.\\

4. $\rho = \frac{6}{1 - cos\theta}$ polyar tenglamasi bilan qanday chiziq berilganini aniqlang.  \\

5. Agar vaqtning xohlagan momentida $M(x;y)$ nuqta $5x - 16 = 0$ to'g'ri chiziqqa qaraganda $A(5;0)$ nuqtasidan 1,25 marta uzoqroq masofada joylashgan. Shu $M(x;y)$ nuqtaning harakatining tenglamasini tuzing.  
\end{tabular}
\vspace{1cm}


\begin{tabular}{m{17cm}}
\textbf{6-variant}\\
1. Koordinata sistemasini almashtirish (birlik vektorlar, o'qlar, parallel ko'chirish, koordinata o'qlarinii burish).\\

2. Sirtning kanonik tenglamalari. Sirt haqqida tushuncha. (Sirtning ta'rifi, formulalari, o'q, yo'naltiruvchi to'g'ri chiziqlar).\\

3. Fokuslari abssissa o'qida va koordinata boshiga nisbatan simmetrik joylashgan giperbolaning tenglamasini tuzing: fokuslari orasidagi masofasi $2c=10$ va o'qi $2b=8$.\\

4. $x^{2} - y^{2} = 27$ giperbolasiga $4x + 2y - 7 = 0$ to'g'ri chizigiga parallel bo'lgan urinmasining tenglamasini toping.  \\

5. $14x^{2} + 24xy + 21y^{2} - 4x + 18y - 139 = 0$ egri chizig'ining tipini aniqlang, agar markazga ega egri chiziq bo'lsa, markazining koordinatalarini toping.  
\end{tabular}
\vspace{1cm}


\begin{tabular}{m{17cm}}
\textbf{7-variant}\\
1. Ellips va uning kanonik tenglamasi (ta'rifi, fokuslari, ellipsning kanonik tenglamasi, ekstsentrisiteti, direktrisalari).\\

2. ITECH-ning markazini aniqlash formulasi (ITECH-ning umumiy tenglamasi, markazini aniqlash formulasi).\\

3. Qutb tenglamasi bilan berilgan egri chiziqning tipini aniqlang: $\rho=\frac{10}{1-\frac{3}{2}\cos\theta}$.\\

4. Koordinata o'qlarini almashtirmasdan ITECH tenglamasini soddalashtiring, qanday geometrik obraz ekanligini ko'rsating: $4x^{2} - 4xy + y^{2} + 4x - 2y + 1 = 0$.  \\

5. $\frac{x^{2}}{3} - \frac{y^{2}}{5} = 1$ giperbolasiga $P(1; - 5)$ nuqtasida yurgizilgan urinmalarning tenglamasini tuzing.
\end{tabular}
\vspace{1cm}


\begin{tabular}{m{17cm}}
\textbf{8-variant}\\
1. Giperbolik paraboloydning to'g'ri chiziq yasovchilari (Giperbolik paraboloydni yasovchi to'g'ri chiziqlar dastasi).\\

2. Parabolaning polyar koordinatalardagi tenglamasi (polyar koordinata sistemasida parabolaning tenglamasi).\\

3. Berilgan chiziqlarning markaziy ekanligini ko'rsating va markazinin toping: $2x^{2}-6xy+5y^{2}+22x-36y+11=0$.\\

4. Ellips $3x^{2} + 4y^{2} - 12 = 0$ tenglamasi bilan berilgan. Uning o'qlarining uzunliklarini, fokuslarining koordinatalarini va ekssentrisitetini toping.  \\

5. $y^{2} = 20x$ parabolasining abssissasi 7 ga teng bo'lgan $M$ nuqtasining fokal radiusini toping va fokal radiusi yotgan to'g'ri chiziqning tenglamasini tuzing.  
\end{tabular}
\vspace{1cm}


\begin{tabular}{m{17cm}}
\textbf{9-variant}\\
1. Ikkinchi tartibli sirtning umumiy tenglamasi. Markazin aniqlash formulasi.\\

2. Ikki pallali giperboloid Kanonik tenglamasi (giperbolani simmetriya o'qi atrofida aylantirishdan olingan sirt).\\

3. Aylananing $C$ markazi va $R$ radiusini toping: $x^2+y^2+4x-2y+5=0$.\\

4. $3x + 4y - 12 = 0$ to'g'ri chizig'i va $y^{2} = - 9x$ parabolasining kesishish nuqtalarini toping.\\

5. Fokusi $F(2; - 1)$ nuqtasida joylashgan, mos direktrisasi $x - y - 1 = 0$ tenglamasi bilan berilgan parabolaning tenglamasini tuzing.  
\end{tabular}
\vspace{1cm}


\begin{tabular}{m{17cm}}
\textbf{10-variant}\\
1. Parabola va uning kanonik tenglamasi ( ta'rifi, fokusi, direktrisasi, kanonik tenglamasi).\\

2. ITECH-ning umumiy tenglamasini koordinata o'qlarini burish bilan soddalashtirish (ITECH-ning umumiy tenglamalari, koordinata o'qin burish formulasi, tenglamani kanonik turga olib kelish).\\

3. Uchi koordinata boshida joylashgan va $Ox$ o'qiga nisbatan o'ng tarafafgi yarim tekislikda joylashgan parabolaning tenglamasini tuzing: parametri $p=3$.\\

4. $\rho = \frac{144}{13 - 5cos\theta}$ ellips ekanligini ko'rsating va uning yarim o'qlarini aniqlang.\\

5. $2x^{2} + 3y^{2} + 8x - 6y + 11 = 0$ tenglamasi bilan qanday tipdagi chiziq berilganini aniqlang va uning tenglamasini soddalashtiring va grafigini chizing.  
\end{tabular}
\vspace{1cm}


\begin{tabular}{m{17cm}}
\textbf{11-variant}\\
1. Bir pallali giperboloid. Kanonik tenglamasi (giperbolani simmetriya o'qi atrofida aylantirishdan olingan sirt).\\

2. Giperbolaning polyar koordinatadagi tenglamasi (Polyar burchagi, polyar radiusi giperbolaning polyar tenglamasi)\\

3. Qutb tenglamasi bilan berilgan egri chiziqning tipini aniqlang: $\rho=\frac{6}{1-\cos 0}$.\\

4. $2x + 2y - 3 = 0$ to'g'ri chizig'iga perpendikulyar bo'lib $x^{2} = 16y$ parabolasiga urinib o'tuvchi to'g'ri chiziqning tenglamasini tuzing.  \\

5. $A(\frac{10}{3};\frac{5}{3})$ nuqtasidan $\frac{x^{2}}{20} + \frac{y^{2}}{5} = 1$ ellipsiga yurgizilgan urinmalarning tenglamasini tuzing.  
\end{tabular}
\vspace{1cm}


\begin{tabular}{m{17cm}}
\textbf{12-variant}\\
1. ITECH-ning invariantlari (ITECH-ning umumiy tenglamasi, almashtirish, ITECH invariantlari).\\

2. Ellipsoida. Kanonik tenglamasi (ellipsni simmetriya o'qi atrofida aylantirishdan olingan sirt, kanonik tenglamasi).\\

3. Berilgan chiziqlarning markaziy ekanligini ko'rsating va markazinin toping: $9x^{2}-4xy-7y^{2}-12=0$.\\

4. $41x^{2} + 24xy + 9y^{2} + 24x + 18y - 36 = 0$ ITECH tipini aniqlang va markazlarini toping koordinata o'qlarini almashtirmasdan qanday chiziq ekanligini ko'rsating, yarim o'qlarini toping.  \\

5. $\frac{x^{2}}{100} + \frac{y^{2}}{36} = 1$ ellipsining o'ng tarafdagi fokusidan 14 ga teng masofada bo'lgan nuqtasini toping.  
\end{tabular}
\vspace{1cm}


\begin{tabular}{m{17cm}}
\textbf{13-variant}\\
1. Elliptik paraboloid (parabola, o'q, elliptik paraboloid).\\

2. Giperbolaning urinmasining tenglamasi (giperbolaga berilgan nuqtada yurgizilgan urinma tenglamasi).\\

3. Aylana tenglamasini tuzing: aylana diametrining uchlari $A(3;2)$ va $B(-1;6)$ nuqtalarda joylashgan.\\

4. $y^{2} = 3x$ parabolasi bilan $\frac{x^{2}}{100} + \frac{y^{2}}{225} = 1$ ellipsining kesishish nuqtalarini toping.  \\

5. Fokusi $F( - 1; - 4)$ nuqtasida joylashgan, mos direktrisasi $x - 2 = 0$ tenglamasi bilan berilgan, $A( - 3; - 5)$ nuqtadan o'tuvchi ellipsning tenglamasini tuzing.  
\end{tabular}
\vspace{1cm}


\begin{tabular}{m{17cm}}
\textbf{14-variant}\\
1. ITECH-ning umumiy tenglamasini koordinata boshin parallel ko'chirish bilan soddalastiring (ITECH-ning umumiy tenglamasini parallel ko'chirish formulasi).\\

2. Ikkinshi tartibli aylanma sirtlar (koordinata sistemasi, tekislik, vektor egri chiziq, aylanma sirt).\\

3. Uchi koordinata boshida joylashgan va $Oy$ o'qiga nisbatan quyi tarafafgi yarim tekislikda joylashgan parabolaning tenglamasini tuzing: parametri $p=3$.\\

4. $\rho = \frac{10}{2 - cos\theta}$ polyar tenglamasi bilan qanday chiziq berilganini aniqlang.  \\

5. $4x^{2} + 24xy + 11y^{2} + 64x + 42y + 51 = 0$ egri chizig'ining tipini aniqlang, agar markazga ega bo'lsa, uning markazining koordinatalarini toping va koordinata boshini markazga parallel ko'chirish amalini bajaring.
\end{tabular}
\vspace{1cm}


\begin{tabular}{m{17cm}}
\textbf{15-variant}\\
1. Ellipsning polyar koordinatalardagi tenglamasi (polyar koordinatalar sistemasida ellipsning tenglamasi).\\

2. ITECH-ning umumiy tenglamasini klassifikatsiyalash (ITECH-ning umumiy tenglamasi, ITECH-ning umumiy tenglamasini soddalashtirish, klassifikatsiyalash).\\

3. Qutb tenglamasi bilan berilgan egri chiziqning tipini aniqlang: $\rho=\frac{12}{2-\cos\theta}$.\\

4. $2x + 2y - 3 = 0$ to'g'ri chizig'iga parallel bo'lib $\frac{x^{2}}{16} + \frac{y^{2}}{64} = 1$ giperbolasiga urinib o'tuvchi to'g'ri chiziqning tenglamasini tuzing.  \\

5. $\frac{x^{2}}{3} - \frac{y^{2}}{5} = 1$, giperbolasiga $P(4;2)$ nuqtadan yurgizilgan urinmalarning tenglamasini tuzing.  
\end{tabular}
\vspace{1cm}


\begin{tabular}{m{17cm}}
\textbf{16-variant}\\
1. Silindrlik sirtlar (yasovchi to'g'ri chiziq, yo'naltiruvchi egri chiziq, silindrlik sirt).\\

2. Ellipsning urinmasining tenglamasi (ellips, to'g'ri chiziq urinish nuqtasi, urinma tenglamasi).\\

3. Tipini aniqlang: $3x^{2}-2xy-3y^{2}+12y-15=0$.\\

4. Koordinata o'qlarini almashtirmasdan ITECH tenglamasini soddalashtiring, yarim o'qlarnin toping: $4x^{2} - 4xy + 7y^{2} - 26x - 18y + 3 = 0$.\\

5. $M(2; - \frac{5}{3})$ nuqta $\frac{x^{2}}{9} + \frac{y^{2}}{5} = 1$ ellipsda joylashgan. $M$ nuqtaning fokal radiuslarida yotuvchi to'g'ri chiziq tenglamalarini tuzing.  
\end{tabular}
\vspace{1cm}


\begin{tabular}{m{17cm}}
\textbf{17-variant}\\
1. ITECH-ning umumiy tenglamasini soddalashtirish (ITECH-ning umumiy tenglamasi, koordinata sistemasin almashtirish ITECH umumiy tenglamasini soddalashtirish).\\

2. Sirtning kanonik tenglamalari. Sirt haqqida tushuncha. (Sirtning ta'rifi, formulalari, o'q, yo'naltiruvchi to'g'ri chiziqlar).\\

3. Aylananing $C$ markazi va $R$ radiusini toping: $x^2+y^2-2x+4y-20=0$.\\

4. $3x + 10y - 25 = 0$ to'g'ri bilan $\frac{x^{2}}{25} + \frac{y^{2}}{4} = 1$ ellipsning kesishish nuqtalarini toping.  \\

5. Fokuslari $F(3;4)$, $F(-3;-4)$ nuqtalarida joylashgan direktrisalari orasidagi masofa 3,6 ga teng bo'lgan giperbolaning tenglamasini tuzing.  
\end{tabular}
\vspace{1cm}


\begin{tabular}{m{17cm}}
\textbf{18-variant}\\
1. Parabolaning urinmasining tenglamasi (parabola, to'g'ri chiziq urinish nuqtasi, urinma tenglamasi).\\

2. Koordinata sistemasini almashtirish (birlik vektorlar, o'qlar, parallel ko'chirish, koordinata o'qlarinii burish).\\

3. Uchi koordinata boshida joylashgan va $Oy$ o'qiga nisbatan yuqori yarim tekislikda joylashgan parabolaning tenglamasini tuzing: parametri $p=1/4$.\\

4. $\rho = \frac{5}{3 - 4cos\theta}$ tenglamasi bilan qanday chiziq berilganini va yarim o'qlarini toping.  \\

5. $4x^{2} - 4xy + y^{2} - 2x - 14y + 7 = 0$ ITECH tenglamasini kanonik shaklga olib keling, tipini aniqlang, qanday geometrik obraz ekanligini ko'rsating, chizmasini eski va yangi koordinatalar sistemasiga nisbatan chizing.  
\end{tabular}
\vspace{1cm}


\begin{tabular}{m{17cm}}
\textbf{19-variant}\\
1. Giperbolik paraboloydning to'g'ri chiziq yasovchilari (Giperbolik paraboloydni yasovchi to'g'ri chiziqlar dastasi).\\

2. Giperbola. Kanonik tenglamasi (fokuslar, o'qlar, direktrisalar, giperbola, ekstsentrisitet, kanonik tenglamasi).\\

3. Qutb tenglamasi bilan berilgan egri chiziqning tipini aniqlang: $\rho=\frac{5}{3-4\cos\theta}$.\\

4. $x^{2} + 4y^{2} = 25$ ellipsi bilan $4x - 2y + 23 = 0$ to'g'ri chizig'iga parallel bo'lgan urinma to'g'ri chiziqning tenglamasini tuzing.  \\

5. Uchi (-4;0) nuqtasinda, direktrisasi $y - 2 = 0$ to'g'ri chiziq bo'lgan parabolaning tenglamasini tuzing.
\end{tabular}
\vspace{1cm}


\begin{tabular}{m{17cm}}
\textbf{20-variant}\\
1. ITECH-ning markazini aniqlash formulasi (ITECH-ning umumiy tenglamasi, markazini aniqlash formulasi).\\

2. Ikki pallali giperboloid Kanonik tenglamasi (giperbolani simmetriya o'qi atrofida aylantirishdan olingan sirt).\\

3. Tipini aniqlang: $x^{2}-4xy+4y^{2}+7x-12=0$.\\

4. Koordinata o'qlarini almashtirmasdan ITECH umumiy tenglamasini soddalashtiring, yarim o'qlarini toping: $13x^{2} + 18xy + 37y^{2} - 26x - 18y + 3 = 0$.  \\

5. $4x^{2} - 4xy + y^{2} - 6x + 8y + 13 = 0$ ITECH markazga egami? Markazga ega bo'lsa markazini aniqlang?  
\end{tabular}
\vspace{1cm}


\begin{tabular}{m{17cm}}
\textbf{21-variant}\\
1. Ellips va uning kanonik tenglamasi (ta'rifi, fokuslari, ellipsning kanonik tenglamasi, ekstsentrisiteti, direktrisalari).\\

2. Ikkinchi tartibli sirtning umumiy tenglamasi. Markazin aniqlash formulasi.\\

3. Aylana tenglamasini tuzing: aylana $A(2;6)$ nuqtadan o'tadi va markazi $C(-1;2)$ nuqtada joylashgan.\\

4. $\frac{x^{2}}{4} - \frac{y^{2}}{5} = 1$, giperbolaning $3x - 2y = 0$ to'g'ri chizig'iga parallel bo'lgan urinmasining tenglamasini tuzing.  \\

5. Fokusi $F( - 1; - 4)$ nuqtasida bo'lgan, mos direktrisasi $x - 2 = 0$ tenglamasi bilan berilgan, $A( - 3; - 5)$ nuqtadan o'tuvchi ellipsning tenglamasini tuzing.  
\end{tabular}
\vspace{1cm}


\begin{tabular}{m{17cm}}
\textbf{22-variant}\\
1. Bir pallali giperboloid. Kanonik tenglamasi (giperbolani simmetriya o'qi atrofida aylantirishdan olingan sirt).\\

2. Parabolaning polyar koordinatalardagi tenglamasi (polyar koordinata sistemasida parabolaning tenglamasi).\\

3. Fokuslari abssissa o'qida va koordinata boshiga nisbatan simmetrik joylashgan ellipsning tenglamasini tuzing: kichik o'qi $24$, fokuslari orasidagi masofa $2c=10$.\\

4. $y^{2} = 12x$ paraborolasiga $3x - 2y + 30 = 0$ to'g'ri chizig'iga parallel bo'lgan urinmasining tenglamasini tuzing.  \\

5. $32x^{2} + 52xy - 7y^{2} + 180 = 0$ ITECH tenglamasini kanonik shaklga olib keling, tipini aniqlang, qanday geometrik obraz ekanligini ko'rsating, chizmasini eski va yangi koordinatalar sistemasiga nisbatan chizing.  
\end{tabular}
\vspace{1cm}


\begin{tabular}{m{17cm}}
\textbf{23-variant}\\
1. ITECH-ning umumiy tenglamasini koordinata o'qlarini burish bilan soddalashtirish (ITECH-ning umumiy tenglamalari, koordinata o'qin burish formulasi, tenglamani kanonik turga olib kelish).\\

2. Ellipsoida. Kanonik tenglamasi (ellipsni simmetriya o'qi atrofida aylantirishdan olingan sirt, kanonik tenglamasi).\\

3. Giperbola tenglamasi berilgan: $\frac{x^{2}}{16}-\frac{y^{2}}{9}=1$. Uning qutb tenglamasini tuzing.\\

4. $\frac{x^{2}}{20} - \frac{y^{2}}{5} = 1$ giperbolasiga $4x + 3y - 7 = 0$ to'g'ri chizig'iga perpendikulyar bo'lgan urinmasining tenglamasini tuzing.  \\

5. Fokusi $F(7;2)$ nuqtasida joylashgan, mos direktrisasi $x - 5 = 0$ tenglamasi bilan berilgan parabolaning tenglamasini tuzing.  
\end{tabular}
\vspace{1cm}


\begin{tabular}{m{17cm}}
\textbf{24-variant}\\
1. Parabola va uning kanonik tenglamasi ( ta'rifi, fokusi, direktrisasi, kanonik tenglamasi).\\

2. ITECH-ning invariantlari (ITECH-ning umumiy tenglamasi, almashtirish, ITECH invariantlari).\\

3. Tipini aniqlang: $2x^{2}+3y^{2}+8x-6y+11=0$.\\

4. ITECH ning umumiy tenglamasini koordinata sistemasini almashtirmasdan soddalashtiring, tipini aniqlang, obrazi qanday chiziq ekanligini ko'rsating: $7x^{2} - 8xy + y^{2} - 16x - 2y - 51 = 0$\\

5. Katta o'qi 26 ga, fokuslari $F( - 10;0), F(14;0)$ nuqtalarida joylashgan ellipsning tenglamasini tuzing.  
\end{tabular}
\vspace{1cm}


\begin{tabular}{m{17cm}}
\textbf{25-variant}\\
1. Elliptik paraboloid (parabola, o'q, elliptik paraboloid).\\

2. Ikkinshi tartibli aylanma sirtlar (koordinata sistemasi, tekislik, vektor egri chiziq, aylanma sirt).\\

3. Aylana tenglamasini tuzing: markazi koordinata boshida joylashgan va $3x-4y+20=0$ to'g'ri chiziqga urinadi.\\

4. $\frac{x^{2}}{4} - \frac{y^{2}}{5} = 1$ giperbolasiga $3x + 2y = 0$ to'g'ri chizig'iga perpendikulyar bo'lgan urinma to'g'ri chiziqning tenglamasini tuzing.\\

5. $y^{2} = 20x$ parabolasining $M$ nuqtasini toping, agar uning abssissasi 7 ga teng bo'lsa, fokal radiusini va fokal radiusi joylashgan to'g'rini aniqlang.
\end{tabular}
\vspace{1cm}


\begin{tabular}{m{17cm}}
\textbf{26-variant}\\
1. Giperbolaning polyar koordinatadagi tenglamasi (Polyar burchagi, polyar radiusi giperbolaning polyar tenglamasi)\\

2. ITECH-ning umumiy tenglamasini koordinata boshin parallel ko'chirish bilan soddalastiring (ITECH-ning umumiy tenglamasini parallel ko'chirish formulasi).\\

3. Fokuslari abssissa o'qida va koordinata boshiga nisbatan simmetrik joylashgan ellipsning tenglamasini tuzing: katta o'qi $10$, fokuslari orasidagi masofa $2c=8$.\\

4. Koordinata o'qlarini almashtirmasdan ITECH tenglamasini soddalashtiring, qanday geometrik obraz ekanligini ko'rsating: $4x^{2} - 4xy + y^{2} + 4x - 2y + 1 = 0$.  \\

5. Giperbolaning ekssentrisiteti $\varepsilon = \frac{13}{12}$, fokusi $F(0;13)$ nuqtasida va mos direktrisasi $13y - 144 = 0$ tenglamasi bilan berilgan bo'lsa, giperbolaning tenglamasini tuzing.  
\end{tabular}
\vspace{1cm}


\begin{tabular}{m{17cm}}
\textbf{27-variant}\\
1. Silindrlik sirtlar (yasovchi to'g'ri chiziq, yo'naltiruvchi egri chiziq, silindrlik sirt).\\

2. Giperbolaning urinmasining tenglamasi (giperbolaga berilgan nuqtada yurgizilgan urinma tenglamasi).\\

3. Ellips tenglamasi berilgan: $\frac{x^2}{25}+\frac{y^2}{16}=1$. Uning qutb tenglamasini tuzing.\\

4. $x^{2} - 4y^{2} = 16$ giperbola berilgan. Uning ekssentrisitetini, fokuslarining koordinatalarini toping va asimptotalarining tenglamalarini tuzing.\\

5. $16x^{2} - 9y^{2} - 64x - 54y - 161 = 0$ tenglamasi giperbolaning tenglamasi ekanligini ko'rsating va uning markazi $C$ ni, yarim o'qlarini, ekssentrisitetini toping, asimptotalarining tenglamalarini tuzing.  
\end{tabular}
\vspace{1cm}


\begin{tabular}{m{17cm}}
\textbf{28-variant}\\
1. ITECH-ning umumiy tenglamasini klassifikatsiyalash (ITECH-ning umumiy tenglamasi, ITECH-ning umumiy tenglamasini soddalashtirish, klassifikatsiyalash).\\

2. Sirtning kanonik tenglamalari. Sirt haqqida tushuncha. (Sirtning ta'rifi, formulalari, o'q, yo'naltiruvchi to'g'ri chiziqlar).\\

3. Tipini aniqlang: $2x^{2}+10xy+12y^{2}-7x+18y-15=0$.\\

4. $3x + 4y - 12 = 0$ to'g'ri chizig'i bilan $y^{2} = - 9x$ parabolasining kesishish nuqtalarini toping.  \\

5. $\frac{x^{2}}{25} + \frac{y^{2}}{16} = 1$, ellipsiga $C(10; - 8)$ nuqtadan yurgizilgan urinmalarining tenglamasini tuzing.  
\end{tabular}
\vspace{1cm}


\begin{tabular}{m{17cm}}
\textbf{29-variant}\\
1. Ellipsning polyar koordinatalardagi tenglamasi (polyar koordinatalar sistemasida ellipsning tenglamasi).\\

2. ITECH-ning umumiy tenglamasini soddalashtirish (ITECH-ning umumiy tenglamasi, koordinata sistemasin almashtirish ITECH umumiy tenglamasini soddalashtirish).\\

3. Aylana tenglamasini tuzing: markazi koordinata boshida joylashgan va radiusi $R=3$ ga teng.\\

4. $\rho = \frac{6}{1 - cos\theta}$ polyar tenglamasi bilan qanday chiziq berilganini aniqlang.  \\

5. $y^{2} = 20x$ parabolasining abssissasi 7 ga teng bo'lgan $M$ nuqtasining fokal radiusini toping va fokal radiusi yotgan to'g'ri chiziqning tenglamasini tuzing.  
\end{tabular}
\vspace{1cm}


\begin{tabular}{m{17cm}}
\textbf{30-variant}\\
1. Giperbolik paraboloydning to'g'ri chiziq yasovchilari (Giperbolik paraboloydni yasovchi to'g'ri chiziqlar dastasi).\\

2. Ellipsning urinmasining tenglamasi (ellips, to'g'ri chiziq urinish nuqtasi, urinma tenglamasi).\\

3. Fokuslari abssissa o'qida va koordinata boshiga nisbatan simmetrik joylashgan giperbolaning tenglamasini tuzing: direktrisalar orasidagi masofa $32/5$ va o'qi $2b=6$.\\

4. $\frac{x^{2}}{16} - \frac{y^{2}}{64} = 1$ giperbolasiga berilgan $10x - 3y + 9 = 0$ to'g'ri chizig'iga parallel bo'lgan urinmasining tenglamasini tuzing.  \\

5. Agar xohlagan vaqt momentida $M(x;y)$ nuqta $A(8;4)$ nuqtasidan va ordinata o'qidan birxil masofada joylashsa, $M(x;y)$ nuqtaning harakat troektoriyasining tenglamasini tuzing.  
\end{tabular}
\vspace{1cm}


\begin{tabular}{m{17cm}}
\textbf{31-variant}\\
1. Koordinata sistemasini almashtirish (birlik vektorlar, o'qlar, parallel ko'chirish, koordinata o'qlarinii burish).\\

2. Ikki pallali giperboloid Kanonik tenglamasi (giperbolani simmetriya o'qi atrofida aylantirishdan olingan sirt).\\

3. Parabola tenglamasi berilgan: $y^2=6x$. Uning qutb tenglamasini tuzing.\\

4. $41x^{2} + 24xy + 9y^{2} + 24x + 18y - 36 = 0$ ITECH tipini aniqlang va markazlarini toping koordinata o'qlarini almashtirmasdan qanday chiziq ekanligini ko'rsating, yarim o'qlarini toping.  \\

5. $14x^{2} + 24xy + 21y^{2} - 4x + 18y - 139 = 0$ egri chizig'ining tipini aniqlang, agar markazga ega egri chiziq bo'lsa, markazining koordinatalarini toping.  
\end{tabular}
\vspace{1cm}


\begin{tabular}{m{17cm}}
\textbf{32-variant}\\
1. Parabolaning urinmasining tenglamasi (parabola, to'g'ri chiziq urinish nuqtasi, urinma tenglamasi).\\

2. ITECH-ning markazini aniqlash formulasi (ITECH-ning umumiy tenglamasi, markazini aniqlash formulasi).\\

3. Berilgan chiziqlarning markaziy ekanligini ko'rsating va markazinin toping: $5x^{2}+4xy+2y^{2}+20x+20y-18=0$.\\

4. Ellips $3x^{2} + 4y^{2} - 12 = 0$ tenglamasi bilan berilgan. Uning o'qlarining uzunliklarini, fokuslarining koordinatalarini va ekssentrisitetini toping.  \\

5. $\frac{x^{2}}{3} - \frac{y^{2}}{5} = 1$ giperbolasiga $P(1; - 5)$ nuqtasida yurgizilgan urinmalarning tenglamasini tuzing.
\end{tabular}
\vspace{1cm}


\begin{tabular}{m{17cm}}
\textbf{33-variant}\\
1. Bir pallali giperboloid. Kanonik tenglamasi (giperbolani simmetriya o'qi atrofida aylantirishdan olingan sirt).\\

2. Giperbola. Kanonik tenglamasi (fokuslar, o'qlar, direktrisalar, giperbola, ekstsentrisitet, kanonik tenglamasi).\\

3. Aylana tenglamasini tuzing: markazi $C(2;-3)$ nuqtada joylashgan va radiusi $R=7$ ga teng.\\

4. $3x + 4y - 12 = 0$ to'g'ri chizig'i va $y^{2} = - 9x$ parabolasining kesishish nuqtalarini toping.\\

5. $\frac{x^{2}}{100} + \frac{y^{2}}{36} = 1$ ellipsining o'ng tarafdagi fokusidan 14 ga teng masofada bo'lgan nuqtasini toping.  
\end{tabular}
\vspace{1cm}


\begin{tabular}{m{17cm}}
\textbf{34-variant}\\
1. Ikkinchi tartibli sirtning umumiy tenglamasi. Markazin aniqlash formulasi.\\

2. Ellipsoida. Kanonik tenglamasi (ellipsni simmetriya o'qi atrofida aylantirishdan olingan sirt, kanonik tenglamasi).\\

3. Fokuslari abssissa o'qida va koordinata boshiga nisbatan simmetrik joylashgan giperbolaning tenglamasini tuzing: asimptotalar tenglamalari $y=\pm \frac{3}{4}x$ va direktrisalar orasidagi masofa $64/5$.\\

4. $\rho = \frac{144}{13 - 5cos\theta}$ ellips ekanligini ko'rsating va uning yarim o'qlarini aniqlang.\\

5. Agar vaqtning xohlagan momentida $M(x;y)$ nuqta $5x - 16 = 0$ to'g'ri chiziqqa qaraganda $A(5;0)$ nuqtasidan 1,25 marta uzoqroq masofada joylashgan. Shu $M(x;y)$ nuqtaning harakatining tenglamasini tuzing.  
\end{tabular}
\vspace{1cm}


\begin{tabular}{m{17cm}}
\textbf{35-variant}\\
1. Ellips va uning kanonik tenglamasi (ta'rifi, fokuslari, ellipsning kanonik tenglamasi, ekstsentrisiteti, direktrisalari).\\

2. ITECH-ning umumiy tenglamasini koordinata o'qlarini burish bilan soddalashtirish (ITECH-ning umumiy tenglamalari, koordinata o'qin burish formulasi, tenglamani kanonik turga olib kelish).\\

3. Qutb tenglamasi bilan berilgan egri chiziqning tipini aniqlang: $\rho=\frac{1}{3-3\cos\theta}$.\\

4. $x^{2} - y^{2} = 27$ giperbolasiga $4x + 2y - 7 = 0$ to'g'ri chizigiga parallel bo'lgan urinmasining tenglamasini toping.  \\

5. $2x^{2} + 3y^{2} + 8x - 6y + 11 = 0$ tenglamasi bilan qanday tipdagi chiziq berilganini aniqlang va uning tenglamasini soddalashtiring va grafigini chizing.  
\end{tabular}
\vspace{1cm}


\begin{tabular}{m{17cm}}
\textbf{36-variant}\\
1. Elliptik paraboloid (parabola, o'q, elliptik paraboloid).\\

2. Parabolaning polyar koordinatalardagi tenglamasi (polyar koordinata sistemasida parabolaning tenglamasi).\\

3. Tipini aniqlang: $9x^{2}+4y^{2}+18x-8y+49=0$.\\

4. Koordinata o'qlarini almashtirmasdan ITECH tenglamasini soddalashtiring, yarim o'qlarnin toping: $4x^{2} - 4xy + 7y^{2} - 26x - 18y + 3 = 0$.\\

5. $A(\frac{10}{3};\frac{5}{3})$ nuqtasidan $\frac{x^{2}}{20} + \frac{y^{2}}{5} = 1$ ellipsiga yurgizilgan urinmalarning tenglamasini tuzing.  
\end{tabular}
\vspace{1cm}


\begin{tabular}{m{17cm}}
\textbf{37-variant}\\
1. ITECH-ning invariantlari (ITECH-ning umumiy tenglamasi, almashtirish, ITECH invariantlari).\\

2. Ikkinshi tartibli aylanma sirtlar (koordinata sistemasi, tekislik, vektor egri chiziq, aylanma sirt).\\

3. Aylana tenglamasini tuzing: $A(3;1)$ va $B(-1;3)$ nuqtalardan o'tadi, markazi $3x-y-2=0$ togri chiziqda joylashgan.\\

4. $y^{2} = 3x$ parabolasi bilan $\frac{x^{2}}{100} + \frac{y^{2}}{225} = 1$ ellipsining kesishish nuqtalarini toping.  \\

5. $M(2; - \frac{5}{3})$ nuqta $\frac{x^{2}}{9} + \frac{y^{2}}{5} = 1$ ellipsda joylashgan. $M$ nuqtaning fokal radiuslarida yotuvchi to'g'ri chiziq tenglamalarini tuzing.  
\end{tabular}
\vspace{1cm}


\begin{tabular}{m{17cm}}
\textbf{38-variant}\\
1. Silindrlik sirtlar (yasovchi to'g'ri chiziq, yo'naltiruvchi egri chiziq, silindrlik sirt).\\

2. Parabola va uning kanonik tenglamasi ( ta'rifi, fokusi, direktrisasi, kanonik tenglamasi).\\

3. Fokuslari abssissa o'qida va koordinata boshiga nisbatan simmetrik joylashgan giperbolaning tenglamasini tuzing: direktrisalar orasidagi masofa $8/3$ va ekssentrisitet $\varepsilon=3/2$.\\

4. $\rho = \frac{10}{2 - cos\theta}$ polyar tenglamasi bilan qanday chiziq berilganini aniqlang.  \\

5. Fokusi $F(2; - 1)$ nuqtasida joylashgan, mos direktrisasi $x - y - 1 = 0$ tenglamasi bilan berilgan parabolaning tenglamasini tuzing.  
\end{tabular}
\vspace{1cm}


\begin{tabular}{m{17cm}}
\textbf{39-variant}\\
1. ITECH-ning umumiy tenglamasini koordinata boshin parallel ko'chirish bilan soddalastiring (ITECH-ning umumiy tenglamasini parallel ko'chirish formulasi).\\

2. Sirtning kanonik tenglamalari. Sirt haqqida tushuncha. (Sirtning ta'rifi, formulalari, o'q, yo'naltiruvchi to'g'ri chiziqlar).\\

3. Qutb tenglamasi bilan berilgan egri chiziqning tipini aniqlang: $\rho=\frac{5}{1-\frac{1}{2}\cos\theta}$.\\

4. $2x + 2y - 3 = 0$ to'g'ri chizig'iga perpendikulyar bo'lib $x^{2} = 16y$ parabolasiga urinib o'tuvchi to'g'ri chiziqning tenglamasini tuzing.  \\

5. $4x^{2} + 24xy + 11y^{2} + 64x + 42y + 51 = 0$ egri chizig'ining tipini aniqlang, agar markazga ega bo'lsa, uning markazining koordinatalarini toping va koordinata boshini markazga parallel ko'chirish amalini bajaring.
\end{tabular}
\vspace{1cm}


\begin{tabular}{m{17cm}}
\textbf{40-variant}\\
1. Giperbolaning polyar koordinatadagi tenglamasi (Polyar burchagi, polyar radiusi giperbolaning polyar tenglamasi)\\

2. ITECH-ning umumiy tenglamasini klassifikatsiyalash (ITECH-ning umumiy tenglamasi, ITECH-ning umumiy tenglamasini soddalashtirish, klassifikatsiyalash).\\

3. Tipini aniqlang: $4x^2+9y^2-40x+36y+100=0$.\\

4. Koordinata o'qlarini almashtirmasdan ITECH umumiy tenglamasini soddalashtiring, yarim o'qlarini toping: $13x^{2} + 18xy + 37y^{2} - 26x - 18y + 3 = 0$.  \\

5. $\frac{x^{2}}{3} - \frac{y^{2}}{5} = 1$, giperbolasiga $P(4;2)$ nuqtadan yurgizilgan urinmalarning tenglamasini tuzing.  
\end{tabular}
\vspace{1cm}


\begin{tabular}{m{17cm}}
\textbf{41-variant}\\
1. Giperbolik paraboloydning to'g'ri chiziq yasovchilari (Giperbolik paraboloydni yasovchi to'g'ri chiziqlar dastasi).\\

2. Giperbolaning urinmasining tenglamasi (giperbolaga berilgan nuqtada yurgizilgan urinma tenglamasi).\\

3. Aylananing $C$ markazi va $R$ radiusini toping: $x^2+y^2-2x+4y-14=0$.\\

4. $3x + 10y - 25 = 0$ to'g'ri bilan $\frac{x^{2}}{25} + \frac{y^{2}}{4} = 1$ ellipsning kesishish nuqtalarini toping.  \\

5. $y^{2} = 20x$ parabolasining $M$ nuqtasini toping, agar uning abssissasi 7 ga teng bo'lsa, fokal radiusini va fokal radiusi joylashgan to'g'rini aniqlang.
\end{tabular}
\vspace{1cm}


\begin{tabular}{m{17cm}}
\textbf{42-variant}\\
1. ITECH-ning umumiy tenglamasini soddalashtirish (ITECH-ning umumiy tenglamasi, koordinata sistemasin almashtirish ITECH umumiy tenglamasini soddalashtirish).\\

2. Ikki pallali giperboloid Kanonik tenglamasi (giperbolani simmetriya o'qi atrofida aylantirishdan olingan sirt).\\

3. Fokuslari abssissa o'qida va koordinata boshiga nisbatan simmetrik joylashgan giperbolaning tenglamasini tuzing: o'qlari $2a=10$ va $2b=8$.\\

4. $\rho = \frac{5}{3 - 4cos\theta}$ tenglamasi bilan qanday chiziq berilganini va yarim o'qlarini toping.  \\

5. Fokusi $F( - 1; - 4)$ nuqtasida joylashgan, mos direktrisasi $x - 2 = 0$ tenglamasi bilan berilgan, $A( - 3; - 5)$ nuqtadan o'tuvchi ellipsning tenglamasini tuzing.  
\end{tabular}
\vspace{1cm}


\begin{tabular}{m{17cm}}
\textbf{43-variant}\\
1. Ellipsning polyar koordinatalardagi tenglamasi (polyar koordinatalar sistemasida ellipsning tenglamasi).\\

2. Koordinata sistemasini almashtirish (birlik vektorlar, o'qlar, parallel ko'chirish, koordinata o'qlarinii burish).\\

3. Tipini aniqlang: $4x^{2}-y^{2}+8x-2y+3=0$.\\

4. $2x + 2y - 3 = 0$ to'g'ri chizig'iga parallel bo'lib $\frac{x^{2}}{16} + \frac{y^{2}}{64} = 1$ giperbolasiga urinib o'tuvchi to'g'ri chiziqning tenglamasini tuzing.  \\

5. $4x^{2} - 4xy + y^{2} - 2x - 14y + 7 = 0$ ITECH tenglamasini kanonik shaklga olib keling, tipini aniqlang, qanday geometrik obraz ekanligini ko'rsating, chizmasini eski va yangi koordinatalar sistemasiga nisbatan chizing.  
\end{tabular}
\vspace{1cm}


\begin{tabular}{m{17cm}}
\textbf{44-variant}\\
1. Bir pallali giperboloid. Kanonik tenglamasi (giperbolani simmetriya o'qi atrofida aylantirishdan olingan sirt).\\

2. Ellipsning urinmasining tenglamasi (ellips, to'g'ri chiziq urinish nuqtasi, urinma tenglamasi).\\

3. Aylananing $C$ markazi va $R$ radiusini toping: $x^2+y^2+6x-4y+14=0$.\\

4. ITECH ning umumiy tenglamasini koordinata sistemasini almashtirmasdan soddalashtiring, tipini aniqlang, obrazi qanday chiziq ekanligini ko'rsating: $7x^{2} - 8xy + y^{2} - 16x - 2y - 51 = 0$\\

5. Fokuslari $F(3;4)$, $F(-3;-4)$ nuqtalarida joylashgan direktrisalari orasidagi masofa 3,6 ga teng bo'lgan giperbolaning tenglamasini tuzing.  
\end{tabular}
\vspace{1cm}


\begin{tabular}{m{17cm}}
\textbf{45-variant}\\
1. ITECH-ning markazini aniqlash formulasi (ITECH-ning umumiy tenglamasi, markazini aniqlash formulasi).\\

2. Ellipsoida. Kanonik tenglamasi (ellipsni simmetriya o'qi atrofida aylantirishdan olingan sirt, kanonik tenglamasi).\\

3. Fokuslari abssissa o'qida va koordinata boshiga nisbatan simmetrik joylashgan ellipsning tenglamasini tuzing: yarim o'qlari 5 va 2.\\

4. $x^{2} + 4y^{2} = 25$ ellipsi bilan $4x - 2y + 23 = 0$ to'g'ri chizig'iga parallel bo'lgan urinma to'g'ri chiziqning tenglamasini tuzing.  \\

5. $4x^{2} - 4xy + y^{2} - 6x + 8y + 13 = 0$ ITECH markazga egami? Markazga ega bo'lsa markazini aniqlang?  
\end{tabular}
\vspace{1cm}


\begin{tabular}{m{17cm}}
\textbf{46-variant}\\
1. Parabolaning urinmasining tenglamasi (parabola, to'g'ri chiziq urinish nuqtasi, urinma tenglamasi).\\

2. Ikkinchi tartibli sirtning umumiy tenglamasi. Markazin aniqlash formulasi.\\

3. Tipini aniqlang: $9x^{2}-16y^{2}-54x-64y-127=0$.\\

4. $\frac{x^{2}}{4} - \frac{y^{2}}{5} = 1$, giperbolaning $3x - 2y = 0$ to'g'ri chizig'iga parallel bo'lgan urinmasining tenglamasini tuzing.  \\

5. Uchi (-4;0) nuqtasinda, direktrisasi $y - 2 = 0$ to'g'ri chiziq bo'lgan parabolaning tenglamasini tuzing.
\end{tabular}
\vspace{1cm}


\begin{tabular}{m{17cm}}
\textbf{47-variant}\\
1. Elliptik paraboloid (parabola, o'q, elliptik paraboloid).\\

2. Giperbola. Kanonik tenglamasi (fokuslar, o'qlar, direktrisalar, giperbola, ekstsentrisitet, kanonik tenglamasi).\\

3. Fokuslari abssissa o'qida va koordinata boshiga nisbatan simmetrik joylashgan ellipsning tenglamasini tuzing: katta o'qi $8$, direktrisalar orasidagi masofa $16$.\\

4. $y^{2} = 12x$ paraborolasiga $3x - 2y + 30 = 0$ to'g'ri chizig'iga parallel bo'lgan urinmasining tenglamasini tuzing.  \\

5. $32x^{2} + 52xy - 7y^{2} + 180 = 0$ ITECH tenglamasini kanonik shaklga olib keling, tipini aniqlang, qanday geometrik obraz ekanligini ko'rsating, chizmasini eski va yangi koordinatalar sistemasiga nisbatan chizing.  
\end{tabular}
\vspace{1cm}


\begin{tabular}{m{17cm}}
\textbf{48-variant}\\
1. ITECH-ning umumiy tenglamasini koordinata o'qlarini burish bilan soddalashtirish (ITECH-ning umumiy tenglamalari, koordinata o'qin burish formulasi, tenglamani kanonik turga olib kelish).\\

2. Ikkinshi tartibli aylanma sirtlar (koordinata sistemasi, tekislik, vektor egri chiziq, aylanma sirt).\\

3. Tipini aniqlang: $5x^{2}+14xy+11y^{2}+12x-7y+19=0$.\\

4. Koordinata o'qlarini almashtirmasdan ITECH tenglamasini soddalashtiring, qanday geometrik obraz ekanligini ko'rsating: $4x^{2} - 4xy + y^{2} + 4x - 2y + 1 = 0$.  \\

5. Fokusi $F( - 1; - 4)$ nuqtasida bo'lgan, mos direktrisasi $x - 2 = 0$ tenglamasi bilan berilgan, $A( - 3; - 5)$ nuqtadan o'tuvchi ellipsning tenglamasini tuzing.  
\end{tabular}
\vspace{1cm}


\begin{tabular}{m{17cm}}
\textbf{49-variant}\\
1. Ellips va uning kanonik tenglamasi (ta'rifi, fokuslari, ellipsning kanonik tenglamasi, ekstsentrisiteti, direktrisalari).\\

2. ITECH-ning invariantlari (ITECH-ning umumiy tenglamasi, almashtirish, ITECH invariantlari).\\

3. Fokuslari abssissa o'qida va koordinata boshiga nisbatan simmetrik joylashgan giperbolaning tenglamasini tuzing: katta o'qi $2a=16$ va ekssentrisitet $\varepsilon=5/4$.\\

4. $\frac{x^{2}}{20} - \frac{y^{2}}{5} = 1$ giperbolasiga $4x + 3y - 7 = 0$ to'g'ri chizig'iga perpendikulyar bo'lgan urinmasining tenglamasini tuzing.  \\

5. Fokusi $F(7;2)$ nuqtasida joylashgan, mos direktrisasi $x - 5 = 0$ tenglamasi bilan berilgan parabolaning tenglamasini tuzing.  
\end{tabular}
\vspace{1cm}


\begin{tabular}{m{17cm}}
\textbf{50-variant}\\
1. Silindrlik sirtlar (yasovchi to'g'ri chiziq, yo'naltiruvchi egri chiziq, silindrlik sirt).\\

2. Sirtning kanonik tenglamalari. Sirt haqqida tushuncha. (Sirtning ta'rifi, formulalari, o'q, yo'naltiruvchi to'g'ri chiziqlar).\\

3. Berilgan chiziqlarning markaziy ekanligini ko'rsating va markazinin toping: $3x^{2}+5xy+y^{2}-8x-11y-7=0$.\\

4. $41x^{2} + 24xy + 9y^{2} + 24x + 18y - 36 = 0$ ITECH tipini aniqlang va markazlarini toping koordinata o'qlarini almashtirmasdan qanday chiziq ekanligini ko'rsating, yarim o'qlarini toping.  \\

5. $y^{2} = 20x$ parabolasining abssissasi 7 ga teng bo'lgan $M$ nuqtasining fokal radiusini toping va fokal radiusi yotgan to'g'ri chiziqning tenglamasini tuzing.  
\end{tabular}
\vspace{1cm}


\begin{tabular}{m{17cm}}
\textbf{51-variant}\\
1. Parabolaning polyar koordinatalardagi tenglamasi (polyar koordinata sistemasida parabolaning tenglamasi).\\

2. ITECH-ning umumiy tenglamasini koordinata boshin parallel ko'chirish bilan soddalastiring (ITECH-ning umumiy tenglamasini parallel ko'chirish formulasi).\\

3. Fokuslari abssissa o'qida va koordinata boshiga nisbatan simmetrik joylashgan ellipsning tenglamasini tuzing: direktrisalar orasidagi masofa $5$ va fokuslari orasidagi masofa $2c=4$.\\

4. $x^{2} - 4y^{2} = 16$ giperbola berilgan. Uning ekssentrisitetini, fokuslarining koordinatalarini toping va asimptotalarining tenglamalarini tuzing.\\

5. Katta o'qi 26 ga, fokuslari $F( - 10;0), F(14;0)$ nuqtalarida joylashgan ellipsning tenglamasini tuzing.  
\end{tabular}
\vspace{1cm}


\begin{tabular}{m{17cm}}
\textbf{52-variant}\\
1. Giperbolik paraboloydning to'g'ri chiziq yasovchilari (Giperbolik paraboloydni yasovchi to'g'ri chiziqlar dastasi).\\

2. Parabola va uning kanonik tenglamasi ( ta'rifi, fokusi, direktrisasi, kanonik tenglamasi).\\

3. Tipini aniqlang: $3x^{2}-8xy+7y^{2}+8x-15y+20=0$.\\

4. $3x + 4y - 12 = 0$ to'g'ri chizig'i bilan $y^{2} = - 9x$ parabolasining kesishish nuqtalarini toping.  \\

5. $16x^{2} - 9y^{2} - 64x - 54y - 161 = 0$ tenglamasi giperbolaning tenglamasi ekanligini ko'rsating va uning markazi $C$ ni, yarim o'qlarini, ekssentrisitetini toping, asimptotalarining tenglamalarini tuzing.  
\end{tabular}
\vspace{1cm}


\begin{tabular}{m{17cm}}
\textbf{53-variant}\\
1. ITECH-ning umumiy tenglamasini klassifikatsiyalash (ITECH-ning umumiy tenglamasi, ITECH-ning umumiy tenglamasini soddalashtirish, klassifikatsiyalash).\\

2. Ikki pallali giperboloid Kanonik tenglamasi (giperbolani simmetriya o'qi atrofida aylantirishdan olingan sirt).\\

3. Fokuslari abssissa o'qida va koordinata boshiga nisbatan simmetrik joylashgan giperbolaning tenglamasini tuzing: asimptotalar tenglamalari $y=\pm \frac{4}{3}x$ va fokuslari orasidagi masofa $2c=20$.\\

4. $\rho = \frac{6}{1 - cos\theta}$ polyar tenglamasi bilan qanday chiziq berilganini aniqlang.  \\

5. $\frac{x^{2}}{25} + \frac{y^{2}}{16} = 1$, ellipsiga $C(10; - 8)$ nuqtadan yurgizilgan urinmalarining tenglamasini tuzing.  
\end{tabular}
\vspace{1cm}


\begin{tabular}{m{17cm}}
\textbf{54-variant}\\
1. Giperbolaning polyar koordinatadagi tenglamasi (Polyar burchagi, polyar radiusi giperbolaning polyar tenglamasi)\\

2. ITECH-ning umumiy tenglamasini soddalashtirish (ITECH-ning umumiy tenglamasi, koordinata sistemasin almashtirish ITECH umumiy tenglamasini soddalashtirish).\\

3. Fokuslari abssissa o'qida va koordinata boshiga nisbatan simmetrik joylashgan ellipsning tenglamasini tuzing: fokuslari orasidagi masofa $2c=6$ va ekssentrisitet $\varepsilon=3/5$.\\

4. $\frac{x^{2}}{4} - \frac{y^{2}}{5} = 1$ giperbolasiga $3x + 2y = 0$ to'g'ri chizig'iga perpendikulyar bo'lgan urinma to'g'ri chiziqning tenglamasini tuzing.\\

5. $\frac{x^{2}}{100} + \frac{y^{2}}{36} = 1$ ellipsining o'ng tarafdagi fokusidan 14 ga teng masofada bo'lgan nuqtasini toping.  
\end{tabular}
\vspace{1cm}


\begin{tabular}{m{17cm}}
\textbf{55-variant}\\
1. Bir pallali giperboloid. Kanonik tenglamasi (giperbolani simmetriya o'qi atrofida aylantirishdan olingan sirt).\\

2. Giperbolaning urinmasining tenglamasi (giperbolaga berilgan nuqtada yurgizilgan urinma tenglamasi).\\

3. Uchi koordinata boshida joylashgan va $Ox$ o'qiga nisbatan chap tarafafgi yarim tekislikda joylashgan parabolaning tenglamasini tuzing: parametri $p=0,5$.\\

4. Koordinata o'qlarini almashtirmasdan ITECH tenglamasini soddalashtiring, yarim o'qlarnin toping: $4x^{2} - 4xy + 7y^{2} - 26x - 18y + 3 = 0$.\\

5. Giperbolaning ekssentrisiteti $\varepsilon = \frac{13}{12}$, fokusi $F(0;13)$ nuqtasida va mos direktrisasi $13y - 144 = 0$ tenglamasi bilan berilgan bo'lsa, giperbolaning tenglamasini tuzing.  
\end{tabular}
\vspace{1cm}


\begin{tabular}{m{17cm}}
\textbf{56-variant}\\
1. Koordinata sistemasini almashtirish (birlik vektorlar, o'qlar, parallel ko'chirish, koordinata o'qlarinii burish).\\

2. Ellipsoida. Kanonik tenglamasi (ellipsni simmetriya o'qi atrofida aylantirishdan olingan sirt, kanonik tenglamasi).\\

3. Fokuslari abssissa o'qida va koordinata boshiga nisbatan simmetrik joylashgan giperbolaning tenglamasini tuzing: fokuslari orasidagi masofa $2c=6$ va ekssentrisitet $\varepsilon=3/2$.\\

4. Ellips $3x^{2} + 4y^{2} - 12 = 0$ tenglamasi bilan berilgan. Uning o'qlarining uzunliklarini, fokuslarining koordinatalarini va ekssentrisitetini toping.  \\

5. $14x^{2} + 24xy + 21y^{2} - 4x + 18y - 139 = 0$ egri chizig'ining tipini aniqlang, agar markazga ega egri chiziq bo'lsa, markazining koordinatalarini toping.  
\end{tabular}
\vspace{1cm}


\begin{tabular}{m{17cm}}
\textbf{57-variant}\\
1. Ellipsning polyar koordinatalardagi tenglamasi (polyar koordinatalar sistemasida ellipsning tenglamasi).\\

2. ITECH-ning markazini aniqlash formulasi (ITECH-ning umumiy tenglamasi, markazini aniqlash formulasi).\\

3. Fokuslari abssissa o'qida va koordinata boshiga nisbatan simmetrik joylashgan ellipsning tenglamasini tuzing: kichik o'qi $6$, direktrisalar orasidagi masofa $13$.\\

4. $3x + 4y - 12 = 0$ to'g'ri chizig'i va $y^{2} = - 9x$ parabolasining kesishish nuqtalarini toping.\\

5. $\frac{x^{2}}{3} - \frac{y^{2}}{5} = 1$ giperbolasiga $P(1; - 5)$ nuqtasida yurgizilgan urinmalarning tenglamasini tuzing.
\end{tabular}
\vspace{1cm}


\begin{tabular}{m{17cm}}
\textbf{58-variant}\\
1. Elliptik paraboloid (parabola, o'q, elliptik paraboloid).\\

2. Ellipsning urinmasining tenglamasi (ellips, to'g'ri chiziq urinish nuqtasi, urinma tenglamasi).\\

3. Fokuslari abssissa o'qida va koordinata boshiga nisbatan simmetrik joylashgan ellipsning tenglamasini tuzing: katta o'qi $20$, ekssentrisitet $\varepsilon=3/5$.\\

4. $\rho = \frac{144}{13 - 5cos\theta}$ ellips ekanligini ko'rsating va uning yarim o'qlarini aniqlang.\\

5. $M(2; - \frac{5}{3})$ nuqta $\frac{x^{2}}{9} + \frac{y^{2}}{5} = 1$ ellipsda joylashgan. $M$ nuqtaning fokal radiuslarida yotuvchi to'g'ri chiziq tenglamalarini tuzing.  
\end{tabular}
\vspace{1cm}


\begin{tabular}{m{17cm}}
\textbf{59-variant}\\
1. Ikkinchi tartibli sirtning umumiy tenglamasi. Markazin aniqlash formulasi.\\

2. Ikkinshi tartibli aylanma sirtlar (koordinata sistemasi, tekislik, vektor egri chiziq, aylanma sirt).\\

3. Fokuslari abssissa o'qida va koordinata boshiga nisbatan simmetrik joylashgan giperbolaning tenglamasini tuzing: direktrisalar orasidagi masofa $228/13$ va fokuslari orasidagi masofa $2c=26$.\\

4. $\frac{x^{2}}{16} - \frac{y^{2}}{64} = 1$ giperbolasiga berilgan $10x - 3y + 9 = 0$ to'g'ri chizig'iga parallel bo'lgan urinmasining tenglamasini tuzing.  \\

5. Agar xohlagan vaqt momentida $M(x;y)$ nuqta $A(8;4)$ nuqtasidan va ordinata o'qidan birxil masofada joylashsa, $M(x;y)$ nuqtaning harakat troektoriyasining tenglamasini tuzing.  
\end{tabular}
\vspace{1cm}


\begin{tabular}{m{17cm}}
\textbf{60-variant}\\
1. Parabolaning urinmasining tenglamasi (parabola, to'g'ri chiziq urinish nuqtasi, urinma tenglamasi).\\

2. ITECH-ning umumiy tenglamasini koordinata o'qlarini burish bilan soddalashtirish (ITECH-ning umumiy tenglamalari, koordinata o'qin burish formulasi, tenglamani kanonik turga olib kelish).\\

3. Aylana tenglamasini tuzing: markazi $C(1;-1)$ nuqtada joylashgan va $5x-12y+9-0$ to'g'ri chiziqga urinadi.\\

4. Koordinata o'qlarini almashtirmasdan ITECH umumiy tenglamasini soddalashtiring, yarim o'qlarini toping: $13x^{2} + 18xy + 37y^{2} - 26x - 18y + 3 = 0$.  \\

5. $2x^{2} + 3y^{2} + 8x - 6y + 11 = 0$ tenglamasi bilan qanday tipdagi chiziq berilganini aniqlang va uning tenglamasini soddalashtiring va grafigini chizing.  
\end{tabular}
\vspace{1cm}


\begin{tabular}{m{17cm}}
\textbf{61-variant}\\
1. Silindrlik sirtlar (yasovchi to'g'ri chiziq, yo'naltiruvchi egri chiziq, silindrlik sirt).\\

2. Giperbola. Kanonik tenglamasi (fokuslar, o'qlar, direktrisalar, giperbola, ekstsentrisitet, kanonik tenglamasi).\\

3. Fokuslari abssissa o'qida va koordinata boshiga nisbatan simmetrik joylashgan ellipsning tenglamasini tuzing: kichik o'qi $10$, ekssentrisitet $\varepsilon=12/13$.\\

4. $y^{2} = 3x$ parabolasi bilan $\frac{x^{2}}{100} + \frac{y^{2}}{225} = 1$ ellipsining kesishish nuqtalarini toping.  \\

5. $A(\frac{10}{3};\frac{5}{3})$ nuqtasidan $\frac{x^{2}}{20} + \frac{y^{2}}{5} = 1$ ellipsiga yurgizilgan urinmalarning tenglamasini tuzing.  
\end{tabular}
\vspace{1cm}


\begin{tabular}{m{17cm}}
\textbf{62-variant}\\
1. ITECH-ning invariantlari (ITECH-ning umumiy tenglamasi, almashtirish, ITECH invariantlari).\\

2. Sirtning kanonik tenglamalari. Sirt haqqida tushuncha. (Sirtning ta'rifi, formulalari, o'q, yo'naltiruvchi to'g'ri chiziqlar).\\

3. Giperbola tenglamasi berilgan: $\frac{x^{2}}{25}-\frac{y^{2}}{144}=1$. Uning qutb tenglamasini tuzing.\\

4. $\rho = \frac{10}{2 - cos\theta}$ polyar tenglamasi bilan qanday chiziq berilganini aniqlang.  \\

5. $y^{2} = 20x$ parabolasining $M$ nuqtasini toping, agar uning abssissasi 7 ga teng bo'lsa, fokal radiusini va fokal radiusi joylashgan to'g'rini aniqlang.
\end{tabular}
\vspace{1cm}


\begin{tabular}{m{17cm}}
\textbf{63-variant}\\
1. Giperbolik paraboloydning to'g'ri chiziq yasovchilari (Giperbolik paraboloydni yasovchi to'g'ri chiziqlar dastasi).\\

2. Ellips va uning kanonik tenglamasi (ta'rifi, fokuslari, ellipsning kanonik tenglamasi, ekstsentrisiteti, direktrisalari).\\

3. Tipini aniqlang: $25x^{2}-20xy+4y^{2}-12x+20y-17=0$.\\

4. $x^{2} - y^{2} = 27$ giperbolasiga $4x + 2y - 7 = 0$ to'g'ri chizigiga parallel bo'lgan urinmasining tenglamasini toping.  \\

5. Agar vaqtning xohlagan momentida $M(x;y)$ nuqta $5x - 16 = 0$ to'g'ri chiziqqa qaraganda $A(5;0)$ nuqtasidan 1,25 marta uzoqroq masofada joylashgan. Shu $M(x;y)$ nuqtaning harakatining tenglamasini tuzing.  
\end{tabular}
\vspace{1cm}


\begin{tabular}{m{17cm}}
\textbf{64-variant}\\
1. ITECH-ning umumiy tenglamasini koordinata boshin parallel ko'chirish bilan soddalastiring (ITECH-ning umumiy tenglamasini parallel ko'chirish formulasi).\\

2. Ikki pallali giperboloid Kanonik tenglamasi (giperbolani simmetriya o'qi atrofida aylantirishdan olingan sirt).\\

3. Aylana tenglamasini tuzing: markazi $C(6;-8)$ nuqtada joylashgan va koordinata boshidan o'tadi.\\

4. ITECH ning umumiy tenglamasini koordinata sistemasini almashtirmasdan soddalashtiring, tipini aniqlang, obrazi qanday chiziq ekanligini ko'rsating: $7x^{2} - 8xy + y^{2} - 16x - 2y - 51 = 0$\\

5. $4x^{2} + 24xy + 11y^{2} + 64x + 42y + 51 = 0$ egri chizig'ining tipini aniqlang, agar markazga ega bo'lsa, uning markazining koordinatalarini toping va koordinata boshini markazga parallel ko'chirish amalini bajaring.
\end{tabular}
\vspace{1cm}


\begin{tabular}{m{17cm}}
\textbf{65-variant}\\
1. Parabolaning polyar koordinatalardagi tenglamasi (polyar koordinata sistemasida parabolaning tenglamasi).\\

2. ITECH-ning umumiy tenglamasini klassifikatsiyalash (ITECH-ning umumiy tenglamasi, ITECH-ning umumiy tenglamasini soddalashtirish, klassifikatsiyalash).\\

3. Fokuslari abssissa o'qida va koordinata boshiga nisbatan simmetrik joylashgan giperbolaning tenglamasini tuzing: fokuslari orasidagi masofasi $2c=10$ va o'qi $2b=8$.\\

4. $3x + 10y - 25 = 0$ to'g'ri bilan $\frac{x^{2}}{25} + \frac{y^{2}}{4} = 1$ ellipsning kesishish nuqtalarini toping.  \\

5. $\frac{x^{2}}{3} - \frac{y^{2}}{5} = 1$, giperbolasiga $P(4;2)$ nuqtadan yurgizilgan urinmalarning tenglamasini tuzing.  
\end{tabular}
\vspace{1cm}


\begin{tabular}{m{17cm}}
\textbf{66-variant}\\
1. Bir pallali giperboloid. Kanonik tenglamasi (giperbolani simmetriya o'qi atrofida aylantirishdan olingan sirt).\\

2. Parabola va uning kanonik tenglamasi ( ta'rifi, fokusi, direktrisasi, kanonik tenglamasi).\\

3. Qutb tenglamasi bilan berilgan egri chiziqning tipini aniqlang: $\rho=\frac{10}{1-\frac{3}{2}\cos\theta}$.\\

4. $\rho = \frac{5}{3 - 4cos\theta}$ tenglamasi bilan qanday chiziq berilganini va yarim o'qlarini toping.  \\

5. $y^{2} = 20x$ parabolasining abssissasi 7 ga teng bo'lgan $M$ nuqtasining fokal radiusini toping va fokal radiusi yotgan to'g'ri chiziqning tenglamasini tuzing.  
\end{tabular}
\vspace{1cm}


\begin{tabular}{m{17cm}}
\textbf{67-variant}\\
1. ITECH-ning umumiy tenglamasini soddalashtirish (ITECH-ning umumiy tenglamasi, koordinata sistemasin almashtirish ITECH umumiy tenglamasini soddalashtirish).\\

2. Ellipsoida. Kanonik tenglamasi (ellipsni simmetriya o'qi atrofida aylantirishdan olingan sirt, kanonik tenglamasi).\\

3. Berilgan chiziqlarning markaziy ekanligini ko'rsating va markazinin toping: $2x^{2}-6xy+5y^{2}+22x-36y+11=0$.\\

4. $2x + 2y - 3 = 0$ to'g'ri chizig'iga perpendikulyar bo'lib $x^{2} = 16y$ parabolasiga urinib o'tuvchi to'g'ri chiziqning tenglamasini tuzing.  \\

5. Fokusi $F(2; - 1)$ nuqtasida joylashgan, mos direktrisasi $x - y - 1 = 0$ tenglamasi bilan berilgan parabolaning tenglamasini tuzing.  
\end{tabular}
\vspace{1cm}


\begin{tabular}{m{17cm}}
\textbf{68-variant}\\
1. Giperbolaning polyar koordinatadagi tenglamasi (Polyar burchagi, polyar radiusi giperbolaning polyar tenglamasi)\\

2. Koordinata sistemasini almashtirish (birlik vektorlar, o'qlar, parallel ko'chirish, koordinata o'qlarinii burish).\\

3. Aylananing $C$ markazi va $R$ radiusini toping: $x^2+y^2+4x-2y+5=0$.\\

4. Koordinata o'qlarini almashtirmasdan ITECH tenglamasini soddalashtiring, qanday geometrik obraz ekanligini ko'rsating: $4x^{2} - 4xy + y^{2} + 4x - 2y + 1 = 0$.  \\

5. $4x^{2} - 4xy + y^{2} - 2x - 14y + 7 = 0$ ITECH tenglamasini kanonik shaklga olib keling, tipini aniqlang, qanday geometrik obraz ekanligini ko'rsating, chizmasini eski va yangi koordinatalar sistemasiga nisbatan chizing.  
\end{tabular}
\vspace{1cm}


\begin{tabular}{m{17cm}}
\textbf{69-variant}\\
1. Elliptik paraboloid (parabola, o'q, elliptik paraboloid).\\

2. Giperbolaning urinmasining tenglamasi (giperbolaga berilgan nuqtada yurgizilgan urinma tenglamasi).\\

3. Uchi koordinata boshida joylashgan va $Ox$ o'qiga nisbatan o'ng tarafafgi yarim tekislikda joylashgan parabolaning tenglamasini tuzing: parametri $p=3$.\\

4. $2x + 2y - 3 = 0$ to'g'ri chizig'iga parallel bo'lib $\frac{x^{2}}{16} + \frac{y^{2}}{64} = 1$ giperbolasiga urinib o'tuvchi to'g'ri chiziqning tenglamasini tuzing.  \\

5. Fokusi $F( - 1; - 4)$ nuqtasida joylashgan, mos direktrisasi $x - 2 = 0$ tenglamasi bilan berilgan, $A( - 3; - 5)$ nuqtadan o'tuvchi ellipsning tenglamasini tuzing.  
\end{tabular}
\vspace{1cm}


\begin{tabular}{m{17cm}}
\textbf{70-variant}\\
1. ITECH-ning markazini aniqlash formulasi (ITECH-ning umumiy tenglamasi, markazini aniqlash formulasi).\\

2. Ikkinshi tartibli aylanma sirtlar (koordinata sistemasi, tekislik, vektor egri chiziq, aylanma sirt).\\

3. Qutb tenglamasi bilan berilgan egri chiziqning tipini aniqlang: $\rho=\frac{6}{1-\cos 0}$.\\

4. $x^{2} + 4y^{2} = 25$ ellipsi bilan $4x - 2y + 23 = 0$ to'g'ri chizig'iga parallel bo'lgan urinma to'g'ri chiziqning tenglamasini tuzing.  \\

5. $4x^{2} - 4xy + y^{2} - 6x + 8y + 13 = 0$ ITECH markazga egami? Markazga ega bo'lsa markazini aniqlang?  
\end{tabular}
\vspace{1cm}


\begin{tabular}{m{17cm}}
\textbf{71-variant}\\
1. Ellipsning polyar koordinatalardagi tenglamasi (polyar koordinatalar sistemasida ellipsning tenglamasi).\\

2. Ikkinchi tartibli sirtning umumiy tenglamasi. Markazin aniqlash formulasi.\\

3. Berilgan chiziqlarning markaziy ekanligini ko'rsating va markazinin toping: $9x^{2}-4xy-7y^{2}-12=0$.\\

4. $\frac{x^{2}}{4} - \frac{y^{2}}{5} = 1$, giperbolaning $3x - 2y = 0$ to'g'ri chizig'iga parallel bo'lgan urinmasining tenglamasini tuzing.  \\

5. Fokuslari $F(3;4)$, $F(-3;-4)$ nuqtalarida joylashgan direktrisalari orasidagi masofa 3,6 ga teng bo'lgan giperbolaning tenglamasini tuzing.  
\end{tabular}
\vspace{1cm}


\begin{tabular}{m{17cm}}
\textbf{72-variant}\\
1. Silindrlik sirtlar (yasovchi to'g'ri chiziq, yo'naltiruvchi egri chiziq, silindrlik sirt).\\

2. Ellipsning urinmasining tenglamasi (ellips, to'g'ri chiziq urinish nuqtasi, urinma tenglamasi).\\

3. Aylana tenglamasini tuzing: aylana diametrining uchlari $A(3;2)$ va $B(-1;6)$ nuqtalarda joylashgan.\\

4. $41x^{2} + 24xy + 9y^{2} + 24x + 18y - 36 = 0$ ITECH tipini aniqlang va markazlarini toping koordinata o'qlarini almashtirmasdan qanday chiziq ekanligini ko'rsating, yarim o'qlarini toping.  \\

5. $32x^{2} + 52xy - 7y^{2} + 180 = 0$ ITECH tenglamasini kanonik shaklga olib keling, tipini aniqlang, qanday geometrik obraz ekanligini ko'rsating, chizmasini eski va yangi koordinatalar sistemasiga nisbatan chizing.  
\end{tabular}
\vspace{1cm}


\begin{tabular}{m{17cm}}
\textbf{73-variant}\\
1. ITECH-ning umumiy tenglamasini koordinata o'qlarini burish bilan soddalashtirish (ITECH-ning umumiy tenglamalari, koordinata o'qin burish formulasi, tenglamani kanonik turga olib kelish).\\

2. Sirtning kanonik tenglamalari. Sirt haqqida tushuncha. (Sirtning ta'rifi, formulalari, o'q, yo'naltiruvchi to'g'ri chiziqlar).\\

3. Uchi koordinata boshida joylashgan va $Oy$ o'qiga nisbatan quyi tarafafgi yarim tekislikda joylashgan parabolaning tenglamasini tuzing: parametri $p=3$.\\

4. $y^{2} = 12x$ paraborolasiga $3x - 2y + 30 = 0$ to'g'ri chizig'iga parallel bo'lgan urinmasining tenglamasini tuzing.  \\

5. Uchi (-4;0) nuqtasinda, direktrisasi $y - 2 = 0$ to'g'ri chiziq bo'lgan parabolaning tenglamasini tuzing.
\end{tabular}
\vspace{1cm}


\begin{tabular}{m{17cm}}
\textbf{74-variant}\\
1. Parabolaning urinmasining tenglamasi (parabola, to'g'ri chiziq urinish nuqtasi, urinma tenglamasi).\\

2. ITECH-ning invariantlari (ITECH-ning umumiy tenglamasi, almashtirish, ITECH invariantlari).\\

3. Qutb tenglamasi bilan berilgan egri chiziqning tipini aniqlang: $\rho=\frac{12}{2-\cos\theta}$.\\

4. Koordinata o'qlarini almashtirmasdan ITECH tenglamasini soddalashtiring, yarim o'qlarnin toping: $4x^{2} - 4xy + 7y^{2} - 26x - 18y + 3 = 0$.\\

5. Fokusi $F( - 1; - 4)$ nuqtasida bo'lgan, mos direktrisasi $x - 2 = 0$ tenglamasi bilan berilgan, $A( - 3; - 5)$ nuqtadan o'tuvchi ellipsning tenglamasini tuzing.  
\end{tabular}
\vspace{1cm}


\begin{tabular}{m{17cm}}
\textbf{75-variant}\\
1. Giperbolik paraboloydning to'g'ri chiziq yasovchilari (Giperbolik paraboloydni yasovchi to'g'ri chiziqlar dastasi).\\

2. Ikki pallali giperboloid Kanonik tenglamasi (giperbolani simmetriya o'qi atrofida aylantirishdan olingan sirt).\\

3. Tipini aniqlang: $3x^{2}-2xy-3y^{2}+12y-15=0$.\\

4. $x^{2} - 4y^{2} = 16$ giperbola berilgan. Uning ekssentrisitetini, fokuslarining koordinatalarini toping va asimptotalarining tenglamalarini tuzing.\\

5. $\frac{x^{2}}{100} + \frac{y^{2}}{36} = 1$ ellipsining o'ng tarafdagi fokusidan 14 ga teng masofada bo'lgan nuqtasini toping.  
\end{tabular}
\vspace{1cm}


\begin{tabular}{m{17cm}}
\textbf{76-variant}\\
1. Giperbola. Kanonik tenglamasi (fokuslar, o'qlar, direktrisalar, giperbola, ekstsentrisitet, kanonik tenglamasi).\\

2. ITECH-ning umumiy tenglamasini koordinata boshin parallel ko'chirish bilan soddalastiring (ITECH-ning umumiy tenglamasini parallel ko'chirish formulasi).\\

3. Aylananing $C$ markazi va $R$ radiusini toping: $x^2+y^2-2x+4y-20=0$.\\

4. $3x + 4y - 12 = 0$ to'g'ri chizig'i bilan $y^{2} = - 9x$ parabolasining kesishish nuqtalarini toping.  \\

5. Fokusi $F(7;2)$ nuqtasida joylashgan, mos direktrisasi $x - 5 = 0$ tenglamasi bilan berilgan parabolaning tenglamasini tuzing.  
\end{tabular}
\vspace{1cm}


\begin{tabular}{m{17cm}}
\textbf{77-variant}\\
1. Bir pallali giperboloid. Kanonik tenglamasi (giperbolani simmetriya o'qi atrofida aylantirishdan olingan sirt).\\

2. Ellips va uning kanonik tenglamasi (ta'rifi, fokuslari, ellipsning kanonik tenglamasi, ekstsentrisiteti, direktrisalari).\\

3. Uchi koordinata boshida joylashgan va $Oy$ o'qiga nisbatan yuqori yarim tekislikda joylashgan parabolaning tenglamasini tuzing: parametri $p=1/4$.\\

4. $\rho = \frac{6}{1 - cos\theta}$ polyar tenglamasi bilan qanday chiziq berilganini aniqlang.  \\

5. $16x^{2} - 9y^{2} - 64x - 54y - 161 = 0$ tenglamasi giperbolaning tenglamasi ekanligini ko'rsating va uning markazi $C$ ni, yarim o'qlarini, ekssentrisitetini toping, asimptotalarining tenglamalarini tuzing.  
\end{tabular}
\vspace{1cm}


\begin{tabular}{m{17cm}}
\textbf{78-variant}\\
1. ITECH-ning umumiy tenglamasini klassifikatsiyalash (ITECH-ning umumiy tenglamasi, ITECH-ning umumiy tenglamasini soddalashtirish, klassifikatsiyalash).\\

2. Ellipsoida. Kanonik tenglamasi (ellipsni simmetriya o'qi atrofida aylantirishdan olingan sirt, kanonik tenglamasi).\\

3. Qutb tenglamasi bilan berilgan egri chiziqning tipini aniqlang: $\rho=\frac{5}{3-4\cos\theta}$.\\

4. $\frac{x^{2}}{20} - \frac{y^{2}}{5} = 1$ giperbolasiga $4x + 3y - 7 = 0$ to'g'ri chizig'iga perpendikulyar bo'lgan urinmasining tenglamasini tuzing.  \\

5. $\frac{x^{2}}{25} + \frac{y^{2}}{16} = 1$, ellipsiga $C(10; - 8)$ nuqtadan yurgizilgan urinmalarining tenglamasini tuzing.  
\end{tabular}
\vspace{1cm}


\begin{tabular}{m{17cm}}
\textbf{79-variant}\\
1. Parabolaning polyar koordinatalardagi tenglamasi (polyar koordinata sistemasida parabolaning tenglamasi).\\

2. ITECH-ning umumiy tenglamasini soddalashtirish (ITECH-ning umumiy tenglamasi, koordinata sistemasin almashtirish ITECH umumiy tenglamasini soddalashtirish).\\

3. Tipini aniqlang: $x^{2}-4xy+4y^{2}+7x-12=0$.\\

4. Koordinata o'qlarini almashtirmasdan ITECH umumiy tenglamasini soddalashtiring, yarim o'qlarini toping: $13x^{2} + 18xy + 37y^{2} - 26x - 18y + 3 = 0$.  \\

5. $M(2; - \frac{5}{3})$ nuqta $\frac{x^{2}}{9} + \frac{y^{2}}{5} = 1$ ellipsda joylashgan. $M$ nuqtaning fokal radiuslarida yotuvchi to'g'ri chiziq tenglamalarini tuzing.  
\end{tabular}
\vspace{1cm}


\begin{tabular}{m{17cm}}
\textbf{80-variant}\\
1. Elliptik paraboloid (parabola, o'q, elliptik paraboloid).\\

2. Parabola va uning kanonik tenglamasi ( ta'rifi, fokusi, direktrisasi, kanonik tenglamasi).\\

3. Aylana tenglamasini tuzing: aylana $A(2;6)$ nuqtadan o'tadi va markazi $C(-1;2)$ nuqtada joylashgan.\\

4. Ellips $3x^{2} + 4y^{2} - 12 = 0$ tenglamasi bilan berilgan. Uning o'qlarining uzunliklarini, fokuslarining koordinatalarini va ekssentrisitetini toping.  \\

5. Katta o'qi 26 ga, fokuslari $F( - 10;0), F(14;0)$ nuqtalarida joylashgan ellipsning tenglamasini tuzing.  
\end{tabular}
\vspace{1cm}


\begin{tabular}{m{17cm}}
\textbf{81-variant}\\
1. Koordinata sistemasini almashtirish (birlik vektorlar, o'qlar, parallel ko'chirish, koordinata o'qlarinii burish).\\

2. Ikkinshi tartibli aylanma sirtlar (koordinata sistemasi, tekislik, vektor egri chiziq, aylanma sirt).\\

3. Fokuslari abssissa o'qida va koordinata boshiga nisbatan simmetrik joylashgan ellipsning tenglamasini tuzing: kichik o'qi $24$, fokuslari orasidagi masofa $2c=10$.\\

4. $3x + 4y - 12 = 0$ to'g'ri chizig'i va $y^{2} = - 9x$ parabolasining kesishish nuqtalarini toping.\\

5. $14x^{2} + 24xy + 21y^{2} - 4x + 18y - 139 = 0$ egri chizig'ining tipini aniqlang, agar markazga ega egri chiziq bo'lsa, markazining koordinatalarini toping.  
\end{tabular}
\vspace{1cm}


\begin{tabular}{m{17cm}}
\textbf{82-variant}\\
1. Giperbolaning polyar koordinatadagi tenglamasi (Polyar burchagi, polyar radiusi giperbolaning polyar tenglamasi)\\

2. ITECH-ning markazini aniqlash formulasi (ITECH-ning umumiy tenglamasi, markazini aniqlash formulasi).\\

3. Giperbola tenglamasi berilgan: $\frac{x^{2}}{16}-\frac{y^{2}}{9}=1$. Uning qutb tenglamasini tuzing.\\

4. $\rho = \frac{144}{13 - 5cos\theta}$ ellips ekanligini ko'rsating va uning yarim o'qlarini aniqlang.\\

5. $\frac{x^{2}}{3} - \frac{y^{2}}{5} = 1$ giperbolasiga $P(1; - 5)$ nuqtasida yurgizilgan urinmalarning tenglamasini tuzing.
\end{tabular}
\vspace{1cm}


\begin{tabular}{m{17cm}}
\textbf{83-variant}\\
1. Silindrlik sirtlar (yasovchi to'g'ri chiziq, yo'naltiruvchi egri chiziq, silindrlik sirt).\\

2. Giperbolaning urinmasining tenglamasi (giperbolaga berilgan nuqtada yurgizilgan urinma tenglamasi).\\

3. Tipini aniqlang: $2x^{2}+3y^{2}+8x-6y+11=0$.\\

4. $\frac{x^{2}}{4} - \frac{y^{2}}{5} = 1$ giperbolasiga $3x + 2y = 0$ to'g'ri chizig'iga perpendikulyar bo'lgan urinma to'g'ri chiziqning tenglamasini tuzing.\\

5. $y^{2} = 20x$ parabolasining $M$ nuqtasini toping, agar uning abssissasi 7 ga teng bo'lsa, fokal radiusini va fokal radiusi joylashgan to'g'rini aniqlang.
\end{tabular}
\vspace{1cm}


\begin{tabular}{m{17cm}}
\textbf{84-variant}\\
1. Ikkinchi tartibli sirtning umumiy tenglamasi. Markazin aniqlash formulasi.\\

2. Sirtning kanonik tenglamalari. Sirt haqqida tushuncha. (Sirtning ta'rifi, formulalari, o'q, yo'naltiruvchi to'g'ri chiziqlar).\\

3. Aylana tenglamasini tuzing: markazi koordinata boshida joylashgan va $3x-4y+20=0$ to'g'ri chiziqga urinadi.\\

4. ITECH ning umumiy tenglamasini koordinata sistemasini almashtirmasdan soddalashtiring, tipini aniqlang, obrazi qanday chiziq ekanligini ko'rsating: $7x^{2} - 8xy + y^{2} - 16x - 2y - 51 = 0$\\

5. Giperbolaning ekssentrisiteti $\varepsilon = \frac{13}{12}$, fokusi $F(0;13)$ nuqtasida va mos direktrisasi $13y - 144 = 0$ tenglamasi bilan berilgan bo'lsa, giperbolaning tenglamasini tuzing.  
\end{tabular}
\vspace{1cm}


\begin{tabular}{m{17cm}}
\textbf{85-variant}\\
1. Ellipsning polyar koordinatalardagi tenglamasi (polyar koordinatalar sistemasida ellipsning tenglamasi).\\

2. ITECH-ning umumiy tenglamasini koordinata o'qlarini burish bilan soddalashtirish (ITECH-ning umumiy tenglamalari, koordinata o'qin burish formulasi, tenglamani kanonik turga olib kelish).\\

3. Fokuslari abssissa o'qida va koordinata boshiga nisbatan simmetrik joylashgan ellipsning tenglamasini tuzing: katta o'qi $10$, fokuslari orasidagi masofa $2c=8$.\\

4. $y^{2} = 3x$ parabolasi bilan $\frac{x^{2}}{100} + \frac{y^{2}}{225} = 1$ ellipsining kesishish nuqtalarini toping.  \\

5. $2x^{2} + 3y^{2} + 8x - 6y + 11 = 0$ tenglamasi bilan qanday tipdagi chiziq berilganini aniqlang va uning tenglamasini soddalashtiring va grafigini chizing.  
\end{tabular}
\vspace{1cm}


\begin{tabular}{m{17cm}}
\textbf{86-variant}\\
1. Giperbolik paraboloydning to'g'ri chiziq yasovchilari (Giperbolik paraboloydni yasovchi to'g'ri chiziqlar dastasi).\\

2. Ellipsning urinmasining tenglamasi (ellips, to'g'ri chiziq urinish nuqtasi, urinma tenglamasi).\\

3. Ellips tenglamasi berilgan: $\frac{x^2}{25}+\frac{y^2}{16}=1$. Uning qutb tenglamasini tuzing.\\

4. $\rho = \frac{10}{2 - cos\theta}$ polyar tenglamasi bilan qanday chiziq berilganini aniqlang.  \\

5. $A(\frac{10}{3};\frac{5}{3})$ nuqtasidan $\frac{x^{2}}{20} + \frac{y^{2}}{5} = 1$ ellipsiga yurgizilgan urinmalarning tenglamasini tuzing.  
\end{tabular}
\vspace{1cm}


\begin{tabular}{m{17cm}}
\textbf{87-variant}\\
1. ITECH-ning invariantlari (ITECH-ning umumiy tenglamasi, almashtirish, ITECH invariantlari).\\

2. Ikki pallali giperboloid Kanonik tenglamasi (giperbolani simmetriya o'qi atrofida aylantirishdan olingan sirt).\\

3. Tipini aniqlang: $2x^{2}+10xy+12y^{2}-7x+18y-15=0$.\\

4. $\frac{x^{2}}{16} - \frac{y^{2}}{64} = 1$ giperbolasiga berilgan $10x - 3y + 9 = 0$ to'g'ri chizig'iga parallel bo'lgan urinmasining tenglamasini tuzing.  \\

5. $y^{2} = 20x$ parabolasining abssissasi 7 ga teng bo'lgan $M$ nuqtasining fokal radiusini toping va fokal radiusi yotgan to'g'ri chiziqning tenglamasini tuzing.  
\end{tabular}
\vspace{1cm}


\begin{tabular}{m{17cm}}
\textbf{88-variant}\\
1. Bir pallali giperboloid. Kanonik tenglamasi (giperbolani simmetriya o'qi atrofida aylantirishdan olingan sirt).\\

2. Parabolaning urinmasining tenglamasi (parabola, to'g'ri chiziq urinish nuqtasi, urinma tenglamasi).\\

3. Aylana tenglamasini tuzing: markazi koordinata boshida joylashgan va radiusi $R=3$ ga teng.\\

4. Koordinata o'qlarini almashtirmasdan ITECH tenglamasini soddalashtiring, qanday geometrik obraz ekanligini ko'rsating: $4x^{2} - 4xy + y^{2} + 4x - 2y + 1 = 0$.  \\

5. Agar xohlagan vaqt momentida $M(x;y)$ nuqta $A(8;4)$ nuqtasidan va ordinata o'qidan birxil masofada joylashsa, $M(x;y)$ nuqtaning harakat troektoriyasining tenglamasini tuzing.  
\end{tabular}
\vspace{1cm}


\begin{tabular}{m{17cm}}
\textbf{89-variant}\\
1. ITECH-ning umumiy tenglamasini koordinata boshin parallel ko'chirish bilan soddalastiring (ITECH-ning umumiy tenglamasini parallel ko'chirish formulasi).\\

2. Ellipsoida. Kanonik tenglamasi (ellipsni simmetriya o'qi atrofida aylantirishdan olingan sirt, kanonik tenglamasi).\\

3. Fokuslari abssissa o'qida va koordinata boshiga nisbatan simmetrik joylashgan giperbolaning tenglamasini tuzing: direktrisalar orasidagi masofa $32/5$ va o'qi $2b=6$.\\

4. $3x + 10y - 25 = 0$ to'g'ri bilan $\frac{x^{2}}{25} + \frac{y^{2}}{4} = 1$ ellipsning kesishish nuqtalarini toping.  \\

5. $4x^{2} + 24xy + 11y^{2} + 64x + 42y + 51 = 0$ egri chizig'ining tipini aniqlang, agar markazga ega bo'lsa, uning markazining koordinatalarini toping va koordinata boshini markazga parallel ko'chirish amalini bajaring.
\end{tabular}
\vspace{1cm}


\begin{tabular}{m{17cm}}
\textbf{90-variant}\\
1. Giperbola. Kanonik tenglamasi (fokuslar, o'qlar, direktrisalar, giperbola, ekstsentrisitet, kanonik tenglamasi).\\

2. ITECH-ning umumiy tenglamasini klassifikatsiyalash (ITECH-ning umumiy tenglamasi, ITECH-ning umumiy tenglamasini soddalashtirish, klassifikatsiyalash).\\

3. Parabola tenglamasi berilgan: $y^2=6x$. Uning qutb tenglamasini tuzing.\\

4. $\rho = \frac{5}{3 - 4cos\theta}$ tenglamasi bilan qanday chiziq berilganini va yarim o'qlarini toping.  \\

5. $\frac{x^{2}}{3} - \frac{y^{2}}{5} = 1$, giperbolasiga $P(4;2)$ nuqtadan yurgizilgan urinmalarning tenglamasini tuzing.  
\end{tabular}
\vspace{1cm}


\begin{tabular}{m{17cm}}
\textbf{91-variant}\\
1. Elliptik paraboloid (parabola, o'q, elliptik paraboloid).\\

2. Ellips va uning kanonik tenglamasi (ta'rifi, fokuslari, ellipsning kanonik tenglamasi, ekstsentrisiteti, direktrisalari).\\

3. Berilgan chiziqlarning markaziy ekanligini ko'rsating va markazinin toping: $5x^{2}+4xy+2y^{2}+20x+20y-18=0$.\\

4. $x^{2} - y^{2} = 27$ giperbolasiga $4x + 2y - 7 = 0$ to'g'ri chizigiga parallel bo'lgan urinmasining tenglamasini toping.  \\

5. $\frac{x^{2}}{100} + \frac{y^{2}}{36} = 1$ ellipsining o'ng tarafdagi fokusidan 14 ga teng masofada bo'lgan nuqtasini toping.  
\end{tabular}
\vspace{1cm}


\begin{tabular}{m{17cm}}
\textbf{92-variant}\\
1. ITECH-ning umumiy tenglamasini soddalashtirish (ITECH-ning umumiy tenglamasi, koordinata sistemasin almashtirish ITECH umumiy tenglamasini soddalashtirish).\\

2. Ikkinshi tartibli aylanma sirtlar (koordinata sistemasi, tekislik, vektor egri chiziq, aylanma sirt).\\

3. Aylana tenglamasini tuzing: markazi $C(2;-3)$ nuqtada joylashgan va radiusi $R=7$ ga teng.\\

4. $41x^{2} + 24xy + 9y^{2} + 24x + 18y - 36 = 0$ ITECH tipini aniqlang va markazlarini toping koordinata o'qlarini almashtirmasdan qanday chiziq ekanligini ko'rsating, yarim o'qlarini toping.  \\

5. Agar vaqtning xohlagan momentida $M(x;y)$ nuqta $5x - 16 = 0$ to'g'ri chiziqqa qaraganda $A(5;0)$ nuqtasidan 1,25 marta uzoqroq masofada joylashgan. Shu $M(x;y)$ nuqtaning harakatining tenglamasini tuzing.  
\end{tabular}
\vspace{1cm}


\begin{tabular}{m{17cm}}
\textbf{93-variant}\\
1. Parabolaning polyar koordinatalardagi tenglamasi (polyar koordinata sistemasida parabolaning tenglamasi).\\

2. Koordinata sistemasini almashtirish (birlik vektorlar, o'qlar, parallel ko'chirish, koordinata o'qlarinii burish).\\

3. Fokuslari abssissa o'qida va koordinata boshiga nisbatan simmetrik joylashgan giperbolaning tenglamasini tuzing: asimptotalar tenglamalari $y=\pm \frac{3}{4}x$ va direktrisalar orasidagi masofa $64/5$.\\

4. $2x + 2y - 3 = 0$ to'g'ri chizig'iga perpendikulyar bo'lib $x^{2} = 16y$ parabolasiga urinib o'tuvchi to'g'ri chiziqning tenglamasini tuzing.  \\

5. $4x^{2} - 4xy + y^{2} - 2x - 14y + 7 = 0$ ITECH tenglamasini kanonik shaklga olib keling, tipini aniqlang, qanday geometrik obraz ekanligini ko'rsating, chizmasini eski va yangi koordinatalar sistemasiga nisbatan chizing.  
\end{tabular}
\vspace{1cm}


\begin{tabular}{m{17cm}}
\textbf{94-variant}\\
1. Silindrlik sirtlar (yasovchi to'g'ri chiziq, yo'naltiruvchi egri chiziq, silindrlik sirt).\\

2. Parabola va uning kanonik tenglamasi ( ta'rifi, fokusi, direktrisasi, kanonik tenglamasi).\\

3. Qutb tenglamasi bilan berilgan egri chiziqning tipini aniqlang: $\rho=\frac{1}{3-3\cos\theta}$.\\

4. $2x + 2y - 3 = 0$ to'g'ri chizig'iga parallel bo'lib $\frac{x^{2}}{16} + \frac{y^{2}}{64} = 1$ giperbolasiga urinib o'tuvchi to'g'ri chiziqning tenglamasini tuzing.  \\

5. Fokusi $F(2; - 1)$ nuqtasida joylashgan, mos direktrisasi $x - y - 1 = 0$ tenglamasi bilan berilgan parabolaning tenglamasini tuzing.  
\end{tabular}
\vspace{1cm}


\begin{tabular}{m{17cm}}
\textbf{95-variant}\\
1. ITECH-ning markazini aniqlash formulasi (ITECH-ning umumiy tenglamasi, markazini aniqlash formulasi).\\

2. Sirtning kanonik tenglamalari. Sirt haqqida tushuncha. (Sirtning ta'rifi, formulalari, o'q, yo'naltiruvchi to'g'ri chiziqlar).\\

3. Tipini aniqlang: $9x^{2}+4y^{2}+18x-8y+49=0$.\\

4. $x^{2} + 4y^{2} = 25$ ellipsi bilan $4x - 2y + 23 = 0$ to'g'ri chizig'iga parallel bo'lgan urinma to'g'ri chiziqning tenglamasini tuzing.  \\

5. $4x^{2} - 4xy + y^{2} - 6x + 8y + 13 = 0$ ITECH markazga egami? Markazga ega bo'lsa markazini aniqlang?  
\end{tabular}
\vspace{1cm}


\begin{tabular}{m{17cm}}
\textbf{96-variant}\\
1. Giperbolaning polyar koordinatadagi tenglamasi (Polyar burchagi, polyar radiusi giperbolaning polyar tenglamasi)\\

2. Ikkinchi tartibli sirtning umumiy tenglamasi. Markazin aniqlash formulasi.\\

3. Aylana tenglamasini tuzing: $A(3;1)$ va $B(-1;3)$ nuqtalardan o'tadi, markazi $3x-y-2=0$ togri chiziqda joylashgan.\\

4. Koordinata o'qlarini almashtirmasdan ITECH tenglamasini soddalashtiring, yarim o'qlarnin toping: $4x^{2} - 4xy + 7y^{2} - 26x - 18y + 3 = 0$.\\

5. Fokusi $F( - 1; - 4)$ nuqtasida joylashgan, mos direktrisasi $x - 2 = 0$ tenglamasi bilan berilgan, $A( - 3; - 5)$ nuqtadan o'tuvchi ellipsning tenglamasini tuzing.  
\end{tabular}
\vspace{1cm}


\begin{tabular}{m{17cm}}
\textbf{97-variant}\\
1. Giperbolik paraboloydning to'g'ri chiziq yasovchilari (Giperbolik paraboloydni yasovchi to'g'ri chiziqlar dastasi).\\

2. Giperbolaning urinmasining tenglamasi (giperbolaga berilgan nuqtada yurgizilgan urinma tenglamasi).\\

3. Fokuslari abssissa o'qida va koordinata boshiga nisbatan simmetrik joylashgan giperbolaning tenglamasini tuzing: direktrisalar orasidagi masofa $8/3$ va ekssentrisitet $\varepsilon=3/2$.\\

4. $\frac{x^{2}}{4} - \frac{y^{2}}{5} = 1$, giperbolaning $3x - 2y = 0$ to'g'ri chizig'iga parallel bo'lgan urinmasining tenglamasini tuzing.  \\

5. $32x^{2} + 52xy - 7y^{2} + 180 = 0$ ITECH tenglamasini kanonik shaklga olib keling, tipini aniqlang, qanday geometrik obraz ekanligini ko'rsating, chizmasini eski va yangi koordinatalar sistemasiga nisbatan chizing.  
\end{tabular}
\vspace{1cm}


\begin{tabular}{m{17cm}}
\textbf{98-variant}\\
1. ITECH-ning umumiy tenglamasini koordinata o'qlarini burish bilan soddalashtirish (ITECH-ning umumiy tenglamalari, koordinata o'qin burish formulasi, tenglamani kanonik turga olib kelish).\\

2. Ikki pallali giperboloid Kanonik tenglamasi (giperbolani simmetriya o'qi atrofida aylantirishdan olingan sirt).\\

3. Qutb tenglamasi bilan berilgan egri chiziqning tipini aniqlang: $\rho=\frac{5}{1-\frac{1}{2}\cos\theta}$.\\

4. Koordinata o'qlarini almashtirmasdan ITECH umumiy tenglamasini soddalashtiring, yarim o'qlarini toping: $13x^{2} + 18xy + 37y^{2} - 26x - 18y + 3 = 0$.  \\

5. Fokuslari $F(3;4)$, $F(-3;-4)$ nuqtalarida joylashgan direktrisalari orasidagi masofa 3,6 ga teng bo'lgan giperbolaning tenglamasini tuzing.  
\end{tabular}
\vspace{1cm}


\begin{tabular}{m{17cm}}
\textbf{99-variant}\\
1. Ellipsning polyar koordinatalardagi tenglamasi (polyar koordinatalar sistemasida ellipsning tenglamasi).\\

2. ITECH-ning invariantlari (ITECH-ning umumiy tenglamasi, almashtirish, ITECH invariantlari).\\

3. Tipini aniqlang: $4x^2+9y^2-40x+36y+100=0$.\\

4. $x^{2} - 4y^{2} = 16$ giperbola berilgan. Uning ekssentrisitetini, fokuslarining koordinatalarini toping va asimptotalarining tenglamalarini tuzing.\\

5. Uchi (-4;0) nuqtasinda, direktrisasi $y - 2 = 0$ to'g'ri chiziq bo'lgan parabolaning tenglamasini tuzing.
\end{tabular}
\vspace{1cm}


\begin{tabular}{m{17cm}}
\textbf{100-variant}\\
1. Bir pallali giperboloid. Kanonik tenglamasi (giperbolani simmetriya o'qi atrofida aylantirishdan olingan sirt).\\

2. Ellipsoida. Kanonik tenglamasi (ellipsni simmetriya o'qi atrofida aylantirishdan olingan sirt, kanonik tenglamasi).\\

3. Aylananing $C$ markazi va $R$ radiusini toping: $x^2+y^2-2x+4y-14=0$.\\

4. $3x + 4y - 12 = 0$ to'g'ri chizig'i bilan $y^{2} = - 9x$ parabolasining kesishish nuqtalarini toping.  \\

5. $M(2; - \frac{5}{3})$ nuqta $\frac{x^{2}}{9} + \frac{y^{2}}{5} = 1$ ellipsda joylashgan. $M$ nuqtaning fokal radiuslarida yotuvchi to'g'ri chiziq tenglamalarini tuzing.  
\end{tabular}
\vspace{1cm}



\end{document}
