\documentclass{article}
\usepackage[utf8]{inputenc}
\usepackage{array}
\usepackage[a4paper,
  left=15mm,
  top=15mm,]{geometry}
\begin{document}

\large
\pagenumbering{gobble}


\begin{tabular}{m{17cm}}
\textbf{1-variant}\\
1. Vektorlardıń skalyar kòbeymesi hàm qàsiyetleri. Skalyar kòbeymeniń koordinatalardaǵı ańlatpası.\\

2. Keńislikte tuwrınıń har tùrli teńlemeleri. \\

3. Tegislikte $A: (-2, 0)$, $B: (3, 2)$, $C: (-4, -3)$ noqatları berilgen. $\overline{AB}, \overline{AC}, \overline{BC}$ vektorardı dúziń; $AB, AC, BC$ uzınlıqlardı tabıń; $\Delta ABC$ úshmúyeshliktiń maydanın esaplań. \\

4. Berilgen ekinshi tártipli iymek sızıqtıń teńlemesin ápiwayılastırıń, tipin anıqlań, orayın hám yarım kósherlerin tabıń: $4x^2+9y^2+8x-72y+112=0$\\

5. \(\overline{a} = \langle 4, 0, -2 \rangle, \quad \overline{b} = \langle -1, 3, 5 \rangle, \quad \overline{c} = \langle 2, 1, -3 \rangle\) vektorları berilgen; \(\overline{a}\) hám \(\overline{b}\) vektorlardan dúzilgen úshmúyeshliktiń maydanın hám usı úsh vektordan dúzilgen parallelepipedtiń kólemin tabıń.
\end{tabular}
\vspace{1cm}


\begin{tabular}{m{17cm}}
\textbf{2-variant}\\
1. Ekinshi tàrtipli sızıq hàm tuwrınıń òzara jaylasıwı. Ekinshi tàrtipli sızıqlardıń urınbası.\\

2. Vektorlardıń vektorlıq kòbeymesi hàm qàsiyetleri. Vektorlıq kòbeymeniń koordinatalardaǵı ańlatpası. \\

3. Tegislikte $A: (-3, 7)$, $B: (4, 3)$, $C: (-1, -4)$ noqatları berilgen. $\overline{AB}, \overline{AC}, \overline{BC}$ vektorardı dúziń; $AB, AC, BC$ uzınlıqlardı tabıń; $\Delta ABC$ úshmúyeshliktiń maydanın esaplań. \\

4. Berilgen ekinshi tártipli iymek sızıqtıń teńlemesin ápiwayılastırıń, tipin anıqlań, orayın hám yarım kósherlerin tabıń: $9x^2+y^2-72x+4y+139=0$\\

5. \(\overline{a} = \langle 5, -2, 1 \rangle, \quad \overline{b} = \langle 0, 3, -4 \rangle, \quad \overline{c} = \langle -3, 1, 6 \rangle\) vektorları berilgen; \(\overline{a}\) hám \(\overline{b}\) vektorlardan dúzilgen úshmúyeshliktiń maydanın hám usı úsh vektordan dúzilgen parallelepipedtiń kólemin tabıń.
\end{tabular}
\vspace{1cm}


\begin{tabular}{m{17cm}}
\textbf{3-variant}\\
1. Tegislikte ekinshi tàrtipli sızıqlardıń kanonikalıq teńlemeleri.\\

2. Ellips hàm onıń kanonikalıq teńlemesi, ekcentrisiteti, grafigi.\\

3. Tegislikte $A: (2, -1)$, $B: (0, 4)$, $C: (3, 0)$ noqatları berilgen. $\overline{AB}, \overline{AC}, \overline{BC}$ vektorardı dúziń; $AB, AC, BC$ uzınlıqlardı tabıń; $\Delta ABC$ úshmúyeshliktiń maydanın esaplań. \\

4. Berilgen ekinshi tártipli iymek sızıqtıń teńlemesin ápiwayılastırıń, tipin anıqlań, orayın hám yarım kósherlerin tabıń: $16x^2+25y^2+64x-150y-111=0$\\

5. \(\overline{a} = \langle -3, 2, -4 \rangle, \quad \overline{b} = \langle 1, 5, 0 \rangle, \quad \overline{c} = \langle 2, -1, 3 \rangle\) vektorları berilgen; \(\overline{a}\) hám \(\overline{b}\) vektorlardan dúzilgen úshmúyeshliktiń maydanın hám usı úsh vektordan dúzilgen parallelepipedtiń kólemin tabıń.
\end{tabular}
\vspace{1cm}


\begin{tabular}{m{17cm}}
\textbf{4-variant}\\
1. Vektorlar ùstinde sızıqlı àmeller.\\

2. Tegisliktiń tùrli teńlemeleri. Tegisliklerdiń òzara jaylasıwı.\\

3. Tegislikte $A: (0, 0)$, $B: (1, 1)$, $C: (-1, 2)$ noqatları berilgen. $\overline{AB}, \overline{AC}, \overline{BC}$ vektorardı dúziń; $AB, AC, BC$ uzınlıqlardı tabıń; $\Delta ABC$ úshmúyeshliktiń maydanın esaplań. \\

4. Berilgen ekinshi tártipli iymek sızıqtıń teńlemesin ápiwayılastırıń, tipin anıqlań, orayın hám yarım kósherlerin tabıń: $25x^2+9y^2+150x-36y+36=0$\\

5. \(\overline{a} = \langle -1, 4, 2 \rangle, \quad \overline{b} = \langle 3, 0, -5 \rangle, \quad \overline{c} = \langle 2, -3, 1 \rangle\) vektorları berilgen; \(\overline{a}\) hám \(\overline{b}\) vektorlardan dúzilgen úshmúyeshliktiń maydanın hám usı úsh vektordan dúzilgen parallelepipedtiń kólemin tabıń.
\end{tabular}
\vspace{1cm}


\begin{tabular}{m{17cm}}
\textbf{5-variant}\\
1. Ekinshi tàrtipli sızıqlardıń ulıwma teńlemeleri, orayı. Orayǵa iye hàm oraysız sızıqlar.\\

2. Vektorlardıń aralas kòbeymesi hàm qàsiyetleri. Parallelepiped kòlemi\\

3. Tegislikte $A: (-1, -3)$, $B: (-4, 6)$, $C: (3, 2)$ noqatları berilgen. $\overline{AB}, \overline{AC}, \overline{BC}$ vektorardı dúziń; $AB, AC, BC$ uzınlıqlardı tabıń; $\Delta ABC$ úshmúyeshliktiń maydanın esaplań. \\

4. Berilgen ekinshi tártipli iymek sızıqtıń teńlemesin ápiwayılastırıń, tipin anıqlań, orayın hám yarım kósherlerin tabıń: $x^2+4y^2+10x+40y+121=0$\\

5. \(\overline{a} = \langle -3, 2, -4 \rangle, \quad \overline{b} = \langle 1, 5, 0 \rangle, \quad \overline{c} = \langle 2, -1, 3 \rangle\) vektorları berilgen; \(\overline{a}\) hám \(\overline{b}\) vektorlardan dúzilgen úshmúyeshliktiń maydanın hám usı úsh vektordan dúzilgen parallelepipedtiń kólemin tabıń.
\end{tabular}
\vspace{1cm}


\begin{tabular}{m{17cm}}
\textbf{6-variant}\\
1. Eki tegisliktiń parallellik, perpendikulyarlıq shàrtleri.\\

2. Ekinshi tàrtipli sızıqlardıń invariyantları.\\

3. Tegislikte $A: (8, 1)$, $B: (2, -3)$, $C: (-6, 5)$ noqatları berilgen. $\overline{AB}, \overline{AC}, \overline{BC}$ vektorardı dúziń; $AB, AC, BC$ uzınlıqlardı tabıń; $\Delta ABC$ úshmúyeshliktiń maydanın esaplań. \\

4. Berilgen ekinshi tártipli iymek sızıqtıń teńlemesin ápiwayılastırıń, tipin anıqlań, orayın hám yarım kósherlerin tabıń: $9x^2+y^2+90x+2y+217=0$\\

5. \(\overline{a} = \langle 1, 2, -3 \rangle, \quad \overline{b} = \langle -2, 3, 0 \rangle, \quad \overline{c} = \langle 4, -1, 2 \rangle\) vektorları berilgen; \(\overline{a}\) hám \(\overline{b}\) vektorlardan dúzilgen úshmúyeshliktiń maydanın hám usı úsh vektordan dúzilgen parallelepipedtiń kólemin tabıń.
\end{tabular}
\vspace{1cm}


\begin{tabular}{m{17cm}}
\textbf{7-variant}\\
1. Tuwrı hàm tegisliktiń òzara jaylasıwı. Ayqasıwshı tuwrılar.\\

2. Ekinshi tàrtipli sızıqlardıń ulıwma teńlemesin kanonikalıq kòriniske keltiriw usılları.\\

3. Tegislikte $A: (4, 8)$, $B: (0, -6)$, $C: (-7, 3)$ noqatları berilgen. $\overline{AB}, \overline{AC}, \overline{BC}$ vektorardı dúziń; $AB, AC, BC$ uzınlıqlardı tabıń; $\Delta ABC$ úshmúyeshliktiń maydanın esaplań. \\

4. Berilgen ekinshi tártipli iymek sızıqtıń teńlemesin ápiwayılastırıń, tipin anıqlań, orayın hám yarım kósherlerin tabıń: $4x^2+4y^2+40x-8y+88=0$\\

5. \(\overline{a} = \langle 0, 1, 3 \rangle, \quad \overline{b} = \langle 4, -2, 0 \rangle, \quad \overline{c} = \langle -1, 2, -5 \rangle\) vektorları berilgen; \(\overline{a}\) hám \(\overline{b}\) vektorlardan dúzilgen úshmúyeshliktiń maydanın hám usı úsh vektordan dúzilgen parallelepipedtiń kólemin tabıń.
\end{tabular}
\vspace{1cm}


\begin{tabular}{m{17cm}}
\textbf{8-variant}\\
1. Giperbola hàm parabola. Kanonikalıq teńlemeleri, ekcentrisiteti, qàsiyetleri, grafikleri.\\

2. Vektorlardıń skalyar kòbeymesi hàm qàsiyetleri. Skalyar kòbeymeniń koordinatalardaǵı ańlatpası.\\

3. Tegislikte $A: (0, 8)$, $B: (-2, -4)$, $C: (5, 1)$ noqatları berilgen. $\overline{AB}, \overline{AC}, \overline{BC}$ vektorardı dúziń; $AB, AC, BC$ uzınlıqlardı tabıń; $\Delta ABC$ úshmúyeshliktiń maydanın esaplań. \\

4. Berilgen ekinshi tártipli iymek sızıqtıń teńlemesin ápiwayılastırıń, tipin anıqlań, orayın hám yarım kósherlerin tabıń: $4x^2+9y^2+32x-18y+37=0$\\

5. \(\overline{a} = \langle 3, 2, 1 \rangle, \quad \overline{b} = \langle -2, 1, -4 \rangle, \quad \overline{c} = \langle 5, 0, 3 \rangle\) vektorları berilgen; \(\overline{a}\) hám \(\overline{b}\) vektorlardan dúzilgen úshmúyeshliktiń maydanın hám usı úsh vektordan dúzilgen parallelepipedtiń kólemin tabıń.
\end{tabular}
\vspace{1cm}


\begin{tabular}{m{17cm}}
\textbf{9-variant}\\
1. Keńislikte tuwrınıń har tùrli teńlemeleri. \\

2. Ekinshi tàrtipli sızıq hàm tuwrınıń òzara jaylasıwı. Ekinshi tàrtipli sızıqlardıń urınbası.\\

3. Tegislikte $A: (-3, -2)$, $B: (-5, 4)$, $C: (6, -1)$ noqatları berilgen. $\overline{AB}, \overline{AC}, \overline{BC}$ vektorardı dúziń; $AB, AC, BC$ uzınlıqlardı tabıń; $\Delta ABC$ úshmúyeshliktiń maydanın esaplań. \\

4. Berilgen ekinshi tártipli iymek sızıqtıń teńlemesin ápiwayılastırıń, tipin anıqlań, orayın hám yarım kósherlerin tabıń: $16x^2+y^2+128x-4y+244=0$\\

5. \(\overline{a} = \langle 2, -1, 3 \rangle, \quad \overline{b} = \langle -4, 5, 1 \rangle, \quad \overline{c} = \langle 0, 2, -3 \rangle\) vektorları berilgen; \(\overline{a}\) hám \(\overline{b}\) vektorlardan dúzilgen úshmúyeshliktiń maydanın hám usı úsh vektordan dúzilgen parallelepipedtiń kólemin tabıń.
\end{tabular}
\vspace{1cm}


\begin{tabular}{m{17cm}}
\textbf{10-variant}\\
1. Vektorlardıń vektorlıq kòbeymesi hàm qàsiyetleri. Vektorlıq kòbeymeniń koordinatalardaǵı ańlatpası. \\

2. Tegislikte ekinshi tàrtipli sızıqlardıń kanonikalıq teńlemeleri.\\

3. Tegislikte $A: (1, 2)$, $B: (-2, 6)$, $C: (4, -3)$ noqatları berilgen. $\overline{AB}, \overline{AC}, \overline{BC}$ vektorardı dúziń; $AB, AC, BC$ uzınlıqlardı tabıń; $\Delta ABC$ úshmúyeshliktiń maydanın esaplań. \\

4. Berilgen ekinshi tártipli iymek sızıqtıń teńlemesin ápiwayılastırıń, tipin anıqlań, orayın hám yarım kósherlerin tabıń: $4x^2+y^2+16x-2y+13=0$\\

5. \(\overline{a} = \langle 1, 3, -2 \rangle, \quad \overline{b} = \langle -4, 2, 1 \rangle, \quad \overline{c} = \langle 2, -5, 0 \rangle\) vektorları berilgen; \(\overline{a}\) hám \(\overline{b}\) vektorlardan dúzilgen úshmúyeshliktiń maydanın hám usı úsh vektordan dúzilgen parallelepipedtiń kólemin tabıń.
\end{tabular}
\vspace{1cm}


\begin{tabular}{m{17cm}}
\textbf{11-variant}\\
1. Ellips hàm onıń kanonikalıq teńlemesi, ekcentrisiteti, grafigi.\\

2. Vektorlar ùstinde sızıqlı àmeller.\\

3. Tegislikte $A: (6, -2)$, $B: (2, 1)$, $C: (-2, 5)$ noqatları berilgen. $\overline{AB}, \overline{AC}, \overline{BC}$ vektorardı dúziń; $AB, AC, BC$ uzınlıqlardı tabıń; $\Delta ABC$ úshmúyeshliktiń maydanın esaplań. \\

4. Berilgen ekinshi tártipli iymek sızıqtıń teńlemesin ápiwayılastırıń, tipin anıqlań, orayın hám yarım kósherlerin tabıń: $4x^2+4y^2+16x+16y+16=0$\\

5. \(\overline{a} = \langle 2, -1, 3 \rangle, \quad \overline{b} = \langle -4, 5, 1 \rangle, \quad \overline{c} = \langle 0, 2, -3 \rangle\) vektorları berilgen; \(\overline{a}\) hám \(\overline{b}\) vektorlardan dúzilgen úshmúyeshliktiń maydanın hám usı úsh vektordan dúzilgen parallelepipedtiń kólemin tabıń.
\end{tabular}
\vspace{1cm}


\begin{tabular}{m{17cm}}
\textbf{12-variant}\\
1. Tegisliktiń tùrli teńlemeleri. Tegisliklerdiń òzara jaylasıwı.\\

2. Ekinshi tàrtipli sızıqlardıń ulıwma teńlemeleri, orayı. Orayǵa iye hàm oraysız sızıqlar.\\

3. Tegislikte $A: (2, 5)$, $B: (7, -3)$, $C: (-4, 9)$ noqatları berilgen. $\overline{AB}, \overline{AC}, \overline{BC}$ vektorardı dúziń; $AB, AC, BC$ uzınlıqlardı tabıń; $\Delta ABC$ úshmúyeshliktiń maydanın esaplań. \\

4. Berilgen ekinshi tártipli iymek sızıqtıń teńlemesin ápiwayılastırıń, tipin anıqlań, orayın hám yarım kósherlerin tabıń: $9x^2+4y^2-36x-16y+16=0$\\

5. \(\overline{a} = \langle 3, 2, 1 \rangle, \quad \overline{b} = \langle -2, 1, -4 \rangle, \quad \overline{c} = \langle 5, 0, 3 \rangle\) vektorları berilgen; \(\overline{a}\) hám \(\overline{b}\) vektorlardan dúzilgen úshmúyeshliktiń maydanın hám usı úsh vektordan dúzilgen parallelepipedtiń kólemin tabıń.
\end{tabular}
\vspace{1cm}


\begin{tabular}{m{17cm}}
\textbf{13-variant}\\
1. Vektorlardıń aralas kòbeymesi hàm qàsiyetleri. Parallelepiped kòlemi\\

2. Eki tegisliktiń parallellik, perpendikulyarlıq shàrtleri.\\

3. Tegislikte $A: (2, -5)$, $B: (6, 3)$, $C: (-3, 0)$ noqatları berilgen. $\overline{AB}, \overline{AC}, \overline{BC}$ vektorardı dúziń; $AB, AC, BC$ uzınlıqlardı tabıń; $\Delta ABC$ úshmúyeshliktiń maydanın esaplań. \\

4. Berilgen ekinshi tártipli iymek sızıqtıń teńlemesin ápiwayılastırıń, tipin anıqlań, orayın hám yarım kósherlerin tabıń: $x^2+16y^2+8x+160y+400=0$\\

5. \(\overline{a} = \langle 4, 0, -2 \rangle, \quad \overline{b} = \langle -1, 3, 5 \rangle, \quad \overline{c} = \langle 2, 1, -3 \rangle\) vektorları berilgen; \(\overline{a}\) hám \(\overline{b}\) vektorlardan dúzilgen úshmúyeshliktiń maydanın hám usı úsh vektordan dúzilgen parallelepipedtiń kólemin tabıń.
\end{tabular}
\vspace{1cm}


\begin{tabular}{m{17cm}}
\textbf{14-variant}\\
1. Ekinshi tàrtipli sızıqlardıń invariyantları.\\

2. Tuwrı hàm tegisliktiń òzara jaylasıwı. Ayqasıwshı tuwrılar.\\

3. Tegislikte $A: (7, 6)$, $B: (1, -2)$, $C: (-5, 1)$ noqatları berilgen. $\overline{AB}, \overline{AC}, \overline{BC}$ vektorardı dúziń; $AB, AC, BC$ uzınlıqlardı tabıń; $\Delta ABC$ úshmúyeshliktiń maydanın esaplań. \\

4. Berilgen ekinshi tártipli iymek sızıqtıń teńlemesin ápiwayılastırıń, tipin anıqlań, orayın hám yarım kósherlerin tabıń: $9x^2+4y^2-54x+32y+109=0$\\

5. \(\overline{a} = \langle 0, 1, 3 \rangle, \quad \overline{b} = \langle 4, -2, 0 \rangle, \quad \overline{c} = \langle -1, 2, -5 \rangle\) vektorları berilgen; \(\overline{a}\) hám \(\overline{b}\) vektorlardan dúzilgen úshmúyeshliktiń maydanın hám usı úsh vektordan dúzilgen parallelepipedtiń kólemin tabıń.
\end{tabular}
\vspace{1cm}


\begin{tabular}{m{17cm}}
\textbf{15-variant}\\
1. Ekinshi tàrtipli sızıqlardıń ulıwma teńlemesin kanonikalıq kòriniske keltiriw usılları.\\

2. Giperbola hàm parabola. Kanonikalıq teńlemeleri, ekcentrisiteti, qàsiyetleri, grafikleri.\\

3. Tegislikte $A: (5, 3)$, $B: (-1, 0)$, $C: (2, 7)$ noqatları berilgen. $\overline{AB}, \overline{AC}, \overline{BC}$ vektorardı dúziń; $AB, AC, BC$ uzınlıqlardı tabıń; $\Delta ABC$ úshmúyeshliktiń maydanın esaplań. \\

4. Berilgen ekinshi tártipli iymek sızıqtıń teńlemesin ápiwayılastırıń, tipin anıqlań, orayın hám yarım kósherlerin tabıń: $25x^2+4y^2-50x-8y-71=0$\\

5. \(\overline{a} = \langle 2, -4, 1 \rangle, \quad \overline{b} = \langle 0, 3, -2 \rangle, \quad \overline{c} = \langle -3, 1, 5 \rangle\) vektorları berilgen; \(\overline{a}\) hám \(\overline{b}\) vektorlardan dúzilgen úshmúyeshliktiń maydanın hám usı úsh vektordan dúzilgen parallelepipedtiń kólemin tabıń.
\end{tabular}
\vspace{1cm}


\begin{tabular}{m{17cm}}
\textbf{16-variant}\\
1. Vektorlardıń skalyar kòbeymesi hàm qàsiyetleri. Skalyar kòbeymeniń koordinatalardaǵı ańlatpası.\\

2. Keńislikte tuwrınıń har tùrli teńlemeleri. \\

3. Tegislikte $A: (-1, 3)$, $B: (4, 6)$, $C: (-2, -5)$ noqatları berilgen. $\overline{AB}, \overline{AC}, \overline{BC}$ vektorardı dúziń; $AB, AC, BC$ uzınlıqlardı tabıń; $\Delta ABC$ úshmúyeshliktiń maydanın esaplań. \\

4. Berilgen ekinshi tártipli iymek sızıqtıń teńlemesin ápiwayılastırıń, tipin anıqlań, orayın hám yarım kósherlerin tabıń: $16x^2+y^2+128x-2y+241=0$\\

5. \(\overline{a} = \langle -1, 4, 2 \rangle, \quad \overline{b} = \langle 3, 0, -5 \rangle, \quad \overline{c} = \langle 2, -3, 1 \rangle\) vektorları berilgen; \(\overline{a}\) hám \(\overline{b}\) vektorlardan dúzilgen úshmúyeshliktiń maydanın hám usı úsh vektordan dúzilgen parallelepipedtiń kólemin tabıń.
\end{tabular}
\vspace{1cm}


\begin{tabular}{m{17cm}}
\textbf{17-variant}\\
1. Ekinshi tàrtipli sızıq hàm tuwrınıń òzara jaylasıwı. Ekinshi tàrtipli sızıqlardıń urınbası.\\

2. Vektorlardıń vektorlıq kòbeymesi hàm qàsiyetleri. Vektorlıq kòbeymeniń koordinatalardaǵı ańlatpası. \\

3. Tegislikte $A: (-5, 0)$, $B: (3, 4)$, $C: (-1, 7)$ noqatları berilgen. $\overline{AB}, \overline{AC}, \overline{BC}$ vektorardı dúziń; $AB, AC, BC$ uzınlıqlardı tabıń; $\Delta ABC$ úshmúyeshliktiń maydanın esaplań. \\

4. Berilgen ekinshi tártipli iymek sızıqtıń teńlemesin ápiwayılastırıń, tipin anıqlań, orayın hám yarım kósherlerin tabıń: $4x^2+16y^2-16x-96y+96=0$\\

5. \(\overline{a} = \langle -2, 1, 0 \rangle, \quad \overline{b} = \langle 3, -4, 2 \rangle, \quad \overline{c} = \langle 1, 0, 5 \rangle\) vektorları berilgen; \(\overline{a}\) hám \(\overline{b}\) vektorlardan dúzilgen úshmúyeshliktiń maydanın hám usı úsh vektordan dúzilgen parallelepipedtiń kólemin tabıń.
\end{tabular}
\vspace{1cm}


\begin{tabular}{m{17cm}}
\textbf{18-variant}\\
1. Tegislikte ekinshi tàrtipli sızıqlardıń kanonikalıq teńlemeleri.\\

2. Ellips hàm onıń kanonikalıq teńlemesi, ekcentrisiteti, grafigi.\\

3. Tegislikte $A: (3, -4)$, $B: (-2, 7)$, $C: (5, 2)$ noqatları berilgen. $\overline{AB}, \overline{AC}, \overline{BC}$ vektorardı dúziń; $AB, AC, BC$ uzınlıqlardı tabıń; $\Delta ABC$ úshmúyeshliktiń maydanın esaplań. \\

4. Berilgen ekinshi tártipli iymek sızıqtıń teńlemesin ápiwayılastırıń, tipin anıqlań, orayın hám yarım kósherlerin tabıń: $x^2+y^2+10x+2y+25=0$\\

5. \(\overline{a} = \langle 3, 0, -2 \rangle, \quad \overline{b} = \langle -1, 2, 4 \rangle, \quad \overline{c} = \langle 2, -3, 1 \rangle\) vektorları berilgen; \(\overline{a}\) hám \(\overline{b}\) vektorlardan dúzilgen úshmúyeshliktiń maydanın hám usı úsh vektordan dúzilgen parallelepipedtiń kólemin tabıń.
\end{tabular}
\vspace{1cm}


\begin{tabular}{m{17cm}}
\textbf{19-variant}\\
1. Vektorlar ùstinde sızıqlı àmeller.\\

2. Tegisliktiń tùrli teńlemeleri. Tegisliklerdiń òzara jaylasıwı.\\

3. Tegislikte $A: (1, 6)$, $B: (-3, -1)$, $C: (7, 4)$ noqatları berilgen. $\overline{AB}, \overline{AC}, \overline{BC}$ vektorardı dúziń; $AB, AC, BC$ uzınlıqlardı tabıń; $\Delta ABC$ úshmúyeshliktiń maydanın esaplań. \\

4. Berilgen ekinshi tártipli iymek sızıqtıń teńlemesin ápiwayılastırıń, tipin anıqlań, orayın hám yarım kósherlerin tabıń: $25x^2+y^2+150x+6y+209=0$\\

5. \(\overline{a} = \langle 1, 3, -2 \rangle, \quad \overline{b} = \langle -4, 2, 1 \rangle, \quad \overline{c} = \langle 2, -5, 0 \rangle\) vektorları berilgen; \(\overline{a}\) hám \(\overline{b}\) vektorla dúzilgen úshmúyeshliktiń maydanın hám usı úsh vektordan dúzilgen parallelepipedtiń kólemin tabıń.
\end{tabular}
\vspace{1cm}


\begin{tabular}{m{17cm}}
\textbf{20-variant}\\
1. Ekinshi tàrtipli sızıqlardıń ulıwma teńlemeleri, orayı. Orayǵa iye hàm oraysız sızıqlar.\\

2. Vektorlardıń aralas kòbeymesi hàm qàsiyetleri. Parallelepiped kòlemi\\

3. Tegislikte $A: (-4, 2)$, $B: (-3, -5)$, $C: (6, 8)$ noqatları berilgen. $\overline{AB}, \overline{AC}, \overline{BC}$ vektorardı dúziń; $AB, AC, BC$ uzınlıqlardı tabıń; $\Delta ABC$ úshmúyeshliktiń maydanın esaplań. \\

4. Berilgen ekinshi tártipli iymek sızıqtıń teńlemesin ápiwayılastırıń, tipin anıqlań, orayın hám yarım kósherlerin tabıń: $x^2+16y^2+10x-32y+25=0$\\

5. \(\overline{a} = \langle 5, -2, 1 \rangle, \quad \overline{b} = \langle 0, 3, -4 \rangle, \quad \overline{c} = \langle -3, 1, 6 \rangle\) vektorları berilgen; \(\overline{a}\) hám \(\overline{b}\) vektorlardan dúzilgen úshmúyeshliktiń maydanın hám usı úsh vektordan dúzilgen parallelepipedtiń kólemin tabıń.
\end{tabular}
\vspace{1cm}


\begin{tabular}{m{17cm}}
\textbf{21-variant}\\
1. Eki tegisliktiń parallellik, perpendikulyarlıq shàrtleri.\\

2. Ekinshi tàrtipli sızıqlardıń invariyantları.\\

3. Tegislikte $A: (-2, 0)$, $B: (3, 2)$, $C: (-4, -3)$ noqatları berilgen. $\overline{AB}, \overline{AC}, \overline{BC}$ vektorardı dúziń; $AB, AC, BC$ uzınlıqlardı tabıń; $\Delta ABC$ úshmúyeshliktiń maydanın esaplań. \\

4. Berilgen ekinshi tártipli iymek sızıqtıń teńlemesin ápiwayılastırıń, tipin anıqlań, orayın hám yarım kósherlerin tabıń: $4x^2+9y^2+8x-72y+112=0$\\

5. \(\overline{a} = \langle 4, 0, -2 \rangle, \quad \overline{b} = \langle -1, 3, 5 \rangle, \quad \overline{c} = \langle 2, 1, -3 \rangle\) vektorları berilgen; \(\overline{a}\) hám \(\overline{b}\) vektorlardan dúzilgen úshmúyeshliktiń maydanın hám usı úsh vektordan dúzilgen parallelepipedtiń kólemin tabıń.
\end{tabular}
\vspace{1cm}


\begin{tabular}{m{17cm}}
\textbf{22-variant}\\
1. Tuwrı hàm tegisliktiń òzara jaylasıwı. Ayqasıwshı tuwrılar.\\

2. Ekinshi tàrtipli sızıqlardıń ulıwma teńlemesin kanonikalıq kòriniske keltiriw usılları.\\

3. Tegislikte $A: (-3, 7)$, $B: (4, 3)$, $C: (-1, -4)$ noqatları berilgen. $\overline{AB}, \overline{AC}, \overline{BC}$ vektorardı dúziń; $AB, AC, BC$ uzınlıqlardı tabıń; $\Delta ABC$ úshmúyeshliktiń maydanın esaplań. \\

4. Berilgen ekinshi tártipli iymek sızıqtıń teńlemesin ápiwayılastırıń, tipin anıqlań, orayın hám yarım kósherlerin tabıń: $9x^2+y^2-72x+4y+139=0$\\

5. \(\overline{a} = \langle 5, -2, 1 \rangle, \quad \overline{b} = \langle 0, 3, -4 \rangle, \quad \overline{c} = \langle -3, 1, 6 \rangle\) vektorları berilgen; \(\overline{a}\) hám \(\overline{b}\) vektorlardan dúzilgen úshmúyeshliktiń maydanın hám usı úsh vektordan dúzilgen parallelepipedtiń kólemin tabıń.
\end{tabular}
\vspace{1cm}


\begin{tabular}{m{17cm}}
\textbf{23-variant}\\
1. Giperbola hàm parabola. Kanonikalıq teńlemeleri, ekcentrisiteti, qàsiyetleri, grafikleri.\\

2. Vektorlardıń skalyar kòbeymesi hàm qàsiyetleri. Skalyar kòbeymeniń koordinatalardaǵı ańlatpası.\\

3. Tegislikte $A: (2, -1)$, $B: (0, 4)$, $C: (3, 0)$ noqatları berilgen. $\overline{AB}, \overline{AC}, \overline{BC}$ vektorardı dúziń; $AB, AC, BC$ uzınlıqlardı tabıń; $\Delta ABC$ úshmúyeshliktiń maydanın esaplań. \\

4. Berilgen ekinshi tártipli iymek sızıqtıń teńlemesin ápiwayılastırıń, tipin anıqlań, orayın hám yarım kósherlerin tabıń: $16x^2+25y^2+64x-150y-111=0$\\

5. \(\overline{a} = \langle -3, 2, -4 \rangle, \quad \overline{b} = \langle 1, 5, 0 \rangle, \quad \overline{c} = \langle 2, -1, 3 \rangle\) vektorları berilgen; \(\overline{a}\) hám \(\overline{b}\) vektorlardan dúzilgen úshmúyeshliktiń maydanın hám usı úsh vektordan dúzilgen parallelepipedtiń kólemin tabıń.
\end{tabular}
\vspace{1cm}


\begin{tabular}{m{17cm}}
\textbf{24-variant}\\
1. Keńislikte tuwrınıń har tùrli teńlemeleri. \\

2. Ekinshi tàrtipli sızıq hàm tuwrınıń òzara jaylasıwı. Ekinshi tàrtipli sızıqlardıń urınbası.\\

3. Tegislikte $A: (0, 0)$, $B: (1, 1)$, $C: (-1, 2)$ noqatları berilgen. $\overline{AB}, \overline{AC}, \overline{BC}$ vektorardı dúziń; $AB, AC, BC$ uzınlıqlardı tabıń; $\Delta ABC$ úshmúyeshliktiń maydanın esaplań. \\

4. Berilgen ekinshi tártipli iymek sızıqtıń teńlemesin ápiwayılastırıń, tipin anıqlań, orayın hám yarım kósherlerin tabıń: $25x^2+9y^2+150x-36y+36=0$\\

5. \(\overline{a} = \langle -1, 4, 2 \rangle, \quad \overline{b} = \langle 3, 0, -5 \rangle, \quad \overline{c} = \langle 2, -3, 1 \rangle\) vektorları berilgen; \(\overline{a}\) hám \(\overline{b}\) vektorlardan dúzilgen úshmúyeshliktiń maydanın hám usı úsh vektordan dúzilgen parallelepipedtiń kólemin tabıń.
\end{tabular}
\vspace{1cm}


\begin{tabular}{m{17cm}}
\textbf{25-variant}\\
1. Vektorlardıń vektorlıq kòbeymesi hàm qàsiyetleri. Vektorlıq kòbeymeniń koordinatalardaǵı ańlatpası. \\

2. Tegislikte ekinshi tàrtipli sızıqlardıń kanonikalıq teńlemeleri.\\

3. Tegislikte $A: (-1, -3)$, $B: (-4, 6)$, $C: (3, 2)$ noqatları berilgen. $\overline{AB}, \overline{AC}, \overline{BC}$ vektorardı dúziń; $AB, AC, BC$ uzınlıqlardı tabıń; $\Delta ABC$ úshmúyeshliktiń maydanın esaplań. \\

4. Berilgen ekinshi tártipli iymek sızıqtıń teńlemesin ápiwayılastırıń, tipin anıqlań, orayın hám yarım kósherlerin tabıń: $x^2+4y^2+10x+40y+121=0$\\

5. \(\overline{a} = \langle -3, 2, -4 \rangle, \quad \overline{b} = \langle 1, 5, 0 \rangle, \quad \overline{c} = \langle 2, -1, 3 \rangle\) vektorları berilgen; \(\overline{a}\) hám \(\overline{b}\) vektorlardan dúzilgen úshmúyeshliktiń maydanın hám usı úsh vektordan dúzilgen parallelepipedtiń kólemin tabıń.
\end{tabular}
\vspace{1cm}


\begin{tabular}{m{17cm}}
\textbf{26-variant}\\
1. Ellips hàm onıń kanonikalıq teńlemesi, ekcentrisiteti, grafigi.\\

2. Vektorlar ùstinde sızıqlı àmeller.\\

3. Tegislikte $A: (8, 1)$, $B: (2, -3)$, $C: (-6, 5)$ noqatları berilgen. $\overline{AB}, \overline{AC}, \overline{BC}$ vektorardı dúziń; $AB, AC, BC$ uzınlıqlardı tabıń; $\Delta ABC$ úshmúyeshliktiń maydanın esaplań. \\

4. Berilgen ekinshi tártipli iymek sızıqtıń teńlemesin ápiwayılastırıń, tipin anıqlań, orayın hám yarım kósherlerin tabıń: $9x^2+y^2+90x+2y+217=0$\\

5. \(\overline{a} = \langle 1, 2, -3 \rangle, \quad \overline{b} = \langle -2, 3, 0 \rangle, \quad \overline{c} = \langle 4, -1, 2 \rangle\) vektorları berilgen; \(\overline{a}\) hám \(\overline{b}\) vektorlardan dúzilgen úshmúyeshliktiń maydanın hám usı úsh vektordan dúzilgen parallelepipedtiń kólemin tabıń.
\end{tabular}
\vspace{1cm}


\begin{tabular}{m{17cm}}
\textbf{27-variant}\\
1. Tegisliktiń tùrli teńlemeleri. Tegisliklerdiń òzara jaylasıwı.\\

2. Ekinshi tàrtipli sızıqlardıń ulıwma teńlemeleri, orayı. Orayǵa iye hàm oraysız sızıqlar.\\

3. Tegislikte $A: (4, 8)$, $B: (0, -6)$, $C: (-7, 3)$ noqatları berilgen. $\overline{AB}, \overline{AC}, \overline{BC}$ vektorardı dúziń; $AB, AC, BC$ uzınlıqlardı tabıń; $\Delta ABC$ úshmúyeshliktiń maydanın esaplań. \\

4. Berilgen ekinshi tártipli iymek sızıqtıń teńlemesin ápiwayılastırıń, tipin anıqlań, orayın hám yarım kósherlerin tabıń: $4x^2+4y^2+40x-8y+88=0$\\

5. \(\overline{a} = \langle 0, 1, 3 \rangle, \quad \overline{b} = \langle 4, -2, 0 \rangle, \quad \overline{c} = \langle -1, 2, -5 \rangle\) vektorları berilgen; \(\overline{a}\) hám \(\overline{b}\) vektorlardan dúzilgen úshmúyeshliktiń maydanın hám usı úsh vektordan dúzilgen parallelepipedtiń kólemin tabıń.
\end{tabular}
\vspace{1cm}


\begin{tabular}{m{17cm}}
\textbf{28-variant}\\
1. Vektorlardıń aralas kòbeymesi hàm qàsiyetleri. Parallelepiped kòlemi\\

2. Eki tegisliktiń parallellik, perpendikulyarlıq shàrtleri.\\

3. Tegislikte $A: (0, 8)$, $B: (-2, -4)$, $C: (5, 1)$ noqatları berilgen. $\overline{AB}, \overline{AC}, \overline{BC}$ vektorardı dúziń; $AB, AC, BC$ uzınlıqlardı tabıń; $\Delta ABC$ úshmúyeshliktiń maydanın esaplań. \\

4. Berilgen ekinshi tártipli iymek sızıqtıń teńlemesin ápiwayılastırıń, tipin anıqlań, orayın hám yarım kósherlerin tabıń: $4x^2+9y^2+32x-18y+37=0$\\

5. \(\overline{a} = \langle 3, 2, 1 \rangle, \quad \overline{b} = \langle -2, 1, -4 \rangle, \quad \overline{c} = \langle 5, 0, 3 \rangle\) vektorları berilgen; \(\overline{a}\) hám \(\overline{b}\) vektorlardan dúzilgen úshmúyeshliktiń maydanın hám usı úsh vektordan dúzilgen parallelepipedtiń kólemin tabıń.
\end{tabular}
\vspace{1cm}


\begin{tabular}{m{17cm}}
\textbf{29-variant}\\
1. Ekinshi tàrtipli sızıqlardıń invariyantları.\\

2. Tuwrı hàm tegisliktiń òzara jaylasıwı. Ayqasıwshı tuwrılar.\\

3. Tegislikte $A: (-3, -2)$, $B: (-5, 4)$, $C: (6, -1)$ noqatları berilgen. $\overline{AB}, \overline{AC}, \overline{BC}$ vektorardı dúziń; $AB, AC, BC$ uzınlıqlardı tabıń; $\Delta ABC$ úshmúyeshliktiń maydanın esaplań. \\

4. Berilgen ekinshi tártipli iymek sızıqtıń teńlemesin ápiwayılastırıń, tipin anıqlań, orayın hám yarım kósherlerin tabıń: $16x^2+y^2+128x-4y+244=0$\\

5. \(\overline{a} = \langle 2, -1, 3 \rangle, \quad \overline{b} = \langle -4, 5, 1 \rangle, \quad \overline{c} = \langle 0, 2, -3 \rangle\) vektorları berilgen; \(\overline{a}\) hám \(\overline{b}\) vektorlardan dúzilgen úshmúyeshliktiń maydanın hám usı úsh vektordan dúzilgen parallelepipedtiń kólemin tabıń.
\end{tabular}
\vspace{1cm}


\begin{tabular}{m{17cm}}
\textbf{30-variant}\\
1. Ekinshi tàrtipli sızıqlardıń ulıwma teńlemesin kanonikalıq kòriniske keltiriw usılları.\\

2. Giperbola hàm parabola. Kanonikalıq teńlemeleri, ekcentrisiteti, qàsiyetleri, grafikleri.\\

3. Tegislikte $A: (1, 2)$, $B: (-2, 6)$, $C: (4, -3)$ noqatları berilgen. $\overline{AB}, \overline{AC}, \overline{BC}$ vektorardı dúziń; $AB, AC, BC$ uzınlıqlardı tabıń; $\Delta ABC$ úshmúyeshliktiń maydanın esaplań. \\

4. Berilgen ekinshi tártipli iymek sızıqtıń teńlemesin ápiwayılastırıń, tipin anıqlań, orayın hám yarım kósherlerin tabıń: $4x^2+y^2+16x-2y+13=0$\\

5. \(\overline{a} = \langle 1, 3, -2 \rangle, \quad \overline{b} = \langle -4, 2, 1 \rangle, \quad \overline{c} = \langle 2, -5, 0 \rangle\) vektorları berilgen; \(\overline{a}\) hám \(\overline{b}\) vektorlardan dúzilgen úshmúyeshliktiń maydanın hám usı úsh vektordan dúzilgen parallelepipedtiń kólemin tabıń.
\end{tabular}
\vspace{1cm}


\begin{tabular}{m{17cm}}
\textbf{31-variant}\\
1. Vektorlardıń skalyar kòbeymesi hàm qàsiyetleri. Skalyar kòbeymeniń koordinatalardaǵı ańlatpası.\\

2. Keńislikte tuwrınıń har tùrli teńlemeleri. \\

3. Tegislikte $A: (6, -2)$, $B: (2, 1)$, $C: (-2, 5)$ noqatları berilgen. $\overline{AB}, \overline{AC}, \overline{BC}$ vektorardı dúziń; $AB, AC, BC$ uzınlıqlardı tabıń; $\Delta ABC$ úshmúyeshliktiń maydanın esaplań. \\

4. Berilgen ekinshi tártipli iymek sızıqtıń teńlemesin ápiwayılastırıń, tipin anıqlań, orayın hám yarım kósherlerin tabıń: $4x^2+4y^2+16x+16y+16=0$\\

5. \(\overline{a} = \langle 2, -1, 3 \rangle, \quad \overline{b} = \langle -4, 5, 1 \rangle, \quad \overline{c} = \langle 0, 2, -3 \rangle\) vektorları berilgen; \(\overline{a}\) hám \(\overline{b}\) vektorlardan dúzilgen úshmúyeshliktiń maydanın hám usı úsh vektordan dúzilgen parallelepipedtiń kólemin tabıń.
\end{tabular}
\vspace{1cm}


\begin{tabular}{m{17cm}}
\textbf{32-variant}\\
1. Ekinshi tàrtipli sızıq hàm tuwrınıń òzara jaylasıwı. Ekinshi tàrtipli sızıqlardıń urınbası.\\

2. Vektorlardıń vektorlıq kòbeymesi hàm qàsiyetleri. Vektorlıq kòbeymeniń koordinatalardaǵı ańlatpası. \\

3. Tegislikte $A: (2, 5)$, $B: (7, -3)$, $C: (-4, 9)$ noqatları berilgen. $\overline{AB}, \overline{AC}, \overline{BC}$ vektorardı dúziń; $AB, AC, BC$ uzınlıqlardı tabıń; $\Delta ABC$ úshmúyeshliktiń maydanın esaplań. \\

4. Berilgen ekinshi tártipli iymek sızıqtıń teńlemesin ápiwayılastırıń, tipin anıqlań, orayın hám yarım kósherlerin tabıń: $9x^2+4y^2-36x-16y+16=0$\\

5. \(\overline{a} = \langle 3, 2, 1 \rangle, \quad \overline{b} = \langle -2, 1, -4 \rangle, \quad \overline{c} = \langle 5, 0, 3 \rangle\) vektorları berilgen; \(\overline{a}\) hám \(\overline{b}\) vektorlardan dúzilgen úshmúyeshliktiń maydanın hám usı úsh vektordan dúzilgen parallelepipedtiń kólemin tabıń.
\end{tabular}
\vspace{1cm}


\begin{tabular}{m{17cm}}
\textbf{33-variant}\\
1. Tegislikte ekinshi tàrtipli sızıqlardıń kanonikalıq teńlemeleri.\\

2. Ellips hàm onıń kanonikalıq teńlemesi, ekcentrisiteti, grafigi.\\

3. Tegislikte $A: (2, -5)$, $B: (6, 3)$, $C: (-3, 0)$ noqatları berilgen. $\overline{AB}, \overline{AC}, \overline{BC}$ vektorardı dúziń; $AB, AC, BC$ uzınlıqlardı tabıń; $\Delta ABC$ úshmúyeshliktiń maydanın esaplań. \\

4. Berilgen ekinshi tártipli iymek sızıqtıń teńlemesin ápiwayılastırıń, tipin anıqlań, orayın hám yarım kósherlerin tabıń: $x^2+16y^2+8x+160y+400=0$\\

5. \(\overline{a} = \langle 4, 0, -2 \rangle, \quad \overline{b} = \langle -1, 3, 5 \rangle, \quad \overline{c} = \langle 2, 1, -3 \rangle\) vektorları berilgen; \(\overline{a}\) hám \(\overline{b}\) vektorlardan dúzilgen úshmúyeshliktiń maydanın hám usı úsh vektordan dúzilgen parallelepipedtiń kólemin tabıń.
\end{tabular}
\vspace{1cm}


\begin{tabular}{m{17cm}}
\textbf{34-variant}\\
1. Vektorlar ùstinde sızıqlı àmeller.\\

2. Tegisliktiń tùrli teńlemeleri. Tegisliklerdiń òzara jaylasıwı.\\

3. Tegislikte $A: (7, 6)$, $B: (1, -2)$, $C: (-5, 1)$ noqatları berilgen. $\overline{AB}, \overline{AC}, \overline{BC}$ vektorardı dúziń; $AB, AC, BC$ uzınlıqlardı tabıń; $\Delta ABC$ úshmúyeshliktiń maydanın esaplań. \\

4. Berilgen ekinshi tártipli iymek sızıqtıń teńlemesin ápiwayılastırıń, tipin anıqlań, orayın hám yarım kósherlerin tabıń: $9x^2+4y^2-54x+32y+109=0$\\

5. \(\overline{a} = \langle 0, 1, 3 \rangle, \quad \overline{b} = \langle 4, -2, 0 \rangle, \quad \overline{c} = \langle -1, 2, -5 \rangle\) vektorları berilgen; \(\overline{a}\) hám \(\overline{b}\) vektorlardan dúzilgen úshmúyeshliktiń maydanın hám usı úsh vektordan dúzilgen parallelepipedtiń kólemin tabıń.
\end{tabular}
\vspace{1cm}


\begin{tabular}{m{17cm}}
\textbf{35-variant}\\
1. Ekinshi tàrtipli sızıqlardıń ulıwma teńlemeleri, orayı. Orayǵa iye hàm oraysız sızıqlar.\\

2. Vektorlardıń aralas kòbeymesi hàm qàsiyetleri. Parallelepiped kòlemi\\

3. Tegislikte $A: (5, 3)$, $B: (-1, 0)$, $C: (2, 7)$ noqatları berilgen. $\overline{AB}, \overline{AC}, \overline{BC}$ vektorardı dúziń; $AB, AC, BC$ uzınlıqlardı tabıń; $\Delta ABC$ úshmúyeshliktiń maydanın esaplań. \\

4. Berilgen ekinshi tártipli iymek sızıqtıń teńlemesin ápiwayılastırıń, tipin anıqlań, orayın hám yarım kósherlerin tabıń: $25x^2+4y^2-50x-8y-71=0$\\

5. \(\overline{a} = \langle 2, -4, 1 \rangle, \quad \overline{b} = \langle 0, 3, -2 \rangle, \quad \overline{c} = \langle -3, 1, 5 \rangle\) vektorları berilgen; \(\overline{a}\) hám \(\overline{b}\) vektorlardan dúzilgen úshmúyeshliktiń maydanın hám usı úsh vektordan dúzilgen parallelepipedtiń kólemin tabıń.
\end{tabular}
\vspace{1cm}


\begin{tabular}{m{17cm}}
\textbf{36-variant}\\
1. Eki tegisliktiń parallellik, perpendikulyarlıq shàrtleri.\\

2. Ekinshi tàrtipli sızıqlardıń invariyantları.\\

3. Tegislikte $A: (-1, 3)$, $B: (4, 6)$, $C: (-2, -5)$ noqatları berilgen. $\overline{AB}, \overline{AC}, \overline{BC}$ vektorardı dúziń; $AB, AC, BC$ uzınlıqlardı tabıń; $\Delta ABC$ úshmúyeshliktiń maydanın esaplań. \\

4. Berilgen ekinshi tártipli iymek sızıqtıń teńlemesin ápiwayılastırıń, tipin anıqlań, orayın hám yarım kósherlerin tabıń: $16x^2+y^2+128x-2y+241=0$\\

5. \(\overline{a} = \langle -1, 4, 2 \rangle, \quad \overline{b} = \langle 3, 0, -5 \rangle, \quad \overline{c} = \langle 2, -3, 1 \rangle\) vektorları berilgen; \(\overline{a}\) hám \(\overline{b}\) vektorlardan dúzilgen úshmúyeshliktiń maydanın hám usı úsh vektordan dúzilgen parallelepipedtiń kólemin tabıń.
\end{tabular}
\vspace{1cm}


\begin{tabular}{m{17cm}}
\textbf{37-variant}\\
1. Tuwrı hàm tegisliktiń òzara jaylasıwı. Ayqasıwshı tuwrılar.\\

2. Ekinshi tàrtipli sızıqlardıń ulıwma teńlemesin kanonikalıq kòriniske keltiriw usılları.\\

3. Tegislikte $A: (-5, 0)$, $B: (3, 4)$, $C: (-1, 7)$ noqatları berilgen. $\overline{AB}, \overline{AC}, \overline{BC}$ vektorardı dúziń; $AB, AC, BC$ uzınlıqlardı tabıń; $\Delta ABC$ úshmúyeshliktiń maydanın esaplań. \\

4. Berilgen ekinshi tártipli iymek sızıqtıń teńlemesin ápiwayılastırıń, tipin anıqlań, orayın hám yarım kósherlerin tabıń: $4x^2+16y^2-16x-96y+96=0$\\

5. \(\overline{a} = \langle -2, 1, 0 \rangle, \quad \overline{b} = \langle 3, -4, 2 \rangle, \quad \overline{c} = \langle 1, 0, 5 \rangle\) vektorları berilgen; \(\overline{a}\) hám \(\overline{b}\) vektorlardan dúzilgen úshmúyeshliktiń maydanın hám usı úsh vektordan dúzilgen parallelepipedtiń kólemin tabıń.
\end{tabular}
\vspace{1cm}


\begin{tabular}{m{17cm}}
\textbf{38-variant}\\
1. Giperbola hàm parabola. Kanonikalıq teńlemeleri, ekcentrisiteti, qàsiyetleri, grafikleri.\\

2. Vektorlardıń skalyar kòbeymesi hàm qàsiyetleri. Skalyar kòbeymeniń koordinatalardaǵı ańlatpası.\\

3. Tegislikte $A: (3, -4)$, $B: (-2, 7)$, $C: (5, 2)$ noqatları berilgen. $\overline{AB}, \overline{AC}, \overline{BC}$ vektorardı dúziń; $AB, AC, BC$ uzınlıqlardı tabıń; $\Delta ABC$ úshmúyeshliktiń maydanın esaplań. \\

4. Berilgen ekinshi tártipli iymek sızıqtıń teńlemesin ápiwayılastırıń, tipin anıqlań, orayın hám yarım kósherlerin tabıń: $x^2+y^2+10x+2y+25=0$\\

5. \(\overline{a} = \langle 3, 0, -2 \rangle, \quad \overline{b} = \langle -1, 2, 4 \rangle, \quad \overline{c} = \langle 2, -3, 1 \rangle\) vektorları berilgen; \(\overline{a}\) hám \(\overline{b}\) vektorlardan dúzilgen úshmúyeshliktiń maydanın hám usı úsh vektordan dúzilgen parallelepipedtiń kólemin tabıń.
\end{tabular}
\vspace{1cm}


\begin{tabular}{m{17cm}}
\textbf{39-variant}\\
1. Keńislikte tuwrınıń har tùrli teńlemeleri. \\

2. Ekinshi tàrtipli sızıq hàm tuwrınıń òzara jaylasıwı. Ekinshi tàrtipli sızıqlardıń urınbası.\\

3. Tegislikte $A: (1, 6)$, $B: (-3, -1)$, $C: (7, 4)$ noqatları berilgen. $\overline{AB}, \overline{AC}, \overline{BC}$ vektorardı dúziń; $AB, AC, BC$ uzınlıqlardı tabıń; $\Delta ABC$ úshmúyeshliktiń maydanın esaplań. \\

4. Berilgen ekinshi tártipli iymek sızıqtıń teńlemesin ápiwayılastırıń, tipin anıqlań, orayın hám yarım kósherlerin tabıń: $25x^2+y^2+150x+6y+209=0$\\

5. \(\overline{a} = \langle 1, 3, -2 \rangle, \quad \overline{b} = \langle -4, 2, 1 \rangle, \quad \overline{c} = \langle 2, -5, 0 \rangle\) vektorları berilgen; \(\overline{a}\) hám \(\overline{b}\) vektorla dúzilgen úshmúyeshliktiń maydanın hám usı úsh vektordan dúzilgen parallelepipedtiń kólemin tabıń.
\end{tabular}
\vspace{1cm}


\begin{tabular}{m{17cm}}
\textbf{40-variant}\\
1. Vektorlardıń vektorlıq kòbeymesi hàm qàsiyetleri. Vektorlıq kòbeymeniń koordinatalardaǵı ańlatpası. \\

2. Tegislikte ekinshi tàrtipli sızıqlardıń kanonikalıq teńlemeleri.\\

3. Tegislikte $A: (-4, 2)$, $B: (-3, -5)$, $C: (6, 8)$ noqatları berilgen. $\overline{AB}, \overline{AC}, \overline{BC}$ vektorardı dúziń; $AB, AC, BC$ uzınlıqlardı tabıń; $\Delta ABC$ úshmúyeshliktiń maydanın esaplań. \\

4. Berilgen ekinshi tártipli iymek sızıqtıń teńlemesin ápiwayılastırıń, tipin anıqlań, orayın hám yarım kósherlerin tabıń: $x^2+16y^2+10x-32y+25=0$\\

5. \(\overline{a} = \langle 5, -2, 1 \rangle, \quad \overline{b} = \langle 0, 3, -4 \rangle, \quad \overline{c} = \langle -3, 1, 6 \rangle\) vektorları berilgen; \(\overline{a}\) hám \(\overline{b}\) vektorlardan dúzilgen úshmúyeshliktiń maydanın hám usı úsh vektordan dúzilgen parallelepipedtiń kólemin tabıń.
\end{tabular}
\vspace{1cm}


\begin{tabular}{m{17cm}}
\textbf{41-variant}\\
1. Ellips hàm onıń kanonikalıq teńlemesi, ekcentrisiteti, grafigi.\\

2. Vektorlar ùstinde sızıqlı àmeller.\\

3. Tegislikte $A: (-2, 0)$, $B: (3, 2)$, $C: (-4, -3)$ noqatları berilgen. $\overline{AB}, \overline{AC}, \overline{BC}$ vektorardı dúziń; $AB, AC, BC$ uzınlıqlardı tabıń; $\Delta ABC$ úshmúyeshliktiń maydanın esaplań. \\

4. Berilgen ekinshi tártipli iymek sızıqtıń teńlemesin ápiwayılastırıń, tipin anıqlań, orayın hám yarım kósherlerin tabıń: $4x^2+9y^2+8x-72y+112=0$\\

5. \(\overline{a} = \langle 4, 0, -2 \rangle, \quad \overline{b} = \langle -1, 3, 5 \rangle, \quad \overline{c} = \langle 2, 1, -3 \rangle\) vektorları berilgen; \(\overline{a}\) hám \(\overline{b}\) vektorlardan dúzilgen úshmúyeshliktiń maydanın hám usı úsh vektordan dúzilgen parallelepipedtiń kólemin tabıń.
\end{tabular}
\vspace{1cm}


\begin{tabular}{m{17cm}}
\textbf{42-variant}\\
1. Tegisliktiń tùrli teńlemeleri. Tegisliklerdiń òzara jaylasıwı.\\

2. Ekinshi tàrtipli sızıqlardıń ulıwma teńlemeleri, orayı. Orayǵa iye hàm oraysız sızıqlar.\\

3. Tegislikte $A: (-3, 7)$, $B: (4, 3)$, $C: (-1, -4)$ noqatları berilgen. $\overline{AB}, \overline{AC}, \overline{BC}$ vektorardı dúziń; $AB, AC, BC$ uzınlıqlardı tabıń; $\Delta ABC$ úshmúyeshliktiń maydanın esaplań. \\

4. Berilgen ekinshi tártipli iymek sızıqtıń teńlemesin ápiwayılastırıń, tipin anıqlań, orayın hám yarım kósherlerin tabıń: $9x^2+y^2-72x+4y+139=0$\\

5. \(\overline{a} = \langle 5, -2, 1 \rangle, \quad \overline{b} = \langle 0, 3, -4 \rangle, \quad \overline{c} = \langle -3, 1, 6 \rangle\) vektorları berilgen; \(\overline{a}\) hám \(\overline{b}\) vektorlardan dúzilgen úshmúyeshliktiń maydanın hám usı úsh vektordan dúzilgen parallelepipedtiń kólemin tabıń.
\end{tabular}
\vspace{1cm}


\begin{tabular}{m{17cm}}
\textbf{43-variant}\\
1. Vektorlardıń aralas kòbeymesi hàm qàsiyetleri. Parallelepiped kòlemi\\

2. Eki tegisliktiń parallellik, perpendikulyarlıq shàrtleri.\\

3. Tegislikte $A: (2, -1)$, $B: (0, 4)$, $C: (3, 0)$ noqatları berilgen. $\overline{AB}, \overline{AC}, \overline{BC}$ vektorardı dúziń; $AB, AC, BC$ uzınlıqlardı tabıń; $\Delta ABC$ úshmúyeshliktiń maydanın esaplań. \\

4. Berilgen ekinshi tártipli iymek sızıqtıń teńlemesin ápiwayılastırıń, tipin anıqlań, orayın hám yarım kósherlerin tabıń: $16x^2+25y^2+64x-150y-111=0$\\

5. \(\overline{a} = \langle -3, 2, -4 \rangle, \quad \overline{b} = \langle 1, 5, 0 \rangle, \quad \overline{c} = \langle 2, -1, 3 \rangle\) vektorları berilgen; \(\overline{a}\) hám \(\overline{b}\) vektorlardan dúzilgen úshmúyeshliktiń maydanın hám usı úsh vektordan dúzilgen parallelepipedtiń kólemin tabıń.
\end{tabular}
\vspace{1cm}


\begin{tabular}{m{17cm}}
\textbf{44-variant}\\
1. Ekinshi tàrtipli sızıqlardıń invariyantları.\\

2. Tuwrı hàm tegisliktiń òzara jaylasıwı. Ayqasıwshı tuwrılar.\\

3. Tegislikte $A: (0, 0)$, $B: (1, 1)$, $C: (-1, 2)$ noqatları berilgen. $\overline{AB}, \overline{AC}, \overline{BC}$ vektorardı dúziń; $AB, AC, BC$ uzınlıqlardı tabıń; $\Delta ABC$ úshmúyeshliktiń maydanın esaplań. \\

4. Berilgen ekinshi tártipli iymek sızıqtıń teńlemesin ápiwayılastırıń, tipin anıqlań, orayın hám yarım kósherlerin tabıń: $25x^2+9y^2+150x-36y+36=0$\\

5. \(\overline{a} = \langle -1, 4, 2 \rangle, \quad \overline{b} = \langle 3, 0, -5 \rangle, \quad \overline{c} = \langle 2, -3, 1 \rangle\) vektorları berilgen; \(\overline{a}\) hám \(\overline{b}\) vektorlardan dúzilgen úshmúyeshliktiń maydanın hám usı úsh vektordan dúzilgen parallelepipedtiń kólemin tabıń.
\end{tabular}
\vspace{1cm}


\begin{tabular}{m{17cm}}
\textbf{45-variant}\\
1. Ekinshi tàrtipli sızıqlardıń ulıwma teńlemesin kanonikalıq kòriniske keltiriw usılları.\\

2. Giperbola hàm parabola. Kanonikalıq teńlemeleri, ekcentrisiteti, qàsiyetleri, grafikleri.\\

3. Tegislikte $A: (-1, -3)$, $B: (-4, 6)$, $C: (3, 2)$ noqatları berilgen. $\overline{AB}, \overline{AC}, \overline{BC}$ vektorardı dúziń; $AB, AC, BC$ uzınlıqlardı tabıń; $\Delta ABC$ úshmúyeshliktiń maydanın esaplań. \\

4. Berilgen ekinshi tártipli iymek sızıqtıń teńlemesin ápiwayılastırıń, tipin anıqlań, orayın hám yarım kósherlerin tabıń: $x^2+4y^2+10x+40y+121=0$\\

5. \(\overline{a} = \langle -3, 2, -4 \rangle, \quad \overline{b} = \langle 1, 5, 0 \rangle, \quad \overline{c} = \langle 2, -1, 3 \rangle\) vektorları berilgen; \(\overline{a}\) hám \(\overline{b}\) vektorlardan dúzilgen úshmúyeshliktiń maydanın hám usı úsh vektordan dúzilgen parallelepipedtiń kólemin tabıń.
\end{tabular}
\vspace{1cm}


\begin{tabular}{m{17cm}}
\textbf{46-variant}\\
1. Vektorlardıń skalyar kòbeymesi hàm qàsiyetleri. Skalyar kòbeymeniń koordinatalardaǵı ańlatpası.\\

2. Keńislikte tuwrınıń har tùrli teńlemeleri. \\

3. Tegislikte $A: (8, 1)$, $B: (2, -3)$, $C: (-6, 5)$ noqatları berilgen. $\overline{AB}, \overline{AC}, \overline{BC}$ vektorardı dúziń; $AB, AC, BC$ uzınlıqlardı tabıń; $\Delta ABC$ úshmúyeshliktiń maydanın esaplań. \\

4. Berilgen ekinshi tártipli iymek sızıqtıń teńlemesin ápiwayılastırıń, tipin anıqlań, orayın hám yarım kósherlerin tabıń: $9x^2+y^2+90x+2y+217=0$\\

5. \(\overline{a} = \langle 1, 2, -3 \rangle, \quad \overline{b} = \langle -2, 3, 0 \rangle, \quad \overline{c} = \langle 4, -1, 2 \rangle\) vektorları berilgen; \(\overline{a}\) hám \(\overline{b}\) vektorlardan dúzilgen úshmúyeshliktiń maydanın hám usı úsh vektordan dúzilgen parallelepipedtiń kólemin tabıń.
\end{tabular}
\vspace{1cm}


\begin{tabular}{m{17cm}}
\textbf{47-variant}\\
1. Ekinshi tàrtipli sızıq hàm tuwrınıń òzara jaylasıwı. Ekinshi tàrtipli sızıqlardıń urınbası.\\

2. Vektorlardıń vektorlıq kòbeymesi hàm qàsiyetleri. Vektorlıq kòbeymeniń koordinatalardaǵı ańlatpası. \\

3. Tegislikte $A: (4, 8)$, $B: (0, -6)$, $C: (-7, 3)$ noqatları berilgen. $\overline{AB}, \overline{AC}, \overline{BC}$ vektorardı dúziń; $AB, AC, BC$ uzınlıqlardı tabıń; $\Delta ABC$ úshmúyeshliktiń maydanın esaplań. \\

4. Berilgen ekinshi tártipli iymek sızıqtıń teńlemesin ápiwayılastırıń, tipin anıqlań, orayın hám yarım kósherlerin tabıń: $4x^2+4y^2+40x-8y+88=0$\\

5. \(\overline{a} = \langle 0, 1, 3 \rangle, \quad \overline{b} = \langle 4, -2, 0 \rangle, \quad \overline{c} = \langle -1, 2, -5 \rangle\) vektorları berilgen; \(\overline{a}\) hám \(\overline{b}\) vektorlardan dúzilgen úshmúyeshliktiń maydanın hám usı úsh vektordan dúzilgen parallelepipedtiń kólemin tabıń.
\end{tabular}
\vspace{1cm}


\begin{tabular}{m{17cm}}
\textbf{48-variant}\\
1. Tegislikte ekinshi tàrtipli sızıqlardıń kanonikalıq teńlemeleri.\\

2. Ellips hàm onıń kanonikalıq teńlemesi, ekcentrisiteti, grafigi.\\

3. Tegislikte $A: (0, 8)$, $B: (-2, -4)$, $C: (5, 1)$ noqatları berilgen. $\overline{AB}, \overline{AC}, \overline{BC}$ vektorardı dúziń; $AB, AC, BC$ uzınlıqlardı tabıń; $\Delta ABC$ úshmúyeshliktiń maydanın esaplań. \\

4. Berilgen ekinshi tártipli iymek sızıqtıń teńlemesin ápiwayılastırıń, tipin anıqlań, orayın hám yarım kósherlerin tabıń: $4x^2+9y^2+32x-18y+37=0$\\

5. \(\overline{a} = \langle 3, 2, 1 \rangle, \quad \overline{b} = \langle -2, 1, -4 \rangle, \quad \overline{c} = \langle 5, 0, 3 \rangle\) vektorları berilgen; \(\overline{a}\) hám \(\overline{b}\) vektorlardan dúzilgen úshmúyeshliktiń maydanın hám usı úsh vektordan dúzilgen parallelepipedtiń kólemin tabıń.
\end{tabular}
\vspace{1cm}


\begin{tabular}{m{17cm}}
\textbf{49-variant}\\
1. Vektorlar ùstinde sızıqlı àmeller.\\

2. Tegisliktiń tùrli teńlemeleri. Tegisliklerdiń òzara jaylasıwı.\\

3. Tegislikte $A: (-3, -2)$, $B: (-5, 4)$, $C: (6, -1)$ noqatları berilgen. $\overline{AB}, \overline{AC}, \overline{BC}$ vektorardı dúziń; $AB, AC, BC$ uzınlıqlardı tabıń; $\Delta ABC$ úshmúyeshliktiń maydanın esaplań. \\

4. Berilgen ekinshi tártipli iymek sızıqtıń teńlemesin ápiwayılastırıń, tipin anıqlań, orayın hám yarım kósherlerin tabıń: $16x^2+y^2+128x-4y+244=0$\\

5. \(\overline{a} = \langle 2, -1, 3 \rangle, \quad \overline{b} = \langle -4, 5, 1 \rangle, \quad \overline{c} = \langle 0, 2, -3 \rangle\) vektorları berilgen; \(\overline{a}\) hám \(\overline{b}\) vektorlardan dúzilgen úshmúyeshliktiń maydanın hám usı úsh vektordan dúzilgen parallelepipedtiń kólemin tabıń.
\end{tabular}
\vspace{1cm}


\begin{tabular}{m{17cm}}
\textbf{50-variant}\\
1. Ekinshi tàrtipli sızıqlardıń ulıwma teńlemeleri, orayı. Orayǵa iye hàm oraysız sızıqlar.\\

2. Vektorlardıń aralas kòbeymesi hàm qàsiyetleri. Parallelepiped kòlemi\\

3. Tegislikte $A: (1, 2)$, $B: (-2, 6)$, $C: (4, -3)$ noqatları berilgen. $\overline{AB}, \overline{AC}, \overline{BC}$ vektorardı dúziń; $AB, AC, BC$ uzınlıqlardı tabıń; $\Delta ABC$ úshmúyeshliktiń maydanın esaplań. \\

4. Berilgen ekinshi tártipli iymek sızıqtıń teńlemesin ápiwayılastırıń, tipin anıqlań, orayın hám yarım kósherlerin tabıń: $4x^2+y^2+16x-2y+13=0$\\

5. \(\overline{a} = \langle 1, 3, -2 \rangle, \quad \overline{b} = \langle -4, 2, 1 \rangle, \quad \overline{c} = \langle 2, -5, 0 \rangle\) vektorları berilgen; \(\overline{a}\) hám \(\overline{b}\) vektorlardan dúzilgen úshmúyeshliktiń maydanın hám usı úsh vektordan dúzilgen parallelepipedtiń kólemin tabıń.
\end{tabular}
\vspace{1cm}


\begin{tabular}{m{17cm}}
\textbf{51-variant}\\
1. Eki tegisliktiń parallellik, perpendikulyarlıq shàrtleri.\\

2. Ekinshi tàrtipli sızıqlardıń invariyantları.\\

3. Tegislikte $A: (6, -2)$, $B: (2, 1)$, $C: (-2, 5)$ noqatları berilgen. $\overline{AB}, \overline{AC}, \overline{BC}$ vektorardı dúziń; $AB, AC, BC$ uzınlıqlardı tabıń; $\Delta ABC$ úshmúyeshliktiń maydanın esaplań. \\

4. Berilgen ekinshi tártipli iymek sızıqtıń teńlemesin ápiwayılastırıń, tipin anıqlań, orayın hám yarım kósherlerin tabıń: $4x^2+4y^2+16x+16y+16=0$\\

5. \(\overline{a} = \langle 2, -1, 3 \rangle, \quad \overline{b} = \langle -4, 5, 1 \rangle, \quad \overline{c} = \langle 0, 2, -3 \rangle\) vektorları berilgen; \(\overline{a}\) hám \(\overline{b}\) vektorlardan dúzilgen úshmúyeshliktiń maydanın hám usı úsh vektordan dúzilgen parallelepipedtiń kólemin tabıń.
\end{tabular}
\vspace{1cm}


\begin{tabular}{m{17cm}}
\textbf{52-variant}\\
1. Tuwrı hàm tegisliktiń òzara jaylasıwı. Ayqasıwshı tuwrılar.\\

2. Ekinshi tàrtipli sızıqlardıń ulıwma teńlemesin kanonikalıq kòriniske keltiriw usılları.\\

3. Tegislikte $A: (2, 5)$, $B: (7, -3)$, $C: (-4, 9)$ noqatları berilgen. $\overline{AB}, \overline{AC}, \overline{BC}$ vektorardı dúziń; $AB, AC, BC$ uzınlıqlardı tabıń; $\Delta ABC$ úshmúyeshliktiń maydanın esaplań. \\

4. Berilgen ekinshi tártipli iymek sızıqtıń teńlemesin ápiwayılastırıń, tipin anıqlań, orayın hám yarım kósherlerin tabıń: $9x^2+4y^2-36x-16y+16=0$\\

5. \(\overline{a} = \langle 3, 2, 1 \rangle, \quad \overline{b} = \langle -2, 1, -4 \rangle, \quad \overline{c} = \langle 5, 0, 3 \rangle\) vektorları berilgen; \(\overline{a}\) hám \(\overline{b}\) vektorlardan dúzilgen úshmúyeshliktiń maydanın hám usı úsh vektordan dúzilgen parallelepipedtiń kólemin tabıń.
\end{tabular}
\vspace{1cm}


\begin{tabular}{m{17cm}}
\textbf{53-variant}\\
1. Giperbola hàm parabola. Kanonikalıq teńlemeleri, ekcentrisiteti, qàsiyetleri, grafikleri.\\

2. Vektorlardıń skalyar kòbeymesi hàm qàsiyetleri. Skalyar kòbeymeniń koordinatalardaǵı ańlatpası.\\

3. Tegislikte $A: (2, -5)$, $B: (6, 3)$, $C: (-3, 0)$ noqatları berilgen. $\overline{AB}, \overline{AC}, \overline{BC}$ vektorardı dúziń; $AB, AC, BC$ uzınlıqlardı tabıń; $\Delta ABC$ úshmúyeshliktiń maydanın esaplań. \\

4. Berilgen ekinshi tártipli iymek sızıqtıń teńlemesin ápiwayılastırıń, tipin anıqlań, orayın hám yarım kósherlerin tabıń: $x^2+16y^2+8x+160y+400=0$\\

5. \(\overline{a} = \langle 4, 0, -2 \rangle, \quad \overline{b} = \langle -1, 3, 5 \rangle, \quad \overline{c} = \langle 2, 1, -3 \rangle\) vektorları berilgen; \(\overline{a}\) hám \(\overline{b}\) vektorlardan dúzilgen úshmúyeshliktiń maydanın hám usı úsh vektordan dúzilgen parallelepipedtiń kólemin tabıń.
\end{tabular}
\vspace{1cm}


\begin{tabular}{m{17cm}}
\textbf{54-variant}\\
1. Keńislikte tuwrınıń har tùrli teńlemeleri. \\

2. Ekinshi tàrtipli sızıq hàm tuwrınıń òzara jaylasıwı. Ekinshi tàrtipli sızıqlardıń urınbası.\\

3. Tegislikte $A: (7, 6)$, $B: (1, -2)$, $C: (-5, 1)$ noqatları berilgen. $\overline{AB}, \overline{AC}, \overline{BC}$ vektorardı dúziń; $AB, AC, BC$ uzınlıqlardı tabıń; $\Delta ABC$ úshmúyeshliktiń maydanın esaplań. \\

4. Berilgen ekinshi tártipli iymek sızıqtıń teńlemesin ápiwayılastırıń, tipin anıqlań, orayın hám yarım kósherlerin tabıń: $9x^2+4y^2-54x+32y+109=0$\\

5. \(\overline{a} = \langle 0, 1, 3 \rangle, \quad \overline{b} = \langle 4, -2, 0 \rangle, \quad \overline{c} = \langle -1, 2, -5 \rangle\) vektorları berilgen; \(\overline{a}\) hám \(\overline{b}\) vektorlardan dúzilgen úshmúyeshliktiń maydanın hám usı úsh vektordan dúzilgen parallelepipedtiń kólemin tabıń.
\end{tabular}
\vspace{1cm}


\begin{tabular}{m{17cm}}
\textbf{55-variant}\\
1. Vektorlardıń vektorlıq kòbeymesi hàm qàsiyetleri. Vektorlıq kòbeymeniń koordinatalardaǵı ańlatpası. \\

2. Tegislikte ekinshi tàrtipli sızıqlardıń kanonikalıq teńlemeleri.\\

3. Tegislikte $A: (5, 3)$, $B: (-1, 0)$, $C: (2, 7)$ noqatları berilgen. $\overline{AB}, \overline{AC}, \overline{BC}$ vektorardı dúziń; $AB, AC, BC$ uzınlıqlardı tabıń; $\Delta ABC$ úshmúyeshliktiń maydanın esaplań. \\

4. Berilgen ekinshi tártipli iymek sızıqtıń teńlemesin ápiwayılastırıń, tipin anıqlań, orayın hám yarım kósherlerin tabıń: $25x^2+4y^2-50x-8y-71=0$\\

5. \(\overline{a} = \langle 2, -4, 1 \rangle, \quad \overline{b} = \langle 0, 3, -2 \rangle, \quad \overline{c} = \langle -3, 1, 5 \rangle\) vektorları berilgen; \(\overline{a}\) hám \(\overline{b}\) vektorlardan dúzilgen úshmúyeshliktiń maydanın hám usı úsh vektordan dúzilgen parallelepipedtiń kólemin tabıń.
\end{tabular}
\vspace{1cm}


\begin{tabular}{m{17cm}}
\textbf{56-variant}\\
1. Ellips hàm onıń kanonikalıq teńlemesi, ekcentrisiteti, grafigi.\\

2. Vektorlar ùstinde sızıqlı àmeller.\\

3. Tegislikte $A: (-1, 3)$, $B: (4, 6)$, $C: (-2, -5)$ noqatları berilgen. $\overline{AB}, \overline{AC}, \overline{BC}$ vektorardı dúziń; $AB, AC, BC$ uzınlıqlardı tabıń; $\Delta ABC$ úshmúyeshliktiń maydanın esaplań. \\

4. Berilgen ekinshi tártipli iymek sızıqtıń teńlemesin ápiwayılastırıń, tipin anıqlań, orayın hám yarım kósherlerin tabıń: $16x^2+y^2+128x-2y+241=0$\\

5. \(\overline{a} = \langle -1, 4, 2 \rangle, \quad \overline{b} = \langle 3, 0, -5 \rangle, \quad \overline{c} = \langle 2, -3, 1 \rangle\) vektorları berilgen; \(\overline{a}\) hám \(\overline{b}\) vektorlardan dúzilgen úshmúyeshliktiń maydanın hám usı úsh vektordan dúzilgen parallelepipedtiń kólemin tabıń.
\end{tabular}
\vspace{1cm}


\begin{tabular}{m{17cm}}
\textbf{57-variant}\\
1. Tegisliktiń tùrli teńlemeleri. Tegisliklerdiń òzara jaylasıwı.\\

2. Ekinshi tàrtipli sızıqlardıń ulıwma teńlemeleri, orayı. Orayǵa iye hàm oraysız sızıqlar.\\

3. Tegislikte $A: (-5, 0)$, $B: (3, 4)$, $C: (-1, 7)$ noqatları berilgen. $\overline{AB}, \overline{AC}, \overline{BC}$ vektorardı dúziń; $AB, AC, BC$ uzınlıqlardı tabıń; $\Delta ABC$ úshmúyeshliktiń maydanın esaplań. \\

4. Berilgen ekinshi tártipli iymek sızıqtıń teńlemesin ápiwayılastırıń, tipin anıqlań, orayın hám yarım kósherlerin tabıń: $4x^2+16y^2-16x-96y+96=0$\\

5. \(\overline{a} = \langle -2, 1, 0 \rangle, \quad \overline{b} = \langle 3, -4, 2 \rangle, \quad \overline{c} = \langle 1, 0, 5 \rangle\) vektorları berilgen; \(\overline{a}\) hám \(\overline{b}\) vektorlardan dúzilgen úshmúyeshliktiń maydanın hám usı úsh vektordan dúzilgen parallelepipedtiń kólemin tabıń.
\end{tabular}
\vspace{1cm}


\begin{tabular}{m{17cm}}
\textbf{58-variant}\\
1. Vektorlardıń aralas kòbeymesi hàm qàsiyetleri. Parallelepiped kòlemi\\

2. Eki tegisliktiń parallellik, perpendikulyarlıq shàrtleri.\\

3. Tegislikte $A: (3, -4)$, $B: (-2, 7)$, $C: (5, 2)$ noqatları berilgen. $\overline{AB}, \overline{AC}, \overline{BC}$ vektorardı dúziń; $AB, AC, BC$ uzınlıqlardı tabıń; $\Delta ABC$ úshmúyeshliktiń maydanın esaplań. \\

4. Berilgen ekinshi tártipli iymek sızıqtıń teńlemesin ápiwayılastırıń, tipin anıqlań, orayın hám yarım kósherlerin tabıń: $x^2+y^2+10x+2y+25=0$\\

5. \(\overline{a} = \langle 3, 0, -2 \rangle, \quad \overline{b} = \langle -1, 2, 4 \rangle, \quad \overline{c} = \langle 2, -3, 1 \rangle\) vektorları berilgen; \(\overline{a}\) hám \(\overline{b}\) vektorlardan dúzilgen úshmúyeshliktiń maydanın hám usı úsh vektordan dúzilgen parallelepipedtiń kólemin tabıń.
\end{tabular}
\vspace{1cm}


\begin{tabular}{m{17cm}}
\textbf{59-variant}\\
1. Ekinshi tàrtipli sızıqlardıń invariyantları.\\

2. Tuwrı hàm tegisliktiń òzara jaylasıwı. Ayqasıwshı tuwrılar.\\

3. Tegislikte $A: (1, 6)$, $B: (-3, -1)$, $C: (7, 4)$ noqatları berilgen. $\overline{AB}, \overline{AC}, \overline{BC}$ vektorardı dúziń; $AB, AC, BC$ uzınlıqlardı tabıń; $\Delta ABC$ úshmúyeshliktiń maydanın esaplań. \\

4. Berilgen ekinshi tártipli iymek sızıqtıń teńlemesin ápiwayılastırıń, tipin anıqlań, orayın hám yarım kósherlerin tabıń: $25x^2+y^2+150x+6y+209=0$\\

5. \(\overline{a} = \langle 1, 3, -2 \rangle, \quad \overline{b} = \langle -4, 2, 1 \rangle, \quad \overline{c} = \langle 2, -5, 0 \rangle\) vektorları berilgen; \(\overline{a}\) hám \(\overline{b}\) vektorla dúzilgen úshmúyeshliktiń maydanın hám usı úsh vektordan dúzilgen parallelepipedtiń kólemin tabıń.
\end{tabular}
\vspace{1cm}


\begin{tabular}{m{17cm}}
\textbf{60-variant}\\
1. Ekinshi tàrtipli sızıqlardıń ulıwma teńlemesin kanonikalıq kòriniske keltiriw usılları.\\

2. Giperbola hàm parabola. Kanonikalıq teńlemeleri, ekcentrisiteti, qàsiyetleri, grafikleri.\\

3. Tegislikte $A: (-4, 2)$, $B: (-3, -5)$, $C: (6, 8)$ noqatları berilgen. $\overline{AB}, \overline{AC}, \overline{BC}$ vektorardı dúziń; $AB, AC, BC$ uzınlıqlardı tabıń; $\Delta ABC$ úshmúyeshliktiń maydanın esaplań. \\

4. Berilgen ekinshi tártipli iymek sızıqtıń teńlemesin ápiwayılastırıń, tipin anıqlań, orayın hám yarım kósherlerin tabıń: $x^2+16y^2+10x-32y+25=0$\\

5. \(\overline{a} = \langle 5, -2, 1 \rangle, \quad \overline{b} = \langle 0, 3, -4 \rangle, \quad \overline{c} = \langle -3, 1, 6 \rangle\) vektorları berilgen; \(\overline{a}\) hám \(\overline{b}\) vektorlardan dúzilgen úshmúyeshliktiń maydanın hám usı úsh vektordan dúzilgen parallelepipedtiń kólemin tabıń.
\end{tabular}
\vspace{1cm}


\begin{tabular}{m{17cm}}
\textbf{61-variant}\\
1. Vektorlardıń skalyar kòbeymesi hàm qàsiyetleri. Skalyar kòbeymeniń koordinatalardaǵı ańlatpası.\\

2. Keńislikte tuwrınıń har tùrli teńlemeleri. \\

3. Tegislikte $A: (-2, 0)$, $B: (3, 2)$, $C: (-4, -3)$ noqatları berilgen. $\overline{AB}, \overline{AC}, \overline{BC}$ vektorardı dúziń; $AB, AC, BC$ uzınlıqlardı tabıń; $\Delta ABC$ úshmúyeshliktiń maydanın esaplań. \\

4. Berilgen ekinshi tártipli iymek sızıqtıń teńlemesin ápiwayılastırıń, tipin anıqlań, orayın hám yarım kósherlerin tabıń: $4x^2+9y^2+8x-72y+112=0$\\

5. \(\overline{a} = \langle 4, 0, -2 \rangle, \quad \overline{b} = \langle -1, 3, 5 \rangle, \quad \overline{c} = \langle 2, 1, -3 \rangle\) vektorları berilgen; \(\overline{a}\) hám \(\overline{b}\) vektorlardan dúzilgen úshmúyeshliktiń maydanın hám usı úsh vektordan dúzilgen parallelepipedtiń kólemin tabıń.
\end{tabular}
\vspace{1cm}


\begin{tabular}{m{17cm}}
\textbf{62-variant}\\
1. Ekinshi tàrtipli sızıq hàm tuwrınıń òzara jaylasıwı. Ekinshi tàrtipli sızıqlardıń urınbası.\\

2. Vektorlardıń vektorlıq kòbeymesi hàm qàsiyetleri. Vektorlıq kòbeymeniń koordinatalardaǵı ańlatpası. \\

3. Tegislikte $A: (-3, 7)$, $B: (4, 3)$, $C: (-1, -4)$ noqatları berilgen. $\overline{AB}, \overline{AC}, \overline{BC}$ vektorardı dúziń; $AB, AC, BC$ uzınlıqlardı tabıń; $\Delta ABC$ úshmúyeshliktiń maydanın esaplań. \\

4. Berilgen ekinshi tártipli iymek sızıqtıń teńlemesin ápiwayılastırıń, tipin anıqlań, orayın hám yarım kósherlerin tabıń: $9x^2+y^2-72x+4y+139=0$\\

5. \(\overline{a} = \langle 5, -2, 1 \rangle, \quad \overline{b} = \langle 0, 3, -4 \rangle, \quad \overline{c} = \langle -3, 1, 6 \rangle\) vektorları berilgen; \(\overline{a}\) hám \(\overline{b}\) vektorlardan dúzilgen úshmúyeshliktiń maydanın hám usı úsh vektordan dúzilgen parallelepipedtiń kólemin tabıń.
\end{tabular}
\vspace{1cm}


\begin{tabular}{m{17cm}}
\textbf{63-variant}\\
1. Tegislikte ekinshi tàrtipli sızıqlardıń kanonikalıq teńlemeleri.\\

2. Ellips hàm onıń kanonikalıq teńlemesi, ekcentrisiteti, grafigi.\\

3. Tegislikte $A: (2, -1)$, $B: (0, 4)$, $C: (3, 0)$ noqatları berilgen. $\overline{AB}, \overline{AC}, \overline{BC}$ vektorardı dúziń; $AB, AC, BC$ uzınlıqlardı tabıń; $\Delta ABC$ úshmúyeshliktiń maydanın esaplań. \\

4. Berilgen ekinshi tártipli iymek sızıqtıń teńlemesin ápiwayılastırıń, tipin anıqlań, orayın hám yarım kósherlerin tabıń: $16x^2+25y^2+64x-150y-111=0$\\

5. \(\overline{a} = \langle -3, 2, -4 \rangle, \quad \overline{b} = \langle 1, 5, 0 \rangle, \quad \overline{c} = \langle 2, -1, 3 \rangle\) vektorları berilgen; \(\overline{a}\) hám \(\overline{b}\) vektorlardan dúzilgen úshmúyeshliktiń maydanın hám usı úsh vektordan dúzilgen parallelepipedtiń kólemin tabıń.
\end{tabular}
\vspace{1cm}


\begin{tabular}{m{17cm}}
\textbf{64-variant}\\
1. Vektorlar ùstinde sızıqlı àmeller.\\

2. Tegisliktiń tùrli teńlemeleri. Tegisliklerdiń òzara jaylasıwı.\\

3. Tegislikte $A: (0, 0)$, $B: (1, 1)$, $C: (-1, 2)$ noqatları berilgen. $\overline{AB}, \overline{AC}, \overline{BC}$ vektorardı dúziń; $AB, AC, BC$ uzınlıqlardı tabıń; $\Delta ABC$ úshmúyeshliktiń maydanın esaplań. \\

4. Berilgen ekinshi tártipli iymek sızıqtıń teńlemesin ápiwayılastırıń, tipin anıqlań, orayın hám yarım kósherlerin tabıń: $25x^2+9y^2+150x-36y+36=0$\\

5. \(\overline{a} = \langle -1, 4, 2 \rangle, \quad \overline{b} = \langle 3, 0, -5 \rangle, \quad \overline{c} = \langle 2, -3, 1 \rangle\) vektorları berilgen; \(\overline{a}\) hám \(\overline{b}\) vektorlardan dúzilgen úshmúyeshliktiń maydanın hám usı úsh vektordan dúzilgen parallelepipedtiń kólemin tabıń.
\end{tabular}
\vspace{1cm}


\begin{tabular}{m{17cm}}
\textbf{65-variant}\\
1. Ekinshi tàrtipli sızıqlardıń ulıwma teńlemeleri, orayı. Orayǵa iye hàm oraysız sızıqlar.\\

2. Vektorlardıń aralas kòbeymesi hàm qàsiyetleri. Parallelepiped kòlemi\\

3. Tegislikte $A: (-1, -3)$, $B: (-4, 6)$, $C: (3, 2)$ noqatları berilgen. $\overline{AB}, \overline{AC}, \overline{BC}$ vektorardı dúziń; $AB, AC, BC$ uzınlıqlardı tabıń; $\Delta ABC$ úshmúyeshliktiń maydanın esaplań. \\

4. Berilgen ekinshi tártipli iymek sızıqtıń teńlemesin ápiwayılastırıń, tipin anıqlań, orayın hám yarım kósherlerin tabıń: $x^2+4y^2+10x+40y+121=0$\\

5. \(\overline{a} = \langle -3, 2, -4 \rangle, \quad \overline{b} = \langle 1, 5, 0 \rangle, \quad \overline{c} = \langle 2, -1, 3 \rangle\) vektorları berilgen; \(\overline{a}\) hám \(\overline{b}\) vektorlardan dúzilgen úshmúyeshliktiń maydanın hám usı úsh vektordan dúzilgen parallelepipedtiń kólemin tabıń.
\end{tabular}
\vspace{1cm}


\begin{tabular}{m{17cm}}
\textbf{66-variant}\\
1. Eki tegisliktiń parallellik, perpendikulyarlıq shàrtleri.\\

2. Ekinshi tàrtipli sızıqlardıń invariyantları.\\

3. Tegislikte $A: (8, 1)$, $B: (2, -3)$, $C: (-6, 5)$ noqatları berilgen. $\overline{AB}, \overline{AC}, \overline{BC}$ vektorardı dúziń; $AB, AC, BC$ uzınlıqlardı tabıń; $\Delta ABC$ úshmúyeshliktiń maydanın esaplań. \\

4. Berilgen ekinshi tártipli iymek sızıqtıń teńlemesin ápiwayılastırıń, tipin anıqlań, orayın hám yarım kósherlerin tabıń: $9x^2+y^2+90x+2y+217=0$\\

5. \(\overline{a} = \langle 1, 2, -3 \rangle, \quad \overline{b} = \langle -2, 3, 0 \rangle, \quad \overline{c} = \langle 4, -1, 2 \rangle\) vektorları berilgen; \(\overline{a}\) hám \(\overline{b}\) vektorlardan dúzilgen úshmúyeshliktiń maydanın hám usı úsh vektordan dúzilgen parallelepipedtiń kólemin tabıń.
\end{tabular}
\vspace{1cm}


\begin{tabular}{m{17cm}}
\textbf{67-variant}\\
1. Tuwrı hàm tegisliktiń òzara jaylasıwı. Ayqasıwshı tuwrılar.\\

2. Ekinshi tàrtipli sızıqlardıń ulıwma teńlemesin kanonikalıq kòriniske keltiriw usılları.\\

3. Tegislikte $A: (4, 8)$, $B: (0, -6)$, $C: (-7, 3)$ noqatları berilgen. $\overline{AB}, \overline{AC}, \overline{BC}$ vektorardı dúziń; $AB, AC, BC$ uzınlıqlardı tabıń; $\Delta ABC$ úshmúyeshliktiń maydanın esaplań. \\

4. Berilgen ekinshi tártipli iymek sızıqtıń teńlemesin ápiwayılastırıń, tipin anıqlań, orayın hám yarım kósherlerin tabıń: $4x^2+4y^2+40x-8y+88=0$\\

5. \(\overline{a} = \langle 0, 1, 3 \rangle, \quad \overline{b} = \langle 4, -2, 0 \rangle, \quad \overline{c} = \langle -1, 2, -5 \rangle\) vektorları berilgen; \(\overline{a}\) hám \(\overline{b}\) vektorlardan dúzilgen úshmúyeshliktiń maydanın hám usı úsh vektordan dúzilgen parallelepipedtiń kólemin tabıń.
\end{tabular}
\vspace{1cm}


\begin{tabular}{m{17cm}}
\textbf{68-variant}\\
1. Giperbola hàm parabola. Kanonikalıq teńlemeleri, ekcentrisiteti, qàsiyetleri, grafikleri.\\

2. Vektorlardıń skalyar kòbeymesi hàm qàsiyetleri. Skalyar kòbeymeniń koordinatalardaǵı ańlatpası.\\

3. Tegislikte $A: (0, 8)$, $B: (-2, -4)$, $C: (5, 1)$ noqatları berilgen. $\overline{AB}, \overline{AC}, \overline{BC}$ vektorardı dúziń; $AB, AC, BC$ uzınlıqlardı tabıń; $\Delta ABC$ úshmúyeshliktiń maydanın esaplań. \\

4. Berilgen ekinshi tártipli iymek sızıqtıń teńlemesin ápiwayılastırıń, tipin anıqlań, orayın hám yarım kósherlerin tabıń: $4x^2+9y^2+32x-18y+37=0$\\

5. \(\overline{a} = \langle 3, 2, 1 \rangle, \quad \overline{b} = \langle -2, 1, -4 \rangle, \quad \overline{c} = \langle 5, 0, 3 \rangle\) vektorları berilgen; \(\overline{a}\) hám \(\overline{b}\) vektorlardan dúzilgen úshmúyeshliktiń maydanın hám usı úsh vektordan dúzilgen parallelepipedtiń kólemin tabıń.
\end{tabular}
\vspace{1cm}


\begin{tabular}{m{17cm}}
\textbf{69-variant}\\
1. Keńislikte tuwrınıń har tùrli teńlemeleri. \\

2. Ekinshi tàrtipli sızıq hàm tuwrınıń òzara jaylasıwı. Ekinshi tàrtipli sızıqlardıń urınbası.\\

3. Tegislikte $A: (-3, -2)$, $B: (-5, 4)$, $C: (6, -1)$ noqatları berilgen. $\overline{AB}, \overline{AC}, \overline{BC}$ vektorardı dúziń; $AB, AC, BC$ uzınlıqlardı tabıń; $\Delta ABC$ úshmúyeshliktiń maydanın esaplań. \\

4. Berilgen ekinshi tártipli iymek sızıqtıń teńlemesin ápiwayılastırıń, tipin anıqlań, orayın hám yarım kósherlerin tabıń: $16x^2+y^2+128x-4y+244=0$\\

5. \(\overline{a} = \langle 2, -1, 3 \rangle, \quad \overline{b} = \langle -4, 5, 1 \rangle, \quad \overline{c} = \langle 0, 2, -3 \rangle\) vektorları berilgen; \(\overline{a}\) hám \(\overline{b}\) vektorlardan dúzilgen úshmúyeshliktiń maydanın hám usı úsh vektordan dúzilgen parallelepipedtiń kólemin tabıń.
\end{tabular}
\vspace{1cm}


\begin{tabular}{m{17cm}}
\textbf{70-variant}\\
1. Vektorlardıń vektorlıq kòbeymesi hàm qàsiyetleri. Vektorlıq kòbeymeniń koordinatalardaǵı ańlatpası. \\

2. Tegislikte ekinshi tàrtipli sızıqlardıń kanonikalıq teńlemeleri.\\

3. Tegislikte $A: (1, 2)$, $B: (-2, 6)$, $C: (4, -3)$ noqatları berilgen. $\overline{AB}, \overline{AC}, \overline{BC}$ vektorardı dúziń; $AB, AC, BC$ uzınlıqlardı tabıń; $\Delta ABC$ úshmúyeshliktiń maydanın esaplań. \\

4. Berilgen ekinshi tártipli iymek sızıqtıń teńlemesin ápiwayılastırıń, tipin anıqlań, orayın hám yarım kósherlerin tabıń: $4x^2+y^2+16x-2y+13=0$\\

5. \(\overline{a} = \langle 1, 3, -2 \rangle, \quad \overline{b} = \langle -4, 2, 1 \rangle, \quad \overline{c} = \langle 2, -5, 0 \rangle\) vektorları berilgen; \(\overline{a}\) hám \(\overline{b}\) vektorlardan dúzilgen úshmúyeshliktiń maydanın hám usı úsh vektordan dúzilgen parallelepipedtiń kólemin tabıń.
\end{tabular}
\vspace{1cm}


\begin{tabular}{m{17cm}}
\textbf{71-variant}\\
1. Ellips hàm onıń kanonikalıq teńlemesi, ekcentrisiteti, grafigi.\\

2. Vektorlar ùstinde sızıqlı àmeller.\\

3. Tegislikte $A: (6, -2)$, $B: (2, 1)$, $C: (-2, 5)$ noqatları berilgen. $\overline{AB}, \overline{AC}, \overline{BC}$ vektorardı dúziń; $AB, AC, BC$ uzınlıqlardı tabıń; $\Delta ABC$ úshmúyeshliktiń maydanın esaplań. \\

4. Berilgen ekinshi tártipli iymek sızıqtıń teńlemesin ápiwayılastırıń, tipin anıqlań, orayın hám yarım kósherlerin tabıń: $4x^2+4y^2+16x+16y+16=0$\\

5. \(\overline{a} = \langle 2, -1, 3 \rangle, \quad \overline{b} = \langle -4, 5, 1 \rangle, \quad \overline{c} = \langle 0, 2, -3 \rangle\) vektorları berilgen; \(\overline{a}\) hám \(\overline{b}\) vektorlardan dúzilgen úshmúyeshliktiń maydanın hám usı úsh vektordan dúzilgen parallelepipedtiń kólemin tabıń.
\end{tabular}
\vspace{1cm}


\begin{tabular}{m{17cm}}
\textbf{72-variant}\\
1. Tegisliktiń tùrli teńlemeleri. Tegisliklerdiń òzara jaylasıwı.\\

2. Ekinshi tàrtipli sızıqlardıń ulıwma teńlemeleri, orayı. Orayǵa iye hàm oraysız sızıqlar.\\

3. Tegislikte $A: (2, 5)$, $B: (7, -3)$, $C: (-4, 9)$ noqatları berilgen. $\overline{AB}, \overline{AC}, \overline{BC}$ vektorardı dúziń; $AB, AC, BC$ uzınlıqlardı tabıń; $\Delta ABC$ úshmúyeshliktiń maydanın esaplań. \\

4. Berilgen ekinshi tártipli iymek sızıqtıń teńlemesin ápiwayılastırıń, tipin anıqlań, orayın hám yarım kósherlerin tabıń: $9x^2+4y^2-36x-16y+16=0$\\

5. \(\overline{a} = \langle 3, 2, 1 \rangle, \quad \overline{b} = \langle -2, 1, -4 \rangle, \quad \overline{c} = \langle 5, 0, 3 \rangle\) vektorları berilgen; \(\overline{a}\) hám \(\overline{b}\) vektorlardan dúzilgen úshmúyeshliktiń maydanın hám usı úsh vektordan dúzilgen parallelepipedtiń kólemin tabıń.
\end{tabular}
\vspace{1cm}


\begin{tabular}{m{17cm}}
\textbf{73-variant}\\
1. Vektorlardıń aralas kòbeymesi hàm qàsiyetleri. Parallelepiped kòlemi\\

2. Eki tegisliktiń parallellik, perpendikulyarlıq shàrtleri.\\

3. Tegislikte $A: (2, -5)$, $B: (6, 3)$, $C: (-3, 0)$ noqatları berilgen. $\overline{AB}, \overline{AC}, \overline{BC}$ vektorardı dúziń; $AB, AC, BC$ uzınlıqlardı tabıń; $\Delta ABC$ úshmúyeshliktiń maydanın esaplań. \\

4. Berilgen ekinshi tártipli iymek sızıqtıń teńlemesin ápiwayılastırıń, tipin anıqlań, orayın hám yarım kósherlerin tabıń: $x^2+16y^2+8x+160y+400=0$\\

5. \(\overline{a} = \langle 4, 0, -2 \rangle, \quad \overline{b} = \langle -1, 3, 5 \rangle, \quad \overline{c} = \langle 2, 1, -3 \rangle\) vektorları berilgen; \(\overline{a}\) hám \(\overline{b}\) vektorlardan dúzilgen úshmúyeshliktiń maydanın hám usı úsh vektordan dúzilgen parallelepipedtiń kólemin tabıń.
\end{tabular}
\vspace{1cm}


\begin{tabular}{m{17cm}}
\textbf{74-variant}\\
1. Ekinshi tàrtipli sızıqlardıń invariyantları.\\

2. Tuwrı hàm tegisliktiń òzara jaylasıwı. Ayqasıwshı tuwrılar.\\

3. Tegislikte $A: (7, 6)$, $B: (1, -2)$, $C: (-5, 1)$ noqatları berilgen. $\overline{AB}, \overline{AC}, \overline{BC}$ vektorardı dúziń; $AB, AC, BC$ uzınlıqlardı tabıń; $\Delta ABC$ úshmúyeshliktiń maydanın esaplań. \\

4. Berilgen ekinshi tártipli iymek sızıqtıń teńlemesin ápiwayılastırıń, tipin anıqlań, orayın hám yarım kósherlerin tabıń: $9x^2+4y^2-54x+32y+109=0$\\

5. \(\overline{a} = \langle 0, 1, 3 \rangle, \quad \overline{b} = \langle 4, -2, 0 \rangle, \quad \overline{c} = \langle -1, 2, -5 \rangle\) vektorları berilgen; \(\overline{a}\) hám \(\overline{b}\) vektorlardan dúzilgen úshmúyeshliktiń maydanın hám usı úsh vektordan dúzilgen parallelepipedtiń kólemin tabıń.
\end{tabular}
\vspace{1cm}


\begin{tabular}{m{17cm}}
\textbf{75-variant}\\
1. Ekinshi tàrtipli sızıqlardıń ulıwma teńlemesin kanonikalıq kòriniske keltiriw usılları.\\

2. Giperbola hàm parabola. Kanonikalıq teńlemeleri, ekcentrisiteti, qàsiyetleri, grafikleri.\\

3. Tegislikte $A: (5, 3)$, $B: (-1, 0)$, $C: (2, 7)$ noqatları berilgen. $\overline{AB}, \overline{AC}, \overline{BC}$ vektorardı dúziń; $AB, AC, BC$ uzınlıqlardı tabıń; $\Delta ABC$ úshmúyeshliktiń maydanın esaplań. \\

4. Berilgen ekinshi tártipli iymek sızıqtıń teńlemesin ápiwayılastırıń, tipin anıqlań, orayın hám yarım kósherlerin tabıń: $25x^2+4y^2-50x-8y-71=0$\\

5. \(\overline{a} = \langle 2, -4, 1 \rangle, \quad \overline{b} = \langle 0, 3, -2 \rangle, \quad \overline{c} = \langle -3, 1, 5 \rangle\) vektorları berilgen; \(\overline{a}\) hám \(\overline{b}\) vektorlardan dúzilgen úshmúyeshliktiń maydanın hám usı úsh vektordan dúzilgen parallelepipedtiń kólemin tabıń.
\end{tabular}
\vspace{1cm}


\begin{tabular}{m{17cm}}
\textbf{76-variant}\\
1. Vektorlardıń skalyar kòbeymesi hàm qàsiyetleri. Skalyar kòbeymeniń koordinatalardaǵı ańlatpası.\\

2. Keńislikte tuwrınıń har tùrli teńlemeleri. \\

3. Tegislikte $A: (-1, 3)$, $B: (4, 6)$, $C: (-2, -5)$ noqatları berilgen. $\overline{AB}, \overline{AC}, \overline{BC}$ vektorardı dúziń; $AB, AC, BC$ uzınlıqlardı tabıń; $\Delta ABC$ úshmúyeshliktiń maydanın esaplań. \\

4. Berilgen ekinshi tártipli iymek sızıqtıń teńlemesin ápiwayılastırıń, tipin anıqlań, orayın hám yarım kósherlerin tabıń: $16x^2+y^2+128x-2y+241=0$\\

5. \(\overline{a} = \langle -1, 4, 2 \rangle, \quad \overline{b} = \langle 3, 0, -5 \rangle, \quad \overline{c} = \langle 2, -3, 1 \rangle\) vektorları berilgen; \(\overline{a}\) hám \(\overline{b}\) vektorlardan dúzilgen úshmúyeshliktiń maydanın hám usı úsh vektordan dúzilgen parallelepipedtiń kólemin tabıń.
\end{tabular}
\vspace{1cm}


\begin{tabular}{m{17cm}}
\textbf{77-variant}\\
1. Ekinshi tàrtipli sızıq hàm tuwrınıń òzara jaylasıwı. Ekinshi tàrtipli sızıqlardıń urınbası.\\

2. Vektorlardıń vektorlıq kòbeymesi hàm qàsiyetleri. Vektorlıq kòbeymeniń koordinatalardaǵı ańlatpası. \\

3. Tegislikte $A: (-5, 0)$, $B: (3, 4)$, $C: (-1, 7)$ noqatları berilgen. $\overline{AB}, \overline{AC}, \overline{BC}$ vektorardı dúziń; $AB, AC, BC$ uzınlıqlardı tabıń; $\Delta ABC$ úshmúyeshliktiń maydanın esaplań. \\

4. Berilgen ekinshi tártipli iymek sızıqtıń teńlemesin ápiwayılastırıń, tipin anıqlań, orayın hám yarım kósherlerin tabıń: $4x^2+16y^2-16x-96y+96=0$\\

5. \(\overline{a} = \langle -2, 1, 0 \rangle, \quad \overline{b} = \langle 3, -4, 2 \rangle, \quad \overline{c} = \langle 1, 0, 5 \rangle\) vektorları berilgen; \(\overline{a}\) hám \(\overline{b}\) vektorlardan dúzilgen úshmúyeshliktiń maydanın hám usı úsh vektordan dúzilgen parallelepipedtiń kólemin tabıń.
\end{tabular}
\vspace{1cm}


\begin{tabular}{m{17cm}}
\textbf{78-variant}\\
1. Tegislikte ekinshi tàrtipli sızıqlardıń kanonikalıq teńlemeleri.\\

2. Ellips hàm onıń kanonikalıq teńlemesi, ekcentrisiteti, grafigi.\\

3. Tegislikte $A: (3, -4)$, $B: (-2, 7)$, $C: (5, 2)$ noqatları berilgen. $\overline{AB}, \overline{AC}, \overline{BC}$ vektorardı dúziń; $AB, AC, BC$ uzınlıqlardı tabıń; $\Delta ABC$ úshmúyeshliktiń maydanın esaplań. \\

4. Berilgen ekinshi tártipli iymek sızıqtıń teńlemesin ápiwayılastırıń, tipin anıqlań, orayın hám yarım kósherlerin tabıń: $x^2+y^2+10x+2y+25=0$\\

5. \(\overline{a} = \langle 3, 0, -2 \rangle, \quad \overline{b} = \langle -1, 2, 4 \rangle, \quad \overline{c} = \langle 2, -3, 1 \rangle\) vektorları berilgen; \(\overline{a}\) hám \(\overline{b}\) vektorlardan dúzilgen úshmúyeshliktiń maydanın hám usı úsh vektordan dúzilgen parallelepipedtiń kólemin tabıń.
\end{tabular}
\vspace{1cm}


\begin{tabular}{m{17cm}}
\textbf{79-variant}\\
1. Vektorlar ùstinde sızıqlı àmeller.\\

2. Tegisliktiń tùrli teńlemeleri. Tegisliklerdiń òzara jaylasıwı.\\

3. Tegislikte $A: (1, 6)$, $B: (-3, -1)$, $C: (7, 4)$ noqatları berilgen. $\overline{AB}, \overline{AC}, \overline{BC}$ vektorardı dúziń; $AB, AC, BC$ uzınlıqlardı tabıń; $\Delta ABC$ úshmúyeshliktiń maydanın esaplań. \\

4. Berilgen ekinshi tártipli iymek sızıqtıń teńlemesin ápiwayılastırıń, tipin anıqlań, orayın hám yarım kósherlerin tabıń: $25x^2+y^2+150x+6y+209=0$\\

5. \(\overline{a} = \langle 1, 3, -2 \rangle, \quad \overline{b} = \langle -4, 2, 1 \rangle, \quad \overline{c} = \langle 2, -5, 0 \rangle\) vektorları berilgen; \(\overline{a}\) hám \(\overline{b}\) vektorla dúzilgen úshmúyeshliktiń maydanın hám usı úsh vektordan dúzilgen parallelepipedtiń kólemin tabıń.
\end{tabular}
\vspace{1cm}


\begin{tabular}{m{17cm}}
\textbf{80-variant}\\
1. Ekinshi tàrtipli sızıqlardıń ulıwma teńlemeleri, orayı. Orayǵa iye hàm oraysız sızıqlar.\\

2. Vektorlardıń aralas kòbeymesi hàm qàsiyetleri. Parallelepiped kòlemi\\

3. Tegislikte $A: (-4, 2)$, $B: (-3, -5)$, $C: (6, 8)$ noqatları berilgen. $\overline{AB}, \overline{AC}, \overline{BC}$ vektorardı dúziń; $AB, AC, BC$ uzınlıqlardı tabıń; $\Delta ABC$ úshmúyeshliktiń maydanın esaplań. \\

4. Berilgen ekinshi tártipli iymek sızıqtıń teńlemesin ápiwayılastırıń, tipin anıqlań, orayın hám yarım kósherlerin tabıń: $x^2+16y^2+10x-32y+25=0$\\

5. \(\overline{a} = \langle 5, -2, 1 \rangle, \quad \overline{b} = \langle 0, 3, -4 \rangle, \quad \overline{c} = \langle -3, 1, 6 \rangle\) vektorları berilgen; \(\overline{a}\) hám \(\overline{b}\) vektorlardan dúzilgen úshmúyeshliktiń maydanın hám usı úsh vektordan dúzilgen parallelepipedtiń kólemin tabıń.
\end{tabular}
\vspace{1cm}


\begin{tabular}{m{17cm}}
\textbf{81-variant}\\
1. Eki tegisliktiń parallellik, perpendikulyarlıq shàrtleri.\\

2. Ekinshi tàrtipli sızıqlardıń invariyantları.\\

3. Tegislikte $A: (-2, 0)$, $B: (3, 2)$, $C: (-4, -3)$ noqatları berilgen. $\overline{AB}, \overline{AC}, \overline{BC}$ vektorardı dúziń; $AB, AC, BC$ uzınlıqlardı tabıń; $\Delta ABC$ úshmúyeshliktiń maydanın esaplań. \\

4. Berilgen ekinshi tártipli iymek sızıqtıń teńlemesin ápiwayılastırıń, tipin anıqlań, orayın hám yarım kósherlerin tabıń: $4x^2+9y^2+8x-72y+112=0$\\

5. \(\overline{a} = \langle 4, 0, -2 \rangle, \quad \overline{b} = \langle -1, 3, 5 \rangle, \quad \overline{c} = \langle 2, 1, -3 \rangle\) vektorları berilgen; \(\overline{a}\) hám \(\overline{b}\) vektorlardan dúzilgen úshmúyeshliktiń maydanın hám usı úsh vektordan dúzilgen parallelepipedtiń kólemin tabıń.
\end{tabular}
\vspace{1cm}


\begin{tabular}{m{17cm}}
\textbf{82-variant}\\
1. Tuwrı hàm tegisliktiń òzara jaylasıwı. Ayqasıwshı tuwrılar.\\

2. Ekinshi tàrtipli sızıqlardıń ulıwma teńlemesin kanonikalıq kòriniske keltiriw usılları.\\

3. Tegislikte $A: (-3, 7)$, $B: (4, 3)$, $C: (-1, -4)$ noqatları berilgen. $\overline{AB}, \overline{AC}, \overline{BC}$ vektorardı dúziń; $AB, AC, BC$ uzınlıqlardı tabıń; $\Delta ABC$ úshmúyeshliktiń maydanın esaplań. \\

4. Berilgen ekinshi tártipli iymek sızıqtıń teńlemesin ápiwayılastırıń, tipin anıqlań, orayın hám yarım kósherlerin tabıń: $9x^2+y^2-72x+4y+139=0$\\

5. \(\overline{a} = \langle 5, -2, 1 \rangle, \quad \overline{b} = \langle 0, 3, -4 \rangle, \quad \overline{c} = \langle -3, 1, 6 \rangle\) vektorları berilgen; \(\overline{a}\) hám \(\overline{b}\) vektorlardan dúzilgen úshmúyeshliktiń maydanın hám usı úsh vektordan dúzilgen parallelepipedtiń kólemin tabıń.
\end{tabular}
\vspace{1cm}


\begin{tabular}{m{17cm}}
\textbf{83-variant}\\
1. Giperbola hàm parabola. Kanonikalıq teńlemeleri, ekcentrisiteti, qàsiyetleri, grafikleri.\\

2. Vektorlardıń skalyar kòbeymesi hàm qàsiyetleri. Skalyar kòbeymeniń koordinatalardaǵı ańlatpası.\\

3. Tegislikte $A: (2, -1)$, $B: (0, 4)$, $C: (3, 0)$ noqatları berilgen. $\overline{AB}, \overline{AC}, \overline{BC}$ vektorardı dúziń; $AB, AC, BC$ uzınlıqlardı tabıń; $\Delta ABC$ úshmúyeshliktiń maydanın esaplań. \\

4. Berilgen ekinshi tártipli iymek sızıqtıń teńlemesin ápiwayılastırıń, tipin anıqlań, orayın hám yarım kósherlerin tabıń: $16x^2+25y^2+64x-150y-111=0$\\

5. \(\overline{a} = \langle -3, 2, -4 \rangle, \quad \overline{b} = \langle 1, 5, 0 \rangle, \quad \overline{c} = \langle 2, -1, 3 \rangle\) vektorları berilgen; \(\overline{a}\) hám \(\overline{b}\) vektorlardan dúzilgen úshmúyeshliktiń maydanın hám usı úsh vektordan dúzilgen parallelepipedtiń kólemin tabıń.
\end{tabular}
\vspace{1cm}


\begin{tabular}{m{17cm}}
\textbf{84-variant}\\
1. Keńislikte tuwrınıń har tùrli teńlemeleri. \\

2. Ekinshi tàrtipli sızıq hàm tuwrınıń òzara jaylasıwı. Ekinshi tàrtipli sızıqlardıń urınbası.\\

3. Tegislikte $A: (0, 0)$, $B: (1, 1)$, $C: (-1, 2)$ noqatları berilgen. $\overline{AB}, \overline{AC}, \overline{BC}$ vektorardı dúziń; $AB, AC, BC$ uzınlıqlardı tabıń; $\Delta ABC$ úshmúyeshliktiń maydanın esaplań. \\

4. Berilgen ekinshi tártipli iymek sızıqtıń teńlemesin ápiwayılastırıń, tipin anıqlań, orayın hám yarım kósherlerin tabıń: $25x^2+9y^2+150x-36y+36=0$\\

5. \(\overline{a} = \langle -1, 4, 2 \rangle, \quad \overline{b} = \langle 3, 0, -5 \rangle, \quad \overline{c} = \langle 2, -3, 1 \rangle\) vektorları berilgen; \(\overline{a}\) hám \(\overline{b}\) vektorlardan dúzilgen úshmúyeshliktiń maydanın hám usı úsh vektordan dúzilgen parallelepipedtiń kólemin tabıń.
\end{tabular}
\vspace{1cm}


\begin{tabular}{m{17cm}}
\textbf{85-variant}\\
1. Vektorlardıń vektorlıq kòbeymesi hàm qàsiyetleri. Vektorlıq kòbeymeniń koordinatalardaǵı ańlatpası. \\

2. Tegislikte ekinshi tàrtipli sızıqlardıń kanonikalıq teńlemeleri.\\

3. Tegislikte $A: (-1, -3)$, $B: (-4, 6)$, $C: (3, 2)$ noqatları berilgen. $\overline{AB}, \overline{AC}, \overline{BC}$ vektorardı dúziń; $AB, AC, BC$ uzınlıqlardı tabıń; $\Delta ABC$ úshmúyeshliktiń maydanın esaplań. \\

4. Berilgen ekinshi tártipli iymek sızıqtıń teńlemesin ápiwayılastırıń, tipin anıqlań, orayın hám yarım kósherlerin tabıń: $x^2+4y^2+10x+40y+121=0$\\

5. \(\overline{a} = \langle -3, 2, -4 \rangle, \quad \overline{b} = \langle 1, 5, 0 \rangle, \quad \overline{c} = \langle 2, -1, 3 \rangle\) vektorları berilgen; \(\overline{a}\) hám \(\overline{b}\) vektorlardan dúzilgen úshmúyeshliktiń maydanın hám usı úsh vektordan dúzilgen parallelepipedtiń kólemin tabıń.
\end{tabular}
\vspace{1cm}


\begin{tabular}{m{17cm}}
\textbf{86-variant}\\
1. Ellips hàm onıń kanonikalıq teńlemesi, ekcentrisiteti, grafigi.\\

2. Vektorlar ùstinde sızıqlı àmeller.\\

3. Tegislikte $A: (8, 1)$, $B: (2, -3)$, $C: (-6, 5)$ noqatları berilgen. $\overline{AB}, \overline{AC}, \overline{BC}$ vektorardı dúziń; $AB, AC, BC$ uzınlıqlardı tabıń; $\Delta ABC$ úshmúyeshliktiń maydanın esaplań. \\

4. Berilgen ekinshi tártipli iymek sızıqtıń teńlemesin ápiwayılastırıń, tipin anıqlań, orayın hám yarım kósherlerin tabıń: $9x^2+y^2+90x+2y+217=0$\\

5. \(\overline{a} = \langle 1, 2, -3 \rangle, \quad \overline{b} = \langle -2, 3, 0 \rangle, \quad \overline{c} = \langle 4, -1, 2 \rangle\) vektorları berilgen; \(\overline{a}\) hám \(\overline{b}\) vektorlardan dúzilgen úshmúyeshliktiń maydanın hám usı úsh vektordan dúzilgen parallelepipedtiń kólemin tabıń.
\end{tabular}
\vspace{1cm}


\begin{tabular}{m{17cm}}
\textbf{87-variant}\\
1. Tegisliktiń tùrli teńlemeleri. Tegisliklerdiń òzara jaylasıwı.\\

2. Ekinshi tàrtipli sızıqlardıń ulıwma teńlemeleri, orayı. Orayǵa iye hàm oraysız sızıqlar.\\

3. Tegislikte $A: (4, 8)$, $B: (0, -6)$, $C: (-7, 3)$ noqatları berilgen. $\overline{AB}, \overline{AC}, \overline{BC}$ vektorardı dúziń; $AB, AC, BC$ uzınlıqlardı tabıń; $\Delta ABC$ úshmúyeshliktiń maydanın esaplań. \\

4. Berilgen ekinshi tártipli iymek sızıqtıń teńlemesin ápiwayılastırıń, tipin anıqlań, orayın hám yarım kósherlerin tabıń: $4x^2+4y^2+40x-8y+88=0$\\

5. \(\overline{a} = \langle 0, 1, 3 \rangle, \quad \overline{b} = \langle 4, -2, 0 \rangle, \quad \overline{c} = \langle -1, 2, -5 \rangle\) vektorları berilgen; \(\overline{a}\) hám \(\overline{b}\) vektorlardan dúzilgen úshmúyeshliktiń maydanın hám usı úsh vektordan dúzilgen parallelepipedtiń kólemin tabıń.
\end{tabular}
\vspace{1cm}


\begin{tabular}{m{17cm}}
\textbf{88-variant}\\
1. Vektorlardıń aralas kòbeymesi hàm qàsiyetleri. Parallelepiped kòlemi\\

2. Eki tegisliktiń parallellik, perpendikulyarlıq shàrtleri.\\

3. Tegislikte $A: (0, 8)$, $B: (-2, -4)$, $C: (5, 1)$ noqatları berilgen. $\overline{AB}, \overline{AC}, \overline{BC}$ vektorardı dúziń; $AB, AC, BC$ uzınlıqlardı tabıń; $\Delta ABC$ úshmúyeshliktiń maydanın esaplań. \\

4. Berilgen ekinshi tártipli iymek sızıqtıń teńlemesin ápiwayılastırıń, tipin anıqlań, orayın hám yarım kósherlerin tabıń: $4x^2+9y^2+32x-18y+37=0$\\

5. \(\overline{a} = \langle 3, 2, 1 \rangle, \quad \overline{b} = \langle -2, 1, -4 \rangle, \quad \overline{c} = \langle 5, 0, 3 \rangle\) vektorları berilgen; \(\overline{a}\) hám \(\overline{b}\) vektorlardan dúzilgen úshmúyeshliktiń maydanın hám usı úsh vektordan dúzilgen parallelepipedtiń kólemin tabıń.
\end{tabular}
\vspace{1cm}


\begin{tabular}{m{17cm}}
\textbf{89-variant}\\
1. Ekinshi tàrtipli sızıqlardıń invariyantları.\\

2. Tuwrı hàm tegisliktiń òzara jaylasıwı. Ayqasıwshı tuwrılar.\\

3. Tegislikte $A: (-3, -2)$, $B: (-5, 4)$, $C: (6, -1)$ noqatları berilgen. $\overline{AB}, \overline{AC}, \overline{BC}$ vektorardı dúziń; $AB, AC, BC$ uzınlıqlardı tabıń; $\Delta ABC$ úshmúyeshliktiń maydanın esaplań. \\

4. Berilgen ekinshi tártipli iymek sızıqtıń teńlemesin ápiwayılastırıń, tipin anıqlań, orayın hám yarım kósherlerin tabıń: $16x^2+y^2+128x-4y+244=0$\\

5. \(\overline{a} = \langle 2, -1, 3 \rangle, \quad \overline{b} = \langle -4, 5, 1 \rangle, \quad \overline{c} = \langle 0, 2, -3 \rangle\) vektorları berilgen; \(\overline{a}\) hám \(\overline{b}\) vektorlardan dúzilgen úshmúyeshliktiń maydanın hám usı úsh vektordan dúzilgen parallelepipedtiń kólemin tabıń.
\end{tabular}
\vspace{1cm}


\begin{tabular}{m{17cm}}
\textbf{90-variant}\\
1. Ekinshi tàrtipli sızıqlardıń ulıwma teńlemesin kanonikalıq kòriniske keltiriw usılları.\\

2. Giperbola hàm parabola. Kanonikalıq teńlemeleri, ekcentrisiteti, qàsiyetleri, grafikleri.\\

3. Tegislikte $A: (1, 2)$, $B: (-2, 6)$, $C: (4, -3)$ noqatları berilgen. $\overline{AB}, \overline{AC}, \overline{BC}$ vektorardı dúziń; $AB, AC, BC$ uzınlıqlardı tabıń; $\Delta ABC$ úshmúyeshliktiń maydanın esaplań. \\

4. Berilgen ekinshi tártipli iymek sızıqtıń teńlemesin ápiwayılastırıń, tipin anıqlań, orayın hám yarım kósherlerin tabıń: $4x^2+y^2+16x-2y+13=0$\\

5. \(\overline{a} = \langle 1, 3, -2 \rangle, \quad \overline{b} = \langle -4, 2, 1 \rangle, \quad \overline{c} = \langle 2, -5, 0 \rangle\) vektorları berilgen; \(\overline{a}\) hám \(\overline{b}\) vektorlardan dúzilgen úshmúyeshliktiń maydanın hám usı úsh vektordan dúzilgen parallelepipedtiń kólemin tabıń.
\end{tabular}
\vspace{1cm}


\begin{tabular}{m{17cm}}
\textbf{91-variant}\\
1. Vektorlardıń skalyar kòbeymesi hàm qàsiyetleri. Skalyar kòbeymeniń koordinatalardaǵı ańlatpası.\\

2. Keńislikte tuwrınıń har tùrli teńlemeleri. \\

3. Tegislikte $A: (6, -2)$, $B: (2, 1)$, $C: (-2, 5)$ noqatları berilgen. $\overline{AB}, \overline{AC}, \overline{BC}$ vektorardı dúziń; $AB, AC, BC$ uzınlıqlardı tabıń; $\Delta ABC$ úshmúyeshliktiń maydanın esaplań. \\

4. Berilgen ekinshi tártipli iymek sızıqtıń teńlemesin ápiwayılastırıń, tipin anıqlań, orayın hám yarım kósherlerin tabıń: $4x^2+4y^2+16x+16y+16=0$\\

5. \(\overline{a} = \langle 2, -1, 3 \rangle, \quad \overline{b} = \langle -4, 5, 1 \rangle, \quad \overline{c} = \langle 0, 2, -3 \rangle\) vektorları berilgen; \(\overline{a}\) hám \(\overline{b}\) vektorlardan dúzilgen úshmúyeshliktiń maydanın hám usı úsh vektordan dúzilgen parallelepipedtiń kólemin tabıń.
\end{tabular}
\vspace{1cm}


\begin{tabular}{m{17cm}}
\textbf{92-variant}\\
1. Ekinshi tàrtipli sızıq hàm tuwrınıń òzara jaylasıwı. Ekinshi tàrtipli sızıqlardıń urınbası.\\

2. Vektorlardıń vektorlıq kòbeymesi hàm qàsiyetleri. Vektorlıq kòbeymeniń koordinatalardaǵı ańlatpası. \\

3. Tegislikte $A: (2, 5)$, $B: (7, -3)$, $C: (-4, 9)$ noqatları berilgen. $\overline{AB}, \overline{AC}, \overline{BC}$ vektorardı dúziń; $AB, AC, BC$ uzınlıqlardı tabıń; $\Delta ABC$ úshmúyeshliktiń maydanın esaplań. \\

4. Berilgen ekinshi tártipli iymek sızıqtıń teńlemesin ápiwayılastırıń, tipin anıqlań, orayın hám yarım kósherlerin tabıń: $9x^2+4y^2-36x-16y+16=0$\\

5. \(\overline{a} = \langle 3, 2, 1 \rangle, \quad \overline{b} = \langle -2, 1, -4 \rangle, \quad \overline{c} = \langle 5, 0, 3 \rangle\) vektorları berilgen; \(\overline{a}\) hám \(\overline{b}\) vektorlardan dúzilgen úshmúyeshliktiń maydanın hám usı úsh vektordan dúzilgen parallelepipedtiń kólemin tabıń.
\end{tabular}
\vspace{1cm}


\begin{tabular}{m{17cm}}
\textbf{93-variant}\\
1. Tegislikte ekinshi tàrtipli sızıqlardıń kanonikalıq teńlemeleri.\\

2. Ellips hàm onıń kanonikalıq teńlemesi, ekcentrisiteti, grafigi.\\

3. Tegislikte $A: (2, -5)$, $B: (6, 3)$, $C: (-3, 0)$ noqatları berilgen. $\overline{AB}, \overline{AC}, \overline{BC}$ vektorardı dúziń; $AB, AC, BC$ uzınlıqlardı tabıń; $\Delta ABC$ úshmúyeshliktiń maydanın esaplań. \\

4. Berilgen ekinshi tártipli iymek sızıqtıń teńlemesin ápiwayılastırıń, tipin anıqlań, orayın hám yarım kósherlerin tabıń: $x^2+16y^2+8x+160y+400=0$\\

5. \(\overline{a} = \langle 4, 0, -2 \rangle, \quad \overline{b} = \langle -1, 3, 5 \rangle, \quad \overline{c} = \langle 2, 1, -3 \rangle\) vektorları berilgen; \(\overline{a}\) hám \(\overline{b}\) vektorlardan dúzilgen úshmúyeshliktiń maydanın hám usı úsh vektordan dúzilgen parallelepipedtiń kólemin tabıń.
\end{tabular}
\vspace{1cm}


\begin{tabular}{m{17cm}}
\textbf{94-variant}\\
1. Vektorlar ùstinde sızıqlı àmeller.\\

2. Tegisliktiń tùrli teńlemeleri. Tegisliklerdiń òzara jaylasıwı.\\

3. Tegislikte $A: (7, 6)$, $B: (1, -2)$, $C: (-5, 1)$ noqatları berilgen. $\overline{AB}, \overline{AC}, \overline{BC}$ vektorardı dúziń; $AB, AC, BC$ uzınlıqlardı tabıń; $\Delta ABC$ úshmúyeshliktiń maydanın esaplań. \\

4. Berilgen ekinshi tártipli iymek sızıqtıń teńlemesin ápiwayılastırıń, tipin anıqlań, orayın hám yarım kósherlerin tabıń: $9x^2+4y^2-54x+32y+109=0$\\

5. \(\overline{a} = \langle 0, 1, 3 \rangle, \quad \overline{b} = \langle 4, -2, 0 \rangle, \quad \overline{c} = \langle -1, 2, -5 \rangle\) vektorları berilgen; \(\overline{a}\) hám \(\overline{b}\) vektorlardan dúzilgen úshmúyeshliktiń maydanın hám usı úsh vektordan dúzilgen parallelepipedtiń kólemin tabıń.
\end{tabular}
\vspace{1cm}


\begin{tabular}{m{17cm}}
\textbf{95-variant}\\
1. Ekinshi tàrtipli sızıqlardıń ulıwma teńlemeleri, orayı. Orayǵa iye hàm oraysız sızıqlar.\\

2. Vektorlardıń aralas kòbeymesi hàm qàsiyetleri. Parallelepiped kòlemi\\

3. Tegislikte $A: (5, 3)$, $B: (-1, 0)$, $C: (2, 7)$ noqatları berilgen. $\overline{AB}, \overline{AC}, \overline{BC}$ vektorardı dúziń; $AB, AC, BC$ uzınlıqlardı tabıń; $\Delta ABC$ úshmúyeshliktiń maydanın esaplań. \\

4. Berilgen ekinshi tártipli iymek sızıqtıń teńlemesin ápiwayılastırıń, tipin anıqlań, orayın hám yarım kósherlerin tabıń: $25x^2+4y^2-50x-8y-71=0$\\

5. \(\overline{a} = \langle 2, -4, 1 \rangle, \quad \overline{b} = \langle 0, 3, -2 \rangle, \quad \overline{c} = \langle -3, 1, 5 \rangle\) vektorları berilgen; \(\overline{a}\) hám \(\overline{b}\) vektorlardan dúzilgen úshmúyeshliktiń maydanın hám usı úsh vektordan dúzilgen parallelepipedtiń kólemin tabıń.
\end{tabular}
\vspace{1cm}


\begin{tabular}{m{17cm}}
\textbf{96-variant}\\
1. Eki tegisliktiń parallellik, perpendikulyarlıq shàrtleri.\\

2. Ekinshi tàrtipli sızıqlardıń invariyantları.\\

3. Tegislikte $A: (-1, 3)$, $B: (4, 6)$, $C: (-2, -5)$ noqatları berilgen. $\overline{AB}, \overline{AC}, \overline{BC}$ vektorardı dúziń; $AB, AC, BC$ uzınlıqlardı tabıń; $\Delta ABC$ úshmúyeshliktiń maydanın esaplań. \\

4. Berilgen ekinshi tártipli iymek sızıqtıń teńlemesin ápiwayılastırıń, tipin anıqlań, orayın hám yarım kósherlerin tabıń: $16x^2+y^2+128x-2y+241=0$\\

5. \(\overline{a} = \langle -1, 4, 2 \rangle, \quad \overline{b} = \langle 3, 0, -5 \rangle, \quad \overline{c} = \langle 2, -3, 1 \rangle\) vektorları berilgen; \(\overline{a}\) hám \(\overline{b}\) vektorlardan dúzilgen úshmúyeshliktiń maydanın hám usı úsh vektordan dúzilgen parallelepipedtiń kólemin tabıń.
\end{tabular}
\vspace{1cm}


\begin{tabular}{m{17cm}}
\textbf{97-variant}\\
1. Tuwrı hàm tegisliktiń òzara jaylasıwı. Ayqasıwshı tuwrılar.\\

2. Ekinshi tàrtipli sızıqlardıń ulıwma teńlemesin kanonikalıq kòriniske keltiriw usılları.\\

3. Tegislikte $A: (-5, 0)$, $B: (3, 4)$, $C: (-1, 7)$ noqatları berilgen. $\overline{AB}, \overline{AC}, \overline{BC}$ vektorardı dúziń; $AB, AC, BC$ uzınlıqlardı tabıń; $\Delta ABC$ úshmúyeshliktiń maydanın esaplań. \\

4. Berilgen ekinshi tártipli iymek sızıqtıń teńlemesin ápiwayılastırıń, tipin anıqlań, orayın hám yarım kósherlerin tabıń: $4x^2+16y^2-16x-96y+96=0$\\

5. \(\overline{a} = \langle -2, 1, 0 \rangle, \quad \overline{b} = \langle 3, -4, 2 \rangle, \quad \overline{c} = \langle 1, 0, 5 \rangle\) vektorları berilgen; \(\overline{a}\) hám \(\overline{b}\) vektorlardan dúzilgen úshmúyeshliktiń maydanın hám usı úsh vektordan dúzilgen parallelepipedtiń kólemin tabıń.
\end{tabular}
\vspace{1cm}


\begin{tabular}{m{17cm}}
\textbf{98-variant}\\
1. Giperbola hàm parabola. Kanonikalıq teńlemeleri, ekcentrisiteti, qàsiyetleri, grafikleri.\\

2. Vektorlardıń skalyar kòbeymesi hàm qàsiyetleri. Skalyar kòbeymeniń koordinatalardaǵı ańlatpası.\\

3. Tegislikte $A: (3, -4)$, $B: (-2, 7)$, $C: (5, 2)$ noqatları berilgen. $\overline{AB}, \overline{AC}, \overline{BC}$ vektorardı dúziń; $AB, AC, BC$ uzınlıqlardı tabıń; $\Delta ABC$ úshmúyeshliktiń maydanın esaplań. \\

4. Berilgen ekinshi tártipli iymek sızıqtıń teńlemesin ápiwayılastırıń, tipin anıqlań, orayın hám yarım kósherlerin tabıń: $x^2+y^2+10x+2y+25=0$\\

5. \(\overline{a} = \langle 3, 0, -2 \rangle, \quad \overline{b} = \langle -1, 2, 4 \rangle, \quad \overline{c} = \langle 2, -3, 1 \rangle\) vektorları berilgen; \(\overline{a}\) hám \(\overline{b}\) vektorlardan dúzilgen úshmúyeshliktiń maydanın hám usı úsh vektordan dúzilgen parallelepipedtiń kólemin tabıń.
\end{tabular}
\vspace{1cm}


\begin{tabular}{m{17cm}}
\textbf{99-variant}\\
1. Keńislikte tuwrınıń har tùrli teńlemeleri. \\

2. Ekinshi tàrtipli sızıq hàm tuwrınıń òzara jaylasıwı. Ekinshi tàrtipli sızıqlardıń urınbası.\\

3. Tegislikte $A: (1, 6)$, $B: (-3, -1)$, $C: (7, 4)$ noqatları berilgen. $\overline{AB}, \overline{AC}, \overline{BC}$ vektorardı dúziń; $AB, AC, BC$ uzınlıqlardı tabıń; $\Delta ABC$ úshmúyeshliktiń maydanın esaplań. \\

4. Berilgen ekinshi tártipli iymek sızıqtıń teńlemesin ápiwayılastırıń, tipin anıqlań, orayın hám yarım kósherlerin tabıń: $25x^2+y^2+150x+6y+209=0$\\

5. \(\overline{a} = \langle 1, 3, -2 \rangle, \quad \overline{b} = \langle -4, 2, 1 \rangle, \quad \overline{c} = \langle 2, -5, 0 \rangle\) vektorları berilgen; \(\overline{a}\) hám \(\overline{b}\) vektorla dúzilgen úshmúyeshliktiń maydanın hám usı úsh vektordan dúzilgen parallelepipedtiń kólemin tabıń.
\end{tabular}
\vspace{1cm}


\begin{tabular}{m{17cm}}
\textbf{100-variant}\\
1. Vektorlardıń vektorlıq kòbeymesi hàm qàsiyetleri. Vektorlıq kòbeymeniń koordinatalardaǵı ańlatpası. \\

2. Tegislikte ekinshi tàrtipli sızıqlardıń kanonikalıq teńlemeleri.\\

3. Tegislikte $A: (-4, 2)$, $B: (-3, -5)$, $C: (6, 8)$ noqatları berilgen. $\overline{AB}, \overline{AC}, \overline{BC}$ vektorardı dúziń; $AB, AC, BC$ uzınlıqlardı tabıń; $\Delta ABC$ úshmúyeshliktiń maydanın esaplań. \\

4. Berilgen ekinshi tártipli iymek sızıqtıń teńlemesin ápiwayılastırıń, tipin anıqlań, orayın hám yarım kósherlerin tabıń: $x^2+16y^2+10x-32y+25=0$\\

5. \(\overline{a} = \langle 5, -2, 1 \rangle, \quad \overline{b} = \langle 0, 3, -4 \rangle, \quad \overline{c} = \langle -3, 1, 6 \rangle\) vektorları berilgen; \(\overline{a}\) hám \(\overline{b}\) vektorlardan dúzilgen úshmúyeshliktiń maydanın hám usı úsh vektordan dúzilgen parallelepipedtiń kólemin tabıń.
\end{tabular}
\vspace{1cm}

\end{document}