Tegislikte $A: (2, 5)$, $B: (7, -3)$, $C: (-4, 9)$ noqatları berilgen. $\overline{AB}, \overline{AC}, \overline{BC}$ vektorardı dúziń; $AB, AC, BC$ uzınlıqlardı tabıń; $\Delta ABC$ úshmúyeshliktiń maydanın esaplań. 
Tegislikte $A: (-1, 3)$, $B: (4, 6)$, $C: (-2, -5)$ noqatları berilgen. $\overline{AB}, \overline{AC}, \overline{BC}$ vektorardı dúziń; $AB, AC, BC$ uzınlıqlardı tabıń; $\Delta ABC$ úshmúyeshliktiń maydanın esaplań. 
Tegislikte $A: (0, 0)$, $B: (1, 1)$, $C: (-1, 2)$ noqatları berilgen. $\overline{AB}, \overline{AC}, \overline{BC}$ vektorardı dúziń; $AB, AC, BC$ uzınlıqlardı tabıń; $\Delta ABC$ úshmúyeshliktiń maydanın esaplań. 
Tegislikte $A: (3, -4)$, $B: (-2, 7)$, $C: (5, 2)$ noqatları berilgen. $\overline{AB}, \overline{AC}, \overline{BC}$ vektorardı dúziń; $AB, AC, BC$ uzınlıqlardı tabıń; $\Delta ABC$ úshmúyeshliktiń maydanın esaplań. 
Tegislikte $A: (-3, -2)$, $B: (-5, 4)$, $C: (6, -1)$ noqatları berilgen. $\overline{AB}, \overline{AC}, \overline{BC}$ vektorardı dúziń; $AB, AC, BC$ uzınlıqlardı tabıń; $\Delta ABC$ úshmúyeshliktiń maydanın esaplań. 
Tegislikte $A: (8, 1)$, $B: (2, -3)$, $C: (-6, 5)$ noqatları berilgen. $\overline{AB}, \overline{AC}, \overline{BC}$ vektorardı dúziń; $AB, AC, BC$ uzınlıqlardı tabıń; $\Delta ABC$ úshmúyeshliktiń maydanın esaplań. 
Tegislikte $A: (4, 8)$, $B: (0, -6)$, $C: (-7, 3)$ noqatları berilgen. $\overline{AB}, \overline{AC}, \overline{BC}$ vektorardı dúziń; $AB, AC, BC$ uzınlıqlardı tabıń; $\Delta ABC$ úshmúyeshliktiń maydanın esaplań. 
Tegislikte $A: (-2, 0)$, $B: (3, 2)$, $C: (-4, -3)$ noqatları berilgen. $\overline{AB}, \overline{AC}, \overline{BC}$ vektorardı dúziń; $AB, AC, BC$ uzınlıqlardı tabıń; $\Delta ABC$ úshmúyeshliktiń maydanın esaplań. 
Tegislikte $A: (1, 6)$, $B: (-3, -1)$, $C: (7, 4)$ noqatları berilgen. $\overline{AB}, \overline{AC}, \overline{BC}$ vektorardı dúziń; $AB, AC, BC$ uzınlıqlardı tabıń; $\Delta ABC$ úshmúyeshliktiń maydanın esaplań. 
Tegislikte $A: (5, 3)$, $B: (-1, 0)$, $C: (2, 7)$ noqatları berilgen. $\overline{AB}, \overline{AC}, \overline{BC}$ vektorardı dúziń; $AB, AC, BC$ uzınlıqlardı tabıń; $\Delta ABC$ úshmúyeshliktiń maydanın esaplań. 
Tegislikte $A: (-4, 2)$, $B: (-3, -5)$, $C: (6, 8)$ noqatları berilgen. $\overline{AB}, \overline{AC}, \overline{BC}$ vektorardı dúziń; $AB, AC, BC$ uzınlıqlardı tabıń; $\Delta ABC$ úshmúyeshliktiń maydanın esaplań. 
Tegislikte $A: (2, -1)$, $B: (0, 4)$, $C: (3, 0)$ noqatları berilgen. $\overline{AB}, \overline{AC}, \overline{BC}$ vektorardı dúziń; $AB, AC, BC$ uzınlıqlardı tabıń; $\Delta ABC$ úshmúyeshliktiń maydanın esaplań. 
Tegislikte $A: (7, 6)$, $B: (1, -2)$, $C: (-5, 1)$ noqatları berilgen. $\overline{AB}, \overline{AC}, \overline{BC}$ vektorardı dúziń; $AB, AC, BC$ uzınlıqlardı tabıń; $\Delta ABC$ úshmúyeshliktiń maydanın esaplań. 
Tegislikte $A: (-3, 7)$, $B: (4, 3)$, $C: (-1, -4)$ noqatları berilgen. $\overline{AB}, \overline{AC}, \overline{BC}$ vektorardı dúziń; $AB, AC, BC$ uzınlıqlardı tabıń; $\Delta ABC$ úshmúyeshliktiń maydanın esaplań. 
Tegislikte $A: (6, -2)$, $B: (2, 1)$, $C: (-2, 5)$ noqatları berilgen. $\overline{AB}, \overline{AC}, \overline{BC}$ vektorardı dúziń; $AB, AC, BC$ uzınlıqlardı tabıń; $\Delta ABC$ úshmúyeshliktiń maydanın esaplań. 
Tegislikte $A: (-1, -3)$, $B: (-4, 6)$, $C: (3, 2)$ noqatları berilgen. $\overline{AB}, \overline{AC}, \overline{BC}$ vektorardı dúziń; $AB, AC, BC$ uzınlıqlardı tabıń; $\Delta ABC$ úshmúyeshliktiń maydanın esaplań. 
Tegislikte $A: (0, 8)$, $B: (-2, -4)$, $C: (5, 1)$ noqatları berilgen. $\overline{AB}, \overline{AC}, \overline{BC}$ vektorardı dúziń; $AB, AC, BC$ uzınlıqlardı tabıń; $\Delta ABC$ úshmúyeshliktiń maydanın esaplań. 
Tegislikte $A: (-5, 0)$, $B: (3, 4)$, $C: (-1, 7)$ noqatları berilgen. $\overline{AB}, \overline{AC}, \overline{BC}$ vektorardı dúziń; $AB, AC, BC$ uzınlıqlardı tabıń; $\Delta ABC$ úshmúyeshliktiń maydanın esaplań. 
Tegislikte $A: (2, -5)$, $B: (6, 3)$, $C: (-3, 0)$ noqatları berilgen. $\overline{AB}, \overline{AC}, \overline{BC}$ vektorardı dúziń; $AB, AC, BC$ uzınlıqlardı tabıń; $\Delta ABC$ úshmúyeshliktiń maydanın esaplań. 
Tegislikte $A: (1, 2)$, $B: (-2, 6)$, $C: (4, -3)$ noqatları berilgen. $\overline{AB}, \overline{AC}, \overline{BC}$ vektorardı dúziń; $AB, AC, BC$ uzınlıqlardı tabıń; $\Delta ABC$ úshmúyeshliktiń maydanın esaplań. 

Berilgen ekinshi tártipli iymek sızıqtıń teńlemesin ápiwayılastırıń, tipin anıqlań, orayın hám yarım kósherlerin tabıń: $x^2+4y^2+10x+40y+121=0$
Berilgen ekinshi tártipli iymek sızıqtıń teńlemesin ápiwayılastırıń, tipin anıqlań, orayın hám yarım kósherlerin tabıń: $9x^2+y^2-72x+4y+139=0$
Berilgen ekinshi tártipli iymek sızıqtıń teńlemesin ápiwayılastırıń, tipin anıqlań, orayın hám yarım kósherlerin tabıń: $4x^2+9y^2+8x-72y+112=0$
Berilgen ekinshi tártipli iymek sızıqtıń teńlemesin ápiwayılastırıń, tipin anıqlań, orayın hám yarım kósherlerin tabıń: $9x^2+y^2+90x+2y+217=0$
Berilgen ekinshi tártipli iymek sızıqtıń teńlemesin ápiwayılastırıń, tipin anıqlań, orayın hám yarım kósherlerin tabıń: $4x^2+4y^2+40x-8y+88=0$
Berilgen ekinshi tártipli iymek sızıqtıń teńlemesin ápiwayılastırıń, tipin anıqlań, orayın hám yarım kósherlerin tabıń: $x^2+y^2+10x+2y+25=0$
Berilgen ekinshi tártipli iymek sızıqtıń teńlemesin ápiwayılastırıń, tipin anıqlań, orayın hám yarım kósherlerin tabıń: $4x^2+16y^2-16x-96y+96=0$
Berilgen ekinshi tártipli iymek sızıqtıń teńlemesin ápiwayılastırıń, tipin anıqlań, orayın hám yarım kósherlerin tabıń: $16x^2+y^2+128x-4y+244=0$
Berilgen ekinshi tártipli iymek sızıqtıń teńlemesin ápiwayılastırıń, tipin anıqlań, orayın hám yarım kósherlerin tabıń: $25x^2+4y^2-50x-8y-71=0$
Berilgen ekinshi tártipli iymek sızıqtıń teńlemesin ápiwayılastırıń, tipin anıqlań, orayın hám yarım kósherlerin tabıń: $9x^2+4y^2-36x-16y+16=0$
Berilgen ekinshi tártipli iymek sızıqtıń teńlemesin ápiwayılastırıń, tipin anıqlań, orayın hám yarım kósherlerin tabıń: $9x^2+4y^2-54x+32y+109=0$
Berilgen ekinshi tártipli iymek sızıqtıń teńlemesin ápiwayılastırıń, tipin anıqlań, orayın hám yarım kósherlerin tabıń: $25x^2+9y^2+150x-36y+36=0$
Berilgen ekinshi tártipli iymek sızıqtıń teńlemesin ápiwayılastırıń, tipin anıqlań, orayın hám yarım kósherlerin tabıń: $25x^2+y^2+150x+6y+209=0$
Berilgen ekinshi tártipli iymek sızıqtıń teńlemesin ápiwayılastırıń, tipin anıqlań, orayın hám yarım kósherlerin tabıń: $16x^2+y^2+128x-2y+241=0$
Berilgen ekinshi tártipli iymek sızıqtıń teńlemesin ápiwayılastırıń, tipin anıqlań, orayın hám yarım kósherlerin tabıń: $16x^2+25y^2+64x-150y-111=0$
Berilgen ekinshi tártipli iymek sızıqtıń teńlemesin ápiwayılastırıń, tipin anıqlań, orayın hám yarım kósherlerin tabıń: $4x^2+9y^2+32x-18y+37=0$
Berilgen ekinshi tártipli iymek sızıqtıń teńlemesin ápiwayılastırıń, tipin anıqlań, orayın hám yarım kósherlerin tabıń: $x^2+16y^2+8x+160y+400=0$
Berilgen ekinshi tártipli iymek sızıqtıń teńlemesin ápiwayılastırıń, tipin anıqlań, orayın hám yarım kósherlerin tabıń: $4x^2+4y^2+16x+16y+16=0$
Berilgen ekinshi tártipli iymek sızıqtıń teńlemesin ápiwayılastırıń, tipin anıqlań, orayın hám yarım kósherlerin tabıń: $x^2+16y^2+10x-32y+25=0$
Berilgen ekinshi tártipli iymek sızıqtıń teńlemesin ápiwayılastırıń, tipin anıqlań, orayın hám yarım kósherlerin tabıń: $4x^2+y^2+16x-2y+13=0$

\(\overline{a} = \langle 2, -1, 3 \rangle, \quad \overline{b} = \langle -4, 5, 1 \rangle, \quad \overline{c} = \langle 0, 2, -3 \rangle\) vektorları berilgen; \(\overline{a}\) hám \(\overline{b}\) vektorlardan dúzilgen úshmúyeshliktiń maydanın hám usı úsh vektordan dúzilgen parallelepipedtiń kólemin tabıń.
\(\overline{a} = \langle -1, 4, 2 \rangle, \quad \overline{b} = \langle 3, 0, -5 \rangle, \quad \overline{c} = \langle 2, -3, 1 \rangle\) vektorları berilgen; \(\overline{a}\) hám \(\overline{b}\) vektorlardan dúzilgen úshmúyeshliktiń maydanın hám usı úsh vektordan dúzilgen parallelepipedtiń kólemin tabıń.
\(\overline{a} = \langle 5, -2, 1 \rangle, \quad \overline{b} = \langle 0, 3, -4 \rangle, \quad \overline{c} = \langle -3, 1, 6 \rangle\) vektorları berilgen; \(\overline{a}\) hám \(\overline{b}\) vektorlardan dúzilgen úshmúyeshliktiń maydanın hám usı úsh vektordan dúzilgen parallelepipedtiń kólemin tabıń.
\(\overline{a} = \langle -3, 2, -4 \rangle, \quad \overline{b} = \langle 1, 5, 0 \rangle, \quad \overline{c} = \langle 2, -1, 3 \rangle\) vektorları berilgen; \(\overline{a}\) hám \(\overline{b}\) vektorlardan dúzilgen úshmúyeshliktiń maydanın hám usı úsh vektordan dúzilgen parallelepipedtiń kólemin tabıń.
\(\overline{a} = \langle 4, 0, -2 \rangle, \quad \overline{b} = \langle -1, 3, 5 \rangle, \quad \overline{c} = \langle 2, 1, -3 \rangle\) vektorları berilgen; \(\overline{a}\) hám \(\overline{b}\) vektorlardan dúzilgen úshmúyeshliktiń maydanın hám usı úsh vektordan dúzilgen parallelepipedtiń kólemin tabıń.
\(\overline{a} = \langle 0, 1, 3 \rangle, \quad \overline{b} = \langle 4, -2, 0 \rangle, \quad \overline{c} = \langle -1, 2, -5 \rangle\) vektorları berilgen; \(\overline{a}\) hám \(\overline{b}\) vektorlardan dúzilgen úshmúyeshliktiń maydanın hám usı úsh vektordan dúzilgen parallelepipedtiń kólemin tabıń.
\(\overline{a} = \langle 3, 2, 1 \rangle, \quad \overline{b} = \langle -2, 1, -4 \rangle, \quad \overline{c} = \langle 5, 0, 3 \rangle\) vektorları berilgen; \(\overline{a}\) hám \(\overline{b}\) vektorlardan dúzilgen úshmúyeshliktiń maydanın hám usı úsh vektordan dúzilgen parallelepipedtiń kólemin tabıń.
\(\overline{a} = \langle 1, 3, -2 \rangle, \quad \overline{b} = \langle -4, 2, 1 \rangle, \quad \overline{c} = \langle 2, -5, 0 \rangle\) vektorları berilgen; \(\overline{a}\) hám \(\overline{b}\) vektorlardan dúzilgen úshmúyeshliktiń maydanın hám usı úsh vektordan dúzilgen parallelepipedtiń kólemin tabıń.
\(\overline{a} = \langle -2, 1, 0 \rangle, \quad \overline{b} = \langle 3, -4, 2 \rangle, \quad \overline{c} = \langle 1, 0, 5 \rangle\) vektorları berilgen; \(\overline{a}\) hám \(\overline{b}\) vektorlardan dúzilgen úshmúyeshliktiń maydanın hám usı úsh vektordan dúzilgen parallelepipedtiń kólemin tabıń.
\(\overline{a} = \langle 2, -4, 1 \rangle, \quad \overline{b} = \langle 0, 3, -2 \rangle, \quad \overline{c} = \langle -3, 1, 5 \rangle\) vektorları berilgen; \(\overline{a}\) hám \(\overline{b}\) vektorlardan dúzilgen úshmúyeshliktiń maydanın hám usı úsh vektordan dúzilgen parallelepipedtiń kólemin tabıń.
\(\overline{a} = \langle 1, 2, -3 \rangle, \quad \overline{b} = \langle -2, 3, 0 \rangle, \quad \overline{c} = \langle 4, -1, 2 \rangle\) vektorları berilgen; \(\overline{a}\) hám \(\overline{b}\) vektorlardan dúzilgen úshmúyeshliktiń maydanın hám usı úsh vektordan dúzilgen parallelepipedtiń kólemin tabıń.
\(\overline{a} = \langle 3, 0, -2 \rangle, \quad \overline{b} = \langle -1, 2, 4 \rangle, \quad \overline{c} = \langle 2, -3, 1 \rangle\) vektorları berilgen; \(\overline{a}\) hám \(\overline{b}\) vektorlardan dúzilgen úshmúyeshliktiń maydanın hám usı úsh vektordan dúzilgen parallelepipedtiń kólemin tabıń.
\(\overline{a} = \langle 2, -1, 3 \rangle, \quad \overline{b} = \langle -4, 5, 1 \rangle, \quad \overline{c} = \langle 0, 2, -3 \rangle\) vektorları berilgen; \(\overline{a}\) hám \(\overline{b}\) vektorlardan dúzilgen úshmúyeshliktiń maydanın hám usı úsh vektordan dúzilgen parallelepipedtiń kólemin tabıń.
\(\overline{a} = \langle -1, 4, 2 \rangle, \quad \overline{b} = \langle 3, 0, -5 \rangle, \quad \overline{c} = \langle 2, -3, 1 \rangle\) vektorları berilgen; \(\overline{a}\) hám \(\overline{b}\) vektorlardan dúzilgen úshmúyeshliktiń maydanın hám usı úsh vektordan dúzilgen parallelepipedtiń kólemin tabıń.
\(\overline{a} = \langle 5, -2, 1 \rangle, \quad \overline{b} = \langle 0, 3, -4 \rangle, \quad \overline{c} = \langle -3, 1, 6 \rangle\) vektorları berilgen; \(\overline{a}\) hám \(\overline{b}\) vektorlardan dúzilgen úshmúyeshliktiń maydanın hám usı úsh vektordan dúzilgen parallelepipedtiń kólemin tabıń.
\(\overline{a} = \langle -3, 2, -4 \rangle, \quad \overline{b} = \langle 1, 5, 0 \rangle, \quad \overline{c} = \langle 2, -1, 3 \rangle\) vektorları berilgen; \(\overline{a}\) hám \(\overline{b}\) vektorlardan dúzilgen úshmúyeshliktiń maydanın hám usı úsh vektordan dúzilgen parallelepipedtiń kólemin tabıń.
\(\overline{a} = \langle 4, 0, -2 \rangle, \quad \overline{b} = \langle -1, 3, 5 \rangle, \quad \overline{c} = \langle 2, 1, -3 \rangle\) vektorları berilgen; \(\overline{a}\) hám \(\overline{b}\) vektorlardan dúzilgen úshmúyeshliktiń maydanın hám usı úsh vektordan dúzilgen parallelepipedtiń kólemin tabıń.
\(\overline{a} = \langle 0, 1, 3 \rangle, \quad \overline{b} = \langle 4, -2, 0 \rangle, \quad \overline{c} = \langle -1, 2, -5 \rangle\) vektorları berilgen; \(\overline{a}\) hám \(\overline{b}\) vektorlardan dúzilgen úshmúyeshliktiń maydanın hám usı úsh vektordan dúzilgen parallelepipedtiń kólemin tabıń.
\(\overline{a} = \langle 3, 2, 1 \rangle, \quad \overline{b} = \langle -2, 1, -4 \rangle, \quad \overline{c} = \langle 5, 0, 3 \rangle\) vektorları berilgen; \(\overline{a}\) hám \(\overline{b}\) vektorlardan dúzilgen úshmúyeshliktiń maydanın hám usı úsh vektordan dúzilgen parallelepipedtiń kólemin tabıń.
\(\overline{a} = \langle 1, 3, -2 \rangle, \quad \overline{b} = \langle -4, 2, 1 \rangle, \quad \overline{c} = \langle 2, -5, 0 \rangle\) vektorları berilgen; \(\overline{a}\) hám \(\overline{b}\) vektorla dúzilgen úshmúyeshliktiń maydanın hám usı úsh vektordan dúzilgen parallelepipedtiń kólemin tabıń.